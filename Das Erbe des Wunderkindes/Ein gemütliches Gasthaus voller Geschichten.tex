\begin{chapterbox}
    \chapter{Ein gemütliches Gasthaus voller Geschichten (2025)}
    \label{Ein gemütliches Gasthaus voller Geschichten (2025)}
    \az{82}
    
    Die Wirtin Gilda richtet eine offizielle Erzählbühne in ihrer Taverne zum Trunkenen Troll ein. Zahlreiche Tavernengäste kommen zusammen und tragen verschiedenste Geschichten vor.
\end{chapterbox}


\section{Gildas glamouröse Gedichtbühne}

Im westlichen Rietland des Königreichs Andor, mittig zwischen dem freien Markt und dem Südlichen Wald, lag eine gut und gern besuchte Taverne, deren legendärer Ruf als gemütlichste Gaststätte in ganz Andor verbreitet war: Die Taverne zum Trunkenen Troll. 

An jedem Nachmittag waren wie an so vielen anderen verschiedenste Feiernde versammelt. Dort drüben am Stammtisch saß Galaphil, der Druide des Verlassenen Turmes der großen Fledermäuse, tief in ein Kartenspiel versunken. Hier vorne stand eine große Steinstatue von Rapa Nui, dem Nabel der Welt, und brummelte fröhlich vor sich hin. Und an einem Randtisch saß der Feuerdämon Kar éVarin, tief in eine von Runen übersäte Schiefertafel versunken.

Soeben war die Wirtin Gilda dabei, ihre Galerie zahlreicher geschenkter Zeichnungen zünftig zufriedener Gäste umzubenennen. Kaum einer achtete darauf. Vielmehr waren die anwesenden Andori und Besucher fernerer Länder beschäftigt mit Tratsch und Klatsch, Trinken und Singen, Würfelspiel und komplizierteren Geisterinselspielen. 

Dies änderte sich, als Gilda neben dem Kamin Tische zur Seite rückte. Von hinter dem Tresen hievte sie eine breite Holzkiste hervor und schob jene bis an den Kamin, unter das dort hängende andorische Wappen. Die auf dem Kaminsims schlafende Katze ließ nur durch zuckende Ohren erkennen, dass sie die Veränderung wahrnahm.

Das angenehme Hintergrundrauschen aus miteinander sprechenden Besuchern ebbte ab und nahm, erfüllt vom Raunen rascher Spekulationen, erneut zu, als Gilda ihre Hände an ihrer Schürze abwischte, auf die Holzkiste stand und sich laut räusperte.

"Liebe Andori! Vieles habe ich in den letzten Jahren hier vorgetragen an Liedern und Sagen aus Andor und darüber hinaus. Gesungen, gedichtet, berichtet. Und so vieles habt ihr erzählt an fantastischen Geschichten, wenn ihr von legendären Heldentaten in Andor, von Legenden und Heldentafeln erzähltet. Es wird an der Zeit für eine offizielle Bühne dafür."

Sie gestikulierte zur Holzkiste, welche sie nun demonstrativ betrat. "Wie die Erfahreneren unter euch bereits wissen, finden Reisende aus aller Welt den Weg in meine bescheidene Taverne. Dann und wann geschieht es, dass einer von ihnen sich am Feuer niedersetzt und eine merkwürdige Geschichte erzählt. Ringsherum wird es still und ein jeder lauscht."

Und als hätte er es erwartet, erhob sich bei Gildas letzte Worten ein massiger Tavernenbesucher von seinem drollig kleinen Vierbein und lief zur Erzählbühne. Wie die dreifingrigen Hände und die Hörner klar machten, war dies ein Troll! Gilda stimmte einen Applaus auf ihn an, während erwartende Stille sich unter dem Publikum breit machte.




\section{Der ewige Rat}

Dies war der Tolle TroII, ein überall gern gesehener Barde der Kreaturen. Der ebenso rundum einzigartige wie ideenreiche, Geschichten erzählende, recht außergewöhnliche Troll. Ein Vogelnest trug er auf dem Kopf, ein Geschichtenbuch in seiner Hand und jede Menge Honig in der Stimme.

Und der Tolle TroII erzählte vom Ewigen Rat. Lange hielt er die lauschenden Andori in Bann, mit geschmiedeten Worten, sodass es selbst den Balladenschreiber Grenolin, die begnadete Sängerin Gilda und den aus den fernen Norden angereisten Barden Canour zu Tränen rührte. Der TroII erzählte davon, wie die Krahder besiegt worden waren, die verschleppten Andori befreit, die Helden von Andor trotz aller Opfer einmal mehr siegreich. Von Thorn, der sich nach Frieden sehnte. Von Drukil, den es dazu drängte, seine menschliche Haut abzustreifen. Von Chada, die ihr Erbe antreten  wollte, während Janis von seinem davonlief. Von Eara, die mit ihrer Dunkelheit rang. Von Leander, dessen Visionen seinen Bruder und eine alte Bedrohung zeigten. Von einem lange unterschätzten Feind, der das Erbe der Krahder antrat. Und von einem Ewigen Rat aus lange totgeglaubten Widersachern der Helden, die so viel mehr als nur Andor vernichten wollten\footnote{"Der ewige Rat" von TroII: \url{https://legenden-von-andor.de/forum/viewtopic.php?f=26&t=11465}}.

Tage- und nächtelang dichtete er an seinem Wunderwerk. Über Wochen kehrte er immer wieder in die Taverne zum Trunkenen Troll ein, stellte sich auf die Holzkiste neben dem Kamin und führte seine Geschichte fort. Die Tavernengäste hingen gebannt an seinen Lippen. Weinten Tränen der Trauer und sangen Lieder der Lobpreisung auf sein Talent und seinen Durchhaltewillens. Selbst manche Tiere des nahe gelegenen Südlichen Waldes näherten sich dem Gasthaus und lauschten. Ein bestimmtes Rotkehlchen landete immer wieder auf dem Sims und hörte der Erzählung des TroIIs gespannt von außerhalb des Fensters zu. Von a bis z arbeitete der TroII sich durch das andorische Alphabet, und als er damit durch war, begann er gleich wieder bei A und arbeitete sich noch einmal bis Z durch.  Es dauerte viele Tage und Nächte, bis er seine epische Erzählung einem Ende zuführte. Jenes Ende war unerwartet und hätte man doch so lange kommen sehen können. Es war der abschließendste aller möglichen Abschlüsse und doch so offen für weitere Abenteuer. Es war ... unübertrefflich. Und es war vorbei. Vollbracht. Vollendet.

Und Gilda sprach: "Wunderbar, eine erste Erzählung für die Taverne! Besten Dank, lieber TroII. Und an alle Zuhörer: Selbstverständlich müssen eure Geschichten solche Ausmaße nicht annehmen. Es steht euch vollkommen frei, wie lang eure Geschichten sind."

Die Hände wund vom Klatschen, blieben die Tavernengäste still in Andacht und fragten sich, bei allen Kreaturen der Tiefe, wie könnte man diesem Wunderwerk je nahe kommen? 




\section{Die ersten Barden der Taverne}


Da meldete sich TroII zu Wort und korrigierte, er wäre nicht der erste gewesen, der hier in der Taverne fiktionale Geschichten präsentiert habe. So viele Leute wären bereits hier eingekehrt und hätten lange Geschichten zu erzählen begonnen.

„Erinnert ihr euch noch an den Bogenschützen Chris aus dem Westerwald? Nur dank ihm erfuhren wir von den heldenhaften Abenteuern der Zwergin Bait, die sich, während ihr Lieblingsbruder Kram an der Seite anderer Helden den Kampf um Cavern bestritt, alleine, nur mit ihrem Mut, ihrem Mithrilring und ihrer Streitaxt bewaffnet, an grausamen Kreaturen, lähmenden Spornen und stinkenden Seemonstern vorbei durch Cavern schlagen musste\footnote{"Bait, die Zwergin" von ChrisW: \url{https://legenden-von-andor.de/forum/viewtopic.php?f=11&t=1587}}.

Was war mit Tapsi Tost vom Schicksalsberg, welcher eine selbstgeschriebene Geschichte zu Andor vortrug? Mit allerlei amüsanten Anmerkungen versehene Erzählungen von einer alternativen Ankunft der Helden, einem ersten Zusammentreffen der vier bekanntesten Helden Thorn, Eara, Chada und Kram, sowie der gefürchteten Rückkehr der Skrale nach Andor, die König Brandur, Fürst Hallgard und Hohepriester Melkart allesamt überraschte. Daraufhin folgte eine Erzählung der Heilung des Königs Brandur, in die auch der Bewahrer Arbon und der Fährtenleser Fenn verwickelt wurden. Zwei bekannte Legenden in einem neuen Licht\footnote{Tosts Geschichte von Andor: \url{https://legenden-von-andor.de/forum/viewtopic.php?f=11&t=1920}}.

Der berufliche Geschichtenerzähler Kleativ holte sogar viel früher in der Zeitlinie aus und malte uns brutale Bilder der Höllenwelt Krahd und erster Vorbereitungen zur Flucht von Brandur, Reka, Harthalt, Drak und sogar des zwielichtigen Jari Dorrs\footnote{Kleativs Legende von Andor: \url{https://legenden-von-andor.de/forum/viewtopic.php?f=11&t=2439}}.

Und kaum einer konnte dem alten Barden Bird das Fabelwasser reichen, einem weiteren beruflichen Geschichtenerzähler aus einem fernen Land. Bei einem ausgezeichneten Becher Met setzte er sich alle paar Mondzyklen vor Gildas behaglich warmes Feuer und brachte lyrische Meisterwerke dar, Legenden von Andor, wie sie in seiner fernen Heimat erzählt werden. Anders als die uns bekannten, und doch so ähnlich. Vogelgleich sang Bird von der Flucht von Brandurs tausenköpfiger Schar und dem Kampf des zukünftigen Königs gegen den Drachen Tarok. Davon, wie Baumeister Kram den Tod fünf guter Zwerge verschuldete und wie die junge Chada sich zwischen Bewahren und Bewachen entscheiden musste. Wie Wachhauptmann Thorn nach einem Blutsturm beim Freien Markt zum Rechten sah. Wie die Helden ausgesandt wurden, die Hexe Reka zu finden, Lehrmeisterin der Einsiedlerin Fennah. Eine meisterhafte Erzählung\footnote{Birds Legenden von Andor: \url{https://legenden-von-andor.de/forum/viewtopic.php?f=11&t=3757}}. Ich bin betrübt, dass sie kein offizielles Ende fand.“

Bei den Worten des TroIIs sinnierte die versammelte Tavernengemeinschaft über die Werke der ersten Barden der Taverne und den Einfluss, die sie auf die bisher gehörten Geschichten gehabt hatten und auf die erst noch kommenden haben würden. Im Hintergrund spielte die Fantasy Folk Band ELANE eine Ballade über die Taverne. Und aus Birds Legenden wurde ein Trinklied hervorgehoben, auf das die Andori fröhlich anstimmten:

"\textit{Und wir trinken uns lustig, wir trinken uns toll,}

\textit{wir trinken im Gasthaus Zum Trunkenen Troll,}

\textit{und wir trinken zum Umfall'n, wir trinken uns voll,}

\textit{wir trinken im Gasthaus Zum Trunkenen Troll!}"





\section{Einzigartige Einzel-Erzählungen}




Kaum war das Trinklied ausgeklungen, erhob sich eine Gestalt. Es war Canour, der Barde aus dem hohen Norden. Andächtig betrat er die Holzkiste und sprach: "Wenn wir schon so fröhlich am Singen sind, will ich mich dem anschließen: Mir wurde aufgetragen, im Namen von Naqoy aus dem Süden einen Lobgesang darzubringen, und dies tue ich nun." Und so tat er dies. In astreinem Versmaß und in einem wunderschönen Bariton sang Canour eine Lobpreisung an auf die Wunder der Tavernengemeinschaft\footnote{"Das Lied von Canour dem Barden – ein Lobgesang" von Naqoy: \url{https://legenden-von-andor.de/forum/viewtopic.php?f=11&t=2930}}. Ein wahrer Dichter!

Da bahnte sich bereits wieder der Tolle TroII einen Weg zur Erzählbühne. Durch seinen vielseitigen Einsatz in der Gaststube hatte er Mjölnirs Olymp bezwungen, und so stimmten die Tavernengäste alliterative Lobpreisungen auf ihn an, und er Lobpreisungen auf die Tavernengemeinschaft – oder zumindest wollte er letzteres, doch schien er leider auf einmal seine vorbereitete Ballade in acht Fragmente zerteilt und selbst alle Links dazu über die ganze Taverne (und, wenn es mit der Modertaion geklappt hätte, auch darüber hinaus) verteilt und in den staubigsten Tavernenecken verloren zu haben. Prompt kamen verschiedenste Tavernengäste zusammen und halfen dem TroII dabei, die diversen Teile seines Lieds zu vereinen. Es war ein Rätsel, wie es ihnen so rasch gelang, die Fragmente zu finden, zu entschlüsseln, zu markieren, nach Alter zu sortieren und von Schmutz zu befreien. Doch am Ende waren sie siegreich und der TroII vermochte es, verschmitzt seine vollständige Ballade vorzutragen. Und als den ersten Gästen klar wurde, dass sich in deren ersten Worten mehr verbarg als vorher vermutet, ja, da grinste der TroII noch breiter. Und ihnen allen wurde klar, dass diese gesamte Aktion von äußerst geschickter dreifingriger Hand geplant worden war, und dass im gemeinsamen Geiste der Balladenvereinigung eine weitere Ode der Dankbarkeit kunstvoll versteckt worden war\footnote{"Die Ballade vom TroII, der seinen Link verloren hat": \url{https://legenden-von-andor.de/forum/viewtopic.php?f=11&t=6118}, \url{https://legenden-von-andor.de/forum/viewtopic.php?p=46782\#p46782}, ...}.

Canours und TroIIs Wunderwerke des Gesangs berührten viele. Gildas Singstimme und Grenolins Schreibkünste waren vereint in beiden, und auch der Rätselhunger der Tavernengäste war fürs Erste gestillt. Canour und der TroII setzten sich. Die schwarze Katze, die bislang auf dem Sims des knisternden Kamins geschlafen hatte, zuckte mit ihren Ohren, räkelte sich und trottete dann mit einer Selbstverständlichkeit, wie sie vielen Katzen angeboren war, auf die Holzkiste. Gleich zwei kurze Erzählungen gab sie zum besten, eine von den Folgen einer unterlassenen Erstretterritterschulung mit darauffolgendem Retterrittereinsatz\footnote{"Rettungsdienst im Lande Andor" von der Schlafenden Katze: \url{https://legenden-von-andor.de/forum/viewtopic.php?p=59497\#p59497}} und eine zweite von einem beliebten Kinderspiel mit Stoffwolfswelpen, Drachen und natürlich Junior-Helden\footnote{"Der Ursprung von Andor Junior" von der Schlafenden Katze: \url{https://legenden-von-andor.de/forum/viewtopic.php?p=63055\#p63055}}. Aus dem Publikum erklangen einige fröhliche Kinderstimmen, die zu einem melodischen "Andor, Andor!" anstimmten.

Als nächstes stand ein Fischer von der Narne auf die Holzkiste beim Kamin. Als Thies stellte er sich vor, und berichten tat er von uralten Aufzeichnungen, welche unter dem geheimen Boden einer von Fürst Kram im Verlassenen Turm wiederentdeckten Kiste gefunden worden waren: Von Orfas, Sohn des Ofanus vom vergessenen Volk der Wölfe aus den Wäldern Lupiens, vom Schicksal des Volks der Krähen aus den Corvidischen Wäldern und vom Gräuel der Schar des Drachentöters Brandur\footnote{"Die wahre Geschichte von den Anfängen Andors" von Thies: \url{https://legenden-von-andor.de/forum/viewtopic.php?f=11&t=6860}}.

Während Thies stolz von der Bühne schritt, stolperte ein kleiner Teddybär darauf zu, zog sich unter etwas Mühe auf die Holzkiste und sprudelte Wörter dort über den Santa Gor los\footnote{"Santa Gor" von BBB: \url{https://legenden-von-andor.de/forum/viewtopic.php?p=101625\#p101625}}. An dessen Legende glaubten natürlich nur noch Kinder, nichtsdestotrotz war es schön, sich auszumalen, wie der Santa Gor in einer Kutsche über den Nachthimmel flog und verschiedensten Figuren diverse passende Geschenke bescherte.

Die Kinder, die fröhlich im Chor "Andor, Andor!", gerufen hatten, wanderten hinüber, um mehr vom Santa Gor zu hören. Sie wurden abgelöst vom Musiker Mario, der extra von den Nebelinseln angreist war und dessen wunderschöne Flötenstücke die Taverne schon oft zum Singen -- oder vielmehr Grölen -- animiert hatten\footnote{Ein Lied zu Marios Melodie: \url{https://legenden-von-andor.de/forum/viewtopic.php?p=90472\#p90472}}.

Der Butterbrotbär wurde indes von einer mysteriösen Botschaft abgelenkt\footnote{Mivos Pergament: \url{https://legenden-von-andor.de/forum/viewtopic.php?f=11&t=4695&start=7439}}, ebenjenem leeren Pergament, auf das eines Tages diese Geschichte niedergeschrieben würde. Stammte es womöglich vom tulgorischen Sternkundigen Mivo?

Auf jeden Fall war mit diesen Liedern der Bann endgültig gebrochen. An diesem und an den nächsten Abenden trauten sich immer mehr Andori nach und nach zur Erzählbühne, um dort auch längere Geschichten zu erzählen.

Flederfluse flatterte aus dem fernen Tulgor herbei und berichtete vom Schiffsbruch der \textsc{Antares} und davon, wie der königliche Krieger Marcus und die Trollmagierin Anna ins ferne Tulgor gelangten und von der Felderwirtschaft der Tulgori erfuhren. Die Beschreibung zahlreicher fremder Tierarten faszinierte besonders, von Hybärnen und Baumtrollen über Feuerschrecken und blauschuppigen Feruns bis hin zum legendären Gepodon, über das noch viel gesprochen würde\footnote{Flederfluses Tulgor: \url{https://legenden-von-andor.de/forum/viewtopic.php?p=103305\#p103305}}. 

Da kam auch Kram, der erste aus den Tiefminen stammende Fürst der Schildzwerge. Er richtete seine Schildkrone und erzählte in Melkarts Namen davon, wie es nach dem Fall der Krahder weiterging. Vom brennenden Baum der Lieder, von der Jagd auf einen gefährlichen Mann in Schwarz, von einem wiedergeborenen Tarok und von einem Hilferuf von den tulgorischen Sümpfen der Traurigkeit\footnote{kram1s tulgorische Sümpfe: \url{https://legenden-von-andor.de/forum/viewtopic.php?f=11&t=5762}}.

Und als der Musiker Mario mit seiner Flöte wieder in den Norden abreiste, kam der Barde Julian auf den Plan, welcher die Andori für die restlichen Abende musikalisch belustigen würde mit Balladen vom Geheimnis des Königs und Liedern über das Rietland, das Zwergenreich, das Graue Gebirge und so einige Kämpfen. Das Lied vom Blutstrom war allerdings nicht darunter.

Von einigen Kämpfen berichtete als nächstes auch der königliche Krieger Malin. Jener brachte spannende Spekulationen an den Tisch, darüber, wie die Geschichte Andors sich entwickelt hätte, wenn Brandur sich vor seiner Flucht aus Krahd geopfert und seine Schar somit niemals Tarok besiegt und nie ein Königreich im Drachenland gegründet hätte. Was, wenn die drei Großen Trolle noch immer das Drachenland unsicher gemacht hätten, als der Yetohe-Stamm von den Krahdern dorthin gescheucht wurde? Was, wenn Fenn vom Büffelclan, Tranuk vom Einhornclan, Barz von den Iquar (?) und Grent von den Grehon statt Tenaya und den Grundspielhelden gegen die Ewige Kälte vorgegangen wären? Was, wenn Hademar einen Großangriff auf Krahd statt Andor angeführt hätte? Diesen und so vielen weiteren Fragen ging Malin nach, äußerst spannend, einsichtsreich und gut durchdacht\footnote{"Was wäre wenn ..." von Malin: \url{https://legenden-von-andor.de/forum/viewtopic.php?f=11&t=9697}}.



\section{Gemeinsame Gast-Geschichten}

Stille machte sich breit während den nächsten Abenden in der Taverne, denn wieder einmal war eine Erzählung schwer zu übertreffen. Bis sich eines Tages nicht eine Einzelperson, sondern ein ganzes kollaboratives Kollektiv aus Tavernengästen vor dem Kamin versammelte (denn auf die Holzkiste, die als Bühne fungierte, passten sie nicht alle gleichzeitig). Diese Gruppe berichtete von einem gemütlichen Jubiläums-Sommerabend am Stammtisch der Taverne, welcher abrupt durch einen Giftmord unterbrochen wurde. Sie erzählte von einer verschlüsselten Botschaft über ein anstehendes königliches Turnier und von Gerüchten über den Hexenmeister Noctis, der leider äußerst geschickt darin war, prefide Illusionen zu weben\footnote{"Der Stammtisch der Taverne": \url{https://legenden-von-andor.de/forum/viewtopic.php?f=11&t=5451}}. 

Ein junger Bewahrer namens Albus erinnerte sich an seinen ersten Besuch in der Taverne, damals, während der Ewigen Kälte, als Garz der stellvertretende Tavernenwirt gewesen war. Prompt spannen andere Andori weiter, wie sie damals auf Albus reagiert hatten, und knüpften gar eine weitere kollaborative Geschichte daran an, eine mit Dutzenden Figuren und Aberdutzenden an Beiträgen. Sie berichteten, was die Tavernengäste für Abenteuer erlebt hatten, während die Helden von Andor im Grauen Gebirge den Nekromanten Hademar gejagt hatten. Wie der junge Albus vom diebischen Varkur entführt und im Verlassenen Turm festgehalten wurde. Wie der Trunkene Troll 30 Fässer von Erloths sagenumwobenen Goldmet gegen hochbrisante Informationen zu tauschen ersuchte. Wie der zaubermächtige Agren Ühra Zwist zwischen den Völkern der Nebelinseln säte. Eine ungleich lange und wilde Geschichte voller Großangriffen auf die Taverne, finsteren Intrigen des Dunklen Magiers Varkur und eines uralten Geschenks eines mysteriösen Reisenden mit Katzenaugen: Des Kodex, einer zerbrochenen Landkarte des Andorversums, deren Vereinigung ungeahnte magische Macht zu verleihen vermag. Eine Erzählung ohne Ende, an der man noch heute weiterdichten könnte\footnote{"Abenteuer der Tavernengäste": \url{https://legenden-von-andor.de/forum/viewtopic.php?f=11&t=8059}}.

Und während vor dem Fenster des Gasthauses der erste Schnee fiel, fand sich eine kleinere Gruppe von Tavernengästen auf der Erzählbühne ein, welche gemeinsam eine kürzere kollaborative Geschichte konstruierte. Darin unterbrach ein verletzter Andori ein Weihnachtsfest in der Taverne und einige Helden von Andor (sowie Wassergeist Vara und Feuertakuri Turr) verirrten sich auf der Suche nach gefürchteten Nachtgors im Südlichen Wald, wo sie auf untote Wardraks, eine mysteriöse zerstörte Kutsche und einen bloß spärlich bekleideten Fährtenleser stießen. Und das ausgerechnet vor der großen Adventsfeier König Thoralds und in jener Nacht, in welcher der Santa Gor seine Geschenke verteilen sollte ... falls er überhaupt existierte\footnote{"Im Schatten des Winters": \url{https://legenden-von-andor.de/forum/viewtopic.php?f=26&t=11528}}.


\section{Spannende Story-Sammlungen}

Nachdem so viele Leute bewiesen hatten, dass sie gemeinsam an einer großen Geschichte spinnen konnten, eine Person nach der anderen, kamen andere Gruppen auf die Idee, gemeinsam verschiedene Geschichten spinnen zu lassen. Lose Sammlungen, die sich doch irgendwie ergänzten.

Die Schulkinder, welche bislang immer wieder "Andor, Andor!" gerufen hatten, hatten sich einmal mehr zur Erzählbühne begeben, als es im Schatten des Winters wieder um den Santa Gor gegangen war. Nun trauten sie sich nacheinander auf die Holzkiste und trugen ihre eigenen Ideen an Kurzgeschichten vor, ganze 13 Stück davon. Abenteuer der Helden Bait, Thorn, Pasco und Eara. Earas treuer Falke Korax wurde erstaunlich oft erwähnt. Geschichten von Kämpfen gegen böse Kreaturen, den Dunklen Magier Tarok und den Drachen Tarok, aber auch von zu rettenden Wölfen. In meisterhafter Erzählstimme vorgetragen, mit epischer Musik unterlegt\footnote{Sammlung "Klasse 3e": \url{https://legenden-von-andor.de/forum/viewtopic.php?f=11&t=3443}}. 

Als die Schüler die Bühne wieder freigaben und sich an ihrem Langtisch feinsten Nachtisch (andorische Knusper und Erloths Sonnenbrot mit Zimt und Zucker) gönnten, versammelte Thies von der Narne eine Gruppe von kulinarisch kreativen Tavernengästen. Wie der Fischer verkündete, war er einst von König Brandur höchstpersönlich dazu berufen worden, durchs Land zu reisen und eine Schrift zu den reichhaltigen Rezepten des Reichs zu sammeln. Angeführt von Dharwyn, ergänzt von Waldfaunen, Speerkämpfern, aufgeweckten Katzen und allerlei anderen Andori, präsentierte die Runde ihre liebsten Rezepte aus dem ganzen bekannten Andorversum  -- die meisten davon ergänzt um atmosphärische Kurzgeschichten. Von Tischmanieren der Taren über zahlreiche Einblicke in den Heldenalltag über Kreatoks letztem Schmaus vor dem Unterirdischen Krieg bis hin zu Forns zahlreichen Einblicken in Kreaturenkultur\footnote{"Das Kochbuch der Andori": \url{https://legenden-von-andor.de/forum/viewtopic.php?p=49839\#p49839}}. Die tavernenbesuchenden Bewahrer Phlegon und Tapta jubelten auf, als die ordentlichen Drachenbohnen, ihre Leibspeise, erwähnt wurden. Und generell waren die Vorträge dieser Rezepte derart beliebt, dass selbst nach Abschluss des Vortrags Bewahrer Melkart höchstpersönlich auftrat, um anlässlich des anstehenden Geburtstags der Taverne weitere Rezepte zu ergänzen, von Apfelnusskuchen der Bewahrer\footnote{Melkarts Apfelnusskuchen: \url{https://legenden-von-andor.de/forum/viewtopic.php?p=98408\#p98408}} über andorische Tomatenfladen zu Schokokeksen der Schildzwerge. 

Zum Geburtstag der Taverne lenkte Melkart den Fokus auch auf spannende Geschichten, die der Wachsame Waldläufer Aoto aus dem Wachsamen Wald zu Krahd zusammengetragen hatte. Dieser Wachsame Aoto betrat nun selbst die Erzählbühne und teilte weitere wilde Sagen aus Andor, Krahd, Silberland und anderen Orten. Von Namenskonventionen der Krahder und davon, wie die uralte Ent den Wald "Wachsam" schuf und den Bewahrern den Baum "Lieder" schenkte. Von den zywallischen Bognern Archor und Archiri, vom Druiden Yelaos und dem Nebelmagier Uhain, der nicht gerne als solchen bezeichnet wurde. Aoto blieb auch nicht alleine mit seinen Geschichten. Der Bewahrer Merrik, der Zwerg Brogg und gleich zwei verschiedene Bären schlossen darauf an und berichteten von eigenen Sagen: Seemannsgarn des alten Käpt'n Maleard, der seine \textsc{Capella} an Kapitänin Mondrianne verloren hatte. Ein Bauer, der sich unfreiwillig einem feurigen Skralreigen anschloss. Wie Forn mit Drukils Tod umging. Sogar detaillierte Ausführungen zur fernen nebligen Insel Zywall mit ihrem schwefligen Mhourlgebirge\footnote{"Geschichten von Andor oder dem Rest der Welt", eröffnet vom Wachsamen Waldläufer und ergänzt von anderen Andori: \url{https://legenden-von-andor.de/forum/viewtopic.php?f=11&t=5100}}.

Einer dieser Bären, der tiny toastförmige Teddy mit einer Vorliebe für Butterbrote, blieb für die nächsten Abende gleich auf der Bühne und begann damit, weitere Geschichten zu erzählen und zu verknüpfen. Vom tödlichen Kampf zwischen Kreatok und Tarok. Von Kjalls zweitem -- oder ersten? ;) -- Diebstahl der Zeitreise-Sphäre. Von den Konsequenzen von Taroks Tod und dem ersten Zusammentreffen der Magischen Helden. Von den Geheimnissen der Wassermagierin Jarid und des Feuerkriegers Trieest. Davon, wie der Seher Leander das Gespenst des Nekromanten Hademar einfing. Selbst von König Thoralds unfreiwilligem Ausflug in ein fernes Land mit einem Sherwood Forest. Lose verbundene Kurzgeschichten, von denen noch einige weitere geplant waren\footnote{"Bärig gebutterte Geschichten": \url{https://legenden-von-andor.de/forum/viewtopic.php?f=11&t=5828}}\textsuperscript{,}\footnote{BBBs Erbe des Wunderkindes: \url{https://legenden-von-andor.de/forum/viewtopic.php?f=26&t=11490}}.

Als nächstes präsentierte sich ein in perfekter Synchronie vor die Gäste tretendes Quartett, bestehend aus dem Zauberer und Hofschreiber Lifornus, dem abtrünnigen Krahder Ragomiter sowie zwei rundonischen Thornen namens Taris Norr und Dagain. Von diesem Quartett tauschte einer mit dem anderen in perfekt einstudiertem Rhythmus den Platz und beendete angefangene Sätze des vorherigen. Und so erzählten die vier vom fernen Land Rundon östlich des Lands der Steppe. Vom weisen König Rastak, dem von Krahdern verstoßenen Prinz Argwydd und dem magiebegabten Agren Ühra\footnote{"Rundon" von Dagain: \url{https://legenden-von-andor.de/forum/viewtopic.php?f=11&t=6345}}. Und von einer legendären Zeit der Söldner\footnote{"Zeit der Söldner" von Dagain: \url{https://legenden-von-andor.de/forum/viewtopic.php?f=11&t=9671}}. 




\section{Famose Fan-Filme}


Manche Andori meinten sogar, man solle nicht nur auf diese Holzkiste stehen und von dort Geschichten erzählen, nein, man könnte sogar welche theatralisch aufführen. Ideen wurden herumgeworfen, von Liboa\footnote{Liboas 3x3 Akte: \url{https://legenden-von-andor.de/forum/viewtopic.php?p=81089\#p81089}} zu Hofschreiber Lifornus\footnote{Lifornus' 3 Akte: \url{https://legenden-von-andor.de/forum/viewtopic.php?p=84464\#p84464}}, dessen viertes Quartettmitglied Dagain schlussendlich einen Vorschlag zu Visionen und Verrat\footnote{Dagains Visionen \& Verrat: \url{https://legenden-von-andor.de/forum/viewtopic.php?p=119840\#p119840}} vortrug, einer Variante der Legende von der Heilung des Königs, in der die Helden einen vergiftenden Diener an der Rietburg enttarnen.

Andere bardische begnadet begabte, theatralisch talentierte Truppen hatten schon längst lyrische Stücke einstudiert und inszenierten sie nun, so gut es ihnen auf und vor der kleinen Bühne ging, vor freudigem Publikum. Gregor aus dem beschaulichen Potsdam erzählte von einem magischen Tisch, der Musik machte und Würfelbecher vorhersah. Ein mechanisches Meisterwerk, von dem sogar Liphardus beeindruckt war\footnote{"Andor meets Pixelsense" von Gregor Tallig: \url{https://www.youtube.com/watch?v=QRZDlJ5JaQQ}}.  Mit "Und wenn sie nicht gestorben sind, dann würfeln sie noch heute" schloss Gregor seine Erzählung und setzte sich wieder hin.

Am nächsten Nachmittag bot eine Gruppe um Meister Felix in einem halbstündigen Stück eine theatralische Interpretation der Legende der Tage des Widerstandes dar, vom Kampf der tapferen Helden Mairen, Fennah, Pasco und Liphardus gegen finstere Kreaturen und die noch finsterere Dunkle Magierin Varkur. Für diese Aufführung verließen die Darsteller sogar die Taverne zum Trunkenen Troll und nutzten den naheliegenden Wald samt der alten Burg Reichenau als Kulisse. Als Kreaturen verkleidete Andori zeigten ihre düsterste Seite. Ein Bewahrer vom Baum der Lieder traute zwei verliebte Bauersleute. Runensteine flüsterten verheissungsvoll von vergangenen Zeiten. Und hin und wieder kehrten manche Schauspieler zurück zur Taverne und spielten die erlebte Geschichte als Würfelspiel durch, auf dass die beiden Perspektiven einander perfekt ergänzten, bis hin zum ausgelassenen Siegesfest, welches selbstverständlich in der Taverne durch den Ausschank von Met, Ziegenmilch und Litschi-Limes ergänzt wurde. Ein beeindruckendes professionelles Werk\footnote{"Die Legenden von Andor" von Felix Merten: \url{https://www.youtube.com/watch?v=vXzOKKTNIbM}}.

Dann kam Yoda, in Dunkle Schatten gehüllt. Ein kleines grünes Männlein (ein Temm, vielleicht, auch wenn er selbst diese Spezies stattdessen als Goblins bezeichnete), das auf einen knorrigen Stock gestützt zur Erzählbühne watschelte. Es reiste schon seit langer, langer Zeit hierher, von einem weit, weit entfernten Ort. Einem Ort, in dem das Wort "Andor" ebenfalls groß geschrieben wurde, wenn auch aus ganz anderen Gründen. Und Yoda bot in ganz korrekter Satzstellung und angenehmer Stimmlage zweierlei Vorgeschichten dar, zunächst eine Erzählung von der Flucht von Brandurs Schar aus Krahd und Brandurs Kampf gegen den Drachen Tarok\footnote{"Die Flucht aus Krahd" von darkshadowyoda: \url{https://www.youtube.com/watch?v=Sy-Qx-av-KE}}, danach einen Überblick über den noch viel weiter zurückligenden Wirren des Unterirdischen Kriegs zwischen Zwergen, Drachen, Trollen und Riesen\footnote{"Der Unterirdische Krieg" von darkshadowyoda: \url{https://www.youtube.com/watch?v=Uzz773Fwy_4}}. Wie das Wunderkind Kreatok und der Drache Nehal die vier mächtigen Schilde aus uralter Zeit schmiedeten. Wie Nomion die Dunkle Hexerei entdeckte und der erste Krahder wurde. Düstere Geschichten, voller Humor und Spannung erzählt, ergänzt mit atmosphärischen Klängen, guter Musik und wunderschönen Bildern mmeisterhafter Hand.




\section{Heldenhafte Hintergrund-Historien}


Und dann wurde es wieder still in der Taverne. So vieles war gehört worden, von langen Geschichten über kollaborative Geschichten zu Kurzgeschichtensammlungen. Von Krahd bis Hadria, von Tulgor bis zum Barbarenland, von Zywall im fernen Westen bis Rundon im fernen Osten. Was gab es noch zu erzählen?

Da klopfte es an der Türe. Herein trat Tapsi Tost vom Schicksalsberg, ein gern gesehener und doch lange nicht mehr gesehener Gast der Taverne. Er winkte in die Runde und rief: "Hallo Leute! Ist ne Weile her, seit ich das letzte Mal hier war." 
Und kaum war Tost die neue Erzählbühne aufgefallen und erklärt worden, sprang er bereits stolz darauf und sprach los: "Na? Wie geht es euch so? Habt ihr es schön warm auf euren Stühlen? Dann kuschelt euch mal vor dem Kamin und lasst mich euch etwas über das Waldtost erzählen."
Und so erzählte Tost vom Waldtost, welches alle Butterbrote auf ihre Marmeladenseite fallen lässt\footnote{"Das Waldtost" von Tost: \url{https://die-legenden-von-andor.fandom.com/de/wiki/Benutzer_Blog:Tapsi_das_Waldtost/Das_Waldtost}}. 

Der Feuerdämon Kar éVarin räusperte sich aus der Ecke. Er führte eine flammende Feuervogelfeder, mit welcher er aschene Schriftzeichen in ein vor ihm liegendes Pergament brannte. Wundersamer Weise benötigte er dafür keine Tinte.
"Ich habe bereits Buch geführt darüber, was alles für Geschichten in unserer gemütlichen Gaststube vorgetragen wurden", sprach der Feuerdämon mit kehliger, kratziger Stimme, ohne aufzusehen\footnote{Kar éVarins Liste der Storytexte: \url{https://legenden-von-andor.de/forum/viewtopic.php?f=11&t=4432}}. Dann hielt er inne und ließ seine glühenden Augen über die angespannt wartenden Gäste wandern. "Aber eine Hintergrundgeschichte fehlt noch. Die meine." Dann erzählte Kar von der kleinen Mivar, die zu fasziniert vom Feuer gewesen war, von der Wassermagierin Jirid, die dem Dämon auf der Spur gewesen war, und von \textbf{Flammen}\footnote{"Kar éVarin": \url{https://legenden-von-andor.de/forum/viewtopic.php?f=11&t=4587}}. Ergänzt wurde die Geschichte um eine handvoll verschiedener Quellen, bei denen der Feuerdämon selbst noch herausfinden musste, wie genau sie zu seinem Wandel vom Dämon zum Feuerkrieger passen: Von der Wassermagierin Jarid, vom danwarischen Feuerschmied Melek, vom Kapitän Lunor und einigen anderen\footnote{"Quellen zu meiner Geschichte" von Kar éVarin: \url{https://die-legenden-von-andor.fandom.com/de/wiki/Benutzer_Blog:Kar_éVarin/Meine_Geschichte}}.

"Und meine Geschichte wurde bereits erzählt, doch will ich sie auch hervorheben!", fiepste eine Stimme. Der Butterbrotbär war zurück, und sprach stolz von einem großen braunen Brummbären, der von Gildas besten Butterbroten stahl und mysteriöse Schiefertafeln hinterließ, deren Entschlüsselung die Tavernengäste lange beschäftigt hatte\footnote{"Die Sage vom Butterbrotbären": \url{https://die-legenden-von-andor.fandom.com/de/wiki/Benutzer_Blog:Butterbrotbär/Die_Sage_vom_Butterbrotbären}}. "Und wie soll aus diesem großen Braunbären du kleiner sprachfähiger Teddy geworden sein?", rief jemand ein. Der kleine Bär schwieg grinsend.

Ein weiterer Bewahrer meldete sich zu Wort: "Ich bin Phlegon, Giftknödel, hier inzwischen wohlbekannt, doch war dem nicht immer so. Damals, vor meinen Recherchen zu den Düsteren Zeiten, war ich nur ein unbekannter Adept am Baum der Lieder, einer von zwei jungen Bewahrern. Und damals, im Sommer des Jahres 67 nach andorischer Zeitrechnung, schrieb ich folgendes ..." Und so erzählte Phlegon die Geschichte seines Spitznamens\footnote{Phlegons "Mein wahrer Name": \url{https://legenden-von-andor.de/forum/viewtopic.php?f=11&t=5205}}.

"Auch ich war einst unter einem anderen Namen bekannt", sprach ein Bursche, ein Wandergeselle von überall und nirgendwo. Einst hatte er als Schusterlehrling ohne Schuhe den Namen fusssohle getragen, doch sein wahrer Name ... war Pahl, der Wanderer. Und davon sprach er\footnote{Pahls Name des Wandergesellen: \url{https://legenden-von-andor.de/forum/viewtopic.php?f=11&t=5905}}.

Nachdem Pahl die Bühne verlassen hatte, lief eine Gestalt in einer schweren Rüstung vor die versammelte Tavernengemeinschaft. Gum Jabbar. Mit rauer Stimme berichtete er von seinen Abenteuern und denen seiner Begleiterin Kaylan. Von seiner Verbindung zu Wölfen und seiner und Kaylans Verfolgung wolfstötender Trolle ... wobei der wahre Wolfstöter womöglich gar kein Troll war\footnote{"Wolfstöter" von Gum Jabbar: \url{https://legenden-von-andor.de/forum/viewtopic.php?f=11&t=7073}}.

Große Augen wurden gemacht, als der schon lange verschollen geglaubte Hexer Meres sich vor die Gemeinschaft wagte und mit tonloser Stimme zu berichten begann. Er sei nicht beim Untergang der Schwarzen Kogge gestorben, nein, und er habe einiges verpasst, was sich seither zugetragen hatte. Doch als er zurückgekehrt war, war er von der Taverne zur Rietburg gereist, hatte Königin Chada auf seine überhebliche Art um Vergebung gebeten und den geflüchteten Kapitän Callem gefolgt ... oder wollte er sich ihm nicht lieber wieder anschließen? Eine Geschichte voller Anspielungen auf verschiedene Tavernenmitglieder und rumlose Piraten von fernen karibischen Inseln\footnote{Geschichten zu Meres' Verbleib: \url{https://legenden-von-andor.de/forum/viewtopic.php?f=11&t=7079}}.

Doch gerade als Meres zum Wiedersehen mit Kenvilars Tochter Kentar kam, wurde er unterbrochen vom Bewahrer Albus, der die Taverne betrat und an diesem Abend erneut eine Geschichte zum besten geben wollte. Bis Albus zur Erzählbühne gelangt war, war Meres ohne ein Wort des Abschieds verschwunden. Und so sprach Albus: "Dann will ich euch erneut davon erzählen, was sich zugetragen hatte, als ich diese Taverne zum ersten Mal betrat, damals, während der Ewigen Kälte. Selten ward ich an einem Ort so offen empfangen ... aber selten kommt es auch vor, dass Bewahrer vom Baum der Lieder den grünen Radius verlässt, und nie ohne guten Grund. Auch ich hatte einen guten Grund dafür." Und so erzählte Albus nicht nur von seinem ersten Tavernenbesuch, sondern auch vom Rätsel um das Verschwinden des Obersten Bewahrers Melkart\footnote{"Ein neuer Gast in der Taverne": \url{https://legenden-von-andor.de/forum/viewtopic.php?f=11&t=7950}}.

Nachdem Albus die Bühne wieder verlassen hatte, um mit einem feurigen Phoenix zusammenzusitzen, erhoben sich zwei weitere Protagonisten des "Abenteuers der Tavernengäste" und erzählte von ihren Hintergründen. Da war zum einen Nicopaos Runensteinen, Zauberer aus dem fernen Lande Magix\footnote{Nics "Nicopaos Runensteinen": \url{https://legenden-von-andor.de/forum/viewtopic.php?f=11&t=9153}}. Und da war zum anderen eine Steppenechse aus einem anderen fernen Land, dem Land des Weißen Gebirges\footnote{Steppenechses Macht der Hydra: \url{https://legenden-von-andor.de/forum/viewtopic.php?f=11&t=9160}}.

Die Tür schlug auf und eine Gestalt in einem grauen Mantel huschte herein. Sie schlug die Kapuze zurück und sprach zur Runde: "Ich habe von einer Wirtin gehört mit einer Stimme wie Gold, so warm wie die Ewigen Feuer aus Hadria. Ich würde gerne ein Lied hören, bitte. Kennt jemand ein Traditionelles Lied aus Andor? Mir ist bisher nur das Lied des Königs bekannt." Nachdem die Fremde einige Lieder vernommen hatte, stellte sie sich als Rekas Tochter vor, gesellte sich zur trauten Runde, erzählte von Varkur und vagen Erinnerungen und spekulierte zu ihrer wundersamen Herkunft\footnote{"Rekas Tochter": \url{https://legenden-von-andor.de/forum/viewtopic.php?p=109258\#p109258}}. 

Inspiriert von Rekas Tochter, trauten sich so viele andere Tavernengäste nun einen nach dem anderen hervor und beglückten die Tavernengemeinschaft mit kurzen Einblicken darin, wie sie zur Taverne fanden: Ein Feuervogel aus Danwar, der hadrische Zauberer Kamuna, das Irrlicht loro, ein als namenloser Wanderer auftretender Bewahrer namens Kahru, Dagain aus Rundon, Qurunatobra zweiter Sohn eines ersten Sohns und Freund von Liphardus siebtem Sohn eines siebten Sohnes, Schwertmeister Harthalts ebenso tapferer Sohn Malin ... verschiedenste Leute, von einem klassischen Schildzwerg aus Cavern über den Kundschafter Legolas Grünblatt aus den Elbenwäldern jenseits Krahds bis hin zum lächelnden Spike -- ein weiteres Zeichen dafür, was für ein Treffpunkt vielseitigster Leute Gildas gemütliches Gasthaus geworden war\footnote{"Storys zu Forumsmitgliedern": \url{https://legenden-von-andor.de/forum/viewtopic.php?f=11&t=9194}}.


\section{Und nun ... ?}

Und wieder einmal blieb die Erzählbühne leer, während die Stammgäste am Stammtisch fröhlich miteinander quatschten. In einer Ecke der Taverne saß eine einsame Gestalt an einem kleinen Rundtisch, die Kapuze tief in ihr Gesicht gezogen.

Ein Tablett voller Ziegenmilchkrüge gekonnt balancierend, bahnte sich die Wirtin Gilda einen Weg zum Rundtisch und stellte einen der bauchigen Krüge vor dir ab. "Aufs Haus!", rief sie dir zu, "Die erste ist immer frei für neue Gäste."

Du schlugst höflich die Kapuze zurück. Gilda lächelte dich an und sagte: "Und, hast Du Geschichten gehört, die Dir gefallen haben?" Sie nickte auffordernd zur leeren Erzählbühne hinüber. "Na, was kannst Du uns erzählen?"\footnote{Sendet neue Fan-Geschichten an \href{mailto:gilda@legenden-von-andor.de}{\texttt{gilda@legenden-von-andor.de}} !}

