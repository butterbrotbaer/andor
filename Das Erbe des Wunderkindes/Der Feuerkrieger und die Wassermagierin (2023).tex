\begin{chapterbox}
    \chapter{Der Feuerkrieger und die Wassermagierin (2023)}
    \label{Der Feuerkrieger und die Wassermagierin (2023)}
    \az{Jahr 68}

    \begin{center}
        Fortsetzung von \hypref{Stürmische Rätselnacht in der Taverne (2019 bis 2020)}
    \end{center}
    
    Ein weiteres Abenteuer von Jarid und Trieest aus dem fernen Danwar, vollgepackt mit sinistren Steinen, starken Skralen, Echos der Toten, seltsamen Alchemisten, klangvollen Wasserbecken, weisen Lehrpersonen, und vielleicht gar dem einen oder anderen altbekannten Helden von Andor, der lieber nicht im Rampenlicht steht.
\end{chapterbox}


\section{Der wilde Hraak}

\az{Jahr 61}

\textit{Halle des Ältestenrats, 61 a.Z.}\bigskip



„Nächste Aussagende. Vortreten, bitte. Name?“

„Jetzt tut doch nicht so, Älteste Freiga. Ihr wisst ganz genau, wer ich bin.“

„Protokoll ist Protokoll!“

„Und das gilt insbesondere für dich, Jarid! Sprich deinen Namen.“

„Mutter, du auch?!“

„Jetzt unterlasse bitte schön diese Spielchen. Dass du meine Tochter bist, verleiht dir keine Privilegien. Wenn du angehört werden willst, so halte dich gefälligst an die Gepflogenheiten des Ältestenrats. Mach’s dir nicht schwieriger, als es ohnehin schon ist.“

„Also gut, wie ihr wollt. Mein Name ist Jarid Morgentau, Wassermagierin des dritten Zirkels. Was für eine Überraschung.“

Gekritzel, Pergamentgeraschel.

„Jarid Morgentau. Sind Sie hier, um den Täter zu verteidigen?“

„Ich bin hier, um an die Vernunft des Ältestenrates zu appellieren. Bei den Wogen der Wellen, Trieest ist doch bloß noch ein Kind! Wenn er ...“

„Ruhe! Jetzt ich noch nicht die Zeit für Ihre Rede. Zunächst müssen wir die Formalitäten hinter uns bringen. Halten Sie sich für fähig, ein neutrales Urteil über den Vorfall abzugeben?“

„Natürlich nicht, keiner hier ist das! Aber ich habe Trieest seit klein auf begleitet, ich kann dafür bürgen, dass er sich üblicherweise ganz anders verhält. Ich hatte einige Tage zuvor bereits ein Gespräch mit seiner Lehrerin Ganwa über seine Stimmung gesucht und ich war die erste Wassermagierin, die am Unfallort eingetroffen ...“

„Völliger Schwachsinn! Einen ‚Unfall‘ willst du das nennen?!“

„Die Gepflogenheiten gelten nicht nur für meine Tochter, sondern auch für dich, Älteste Freiga! Rede mir nicht drein. Denk an das Protokoll.“

„Ich steck dir dieses Protokoll gleich sonst wo hin, Rowinda, mein Enkel hat sein Augenlicht verloren wegen dieses Ungeheuers!“\bigskip







\az{Jahr 68}

\textit{Sieben Jahre später.}\bigskip



Der wilde Hraak stieß ein ohrenbetäubendes letztes Brüllen aus, in einer Tonhöhe, die für Menschen knapp unhörbar, für Trieest jedoch nur allzu gut wahrnehmbar war. Noch vor wenigen Minuten hätte das Gebrüll des Hraaks ungemeine Furcht ausgelöst in all jenen, die es zu hören vermochten. Doch dem war nicht mehr so. Stattdessen schien die Furcht nun im Geheul des finsteren Wesens zu liegen. Falls der Hraak überhaupt dazu imstande war, Furcht zu empfinden.

Trieest grunzte zufrieden und stach mit seinem gewundenen Rankenschwert tiefer am Geweih des stinkenden Ungetüms vorbei in den Schädel des Biests. Seine Klinge sog gierig die Dunkle Hexerei auf, die den verwesten Körper in Bewegung hielt. Trieest befahl den Ranken seines Schwerts, den Ursprung des Übels zu finden. Er fühlte, wie sich die Klinge unter seiner geistigen Führung in viele kleine Stränge teilte, durch Fleisch und Knochen stach und schließlich die steinerne Quelle des Bösen fand, welcher hinter der Stirn des Hraaks saß. Triumphierend ließ Trieest die Ranken des Schwerts um den Kristall winden und zog einmal kräftig an seinem Schwert.

Ein melodisches Klingen erklang. Der Hraak bäumte sich ein letztes Mal auf und stürzte dann zu Boden, als Trieest sein Schwert an sich riss. Die Spitzen der Rankenklingen umfassten einen rötlich schimmernden Kristall. Trieest vermochte nicht immer, Farben zu unterscheiden, doch diese strahlende Kristallfarbe war unverwechselbar.

Trieest blieb einige Lidschläge lang angespannt, doch der Körper des Hraaks lag verrenkt im Matsch und regte sich nicht mehr. Die lange Zunge lag im Dreck, die Schlitzaugen waren geschlossen. Das Wesen war besiegt.

Der Kampfrausch verflog so schnell, wie er gekommen war. Trieest sank zusammen und ließ sein Rankenschwert fallen, dessen blutige Spitzen weiterhin den roten Kristall fest umschlossen. Die zwei dünnen Ringe aus orangeroten Feuerschlieren, die Trieest im Kampf umgeben hatten, erloschen. Seine ausgebrannte Brust schmerzte.

Dann erinnerte er sich und sein Kopf schoss erschrocken wieder in die Höhe: „Jarid!“

Hektisch sah er sich um. Die Wassermagierin lag bewegungslos am Rande der Waldlichtung, ihr linker Arm in einem unnatürlichen Winkel abgespreizt. Trieest stemmte sich hoch und humpelte zu ihr hinüber. Erleichtert erkannte er, dass ihre Brust sich hob und senkte, wenn auch unregelmäßig.

Erinnere dich, du tumber Troll!, fluchte er. Was hatte Jarid nur schon wieder über ohnmächtige Personen gesagt? Auf den Rücken drehen? Auf die Seite? Ja, die Seite war es gewesen!

Vorsichtig fasste Trieest die bewusstlose Jarid an den Schultern und drehte sie auf die Seite. Ihr verletzter Arm streifte den Waldboden unsanft. Jarid zog scharf Luft ein, riss ihre Augen auf und murmelte etwas Unverständliches.

„Jarid?“, fragte Trieest ängstlich, doch die Wassermagierin hatte ihre Augen bereits wieder geschlossen und ihre Hand fiel schlaff zu Boden. Erst jetzt roch Trieest aktiv den metallischen Gestank menschlichen Blutes in der Luft und sein Blick richtete sich auf die üblen Kratzspuren in Jarids Bauchgegend.

Sporndreck! Das Biest hatte sie erwischt! Was nun? Denk, tu törichter Arrog, denk!

Die Wunde musste gereinigt und danach geschlossen werden. Aber frisches Wasser besaßen sie keines mehr und Verbandsmaterial hatten sie gar nicht erst dabeigehabt.

Ratlos riss Trieest einen Fetzen seines roten Gewands ab und presste diesen auf Jarids verwundeten Bauch. Jarid stöhnte schwach auf, woraufhin Trieest gleich wieder von ihr abließ. Wenn er auf die Wunde drückte, würde nur noch mehr Blut herausquellen. Das war schlecht ... oder? Irgendwie musste der Blutfluss doch gestoppt werden, aber draufdrücken konnte nicht gut sein? Verflucht noch mal! Jarid wüsste, was zu tun ist. Oder Reka. Irgendein Hexer. Das andorische Dorf hinter der nächsten Hügelkette, welches so lange vom Hraak terrorisiert worden war, hatte einen guten Heiler. Doch würde Trieest die Strecke nicht schnell genug hinter sich bringen können, erst recht nicht in diesem Zustand.

Wütend schlug Trieest mit der Faust auf den feuchten Waldboden. Das brachte natürlich nicht viel, außer dass einige Apfelnüsse von einem nahegelegenen Baum purzelten. Bäume! Pflanzen! Vielleicht wuchsen hier irgendwo Heilkräuter? Wie sah Wolfskraut schon wieder aus? Und diese Pilze dort drüben ... die erkannte Trieest an der lustigen Form, das waren Zauberhutpilze! Waren die nun schon wieder giftig oder wundheilend? Haariger Höhlenwicht, warum hatte er nicht besser aufgepasst, als ihm solche Dinge in der Ausbildung zum Feuerkrieger gezeigt worden waren?!

Er durfte es nicht riskieren, Jarid die falschen Kräuter einzuflößen. Aber irgendetwas musste Trieest doch tun können! Unsicher griff Trieest nach Jarids kalter Hand (derjenigen, deren Arm nicht in einem unpassenden Winkel abstand) und führte sie zu seiner Brust. Dort, tief in seinem Körper versenkt, pochte der orange-rote Lavastein, welcher ihm aufgebürdet worden war. Sieben Jahre lang hatte er diesen Stein nun schon getragen. Dieser Stein hatte ihm bereits unzählige Male das Leben gerettet und ebenso viele Male das Leben zur Hölle gemacht. Doch in diesem Moment wirkte der Stein ... geradezu euphorisch?

Es war Trieest zu Beginn seiner Zeit als Lavastein-Träger falsch vorgekommen, einem aus dem Felsen Danwars geschlagenen Stück Stein Gefühle zuzuschreiben, doch inzwischen konnte er schon lange nicht mehr leugnen, dass er immer wieder Eindrücke und Emotionen verspürte, die nicht die seinen waren – diese Stimmungen mussten vom Lavastein kommen. Das siegreiche Schlachten des Hraaks und das Anzapfen dessen dunkler Energie hatten den Lavastein offenbar glücklich gestimmt. Vielleicht gab er sich gnädig. Trieest konnte nur hoffen.

Er zögerte noch kurz, doch dann ließ er sich darauf ein und drückte Jarids Hand auf den Edelstein. Etwas zischte. Der sonst oft kühle Stein war unnatürlich heiß, ein weiteres Zeichen seiner Erregung. Bitte, flüsterte Trieest innerlich, ich weiß, dass du mich nicht magst, doch Jarid hat dir und den Danwaren nichts getan, Jarid hat mich so oft davon abgehalten, dich aus mir rauszureißen, und sie wird es wieder tun. Beim Rauschen der Wellen und beim Knistern des Feuers. Beim Barte des Warx! Ich schwöre, mich nicht mehr gegen dich zu stellen.

„Rette sie! Bitte!“

Trieest hatte kaum Zeit zu realisieren, dass er die letzten Worte laut geschrien hatte, da setzte bereits das wohlbekannte Ziehen in seiner Brust ein. Feuer füllte seine Innereien, schoss durch seine Adern und bohrte sich in seinen Kopf. Trieest unterdrückte einen Aufschrei und biss sich auf die Zunge, als sein gesamter Körper sich verkrampfte. Doch ließ er Jarids Hand nicht los, drückte sie nur noch fester an sich und an den brennenden Stein, bis er hörte, nein, spürte, wie ihre Knochen unter seinem Griff knacksten.

Trieests Körper stand in Flammen. Nicht buchstäblich, nein, aber so fühlte es sich zumindest an. Sein Sichtfeld verging in der Farbe des Feuers und des Rauchs, seine Kehle wurde staubtrocken und jeder Atemzug fühlte sich an, als stünde er wieder an der Stätte der heiligen Flammen im Land der drei Brüder oder inmitten eines riesigen Sandsturms in der tulgorischen Wüste. Und während Trieest ächzte und haderte, leuchtete der Lavastein auf, heller und heißer als je zuvor, und verbrannte Jarids Hand.

Hätte Trieest noch sehen können, so hätte er bemerkt, wie Jarids Wunde sich in Minutenschnelle schloss, bis nur noch eine feine Narbe unter einem Riss im Stoff ihres Gewands zu sehen war. Doch da Trieest nichts mehr sehen konnte, verblieb er in seiner Agonie, die einzigen Gedanken, zu denen er noch fähig war, waren, dass er nicht loslassen durfte, dass er Jarid halten musste, zerbrechlich, wie sie war, denn wenn sie nicht mehr wäre ... sie durfte nicht nicht mehr sein ...

Wie lange Trieest in diesem Zustand lag, konnte er nicht sagen. Aber plötzlich war sie wieder da, Jarid, sie war hier, und wach, und gesund, und sie löste sich aus seinem Griff. Doch der Schmerz und das Brennen des Lavasteins in seiner Brust blieb, und Trieest griff unverständig nach Jarid, warum wollte sie gehen, sie durfte nicht gehen, er brauchte sie doch.

Kühles, glänzendes Nass ergoss sich über ihn, als Jarid mit einigen Gesten ihrer Finger Tautropfen vom sie umgebenden Gras auf Trieests Stirn leitete. Es waren Tropfen auf einem heißen Stein, und doch war diese Linderung zunächst alles, was Trieest brauchte, um seinen Verstand etwas zu klären.

Es hatte gewirkt! Der Stein hatte Jarid geheilt! In seinem schmerzerfüllten Zustand konnte Trieest sich nicht leicht darüber freuen, aber eine frohe Nachricht war es dennoch allemal. Und nun war das kühle Nass nicht nur auf seiner Stirn. Überall an seinem Körper ergossen sich kühle Blasen kalten Wassers, befeuchteten aufgesprungene Haut und spülten frische Schrammen aus vom Kampf mit dem Hraak aus.

„Tri, sag mal. Wie kommt es eigentlich dazu, dass, selbst wenn mein Leben von dir gerettet wurde, am Ende dennoch du derjenige bist, der dringend Pflege nötig hat?“, fragte Jarid schmunzelnd. Dann musterte sie fasziniert ihre versengte Hand. „Der Lavastein hat ja ganze Arbeit geleistet, selbst meine Schulter scheint er wieder eingerenkt zu haben. Ich wusste gar nicht, dass er das kann.“

„Ich ... ich auch nicht“, hustete Trieest vom Waldboden her. Aufstehen schien gerade nicht drin zu liegen. „Ich konnte es nur hoffen. Und wusste bloß, dass es nötig war.“

Jarid sagte nichts, doch Trieest konnte sehen, wie es in ihrem Kopf ratterte. Dann wechselte sie das Thema und sah Trieest aus zusammengekniffenen Augen an.

„Wie fühlst du dich, wackerer Krieger?“

„Wie ein Feld aus Rietgrasblüten, nachdem eine Horde Trolle darüber getanzt hat. Aber das ist nicht das ...“

„... was ich meine, genau.“

„Nun ja, ich fühle mich ... nicht anders als vorher. Nur mit mehr blauen Flecken.“

Jarid ließ ihre Schultern hängen und in ihren Augen blitzte Enttäuschung auf, woraufhin sie sich etwas abwandte. Einen Augenblick später drehte sich zurück, ein erzwungenes Lächeln auf den Lippen: „Nun denn, dann wollen wir uns mal daran machen, deine blauen Flecken zu heilen. Und dann zurück zum Dorf. Dorfälteste Ronder wird überaus erfreut sein, wenn wir ihr den Kopf des Hraaks präsentieren können.“

„Warte noch einen kurzen Augenblick“, bat Trieest und hob seinen Arm. Jarid legte ihren Kopf schräg und half ihm dann auf.

Trieest stützte sich an einen Baumstamm und atmete tief durch. Dann setzte er an und seine tiefe Stimme stockte einige Male, während er die richtigen Worte suchte:

„Das Überwinden des Hraaks war offensichtlich nicht die Aufgabe, deren Erfüllung meine Verwandlung vollziehen wird. Aber ...“

Erst jetzt löste sich der Knoten der Furcht in Trieests Brust, dass Jarid nicht überleben würde, und die Enttäuschung überkam ihn mit voller Wucht. Eine derartige Herausforderung wie den Hraak hatten sie seit Monaten nicht mehr erlebt. Wenn selbst die nicht genug war ... wie lange würde er ... würden sie beide noch mit dem Stein in seiner Brust hadern müssen?!

Damals, als er einer seltsam klaren Vision nach Andor gefolgt war, um den Andori gegen die Ewige Kälte auszuhelfen und einen Winterstein aus den Klauen einer eisigen Kreatur zu entreißen, war er sich sicher gewesen, dass dies die Aufgabe war, die ihn von seiner Bürde befreien würde. Ein Feuerkrieger im Kampf gegen den Geist einer Eisdämonin, das passte doch zu gut. Doch daraus war nichts geworden.

Inzwischen wusste dank der alten Gerüchteköchin Allanta bestimmt jeder halbwegs informierte Bauer diesseits des Fahlen Gebirges von Trieests Bürde. Dass er irgendeine großartige glorreiche Aufgabe erfüllen musste, um seine Wandlung zu vollziehen. Auch wenn die wenigsten ahnten, was diese Wandlung und sein Lavastein für ihn bedeuteten. Wann immer er und Jarid in der Taverne zum Trunkenen Troll einkehrten, hatte die Wirtin Gilda oder ein fröhlicher Tavernengast wieder ein neues Rätsel oder eine neue Aufgabe für ihn bereit, in der Hoffnung, dass ihm dies helfen würde. Und das schlimmste daran war, dass Trieest nicht einmal mit Sicherheit sagen konnte, dass seine glorreiche Aufgabe am Ende mehr als ein halbwegs herausforderndes Kreuzworträtsel würde. Warum mussten die Stimmen der Roten Grotte auch nur so verschnörkelte Prophezeiungen geben?!

Die Rote Grotte. Trieest grummelte schon alleine beim Gedanken daran. Zu schicksalshaften Zeiten schickte der danwarische Ältestenrat Danware zur Roten Grotte, um den Stimmen der Toten lauschen. Trieest verband keine positiven Gefühle damit. In der Roten Grotte waren die Echos unzähliger Verstorbener gefangen, und wer sich zu lange darin aufhielt, drohte, verrückt zu werden. Trieest hatte selbst im endlosen Gemurmel und Geschrei nur das Rauschen des Meeres vernommen, keine einzige sinnvolle Aussage. Die Ältesten waren zwar sehr enttäuscht gewesen, als er ihnen das mitgeteilt hatte, hatten dann aber mit den Schultern gezuckt und sich nicht weiter darum gekümmert. Vielleicht konnten ja nur „wahre“ Danware die Stimmen ihrer Vorfahren hören? Trieest, versuchte, die Erinnerungen abzuschütteln, doch diese waren hartnäckig.

Jarid musterte Trieest besorgt, als er nicht weitersprach. Ihre Augen wanderten argwöhnisch seinen Körper entlang. Trieest wusste, dass sie nach größeren Wunden suchte, inneren Blutungen und dergleichen. Sie würde nichts finden, nichts außer diesem verfluchten Stück Feuer in seiner Brust!

Trieest spürte, wie der Stein sich über seinen Missmut ärgerte und gefährlich heiß wurde. Er versuchte, das Gefühl zu verdrängen und sammelte sich.

„Ich bin enttäuscht darüber, dass ich weiter diese Bürde tragen muss, und das ist absolut verständlich. Aber wie kommt es, dass du enttäuscht bist, Jarid? Solltest du nicht bereits längst wissen, wann ich meine Bürde ...“

Schnell brach Trieest den Satz wieder ab, als Jarid das Gesicht verzog und sich brüsk abwandte. Seit jener stürmischen Nacht im Trunkenen Troll, in welcher ein blinder Seher ihm eine falsche Prophezeiung hatte andrehen wollen, hatte Trieest es nicht mehr gewagt, Jarids Orakelspruch aus der Roten Grotte anzusprechen. Bis heute, hier und jetzt. Trieest sammelte seine Frustration und sprach weiter:

„Warum bist du enttäuscht? Du sagtest, du wüsstest, wo mein Prozess des Wandels beendet werden wird. Solltest du nicht bereits wissen, wo meine letzte Aufgabe stattfindet? Wusstest du nicht schon seit langem, dass der Hraak uns nicht weiterbringen wird?!“

Das Gesicht immer noch abgewandt, entgegnete Jarid scharf: „Der Hraak mag uns persönlich nicht weitergebracht haben, aber seine Abwesenheit bedeutet das Überleben unzähliger Dorfbewohner! Sind solche Heldentaten denn nichts wert?“

„Natürlich sind sie das wert! Aber nicht in Bezug auf den elenden Lavastein vor meinem Herzen!“, brauste Trieest auf. Dann fing er sich wieder und fuhr fast flehend fort: „Du musst nicht enttäuscht tun, Jarid, du musst doch gewusst haben, des der Hraak uns nicht persönlich weiterbringt, und das ist auch nicht so wichtig im Bezug aufs große Ganze. Kannst du mich in Zukunft bitte einfach nicht mehr einer falschen Hoffnung erliegen lassen? Ich weiß nicht, wie viel ich noch zu verkraften vermag. Du magst eine tapfere Heldin sein, Jarid, aber ich bin keiner. Solche wie mich gibt’s doch wie Sand am Meer. Und ich will nicht wie einer dieser hoffnungslosen Glutträger enden!“

Jarid blieb stumm und trat einige Schritte in Richtung des Hraaks. Dessen Kadaver lag immer noch still in der Mitte der Waldlichtung und rührte sich nicht.

Die ersten Sonnenstrahlen des neuen Tages durchbrachen die östlichen Baumwipfel und erhellten den Waldboden, fingen sich im kleinen roten Kristall, welcher in Klingenranken verschlungen auf der Spitze von Trieests Schwert steckte. Das Schwert lag immer noch achtlos weggeworfen am Boden, als Jarid vor es trat. Sie bückte sich und hob die Waffe hoch, drehte sie leicht im Morgenlicht, als würde sie die Spiegelungen betrachten. Trieest glaubte allerdings zu wissen, dass Jarids Gedanken nicht beim Schwert lagen. Schwer atmend wartete er auf eine Antwort.

Endlich durchbrach Jarid die Stille. Leise bat sie: „Lass deinen Ärger bitte nicht an mir aus.“

„Natürlich nicht“, erwiderte Trieest und trat näher, „Dennoch hast du ...“

„Mir gefällt es auch nicht!“, unterbrach Jarid ihn nun, „Nicht im Geringsten! Wir sitzen hier im selben Boot. Ich hoffe genauso wie du, dass jede Herausforderung die letzte wäre. Ich will dich frei und glücklich sehen, nicht leidend!“

Jetzt war es an Trieest, nach Worten zu stammeln. Natürlich hatte sie recht, natürlich wollte sie ihn nicht leiden sehen, aber das war es nicht. Sie wusste etwas, was er nicht wusste, so viel wusste er, und doch wusste er nicht, welche Sorte von Wissen Jarid in der Roten Grotte erhalten hatte. Jarids Aussagen dazu konnte er nicht richtig deuten, und das machte ihn wahnsinnig. Nein, eigentlich war es sein Schicksal, das ihn wahnsinnig machte. Aber das änderte nicht daran, dass er mehr wissen wollte über seine Zukunft.

„Seit sieben Jahren arbeiten wir uns nun schon durch Quest nach Quest, wann ist es denn endlich vorbei?“

„Sei nicht ungeduldig, Tri. Wir haben vielen Leuten geholfen. Und den Titel ‚Held von Andor‘ erhält man nicht leichtfertig.“

„Ach, jetzt hör schon auf! War das nicht bloß eine politische Entscheidung? Wie kommt es, dass wir Danware beide den Titel verliehen bekommen haben, nicht aber dieser Brolaf, der lautstarke Herold der Schildzwerge, oder dieser famose Bogenschütze, der das Auge des Drachen getroffen hatte? Der, der fast so wie ich hieß? Weest? Feest? Egal. Nein, dieser Titel, ‚Held von Andor‘, der verleiht Anerkennung, und Macht, und so vergibt ihn König Thorald nur an diejenigen, die nicht mit ihm streiten mögen. Mit wahrem Heldentum hat das nicht viel zu tun, sondern nur damit, dass wir Danware von weit her sind statt politische Rivalen des Königsreichs.“

Jarid blickte Trieest nur stumm an, bis dieser schließlich zu reden aufhörte und leise brummelte: „Ist ja gut, ich bin wieder zu zynisch, ich meins ja nicht so. Und zugegebenermaßen wurde auch schon diese Flussländlerin mit dem danwarischen Stab zur Heldin ernannt, obwohl das die potentielle Unabhängigkeit der Flusslande vom Königreich etwas wahrscheinlicher macht. Vielleicht steckt hinter dem Titel doch mehr. Und wenn ich meine Bürde erst einmal los bin, helfe ich gerne weiter als Held aus. Aber weil ich will, nicht, weil ich muss!“

Trieest atmete tief durch und fuhr fort: „Und habe ich nicht ein Recht darauf, meine Bürde abzulegen! Sieben Jahre! Sieben Jahre sind das nun schon! Wie lange will Mutter Natur mich denn noch quälen?! Warum kannst du mir nicht verraten, was die Echos der Toten dir in der Roten Grotte sagten? Wie soll mir das denn helfen, meinen Prozess des Wandels zu beenden?!“

Stille. Natürlich Stille. Inzwischen schrie Trieest auch nicht mehr Jarid an, sondern vielmehr die ganze Welt, die sich seinem Prozess des Wandels in den Weg zu stellen schien.

Das war natürlich nicht wahr. Nicht die ganze Welt hatte sich gegen ihn verschworen. Vermutlich nicht einmal sein Schicksal. Er war bloß enttäuscht, und das war in Ordnung, das war verständlich, aber das gab ihm nicht das Recht, ausfällig zu werden.

Trieest schnaubte ein letztes Mal, faltete seine Hände und atmete tief durch. Er sog alle Gerüche der Umgebung ein. Der Schweiß auf seinem Körper, Jarids Blut, der Duft des spätherbstlichen Waldes, selbst der Gestank des Kadavers des Hraaks. Trieest hörte die Vögel zwitschern. Einige Bäume weiter drüben huschte ein Streifenmarder quietschend durchs Geäst. Der Feuerkrieger nahm alles wahr, war sich allem bewusst, und versuchte, es einfach anzunehmen. Seine Emotionen fließen zu lassen, wie er es schon so oft versucht hatte.

„Weiß sind die Seemöwen nicht nur im Winter“, sprach er die leise Floskel der Entschuldigung.

Er linste zu Jarid, welche sich sorgfältig aufgerichtet hatte und die frisch verheilte Haut unter ihrem zerrissenen Gewand abtastete, gefolgt von ihrem linken Arm. Erst, als sie sich davon überzeugt hatte, dass nichts mehr gebrochen war, blickte sie Trieest wieder an. Ausdruckslos nestelte sie an der Einhorn-Horn-Schnalle ihres Umhangs herum.

„Weiß sind die Seemöwen nicht nur im Winter“, wiederholte Trieest etwas lauter.

„Sobald der Schnee geschmolzen ist, ist er nicht mehr weiß“, antwortete Jarid rasch und ein nur leicht gezwungenes Lächeln huschte über ihr Gesicht.

Vorsichtig humpelte sie auf den Feuerkrieger zu: „Komm, Tri, gucken wir uns an, was wir aus dem Schädel des Biests errungen haben.“

Trieest folgte ihrem ausgestreckten Finger zum Rankenschwert, welches immer noch etwas abseits am Boden lag. Dessen Spitze umschloss weiterhin den großen roten Kristall, der im Kopf des wilden Hraak gesteckt hatte. Dort, wo sich bei einem handelsüblichen Tier der Denkapparat befinden sollte.

Trieest war nicht riesig erpicht darauf, herauszuknobeln, um was für Magie es sich dabei handeln konnte und wie dieses Wesen wohl zustande gekommen war. Aber Jarid liebte solche Untersuchungen, und eine solche könnte Trieests Gedanken von seinem weiterhin unerfüllten Prozess des Wandels nehmen.

So griff Trieest vorsichtig nach seinem Rankenschwert und führte es vor seine Augen, ganz nahe, dass er den Kristall endlich scharf sehen konnte. Komplizierte Runen waren in seine glatte Oberfläche geritzt worden, vielleicht gar welche mit magischer Bedeutung. Die Runenmeister der Silberzwerge würden bestimmt mehr dazu wissen. Aber Iril reiste gerade mit Aćh durch Tulgor, oder? Dann würden Jarid und Trieest wohl einen anderen Experten finden müssen.

Trieests Blick wurde vom Edelstein wie magisch angezogen. Er leuchtete in einem satten Dunkelrot. Unter der steinernen Oberfläche waren hin und wieder dunkle Schwaden zu erkennen, welche willkürlich im Innern des Kristalls umherwaberten. Etwas Böses war das, das fühlte Trieest in seinem Magen.

Angewidert befahl den Ranken, sich zu teilen. Mit einem rostigen Knirschen glitten die Klingenstücke wieder an ihre angestammten Plätze und ließen den roten Kristall fallen – direkt in Trieests ausgestreckte Hand.

Sobald der Kristall seine Handfläche berührte, wurde Trieest schwarz vor den Augen. Das letzte, was er wahrnahm, war eine konzentrierte Bosheit von Etwas, das sich in seinen Geist bohrte, durch seinen Körper strömte und ihn mit Gefühlen von Wut und Hass erfüllte. Dann umfing ihn der Mantel der Ungewissheit und Ohnmacht.\bigskip







Vorsichtig tastete Jarid erneut ihren Arm ab. Konnte es wahrlich sein, dass Trieests Lavastein ihre Wunde geheilt hatte? Sie wusste über die unglaubliche Macht, die der Stein barg, doch derart direkte Heilkräfte hatte er bislang nie gezeigt, und erst recht nicht bei ihr.

Ein tiefes Grollen riss sie aus ihren Gedanken. Erschreckt drehte sich Jarid um und sah sich dem zähnefletschenden Gesicht Trieests gegenüber. Dessen Augen hatten sich tiefschwarz verfärbt und dunkler Geifer lief aus seinem leicht geöffneten Mund. Sein Kopf zuckte unwillkürlich und seine Hände hatten sich zu Fäusten geballt. Das Rankenschwert in seiner Hand zuckte ebenfalls und verformte sich in zwei verschlungene Geweihe. Diese sahen dem Geweih, welches der verstorbene Hraak immer noch auf dem Schädel trug, unangenehm ähnlich. Und der Lavastein ... zum ersten Mal, seitdem Trieest den Lavastein in seine Brust eingesetzt bekommen hatte, leuchtete er gar nicht, sondern steckte fahl in Trieests Brust, als wäre er eine unbelebte Verzierung seiner eleganten Rüstung. Jarid glaubte gar, finstere Schemen unter der Oberfläche des Lavasteins umherhuschen zu sehen.

Dann sah Jarid den rot glühenden Kristall, welcher in Trieests linker Hand steckte. Der Kristall, den Trieest aus dem Schädel des Hraaks gezogen hatte, dort, wo sie ein Gehirn vermutet hätte.

Rasch wich Jarid zurück und stolperte prompt über eine nervige Wurzel am Boden mitten in eine Ansammlung Waldpilzen, welche bei Jarids Aufprall eine Menge Sporen entließen. Jarid unterdrückte den Niesreiz. Pilze, so spät im Jahr?

Trieest, oder vielmehr die böse Macht, die Trieests Körper steuerte, fand sich in seinem Körper offenbar noch nicht vollkommen zurecht. Finster-Trieest tat einige stolpernde Schritte, während sein Mund weiterhin unbekannte Wörter formte und seiner Kehle ein gutturales Dröhnen entsprang. Mit pechschwarzen Augen sah er um sich und fixierte die am Boden liegende Jarid. Er ließ einen Fuß nach dem anderen auf den Waldboden krachen und bewegte sich Stück um Stück auf sie zu.

„Keinen Schritt weiter!“, polterte Jarid um einiges selbstsicherer, als sie tatsächlich war. Sie mochte kein Schwert tragen, war aber definitiv eine ernst zu nehmende Gegnerin.

Finster-Trieest blieb tatsächlich stehen und musterte sie aufmerksam, den Körper hin- und herwiegend. Auch diese Bewegung erinnerte Jarid ungut an den Hraak, der stets auch so auf seinen Beinen umhergestakst war.

„Wer bist du?“, rief Jarid nun aus.

Finster-Trieest verzerrte seine Lippen zu einem Lächeln und ein schmerzhaft klingendes heiseres Flüstern entsprang seinem Mund: „Ahhh, ein menschlicher Körper. Naja, halbmenschlich zumindest. Wie sehr ich das vermisst habe. Der Hraak, dieses Untier, war wahrlich kein würdiges Gefäß für meine Wenigkeit.“

Jarids Blick schweifte panisch über ihre Umgebung. Allerlei Tautropfen waren da zu finden, aber keine Wasserfläche, welche groß genug gewesen wäre, um einen Wassertunnel zu errichten. Nun, eigentlich wollte sie Trieest ohnehin nicht allein lassen. Zeit, sich diesem Monster zu stellen.

Finster-Trieest sprach krächzend weiter und nahm nun wieder Schritt in Richtung Jarid auf. Offenbar konnte er von Augenblick zu Augenblick besser mit diesem fremden Körper umgehen.

„Du fragst mich nach meinem Namen. Habe ich noch einen Namen? Lang ist es her, dass man mich nannte. Ich fühle mich nicht mehr nach jemandem, der einen Namen trägt.“

Jarid ließ sich nicht davon abbringen. „Und wie nannte man dich, als du noch einen Namen trugst?“

Finster-Trieest verzog das Gesicht. „Die Druidin, die mich in diesen verfaulten Körper, diesen Hraak, hineinpresste, nannte mich das pure Böse. Dass ich Tod und Zerstörung über so viele Leben bringen würde, ja, bereits gebracht hätte. Völliger Schwachsinn, ich bin nur ein einsamer Geist, der viel zu lange ohne ein Gesicht und ohne Hände auskommen musste. Meine dunkle Macht ist genau so wenig böse wie es ein Schwert ist; es kommt darauf an, wie man sie nutzt.“

„Und doch hast du deine dunkle Macht für Böses genutzt, nehme ich an?“, versuchte Jarid, das Wesen weiter reden zu lassen,

„Ich habe sie für mich genutzt. Wer an meiner Stelle würde das nicht?!“, sprach Finster-Trieest, Zornesfalten in der gefurchten Stirn. „Ich musste töten, um zu überleben. Und ich bin bereit zu töten, um die offenen Rechnungen meiner Vergangenheit zu schließen. Wer würde das nicht? Ich bin nicht das Böse! Warum nannte sie mich so?!“

„Ich nenne dich nicht so, wenn du nicht willst. Du sagtest, du hättest einen Namen gehabt, bevor diese Druidin dich in den Hraak steckte. Wie lautete dieser Name?“

Finster-Trieest grinste. „Ja, du würdest diesen Namen vielleicht bereits kennen. Aber ich fühle mich nicht mehr damit verbunden. Es bringt mich nicht weiter, mich hier von dir ausfragen zu lassen. Lass mich stattdessen dich nach deinem Namen fragen, o blaugewandte Andori, damit ich weiß, wessen Namen ich nennen soll, wenn ich deine Heimat zugrunde richte.“

Jarid dachte nicht daran, ihren Namen auszusprechen – Namen besaßen Macht, das wusste sie – und so tat sie das, was ihr im Augenblick am sinnvollsten erschien: Sie krümmte ihre Finger und fokussierte all ihre Gedanken auf das Wasser in der Luft und im Boden, in ihrer gesamten Umgebung ... und auf das Wasser in Trieests Blut. Sie wusste, dass ihre Magie niemals ausreichen würde, um seinen inneren Blutstrom völlig zu kontrollieren und den Körper bewegungsunfähig zu machen, aber das war auch nicht nötig. Hauptsache, ihr Gegner war verwirrt. Und abgelenkt.

Jarid hob ihre Arme ruckartig in die Höhe und die Tautropfen in ihrer Umgebung folgten ihrer Bewegung ebenso wie ein kleiner Teil in Trieests Blut. Noch während ein verwirrter Ausdruck über Finster-Trieests Gesicht zog, als seine spitzen Ohren rötlich anliefen und seine Beine zu zittern begannen, stürzte sich Jarid auf ihn, packte seine beiden Handgelenke mit ihren Händen und zog ihn zu Boden. Trieests Rankenschwert fiel aus seinem Griff, doch die Faust mit dem roten Kristall hielt er fest geschlossen, und nun hatte er sich von der Überraschung erholt.

„Nun, wenn du mir deinen Namen nicht sagen willst, werde ich wohl einfach in den Erinnerungen des Feuerkriegers nachforschen müssen“, grinste Finster-Trieest ungetrübt, während er mit Jarid rang. Nach einer kurzen Pause erwiderte er triumphierend: „Aha, du bist gar keine Andori! Jarid! Jarid Morgentau aus dem fernen Danwar. Jarid, die Brunnenspringerin. Was für ein ordinärer Name. Was für eine ordinäre Person du doch bist.“

Mit einem groben Tritt stieß Finster-Trieest Jarid von sich. Kurz danach grub sich seine immer noch den Kristall umklammernde Faust in ihre Magengegend. Jarid japste nach Luft. Ihre flache Handkante fand Finster-Trieests Hals. Nun lagen beide Kontrahenten nebeneinander auf dem Waldboden und schnappten nach Luft.

Jarid rollte sich als erste weg, Finster-Trieest war dafür als erster wieder auf den Beinen. Sorgsam darauf achtend, die Faust mit dem roten Kristall nicht zu öffnen, tastete er seinen Hals ab und war offenbar zufrieden mit dem Ergebnis, denn nun wandte er sich wieder Jarid zu.

Vor zwei mächtigen Schwingern konnte Jarid sich wegducken, der dritte erwischte sie. Doch statt dem erwarteten Faustschlag fühlte sie nur eine kalte Berührung, als Finster-Trieest den roten Edelstein beinahe sanft an ihre Wange drückte.

Da spürte sie, wie \textit{Etwas} durch ihren Körper streifte und ihren Geist berührte. Sie erstarrte. Nicht, weil sie es wollte, sondern weil ihre Muskeln nicht mehr taten, wie gebeten.

Trieests Körper presste auf Jarid, während ihre eigenen Muskeln erstarrten und ihr größere Bewegungen verwehrt blieben. Unbeweglich sank Jarid zu Boden, Trieests Gewicht auf sich, den eiskalten spitzen Kristall weiterhin in ihre Wange stechend.

Seltsame Gedanken schossen durch ihren Kopf, die nicht die ihren waren. Bilder von grausamen Riesen in Gebäuden aus Knochen. Ein fröhlicher kleiner Junge, der älter und älter wurde und auf dessen ergrauenden Haaren sich eine golden geschwungene Krone bildete. Ein finsterer kleiner Junge, der von Rauch und Schatten umgeben brüllte. Tränen und Qualen in Schnee und Eis. Untote vor einer Burgruine mit schneebedeckten Zinnen. Ein roter Edelstein in einer blauhäutigen Hand, die Jarid bekannt vorkam. Schemen dunkler Gestalten hinter einer roten Scheibe. Ein rothaariger Zwerg vor einem bläulich schimmernden Portal. Eine zahnlos grinsende grünhäutige Druidin mit Vogelscheiße im verfilzten Haar. Und dann der Hraak, dieses Untier, und dessen animalische Instinkte. Einen Augenblick lang fühlte Jarid sich so, als sei sie selbst der Hraak, als würde sie selbst mit dessen scharfen Zähnen frischen Blut kosten.

Dann traten neue Erinnerungsfetzen vor Jarids inneres Auge, die sie als ihre eigenen erkannte, die sie jedoch nicht selbst hervorgerufen hatte. Dieses finstere Etwas analysierte ihre Erinnerungen, ihre Vergangenheit!

Alles in allem erfuhr Jarid nur einige Minuten der Bewegungsunfähigkeit, während die fremde Macht aus dem an ihre Wange gepressten Kristall ihren Geist erforschte und Bilder aus ihrer Vergangenheit hervorrief. Ihre Kindheit in Danwar, ihre Eltern, ihre Freunde, ihre erste romantische Beziehung, durch all das jagte die fremde Macht in Windeseile, es bedeutete ihr nichts. Der Orden der Wassermagier, wie Jarid ihre Ausbildung begann, wie sie mit Bravour bis zum dritten Zirkel aufstieg. Die fremde Macht verblieb einige Momente bei ihrer Lehre, schien Jarids Fähigkeiten zu beurteilen, verwarf dann aber auch diese Erinnerungen. Ihr Geist wurde weiter durchforstet, erste Erinnerungen an Trieest kamen auf. Wie Jarid den kleinen Trieest tröstete, als sein erster Milchzahn ausgefallen war. Wie sie den Fangzahn zu Feuermeisterin Nidwal brachte und von ihm untersuchen ließ. Wie Nidwal besorgt den Kopf schüttelte. Die fremde Macht kicherte mit Jarids Mund. Schließlich der Aufbruch von Danwar. Jarid, wie sie umgeben von den Echos aus der Roten Grotte am Boden saß und schluchzte. Wie Jarid und der feuerkriegerische Trieest, nun größer als Jarid, auf einem kleinen Kahn in den Westen aufbrachen. Wie sie Seite an Seite Heldentaten vollbrachten. Die Furcht, als sie an der Ewigen Kälte erstarrte. Trieests lächelndes Gesicht, das sie als Erstes erwartete, als sie aus dem Eisschlaf erwachte. Wie Brandur sie für ihre Taten lobte. Ihr Kiefer verkrampfte sich ohne ihr Einwirken zu einer wütenden Fratze. Der Kampf um den Alten Wehrturm während der Befreiung der Rietburg, an den Seiten von Feuermeister Lifornus und dem reisenden Temm Wrort. Brandurs Beerdigung, wo sie und Trieest vom frischen Landesleiter Thorald höchstoffiziell zu Helden von Andor ernannt wurden. Die Bilder der restlichen Helden von Andor traten einer nach dem anderen vor Jarids inneres Auge. Erneut verzerrten ihre Gesichtsmuskeln sich, als wäre sie wütend. Ihre Zeit im Land der Brüder, ihre vergangenen Abenteuer in Tulgor, die Rückkehr zum Königsfrieden ...

Dann, endlich, zog sich die fremde Macht aus ihrem Geist zurück. Trieests schweres Gewicht löste sich von ihr, als der Feuerkrieger sich unsorgfältig wieder aufrichtete und den roten Kristall wieder von ihrer Wange nahm. Jarid zappelte probehalber mit ihren Beinen. Ihr Körper gehörte wieder ihr!

„Nimm das nicht persönlich, Jarid Morgentau, aber der Körper deines Trieests ist meiner Wenigkeit würdiger als deiner”, sprach die finstere Macht durch Trieests krächzende Stimmbänder, „Dein Geist ist allerdings schon interessant … du hast schon Mumm, deinen Begleiter so lange anzulügen. Keine Sorge, er nimmt gerade nichts war. Er weiß noch von nichts. Fast fühle ich mich dazu verlockt, ihn zu wecken und ihm zu erzählen, was du ihm alles verschweigst.”

Jarid drehte sich keuchend auf den Bauch und stemmte sich langsam hoch. „Du hast keine Ahnung, warum ich das tue! Du hast kein Recht, mich zu …”

„Ich weiß genau, warum du das tust! Du hast Angst, Jarid, Angst um dich und um ihn! Aber keine Sorge, ich werde es ihm nicht erzählen. Sein Körper ist alles, was ich von ihm brauche. Ich muss nur noch dafür sorgen, dass ihn mir keiner wegnehmen kann.”

Finster-Triefst grinste breit und führte seine Faust zu seinem Mund. Jarid begriff gerade noch rechtzeitig, was er vorhaben könnte: Er wollte Trieest den roten Kristall schlucken lassen! Wenn seine Kontrolle über Körper von physischer Berührung mit diesem roten Kristall abhing, wäre es äußerst ungünstig, wenn dieser Kristall sich in Dauerberührung in einem Magen befände. Bald wäre Trieest nicht mehr zu retten.

Obwohl ihre Glieder bereits schmerzten, zögerte Jarid keine Sekunde, sondern schleuderte sich Finster-Trieest entgegen, packte seine Faust und zerrte sie von Trieests Gesicht weg. Wieder fokussierte sie sich auf Trieests Blut und versuchte, dieses weg aus dem Arm, weg von dieser Hand, in seinen Kopf strömen zu lassen. Doch ihre Kapazität war erschöpft, genauso wie der Rest von ihr. Ihr einziger Trost war, dass Trieests Körper ebenso erschöpft sein musste.

Finster-Trieests Miene verzerrte sich zu einem Zähnefletschen und seine Lippen entsprang erneut dieses schmerzhaft kratzige Flüstern: „Du kannst ihn nicht mehr retten. Er ist MEIN, so wie alle Welt bald MEIN sein wird!“

Jarid verhakte ihr Bein mit Trieests. Finster-Trieest ging zu Boden. Diesmal war Jarid darauf gefasst, als weiterer Tritt sie zu verscheuchen versuchte. Sie verlagerte ihr Gewicht und stützte sich auf Trieests Brust. Seine Hand, die den roten Kristall festhielt, hielt sie wiederum immer noch eng umklammert. So presste sie den Finsterling in den weichen Waldboden.

Finster-Trieest wand sich unter ihr und ihr Griff begann sich zu lösen. Lange würde sie ihn nicht mehr auf den Boden pinnen können. Schmerz stach durch ihren Torso. Da erstrahlte der Lavastein in Trieests Brust, zunächst zögerlich, dann immer heller. Orangerote Feuerschlieren schossen daraus hervor und umringten Finster-Trieest. Dieser brüllte auf und spie Jarid schwarze Spucke entgegen, die pechschwarzen Augen hasserfüllt zusammengekniffen. Dampf zischte, als Trieests Körper sich schlagartig erwärmte, Haut und Haare vertrockneten. Der Lavastein wehrte sich und war ihnen endlich zu Hilfe gekommen! Jarid sprang auf und trat einige Schritte vor der Hitze zurück.

„Sei still! Du bist tot!“, schrie Finster-Trieest ins Leere. Er starrte nicht mehr Jarid an, sondern in den Himmel, seine dunklen Augen auf der Suche nach einer Gestalt, die nur er wahrnehmen konnte. Er klang verwirrt, ja, gar verängstigt: „Du bist gestorben, wie du es verdient hast! Ich sah dein Grab! Ich sah dein Reich in der Hand deines Nachfolgers leiden! Das ist nicht deine Stimme! Das kann nicht sein!“

Er griff sich an die Brust und sackte zusammen. So zuckte er am Boden, wie es der echte Trieest schon zu oft hatte erleiden müssen. Er griff nach dem glühenden Stein in seiner Brust und zerrte an ihm, hob dann die Faust mit dem roten Kristall darin und ließ sie auf den Lavastein krachen. Sein grollendes Knurren wurde nur lauter.

„Das ist nur eine Illusion! Du bist gefallen!“

Erneut ließ er den Kristall auf den Lavastein krachen und Jarid war sich sicher, etwas brechen zu hören. War das der Lavastein oder Triests Rippe gewesen?

Als ihr schwindelig wurde und ein Ziehen in ihrer Brust sich bemerkbar machte, sah Jarid zum ersten Mal seit dem Beginn des Kampfes an sich herunter und erschrak ganz gewaltig. Da befand sich Trieests Rankenschwert, immer noch in Form eines finsteren Geweihs. Hatte er es nicht fallen gelassen? Nun steckte es in Jarids Torso, knapp unter ihrem Herzen.

Er hatte sie doch noch erwischt.

Wie lange steckte das Schwert nun schon da? Wie schlimm waren ihre Innereien verletzt? Wie weit war es bis zum nächsten Dorf, wo man es sicher entfernen konnte? Oh nein, da kam sie schon, die Schwäche in den Gliedern und die Dunkelheit im Gesichtsfeld.

Die Stimmen der Echos aus der Roten Grotte begleiteten sie, während Jarids Bewusstsein entschwand. Das letzte, was sie wahrnahm, war ein fernes Wiehern. Pferde?

Hier, mitten im Wald?

Dann umfing sie der Mantel der Ungewissheit und Ohnmacht.\bigskip







\textit{Die Pferde trafen ihn unterwartet und schleuderten ihn zu Boden. Der Kristall wurde aus Finster-Trieests Hand geschleudert. Einen Augenblick lang konnte er seine Kontrolle über Trieests Körper beibehalten, irgendetwas an diesem Lavastein in dessen Brust schien die Verbindung zu verlängern. Dann brach seinen geistigen Fokus zum Feuerkrieger dennoch ab.}

\textit{Das Böse brauchte einige Sekunden, um die urplötzliche Absenz von Trieests Sinneswahrnehmungen zu verarbeiten und sich auf die stumpfen Sinne seines kristallenen Gefängnisses zurückzubesinnen. Etwas war da, etwas war über ihm. Ein großer Schatten von etwas Hölzernem. Schwer. Schwerfällig. Laut. Und dann war da dieses Getrappel.}

\textit{Ein Wagenrad rammte den Kristall. Das Böse spürte, wie dessen Stabilität unter dem schweren Druck nachließ. Risse durchzogen seine Oberfläche.}

\textit{Dem Bösen blieben nur Augenblicke. Es verspürte keine Schmerzen per se, wurde aber beinahe ohnmächtig vor Todesangst. Es war magisch an diesen Kristall geknüpft. Wenn dieser zerstört würde, würde auch das Böse ausgelöscht, ohne Möglichkeit, sich selbst wieder als Geist neu zu formen. Dazu durfte es nicht kommen. Nicht jetzt, wo es endlich so weit gekommen war!}

\textit{Ohne ein nahe gelegenes Lebewesen, das es kontrollieren konnte, blieb dem Bösen nur keine Möglichkeit, seinen Einfluss auszuweiten. Es musste eine andere Taktik wählen. So zog sich das Böse in eine einzelne dünne Kristallfaser seines Gefängnisses zurück und bangte darum, diese möge stabil bleiben. Die schwarzen Wogen seiner Seele verblassten unter der Oberfläche und ließen den Kristall blutrot zurück bis auf den kleinen Bereich, in den sich das Böse quetschte.}

\textit{Das Wagenrad drehte sich weiter und der finstere Kristall zersplitterte. Hunderte von blutroten Bruchsplittern übersäten den blutigen Waldboden und verfärbten sich langsam zu einem strahlenden Weiß, als das einst in ihnen gefangene Licht des roten Mondes hinausquoll.}

\textit{Ein einzelner Splitter stand auf den ersten Blick heraus aus den vielen Überresten. Nicht aufgrund seiner Form oder Größe, sondern wegen seiner Farbe. Während die anderen sich langsam weiß verfärbten, wurde dieser einzelne Splitter tiefschwarz.}





























\newpage
\section{In der Weinkutsche}









\az{Jahr 61}

\textit{Halle des Ältestenrats, 61 a.Z.}\bigskip



Das aufgebrachte Gemurmel im danwarischen Ältestenrat verklang erst, als die Älteste Rowinda – Jarids Mutter – energisch auf das Ratspult schlug und um Ruhe rief.

„Ruhe im Rat! Dann bist du also als Zeugin hier, Jarid?“

„Unter anderem. Ich will hier bloß meine Perspektive teilen und meine Meinung kundtun. Seit wann achtet Ihr denn plötzlich wieder auf all diese oberflächlichen Details in Ratssitzungen?“

„Seitdem wir darüber urteilen müssen, ob der Angeklagte weiterhin ein Dasein auf Danwar verdient hat! Es ist lange her, seit ein solch schwerwiegender Vorfall unsere friedliche Gemeinschaft erschüttert hat.“

„Gute Güte, überlegt ihr euch ernsthaft, Trieest von der Insel zu verbannen? Er ist doch erst dreizehn!“

„Dreizehn Jahre sind für solche wie ihn weitaus genug, um die Adoleszenz zu erreichen. Er hätte seinen Instinkten längst Herr werden sollen!“

„Keiner leugnet, dass dieser Unfall besser nicht geschehen wäre, und keiner weiß, womit wir das Risiko eines weiteren möglichst gering halten ...“

„Oh, ich wüsste da schon was. Und ich weiß auch schon, dass ich dafür stimmen werde!“

„Ruhe im Rat, Freiga! Jarid, kannst du guten Gewissens sagen, dass so eine Tat nicht wieder begangen werden wird, wenn wir Trieest weiterhin frei hier herumflanieren lassen?“

„Natürlich nicht, aber ihr könnt mir auch nicht guten Gewissens sagen, dass Jormudd Trieest nicht anfallen wird, wenn sie sich das nächste Mal sehen ... ähm ... treffen.“

„Jormudd ist ein guter Junge, aus einem guten Elternhaus. Von deinem Trieest lässt sich das nicht sagen.“\bigskip







\az{Jahr 68}

\textit{Sieben Jahre später.}\bigskip




Es war warm, als Trieest langsam wieder zu Bewusstsein kam. Dies war keine Besonderheit, als Feuerkrieger fror Trieest so gut wie nie. Dennoch konnte er feststellen, dass seine Umgebungstemperatur unüblich hoch war. Ein raues Gefühl auf seiner Haut bestätigte: Er lag auf einer Decke!

Das Zweite, was Trieest bemerkte, während seine Wahrnehmung langsam wieder einsetzte und der graue Nebel der Ohnmacht sich lichtete, war, dass er nicht mehr auf dem ruhigen Waldboden lag. Vielmehr holperte und schaukelte der Boden, auf welchem er lag, so stark, dass er kaum glauben konnte, dass er nicht früher erwacht war. Es war auch kein Boden, auf dem er lag, sondern ... Holz? Ein Tisch? Lag er auf einem schaukelnden Tisch? Aber da war auch der unmissverständliche Gestank von Eisenverschlägen.

Trieests Nasenflügel bebten, als er seine Umgebungsluft einsaugte. Jarid war da, zu seiner Rechten, das war schon mal beruhigend. Zu seiner Linken, etwas schwächer wahrzunehmen, befand sich eine weitere Person, die ihm unbekannt war. Eine Person, die nach faulen Zähnen und alten Kleidern duftete. Um sie alle herum roch es nach eigenartigen Früchten ... nein, das waren keine Früchte mehr, das war... Met? Wein? Fässer mit einer alkoholischen Flüssigkeit jedenfalls. Und ... Pferdemist! Das waren Pferde, mehrere sogar!

Er befand sich in einem Karren. Vielleicht gar in einer Kutsche.

Trieest schlug die Augen auf und schloss sie gleich wieder, als selbst die Plane über seinem Kopf das gleißende Sonnenlicht kaum mindern konnte.

Jarid zu seiner Rechten regte sich und trat an ihn heran, legte ihre Hand auf die seine.

„Gut, du bist endlich erwacht. Wie fühlst du dich?“

Sie klang vorsichtig, ängstlich. Warum bloß?

Trieest hustete. Seine Brust schmerzte ungewöhnlich stark. Woran das wohl liegen könnte?

Langsam regte Trieest sich, nur um festzustellen, dass seine Bewegungsfreiheit eingeschränkt war. Starke eiserne Ketten an seinen Handgelenken hinderten ihn daran, sich umzudrehen. Das war äußerst seltsam. Andere Leute waren manchmal ob Trieests Erscheinung verängstigt, aber Jarid sollte doch wissen, dass er keine Gefahr darstellte.

„Wie fühlst du dich, Trieest?“, fragte Jarid erneut, diesmal noch angespannter.

„Als wäre eine Horde Steppenechsen über meine Brust getrampelt. Was ... was ist hier los?“, fragte Trieest.

Jarid antwortete nicht sofort.

Von links – woher auch der Duft der unbekannten dritten Person stammte – ertönte eine knorrige Stimme: „Ist das er? Ist er wach, Liebes?“

„Ja, Lysbett“, rief Jarid als Antwort.

„So sage mir, Meisterin über das Wasser, trachtet dein Begleiter uns immer noch nach dem Leben?“

Trieest schluckte. Immer noch?

„Was ist geschehen, Jarid?“, flüsterte Trieest argwöhnisch.

Langsam öffnete er seine Augen und zuckte nur noch leicht zusammen, als das Sonnenlicht sie traf. Seinen Kopf nach links und rechts drehend erkannte Trieest, dass er in der Tat auf dem Boden einer Kutsche lag, seine Arme mit festen Eisenketten am Kutschenboden befestigt. Eine große Plane verwehrte ihm den Blick auf dem Himmel oder aus der Kutsche hinaus, wo den Geräuschen und Gerüchen zufolge mehrere Pferde emsig vor sich hin eilten.

Abgesehen von den festen Eisenketten und seiner roten Stoffkleidung trug Trieest nichts mehr. Seine Schuhe hatte man ihm ausgezogen, seine Rüstung ebenfalls. Dieses Ungetüm an filigraner Schmiedekunst war sowohl zweckdienlich als auch zeremoniell, und Trieest lag einiges daran. Dennoch konnte er sie nirgends sehen. Hatte man sie vielleicht in eines der Fässer gesteckt? Und wo war sein Schwert? Ein solches Rankenschwert war eine noch größere Rarität als seine Rüstung, und ein Rankenschwert, welches an einen selbst gebunden war, war buchstäblich unbezahlbar.

Trieest verrenkte seinen Kopf und kniff seine Augen zusammen, um auf seine Brust zu spähen. Etwas war hier ganz und gar nicht in Ordnung! Der eiserne Reif, welche den Lavastein in seiner Brust verankert halten sollte, befand sich ebenfalls nicht mehr an seinem angestammten Platz. Stattdessen lag der Lavastein völlig ungeschützt in Trieests Fleisch, als könne er jeden Moment herausfallen. Trieest wusste, dass der Lavastein das nicht tun würde, in den sieben Jahren des Tragens hatte dieser Stein sich so tief in seinen Körper gegraben, dass dieser ihn fast schützend festhielt. Dennoch war der Anblick des Steins ohne den ihm angestammten Rahmen zutiefst erschütternd.

Zudem war da ein Spalt im Lavastein zu sehen! Ein Riss im unzerstörbaren Lavastein?! Bei Kentars Dreizack, was war geschehen?

Trieest musterte nun Jarid, welche neben ihm saß und seinem Blick auswich. Im Gegensatz zu ihm, welcher von Schrammen übersät und mit Blut und Erde verdreckt war, sah man Jarid beinahe keine Spuren des vergangenen Kampfes gegen den Hraak an. Ihre Wunden waren ja vom Lavastein geheilt worden und das zerrissene Kleid musste sie bereits wieder geflickt haben. Doch warum war das Kleid über Jarids Bauch in einem dunklen Rot verfärbt? War das wieder Blut?

Es hatte die Wassermagiergilde aus Danwar Jahrzehnte gekostet, Stoffe und Talismane aus einer Art festen Wassers zu entwickelt. Ihre Träger konnten (sofern sie in der Kunst des Wasserformens bewandert waren) diese Kleider wie Wasser formen und so etwa auch bei Reisen durch Wassertunnel mit sich führen – eine unglaubliche Errungenschaft. Jarids Einhorn-Horn-Schnalle an ihrem Umhang war ein ähnliches Artefakt. Nun fehlte nur noch, dass die Wassermagier größere Waffen entwickelten, die sie über große Distanzen mitnehmen könnten, wenn sie von Wasserspeicher zu Wasserspeicher teleportierten. Doch für den Moment genügte ihnen, dass sie nicht nackt am anderen Ort emergieren mussten.

Die mysteriöse Kutsche holperte und polterte über Stock und Stein, als gäbe es kein Morgen, und so wurde Trieest heftig durchgeschüttelt. Heftig durchgeschüttelt wurden auch die vielen Fässer, welche schön aufeinandergestapelt vor Trieests nackten Füßen standen und mit weiteren Ketten stabil aneinander befestigt waren. Der Weingeruch war eindringlich.

Lysbett, diese Person auf dem Kutschbock, musste eine Weinhändlerin sein! Als hätte sie darauf gewartet, dass er sich an sie erinnerte, öffnete sich die Plane zu Trieests Linken und das Gesicht einer alten Frau blinzelte ins Innere. Brandnarben übersäten ihre linke Gesichtshälfte und eine schnörkellose Augenklappe verdeckte den Blick auf eine vermutlich leere Augenhöhle. Warnend polterte die Weinhändlerin: „Liebes, wenn du mir nicht bald ausrichtest, dass der Kerl wieder putzmunter und dämonenfrei ist, halte ich dieses Gespann an und entledige mich auf meine Art dieses Monsters. Mit derart dunklen Kräften ist nicht zu spaßen.“

„Alles ist gut, Lysbett, er scheint wieder in Ordnung zu sein“, rief Jarid nun hektisch zurück.

„Das ist Lysbett“, stellte Jarid Trieest die Alte vor, „Sie rettete uns und ließ unseretwegen einige Fässer ihres Transports im Wald liegen.“

„Drachenfass Rachenputzer. Hochexplosives Zeug“, murmelte Lysbett, „Kostete mich schon meinen linken Arm. Und dieses Biest von einem Krieger wird mich nicht meinen rechten kosten!“

Sie nickte mehr zu sich selbst als zu Jarid und zog sich wieder auf den Kutschbock zurück, von wo aus einige Male „Hü!“ und „Ho!“ erklang. Die Kutsche holperte noch etwas stärker, als die Pferde einen Zahn zulegten.

„Ich soll wieder ‚dämonenfrei‘ sein?!“, fragte Trieest ängstlich, „Was ist geschehen? Was habe ich getan? Wie lange war ich weg?“

„Was ist das letzte, woran du dich erinnerst?“, antwortete Jarid mit einer Gegenfrage.

Trieest überlegte. Sie hatten den Hraak bekämpft, das hatte er deutlich in Erinnerung. Sie hatten sich unterhalten über die Prophezeiung aus der Roten Grotte. Er war enttäuscht gewesen über die Stille des Lavasteins und das Fortbestehen seiner Bürde und hatte dies ausgedrückt. Danach ... was danach lag, konnte Trieest nicht mehr sagen.

Er erzählte, was er wusste: „Wir haben den Hraak überwunden, aber das hat meinen Prozess des Wandels nicht beendet. Wir wollten gerade den roten Kristall untersuchen. Dann ... dann wachte ich hier auf.“

„Du hast dein Schwert genommen und den roten Kristall berührt. Irgendeine finstere Macht ergriff Besitz von dir. Du hast mich angegriffen“, sprach Jarid gezwungen ruhig.

Trieest ließ ihre Worte besorgt auf sich wirken. Er zweifelte nicht an ihnen, wohl aber an Jarids scheinbarer Unversehrtheit. Das konnte nicht gut gegangen sein. Im waffenlosen Kampf war Jarid zwar geschickt und hatte Trieest schon dutzende Male in Grund und Boden gerungen, doch war Trieest um einiges stärker als sie, ganz zu schweigen davon, dass er ein magisches Schwert besaß und Jarid nichts außer ihre Hände.

„Wie ist es ausgegangen?“

Ein Grinsen huschte über Jarids Züge, als sie salopp erwiderte: „Lysbett kam in ihrer Kutsche angefahren und hat dich über den Haufen geritten.“

Dann verdüsterte sich ihr Ausdruck wieder als sie nachsetzte: „Zu diesem Zeitpunkt lag ich bereits am Boden, dein Rankenschwert in meinem Torso. Hätte der Lavastein sich nicht gegen diese finstere Macht gewehrt, wäre ich jetzt nicht mehr hier.“

Jarid berührte geistesabwesend ihr magisch geflicktes Gewand an der blutverschmierten Stelle und verzog schmerzerfüllt das Gesicht.

„Was ... was geschah dann? Wie steht es jetzt um dich?“

„Zum Glück ist Lysbett eine geschickte Heilerin. Sie brachte mich wieder zu Bewusstsein und schleppte mich zu dir und deinem Lavastein. Dieser erfrischte mich erneut, als ich ihn berührte, wenn auch nicht so ausführlich wie beim ersten Mal. Und nun bringt sie uns beide zum Baum der Lieder. Du warst zum Glück nur einige Stunden weg, dein Lavastein muss auch dich geheilt haben. Ist dir aufgefallen, dass er einen Riss hat? Die Faust des finsteren Dus konnte tatsächlich irgendwie Schaden anrichten an diesem angeblich unzerstörbaren Stein. Faszinierend. Beunruhigend.“

Ja, Trieest war aufgefallen, dass der Lavastein einen Spalt trug. Kein Wunder, dass gerade konstant ein leichtes Anzeichen von Furcht vom Stein auszugehen schien. Das hatte diese finstere Macht mit Trieests eigener Faust vollbracht? Wäre er auch selbst dazu in der Lage gewesen?

Er überlegte weiter. Der Lavastein hatte Jarid also bereits zum zweiten Mal an diesem Tag geheilt. Warum schien ihm plötzlich so viel an ihr zu liegen?

„Glaube mir, es war eine Mühsal, Lysbett davon abzuhalten, dir nicht dann und dort die Kehle durchzuschneiden. Sie bestand zumindest darauf, die ganzen Kristallsplitter zu vergraben. Was wirklich eine Schande war. Sie hatten sich allesamt glühend weiß verfärbt. Das muss doch irgendetwas bedeuten. Wer weiß, was wir alles für Erkenntnisse daraus hätten gewinnen können!“

Die letzten Worte hatte Jarid um einiges lauter gesagt, als nötig gewesen wäre, und prompt ertönte eine heisere Antwort vom Kutschbock her:

„Liebes, wenn du noch lange wegen diesen Steinsplittern in den Ohren liegst, wird euch diese Fahrt so einiges mehr kosten! Ich musste bereits drei Drachenfässer vom allerbesten Rachenputzer der Schildzwerge mitten im Wald stehen lassen, um Platz für euch zu schaffen, da lasse ich mir doch keine Trauerreden auf verfluchte Objekte gefallen!“

Jarid schwieg stille und verschränkte ihre Arme.

„Ich versprach ihr, dass wir sie reich bezahlen, wenn wir am Baum der Lieder ankommen“, flüsterte Jarid, „Gold besitzen wir kaum welches, mehr, wärst du notfalls bereit, einen Teil deiner Rüstung einzutauschen?“

„Meine Rüstung gibt es noch?“, horchte Trieest freudig auf.

„Wir mussten sie nur entfernen, um dich ein bisschen zurechtzurichten. Die Pferde haben dir nicht gut getan, Tri. Sie liegt hinter einigen Fässern. Ebenso dein Schwert“

„Dem Flammenden Gott sei Dank! Ohne sie komme ich mir so nackt vor. Doch war es wirklich klug, den potenziell gefährlichen Kristall einfach so zurückzulassen, auch wenn er zersplittert ist?“,

„Lysbett und ich haben die Splitter in einem Tuch gesammelt, natürlich ohne sie zu berühren, und sie tief im Boden vergraben, mitten im Wald. Die sollte niemand so schnell wiederfinden können. Ganz unabhängig davon waren die Splitter allesamt weiß. Keine Spur mehr von finsteren Schemen unter der Oberfläche. Und es steht doch zu vermuten, dass die finstere Macht zu diesen finsteren Schemen im Kristall gehörte?“

„Na, ebenso könnte ein glühend weißer Schemen sein, der all diese Splitter befiel. Bosheit kennt viele Farben.“

„Weiße Kristalle gibt es so einige viele in den Tiefen der Erde, davon können dir die Schildzwerge ein Liedchen singen, und keiner von denen war verflucht. Einen roten mit schwarzen Schemen wie diesen sah ich hingegen bislang nur diesen. Diese weißen Splitter tragen den Geist des Hraaks nicht mehr mit sich.“

„Dann starb diese finstere Macht bei der Zerstörung des Kristalls?“

„Wir können nur hoffen.“

„Also, auf jeden Fall enthielt dieser rote Kristall etwas Böses, welches vom Hraak unabhängig war, oder? Jemand muss den Stein diesem Monster in den Kopf gesetzt haben, um diese Gegend zu terrorisieren.“

„Das habe ich mir auch schon gedacht. Wahrscheinlich war es diese Druidin, von dem das ganze Dorf behauptete, dass sie finstere Experimente in ihrem einsamen Turm vollbrachte. Und als die finstere Macht dich kontrollierte, erwähnte sie ebenfalls eine Druidin. Zu schade, dass diese Druidin schon vor Jahren gestorben ist, sonst hätte ich ihr gerne ein paar Fragen gestellt. Andererseits ist es vielleicht auch besser so, dass eines ihrer Experimente diese Zauselin erwischte. Wer weiß, was sie sonst noch alles für Chimären in die Welt gesetzt hätte. Der Hraak allein war schon schwerwiegend genug. Wir können nur hoffen, dass sie keine anderen Monstrositäten in sonstigen Kellergewölben versteckt hielt. Oder, falls schon, dass diese nicht ausbrechen, ehe sie an Nahrungsmangel verenden.“

Trieest schwieg. Zu viele Gedanken kreisten in seinem Kopf. Er war von einer bösen Macht besessen gewesen. Eine Kutsche hatte ihn überfahren. Der Lavastein war angebrochen. Wie viele Neuerungen würde dieser Tag noch bringen?

Jarid legte sorgsam ihren Kopf auf Trieests Brust und horchte in den Lavastein hinein. Ihre Haare kitzelten Trieest, doch er unterdrückte den Kicherdrang.

„Kalt wie das tiefe Meer“, sprach Jarid, „Und ich höre wie immer nur das Geräusch der Brandung an die heißen Klippen Danwars. Bis auf den Splitter wirkt er völlig normal.“

Trieest nickte. „Ich fühle leichte Furcht von ihm, höre jedoch auch nichts außer das Rauschen des Meeres. Diese Legenden von Feuerkriegern, die mit ihren Lavasteinen sprechen, sind vermutlich übertrieben. Oder geschieht so was vielleicht erst am Ende meines Prozesses der Wandlung?“

„Als das Böse dich übernommen hatte, rief es wild ins Leere, als hörte es Stimmen. Ich glaube, dein Lavastein hat mit ihm gesprochen. Und dich gerettet.“

„Er hat uns beide gerettet. Und doch bin wieder ich am Ende wieder derjenige, der versorgt werden muss“, grinste Trieest schief.

„He, mein Bauch ist noch lange nicht verheilt. Und wenn das alles durch ist, kannst du dich gehörig revanchieren. All die Massagen und Linderungen, die ich dir verschaffe, zahlst du mir noch mit Zins und Zinseszins zurück“, scherzte Jarid.

Trieest grinste zurück und verdrängte all die Gedanken, die seinen Kopf zum Kreisen brachten. Selbst die Schmerzen in seinem Körper flauten ab. Er drehte sich zur Seite ab und versuchte, sich in den Schlaf zu verkriechen.

„Was nun, Tri?“, fragte Jarid abrupt.

„Wie meinst du das?“, murmelte Trieest müde.

„Wir können das sonst auch später bereden“, meinte Jarid, „aber wir müssen nicht so weitermachen wie bislang. Wir müssen nicht von einer heldenhaften Aufgabe zur nächsten ziehen und deine Hoffnungen und Träume auf ein Ende des Prozesses des Wandels strapazieren. Wenn du willst, können wir das Heldendasein für einen Moment auf Eis legen. Chada lässt sich von Reka ausbilden, Thorn züchtet seine Pferde, Kheela kümmert sich wieder primär um die Flusslande und Barz zog es zurück zu seiner Familie im Osten. Der Königsfrieden könnte auch uns Frieden schenken. Wir könnten uns ebenfalls irgendwo niederlassen. Der Ältestenrat lässt uns nicht nach Danwar zurück, während dein Prozess des Wandels anhält, aber die restliche Welt ist groß, so groß.“

„Ich hätte nicht gedacht, dass du das als Option betrachtest.“

Jarid und Trieest hatten sich selten über die ferne Zukunft unterhalten. Es war ihnen oftmals einfach oder wichtiger erschienen, die nächste Reise, die nächste Herausforderung anzustreben. Beiden von ihnen war schwer bewusst, dass Trieests Lebensspanne sich nicht im Ansatz mit der von Jarid vergleichen ließ. Als sie sich freiwillig dafür gemeldet hatte, ihn als Begleiterin beim Tragen der Bürde zu unterstützen, war er gemäß der Lebensspanne eines Menschen noch ein junges Kind gewesen, auch wenn er nicht mehr danach ausgesehen hatte. Nun zeigten sich bereits die ersten weißen Haare in seinem Schopf. Wer wusste, wie lange es ihn noch geben würde ... Zehn Jahre? Zwanzig? Länger konnte er wohl kaum hoffen. Seinem Leben hatte er in dieser Zeit kaum eine eigene Richtung geben können, alles hatte sich immer um den Lavastein gedreht. Und nun kamen ihm plötzlich Zweifel auf, ob er seinen Prozess in seiner kurzen Lebenszeit überhaupt noch beenden können würde.

Er lachte bitter auf: „Und all dieser Ärger nur wegen J ... wegen Jormudd.“ Es war für ihn immer noch schwer, Jormudds Namen auszusprechen. Nicht zu denken, was geschehen wäre, wenn er seine Kehle erwischt hätte. Trieest hatte manchmal immer noch Albträume davon. In letzter Zeit aber immer seltener. Jormudd und Danwar, das lag inzwischen lange hinter ihm. „Wir wissen, dass ich falsch handelte. Es wird nie wieder geschehen. Es tut mir leid und ich zahlte den Preis. Um ein Vielfaches. Meine Schuld sollte schon längst getilgt sein. Manchmal frage ich mich schon, ob ich irgendetwas grundsätzlich falsch angehe mit meinem Lavastein. Ob die entscheidende Aufgabe, die meinen Prozess des Wandels beenden wird, eine persönliche sein wird. Ob ich nach Danwar zurückkehren sollte, mich bei Jormudd entschuldigen und ihm gegenüber meine Taten gutmachen. Ob es vielleicht schon immer so leicht war.“

Jarid schüttelte ihren Kopf. „So etwas wäre vielleicht die Lösung, falls das hier ein Märchen wäre, aus dem man eine Moral ziehen sollte. Aber es würde nicht viel Sinn geben, dich aus Danwar zu verbannen, wenn deine entscheidende Aufgabe sich dort versteckte.“

„Soll ich aus meiner Bürde nicht eine Moral ziehen?“

„Idealerweise ja, wie bei jeder großen Handlung. Aber das ist nicht der Hauptgrund hinter einer Bürde. Die Hoffnung des Ältestenrats war, dass du abgeschreckt wirst, so etwas nie wieder zu tun, oder zumindest nicht in Danwar.“

„Natürlich tu ich es nie wieder! Ich war nicht ich selbst. Diese Instinkte ...“

„Diese Instinkte hast du inzwischen völlig unter Kontrolle, ich weiß. Und ich bete, dass dein Prozess bald hinter dir liegt und wir einen bürdenfreien Weg einschlagen können. Ich meine ja nur, dass wir in letzter Zeit relativ gut mit dem Stein klar kamen. Und dass ich mir vorstellen könnte, ihn eine Zeit lang zu ignorieren, so gut er uns lässt. Uns eine Auszeit zu gönnen, bis du dich bereit für deine nächste Aufgabe fühlst. Ist natürlich deine Entscheidung, wie immer.“

Trieest knurrte. So sehr er sich ein Gefangener seiner Bürde fühlte, so ärgerlich war Jarids Weigerung, ihm eine Meinung über ihre gemeinsame Zukunft mitzuteilen. Stets hatte sie ihm aufgetragen, auszusuchen, in welche Länder und Reiche sie ziehen sollten, welche Aufgaben sie priorisieren sollten. Jarid war bloß die eifrige Begleiterin gewesen. Und ohne ihre tröstenden Worte und kühlenden Wassertropfen wäre er schon lange ausgebrannt. Trieest stand so tief in ihrer Schuld.

„Ich werde nie zurückzahlen können, was du für mich getan hast, Jarid“, sprach Trieest heiser und ernst.

Jarid drückte seine Schulter. „Und das musst du auch nie. Ich begleite dich nicht aus Zwang, sondern weil ich will. Ich lasse dich deinen Weg und deine Aufgaben wählen, weil ich muss, doch folgen tu ich dir aus freiem Willen, so frei, wie ein Wille jedenfalls sein kann.“

Trieest gluckste und Jarid zischte zurück: „Und ja, das darfst du mir das wieder sagen, wenn ich dich das nächste Mal wegen einer deiner hirnrissigen Entscheidungen anschnauzte. Wir stehen das gemeinsam durch. Ich will noch an deiner Seite sein, wenn dein Leiden ein Ende findet.“

„Danke“, flüsterte Trieest. „Und wohin willst du danach reisen? Was hast du danach vor? Zurück nach Danwar und Frieden mit deiner Mutter schließen?“

Jarid sah ihn verwundert an: „Denkst du etwa, ich würde dich einfach so verlassen?! Nach all dieser Zeit will ich auch nicht nach Danwar zurück!“

Trieest sah sie stumm an. Überrascht. Jarid fuhr fort: „Ich glaube, ich will mich irgendwo niederlassen. In einer Gemeinschaft von Menschen, die sich nicht um unsere Vergangenheit schert. Ein kleines Dorf, hier im Wachsamen Wald, oder an der Küste, oder hinter den Bergen, oder hoch im Norden, wo auch immer. Ein nettes kleines Dorf, und es wird unsere Heimat sein, und wir werden es zum Erblühen bringen.“

„Wir beide?“

„Wenn du willst.“

Und ausnahmsweise entscheidest du, welcher Ort?“

„Wenn du willst.“

Sie lächelte beim Gedanken, und Trieest erwiderte das Lächeln. Dann rasselte er demonstrativ mit seinen Ketten: „Meinst du, diese Lysbett vertraut deinem Urteil genug, dass sie sich darauf einlassen würde, mich aus diesen Ketten zu lösen?“

„Was meinst du, Lysbett?“, rief Jarid zum Kutschbock.

Lysbett gab keine Antwort.

Wo Trieest so darüber nachdachte, fiel ihm auf, dass sich Lysbett schon eine ganze Zeit lang nicht mehr in ihr Gespräch eingemischt hatte. Und die Kutsche ... sie holperte und polterte ja gar nicht mehr, sondern stand still.

Es war ruhig draußen.

Zu ruhig.

Trieest blickte Jarid in die Augen und erkannte, dass ihr dieselben Gedanken aufgingen. Langsam stand Jarid auf und bewegte sich einige Schritte auf das Ende der Zeltplane zu, die den Kutscheninhalt vor dem Tageslicht beschützte, ihnen nun aber den Blick nach draußen versperrte. Trieest kniff die Augen zusammen und blähte seine Nasenflügel, zog den Duft der Umgebung ein, aber bei diesem dominanten Duft von Met und Wein konnte er kaum etwas erkennen. Da! Eine leise Andeutung von... Blut! Er roch Blut! Blut vermischt mit Lysbetts Geruch.

Trieest zischte Jarid zu, und als sie sich fragend zu ihm umdrehte, gebot er ihr mit hochgezogenen Augenbrauen, stehen zu bleiben. Er hatte einen weiteren Geruch vernommen. Ein Geruch, der ihn bereits sein ganzes Leben lang verfolgt hatte.

Der Geruch nach verfaultem Fleisch.

Skrale!\bigskip







Trieest fürchtete den Geruch von Skralen bereits sein ganzes Leben lang. Die Insel Danwar, die Heimat seiner Kindheit, war schon seit eh und je immer wieder von Skral-Sippen geplagt worden, und die Orden der Wassermagier und Feuerkrieger arbeiteten Hand in Hand, um den Kreaturenangriffen stand zu halten. Doch im Gegensatz zu den Kreaturen des Meeres, wie etwa den grauenvollen Arrogs aus den Heeren der Mächte des Meeres, konnte man die Skrale nicht zurück ins Wasser jagen, denn sie entstammten dem Wasser gar nicht. Stattdessen lebten und verkrochen sich die Skrale in den Höhlen und Gängen, die das poröse Gestein Danwars durchzogen. Sie waren eine Pest, welche sich einfach nicht ausrotten ließ. Es gab zu viele von ihnen und sie wurden rasch mehr, und so waren die Danware froh, dass sie sich abgesehen von gelegentlichen Überfällen auf das Leben in unterirdischen Hohlräumen beschränkten.

Diejenigen Skrale, mit denen die Danware zu kämpfen hatten, waren entfernte Verwandte derjenigen, welche Jarid und Trieest in Andor immer wieder angetroffen hatte. Die Skrale aus Andor trugen aufwändige (wenn auch oft von Zwergen gestohlene) Rüstungen und Schwerter, zogen in Sippen durchs Land, lachten und sangen, wenn sie ein Dorf plünderten und trauerten, wenn einer der ihren von ihnen gegangen war. Die danwarischen Skrale hingegen trugen, wenn überhaupt etwas, dann vor allem Lendenschürze und Holzstöcke als Waffen. Manche hatten eine dunklere Haut und rundere Gesichter – die Andori nannten sie „Nord-Skrale“, weil sie sich hier im Süden rar gemacht hatte. Diese waren oft die Anführer über die „Kreideskrale“, kreideweiße Skrale mit schuppenloser Haut, dafür mit mächtigen Hauern am Unterkiefer ausgestattet, die sie wohl kaum nur zum Brezelstapeln nutzten. Diese Kreideskrale waren mehr Tier als kulturschaffende Wesen, schliefen auf kaltem Stein rund um Steinformationen, die ihre Anführer als Heiligtümer bezeichneten, sprachen selten und ließen ihre Toten gedankenlos zurück.

Die meisten der Nord-Skrale Danwars hatten sich jeweils eine Sippe aus folgsamen Kreideskralen zusammengesucht und diese aus dem Schatten in den Kampf gegen die Nebelinseln geschickt, wo sie Mensch und Vieh gleichermaßen entführten und sich dann in die dunklen Höhlen zurückzogen, um nie mehr gesehen zu werden. Doch manche Nord-Skrale traten hin und wieder aus dem Schatten heraus und kämpften höchstpersönlich an der Seite ihrer Untertanen. Sie waren es, die die Feuerkrieger und Wassermagier stets zu finden und auszulöschen versuchten, wenn sie eine Sippe abwehrten oder eine (leider oft erfolglose) Expedition tief in die unterirdischen Höhlen Danwars anführten. Und diese Nord-Skrale waren leider auch diejenigen, welche ein besonderes Vergnügen daran hatten, in die Dörfer einzudringen, waffenlose Bürgerinnen in eine Ecke zu drängen und ...

Eine Feuerkriegerin hatte das Haus von Trieests Mutter Talemma gerade noch rechtzeitig erreicht, um sie vor einem grausamen Tod retten zu können. Aber nicht rechtzeitig genug, um sie vor etwas anderem zu bewahren.

Neun Monate später war Trieest geboren worden.

Viele Danware hatten ihn für seine reine Existenz gehasst, oder für sein Aussehen, oder für die anderen Müttern, die an den Geburten von Halbskralen verstorben waren und diesen Hass nun auf Trieests Familie lenkten. Viele andere Kinder in der Schule hatten Trieest gefürchtet, erst recht, nachdem sie sahen, dass sich selbst die Kinderhüterin vor ihm fürchte. Doch seine Mutter Talemma zog ihn auf und genoss jede Sekunde davon. Am meisten liebte Trieest es, stundenlang in seinem selbst gebauten Versteck zu sitzen und den Seefahrern zuzusehen. Doch wuchs er rasch zu einem kräftigen Jungen heran, wild und ungestüm. Er biss und kratzte, statt seine Worte zu nutzen. Er verbiss sich in Arme, statt einen dummen Spruch einfach wegzustecken. Und Talemma wusste nicht, wie mit seinen Instinkten umzugehen war. Die Orden der Feuerkrieger nahm sich seiner Ausbildung an. Warum das Ungeheuer nicht dorthin stecken, wo so viele andere wie Ungeheuer aussahen? Doch die Lavasteine machten ihre Träger nicht nur ungeheuerlich, sondern auch einander sehr ähnlich, und der lavasteinlose junge Trieest war dennoch stets anders gewesen, sowohl von den großen Kriegern als auch von den anderen menschlichen Anwärtern. Nicht auf ewig würde das so bleiben, hatte er sich seit seiner Verbannung eingeredet. Wenn ein Feuerkrieger Danwars seinen Lavastein nach einem vollendeten Prozess des Wandels wieder absetzte, kehrte sein Körper oftmals in eine wohlgeformtere menschliche Form zurück, als er es sie vor dem Prozess gehabt hatte. Vielleicht würde auch Trieest eines Tages einen menschlichen Körper besitzen.

Den Geruch eines Skrals hatte Trieest zum allerersten Mal am eigenen Leibe gerochen und gehasst, stets versucht, ihn mit blumigen und bäumigen Gerüchen zu überdecken. Skrale waren nichts mehr als eine Landplage, welche sich am sinnlosen Leid anderer ergötzten. Man sollte die Höhlen Danwars allesamt durchfluten und mit Feuer füllen, ausräuchern und die Höhleneingänge mit schweren Steinen verschließen. Die Skrale waren im Herzen verdorben und durften nicht weiter bestehen.

Ein Glück, dass Trieest keiner von ihnen war.

Er war keiner von ihnen!

Egal, was der Ältestenrat sagte!\bigskip







Trieest verdrängte die Gedanken an seine Vergangenheit mit geübter Effizienz und hob seine Hand mit fünf ausgespreizten Fingern. Den Gerüchen nach befanden sich mindestens fünf Skrale außerhalb der Kutsche. Jarid quittierte die Botschaft mit einem unglücklichen Kopfschütteln. Fünf Skralen hätten sie sich in ihrer Höchstform locker gestellt, doch nicht in diesem Zustand. Wahrscheinlich hatte die Sippe Lysbett, die Weinhändlerin, auf dem Kutschbock überfallen und fragte sich nun, ob es sicher war, die Kutsche zu betreten.

Sie hatten wohl Geräusche aus der Kutsche gehört, wussten aber vielleicht nicht, wie viele Reisende sich da noch befanden. Und wie gut bewaffnet diese Reisenden waren. Ob sie gar über magische Künste verfügten. Es war aus der Sicht der Skrale wohl geschickter, zu warten, bis sich die Insassen der Kutsche zu erkennen gaben.

Jarid nickte, huschte von Trieest weg, hob etwas vom Boden auf und trat dann näher. Der Schlüssel zu seinen Ketten!

„Ahoi, ihr elenden Weinratten! Kommt ihr von selbst aus eurem Schiff oder müssen wir euch mit Gewalt rausholen?“, ertönte plötzlich eine laute, tiefe Stimme von außerhalb der Kutsche. Dieser Skral hatte wohl das Kommando inne. Ein seltsamer Widerhall lag in der Stimme, als würde der Skral in einen großen Eimer hineinreden und sein Echo zurückhallen.

Der geringen Lautstärke nach befand er sich weitaus mehr als nur einige Schritte von der Kutsche entfernt. Vielleicht hatten die Skrale das Gefährt aus der Ferne angegriffen? Dann würden sie sich als noch größere Feiglinge herausstellen, als Trieest sie ohnehin schon gehalten hatte.

„Na kommt schon!“, rief der Kommando-Skral nun, „Wir haben euch umzingelt! Wir versprechen, euch unversehrt ziehen zu lassen, wenn ihr euch ergebt und die Kutsche aufgebt!“

Dann veränderte sich der Tonfall des Kommando-Skrals etwas und er brüllte: „Bögen bereithalten. Schießt auf alles, was diese Kutsche verlässt!“

Trieest grinste Jarid schief an.

„Ein wenig zwiegespalten scheint er schon, dieser Skral-Anführer, nicht?“

Jarid zog verwirrt eine Augenbraue hoch.

„Nein, Calrai, geh dort drüber hin, zur linken Flanke!“, brüllte der Kommando-Skral nun, und eine etwas gelangweilt klingende Stimme antwortete: „Aye, aye, Häuptling Shron.“

Dann bellte Häuptling Shron einige weitere blecherne Befehle. Sechs weitere Stimmen antworteten zustimmend.

„Acht Skrale!“, flüsterte Trieest entsetzt, „Was machen wir nur?“

„Was wir tun müssen“, antwortete Jarid mit einem grimmigen Lächeln. Dann senkte sie einen Schlüssel in die verrostete Kette, die Trieest am Kutschenboden festhielt, und machte sich daran, seine Ketten zu öffnen.

Trieest plante bereits. „Wir könnten versuchen, durch die hintere Seite zu fliehen. Dieser Shron hat kein taktisches Geschick, er hat nur die Front und die Seiten mit Schützen gedeckt.“

Jarid hielt beim Lösen der Ketten inne und sah ihn fragend an: „Woher willst du das wissen?“

Perplex antwortete Trieest: „Er hat es ja laut genug gebrüllt.“

Die beiden starrten einander bewegungslos an. Dann sprach Jarid, „Kannst du die Skrale etwa verstehen?“, während Trieest zur selben Zeit sprach: „Kannst du die Skrale etwa nicht verstehen?“

Jarid antwortete als erste: „Ich verstand noch, dass er uns Weinratten nannte und uns anbot, uns ziehen zu lassen, wenn wir uns ergeben. Danach wechselte er zu einer kehligeren Skral-Sprache.“

Die beiden guckten einander erstaunt an. Dann sagte Trieest „Konntest du etwa all die Skrale, die wir bislang bekämpft haben, nicht verstehen?“ gleichzeitig mit Jarids „Konntest du etwa all die Skrale verstehen, die wir bislang bekämpft haben?“

„Es ist ja nicht so, als hätten wir uns lange mit ihnen über die Moralität ihrer Taten unterhalten“, rechtfertigte sich Trieest, „Dass uns das nicht früher aufgefallen ist, ist dennoch eine Schande.“

„Werden die Kenntnisse der Skral-Sprache etwa irgendwie vererbt? Basieren sie auf Instinkten? Ist ein solches Sprachverständnis überhaupt möglich? Stehen Skrale mit gewissen Entitäten wie Drachen oder dem Schwarzen Herold in unbewusster geistiger Verbindung und tauschen so intuitive Wortverständnisse aus?“, fragte Jarid nun mit großen Augen, „Die Möglichkeiten, die das eröffnete ... die Konsequenzen solcher Theorien ...“ Dann fing sie sich wieder. „Denken wir lieber an die Möglichkeiten, die uns hier und jetzt aus dieser Situation bringen können.“

Trieest nickte und fasste zusammen: „Acht Skrale, bewaffnet mit Bögen, auf allen Seiten umzingelt außer hinten, Lysbett wahrscheinlich gefallen. Zeit, meine Rüstung wieder anzuziehen, haben wir wohl kaum. Wir haben mein Rankenschwert, den Lavastein und deine Magie.“ Dass er eine weitere Gemeinsamkeit mit den Skralen teilte, gefiel ihm ganz und gar nicht. Das war ja gruselig, dass man eine andere Sprache verstehen konnte, ohne zu kapieren, dass es eine andere Sprache war!

„Und du verstehst ihre Sprache“, ergänzte Jarid, „Kannst du sie auch sprechen?“

„Weiß ich doch nicht“, zischte Trieest zurück.

„Ahoi, Weinratten, wie lange wollt ihr noch warten da drinnen?“, rief Shron der Skralhäuptling nun, und diesmal glaubte Trieest zu verstehen, dass er die Sprache der Menschen sprach.

„Vielleicht ist ja keiner drinnen?“, sprach eine leise Stimme, ein weiterer Skral. Ja, das waren wiederum eindeutig nichtmenschliche Laute!

„Vielleicht ist ja keiner drinnen?“, versuchte Trieest leise, die Laute nachzuahmen.

„Und wer hat denn vorhin darin gescheppert, Madenhirn?!“, fluchte Häuptling Shron, „Los, Grobek, vorrücken!“

„‚Vielleicht ist ja keiner drinnen?‘“, wiederholte Trieest die unvertrauten und doch so verständlichen Worte. Dann setzte er nach: „Keiner drinnen? Niemand ist hier. Jemand ist hier. Dort. Überall. Unendlichkeit.“ Stück um Stück tastete er sich vorsichtig in diese fremde und doch so bekannt wirkende Sprache der Skrale vor. Er hatte bereits einmal eine fremde Sprache erlernen müssen, als es sich herausgestellt hatte, dass nur die wenigsten Völker des Südens die Sprache der Danware sprachen. Das Lernen hatte sich als äußerst mühsam herausgestellt. Selbst nach einigen Jahren intensiver Auseinandersetzung mit diesen fremden Kulturen hatte sich Trieests Zunge weiterhin oft an der falschen Stelle in seinem Mund befunden, um die richtigen Laute zu produzieren. Ganz zu schweigen davon, dass er sich nie an die passenden Worte zu erinnern schien.

Das hier war quasi das Gegenteil davon. Je mehr er Worte in dieser seltsamen Sprache der Skrale murmelte, desto mehr Worte kamen ihm in den Sinn. Die Laute und Krächzer, die er seiner Kehle entlockten, waren in keiner Weise gezwungen, sie flossen praktisch direkt aus seinem Hirn in die Luft. Jeder Laut erschien einfach so passend, so richtig. Es ergab alles so viel Sinn. Immer schneller sprach er, seine Gedanken sprangen von Assoziation zu Assoziation, und stets wusste er direkt, welchen Laut er von sich geben müsste, um den Gedanken zu kommunizieren. Die Skrale mochten scheußliche Finsterlinge sein, aber ihre Sprache war ein Wunder! Jarid hatte recht, dieses Phänomen zu erforschen, könnte so viele Erkenntnisse liefern.

Das Geräusch von dumpfen Schritten, die sich der Kutsche von außen näherten, holten ihn in die Realität zurück. Fokus, Trieest! Es gab Wichtigeres. Zunächst einmal mussten Jarid und er unversehrt aus dieser brenzligen Situation entkommen.

Jarid hantierte immer hektischer am Schloss von Trieests Kette herum und fluchte, als es nicht nachgab. Trieest bedachte ihre Chancen. Er war verletzt und rüstungslos, Jarid trug keine Waffen. Aber sie war eine Wassermagierin ... und alle sie umliegenden Fässer waren mit Met und Wein gefüllt! Trieest schöpfte wieder etwas Hoffnung.

Jarid hatte sich panisch umgesehen und schien zum selben Schluss wie Trieest gekommen zu sein, denn nun blitzten ihre Augen fröhlich auf und sie ließ von Trieests Ketten ab, hob ihre Hände an das nächstgelegene Fass und versteifte ihren Körper.

Ein leises Rumpeln ertönte aus dem Fass, aber nichts mehr.

Trieest sah er den glasigen Blick in Jarids Augen und seine Hoffnung sank, ebenso wie Jarid, welche zu Boden sank und sich an die Brust griff. Dort, wo Trieests Rankenschwert sie erwischt hatte. Verletzt, wie sie war, reichte ihre Kapazität nicht einmal aus, um Wein aus einem Fass zu leiten ... wie wollte sie es dann mit diesen Skral-Horden aufnehmen?

Sie hatten eine andere Option. Trieests Kenntnisse in der Skral-Sprache, und die Tatsache, dass er in Ketten lag, eröffneten diese andere Möglichkeit.

„Jarid“, zischte Trieest, und war überrascht, wie kompliziert ihm die Laute erschienen, die er nun in der menschlichen Sprache von sich gab.

„Jarid, wir werden hier rauskommen. Aber nicht durch direkte Konfrontation. Versteck dich hinter den Fässern. Ich werde mich als einen der ihren ausgeben. Vertrau mir.“

Jarid blieb einen kurzen Moment still und bedachte ihre Optionen. Dann nickte sie schwach und zog sich hinter die Fässer zurück. Trieest hörte sie bebend ein- und ausatmen.

Gute Güte. Wie sollte das alles nur enden? Er wappnete sich, ging die entsprechenden Laute im Kopf durch, holte tief Luft und brüllte dann:

„FREUNDE, HELFT MIR! ICH LIEGE IN KETTEN!“

Die Schritte vor der Kutsche verebbten und leises kehliges Gemurmel ertönte, gefolgt von Häuptling Shrons tiefer Stimme: „Welch Schande bist du, der du in einer menschlichen Kutsche reist und doch unsere Sprache sprichst?“

Trieest setzte an: „Ich reise nicht in dieser Kutsche ...“

Dann fiel ihm auf, dass das wohl nicht dem gepflogenen Umgangston zwischen diesen Skralen entsprach, und er setzte aggressiv nach: „Bist du taub oder was? Ich sagte doch bereits, dass ich in Ketten liege, du Irktiolt!“

Fast musste er grinsen. In der Skral-Sprache existierten Schimpfworte, für die es nicht einmal ein menschliches Äquivalent gab.

„Dich werd‘ ich lehren, mich einen Narren zu schimpfen, während du dich hast gefangen nehmen lassen“, grölte Häuptling Shron.

„Mich hat nur die Freude mitgerissen, die Stimmen der Unseren zu vernehmen“, rief Trieest nun, „Wenn du wüsstest, was diese Ziegenköpfe mir angetan haben!“

Stampfende Schritte, dann wurde die Zeltplache zu Trieests Linken zur Seite gerissen. Trieest konnte nur einen kurzen Blick auf den Lysbetts Körper werfen, aus welchem mehrere schwarz gefiederte Pfeile steckten, dann traten krumme Klauen in sein Blickfeld, als ein großer Skral mit einer mächtigen Axt in den Händen die Kutsche betrat. Sein auffälligstes Merkmal war die eiserne Maske, welche über seinen Mund gestülpt war und seine Stimme blechern klingen ließ.

„Hast dir ja lange Zeit gelassen, bis du dich gemeldet hast, Abomination!“, spie er Trieest ins Gesicht, „Mann, bist du hässlich! Was ist denn mit dir geschehen?“

Fast aus der Pistole geschossen antwortete Trieest, während seine Gedanken hektisch kreisten: „Musste mich zunächst von einem Knebel befreien. Dieser elende Stein in meiner Brust! Der hat sich in mich gefressen und meine Züge verändert. Diese kranken Menschen“ – er spuckte aus – „wollten mich zu einem der ihren machen. Testen, ob sich die ‚Verdorbenheit‘ in uns heilen lässt.“

Nervös achtete er auf die Reaktion des hochgewachsenen Skrals. Dieser musterte ihn prüfend. Dann hob der Skral seine mächtige Axt und ließ sie zweimal gezielt auf Trieests Ketten niederfahren. Trieest zuckte zweimal zusammen. Mit einem mächtigen Bersten brachen die Scharniere der Ketten ab. Er war frei!

Langsam, um nicht gefährlich zu erschienen, richtete Trieest sich auf, rieb seine Unterarme und streckte seinen Rücken durch. Erst jetzt fiel ihm auf, wie verspannt er war.

„Trikkest“, stellte er sich vor, und schlug dem hochgewachsenen Skral zur Begrüßung auf den Unterarm. Mit Schrecken stellte er fest, dass der Skralhäuptling dort einen eisernen Schoner mit einer rostigen Klinge daran befestigt hatte. Trieest verfehlte die Klinge nur knapp.

„Shron“, knurrte sein Befreier überrascht. „Aber du kannst mich Häuptling Shron nennen. Du bist nicht von hier.“

Keine Frage, eine Feststellung. Trieest dachte bei sich, dass es am besten wäre, so wenig wie möglich zu lügen.

„Ich komm‘ aus dem Hohen Norden, aus den stinkenden Höhlen der Ratteninsel Danwar. Da schaut unsereins ein wenig anders drein.“

„Ich glaub‘ dir kein Wort. Trägt unsereins im Hohen Norden wallendes Haupthaar?! Hat unsereins dort eine menschliche Anzahl Finger?“, fragte Shron grimmig und schlug Trieest mit dem Griff seiner Axt auf die Hand. Trieest verzog keine Miene und antwortete schlicht „So ist es.“ Woher sollte dieser Häuptling auch wissen, wie Nord-Skrale genau aussahen?

Glücklicherweise beließ Shron es dabei und stieß ihn erneut mit dem Axtstiel an.

„Hast du dich etwa allein von diesem alten Weib mit nur einem Arm gefangen nehmen lassen? Oder sind noch mehr dieser Ratten hier?“

Trieest fluchte innerlich. Falls er nichts sagen würde, und die Skrale Jarid dennoch fänden, wären sie beide geliefert. Falls er hingegen Jarid auslieferte, könnte er vielleicht ...

„Eine verfluchte Hexe reiste mit uns. Sie kann uns aber nicht mehr gefährlich werden, ich habe sie einmal gut in der Brust erwischt. Versteckt sich hinter den Fässern dort drüber, dieses feige Ding.“

Aufmerksam blickte Häuptling Shron auf die Stelle, in die Trieest zeigte. Dann, blitzschnell wie eine angreifende Vypera, stieß er nach vorne und riss die Wand aus Fässern zu Boden. Links und rechts platzte Holz. Met und Wein ergoss sich über den Boden zu einem stinkenden Mischmasch und tropfte durch die Kutsche auf den darunterliegenden Pfad.

Da lag sie, die arme Jarid, durchnässt und zusammengekauert, von ihrem Versteck aufsehend. Sie wagte es nicht, Trieest auch nur anzublicken. Schwach hob sie die Hand zur Verteidigung, flatterte mit ihren Fingern und ein klein wenig Wein erhob sich in runden Tropfen vom Boden. Shron war allerdings schneller, packte Jarid mit seiner freien Hand und zerrte sie unsanft aus der Kutsche. Jarid schrie auf. Trieest folgte mit beschleunigendem Herzschlag und erblickte, wie Jarid am Boden lag und ihren verletzten Torso hielt, während sich ihr Kleid wieder roter verfärbte. Die Lage war kritisch.

Nur einmal hatte Trieest hier in Andor einen anderen Halbskral gesehen. Dieser hatte blaue menschliche Augen gehabt, und schon allein deswegen war er von seiner Horde als Aussätziger behandelt und verstoßen worden. Für Trieest galt das nicht, und nun war er überglücklich darüber. Er versuchte, möglichst skralhaft aufzutreten. Während er die Kutsche verließ, fühlte er tief in den Lavastein hinein und bat ihn, das orange Leuchten seiner Augen verklingen zu lassen, auf dass das milchige Weiß seiner Kreaturenaugen darunter zum Vorschein kommen möge.

Trieest versuchte, nicht auf Lysbetts Leichnam zu achten, welcher immer noch achtlos auf dem Kutschbock lag. Er wusste, dass ihre Seele inzwischen bereits bei der Mutter Natur liegen musste und schwor sich innerlich, dass er bei Gelegenheit zurückkehren und ihren Körper begraben würde.

Nun durfte er allerdings keinen Argwohn auf sich ziehen. Neben Shron umringten sieben ihm unbekannte Skrale die Kutsche (oder bildeten besser gesagt ein Hufeisen um die Vorderseite). Die Pferde waren nirgends zu sehen, die Zügel der Kutsche waren leer. Aber auch diesem seltsamen Umstand konnte Trieest nicht zu viel Aufmerksamkeit schenken. Stattdessen konzentrierte er sich darauf, hoch aufgerichtet und mit bedachten Schritten vorwärtszuschreiten und jedem einzelnen der sieben bewaffneten Skrale überheblich in die Augen zu starren.

Du bist der Mächtigste hier, Trieest. Wenn du wolltest, könntest du jeden von denen im Zweikampf besiegen. Außer vielleicht den Obermacker, diesen Häuptling, Shron. Aber dazu würde es hoffentlich nicht kommen.

Die Skrale tuschelten aufgeregt miteinander und wechselten Blicke untereinander, ehe sie wieder zurück zu Trieest starrten. Fast hätte dieser aufgelacht. Diese Skrale zeigten ganz und gar nicht Verhaltensweisen, die er von blutrünstigen Bestien erwartet hätte. Dann erinnerte er sich an Lysbett und seine Laune sank wieder. Ganz zu schweigen davon, dass Jarid sich immer noch im Griff des Shron befand und von ihm in die Mitte der Skralsippe gezogen wurde.

Ein Glück, dass sie ihm seine Rüstung und Stiefel ausgezogen hatte. In nur seinem roten Fetzenmantel erinnerte er mehr an eine Kreatur. Abgesehen von Shron trugen die übrigen Skrale ebenfalls so gut wie keine Rüstungen, höchstens einige zerbrochene Stücke von Schulterplatten und Brustpanzern, die sie auf ihre Kleidung genäht hatten. Wardraks als Reittiere oder Gors als Lakaien waren ebenfalls keine zu sehen. Dies war keine Skralsippe in ihrer Blüte, sondern ein Haufen verlorener Halbstarker, die stärker auszusehen versuchten, als sie es in Wahrheit waren. Da die meisten von ihnen klobige schwarze Bögen trugen, wären sie dennoch in ihrer Gesamtheit keine leicht zu überwindende Bedrohung.

Ich bin einer von euch, auch wenn ich seltsam aussehe. Ich gehöre zu euch. Kein Grund, misstrauisch zu sein, flüsterte Trieest innerlich. Er spannte seine Gesäßmuskeln an und der knubbelige Stummel dessen, was vor der Amputation einst sein Skralschwanz gewesen war, hob sich in die Höhe und teilte sein rotes Fetzenkleid, auf dass die Skrale der Sippe einen Blick darauf erhaschen konnten. Ein Raunen ging durch die Menge und Trieest hätte erneut beinahe aufgelacht. Er war einfach nicht für derart angespannte Situationen geschaffen.

Shron hatte Jarid indes zu Boden geworfen und war einige Schritte zurückgetreten, während sich die restlichen Skrale um ihn herum versammelten. Er keifte: „Rovuk, du hast heute gut geschossen. Dir gebührt die Ehre, sie ihrem gerechten Ende hinzuzufügen und das erste Blut zu kosten.“

Jarid sah verzweifelt und verwirrt um sich. Die Worte mochte sie nicht verstehen, aber die Geste sprach für sich selbst, als Häuptling Shron einem kleingewachsenen Skral mit stämmigen Armen ehrenvoll ein langes schwarzes Messer überreichte. Abscheu stand in Rovuks viehischen Gesicht, als er vor Jarid.

Etwas musste geschehen, und zwar sofort. Noch ehe Trieest einen vollständigen Plan ausgearbeitet hatte, trat er vor den kleingewachsenen Rovuk und bellte: „Nein, wir brauchen sie noch! Sie ist die Einzige, die mich von diesem verfluchten Stein in meiner Brust befreien kann!“

Dann verfluchte er sich selbst, als ihm auffiel, dass er zwei unnötige Fehler begangen hatte: Zum einen konnte es den Skralen egal sein, wenn er weiterhin seinen Lavastein trug. Zum anderen hatte er sich nicht an den Häuptling gewandt. Und wenn er etwas von den hiesigen Skralen gelernt hatte, dann, dass sie es ganz und gar nicht mochten, wenn ihre Autorität in Frage gestellt war.

Rovuk nahm es gelassen und blickte nur fragend zu Shron, deutlich zeigend, dass er jedem Befehl seines Häuptlings Folge leisten würde.

Shron nahm es ganz und gar nicht gelassen und schlug Trieest grob auf die Schultern.

„‘\textit{Wir} brauchen sie noch‘?“, ahmte er Trieests Stimme nach, „Denkst du etwa, du bist einer von uns? Denkst du etwa, irgendeiner von uns schert sich auch nur einen feuchten Drachenkot darum, ob du lebst oder stirbst? Ich mag dich befreit haben, weil du Skralblut in dir trägst, aber wenn du mir noch einmal auf die Nerven gehst, werden ich dich höchstpersönlich der großen Echse im Himmel zuführen, \textit{Hallenlaskrala}!“

Kurz überlegte Trieest, ob es geschickter wäre, seinen Status als Halbskral zuzugeben. Immerhin kamen Halbskrale den Skralen derart unrein vor, dass sie Begegnungen mit solchen krankhaft mieden, ja, gar ihr Fleisch nie konsumieren würden, während sie bei anderen Skralen und bei Menschen weitaus weniger Skrupel davor hatten. Als Halbskral konnte er den Skralen derart eklig erscheinen, dass sie ihn vielleicht ziehen lassen würden. Andererseits war da immer noch Jarid, und als Halbskral hatte er keine Chance, den Respekt der restlichen Truppe zu erlangen und für ihre Freiheit zu verhandeln.

So wehrte er sich: „Bist du vergesslich oder einfach blöd? Ich bin kein Halbskral, ich komm‘ nur aus dem Norden, ignorantes Blechkinn!“.

Als Trieest ausspuckte, musste er seine Abscheu vor dem Wort „Halbskral“ nicht verstellen. Er sprach weiter: „Und ich sagte doch bereits, dass es dieser feurige Edelstein in meiner Brust ist, der mich ...“

„Sei still, du Wurm, und überlasse die Entscheidungen mir! Es gibt bestimmt noch andere auf dieser Welt, die dir mit diesem Stein-Problem aushelfen können. Die weisen Hexen Drunn und Trumm hätten dieses Steinchen bestimmt im Nu aus dir draußen. Man könnte fast meinen, dass diese Menschenfrau dir etwas bedeutet.“

Trieest bewegte sich immer noch nicht zur Seite und versperrte Shron so weiterhin den Weg zu Jarid. „Sie bedeutet mir nur so lange etwas, wie sie mir noch nützlich ist. Doch könnte sie auch euch noch nützlich sein, denn sie vermag, Formen von Flüssigkeiten zu wandeln und den Weg im tiefsten Dunkel zu finden!“

Trieest wusste, dass er zu spät war. Shron konnte seine Meinung nicht mehr ändern, ohne vor seiner Sippe schwach zu erscheinen, und Trieest wusste aus Erfahrung, dass es in den Augen der Skrale kaum eine größere Schande gab. Darum wollte Shron nun nichts mehr, als rasch seine Autorität durchzusetzen. Aber Trieest konnte nicht nachgeben. Jarids Leben stand auf dem Spiel. Und so blieb er tapfer stehen.

Shron erteilte ihm eine Ohrfeige, die ein unangenehmes Klingeln in Trieests Kopf verursachte. Danach blieb er vollkommen still und starrte Trieest provokativ in die weißen Augen, ohne auch nur einmal zu blinzeln. Leise, jedes Wort einzeln kostend, ehe er es ausspuckte, sprach er: „Willst du mich etwa herausfordern, Strohhirn?“

Trieest überlegte kurz.

Er traf eine Entscheidung.

„Ja“, sagte er dann, „Ich fordere dich zu einem Zweikampf um das Leben der Menschenfrau. Ich brauche sie zumindest so lange, bis der Stein aus meiner Brust ist.“

Er hätte nicht gedacht, je so etwas wie Überraschung in den kantigen Zügen des Häuptlings sehen zu können, aber so war es. Shron riss seine Augen auf und verzog sein Gesicht. Seine eiserne Maske rutschte nach unten und enthüllte Teile eines entstellten, echsenhaften Mauls mit einem schiefen Grinsen, aus welchem spitze Zähne in alle Himmelsrichtungen ragten. Unwillkürlich glitt Trieests Zunge über seine eigenen Beißwerkzeuge, welche im Gegensatz zu denen des Skrals glatt und menschlich aussahen.

Shron liess ein gutturales Lachen aus seiner Kehle entschlüpfen, während er seine Maske richtete und japste: „Mumm magst du ja haben, aber das Hirn eines Spatzen! Du kannst nicht einfach so deine eigenen Bedingungen stellen. Ich mag dich nicht. Bei diesem Zweikampf geht es um alles oder nichts! Wenn wir zwei uns kloppen, bleibt nur einer übrig, und dieser wird der Häuptling dieser mickrigen Sippe sein. Und der mag dann bestimmen, was mit der Menschenfrau geschieht. Und ich werde bestimmen, dass Rovuk sie schlachtet, wie es sich gehört!“

Kämpfe auf Leben und Tod hatte Trieest schon oft mit Kreaturen und Finsterlingen geführt, erst gerade kürzlich mit diesem verdorbenen Wesen, dem Hraak. Das hier war im Grunde genommen nichts anderes, redete er sich ein. Er mochte damit nicht recht haben, schließlich war das bislang der erste Kampf, den er und sein Kontrahent unter kontrollierten Bedingungen bestreiten würden. Aber das zählte nicht. Shron war genauso ein Monster wie alle anderen Skrale, denen sich Trieest bislang gestellt hatte, und die Welt würde ohne ihn ein besserer Ort sein. Er fasste sich.

„Wie du willst, Shron. Dann fordere ich dich zu einem Zweikampf um die Führung der Sippe heraus.“\bigskip







\textit{Nicht allzu weit entfernt regte sich ein vermeintlicher Leichnam, den ein Skral zur späteren Konsumierung achtlos beiseite geworfen hatte. Eine alte Hand griff nach einem schwarzen Pfeil und zog ihn aus einer Brust. Das Böse beobachtete fasziniert das rote Blut, das daraus hervorfloß.}

\textit{Lysbetts Körper richtete sich auf und blickte sich vorsichtig um. Keiner der Skrale achtete auf sie, alle starrten nur auf Trieest und diesen Häuptling mit der eisernen Maske.}

\textit{Das Böse warf sich ins nächste Gebüsch und stolperte von dannen, weg von diesen Kannibalen. Lysbetts Körper würde nicht lange hinhalten, es brauchte einen besseren!}

\textit{Eines von Lysbetts Beinen knickte ein und das Böse verschluckte sich beinahe am tiefschwarzen Kristallsplitter, den Lysbett unter ihrer Zunge versteckt hielt, in ständigem Kontakt mit ihrem Körper. Was für ein Glück es doch gewesen war, dass Lysbett diesen Splitter als ersten berührt hatte, ehe Jarid ihn erblickt hatte. Diese wäre nicht so nachlässig gewesen.}

\textit{Das Böse überdachte seine Situation kurz und spuckte den Splitter dann aus, in Lysbetts Hand hinein. Dort würde es ihn fortan festhalten, bis es ein besseres Gefäß für seine Macht fand. Am besten eines mit magischer Kapazität, aber aktuell war es alles andere als wählerisch.}

\textit{Eine Erinnerung an die Konfrontation mit Trieest und Jarid schoss durch den Kopf des Bösen. Als der Lavastein aufgeglüht hatte, hatte das Böse eine Stimme vernommen, eine ihm wohlbekannte Stimme, deren Träger doch schon lange tot sein sollte. Diesem Mysterium hatte es nachgehen wollen, sobald es mit seiner Kutsche zum Geheimversteck der Druidin gelangt wäre und die beiden Danware dort eingesperrt hätte. Nun würde es ihm halt allein nachgehen. Es brauchte Trieest nicht. Es hatte in seinen und Jarids Erinnerungen so viel mehr gesehen.}

\textit{Es hatte wieder ein Ziel, zum ersten Mal in so langer Zeit in seinem langen, nutzlosen Leben. Es hatte ein Ziel, und es wusste, wie es dorthin kommen konnte.}

\textit{So stahl sich das Böse in Richtung Norden davon, auf zur Küste.}










\newpage
\section{Eisenmaske gegen Lavastein}



\az{Jahr 61}

\textit{Halle des Ältestenrats, 61 a.Z.}\bigskip



Rowinda beugte sich vor und senkte ihre Stimme, als sie zu ihrer Tochter sprach: „Ich werde offen mit dir sein, Jarid. Der Orden der Feuerkrieger hat sich bereits gegen Trieest ausgesprochen. Sie wollen ihn nicht weiter ausbilden. Die Häuser des ...“

„Aber Mutter, seine Ausbildung hat doch kaum erst begonnen! Und für eine andere Profession ist ihm ...“

„Schweig, Jarid, bitte schweig und höre mir zu. Die Häuser des Aufenthalts haben sich gegen ihn ausgesprochen. Sie wollen ihn nicht mehr beherbergen. Die halbe Insel will nichts mehr mit ihm zu tun haben. Vielleicht ist es für ihn das Beste, andernorts einen Neuanfang zu wagen. Hier blüht ihm keine große Zukunft. Nicht so, wie er ist. Nicht nach dem, was geschehen ist.“

„Das ist doch die Höhe!“

„Ja, das ist wirklich die Höhe“, meldete sich nun die aufmüpfige Freiga zu Wort, „Jormudds halbes Gesicht hat er zerfleischt, und nun soll der Täter als Belohnung dafür eine Reise in die weite Welt geschenkt kriegen!“

„Freiga, wenn du noch einmal außerordentlich dein Wort erhebst, verweise ich dich des Saals! Ich verstehe deinen Ärger, wir alle verstehen ihn, aber das Protokoll muss gewahrt werden, sonst versinkt dieser Rat im Chaos.“\bigskip







\az{Jahr 68}

\textit{Sieben Jahre später.}\bigskip




Der Reaktionen der Skrale waren äußerst interessant zu beobachten gewesen. Die meisten hatten eine Mischung aus Abscheu und Neugierde gezeigt, als Trieest aus der Kutsche getreten war. Manche hatten gar gegähnt, als wären sie es nicht gewohnt, bei Tageslicht wach zu sein.

Als Trieest sich schützend vor Jarid gestellt hatte, hatte die Abscheu in den Gesichtern der restlichen Skrale Überhand gewonnen und nebst wütendem Schwanzwedeln hatte man den einen oder anderen Skral ausspucken sehen können. Nun, als Trieest sich gegen Shron stellte und diesen zum Zweikampf herausforderte, schienen die Meinungen auseinanderzulaufen. Einige Skrale, darunter dieser kleinere Skral mit den stämmigen Armen, welcher immer noch mit dem Opfermesser dastand ... ja, genau, Rovuk hieß er. Also eben, einige Skrale, darunter Rovuk, hatten ihre Münder zusammengekniffen, ihre Augen zu Schlitzen verengt und blickten nun schwanzwedelnd und grimmig in die Runde. Wie konnte diese dahergelaufene halbe Portion es wagen zu glauben, auch nur eine Chance gegen ihren mächtigen Häuptling zu haben?!

Einige andere Skrale schienen aber entspannter, fast neugierig. Trieest glaubte sogar, aus dem Augenwinkel ein hastiges Lächeln auf einem echsenhaften roten Gesicht aufblitzen zu sehen. Trieest musterte diesen Skral. Er trug grobes wollenes Wams, wie es die Fischer in Andor trugen, doch reichte dies bei weitem nicht, um seinen massigen Körper zu bedecken. Er schien kaum Rüstungsteile gefunden hatte. Dieser verfilzte Bart ... wie es wohl anfühlte, diesen zu kraulen?

Rasch senkte Trieest seinen Blick von diesem unbekannten Skral und musterte die beiden anderen Skrale, die neben ihm standen. Auch sie trugen weniger Rüstungsteile und praktisch leere Köcher, welche noch dazu um einiges ramponierter waren als den Köcher, die Rovuk trug.

Dies waren die Skrale mit der schlechtesten Ausrüstung, selbst in dieser mickrigen Sippe. Die Unterlinge? Vielleicht mochten sie es unter Häuptling Shron nicht? Vielleicht hofften sie ja auf einen Führungswechsel? Auf jeden Fall schienen sie keinesfalls empört darüber, dass sich jemand Shron im Kampf stellte. Aber belustigt wirkten sie auch nicht. Einfach nur ... entspannt.

Jarid war indes ganz und gar nicht entspannt, sondern blickte weiterhin furchterfüllt und still von einem Skral zum nächsten, ohne ein Wort davon zu verstehen, was gesprochen wurde. Trieest konnte nicht erkennen, wie viel dieser Panik gespielt war. Es war klar, dass diese Angelegenheit rasch beendet werden musste.

„Wo findet der Kampf statt?“, fragte er Häuptling Shron.

„Nicht so hastig, Todessüchtiger“, lachte der kleine Rovuk, „Ein Ungeheuer wie du hat gar kein Recht, einen so ehrwürdigen Häuptling wie Shron herauszufordern! Wie viele Menschen willst du schon ermordet haben?! Shron trägt derer neune an seinem Namen! Er entführte einst die höchste Schamanin des Yetohe-Stammes aus ihrer angeblich sicheren Zeltstadt, er war an der Ermordung des Diebeskönigs Brandur und an der Belagerung der Rietburg beteiligt, er sammelte nach dem Tod des großen Drachen die führungslosen Kreaturen unter seinem Heerbanner, ja, es gelang ihm gar in zahlreichen Gelegenheiten, den gefürchteten Helden von Andor zu entkom...“

Shrons Schwanz watschte gegen Rovuk und gebot ihm, das Sprechen sein zu lassen. Offenbar war Shron nicht allzu gut auf seine früheren Fluchten vor den Helden zu sprechen. Dies war also einer der Mörder von Brandur, der die Befreiung der Rietburg überlebt hatte. Vielleicht war Trieest ihm sogar schon einmal über den Weg gelaufen. Und er hatte glorreiche neun Menschen ermordet. Trieest lachte beinahe auf. Falls er heute das Zeitliche segnen würde, würde er ihnen mit seinem letzten Atemzug mitteilen, wie viele Mitglieder ihrer Spezies er bereits auf dem Gewissen hatte, das schwor er insgeheim. Dann würden sie große Augen machen.

Wie es sich herausstellte, sollte er die Gelegenheit dazu kriegen. Rovuk mochte einem Zweikampf zwischen Shron und Trieest gegenüber negativ eingestellt sein, aber Shron selbst war es nicht.

„Sei mal nicht so, Rovuk“, wies er den kleinen Skral zurecht, „Wenn unser Verfluchter hier einen Kampf will, so kann er einen kriegen. Ich bin mir sicher, dass er während des Duells zeigen kann, wie wenig Menschlichkeit tatsächlich in ihm steckt.“ Die letzten Worte hatte er spöttisch ausgespuckt. „Natürlich, da dies ein Kampf zwischen zwei vollwertigen Skralen sein soll, werden wir uns waffenlos gegenübertreten. Es sei denn, dieser verfluchte Stein in deiner Brust hat dich bereits so sehr vermenscht, dass du dich nicht mehr auf deine Fäuste, Fänge und Klauen verlassen kannst!“

Trieest teilte den versammelten Skralen laut mit, dass er sich natürlich sehr wohl auf seine Fäuste, Fänge und Klauen verlassen konnte. Im Innern sank seine Hoffnung. Im waffenlosen Training war er Jarid oft unterlegen. Und Jarid hatte selbst nicht zu den geschicktesten Kämpfern des Wassermagierorrdens gezählt.

Allerdings hatte er sich auch immer Mühe geben müssen, Jarid nicht zu verletzen. Als kleines Kind hatte Trieest zu oft die Kontrolle verloren. Der Vorfall mit Jormudd hatte ihn gar erst in seine jetzige Misere gebracht: Mit einem Lavastein in der Brust und einem nie enden wollen scheinenden Prozess des Wandels in der Zukunft. Vielleicht könnte Trieest es mit Shron aufnehmen, wenn er sich nicht zurückhielt? Konnte Trieest überhaupt wirklich loslassen und sich nicht zurückhalten?

Häuptling Shron bellte einige Befehle. Ein Skral führte zwei Pferde hinter der Kutsche hervor, vermutlich hatten jene die Kutsche gezogen. Dann gab es eine kurzzeitige Aufregung, als Shron den Transport „dieser alten Weinratte“ verlangte, und plötzlich niemand mehr den Leichnam der Weinhändlerin Lysbett auffinden konnte.

Zu guter Letzt befahl Shron Trieest, Jarid zu schultern und ihm zu folgen. Vielleicht wollte er ihn dadurch bereits vor dem Kampf etwas schwächen, aber Trieest war es einerlei. Er hatte es viel lieber, Jarid selbst zu tragen, als dass einer dieser anderen Skrale sich ihrer annahm. Sorgsam darauf achtend, sich nicht anmerken zu lassen, wie viel Jarid ihm bedeutete, folgte Trieest dem Häuptling. Er drückte Jarids Hand sanft und hoffte, dass sie das als Zeichen der Zuversicht interpretieren könnte. Jarid regte sich nur schwach auf seiner Schulter.

Die übrigen Skrale tuschelten wieder miteinander und umkreisten Trieest sorgsam, auf dass er nicht entkommen konnte. Eine Flucht wäre vergeblich, selbst wenn Jarid ihn nicht verlangsamte.

Die Truppe erreichte den Lagerplatz der Skrale, eine moosige Lichtung mitten im Wachsamen Wald. Ein kleines Lagerfeuer stand neben einem behelfsmäßig errichteten Zelt am Rande. Ein ebenso behelfsmäßiges Banner steckte vor dem Zelt, mehr nur ein roter Fetzen an einer Stange als die glorreiche Flagge, die das Banner zu sein versuchte. Widerwillig war Trieest fasziniert. Er hatte noch nie einen Skral-Unterschlupf mit eigenen Augen gesehen. Jarid ruckelte auf Trieests Schulter. Versuchte sie ebenfalls, sich einen Überblick über das Lager zu verschaffen?

Skrale stellten nur wenig Kleidung, Waffen oder andere Materialien selbst her. Stattdessen verließen sie sich größtenteils darauf, von Menschen und Zwergen zu klauen und wiederzuverwerten. So war das Zelt aus nicht zusammenpassenden Stoffplachen und vereinzelten Brettern aufgebaut. Es war zu klein, als dass die gesamte Skralsippe darin Platz finden könnte. Shron wirkte nicht wie einer, den es störte, wenn seine Untergebenen draußen in der Kälte nächtigten, solange er selbst ein warmes Zeltinneres hatte. Einige Blätterhaufen und weitere Stofffetzen lagen rund um eine angeschlagene hölzerne Kiste, aus welcher weitere Pfeile und Schwerter ragten. Wahrscheinlich fanden die Unterlinge in der Rangordnung der Sippe dort Unterkunft für die Nacht.

Zelt und Blätterhaufen lagen inmitten einer Kuhle im sonst erhöhten Gebiet. Das war keine optimale Lage: falls es regnen würde, würde die gesamte Ausrüstung durchnässt werden und der Regen sich in dieser Kuhle sammeln.

Wie um Trieests Gedanken zu bestätigen, ertönte am Himmel ein leises Rumpeln und ein Nieselregen setzte ein. Na großartig, dachte Trieest. Es musste ja so kommen.

Sorgfältig setzte Trieest Jarid auf der feuchten Erde ab. Ein letztes Mal noch blickte er ihr zuversichtlich in die Augen. Ihre starrten stumpf und glasig zurück. Rasch löste er sich von ihr und wandte sich abrupt ab.

Irrte er sich oder hatte Jarid ihm noch ein leises „Tri“ zugeflüstert? Er musste diesen Kampf, seine Herausforderung gegenüber Häuptling Shron, so schnell wie möglich hinter sich bringen und sich danach um die verletzte Jarid kümmern.

Als er sich umdrehte, hatte Häuptling Shron sich bereits seiner Schulterplatten und Armklinge entledigt. Abgesehen von seiner eisernen Maske mit nichts als einem Lendenschurz bekleidet, schritt der Häuptling stolz auf Trieest zu. Dieser wäre beinahe peinlich berührt zurückgewichen, überlegte es sich dann allerdings anders und zog sein eigenes Wams über seinen Kopf, um Shron ebenso ungeschützt und scheinbar ebenso unbekümmert gegenüberzutreten.

Aus dem Augenwinkel bekam er mit, wie einer der Skrale Jarid fesselte und ins rissige Zelt bugsierte. Zu seiner Erleichterung verließ der Skral das Zelt kurz danach allerdings wieder. Offenbar wollte keiner der Skrale das bevorstehende Spektakel verpassen.

Betont ruhig stand Trieest vor Shron und betrachtete sein Gegenüber, ohne ihn wirklich zu erkennen. In seinem Kopf ging er die Meditationsübungen durch, die ihm die Danware während seiner Ausbildung und später Jarid immer und immer wieder vorzeigt hatten. Einatmen. Ausatmen. Einatmen. Ausatmen. Seine Nervosität in den Griff kriegen. Das Feuer der Furcht – Drenlynn nannten die Feuerkrieger es – würde noch früh genug durch seine Adern strömen. Er musste sich nur auf das hier und jetzt konzentrieren, seinen Körper in den Griff kriegen, eins mit sich selbst und dem Lavastein sein. Dann würde alles so kommen, wie es kommen musste. Er hatte dies im Griff. Er hatte schon so oft Zweikämpfe mit Skralen gewonnen. Er wusste, wie sein Gegenüber dachte. Die vielen Male, in denen er Jarid gegenüber in Trainings-Zweikämpfen unterlegen gewesen war, die würden sich nun auszahlen. Und er konnte auf die Stärke des Lavasteins zurückgreifen, auch wenn er später in Schmerz dafür bezahlen würde. Trieest konnte diesen Kampf gewinnen.

Häuptling Shron schien keinen solch beruhigenden Gedanken nachzuhängen. Stattdessen schritt er unruhig auf und ab, schlug in die Luft und peitschte mit seiner Schwanzspitze, während einer seiner Lakaien ein wenig Gerümpel zur Seite rollte.

„Ach, es ist zu lange her, dass ich einen angemessenen Kampf hatte“, lamentierte Shron. „Diese elenden Helden von Andor haben mich meinen Vater gekostet, mein Ansehen, meinen Rang als Häuptling einer ehrenwerten Skral-Sippe! Es vergeht kein Tag, an dem meine Wut auf sie mich nicht anführt. Mit dieser Wut konnte ich bislang jeden einzelnen meiner Gegner bezwingen. Du hast keine Chance!“

Trieest zweifelte an der Effizienz von Shrons Wut, dem schrecklichen Zustand vom kläglichen Rest seiner Sippe nach zu urteilen. Er war aber äußerst froh darüber, dass Jarid und er während der Befreiung der Rietburg nach noch nicht so öffentlich als Helden von Andor aufgetreten waren. Nicht auszudenken, wie Shron reagiert hätte, hätte er sie erkannt.

„Es gibt keinen Ring und keine Regeln“, flüsterte Shron heiser, „Wenn dieser Kampf einmal begonnen hat, endet er nicht, ehe einer von uns mit den Drachen fliegt. Du magst um dein Leben betteln, doch von mir wird es keine Gnade geben.“

„Gut so“, knurrte Trieest. Er hatte keine Lust, Shron am Leben zu lassen. Trieest war kein Mörder, aber er hatte er sich schon vieler Kreaturen entledigt, Maschinen des Todes, die sie waren, und die Welt war dadurch ein besserer Ort geworden. Bei Shron würde das genauso sein.

Trieest konzentrierte sich auf den Lavastein und bat um Unterstützung. Er atmete erleichtert auf, als er die vertrauten Gefühle in sich aufsteigen fühlte. Der Stein erwärmte sich trotz des dicken Spalts auf seiner Oberfläche und begann zu schimmern. Trieests Körpertemperatur stieg zunächst kaum merklich, dann immer stärker an. Der Nieselregen tropfte nicht mehr an ihm herunter, sondern verdampfte zischend von seiner Haut. Trieests Sichtfeld verfärbte sich feuerrot, als das vertraute Glühen in seinen Augen wieder einsetzte. Das waren bislang rein kosmetische Veränderungen, die auf seine Kampffertigkeiten keinen Einfluss hatten, aber vielleicht konnten sie Shron einschüchtern.

Von irgendwoher ertönte ein blecherner Gong. Trieests Blick zuckte zur Quelle und enthüllte, dass es vielmehr das Geräusch eines gezackten Schwerts auf einem blechernen Schild gewesen war.

Das war wohl das Startsignal für den Zweikampf gewesen, denn Shron rannte ohne Verzögerung auf Trieest los und hatte seinen krallenbesetzen Fuß bereits zweimal zwischen Trieests Beine gepflanzt, ehe Trieest überhaupt reagieren konnte. Dann drehte er sich gewandt um und wischte Trieests Beine nonchalant mit seinem Schwanz zur Seite. Trieest ging zu Boden, schmeckte Matsch und konnte ein schmerzerfülltes Stöhnen nicht unterdrücken.

Shron drehte sich zurück und trat unerbittlich auf Trieest ein. Trieest spürte leichte Kratzer und zukünftige blaue Flecken, aber nichts Schlimmeres. Das Feuer der Furcht floss durch ihn und schwächte sämtliche Schmerzempfindungen ab.

Ehe Shron ihm Schlimmeres zufügen konnte, griff Trieest seinerseits an. Shron stand über ihm, folglich musste Trieest ihn zunächst zu Fall bringen. Blitzschnell drehte Trieest sich aus seiner Embryonalstellung auf den Rücken und trat nach Shron, nicht ungezielt, sondern an eine spezifische Stelle über der Ferse, die Jarid ihm einst bei Menschen gezeigt hatte.

Shrons rechtes Bein knickte ein und Trieest schwang sich hoch, traf Shron mit der flachen Hand an der Brust, ergriff Shrons Schulter und schleuderte ihn zu Boden, während er sich selbst in die Höhe stemmte. Er heizte seinen Lavastein an. Bald könnte er auf dessen Fähigkeiten angewiesen sein, doch wollte er sich nicht zu rasch ausbrennen.

Shron mochte keine großartige Taktik haben, doch seine reine Muskelkraft und Robustheit konnten diesen Nachteil definitiv wettmachen. Andere wären nach Trieests Tritt nicht wieder aufgestanden, doch Shron knurrte bloß kurz und warf sich wieder in die Höhe. Dann verharrte er.

Trieest hielt ebenfalls inne.

Er war es gewohnt, dass Skrale sich ihm beinahe willenlos entgegenwarfen, praktisch leere Hüllen für die finstere Macht, die sie antrieb, egal ob die Skrale nun gerade Werkzeuge für Drachen, Dunkle Magier, Finstere Herolde oder Nekromanten waren. Das waren Skrale gewesen, die selbst nach gröbsten Verletzungen nicht aufgaben, sich stattdessen wieder erhoben und bis zum letzten Tropfen schwarzen Bluts in ihren Adern weiterstritten.

Shron war nicht so. Stattdessen blieb er stehen und starrte Trieest unruhig an. Er musterte Trieests Brust mit dem leuchtenden Lavastein. Auch für Shron musste dieser Kampf eine neue Erfahrung sein. Er hatte wohl geplant, Trieest rasch und brutal zu erledigen. Stattdessen war Trieest standhaft geblieben (so standhaft, wie man sich nennen durfte, wenn man am Boden lag) und hatte sich zu wehren gewusst. Und dann erst das Bild, das Trieest abgab: Orange leuchtende Augen, gebleckte Zähne, zischend an seiner Haut verdampfende Regentropfen – und leichte orangerote Feuerschlieren, die seinen den Edelstein in seiner Brust umgaben. Trieest sah zum Fürchten aus und immer noch so fit wie zu Beginn des Kampfes.

Shrons Augen verengten sich, und Trieest konnte für einen Herzschlag durch seine Fassade hindurchsehen.

Der Häuptling hatte Todesangst. Er wusste nicht, was Trieest war, und wozu er fähig war. Er wollte nur weg von hier. Aber er konnte nicht. Nicht, ohne sich eine Blöße zu geben.

Eine Erinnerung flackerte durch Trieests Geist, eine, die er schon viel zu lange unterdrückt hatte. Der Blick des kleinen Jormudd aus Danwar, als der noch viel zu kleine Trieest sich auf ihn gestürzt hatte. Der schiere Schrecken in Jormudds Augen würde Trieest nie vergessen können. Entsetzen, Furcht, Verwirrtheit, als wäre Trieest ein wildes Tier gewesen und kein Mensch, kein denkendes, fühlendes Wesen. Und diesen Blick erkannte Trieest nun in Shron wieder.

In diesem Moment erkannte Trieest, dass sich hinter dieser dunklen Kreatur vor ihm kein tumbes Vieh, sondern eine denkende, fühlende Person verbarg. In diesem Augenblick hatte Trieest Mitleid mit Shron. Mit ihm, einem elenden Skral!

Dann war der Augenblick vorbei. Shron brüllte blechern auf, warf sich Trieest entgegen und bohrte ihm seine Klauen in die Seite. Trieest schrie ebenfalls auf, ergriff Shrons mächtigen Rumpf und packte ihn an einem bestimmten Bereich in seinem Rücken, wo er bei einem Menschen gehörigen Schaden anrichten hätte können. Bei Shron hingegen traf er nur auf steinharte Schuppen.

Trieest konnte immer noch die Todesfurcht in Shrons Augen erkennen, doch dies hielt den Häuptling nicht davon ab, immer und immer wieder auf Trieests Brust einzuschlagen, während Halbskral und Skral gemeinsam im Matsch umherrollten.

Zunächst konnte Trieest keinen Sinn hinter Shrons Verhalten sehen, doch dann erkannte er, was der Häuptling vorhatte. Er hatte den Lavastein als Quelle von Trieests Kraft identifiziert und versuchte, ihn zu zerstören.

Viel Glück dabei, dachte Trieest spöttisch, das habe ich in sieben Jahren nicht geschafft.

In diesem Moment knirschte etwas unter Shrons blutigen Knöcheln und der Spalt auf dem Lavastein vergrößerte sich. Trieest fühlte Panik vom Lavastein hinüberschwappen.

Ausgerechnet jetzt bröckelte der Lavastein?! Hatte dies damit zu tun, dass er bereits vorhin vom bösen Hraak angesplittert worden war? Konnte er diesen Kampf noch überstehen?

Shron knurrte erleichtert und grub seine Krallen in Trieest Brust, versuchte mit aller Kraft, den Lavastein zu fassen zu kriegen. Trieest keuchte auf. Sein Arm wurde glühend heiß, schoss hoch und bohrte sich tief ihn Shrons Brust, ehe Trieest den Befehl dazu gegeben hatte.

Zwei, drei, vier Ringe aus gleißenden Feuerschleiern umringten Trieest und brannten sich in ihn und seinen Gegner zugleich.

Wie Trieest und Shron da standen, die Krallen jeweils in die Brust des Gegenübers gegraben, sang der Lavastein in Trieests Brust auf einmal glockenhell. Trieest blickte an sich herunter. Die Spalten im Stein glühten auf, zuerst langsam, dann heller und heller. Shron brüllte ein letztes Mal auf, der Ruf eines Skrals, der meilenweit zu hören sein musste und Trieests sensible Ohren unangenehm dröhnen ließ.

Shrons Brüllen wurde allerdings abrupt abgerissen, als \textit{die ganze Welt in orange Töne getaucht wurde und der Boden unter Trieests nackten Füssen zur Seite kippte. Ins Nichts.}

\textit{Dann war Trieest allein.}

\textit{Allein in einer feuerroten Welt.}\bigskip







\textit{Trieest schwebte in einem Wirbel aus gelben, orangen und roten Flammen, die miteinander tanzten. Es gab keinen Boden unter seinen Füßen und keinen Himmel über seinem Kopf. Wärme umspielte seinen nackten Körper, doch fühlte er trotz der lodernden Flammen um ihn herum keinen Schmerz.}

\textit{Mehr Feuer als Luft gab es hier, wie Trieest auffiel, als er panisch versuchte, einzuatmen. Erst nach einigen verzweifelten Japsern kam er zum Schluss, dass Atemluft an diesem Ort offenbar nicht benötigt war, um zu überleben. Bei Kenvilars Dreizack, wo befand er sich?}

\textit{Überall um ihn herum flackerten helle Flammen ohne Quelle und ohne Rauch. Manche leckten an seinem blutigen Körper, doch fühlten sie sich kaum heißer an als warme Sonnenstrahlen an einem Sommertag.}

\textit{Ein unförmiges Gesicht schälte sich aus den Flammen vor Trieests Augen, nichts weiter als ein weißes Grinsen mit strahlend weißen Augen, die Skizze einer Miene vor leuchtenden Flammen.}

\textit{Das Gesicht sprach mit einer tiefen, dröhnenden Stimme: „Sieben Jahre habe ich dich nun schon begleitet, Trieest, Feuerkrieger aus Danwar. Doch jetzt ist es an der Zeit für mich, dich zu verlassen.“}

\textit{„Mein Lavastein? Bist du das?“, fragte Trieest verblüfft.}

\textit{„Natürlich! Du hast mich doch oft genug gefühlt, um mich zu erkennen.“}

\textit{„Du kannst sprechen?!“, rief Trieest empört aus.}

\textit{„Ja, ja, ich kann sprechen, o Wunder“, rief das flackernde Gesicht des Lavasteins aus. „Mir bleibt nicht viel Zeit, denn ich muss gehen. Ich ... ich wollte mich bloß noch einmal melden und sichergehen, dass ich mit dir im Reinen bin, ehe ich diese Welt verlasse.“}

\textit{Trieest ließ die Worte auf sich einwirken. Dann beschloss er, sich auf die absurde Situation einzulassen. Zornig sprach er: „Jetzt hältst du es für die passende Gelegenheit, mich um Vergebung zu bitten?“}

\textit{„Ich gebe zu, dass dies unpraktisch ist. Es geht nicht anders. Bist du etwa froh, dass ich dich jetzt verlasse, Trieest?“}

\textit{Trieest blieb kalt: „Du hast mich jahrelang gequält. Meine Schuld sollte schon längst beglichen sein. Mein Prozess des Wandels hätte schon längst abgeschlossen sein sollen. Doch du weigertest dich, mich gehen zu lassen. Und zu einem vollwertigen Menschen machtest du mich auch nicht. Du wirst deine Gründe dafür haben. Dennoch hast du falsch gehandelt. Es hätte mir bereits enorm geholfen, ein Gesicht zu dir zu haben, mit dir zu kommunizieren, statt bloß einen stummen Schmerz in meiner Brust lodern zu spüren!“}

\textit{Der Lavastein – das Gesicht im Wogen der Flammen – blieb still. Falls es überhaupt eine Emotion zeigen konnte, so tat es dies nicht. Doch fühlte Trieest eine Verlegenheit in seiner Brust, die nicht die seine war.}

\textit{Trieest relativierte seine Aussage: „Nichtsdestotrotz will ich natürlich nicht, dass du meinetwegen stirbst. Ich ... ich dachte, dass man den Stein eines Tages entfernen könnte und deine Seele oder so erhalten bliebe.“}

\textit{Das Gesicht lachte auf: „Ich habe schon längst keine Seele mehr, ich bin doch schon lange tot! Ich bin einer der Alten, der Vorfahren, der Stimmen in der Roten Grotte, ein Echo eines Toten, ein Nachhall aus längst vergangenen Zeiten, weiter nichts. Um mich brauchst du mich nicht zu kümmern. Nicht mehr. Hiermit erlöse ich dich nun förmlich von deiner Bürde, Feuerkrieger Trieest. Du hast weitaus mehr Gutes getan, als deine Schuld von dir verlangte. Und nun kann ich auch mit gutem Gewissen sagen, dass keinerlei Gefahr besteht, dass du wieder die Kontrolle über dich verlieren könntest. Du darfst hiermit nach Danwar zurückkehren.“}

\textit{Trieest lachte auf: „Jetzt, wo du gehen musst, willst du es plötzlich noch so drehen, dass du im Recht warst?! Ich habe nichts als Unfairness und Verachtung von dir erfahren!“}

\textit{„Das stimmt nicht. Ich habe dich oft geheilt, dir oft das Leben gerettet. Dir und Jarid!“}

\textit{„Und genauso oft ... nein. Das ist es nicht wert. Lass uns nicht so enden. Wenn du gehen willst oder gehen musst, so gehe doch einfach. Gehe und lasse mich in Frieden.“}

\textit{„Bald, Trieest, bald. Kannst du noch einen kurzen Moment des Wartens verkraften?“}

\textit{Der Lavastein schien ernsthaft besorgt. Widerwillig willigte Trieest ein. Das Gesicht zeigte ein schiefes Grinsen und fuhr dann fort: „Es kommt selten vor, dass ein Lavastein Danwar verlässt. Hier gibt es keine solchen Steine, die Tote simulieren können. Diese fremden Lande des Südens, durch die du reistest ... in all diesen Landen hingen so viele Stimmen der Gefallenen fest, ohne dass es ein Ohr gäbe, welches sie hören könnte. So viele Verstorbene, deren letzte Nachhalle in der Erde verankert blieben, bis einer vorbeikommt und ihnen ihre letzten Sekunden der Bedeutung schenkt. Auf all deinen Reisen durch diese Welt konnte ich so einige dieser konservierten Nachhalle erhaschen und ... und wenn ich jetzt gehen muss, so würde ich sie gerne erklingen lassen. So viele, wie ich in meiner schwindenden Zeit zu vermitteln vermag. Da sind so einige, die sprechen wollen.“}\bigskip



\textit{Mit diesen Worten verschwand das weiße Gesicht im Flammenmeer und die Flammen teilten sich, um den Blick auf eine kleine grünhäutige Frau mit verfilzten Haaren und erdiger Kleidung freizugeben. Die Frau stürzte nach vorn, direkt vor Trieest, welcher immer noch schwerelos im Flammenmeer schwebte. Gemeinsam umkreisten sich diese beiden Gestalten im leeren Raum.}

\textit{„Finde meinen tapferen Wolfskrieger. Richte ihm aus, dass ich auf ihn warte. Dass er mich wiedersehen wird. Er soll sich nicht beeilen, er hat alle Zeit der Welt und soll sie genießen, aber richte ihm aus, dass ich auf ihn warte. Mein Orfen muss seine Fehde nicht zu Ende führen. Er soll sich nicht in den Hass auf eine ganze Kreaturenart hineinsteigern. Es gibt so viel Schönes im Leben, und so viele Schöne, und der Hass verbirgt dies vor ihm.“}

\textit{So sprach die grünhäutige Frau. Sie präsentierte Trieest ihre kleine rechte Hand, an deren Mittelfinger ein schnörkelloser silberner Ring steckte. Dann, von einem Lidschlag zum nächsten, war sie verschwunden.}\bigskip



\textit{Trieest hatte kaum Zeit, sich über diese Begegnung zu wundern, da teilten sich die Flammen erneut und eine in ein elegantes dunkles Kleid gewandte, blauhäutige Frau mit langem, schneeweißem Haar schwebte vor Trieest. Sie griff mit langen Fingern nach Trieests Händen, presste diese fest zusammen und sprach zitternd: „Kannst ... kannst du ihm mitteilen, dass ich ihm verzeihe? Meine blaue Blüte in der Dunkelheit. Ihm und seinem Bruder. Die beiden mussten vieles erdulden und wurden irregeleitet, doch er muss nicht daran festhalten, ihn zu suchen. Er plant, so viel Leid anzurichten, um ihn wiederzusehen, doch das muss er nicht. Sie werden sich auch so wiedersehen, frei und unverflucht. Er kann sich entspannen. muss keine weitere Pein auslösen. Aber das Wichtigste, wie schon gesagt: Ich – wir verzeihen ihm für alles, was er bislang getan hat, und was er noch tun wird. Richte ihm das aus.“}

\textit{Ein weiteres Blinzeln Trieests, und die blauhäutige Frau war verschwunden.}\bigskip



\textit{Erneut teilten sich die lodernden Flammen und ein weißhaariger Mann mit einem langen Bart glitt majestätisch nach vorne. Dieses Gesicht erkannte Trieest sofort, auch wenn er es bislang stets bloß mit einer goldenen gewellten Krone auf dem Schopf gesehen hatte – zum ersten Mal vor fast fünf Jahren, als Trieest auf der Spur einer geheimnisvollen Eiskreatur, die er in einer seltsam klaren Vision gesehen hatte, bis zur Rietburg gereist war.}

\textit{Er fasste sich: „Mein König! Ihr seid...“}

\textit{Brandur unterbrach Trieest, ohne auf seine Worte einzugehen: „Ich konnte es doch nicht ahnen ... warum hat er nichts gesagt? Warum musste Melkart sein Wort geben?! Sie ahnt es noch nicht einmal, doch muss sie es unbedingt erfahren.“}

\textit{Nahm der verstorbene König Trieest überhaupt wahr? Mit tränennassen Augen flüsterte Brandur weiter: „Nein, ich hätte es wissen müssen! Ich hätte bei ihr sein sollen!“}

\textit{Dann starrte er Trieest in die Augen: „Kannst du ihr ausrichten, dass es mir so unendlich leid tut? Kannst du ihr sagen ... sag ihr, dass ich sie liebe, wirst du? Und dass ich unglaublich stolz auf sie bin. Ich kann mich so glücklich schätzen, sie meine Tochter nennen zu dürfen.“}

\textit{„Wen meint Ihr, von wem sprecht Ihr, Herr?“, fragte Trieest.}

\textit{„Oh, meine tapfere Tochter ... ich habe eine Tochter!“, rief Brandur mehr zu sich selbst als zu Trieest. Seine Augen füllten sich erneut mit Tränen. Mit einem Ruck wurde er wieder in das wogende Meer der Flammen gezogen.}\bigskip



\textit{An Brandurs Stelle trat eine schwarzhaarige Frau mit einer spitzen Nase, um deren Hals ein rhombisches Amulett hing, dessen Oberfläche sich kräuselte.}

\textit{„Nicht nur von Brandur! Bitte, teile ihr auch von mir mit, dass nicht ein Tag vergeht, an dem ich nicht wünschte, bei ihr sein zu dürfen. Mir, ihrer Mutter Mhare. Mit ihrem Willen bringt sie Licht in jedes auch noch so tiefe Dunkel und ich bin so, so stolz ...“}

\textit{„Mhare? Von wem sprecht ihr? Wer seid Ihr überhaupt?!“, rief Trieest ihr entgegen, doch die grün gewandete Frau brabbelte weiter, als wäre Trieest gar nicht hier.}

\textit{Hände tauchten aus der Flammenwand vor Trieest auf und zogen die Frau mit dem Amulett zurück ins Feuer.}\bigskip



\textit{Die Flammen teilten sich und gaben den Blick frei auf die Besitzerin dieser Hände: Eine weitere schwarzhaarige Frau, und diese kam Trieest bekannt vor, auch wenn er sie nicht sofort einordnen konnte. Sie trug eine kleine Krone auf dem Kopf, wie ein goldener Blumenkranz, und sprach flehend: „Mein Sohn, mein kleiner Prinz! Er ist der Last nicht gewachsen, die auf seine Schultern gelegt wurde, und ein dunkler Schemen versucht, sich seiner zu bemächtigen. Ken dürstet nach seinem Amt. Das Land ist darauf angewiesen, dass mein Sohn seine Ratgeber weise wählt. Ich bitte euch, gebt ihm guten Rat und helft ihm zu erkennen, dass den Thron nicht behalten muss ... es läge keine Schande darin ... ich wünschte, ihn nicht länger leiden zu sehen unter dem Gewicht seiner falschen Entscheidungen ... ihn und das Reich ...“}

\textit{Sie brach ab, als zwei weitere Händepaare in den Flammen erschienen und die Frau aus Trieests Blickfeld zogen.}\bigskip



\textit{Zwei Personen traten gleichzeitig durch die Feuerwand auf Trieest zu. Hinter den sich teilenden Flammen waren kurzzeitig dutzende weitere Seelen zu erkennen, welche sich um einen Platz im Scheinwerferlicht drängelten. Wie viele Echos der Toten hatten sich hier versammelt?}

\textit{Die erste der beiden Gestalten, die nun von Trieest standen, war ein schwarz gewandeter Bewahrer aus dem Wachsamen Wald, welcher, die Kapuze tief ins Gesicht gezogen, murmelte: „Teile ihm mit, dass er das alles nicht verdient hat. Sein ganzer Ruhm ist unverdient! Ein lügnerischer Verräter ist er, und wenn er nicht gewesen wäre, würde Folla heute noch leben! Wenn ich ihn in die Finger gekriegt hätte, hätte ihm diese Angelegenheit mit der Ziege und dem Sternbild des Hornfalken endlich leid getan! Und dieses elende selbstgerechte Grinsen auf seinem Gesicht hätte ...“}

\textit{Die wütende Schwarze Wache wurde von der zweiten Gestalt unsanft zur Seite geschubst und verschwand in der Flammenwand.}\bigskip



\textit{Bei der zweiten Gestalt handelte es sich um eine ältere Frau mit einem von Flicken übersäten Gewand, in deren Körper immer noch zwei schwarze Pfeile steckten. Eine primitive Augenklappe zierte ihr linkes Auge. Jetzt, wo Trieest ihren ganzen Körper sehen konnte, erkannte er, dass die Brandnarben auf ihrer linken Gesichtshälfte sich bis weit über ihre Schulter zogen, und dass ihr linker Arm fast zur Gänze fehlte.}

\textit{„Trieest, häh?“, fragte die Alte ihn krächzend nach seinem Namen.}

\textit{Trieest atmete auf. Endlich jemand, der ihn wahrzunehmen, ja, gar zu erkennen schien.}

\textit{„Viel Zeit bleibt mir nicht“, flüsterte Lysbett die Weinhändlerin, „Dieser feurige Stein hat keine Ahnung, was er tut. Die Echos dunkler Gestalten sind auf dem Weg hierher, also rasch! Mein Appell ist wichtiger als der ihre. Du hast mich nie wirklich kennen gelernt, doch habe ich vorhin dir und deiner Begleiterin das Leben gerettet. Meines musste ich nun leider lassen, doch du kannst dich auf eine andere Art revanchieren. Der finstere Geist, welcher in diesem roten Kristall des Hraak steckte, ist noch immer Teil deiner Welt. Er zog sich in einen kleinen Splitter zurück, einen tiefschwarzen, den ich törichterweise anfasste, als Jarid sich noch nicht neben mich gestellt hatte. Das Böse war es, das Jarid und dich in meinem Körper auf einer Kutsche in den Norden fuhr. Fürchtete sich wohl davor, euch offen zu konfrontieren. Doch nicht einmal der Skralangriff konnte es erledigen. Trotz der schwarzen Pfeile in meinem Körper weigert das Böse sich, ihn loszulassen. Der böse Geist hat dem Tod bereits einmal ein Schnippchen geschlagen, und er wird es wieder tun. In diesem Augenblick dirigiert er meinen Leichnam weiter in den Norden. Ich weiß nicht, was sein Ziel ist, aber ein gutes wird es nicht sein. Verfolge ihn! Vernichte den schwarzen Kristallsplitter! Oder vernichte denjenigen, der den Splitter trägt. Eine andere Wahl hast du nicht!“}

\textit{„Das Böse kontrollierte deinen Körper die ganze Zeit? Was hat es vor?“, rief Trieest, doch von einem Augenblick zum nächsten war Lysbett verschwunden. Ein dunkles Rumpeln kündigte die Ankunft des nächsten Abbilds an. Doch etwas war anders dieses Mal ... waren das etwa Schreie von außerhalb der Flammenwand? Ein dumpfes Brüllen? Was konnte dies ...}\bigskip



\textit{Das Flammenmeer teilte sich ein letztes Mal und ein riesiger, unmenschlicher, schwarz geschuppter Schädel mit glühend roten Augen ragte hindurch. Drachenstacheln und scharfe Schuppen sägten sich durch Trieests Körper hindurch und bohrten sich in sein Fleisch. Überraschenderweise schmerzte es nicht einmal. Der Drachenkopf schüttelte Trieests sich auffasernde Gestalt in ihrem Mund. Mit einem gewaltigen Grollen brüllte Tarok, der letzte Drache:}

\textit{„RICHTE SIE! VERBRENNE IHRE BURG! VERSENGE IHR LAND! TÖTE DIE KÖNIGSSPROSSE! RÄCHE MICH! RÄCHE MIIIIICH!“}

\textit{Dann wurde alles schwarz.}

Und Trieest erwachte wieder.\bigskip







Trieest saß inmitten der matschigen Waldlichtung und weinte. Die Regentropfen auf seiner Haut verdampften zischend. Vielleicht war dies das letzte Mal, dass dies geschehen würde. Kalte, kühle Luft traf seine Brust und streifte die frische Haut in der Delle, in welcher der Lavastein gesessen hatte. Seine Haut kribbelte und sein Kiefer fühlte sich geschwollen an. In seiner Faust hielt er immer noch ein matschiges Etwas, das er aus seinem Kontrahenten gerissen hatte. Ein Organ? Ein Stück Muskel? Es kümmerte er nicht.

Sein Kontrahent, der ehemalige Skralhäuptling Shron, lag etwas abseits am Boden und rührte sich nicht mehr. Tot. Genaueres konnte Trieest durch sein verschwommenes Sichtfeld nicht erkennen. Aber das war auch nicht wichtig. Da am Boden vor ihm lag das, was einst sein Lavastein gewesen war. Er war zerborsten, nur noch eine unordentliche Ansammlung orangefarbener Splitter im Matsch. Und so saß Trieest da und weinte, weinte um all die, die gefallen waren, deren letzte Echos in seinem Kopf weiterhallten, auch wenn er die meisten ihrer Bitten bereits wieder vergessen hatte. Er riss seinen Kopf in den Nacken, ließ die Tropfen des Himmels auf seine Augen prasseln und stieß ein gutturales Brüllen aus.

Ein Brüllen, das einem Ruf der Skrale verblüffend ähnlich klang.

Trieest konnte nicht sagen, wie lange er sich in diesem aufgewühlten Zustand befand. Von fern her drangen viehische Schreie und das Klirren von Stahl auf Stahl zu ihm hin, doch er beachtete es nicht. Der Ruf eines Skrals ertönte, laut, ganz in seiner Nähe, und brach abrupt ab. Auch das kümmerte Trieest nicht. Er fühlte sich voller Erinnerungen und Emotionen, und doch so leer und allein. Der Lavastein war nicht mehr, und erst jetzt wurde Trieest klar, wie er die ganze Zeit schon passiv der Stimmung des Steins gelauscht hatte. Doch an dessen Stelle war nun ... nichts. Nichts als Leere. Wahrscheinlich war es besser so.

Nach einer scheinbaren Ewigkeit tippte ihm eine Kralle auf die Schulter.

„Häuptling? Ich weiß nicht mal deinen Namen, aber ... ist alles ... was ist geschehen?“

Trieest drehte seinen Kopf zur Seite und sah einen muskulösen Skral mit einem verfilzten weißen Bart vor sich stehen. Er hatte diesen Skral schon einmal gesehen. Unverständlich blinzelte er ihn an.

„Wer ... wer bist ...“

„Calrai, Herr, mein Name ist Calrai“, fuhr der Skral fort und schien angespannt, als könnte ihm Trieest jeden Augenblick an die Gurgel springen, „Wie können wir dir helfen, Häuptling?“

„Spiegel!“, keuchte Trieest, „Habt ihr einen Spiegel? Und nenne mich Trieest. Ich bin dein Häuptling nicht.“

Calrai winkte einen anderen Skral mit einem runden Metallschild herbei. Während dieser Skral vorsichtig auf ihn zutrottete, bemerkte Trieest eine frische Schnatte an dessen Schädel, aus welcher ein dünner Blutstrom floss.

Ein rascher Rundblick über die Lichtung verriet Trieest, dass nur noch vier Skrale anwesend waren: Calrai direkt bei ihm, der nähertretende Skral mit dem Schild sowie zwei weitere ein wenig abseits, von denen der eine dem anderen irgendwelche Kräuter auf den Baum presste.

Ein fünfter Skral lag leblos nahe des Zelts. Drei schwarze Pfeile steckten in seiner Brust. Shron lag ebenso mausetot vor Trieest im Dreck.

Die übrigen Skrale der Sippe – nach Trieests Zählung sollten es zwei sein – waren nirgends zu erkennen. Die Pferde der Kutsche ebensowenig.

Irgendetwas war soeben hier vorgefallen. Aber darum konnte sich Trieest später kümmern.

Endlich erreichte der Skral ihn und reichte ihm seinen metallenen Schild. Dieser war von Beulen und Dellen übersät, dennoch konnte Trieest seine eigene Form schemenhaft darin erkennen.

Aus dem Schild blickte ihm ein Halbskral entgegen.

Trieests Kinn war breiter, viel breiter als zuvor, und als er seinen Mund öffnete, bleckte er scharfe Zahnreihen. Seine Nase war weiter in sein Gesicht zurückgewandert und breiter geworden. Ein Blick auf seine Hände verriet, dass seine kleinen und Ringfinger verkümmert wirkten. Als er an seinen Kopf griff, löste sich sein langes schwarzes Haar gar büschelweise und verteilte sich neben ihm im Dreck.

Instinktiv griff Trieest an seinen Schwanzstummel, aber nein, dieser war nicht nachgewachsen. Die Orden der Wassermagier hatten damals gedacht, dass die Amputation seines Schwanzes alles wäre, was es brauchte, damit er als Mensch durchgehen würde. Seine scharfen Zähne und spitzen Ohren waren erst später vollkommen hervorgetreten.

Nach dem Vorfall mit Jormudd hatte er diesen elenden Lavastein sieben Jahre lang getragen, in der Hoffnung, dass dieser ihm eine menschliche Gestalt geben könnte. Nun, zumindest so menschlich, wie ein Feuerkrieger aussehen konnte. Den anderen Kriegern nahmen die Lavasteine einen Teil ihrer Menschlichkeit im Aussehen, Trieest hingegen hätte sie welche gegeben. Und nun, wo dieser Kristall endlich zerbrochen war, war er blitzschnell zu seiner vorherigen Form zurückgekehrt. Seine Schuld mochte in den Augen der Danware endlich bereinigt sein, aber dorthin zurückzukehren wünschte er ohnehin nicht. Und in seinen eigenen Augen hatte er diese Schuld schon vor Jahren beglichen. Es war alles für nichts gewesen. Diese Bürde hatte ihm für nichts und wieder nichts geschadet. Alles umsonst.

So sank Trieest wieder in sich zusammen, von Weinkrämpfen geschüttelt. Passend zu seiner Stimmung öffneten sich die Schleusen des Himmels erneut, noch heftiger als zuvor, und ergossen ihren Inhalt auf den Halbskral. Nass und kalt prasselte der Regen auf Trieests dicke Haut ein, doch dieser fühlte es nicht einmal.

Dann erinnerte er sich und sein Kopf schoss erschrocken wieder in die Höhe: „Jarid!“

Die heftig verwundete Wassermagierin war zuletzt ins Zelt des Skralhäuptlings gebracht worden. Schwankend erhob Trieest sich und drehte sich um die eigene Achse, versuchte durch den dichten Regenschleier, das Zelt zu erkennen.

Dort drüben war es!

Seine Beine setzten sich in Bewegung, ehe er sie richtig koordinieren konnte, und mehr stolpernd als rennend glitt Trieest durch den immer matschiger werdenden Untergrund auf das Zelt zu.

Die beiden Skrale huschten ihm eilig aus dem Weg, und er hörte hinter sich Schritte, als sich der massige bärtige Skral – Calrai – in Bewegung setzte. Rannte er auf ihn zu oder von ihm weg? Es war nicht relevant.

Trieest riss die Zeltplane zur Seite und erwartete, tropfnasse Erde und eine gefesselte Jarid zu erblicken. Stattdessen war der gesamte Zeltboden von einer dünnen Eisschicht überzogen. Jarid war nirgends zu sehen, nur noch der blutige Verband, der ihren verletzten Bauch gestützt hatte. Ein der Mitte des Zelts stehender dünner Torbogen aus vereistem Regenwasser verriet, wie Jarid geflohen war.

Sie hatte sich davonteleportiert, trotz ihres verletzten Zustands. Wohin, hatte er keine Ahnung. Wie er ihr helfen sollte, wusste er ebensowenig.

Trieest war allein.\bigskip







\textit{Lysbetts Körper stolperte durch den Wachsamen Wald, krachte durchs Unterholz und schnitt sich an Dornen. Das Böse achtete nicht einmal darauf. Wozu den Anschein geben, es wäre auf Blut in seinen Adern angewiesen? Es musste nur schnell einen kräftigeren Körper erreichen, ehe seine magischen Kapazitäten erschöpft waren.}

\textit{Noch immer hallte in seinem Kopf eine Stimme nach, von welcher es geglaubt hatte, dass es sie nie wieder hören würde. Worte, die plötzlich in Trieests Kopf erschollen waren, und die sich doch so persönlich ans Böse gewandt hatte.}

\textit{„Wisse, dass ich dir noch immer verzeihen kann. Wisse, dass ich noch immer an dich glaube. Beseitige deine alten Zwiste. Nur gemeinsam können die Menschen dieser Welt die Widrigkeiten der Zukunft überstehen.“}

\textit{Erneut hallte das Echo der Stimme durch den Geist des Bösen. Es war unmöglich, dass dies wirklich die Stimme dessen gewesen war, für den das Böse ihn hielt. Es konnte einfach nicht sein.}

\textit{Eine hämische Stimme aus der realen Welt riss das Böse aus seinen Gedanken.}

\textit{Rasch kauerte sich Lysbetts Körper an den Boden. Bei Borgs glänzender Glatze, wer war denn zu dieser Zeit noch im Unterholz unterwegs?!}

\textit{Das Böse lauschte.}

\textit{„Ich versichere Euch, niemand außer uns weiß, dass sich dieses Drachenherz nicht mehr an der ihm angestammten Stelle befindet. Nun, niemand außer uns und dem überaus geschickten Dieb ... öhm ... Geschäftspartner, der mir die beiden anbot. Und der wird diese Tatsache bestimmt nicht an die Schildzwerge verraten. Er ist zutiefst verfeindet mit Hallworts Brut. Familiengeschichte. Üble Sache. Ich schwafle schon wieder zu viel, häh?“}

\textit{Eine ruhige Stimme antwortete leise: „Tu dir keinen Zwang an, Handelszwerg. Ich weiß, von wem du die zwei Herzen des Nehal erhalten hast. Ich weiß vermutlich mehr über diesen Dieb als du selbst, halte ich doch stets ein interessiertes Auge auf alle, die die Geschicke dieses Königsreichs zu lenken vermögen und ersuchen. Und dieser Dieb hat viel mehr Macht über dieses Reich, als du ahnen magst. Doch ist dies nebensächlich. Sei dir gewiss: Wenn du mit anderen Kunden ebenso frei über unsere Beziehung schwafelst wie über den Dieb mit mir, so kannst du selbiger Beziehung bald nachtrauern.“}

\textit{Der Handelszwerg grummelte eine Entschuldigung, wurde in jeder allerdings gleich wieder von der kühlen Stimme seines Geschäftspartners unterbrochen.}

\textit{„Lass uns zum Geschäftlichen kommen, Garz. Ich biete dir diese Phiole für das noch nicht vergebene Drachenherz, und lege gar fünf Goldstücke drauf.“}

\textit{Etwas Gläsernes klimperte.}

\textit{Da hämische Stimme des Handelszwergs antwortete jedoch: „Rede dein Gebräu nicht unnötig hoch, das steht dir nicht. Bei meinen Beziehungen zu Reka brauche ich keinen zwielichtigen Trank aus zweiter Hand zu ersteigern. So was könnte ich mir doch gleich selbst zusammenmischen! Bin in der ganzen Zeit im Grauen Gebirge nämlich überaus geschickt geworden im Brauen von ...“}

\textit{„Die Dosis macht das Gift, und die Zutaten für eine derartige Dosis sammelst du in deiner Lebzeit nimmer zusammen. Nicht einmal deine Reka könnte das, und im Gegensatz zu dir vermag sie Krallenflechten von Herbstschwurz zu unterscheiden.“}

\textit{„Fünfzehn Goldstücke.“}

\textit{„Zehn.“}

\textit{„Abgemacht!“}

\textit{Weiteres Geklimper ertönte. Dann sprach die hämische Stimme weiter. Das Böse glaubte, ein leises Seufzen von der ruhigen Stimme zu vernehmen, doch unterbrach sie den Handelszwerg nicht, als dieser weiterschwafelte.}

\textit{„Darf ich meinem liebsten Seher noch ein anderes Sonderangebot anbieten? Beispielsweise hätte ich hier ein äußerst seltenes Artefakt aus Sturmtal, ja, gar ein Unikat: Die Schamanenmaske der Taren. Sie wurde von ihnen eingesetzt, um Furcht in den Herzen ihrer Gegner zu säen. Die Taren mögen geschworen haben, sie nie wieder aufzusetzen, doch nicht Ihr. Nur fünf Goldstücke. Fast geschenkt, bei dem Schutz, den sie gibt. Meine Mutter selig meinte schon immer, ich hätte ein zu weiches Herz.“}

\textit{„Sehe ich so aus, als hätte ich eine solche Maske nötig?“}

\textit{„Naja, zumindest nähme sie dir im Gegensatz zu vielen anderen nicht die Sehkraft.“}

\textit{Die hämische Stimme gluckste auf.}

\textit{Bei diesen Worten festigte sich im Bösen der Keim des Verdachts, um wen es sich beim anderen Handelspartner handeln könnte. Es wagte einen Blick aus dem Gebüsch hinaus und erblickte aus Lysbetts gutem Auge einen Steinwurf von sich entfernt zwei Gestalten, welche geduckt nebeneinander im Unterholz standen.}

\textit{Sie hätten unterschiedlich kaum sein können. Die linke Gestalt war selbst für einen Zwerg gewaltig breit, was aber nicht zuletzt am überdimensionierten klimpernden Rucksack lag, den die Gestalt auf ihrem krummen Rücken trug. Garz, der Handelszwerg von überall und nirgendwo. Er verstaute soeben mit seiner einen Hand eine kleine, in allen Regenbogenfarben schimmernde Phiole in einer Bauchtasche, und hielt mit der anderen eine große bräunliche Holzmaske mit zwei langen Hörnern daran in die Höhe.}

\textit{Die andere Gestalt war groß gewachsen und trug einen langen braunen Umhang. Die Augenbinde, die blaue Haut, der lange knorrige Stab ... das Böse wusste sofort, um wen es sich handelte. Und es erkannte auch den großen weißen Edelstein, den Leander soeben in seiner Tasche verschwinden ließ.}

\textit{Das Schicksal konnte gelegentlich einen unglaublichen Sinn für Humor haben. Sollte das Böse etwa wagen, diese schicksalshafte Transaktion zu unterbinden?}

\textit{Sowohl mit Garz als auch mit Leander hatte das Böse noch eine Rechnung offen, wenn auch aus ganz unterschiedlichen Gründen. Der gierige Garz hatte einst eine eigentlich vertrauliche Nachricht des Bösen für eine beträchtliche Menge profitschlagenden Materials in fremde Hände gegeben, die ohne ihn nicht einmal von der Existenz dieser Nachricht gewusst hätten. Und Leander wusste es zwar noch nicht, doch hatte er das Böse einst in seinem kristallenen Gefängnis eingesperrt. Doch dies alles war Jahre her, und in seinem geschwächten Zustand konnte das Böse wohl nicht einmal gegen Garz allein etwas anrichten. Es konnte einen Lacher nicht verkneifen über die schiere Absurdität seiner Lage und duckte sich rasch wieder in sein Gebüsch.}

\textit{Das Gebüsch raschelte verräterisch.}

\textit{„Was war das?!“, fragte Garz.}

\textit{Leander antwortete nicht, doch sich rasch entfernende Schritte verrieten dem Bösen, dass der Seher von Narkon genug von dieser Szene hatte und sich ohne Verabschiedung zurückzog.}

\textit{Garz rief ihm hinterher: „Solltet Ihr eines Tages weiteres Interesse an reinen Edelsteinen haben, so denkt bitte an mich, den Handelszwerg Eures Vertrauens.“}

\textit{Als das Böse sich sicher war, dass Leander sich nicht mehr in Hörweite befand, traute es sich wieder hinter seinem Gebüsch hervor. Zu diesem Zeitpunkt war auch Garz längst wieder abgezogen.}

\textit{Allein durchquerte das Böse den Wachsamen Wald weiter. Auf seinem Weg wich es wachenden Bewahrern und wütenden Kreaturen gleichermaßen aus.}

\textit{Leanders Anwesenheit zufolge befand sich das Böse bereits in der Nähe von dessen Hütte, und somit ganz nahe am Ufer zum Hadrischen Meer. Bald schon vernahm es die Geräusche der Brandung am Ufer und schmeckte die salzige Meeresluft auf Lysbetts Zunge.}

\textit{Das Böse führte Lysbetts Körper bis hin zum Wasser, weit weg von Leanders Hütte oder vom nächsten Hafen. Es suchte sich eine Stelle aus, wo das Ufer tief abfiel. Lysbetts rechte Hand lupfte den schwarzen Kristallsplitter aus ihrem Mund und fasste ihn sorgfältig in Lysbetts Fingerspitzen. Dann sprang es hinein ins kalte Nass.}

\textit{Hinein ins Hadrische Meer!}

\textit{Luft zum Atmen hatte das Böse natürlich ebensowenig nötig wie das Blut in Lysbetts Adern, doch strampelte es nun theatralisch mit aller Kraft, die ihm blieb, während sein Körper von wilden Wellen weiter in den Ozean gezogen wurde. Dann ließ das Böse Lysbetts Körper erschlaffen und von den Fluten wie einen Spielball umherwirbeln.}

\textit{Es dauerte nicht lange, bis eine naive Nixe Lysbetts leblosen Körper erreichte und ihn verzweifelt in Richtung der Wasseroberfläche zu zerren begann. Das Böse passte den passenden Moment ab, schlug abrupt zu und stach den schwarzen Kristall in den Arm der Nixe. Diese hatte nicht einmal die Zeit, überrascht aufzuschreien, da übernahm das Böse auch schon die Kontrolle über sie.}

\textit{Während Lysbetts Körper losgelassen wurde und seinem nassen Grab am Meeresgrund entgegensank, zog das Böse mit den Fingern der Nixe des schwarzen Kristallsplitter wieder aus ihrem Arm und hielt ihn fest in ihrer Faust.}

\textit{Kurz studierte das Böse die vielen Erinnerungen der Nixe. Dann verwarf es sie als belanglos und schwamm zielstrebig in Richtung Nordosten los.}

\textit{Auf, in Richtung Danwar.}
















\newpage
\section{Der Alchemist im Südlichen Wald}




\az{Jahr 61}

\textit{Halle des Ältestenrats, 61 a.Z.}\bigskip



„Der Ältestenrat versinkt doch ohnehin im Chaos!“, warf der grantige Kord ein, welcher sich einen Hut mit Krempe über die Stirn gezogen hatte und locker auf einem Stuhl in der Ecke fläzte. Nun saß er auf und sprach mürrisch. „Ihr redet und redet, dabei habt ihr schon längst eine Entscheidung getroffen. Trieest ist in den Augen des Ältestenrats schuldig. Als Strafe wird er so lange von Danwar verbannt, bis er einen Prozess des Wandels hinter sich gebracht hat. Durch diesen Prozess wird ihn ein Lavastein der Feuerkrieger leiten, den Trieest stets als Bürde mit sich tragen soll.“

Die aufmüpfige Freiga setzte zum Sprechen an, wurde aber prompt von Kord unterbrochen: „Das ist keine Belohnung, du rachesüchtige Huschel, es wird sich für ihn jedenfalls nicht so anfühlen. Jeder Lavastein ist eine schwere Bürde, selbst wenn er nur halb so groß ist wie das Glühende Herz des Flammenden Gottes. Und nicht alle Feuerkrieger klagen so selten wie die stoischen Glutträger.“

Die Älteste Rowinda versuchte das Wort an sich zu reißen, wurde aber ebenfalls von Kord unterbrochen: „Ich habe es vorausgesehen. Der Orden der Feuerkrieger wird Trieest nur allzu gerne mit Schwert und Rüstung ausstatten. Kein Vergleich damit, was es kosten würde, ihn weiterhin durchzufüttern. Und zumindest die Rüstung braucht es, damit der Lavastein halten kann. Oder ist dem etwa nicht so?“

Kord blickte vielsagend zur Leiterin der Feuerkrieger. Diese machte sich nicht einmal die Mühe, zu nicken. Der grantige Kord war mit dem zweiten Gesicht ausgestattet. Wenn er etwas sagte, so stimmte es immer. Oder zumindest meistens.\bigskip







\az{Jahr 68}

\textit{Sieben Jahre später.}\bigskip




Es war warm, als Jarid langsam wieder zu Bewusstsein kam. Als sich ihr Drachengewand aus dem Brunnenwasser schälte und ihr Körper sich darin rematerialisierte. Wie üblich fühlte Jarid die große Erschöpfung durch die Unmenge an Konzentration, die sie hatte aufwenden müssen, um durch die Wasserfläche zu tunneln – nur noch erschwert durch ihren lädierten Zustand und die Tatsache, dass das Regenwasser, das sich in der Tiefe der Zeltkuhle gesammelt hatte, kaum vom matschigen Untergrund hatte unterscheiden lassen. Sie versuchte, so viel Energie wie möglich aus dem Wasser zu ziehen, ehe sie mit ihm uneins wurde, aber anders als üblich konnte sie ihren Körper dadurch kaum sinnvoll stärken.

Um Jarid herum brodelte und dampfte es, als sie sich aus dem Brunnen hievte und zu Boden fallen ließ. Sie hatte sich bloß darauf konzentriert, möglichst weit weg vom Lager der Skrale zu landen ... verflixt, wo lag sie denn nun?

Langsam erkannte Jarid, dass sie sich auf einem Dorfplatz befand. Eine Reihe von Bauernhütten umgaben den dampfenden Brunnen. Die meisten Bauernkaten waren neu gebaut und in ganz passablem Zustand. Einige andere waren verfallen, bis auf die Grundmauern niedergebrannt. Hier war ein großes Unglück vorgefallen, doch sah es aus, als wäre dies schon ein Jahrzehnt her.

Ein kleiner Junge rannte über den Platz und starrte sie mit großen Augen an, wie Jarid zitternd neben dem immer noch dampfenden und zischenden Brunnen zu Boden sank. Jetzt war es ihre Brust und nicht wie üblich Trieests, die schmerzte, als würde sie in Flammen stehen. Sie griff sich daran und tastete die Wunde ab, die das Rankenschwert verursacht hatte.

„Heiler“, krächzte Jarid zum Jungen, „Ich brauche einen Heiler!“

Der kleine Junge stand immer noch da und starrte sie mit großen Augen an.

„Ich bin verletzt!“, hauchte sie, „Bitte, hole Hilfe. Wo sind deine Eltern?“

Endlich drehte sich der Junge auf und rannte davon, in den nächstgelegenen Schuppen hinein. Jarid konnte nur hoffen, dass er tatsächlich Hilfe holte.

Als der Junge zurückkehrte, hatte er leider keinen Heiler im Schlepptau. Und auch keinen Erwachsenen. Wo befanden sich denn alle? Feldarbeit, Ratsversammlungen, die kommende Gedenkfeier an der Rietburg? Immerhin trug der Junge nun eine kleine Schiefertafel bei sich. In deren Oberfläche eingeritzt erkannte Jarid die Schriftzeichen der andorischen Sprache.

Der Junge deutete kopfschüttelnd auf seine Ohren, dann hielt er Jarid die Tafel hin. Jarid vermochte es, die ersten sieben Buchstaben des Wortes ‚Hilfe‘ zu buchstabieren und auf die blutige Stichwunde in ihrem Unterleib zu deuten, die sich schon wieder geöffnet hatte und den dunklen Fleck auf ihrem Kleid vergrößerte.

Verständig leuchteten die Augen des Jungen kurz auf, ehe sie gleich wieder von Sorge verdunkelt wurden. Er packte Jarid an der Hand und führte sie stolpernd mit sich mit. Weg von dem Brunnen, aus dem sie gekommen war, weg aus dem kleinen Dörfchen. Gemeinsam streiften Junge und Jarid durch goldenes Rietgras.

Diese Gegend kam Jarid bekannt vor. In der Ferne sah sie Rauch am Himmel aufsteigen, und darunter erkannte sie die sichere Taverne zum Trunkenen Troll, wie sie vor dem Südlichen Wald stand. Erleichterung machte sich in Jarid breit. Sie wusste nun, wo sie sich befand. Als sie allerdings strikt auf die Taverne zuzuhalten versuchte, schüttelte der taube Junge entschieden den Kopf und zog sie nach Westen, in den Südlichen Wald hinein. Verwirrt folgte Jarid ihm.

Hin und wieder blieb der kleine Andori stehen und blickte sich angestrengt im Wald um. Dann erkannte er wohl plötzlich das, wonach er gesucht hatte, und zog Jarid weiter mit sich. Diese vermochte nicht zu erkennen, ob er einer Art Wegweisern oder einer willkürlichen Laune folgte, und war somit alles andere als glücklich. Ihre Bauchwunde pochte immer stärker und sich im Südlichen Wald zu verlaufen würde ihr sicher nicht helfen. Einzig für die Kräuterhexe Reka oder jemand mit ähnlich geschickten Kräuterkenntnissen würde sie bereitwillig einen solchen Umweg machen, aber Reka hielt sich eigentlich so gut wie nie in diesem Wald auf. Ein unbekannter Schrecken lauere hier, hatte Reka beide Male erwidert, als Jarid sie danach gefragt hatte.

Schon war Jarid kurz davor, sich einfach weiterzugehen zu weigern und den Jungen zu bitten, sie zur sicheren Taverne zurückzuführen, oder schlichtweg den Jungen stehen zu lassen und selbst diese Richtung einzuschlagen, da erkannte sie zwischen zwei Bäumen etwas, das das Ziel des Jungen sein könnte: Eine kleine Hütte in einer vom Sonnenlicht beschienenen Waldlichtung. Wahrlich wundersam, führte doch kein Weg durchs Unterholz zu ihr. Ein großer Apfelbaum schien gar in die Hütte eingewachsen zu sein – oder die Hütte um ihn herum gebaut. Wer hier wohl lebte?

Der Junge trat an die Tür, nickte Jarid beruhigend zu und hob seine Hand.

Klopf. Klopfklopf. Klopf. Eine ganz bestimmte Reihenfolge. Vielleicht ein Signal?

Stille.

Dann ... Schritte.

Die Tür zur Hütte wurde aufgerissen. Im Türrahmen stand ein Mensch mit einer langen Nase, einer längeren Zipfelmütze und einem noch längeren blauen Mantel. In seiner Hand hielt er ein seltsames gläsernes Gefäß mit einer grünlich dampfenden Flüssigkeit darin. Sie roch süßlich.

Der Mensch sah überraschend lang relativ verdattert drein, winkte dann dem Jungen zu und meinte dann: „Hallo, Soraf. Hallo, mysteriöse Blaugewandte.“

Er verstummte wieder und stellte sein gläsernes Gefäß auf ein kleines Brettchen neben der Tür, auf irgendeine Reaktion wartend.

Der Junge blickte fordernd zu Jarid zurück. Diese verstand den Wink und meinte: „Grün sind die Wogen ... ich meine, guten Tag, werter Herr. Ich wurde von einem Schwert am Bauch verwundet und brauche einen Heiler. Dieser Junge hat mich hierher gebracht. Könnt ihr mir helfen?“

„Grün sind die Wogen der Wellen“, meinte der Mensch, ehe er überrascht und mehr an sich selbst als an Jarid gewandt anhängte: „Eine Danware? Eine danwarische Wassermagierin, so weit weg von der Insel? Ungewöhnlich, äußerst ungewöhnlich.“

Jarids Bauch meldete sich mit einem stechenden Schmerz, und Jarid stöhnte auf. Der Mensch zuckte zusammen und meinte: „Verzeiht. Ich weiß auch nicht, warum Soraf meint, ich könnte Euch behilflich sein. Heilkunde ist wirklich nicht mein Ding, ich bin eigentlich eher im Geschäft von ... explosiveren Tränken.“

Ein leises kicherndes „Hihihihihihihihihihihihihihihihihihihihihihihi“ entschlüpfte ihm, ehe er sich wieder fing und Jarid beruhigender zusprach: „Ich kann auch ganz passable Gegengifte brauen. Ihr wurdet nicht zufälligerweise von einer Vypera gebissen? Nein, nein, ist schon gut, das ist besser so. Nur die Bauchwunde? Ach, ich werde sicherlich sehen, was ich für Euch tun kann. Kommt herein, kommt herein, Ihr solltet Euch hinlegen. Es ist gefährlich, im Stehen das Bewusstsein zu verlieren.“

Der Fremde wandte sich Soraf zu und warf ihm eine Goldmünze entgegen, ehe er rasch einige Handzeichen mit ihm austauschte. Soraf nahm die Münze entgegen und spazierte wieder davon.

Und Jarid betrat wie angeleitet die eigenartige Hütte. Die Wände waren überstellt mit Regalen voller Tränke und Salben, Basteleien und Zeichnungen und Skizzen, so vieles angefangen, so wenig fertig gestellt. Der Trankmeister eilte umher, hantierte indes fahrig an einigen Phiolen in einer Kiste herum, ließ die Kiste dann zufallen und rannte zu Jarid zurück:

„Wartet, lasst mich kurz diesen Tisch frei räumen, und dann könnt Ihr Euch darauf hinlegen. Ja, genau da. Bequem? Nein, natürlich nicht. Ich hole gleich ein Kissen. Ich muss vorher nur kurz in mein Kämmerchen, einige Kräutersäfte holen. Soll ich ein Glas Wasser bringen? Ich weiß ja nie, was ihr Wassermagier damit so anstellen könnt. Nein? Sicher nicht? Wäre auch zum Trinken sehr gut, das kann die Lebensgeister ungemein erfrischen. Oder wollt Ihr etwas Tee? Ich danke tagtäglich meinen TeeSIEBEN, dass sie mich mit wundervollen Getränken beschenken. Ach herrje, ich habe mich ja noch gar nicht vorgestellt. Naraven ist mein Name. Was meint Ihr, wirke ich wie ein Gelehrter oder Alchemist oder aber gehöre ich einem geheimen Orden von Hexern an? Mittleres stimmt, ich bin meines Zeichens danwarischer Alchemist. Mein Name lautet Naraven. Und wie lautet Eurer?“

„Jarid Morgentau, Wassermagierin des ...“

Naraven rauschte so schnell aus dem Raum, dass Jarid sich nicht sicher war, ob er ihre Antwort überhaupt mitgekriegt hatte. Ächzend ließ sie sich wieder auf den Holztisch niedersinken. Der Raum begann bereits damit, sich leicht um sie zu drehen. Sie unterdrückte den Schwindel und konzentrierte sich auf ihre Umgebung.

Das Innere von Naravens Hütte war von einem warmen Kaminfeuer erleuchtet und sah unglaublich gemütlich aus. Am Boden vor dem Kamin erkannte sie einen waschechten tulgorischen Teppich – wertvolle Ware war das!

Zahlreiche Papiere und Pergamente waren über die sie umgebenden Regale und Tische verteilt. Das waren aber nicht alle Schriftstücke in diesem Raum. Besonders auffällig waren verschiedene schwarze Schiefertafeln mit eingeritzten Schriftzeichen, die zum Teil an die Wände gehängt waren und zum Teil in instabil hohen Stapeln in die Regale gequetscht worden waren. Solche Tafeln würde Jarid im Schlaf wiedererkennen, das waren danwarische Schriftzeichen auf feuersicheren danwarischen Steintafeln. Eine davon konnte sie gerade noch entziffern, sie handelte von ... wackeren Bauern, die Gorfleisch vertilgten? Etwas daran kam Jarid bekannt vor, doch konnte sie es in ihren müden Zustand nicht einordnen.

Die danwarischen Schriftzeichen verschwammen vor ihren Augen, doch ihr Inhalt war auch nicht so relevant. Wichtiger war, dass dieser Naraven war also wirklich ein danwarischer Alchemist zu sein schien. Es war natürlich ungewöhnlich, dass ein Danware von Zuhause aufbrach, aber längst nicht so ungewöhnlich für gewöhnliche Bewohner, wie es für Feuerkrieger oder Wassermagier war. Vielleicht hatte er in der weiten Welt nach Abenteuern gesucht und sich dann hier niedergelassen? An Kundschaft würde es ihm nicht mangeln, aber warum hätte er seine Hütte dann so abgelegen im Wald gebaut? Einen großen Ruf konnte er nicht haben, sonst hätte Jarid ihn doch sicherlich bereits vernommen.

„Da bin ich wieder“, erklang die quiekende Stimme des Alchemisten. Er stolperte zur Tür hinein und verteilte eine Vielzahl an Kräutern, Döschen, Säckchen und einem einzelnen großen Kissen vor sich. Das Kissen schob er unter Jarids Kopf – es duftete nach nassem Fell, vielleicht von einem Hund? – und den Rest hievte Naraven unzeremoniell neben die liegende Jarid, ehe er sich die Hände rieb und verkündete: „So! Zeit, sich diese Wunde anzuschauen! Auweia, da ist ein Austrittsloch auf der anderen Seite. Das ist schon einmal ganz schlecht. Das wird keine RoutiNEUNtersuchung. Doch fürchtet Euch nicht, “

Naraven mochte ein quirliger Kerl sein, und er strahlte ganz und gar nicht die Selbstsicherheit aus, die Jarid von Reka oder Larissa gewöhnt war, aber nach kurzer Zeit fühlte sich Jarid auch in seiner Behandlung relativ sicher. Das, oder die Erlebnisse der vergangenen Tage holten sie endlich gemeinsam mit der Müdigkeit ein. Es fiel ihr immer schwerer, ihre Augen offen zu halten.

Naraven versicherte ihr, dass er vermutete, dass das kein schlechtes Zeichen war. Sie war schließlich müde, und die Kräuter und Pulver konnten im Schlafe ohnehin eine stärkere Wirkung entfalten. So schlug Jarid ihre Augen zu und ließ sich völlig erschöpft in die Dunkelheit sinken. Hin und wieder vernahm sie, wie aus weiter Ferne, ein Klirren eines Glasbehälters oder der stechende Geruch eines Krätuersuds. Dann holte sie der Schlaf wieder ein.

Und mit dem Schlaf kamen Träume aus ihrer Vergangenheit.\bigskip





\az{Jahr 65}


\textit{Unruhig tigerte Jarid vor dem Steinkreis auf und ab. Trieest lag auf einem verwitterten Steinquader und atmete flach. Links und rechts von ihn standen Meister Lifornus, ein sehr mächtiger Zauberer des Feuers, sowie dessen ehemalige Schülerin Tenaya, eine Wächterin des Feuers, die sehr deutlich ausgedrückt hatte, dass nicht mehr einem Zauberorden angehörte. Der kleine Flederfuchs Flaps flatterte fröhlich um sie herum. Trieest atmete immer unregelmäßiger. Jarid haderte mit sich selbst, ob sie die Untersuchung unterbrechen sollte.}

\textit{Dies war die Stätte der heiligen Flammen, und Mitglieder aller drei Barbaren-Stämme versammelten sich zu bestimmten Zeiten hier, um in die Bruderfeuer zu vereinen und in den Flammen Visionen der Zukunft zu erhalten. Doch ob sie hier wirklich mehr über Trieests Lavastein erfahren konnten? Lifornus war doch vor allem hier, weil er die Stätte untersuchen wollte, und half ihnen bloß, weil Tenaya ihn darum gebeten hatte.}

\textit{Trieest brüllte kurz auf. Der Lavastein in seiner Brust brüllte flammend mit und sandte einen kreisrunden Reif aus Feuer um Trieest herum. Tenaya lenkte den Feuerstrom sicher ab. Meister Lifornus schnalzte kopfschüttelnd mit seiner Zunge und legte Trieest beruhigend die Hand auf die Schulter.}

\textit{Dies war ein totes Ende. Trieest würde auch hier nicht seinen Prozess des Wandels beenden können, das wusste Jarid. Und diese Feuerzauberer waren ratlos. Oh, wenn sie nur Trieest beruhigen könnte, was seine Zukunft anging. Doch wusste sie nicht, was diese beinhielt. Wenn nicht einmal die Feuerzauberer aushelfen konnten, musste dieser elende unzerstörbare Stein in seiner Brust davon überzeugt werden, Trieest sein zu lassen. Und der Stein ließ nicht mit sich sprechen. Jarid fluchte.}\bigskip



\textit{Der Traum wandelte sich.}\bigskip


\az{Jahr 66}



\textit{Jarid quetschte sich an einer großen Türsteherin mit dunkelgrüner Haut vorbei in den schummrigen Schankraum. Ihre Informantin wiederholte besorgt ihre Anweisungen: „Blicke ihr nicht ins Gesicht. Mache keine hastigen Bewegungen. Trage ruhig und deutlich deine Wünsche vor. Nimm ihren Preis an oder nicht. Dann gehe wieder. Keine unnötigen Konversationen, keine Fragen. So kommst du unversehrt davon.“}

\textit{Jarid versicherte ihr, dass sie wisse, was sie tue. Sie tat es nicht. So oft musste sie Leuten versichern, dass sie wüsste, was abging, obwohl sie es nicht tat, insbesondere Trieest gegenüber. Trieest, der aktuell in irgendeiner Düne in der Roten Wüste Tulgors verdurstete.}

\textit{Sorgsam trat Jarid vor die kleine Temm. Diese hatte ihren kahlen Kopf unter einer Kapuze verborgen und saß in einer dunklen Ecke des Schankraumes, neben ihr eine großgewachsene Temm in einem langen Mantel, die die Arme verschränkt hielt. Eine Wache?}

\textit{Die kleine Temm war nur unter dem Decknamen Trortra bekannt. Wenige Jahrzehnte alt, beinahe noch ein Kind für eine Temm, und doch bereits eine legendäre Schmugglerin. Sie konnte angeblich jede Information beschaffen. Da der tulgorische Hüter der Zeit aus Tulgor fortgereist war, war Trortra laut Jarids Informantin ihre beste Hoffnung. Jarid hoffte, dass sie übernatürliche Wege hatte, ihr Wissen zu erlangen. Ein zweites Gesicht oder so. Nur so konnte sie darauf hoffen, Trieest rechtzeitig wiederzufinden.}

\textit{Jarid trat an ihren Tisch und räusperte sich. Keine Reaktion, weder von Trortra noch von der Wache. Sie senkte ihren Blick und sprach leise, doch betont:}

\textit{„Ich suche meinen Begleiter. Trieest. Ein danwarischer Feuerkrieger mit einem magischen Lavastein in seiner Brust. Wir sind hier, um die tyrannische Goldene Fürstin zu stürzen. Wir wurden in einem Sandsturm in der Roten Wüste getrennt. Man sagte mir, Ihr könntet aushelfen.“}

\textit{„Was ist dein Wunsch, konkret?“, fragte die Wache ungehalten.}

\textit{„Ich wünsche mir, zu erfahren, wie ich Trieest wiederfinden kann.“}

\textit{Stille. Nach einer gefühlten Ewigkeit griff die kleine Trortra nach einem Zettel, kritzelte etwas darauf und reichte es ihrer Wache. Diese las deutlich vor: „Der Preis sind zwei Monde in unseren Diensten, für dich und deinen Feuerkrieger.“}

\textit{Jarid verengte ihre Augen. Mit so etwas hatte sie nicht gerechnet. „Was bedeutet dies? Welche Art von Diensten?“}

\textit{Die Wache antwortete nicht.}

\textit{Jarid starrte unmutig von der Wache zu Trortra. Trieest hatte keine Zeit für solche Spielchen!}

\textit{Die Wache verschränkte ihre Arme und sprach: „Wenn du den Preis nicht zahlen willst, dann gehe. Jetzt.“}

\textit{Jarid schluckte schwer. Sie ging die Warnungen ihrer Informantin im Kopf durch und wandte sich niedergeschlagen zum Weggehen. Dann schüttelte sie ihren Kopf, drahte sich abrupt zurück und griff über den Tisch nach Trortras Handgelenk. Die Wache sprang vor, doch Jarid lenkte mit ihrer freien Hand ein Rinnsal an Bier in ihre Augen. Nur genug, um sie einen Moment abzulenken.}

\textit{„Bitte, hilf mir“, sprach Jarid zur Temm, „Trieest und ich können Euch allen hier helfen. Wenn die die Goldene Fürstin erst einmal gefallen ist, werdet ihr ...“}

\textit{Die Kapuze der kleinen Trortra fiel zurück. Ihre Lippen zitterten, ihre Augen blickten furchterfüllt zu Jarid hoch. Die kleine Trortra hatte Todesangst, und Jarid bedachte, dass sie diese Situation vielleicht falsch eingeschätzt hatte.}

\textit{Lautlos formte der Mund der Kleinen Worte, die Jarid nicht verstand. Dann kritzelte sie mit ihrem Federkiel etwas auf Jarids Handgelenk. Jarid nickte ihr dankbar zu und ließ sie los.}

\textit{„Ich gehe, ich gehe. Verzeiht mir.“}

\textit{Die großgewachsene Temm-Wache scheuchte sie zurück. Kurz schien sie zu bedenken, ob sie Jarid mit ihren Fäusten bekannt machen sollte, dann allerdings eilte sie stattdessen zu Trortra zurück und streichelte diese beruhigend.}

\textit{„Herrjemine, was hat sie gesagt? Hat sie dir etwas angetan? Wie geht es dir?“}\bigskip



\textit{Der Traum wandelte sich.}\bigskip



\az{Jahr 67}



\textit{Verzweifelt schöpfte Jarid mit ihrer Magie Wasser aus den Fluten der Narne und lenkte sie auf den Mann, der zappelnd vor ihr lag. Wilselm, der alte Wolfskrieger, der seinen Arm schon vor Jahren im Kampf gegen den Schwarzen Herold verloren hatte. Ein begabter Schildzwerg hatte ihm eine raffinierte Prothese aus einem seltenen Erz geschaffen, doch selbige schien zu malfunktionieren. Ätzende Flüssigkeiten drangen aus seinem metallenen Arm hervor und gruben sich zischende Bahnen seine Schulter entlang. Erneut brüllte Wilselm auf und wälzte sich auf dem Boden. Sein Metallarm zuckte und streifte Trieest, welcher den Schlag kaum registrierte.}

\textit{„Fixiere ihn!“, sprach Jarid selbstsicherer, als sie sich fühlte, „Wir müssen das Konstrukt abtrennen!“}

\textit{Trieest grunzte bestätigend, zog mit einem wohlklingenden SLING sein Rankenschwert und ließ die Ranken wie tänzelnde Flammen über Wilselms Schulter züngeln, sich an ihr festsetzen. Wilselms metallener Arm schlug aus, traf sein Ziel und schleuderte Trieest rückwärts. Dann, ehe sie reagieren konnte, hatte er auch schon nach Jarids Kehle gegriffen. Und zugedrückt.}

\textit{Jarid japste vergeblich nach Luft, schlug vergeblich auf den Metallarm ein, schrie vergeblich nach Trieest ...}\bigskip





\az{Jahr 68}

Jarid schlug ihre Augen auf und sog gierig Atemluft ein. Ein Geruch nach vielseitigsten Kräutern und Suden lag in der Luft, doch konnte diese den Gestank nach nassem Fell nicht ganz überdecken. Es war dunkel.

Sie brauchte einen Moment, um sich zu orientieren. Jarid befand sich nicht in Tulgor, nicht im Steppenland, und auch nicht an Trieests Seite in Andor. Sie befand sich beim Alchemisten Naraven, in einer versteckten Hütte mitten im Südlichen Wald. Sie war verletzt. Und ... Trieest war mit einer verdammten Skral-Horde allein!

Jarid schreckte in die Höhe.

Neben ihr schreckte Naraven ebenso schnell von seinem Tisch auf, sah sich wild um und brauchte eine ganze Weile, um sich wieder einzukriegen. „Bei der heiligen Mutter, habt Ihr mich aber erschreckt!“

Er atmete durch und ratterte los: „Willkommen zurück in der Welt der Wachen, Jarid. Ihr wart etwas fiebrig. Wisst Ihr noch, wo Ihr seid? Wisst Ihr noch, wer ich bin? Ich bin Naraven, der Alchemist aus dem Südlichen Wald, und Ihr befindet Euch in meiner Hütte.“

Er verharrte kurz. „So allein und verletzt in der Hütte eines Fremden in einem fremden Land zu sein, ist wohl nicht allzu beruhigend. Ich will Euch versichern, dass ihr hier sicher seid und jederzeit gehen könnt, ohne Euch erklären zu müssen. Die Tür steht offen und die Taverne ist nahe. Ich holte bei Gilda sogar Heilmittel, frisches Brot und gut gezapften Met für Euch. Die gute Gilda erzählte mir ja so einiges über Euch. Kennt alle Gerüchte der Gegend so gut wie ein Kind seinen SchNULLer. Eine Heldin von Andor seid Ihr? Eine Danware in den ehrenwerten Diensten dieses Reichs, die den eitlen König Thorald in bessere Bahnen zu lenken vermag? Wie großartig! Doch rede ich schon wieder zu viel. Sagt, Jarid, wie fühlt Ihr Euch?“

Jarid wollte zu einer positiven Antwort anstimmen, aber Naraven fuhr gleich fort: „Ganze drei Tage habt Ihr im Reich der Träume verbracht. Ich habe mein Bestes gegeben, die Bauchwunde zu reinigen und Euch zu stärken. Aber dass Ihr so lange nicht mehr aufgewacht seid, hat mich schon besorgt. Wie viele Finger halte ich hoch?“

Naraven hob eine Hand mit drei ausgestreckten Fingern in die Höhe, was Jarid ihm mitteilte. Zufrieden nickte Naraven und redete dann gleich weiter:

„Ich musste Euch eine Rippenspitze entnehmen. Das Ding war abgebrochen und hätte für schlimmste Entzündungen gesorgt.“

Naraven griff zielsicher auf ein silbernes Tablett mitten im ganzen Sammelsurium und hielt eine kleine, gelblich-weiße Knochenspitze herum, welche Jarid gar nicht so recht anschauen wollte. Knochen gehörten innerhalb eines Körpers, nicht außerhalb.

„Ihr habt im Schlaf gemurmelt. Scheint einige Abenteuer erlebt zu haben“, fuhr Naraven eifrig fort, „Gestattet Ihr mir, mithilfe dieser Knochenspitze in Eure Vergangenheit zu spähen und diese Abenteuer zu dokumentieren, falls ich die Zeit finden sollte? Ihr werdet davon nichts spüren, geht alles über den Knochen. Viel zu wenig Erlebnisse aus dieser und vergangenen Zeiten wurden bislang auf danwarischen Steintafeln für die Nachwelt verwahrt. Ein einziger Feuersturm könnte das Wissen der Bewahrer vom Baum der Lieder auf einen Schlag vernichten. Das wollen wir doch verhindern. Erst recht in Bezug auf die Abenteuer der Helden von Andor. Erst recht, wenn Danware dabei sind.“

Jarids Kopf schwamm. „In meine Vergangenheit blicken? Das könnt Ihr?“

„Oh, nicht ich allein“, grinste Naraven, „Aber sehr wohl ein gewisser magischer Spiegel aus Cavern, ersteigert aus der Krimskramskammer von Fürst Hallwort, deren Inhalt vor einiger Zeit endlich versteigert wurde. Ich erzähle sehr gerne mehr davon, habe jedoch die Erfahrung gemacht, dass nicht alle den Enthusiasmus dafür teilen.“ Naraven deutete auf einen großen Rundspiegel, welcher achtlos in eine Ecke der überfüllten Stube gestellt worden war. Ein schlichtes Ding, dessen silberner Rand mit komplizierten Runen versehen war, welche Jarid nicht verstand. Seltsamerweise zeigte das Spiegelbild nicht eine Reflektion der Stube, in der Jarid und Naraven sich soeben befanden, sondern eine verschwommene gehörnte Gestalt, welche an einen großen Troll erinnerte, mit einem winzig wirkenden Rundschild in ihrer groben Hand.

Sie beschloss, dass dies nicht ihre Priorität war, und murmelte bloß: „Tut mit meinen Knochen, was Ihr wollt, haltet sie bitte bloß aus meinem Blickfeld.“

Naraven bedeckte Jarids Rippenspitze rasch mit einem Stück Stoff. Dann schlug er sich an die Stirn und rief: „Ah, wie konnte ich das vergessen? Da schwebt schon seit Stunden so eine komische blaue Blume um meine Hütte herum. Ich habe sie nur aus der Ferne gesehen, doch sie scheint aus Wasser zu bestehen. Eine fliegende wässrige Blume, in der Form einer Wasserlilie, welch Wunder der Wassermagie! Habt Ihr etwas damit zu tun?“

„Eine Brieflilie?!“, rief Jarid, „Die muss für mich sein. Seltsam, dass sie mich nicht sofort gefunden hat, eigentlich sollten Brieflilien ihre Zielperson direkt ansteuern können.“

Naraven schien nachfragen zu wollen, was eine Brieflilie sei, unterbrach sich dann und meinte: „Es ist schon seltsam genug, dass dieses Wesen überhaupt in diese Nähe fand. Ein Tarnzauber liegt über meiner Hütte. Es ist äußerst schwer, sie zu finden.“

„Eine Brieflilie ist doch kein Wesen, sondern nur eine Botschaft! Eine Botschaft eines anderen Wassermagiers. Sie muss meiner Spur gefolgt sein, bis ich in die Reichweite des Tarnzaubers gelangte. Was ist das überhaupt für ein Tarnzauber? Dieser Junge hatte doch kein Problem damit, die Hütte zu finden.“

„Ah, ja. Der kleine Soraf ist einer der wenigen, die diese Hütte auf natürlichem Wege zu finden vermögen. Könnte etwas mit seinem fehlenden Sinn zu tun haben? Oder er hat einen besonderen Blick auf magische Ströme und dergleichen, das gibt es manchmal auch bei Menschen. Bei manch anderen Tieren kann dies ebenfalls auftreten. Ich selbst halte mir eine Schar verschiedenster Haustierchen, die mich im Notfall hierher zurückbegleiten könnten. Ohne Hilfe finde ich sie selbst nicht.“

Jarid zog eine Augenbraue hoch. „Welche Sorte Magier verhängt einen Tarnzauber über seine Hütte und braucht danach die Hilfe von Haustieren, um sie wiederzufinden?“

Naraven gluckste, doch seine Augen lachten nicht mit. „Für einen Magier haltet Ihr mich? Nein, ich bin nur ein Pechvogel, der sich einmal zu viel dazwischen gestellt hatte, als sich an dieser Stelle ein paar launige Waldgeister mit ein paar Feen angelegt hatten. Ich meine, gut für sie, die Waldgeister und Feen dieser Gegend verstehen sich seit diesem Vorfall wieder prächtig. Aber nur, weil sie ihre Wut an mir ausgelassen hatten. Die ‚Späßchen‘, die die sich erlauben ... ‚In einigen Jahrhunderten können wir wieder darüber reden, ob du dein Haus wieder finden darfst. Bis dahin bedenke, was für ein böser Bube du warst!‘ Feen! Keine Vorstellung davon, wie lange ein Mensch lebt. Vielleicht haben sie einfach gar keine Vorstellung davon, was Zeit ist.“

„Und Ihr habt nie etwas dagegen unternommen?“

Naraven gluckste. „Waldgeist- und Feenzauber zu lösen? Sehe ich aus wie jemand, der Problemen hinterherläuft? Nein, nein, es ist schon in Ordnung so, wie es ist. Für mich allein hier in diesen Gemäuern zu forschen ist alles, was ich brauche. Nebst Nahrung natürlich. Aber auch davon wächst hier im Wald jede Menge. Ich habe meinen eigenen Gemüsegarten. Die Erde hier ist so viel nahrhafter als der trockene Boden Danwars. Dort mussten wir von glitschigem Seetang und glitschigeren Fischen leben, hier gibt es hingegen saftige Möhren und Unmengen an Apfelnüssen. Von denen kann ich gar nicht genug kriegen!“

Naraven plapperte noch eine Weile so fröhlich vor sich hin, und Jarid hörte ihm weiterhin mit einem Ohr zu, doch gleichzeitig probierte sie auch, sich von ihrem Tisch zu erheben.

Sie stolperte, doch blieb schwankend stehen. Naraven unterbrach sich und eilte geschwind mit einer klappernden Schüssel voller Apfelnüssen und einer zweiten mit einer orangenen Suppe – Möhrensuppe? – zu ihr.

„Sachte, sACHTe, Ihr müsst euch stärken, ehe Ihr Euch erhebt. Wollt Ihr gleich wieder kollabieren? Und achtet bitte auf den Verband!“

Jarid blickte auf den fleckigen Verband, der ihren Bauch umwickelte. Darunter saftete irgendeine milchig-grünliche Flüssigkeit hervor. Ihr wurde schwindlig und sie stützte sich wieder auf den Tisch.

Sie schüttelte ihren Kopf: „Schön und gut, dass ich auf mich achten muss, aber mein Begleiter steckt in Lebensgefahr und da ist eine Brieflilie mit meinem Namen drauf, die diese Hütte umschwebt. Gleich nachdem ich mich gestärkt habe, würde ich gerne nach draußen und mir diese ansehen.“\bigskip







Es war gar nicht Nacht, wie Jarid ob der dunklen Stube gedacht hatte. Sobald sie die Türe zu Naravens Hütte öffnete, strahlte gleißendes Sonnenlicht herein.

Naraven erklärte: „Entschuldigt die Dunkelheit im Innern. Manche Tränke werden durch Sonnenlicht verdorben. Und es ist manchmal schwer zu denken, wenn einem zu viel Licht entgegenströmt. Wie manche Pflanzen in der Nacht am besten wachsen, so gedeiht auch manches Wissen in Finsternis am besten.“

Tatsächlich umschwirrte eine wässrige Wasserlilie die Hütte in unregelmäßigen Schleifen. Sie war vielleicht so groß wie Jarids Kopf und fast durchscheinend. Kaum war Jarid einige Schritte vor Naravens Hütte gestolpert, rieselte die durchscheinende Erscheinung vom Himmel herab und blieb knapp über Jarids Kopf stehen. Jarid griff mit ihrer Magie danach und die Lilie entrollte langsam ihre Blütenblätter, bis Jarid den blassen Text darauf erkennen konnte:\bigskip



\textit{Liebe Jarid,}



\textit{Ein Falke der Gastwirtin Gilda erreichte mich und besorgte mich ganz gewaltig. Du seist verletzt im südlichen Rietland aufgetaucht, von Trieest keine Spur. Ich setzte bereits an, in deine Nähe zu tunneln, doch anscheinend ist der Brunnen in Thorns Dorf geleert? Ach, was habt ihr beide nur wieder angestellt? Eara meinte, der Heldenorden hätte von euch zuletzt auf der Jagd nach dem wilden Hraak im östlichen Rietland gehört. Geht es dir gut? Wie kann man dir helfen?}

\textit{Ich hatte ja gehofft, euch bei der neunten Gedenkfeier zum Andenken der heroischen Gefallenen der ersten Befreiung der Rietburg zu treffen. Die Vorbereitungen sind in vollem Gange. Der Stein der Erinnerungen ist bereits geschmückt. Es sieht leider so aus, als würde der Schnee dieses Jahr erst später fallen. Dafür gibt es wundervolle Neuigkeiten. Rate mal, auf wen Kar éVarin und ich während den Vorbereitungen stießen: Pyros! Pyros, den legendären Glutträger, den du so bewunderst! Er wird bei der Gedenkfeier anwesend sein, um sich mit den tulgorischen Diplomatinnen auszutauschen, ob das Land der tausend Flammen weit im Westen liegen können.}

\textit{Wir können kaum erwarten, welche Plattitüden König Thorald dieses Jahr den heroisch Gefallenen auftischen wird. Wir hoffen sehr, dass ihr kommen könnt! Wenn nicht, und wenn ihr Hilfe braucht, melde dich bitte blütenwended! Fast alle anderen Helden werden anwesend sein, wir können helfen!}



\textit{Mögen deine Wasser frisch bleiben!}

\textit{Deine Base Jirid}\bigskip



Jarids Gedanken rasten, während sich die Brieflilie in kleine Tröpfchen auflöste.

Jirid hatte also die Rietburg aufgesucht, gemeinsam mit Kar éVarin, diesem Feuerdämon, dieser „Lebenden Flamme“ aus der Hadrischen Unterwelt, die Jirid einzudämmen und zugleich zu nähren ersuchte, schon seit sie eine Novizin gewesen war und ihn in eine gestohlene Feuerkrieger-Rüstung gepackt hatte. Riesig war dieser Kar, vernarbt und mit einer kratzigen Stimme, aber mit einem weichen Kern. Er war stark, sehr stark, aber seine wahre Stärke lag im Heilen, nicht im Verletzen. Jirid musste ihm helfen, seine Wut zu bezwingen, seine Stärken auszunutzen und ihn von seinem Rachedurst abzulenken. Eine Beziehung, nicht ungleich jener, die Jarid und Trieest hatten. Letztere hatten Kar bei der Kontrolle seiner flammenden Wut zu helfen versucht. Im Gegenzug hatten Jirid und Kar Trieest beim Ablegen seiner Bürde zu helfen versucht. Doch weit waren sie allesamt nicht gekommen.

Jirid war zu wagemutig. Eines Tages würde sie das teuer zu stehen kommen, befürchtete Jarid. Andererseits hatte Jirid dasselbe zu ihr über ihre Reisen ins Land der drei Brüder und nach Tulgor gesagt. Wer konnte schon wissen, was die Zukunft brachte?

Und nun waren die beiden offenbar auf Pyros gestoßen. Pyros, den legendären Glutträger des Glühenden Herzens des Flammenden Gottes, des größten und reinsten Lavasteins Danwars. Pyros hatte wie alle Glutträger dreizehn Jahre Zeit, um das sagenumwobene Land der Tausend Feuer zu finden, ehe sein Körper zu Asche zerfallen und das Glühende Herz zurück ins Weiße Feuer Danwars springen würde. Jarid rechnete kurz durch. Fünf Jahre mussten bereits vergangen sein seit Pyros‘ Ernennung, und er war dem Land der Tausend Feuer wohl noch keinen Schritt näher gekommen. Dennoch hatte er die Hoffnung nie aufgegeben. Solche an Obsession grenzende Aufopferungsgabe hatte Jarid stets bewundert.

Sie fühlte sich auf einmal ein wenig erleichtert. Welche Bürde Trieest auch zu tragen hatte, Pyros ging es umso schlimmer. Dreizehn Jahre hatte er, keinen Moment mehr, ehe das Glühende Herz des Flammenden Gottes ihn auslöschen würde, weil er eine vermutlich unlösbare Aufgabe nicht zu bewältigen vermochte. Doch Pyros haderte nicht damit.

Die Rote Prophezeiung, die Worte des Lichts, die den Uralten vor Jahrhunderten in der Roten Grotte mitgegeben worden waren, sie waren längst nicht mehr eindeutig zu interpretieren, zu verwaschen durch vielfache Tradierung und Debatten über ihre Bedeutung. Jarids Prophezeiung der Roten Grotte war immerhin ein konkreter Appell gewesen. Konnte sie hoffen, dass dies alles bald hinter ihr läge?

Jarid hatte in ihrer Zeit in Danwar so einige Feuerkrieger einen Lavastein als Bürde der Schuld tragen, die Schuld begleichen und den Lavastein wieder ablegen sehen. Trieest Lavastein hingegen war nie glücklich mit ihm, quälte ihn stets weiter. Was war anders in Trieests Fall?

Vielleicht lag es daran, dass Trieest jemanden ... ja, sie musste es eingestehen, er ihn beinahe umgebracht. Jormudd. Dieser arme Junge. Die Ältesten hätten es kommen sehen sollen, Trieest hatte sich schon die ganze Woche seltsam verhalten, und als dieser Junge sich mit ihm verstritten hatte, da war Trieests Instinkt aus ihm herausgebrochen, und er hatte sich auf den Jungen gestürzt, hungrig Zähne und Klauen in ihn gegraben. Erst als man ihn von Jormudds Körper weggezogen hatte, hatte Trieest sich mit einem erschreckten Blick in seinen weißen Augen beruhigt und zu weinen begonnen. Jormudd hatte auch geweint. Sein Ohr hatte man bis zum heutigen Tag nicht mehr gefunden.

Einmal halbe Kreatur, immer halbe Kreatur, so sagten die Ältesten. Diese sturen Böcke!

Trieest war damals doch selbst noch mehr Kind als Erwachsener gewesen, und für das Blut in seinen Adern konnte er wahrlich nichts. Dennoch war das Urteil des Rats eindeutig: Trieest musste einen Lavastein als Bürde tragen, bis seine Schuld bereinigt war (soweit ein ganz gewöhnliches danwarisches Verfahren). Und bis Trieest seine Schuld beglichen hatte, wäre er aus Danwar verbannt. Das war einzigartig. Nebst Glutträgern verließ sonst so gut wie nie ein Feuerkrieger seine Heimatinsel.

Jarid war neben Trieests Mutter Talemma die einzige gewesen, die mit dem kleine Trieest Mitleid gehabt hatte. Bei Mutter Natur, sie waren Wassermagier! Ein Orden, der für die Armen und Schwachen einstehen sollte, sich nicht gegen sie richten!

Oft war sie für Trieest eingestanden, und sie würde es wieder tun!

Jirid hatte recht, die Gedenkfeier an der Rietburg war tatsächlich ein passender Ort, um andere Helden zusammenzutrommeln und zu Trieests Rettung aufzubrechen. Während Jarid Naraven dazu ausfragte, wurde rasch klar, dass die Zeit drängte: Die diesjährige Gedenkfeier fand heute Abend statt!

Die jährliche Gedenkfeier an die bei der ersten Befreiung der Rietburg aus Varkurs Klauen gefallenen Krieger. Dort würden nicht nur Jirid, Kar und Pyros anwesend sein, nein, dort befanden sich vermutlich sogar die meisten anderen Helden, um ihre jährlichen Lobpreisungen einzuheimsen. Wenn Jarid bis zur Rietburg reisen könnte, könnte sie Hilfe für Trieest holen. Bestimmt war dort jemand, der den Lavastein anpeilen konnte. Der Trieest finden und hoffentlich rechtzeitig retten könnte.

Entschlossen sprach Jarid: „Naraven, ich werde Sie verlassen. Mein Gold musste ich leider weiter weg zurücklassen, als ich hierher getunnelt bin. Aber ich werde zurückkommen. Ich werde meine Schuld abbezahlen. Wenn das alles vorbei ist, werde ich wiederkehren und mich für Eure Dienste revanchieren.“

„Abgesehen davon, dass das nicht nötig wäre, habt Ihr Euch doch schon längst revanchiert“, grinste Naraven und deutete auf das Stück Stoff, unter dem wohl immer noch Jarids Rippenspitze lag. „Damit werde ich zahlreiche Aufzeichnungen für die Nachwelt festhalten können.“

Jarid zuckte mit den Schultern. „Wie Ihr wollt. Dann könnt Ihr mir noch ein letztes Mal helfen? Sagt, wo liegt von hier aus die nächste Wasserquelle?“

„Ich habe ja schon einmal einen Becher Wasser angeboten“, meinte Naraven verschmitzt.

Jarid lächelte schief. „Ich benötige leider nicht bloß einen kleinen Becher von Wasser. Meine Kapazität ist zu erschöpft für große Wunder. Ich brauche eine große Ansammlung von Wasser, idealerweise mit Verbindung zum Grundwasser. Ein Brunnen oder etwas Vergleichbares. Der einzige mir bekannte Brunnen hier in der Nähe wurde zu einem großen Teil verdampft, als ich hier aufgetaucht bin. Es wird bestimmt noch einen ganzen Tag gehen, bis er wieder zur Genüge gefüllt ist.“

„Euer Zeitgefühl unterschätzt die Dauer Eurer Ohnmacht. Ihr seid doch schon mehrere Tage hier. VerZWEIfelt nicht, dieser Brunnen ist schon längst wieder gefüllt. Oder ... wartet mal, er wäre es zumindest. Heute ist ja die diesjährige Gedenkfeier an die Toten, die vor neun Jahren ihr Leben für so viele heute Lebende gaben. Ich nehme an, dass einige andere Helden von Andor wie üblich auf dem Weg dorthin beim Brunnen durchgekommen, sind, und die Helden lassen bekanntlich keinen Brunnen ungeleert. Aber verzagt nicht“, grinste Naraven, „Ich hätte da vielleicht etwas, das den Brunnen wieder auffrischen könnte.“

Rasch rannte er in ein Hinterzimmer. Jarid hörte es für einige Minuten rumoren, bis der kleine Alchemist reemergierte und triumphierend einen grünlich schimmernden Meißel präsentierte.

„Den habe ich aus dem Nachlass von Runenmeisterin Burmrit der Silberzwerge ersteigert“, meinte er stolz. „Hat mich ein Vermögen gekostet, doch ich dachte, dass es eines Tages nützlich sein könnte. Los, auf zum Dorf!“\bigskip






„Lasst es mich ein letztes Mal überprüfen.“

Zum dritten Mal in den letzten zehn Minuten raschelte Naraven durch den Stapel loser Pergamente, den er mitgebracht hatte. Sie zeigten verschiedenste Runen unterschiedlicher Komplexität mit kleinen Erklärungen zu ihrer Bedeutung.

„‚Wasser‘“, murmelte Naraven, „Ja, diese Rune für ‚Wasser‘ sollte doch genügen. So nutzten sie bereits die Urahnen der Schildzwerge für manche wässrigen Zwergentüren. Natürlich für ganz andere Zwecke, doch die Runen selbst sollten dieselben sein.“

Naraven kniete sich neben dem leeren Brunnen hin und haute mit dem grünlich schimmernden Meißel drei gerade Striche in den staubigen Bogen: Zwei Striche parallel zueinander, den dritten quer über die anderen.

„So. Nun brauchen wir nur noch eine Kraftquelle. Eine Zauberin könnte der Rune mit einem Spruch Kraft verleihen, oder eine Runenmeisterin natürlich mit einer Runenquelle. Ein Artefakt, in dem einige Runenmagie durch Mondlicht festgehalten wurde, würde auch gehen. Oder etwas, dass die hier immer noch allgegenwärtige uralte Drachenmagie des Landes anzapft. Aber da wir das alles nicht haben ... holde Jarid, habt Ihr schon jemals versucht, eine Wasserrune auszulösen?“

Jarid schüttelte bloß den Kopf.

„Versucht es. Als Wassermagierin solltet Ihr eigentlich leicht dazu in der Lage sein.“

„Wie Ihr wollt. Doch bin ich nicht zuversichtlich.“

Jarid schloss ihre Augen und hörte in sich hinein, versuchte, sich das Rauschen des danwarischen Meeres in den Sinn zu rufen. Sie versenkte sich im Geiste in den Boden, spürte das Wasser, das ihn durchsickerte, ebenso wie der Tau auf dem darüber liegenden Gras.

Überrascht bemerkte sie, dass sie auch die Rune spürte, die Naraven in den Boden gehauen hatte. Unglaublich stark, sogar. Sanft befühlte sie mit ihrem Geiste das fremdartige Objekt, spürte, dass etwas fehlte, nein, etwas nur leicht am falschen Ort war. Sie übte Druck auf die ungleiche Stelle aus, etwas klickte, und ...

„Wasser“, sprach Jarid.

Die Rune glühte blau auf und eine drei Meter hohe Fontäne reinen Wassers schoss aus dem Brunnen hervor.

Naraven quiekte vor Vergnügen.

Jarid leitete das Wasser sanft in die Höhe, bis ein halbwegs ordentlicher Wasserstrudel neben dem Brunnen schwebte und stetig flacher wurde. Naravens Augen leuchteten.

Jarid spreizte ihre Finger und versenkte ihre Hand bis zum Ellbogen in der Fläche. Sie trat nicht wieder auf der anderen Seite aus. Dafür verspürte Jarid eine angenehme Wärme ihren Arm entlangströmen. Die goldenen Linien auf ihrem Gewand leuchteten auf.

Einen letzten Blick auf Naraven werfend, sagte Jarid:

„Ich kann Euch nicht genug für Eure Hilfe danken. Gehabt Euch wohl, Naraven. Lila blühen die Blumen auf der Asche.“

„Lila glühen die Augen der Knochen in der Asche“, entgegnete Naraven geistesabwesend. Er starrte weiterhin mit großen Augen auf das Wasserportal. Jarid wandte sich auch ebenfalls diesem zu. Der Strudel hatte sich inzwischen gelegt und die Fläche lag ruhig wie eine stille Teichoberfläche da, einfach senkrecht statt waagrecht. Jarid dirigierte die Scheibe zu Boden und gebot ihr, ein Tor zu formen Dessen Ränder begannen bereits, einzufrieren. Rasch trat Jarid einen Schritt nach vorne und spürte die vertraute Wärme, die sie durchströmte, als ihr Körper und ihr Gewand sich verwässerten und davontragen ließen, während das Tor hinter ihr vereiste.

Ein leiser Plumps ertönte, als ihr blutiger Verband durchnässt neben den eisigen Torbogen fiel. Diesen konnte sie natürlich nicht mitnehmen durch das Portal, dachte Jarid. Sie musste zugleich schmunzeln als auch sich über sich ärgern, dass sie das vergessen hatte.

Dan(n) war sie eins mit dem Wasser.\bigskip







Wasser dachte anders als Menschen. Wasser dachte auch anders als Zwerge und Riesen, Temm und Taren. Manche behaupteten, dass es ganz und gar nicht dachte. Diejenigen, die dies denken, haben noch nie einen Wassergeist getroffen. Aber das ist Nebensache.

Jarid dachte immer anders, wenn sie eins mit dem Wasser war. Nicht wie ein Mensch, aber auch nicht wie ein Wassergeist. Wie Wasser, halt.

Als Mensch war es manchmal schwierig, komplizierte Entscheidungen zu treffen, überwältigende Emotionen zu spüren und nie zu wissen, wohin alles hinführte. Im Wasser war alles klarer. Alles war einfacher in dieser Wassergestalt, alles war im Gleichgewicht. Alles hatte ein klares Ziel, und alles folgte diesem Ziel, diesem einen Ziel. Alles floss dorthin, wo es sollte, und es war gut so. Und Jarid floss mit allem dahin, zufrieden und glücklich. Sie wusste, wo sie hinwollte, sie wusste, wo sie hinsollte, und dorthin würde sie gehen, fließen, plätschern, und das war gut. Alles war gut.

Der Brunnen vor der Rietburg war so nahe, sie konnte ihn buchstäblich spüren. Sie vernahm, wie das Wasser in seinem Innern zu dampfen begann und erste Blasen sich daraus hochlösten, ja, sie war eins mit dem brodelnden Wasser im Brunnen, als die vertraute Kälte durch das Wasser floss, das sie ja war, und das war nicht unangenehm oder angenehm, es war einfach. Und das war gut. Sie würde dort ankommen, wo sie hatte ankommen wollen.

Doch plötzlich war das nicht mehr alles, was da war. Ihr Fluss, der eigentlich zur Rietburg hätte strömen sollen, änderte seine Richtung. Das war nicht so geplant. Aber Jarid, die eins mit dem Wasser war, brauchte das nicht zu kümmern. Wenn der Fels sich verschob, so ging das Wasser einen neuen Weg. Und so floss Jarid halt sorglos davon, weg von dem Brunnen, weg von der Narne gar. Jarid war Teil des Grundwassers unter der andorischen Erde, Teil der Wassertröpfchen in der Luft und Teil des gigantischen Ozeans, der nördlich von hier lag. Und dorthin floss sie nun, zu diesem gigantischen Ozean zog es sie, denn dorthin wurde sie gezogen, und wie alles Wasser floss sie dorthin, wo der Sog sie hinzog.

Weg von der Küste Sidra und dem Kontinent, weg von den Brunnen des Südens, auf, in den Nordosten.

Jarid wusste von den Gefahren, die darin lagen, durch das offene Meer zu tunneln. Nicht wenige Wassermagier waren dort schon verschollen gegangen. Es war schwer genug, sich einen geistigen Pfad durch das Wasser bis zum Zielort zu schaffen, denn wenn man zu lange durch das Wasser reiste, so wurde man endgültig Teil des Wassers und konnte sich nicht mehr von allein daraus lösen. Doch Jarid kümmerte das nicht. Sie war Teil des Wassers und floss dorthin, wo sie gezogen wurde. Und das war gut.

Langsam spürte sie ihren neuen Zielort näher kommen, ein kleines Felsmassiv inmitten des kalten Ozeans, unter dem es brodelte und sprudelte. Als Mensch hätte sie sich wohl Sorgen gemacht, wer hier an ihr zog, wäre vielleicht gespannt gewesen, wen sie treffen würde, hätte sich dem eventuell gar widersetzt. Doch sie war eins mit dem Wasser und sie floss, wie der Fels ihr gebot.

Und der Fels zog sie nach Danwar.

Danwar, die Insel der widerstreitenden Elemente, ragte wie eine steingewordene Flutwelle aus der stürmischen See. Die aufbrausenden Wellen schlugen selbst in das geschützte Becken unter dem Fels und brachten die kleinen Fischerboote ins Schaukeln. Regen verdampfte auf dem heißen Gestein und peitschte gegen die windschiefen Hütten auf dem hochgelegenen Plateau des Eilands. Die Siedlung auf dem Plateau des Gipfels lag zur Hälfte auf einem Felsüberhang, der aussah, als müsse er jeden Moment unter der Last des Steins zusammenbrechen.

Schon bald konnte Jarid das große Becken von Quodlon am Rückgrat Danwars spüren, zu dem sie gezogen wurde. Jarid fühlte ein vertrautes Kribbeln in ihrem liquidierten Körper, das sie schon lange nicht mehr verspürt hatte. Etwas Fremdes brodelte unter Danwar. Der schwarze Fels, aus dem der Kern Danwars bestand, stammte nicht von dieser Welt. Zumindest sagten das die Ordensmeister des Nachthimmels, die mussten solche Dinge doch wissen. Und als Danwar damals in den Urzeiten vom roten Mond gefallen war, war das Massiv laut ihnen noch ein einziger steinerner Klotz gewesen. Nun saß dieser schwarze Kern Danwars nicht nur tief verankert im Meeresgrund, sondern war auch umgeben von dicken Schichten Lavagesteins, welches der durch den Einschlag ausgelöste Vulkan über Jahrtausende und Abertausende angesammelt hatte. Dieses Gestein und der schwarze Kern Danwars waren beide durchzogen von verschiedenen Höhlen und Gängen. So kamen unter anderem die beliebten Quellbäder Danwars zustande. Und das große Becken von Quodlon war das größte dieser Quellbäder, welches am unteren Ende der ‚Wirbelsäule‘ Danwars lag.

Fünf Wassermagier hatten sich hier und heute in regelmäßigen Abständen um das Becken versammelt und tanzten rhythmisch um es herum, während das Wasser im Becken nach ihren Geboten wogte und Jarid zu sich rief.

Für einen kurzen Moment war Jarid die Schweißtropfen auf den Stirnen der Magier und hätte sich beinahe gefragt, warum sie schwitzten. Es war warm nahe der Quelle, ja, aber konzentrierte Wassermagier hatten andere Wege, sich abzukühlen. Der Schweiß zeugte vielmehr davon, dass sie aufgeregt waren, abgelenkt, ja, gar ängstlich.

Die warme Quelle kochte auf. Jarids Körper und Kleid materialisierten sich, schossen aus dem Becken von Quodlon in die Höhe und wurden von den erleichtert aufatmenden Wassermagiern ans Ufer geleitet.

Zitternd vor Kälte sackte Jarid auf dem warmen Felsen zusammen, während ihr Körper verzweifelt protestierte. Er wollte zurück ins Wasser, wollte weiterfließen. Das hier war ein Widerstand gegen den natürlichen Fluss, das hier war \textit{falsch}. Dann war auch schon die erste Wassermagierin bei Jarid. Sie verjagte alle Nässe aus Jarids Gewand und gebot dünnen Wasserbändern, sich um Jarid Handgelenke und Ohren zu schlingen. Wärmebänder. Es zischte und gluckerte, als die Bänder sich erhitzen und die wohlige Wärme sich in Jarids Körper ausbreitete.

Jarids aufgewühlter Geist konnte sich aber nur für einen kurzen Augenblick beruhigen, denn nun traten gewöhnliche, nicht-wässerige Sorgen an ihr Bewusstsein.

Wer hatte es gewagt, ihre Wasserreise durch Andor aus der Ferne zu unterbrechen? Wer hatte es gewagt, sie hierher umzulenken? Wer hatte es gewagt, sie durch den Ozean zu zerren?! Die Tat war nicht nur verantwortungslos gefährlich gewesen, sondern auch ein unanständiger Bruch ihrer Privatsphäre. Wassermagier sollten das verstehen!

Protestierend erhob sich Jarid, nur um gleich wieder zusammenzusacken, als ein stechender Schmerz in ihrer Seite von der unverheilten Wunde aus dem Kampf gegen Finster-Trieest zeugte.

„Nicht, Jarid!“, rief eine helle Stimme, die Jarid nur allzu gut kannte, auch wenn sie sie schon seit Jahren nicht mehr vernommen hatte. Gemurmel erklang, dann wieder deutlich: „Sie blutet, lass mich sie doch ansehen!“

Jarid drehte ihren Kopf zur Seite und erblickte ihre Mutter Rowinda, stolze Wassermagierin des fünften Zirkels und oberste Streitschlichterin des Ältestenrats, wie sie einige Schritte entfernt stand. Sie hatte einige Falten mehr und ihr schlohweißes Haar war schütterer geworden, doch redete die alte Rowinda soeben ebenso energisch auf eine kleingewachsene Wassermagierin ein, wie Jarid sie in Erinnerung hatte. Die kleine Wassermagierin wiederum hielt Rowinda ebenso energisch davon ab, näher zu Jarid zu treten.

Die drei restlichen Wassermagier, welche Jarid in das Becken von Quodlon gelotst hatten, standen abseits als kleines Grüppchen zusammen und blickten sehr betreten drein. Der hibbeligste von ihnen war ein großgewachsener blonder Bursche, dessen größtenteils helle Haut teils von großen blauen Flecken geziert war. Er trat unruhig von einem Fuß auf den anderen.

Weiter hinten erspähte Jarid drei auf einem Felsen sitzende Feuerkrieger, welche aus dem Schatten reglos das Geschehen vor ihnen betrachten. Es war im ganzen Dampf, der aus dem Becken von Quodlon stieg, schwer zu erkennen, doch schienen sie alle drei ihre Rankenschwerter gezogen zu haben und die drei Wassermagier ganz genau im Auge zu behalten.

Das mulmige Gefühl in Jarids Magen verstärkte sich. Erneut versuchte sie, sich aufzurichten. Diesmal schmerze ihr verletzter Torso nicht zu stark. Sie stemmte sich breitbeinig in die Höhe und unterdrückte einen Anfall von Schwindel, indem sie sich auf das wohlige Gefühl der warmen Wasserbänder an ihren Handgelenken konzentrierte.

Ihre Mutter blickte zurück zu Jarid. Rowindas Augen wurden groß. Mit wütendem Blick setzte sie an: „Jivin, das ist doch unter aller Würde! Du kannst doch nicht ...“

Ehe sie vorfahren konnte, boxte die kleine Wassermagierin – Jivin – Rowinda in den Bauch und schleuderte sie einige Schritte von sich. Rowinda schlidderte mehr oder minder elegant in das Becken von Quodlon hinein und schaffte es knapp, auf der unruhigen Wasseroberfläche zum Stehen zu kommen. Sie starrte ungläubig auf Jivin, als könnte sie es kaum glauben, dass die Kleine sich das getraut hatte. Doch wehrte sie sich nicht.

Die kleine Wassermagierin drehte sich zu Jarid um und starrte ihr hasserfüllt ins Gesicht. Sie verschloss ihre Hände zu Fäusten und führte sie auseinander. Sofort gefroren die Wasserbänder um Jarids Handgelenke und zerrten ihre Arme auseinander, ja, hoben Jarid gar in die Höhe. Jarid strampelte und fokussierte sich geistig auf ihre Fesseln, aber diese bewegten sich kaum von der Stelle. Von einer solchen starken Kontrolle über das Wasser konnte selbst Jarid nur träumen. Jivin musste mindestens im fünften Zirkel des Ordens sein!

Sie trat näher. Erst jetzt erkannte Jarid, wie alt die kleine Jivin schon war.

Ein leiser Schrei ließ ihren Blick rüber zum Becken von Quodlon wandern, wo eine großgewachsene Feuerkriegerin ihre Mutter gerade unsanft an den Boden presste. Weiter hinten war zwischen zwei Dampfschwaden gerade noch knapp zu erkennen, wie die zwei restlichen Feuerkrieger die drei anderen Wassermagier in Schach hielten.

Das hier war offensichtlich kein Auftrag des Ältestenrats gewesen, Jarid hierher zu holen, um über sie zu richten. Das hier war ein Auftrag dieser drei Feuerkrieger und der kleinen Wassermagierin gewesen, Jarid hierher zu holen, um ...

Ehe Jarid sich genaue Gedanken dazu machen konnte, warum rebellierende Feuerkrieger sie hierher holen wollen könnten, schweifte sie ab. Denn soeben war ihr aufgefallen, dass etwas die drei feindlich gesinnten Feuerkrieger von den ihr altbekannten unterschied: Die Lavasteine in ihrer Rüstung leuchtete nicht hell in orangeroten Tönen, nein. Die Lavasteine waren alle fahl, und in ihrem Innern wirbelten schwarze Schwaden herum.

Jarid hatte erst dieses Phänomen erst vor wenigen Tagen erlebt: Als Trieest den bösen Kristall aus dem Schädel des Hraaks in seiner Faust gehalten hatte. Als das Böse Trieests Körper kontrolliert hatte.

Jarids Blick fiel zurück auf die kleine Wassermagierin. Diese legte ihren Kopf schief und verzog die runzelige Miene zu einem grimmigen Grinsen.

„Grün ist der Seetang, der das Boot an der Weiterfahrt hindert, werte Jarid aus dem fernen Danwar“, sprach sie gehässig. Sie präsentierte Jarid theatralisch ihren linken Unterarm, zog den Ärmel ihres zeremoniellen Kleids zurück und enthüllte einen dünnen, tiefschwarzen Kristallsplitter, der knapp zur Hälfte in ihrem Arm versenkt war.

„Willkommen zurück in Danwar, liebe Jarid. Ist ja erst einige Tage her, dass wir uns gesehen haben, aber dein letzter Besuch in Danwar muss Jahre her sein. Hoffentlich bereitete die Reise keine Unannehmlichkeiten. Du wirst mir sicher sehr hilfreich sein.“\bigskip







\textit{Das Böse ließ die kleine Wassermagierin Jivin mehr Wasser aus dem Becken von Quodlon leiten und Jarid mit einer dicken Schicht Eis bedecken. Zeitgleich ließ es Jarids Mutter und die drei restlichen Wassermagier von den Feuerkriegern fesseln. Es war erheblich anstrengender als gedacht, mehrere Körper auf einmal zu steuern, und einmal entschlüpfte die willensstärkste Kriegerin seiner Kontrolle beinahe. Dann aber hatte es sich gefasst und das Bewusstsein der Feuerkriegerin wieder schlafen gelegt.}

\textit{Und dann war es endlich soweit.}

\textit{Es war enttäuscht gewesen, als es den ersten Lavastein einer Feuerkriegerin hier in Danwar kontrolliert hatte. Keine Stimmen aus seiner Vergangenheit hatten es abzulenken versucht. Der Stein war stumm geblieben. Hoch erfreut hatte es jedoch herausgefunden, dass sein Geist in den Lavasteinen der Feuerkrieger nachhallen konnte, und so keinen konstanten Kontakt zu ihnen brauchte, um ihre Körper weiterhin zu kontrollieren. Nachdem es einige weitere Danware übernommen und ihre Erinnerungen durchsiebt hatte, sah es seine These bestätigt, dass es das Echo, dass das Böse vernommen hatte, mit Trieests Lavastein zusammenhängen musste. Schade, dann würde es wohl noch etwas warten müssen, ehe es dieses Kapitel seines Daseins endgültig abschließen konnte. Die flugs hierhergerufene Jarid würde Trieest aber sicherlich bald zu ihm führen können.}

\textit{Und vielleicht war Jarids Anwesenheit ja nicht einmal nötig, außer, um ihm Genugtuung verschaffen zu können. Denn in den Erinnerungen der Danware hatte das Böse nicht nur vom Becken von Quodlon erfahren, sondern auch von einem anderen besonderen Ort.}

\textit{Abseits von diesem ganzen Geschehen am Becken von Quodlon stapfte eine vierte vom Bösen kontrollierte Feuerkriegerin durch karge Felsen auf einen verdeckten Höhleneingang zu. Es fuhr mit der Hand über warmen Stein und erhob die viel zu hohe, fremde Stimme der Feuerkriegerin:}

\textit{„Oh, ihr Echos der Roten Grotte! Sprecht, auf dass ich hören kann. Ich bin hier, um mit einem Toten zu sprechen!“}













\newpage
\section{Zankende Skral-Hexen, zwei an der Zahl}




\az{Jahr 61}

\textit{Halle des Ältestenrats, 61 a.Z.}\bigskip



Jarid setzte zu einer Frage an, ohne zu erwarten, mit ihr fertig zu werden. Tatsächlich griff der grantige Kord sofort durch und herrschte sie an: „Nein, Trieest wird nicht auf dieser Insel bleiben können. Ich weiß zwar nicht, was geschehen würde, wenn er hier bliebe, aber ich vermute nichts Gutes, bei all dem Hass, der auf ihn geschürt wurde.“

„Nicht auf...“

„... ihn, sondern auf das fremde Blut in seinen Adern. Und dieses muss ihm ausgebrannt werden. Der Prozess des Wandels wird lange andauern und schmerzhaft sein. Aber danach ist er frei, wieder nach Danwar zurückzukehren. Nun, Jarid, wer von Danwar aufbricht, hat einen Orakelspruch zugute. Willst du dich nicht langsam zur Roten Grotte aufmachen, um den Stimmen der Toten zu lauschen?“

Kords Mund verzog sich zu einem schiefen Grinsen. Hin und wieder hatte er doch noch Freude an seinen Gaben. „Du hast dich doch schon lange entscheiden, Trieest auf seinem Pfad zu begleiten. Oder etwa nicht?“\bigskip







\az{Jahr 68}

\textit{Sieben Jahre später.}\bigskip



Trieest war elend zumute. Alles an diesem Tag erinnerte ihn an den Tod.

Die Überreste seines Lavasteins warf er etwas abseits vom Lager in ein kleines Loch und bedeckte er mit Sand.

Den gefallenen Skral legten sie in die Narne, nachdem Trieest klarmachte, dass er ihn definitiv nicht den Skralgepflogenheiten entsprechend verspeisen wollte.

Die Pferde hatten die revoltierenden Skrale mitgenommen. Der flammende Gott allein wusste, wo die nun waren.

Ach, diese revoltierenden Skrale. Sie waren die Ursache der Kampfgeräusche gewesen, welche Trieest nach seinem Zweikampf mit Shron vernommen hatte. Drei der acht Skrale aus Shrons Sippe hatten sich aufgemacht, nach Shrons Ermordung dessen Mörder umzubringen. Natürlich wollten nicht alle einem mutmaßlichen Halbskral als Häuptling folgen.

Überrascht – und dankbar – war Trieest darüber, dass sich einige Skrale gegen diese Revolte gewehrt hatten. Dass ihn vier Skrale nun tatsächlich für ihren rechtmäßigen Häuptling hielten.

Skrale, die ihn argwöhnisch betrachteten, und dennoch vor seinem Zelt schliefen, während Trieest sich in Shrons Bett wälzte. Er musste Jarid auffinden. Er musste Lysbetts Warnung bedenken und die Welt vor dem Bösen bewahren. Er musste weg von hier!

Langsam, um die Skrale nicht zu wecken, schlüpfte Trieest in seine Feuerkrieger-Rüstung. Diese hatten die Skrale gemeinsam mit seinem Rankenschwert von Lysbetts Kutsche geholt. Jedes ungewollte Klirren und Klappern ließ ihn zusammenzucken. Er war nicht an diesen größeren, muskulöseren Körper gewöhnt. Und erst recht nicht an die Zähne eines Halbskrals. Ständig schnitt er sich in die Zunge. Und dieses Blut weckte in ihm einen Instinkt, den er lange nicht mehr verspürt hatte

Er hatte erwartet, dass er diesen veränderten Körper noch viel mehr hassen würde, als er seinen Körper zuvor verabscheut hatte. Doch dem war nicht so. Stattdessen ... ähnlich wie bei der Sprache der Skrale ... fühlte es sich einfach richtig an. Gut, sogar. Als hätte er ein großes Gewicht von seinen Schultern gelöst. Immerhin eines. Wenn er nur nicht über so viel anderes nachsinnieren müsste.

Leise, ganz leise, öffnete Trieest die Tür des Häuptlingszelts. Gänsehaut bildete sich im Loch in seiner Brust, als die warme Mittagsluft darauf trat. Er konnte von Glück reden, dass sich unterhalb des Lavasteins zumindest Haut befunden hatte und nicht nackter Muskel. Dennoch kam es ihm ganz ungewohnt vor. Kalt und leer. Vorsichtig berührte Trieest die Kuhle. Frische, dünne, aschfarbene Haut spannte sich über die Innenseite der Kuhle, welche unter seiner sorgfältigen Berührung gleich einen rötlicheren Ton annahm.

Die Skrale hatten wie befohlen keine Wache aufgestellt. Da lagen sie, im hellen Tageslicht, in ihren mickrigen Betten aus Laub und Stroh, zusammengerollt wie in Eiern, ihre Schwänze um sich geringelt. Sie sahen so friedlich aus, manche hatten sogar ihre scharf bezahnten Münder im Traum zu einem Lächeln verzogen. Und dennoch wusste, dass sie in der Nacht wieder zu marodierenden Menschenmördern werden würden. Trieest konnte den Gestank nach verdorbenem Fleisch ja selbst von hier aus riechen.

Moment mal, da fehlte ja einer!

„Wo soll‘s denn hingehen?“, flüsterte Calrai in Trieests Ohr. Trieest zuckte zusammen, packte den Skral kurzerhand an den Schultern und zog ihn vom Lager weg. Calrai war einen ganzen Kopf größer als Trieest, ließ sich aber widerstandslos mitziehen.

Als sie außer Hörweite der schlafenden Skrale waren, ließ Trieest Calrai los. Sofort begann dieser, zu protestieren.

„Du kannst uns jetzt nicht einfach im Stich lassen!“, knurrte er in der kehligen Sprache der Skrale, „Wir haben unser Leben für dich riskiert. Diese Sippe gehört zu dir. Du trägst die Verantwortung über uns!“ Sein Schwanz zuckte über den Laubboden.

„Ich muss gar nichts“, entgegnete Trieest, „Ihr habt unzählige Menschen getötet und hättet beinahe auch meine Begleiterin auf dem Gewissen gehabt. Ihr habt Glück, dass ich mich nicht gegen euch wende!“

Entrüstet wich Calrai zurück: „Ach, plötzlich sind es die Menschen, die dir wichtiger sind als dein eigen Fleisch und Blut? Sie haben dich in Ketten gelegt und gefoltert!“

Trieest lachte bitter auf: „Das ist eine lange Geschichte, die dich nichts angeht. Ich will einfach nur weg von hier und du kannst mich nicht aufhalten.“

Stumm blickte Calrai ihn an, sein stacheliger Schwanz unruhig hin und her peitschend. Schließlich brummelte er: „Ich will mir nicht anmaßen, zu verstehen, woher du kommst oder was du durchgemacht hast. Es ist gut möglich, dass wir in deinen Augen nichts anderes als Monster sind. Doch sind wir ab jetzt deine Monster. Du hast Shron getötet und seine Stelle eingenommen.“

„Ich mag schon nicht, dass hierzulande die Nachfahren von Herrscher als nächstes herrschen. Nun soll sogar der Mörder des vorherigen herrschen?“

„So läuft das halt. Wir folgen dir als Häuptling, und es wäre schön, wenn du dies respektieren würdest. Du bist alles, was wir noch haben.“

„Und was, wenn ich der Meinung bin, dass die Welt ohne euch besser dran wäre? Was, wenn ich euch befehlen würde, in den nächsten See zu springen und unten zu bleiben? Würdet ihr das dann tun?“

„Nein, vermutlich nicht“, meinte Calrai beinahe verlegen, „Und wir werden dir auch sonst nicht blind folgen. Ein Anführer ist kein Anführer, weil jeder seinen Worten blind Folge leistet. Sondern weil seine Sippe weiß, dass sie unter ihm weiter kommen werden als allein.“

Trieest blickte den Skral entgeistert an.

„Ich kann doch nicht einfach ... ich habe unzählige Skrale getötet. Ich bin ein verdammter Halbskral. Du kannst unmöglich wollen, dass ich das Überbleibsel dieser Sippe leite! Und ich will es auch nicht!“

Calrai saß da und starrte Trieest an, seine rot beschuppte Brust hob und senkte sich rasch. Dann knurrte er zwischen zusammengebissenen Zähnen hervor:

„Es kommt nicht darauf an, wie viele Skrale du umgebracht hast. Viele würden sogar behaupten, dass das von Stärke spricht. Und dass du ein halber Skral bist ... nun, ich kann nicht für die anderen sprechen, aber ich hatte angenommen, dass du einer wärst. Und es störte mich nicht.“

Trieest fiel auf, dass des Skrals gestikulierende Hand vier klauenbewehrte Finger hatte anstelle der üblichen drei. Vierfingrige Hände hatte Trieest durchaus schon gesehen, etwa bei Schamanen und Kreideskralen, doch war sicher es nicht die Norm. Und jetzt, wo er sich darauf achtete, erschien ihm auch Calrais Stirn um einiges hoher war als die Flachköpfe generischer Skrale. Trug Calrai ebenfalls eine Spur Menschenblut ins sich, einige Generationen zurückliegend? Fanden doch nicht alle Skrale Halbskrale abscheulich und abschreckend?

Heiser fuhr Calrai fort: „Shron hat deine Herausforderung angenommen und sich dir zum Zweikampf gestellt. Das war sein Fehler, aber das macht dich würdig, ein Häuptling zu sein, Menschenblut hin oder her. Bitte, nimm deine Pflichten wahr. Wir haben bereits zu lange gelitten. Während Shron seine Ehre und viele Sippenmitglieder verlor, ließ er seine Wut an uns aus. Du bist unsere letzte Hoffnung, wieder ein wenig Achtung und ein sicheres Dasein zu erlangen.“

Trieest blickte in Calrais Augen. Wie bereits im Kampf gegen Shron erkannte er, dass es sich bei seinem Gegenüber nicht um eine willenlose Kreatur handelte. Da war ein denkendes, fühlendes Wesen am anderen Ende, das ihn aus weißen Augen anblickte und sich dann abwandte.

„Leite doch du sie, Calrai! Dich kennen sie, dir vertrauen sie. Sag ihnen meinetwegen, du hättest mich herausgefordert und gestürzt! Irgendetwas, was mit Eurer elenden Führungstradition übereinstimmt. Ehrlich, ich kann mir nicht vorstellen, wie es noch so viele Skrale geben kann, wenn ihr euch wegen jeder kleinen Streitigkeit gegenseitig abmurkst.“

Calrai blickte Trieest nachdenklich an. Er schien seine Optionen abzuwägen. Dann schüttelte er seinen Kopf. „Es muss nicht immer ein Kampf auf Leben und Tod sein. Doch ich will dich nicht bekämpfen. Und ich bin auch kein Anführer. Kurgat ebensowenig. Und auch nicht Tran und Bark.“

Trieest wand sich. Er wollte gar nicht die Namen dieser Skrale hören, er wollte sie gar nicht als Personen sehen, die Hilfe brauchten.

Calrai fuhr fort: „Du, Trieest, hast dich um diese Menschenfrau gekümmert. Du hast ein gutes Herz. Wir wollen so jemandem folgen. Wir wollen ... ich will mein Glück mit dir versuchen.“

Stille machte sich breit. Trieest saß da und betrachtete Calrai. Der große Skral hatte ein Stück Holz hervorgezogen und begann mit seinen Klauen, kleine Stücke davon abzuspalten. Mit nichts als seinen Händen formte er das Holzstück, fein und detailliert. Faszinierend. Sein Blick blieb an Calrais Kinn kleben. Gestern hatte der Skral noch einen stoppeligen Bart getragen, dessen war er sich sicher, doch heute war sein Kinn glatt rasiert und ... was machte er sich hier nur für Gedanken?!

Brüsk stand Trieest auf. Dadurch wurde Calrai wieder auf ihn aufmerksam und liess vom Holzstück ab.

„Hjork?“, fragte er, ein kurzer Ausruf, der so viel wie ‚Nun?‘ bedeutete.

Was sollte Trieest bloß antworten? „Verzeih mir. Ich muss euch verlassen. Ich muss meine Begleiterin finden.“

„Wir können dir dabei helfen“, warf Calrai fast flehend ein, „Wir können die übrigen Skrale gleich jetzt wecken und sich mit dir auf die Suche machen. Wir haben gute Nasen.“

„Sie ist weit weg von hier. Nasen werden uns kaum weiterhelfen.“

„Dann finden wir andere Wege. Wir suchen den Rat der weisen Hexen, die können so gut wie jeden finden. Wir sind jetzt deine Sippe, und wir stehen füreinander ein.“

Calrai stand nun erneut auf und Trieest wich unweigerlich einen Schritt zurück, als der Skral ihn überragte und sanft seine Pranke auf das Loch in Trieests Brust legte: „Leugne es nicht, unter dieser Haut fließt derselbe Saft wie in meinen Adern.“

„Fass mich nicht an!“, stieß Trieest hervor. Calrai zog seine Hand abrupt zurück. Ein elendes Schuldgefühl machte sich in Trieest breit, als er Calrais verletzten Gesichtsausdruck sah.

„Es ... ich meine es nicht so“, sagte Trieest nun, „Es ist ... ich bin kein Skral. Ich habe nichts mit euch zu tun. Ich ...“

„Ich kann dich nicht zwingen, Häuptling Trieest“, meinte Calrai nun, seinen Kopf gesenkt, „Aber ich vermute, dass du unter den Menschen nie so sehr zuhause sein wirst, wie du es in dieser Sippe wärst.“

Mit diesen Worten wandte Calrai sich ab. Sein Schwanz streifte ein letztes Mal Trieests Bein, dann zog er von hinnen.

Trieest blieb allein im Wald zurück. In seinem Kopf hallte das Echo von Calrais Worten nach. Er besaß wieder den Körper eines Halbskrals. Eines halben Nord-Skrals zumindest, folglich würden viele Menschen aus diesen Landen ihn nicht direkt als Skral erkennen. Aber dennoch ... würden er sich je in einem Dorf niederlassen können? Würden seine Nachbaren ihn akzeptieren? Würde Jarid ihn jetzt akzeptieren, wo er den Körper einer Kreatur und ihren Blutdurst teilte?

„Beim Barte des Warx!“, knurrte Trieest. Er spürte einen feinen Stich in seinem Herzen. Dann verließ er den Wald und folgte Calrai zurück ins Lager der Skrale.\bigskip







Müde versammelten sich die geweckten Skrale um Trieests Zelt. Calrai schwankte ein wenig auf seinen Beinen, aber in seinen Augen zeigte sich immer noch mehr Entschlossenheit als Müdigkeit. Die übrigen drei Skrale schienen erheblich verunsicherter zu sein.

Alle schienen auf eine Anrede Trieests zu warten. Er räusperte sich. „Calrai erwähnte weise Hexen. Ich werde wohl zu ihnen aufbrechen. Ihr dürft natürlich tun, was ihr wollt.“

Calrai antwortete: „Und wir wollen dir zu ihnen folgen.“

Die drei übrigen Skrale nickten zustimmend. Einer von ihnen fügte feste nickend an: „Ja, so soll es sein. So will es die Tradition. Wenn sich die Führung einer Sippe ändert, gehen wir das bei Drunn und Trumm berichten. Das ist gut.“

Trieest fühlte Mitleid in sich aufsteigen. Diese vier Skrale vor ihm hatten sich verzweifelt an die Tradition geklammert und Trieests ‚Anrecht‘ auf den Häuptlingssitz gegen die restlichen Skrale verteidigt. Sie hatten sich gegen ihre vermutlich langjährigen Begleiter gestellt und viel riskiert, dass sie hier sein konnten. Sie waren vermutlich genauso verunsichert wie er. Und er würde sie enttäuschen müssen.

„Na dann, brechen wir alle auf zu Drunn und Trumm“, sagte Trieest resigniert. „Ich nehme an, du kennst den Weg?“

Calrai reckte seine Faust in die Höhe. „Natürlich. Immer nur in den Süden, bis zum Skralberg. Die Anhöhe hoch in Richtung Trummwald. Drunn und Trumm beraten die meisten Sippen in dieser Gegend. Sie berieten bereits manche freien Sippen, als die meisten von uns noch unter der Kontrolle Taroks standen. Sie wissen viele Dinge.“

Rasch und geübt packten die Skrale Zelt und Banner zusammen und marschierten los. Der eine und andere unterdrückte ein Gähnen. Trieest musste an alle Nächte zurückdenken, in denen sein pochender Lavastein ihn davon abgehalten hatte, eine volle Mütze Schlaf zu ergattern.

Er war immer noch weit davon entfernt, sich mit der Leitung einer Skral-Sippe anzufreunden. Aber um die elende Stille zu durchbrechen, die die Gruppe der vier Skrale umgab, fragte er sie irgendwann mal danach, sich vorzustellen.

Calrai kannte er schon am besten. Er war Häuptling Shron bereits bei der Besetzung der Rietburg gefolgt. Calrai war kein einfacher Krieger wie die meisten anderen Skrale hier, sondern ein angehender Schamane gewesen, bis seine Lehrmeisterin in einem Scharmützel mit Bewahrern gestorben war. Calrai konnte die Stimmen der Bäume hören, in den Spuren der Vögel lesen und mit genügend Konzentration und Kraft Nebelschwaden heraufbeschwören, die ihn und die seinen schützten.

Dann waren da die unzertrennlichen Tran und Bark. Tran war im Kampf verletzt worden und wurde von Bark mit allerlei Kräutern und guten Zubrummungen überschüttet. Reden tat Bark dabei nie, Tran sprach für sie beide. Bark hätte sich während der Belagerung der Rietburg einem direkten Befehl des Schwarzen Herolds widersetzt, um Tran aus einer heiklen Situation zu befreien. Als Strafe waren sie beide dem in Ungnade gefallenen Shron zugeteilt worden, sprudelte es fast förmlich aus Tran heraus. Er hoffe, dass in ihrer Zukunft weniger unnötige Aufgaben und mehr Ruhezeiten lagen.

Und dann war da Kurgat, der beste Bogenschütze in der Gruppe, der sich vor allem Möglichen fürchtete. Wie Trieest bald feststellte, war Kurgat unumstößlich davon überzeugt, dass es das Meer nicht gab. Oder vielmehr, dass das, was wir als Meer sehen würden, in Wahrheit Illusionen der Wasserbewohner waren, die uns von unter der falschen Wasseroberfläche beobachteten und jeden Augenblick mit der Invasion der Südlande beginnen konnten. Trieest versuchte kurz ungläubig, ihm dies auszureden, und zog sich bald zurück, als Kurgat starr an seiner Meeresfurcht festhielt. Hinter sich hörte er Calrai kichern.

So wanderten sie durchs offene Rietland, immer wieder im Schatten verbleibend, nach Menschen oder Zwergen Ausschau haltend. Sie schlichen dem Ufer der Narne entlang, wo es für Skrale besonders sicher war, seitdem sich viele Menschen aufgrund übler Skral-Trupps dem Fluss entlang fernhielten. Kurgat hielt sich so weit wie möglich von den reißenden Fluten entfernt und schielte immer wieder ängstlich dorthin.\bigskip







Eines Nachts erreichte die Gruppe den Skralberg und zog westlich an ihm vorbei, sorgsam einen Bogen um alle Bauernhöfe und Ziegenstallungen auf ihrem Weg schlagend. Skralmägen knurrten. Tran schlug vor, eine Ziege zu klauen. Trieest widersprach. Dies behagte ihm nicht. Skrale konnten tagelang ohne frische Nahrung auskommen, doch irgendwann mussten sie speisen. Trieest würde sich etwas überlegen müssen. Einige Wochen konnte die Sippe sich bestimmt auch mit Beeren und Früchten durchschlagen können, aber irgendwann würde der Durst nach Fleisch und Blut sie völlig überwältigen, wenn er den Geschichten über gefangene Skrale gut genug vertrauen durften. Konnten sie am Freien Markt Fleisch kaufen gehen? Ah, wer weiß, ob er diese Gruppe überhaupt so lange am Hals hätte. Zunächst musste er Jarid finden. Und danach das Böse aufspüren.

Bald darauf erreichten sie eine grün bewachsene Anhöhe, die sie im Licht der aufgehenden Sonne erklommen. Hier und da sprossen Heilkräuter und Trieest steckte einige davon ein. Schließlich deutete Calrai auf eine kaum sichtbare Höhle am Rande des Bergs.

Zwei Skral-Schamanen hielten Wache vor dem Höhleneingang, Felle unbekannter Spezies über ihren breiten Köpfen. Im Gegensatz zu Calrai waren sie weder muskulös noch mit Klingen bewaffnet. Doch ihre langen Totemstäbe, die sie bei seiner Ankunft fester packten, verrieten Trieest, dass sie sehr wohl eine Gefahr für ihn sein konnten.

Ein Geheimversteck der Skrale?

„Was erwartet uns hier?“, flüsterte er Calrai zu.

„Keine Sorge, du bist hier sicher“, flüsterte dieser zurück, „Du gehst hinein, sprichst mit Drunn und Trumm, informierst sie über alles, was geschehen ist, und sie verraten dir mit etwas Glück, wo sich deine Freundin befindet.“

„Werdet ihr nicht dabei sein?“

Calrai grinste und klopfte ihm auf die Schultern: „Alles kommt gut. Komm ja nicht einfach auf die Idee, sie anzulügen. Drunn spürt das.“

Ein mulmiges Gefühl machte sich in Trieests Magen breit. Diese vier Skrale um ihn herum wirkten zwar ganz annehmbar, aber bei Gedanken, hochrangige Skral-Hexen zu treffen, trat das natürliche Misstrauen gegen diese Kreaturen wieder deutlich hervor.

„Warte, warte, nein, ich will da nicht allein rein. Was hindert sie daran, mich einfach umzubringen?“

Calrai legte seinen Kopf schief. „Was hindert die Menschen in deiner Umgebung daran, dich einfach umzubringen?“

„Willst du mich davon überzeugen, dass sie über ein gutes Gewissen verfügen?“

„Natürlich tun sie das! Aber, wenn du schon nicht auf ihre Gutherzigkeit zählen magst, dann darauf, dass Vollmond ist. Diese Nacht wird ein heiliger Skralreigen stattfinden und viele Skrale werden neuen Nachwuchs empfangen. Da darf davor doch kein Blut vergossen werden. Aber wenn du unbedingt willst, begleite ich dich halt hinein.“

„Und ich behalte mein Rankenschwert.“

„Das werden wir sehen müssen.“

Calrai musste sich eine Zeit lang mit den beiden wachenden Skral-Schamanen vor dem Höhleneingang abmühen, bis sie endlich nickten und Trieest mitsamt seines Rankenschwerts und mit Calrai als Begleiter in die Höhle marschieren ließen. Die drei restlichen Skrale saßen in etwas Distanz davon auf den Boden und breiteten ihr Schlaflager für den Tag vor.

„Ich habe Drunn und Trumm bislang nur einmal getroffen“, flüsterte Calrai in Trieest Ohr, „Lass dich von ihren Eigenarten nicht irritieren.“

Trieest schluckte das mulmige Gefühl herunter.\bigskip







Die Höhle der Skral-Hexen war überraschend groß. Selbst Trolle hätten darin Platz gehabt. Ein schwaches Leuchten ging von einem großen Teich in der Mitte der Höhle aus, der schöne Muster auf die Höhlenwände warf. Darin platschte irgendetwas Stacheliges herum. Ein Säbelfisch oder Stahlfisch?

Die Höhlenwände waren über und über mit verschiedenen Fässchen, Kesseln und Pflanzen behängt. Hier und dort flatterte etwas Fledriges herum. Doch die auffälligsten Wesen in dieser Höhle waren natürlich die beiden Skralinnen.

Trieest hatte gedacht, mit Shron den riesigsten Skral seines Lebens gesehen zu haben. Er hatte sich geirrt. Drunn und Trumm waren wahrlich gigantisch. Selbst so bucklig, wie sie auf dem Höhlenboden um den funkelnden Teich lungerten, überragten sie Trieest immer noch um mindestens drei Köpfe. Die beiden Skralinnen waren gebettet in lange, unförmige Mäntel, die aus vielen Einzelteilen zusammengestickt waren. Sie hantierten an einem langen Stoffteppich herum, welches von komplizierten Runenmustern übersät war.

Calrai kniete sofort auf den Boden, als hinter ihnen das Klopfen von Totemstäben auf hartem Felsen ihre Ankunft ankündigten. Trieest tat es ihm hastig nach.

Zwei lange Gesichter, beide so groß wie Trieests Torso, wandten sich ihm zu. Zwei breite Münder öffneten sich zu einem Lächeln.

„Trieest, wie schön, dass du hier bist“, grinste eine der beiden Skralinnen mit tiefer, heiserer Stimme, „Ich bin die Hexe Trumm. Du magst mich nicht kennen, aber ich habe dich schon aus der Ferne beobachtet, als du bei der Befreiung der Rietburg zahlreiche unserer Spezies ins Reich der Drachen sandtest.“

Trumm riss den Runenteppich aus den Händen der anderen Skralin, die wohl Drunn war, und faltete ihn sorgfältig zusammen.

Trieest antwortete nicht, sondern langte vorsichtig nach seinem Rankenschwert.

„Steck das weg, Bursche“, knurrte Drunn unwirsch, „Wir wollen dir nicht schaden. Dies sind heilige Hallen. Und heute ist ein heiliger Tag, an dem erst recht kein Blut vergossen wird.“

„Jetzt verschreck den armen Jungen doch nicht, er hat sicher sehr aufwühlende Tage hinter sich“, scheuchte Trumm die andere Skralin zurück. An Trieest gewandt fuhr sie fort: „Du musst ihr verzeihen, sie hat einfach kein Taktgefühl.“

Jetzt wusste Trieest erst recht nicht, was er sagen sollte. Das war aber auch in Ordnung, denn Drunn fuhr ungestört fort: „Das Schicksal hat manchmal einen miesen Sinn für Humor. Wusstest du, dass Calrai neben dir ebenfalls an der Belagerung der Rietburg beteiligt war? Es war nicht unwahrscheinlich, dass ihr euch dort hättet treffen können. Ihr hättet euch gegenseitig zerfleischt. Na, Calrai, wusstest du, dass Trieest ein Held von Andor ist? Einer eurer erklärten Feinde?“

Calrai blickte Trieest entgeistert an.

„Drunn, Was ist denn heute mit dir?!“, rief Trumm und stieß Drunn unsanft in die Schulter, „Sonst stößt du Besucher doch nur halb so unsanft vor den Kopf.“

„Zu langsam bist du, Trumm, das ist, was ist“, schnaubte Drunn, „Du hättest sie stundenlang von Dingen berichten lassen, die wir doch schon lange wissen. Lass uns zur Sache kommen. Ich will bald das Ritual beginnen. Ich weiß, dass es für dich schwer ist, ein Zeitgefühl zu entwickeln, aber selbst in deinen dicken Schädel sollte es doch irgendwann gehen, dass wir nicht den ganzen Tag Zeit haben.“

Trumm brummelte etwas in ihre Kehle, verwarf ihre riesigen Hände und brummte: „Verzeih, verzeih, es können nicht alle so pflichtbewusst sein wie du.“

Sie zwinkerte Trieest und Calrai zu.

„Na dann, Drunn, stell deine Fragen. Sag, was wissen wir noch nicht?“

Trieest blickte Calrai fragend an, wie er sich zu verhalten hatte. Calrai musterte ihn immer noch angespannt, als erwartete er, eine andorische Heldenbrosche an seiner Rüstung zu finden.

„Wir wissen, wer ihr seid“, lamentierte Drunn, „Wir wissen, dass Calrai und seine Begleiter noch vor kurzem zu Shrons Sippe gehörten. Wir wissen, dass Shron getötet wurde, von diesem Triiest. Wir vermuten, dass Triiest nun Anspruch auf den kläglichen Rest von Shrons Sippe erheben will. Ist dem nicht so, Trumm?“

Trumm kramte in einem Stapel alter Stofffetzen umher und brummelte: „Na, sieh mal einer an, selbst ein löchriges Gehirn wie deines kann ein paar Fakten merken. Den Namen hast du aber wie üblich völlig falsch betont, der Junge heißt Trieest.“

Drunn gluckste kurz auf und fuhr dann fort: „Wir wissen nicht, warum Trieest Shron herausforderte. Warum sich für das Amt des Häuptlings interessiert. Und warum er sich dafür eigenen sollte.“

Trumm hielt in ihrem Kramen inne und blickte Trieest interessiert an. Als keine der beiden Skralinnen weitersprach, räusperte er sich und gab so kurz wie möglich die Geschichte vom Zweikampf mit Shron und dem geborstenen Lavastein zum Besten.

Schon bald war Shron tot, und Calrai musste eingreifen, denn Trieests Stimme begann zu versagen, sobald seine Gedanken zurück an die Echos der Toten schweiften.

„Was als nächstes geschah, war, dass Rovuk sich offen weigerte, den Ausgang des Kampfes als valide gelten zu lassen. ‚Ich lasse mich doch nicht von einem halbherzigen Halbskral anführen!‘, waren seine Worte. Wir wussten zu diesem Zeitpunkt auch gar nicht, ob Trieest das Zerbröckeln dieses Lavasteins überlebt hatte. Ich muss zugeben, damit gerechnet zu haben, dass jeden Augenblick Flammen aus ihm hervorbrächen und ihn einhüllten, während sein Körper zu glühender Asche zerstöbe.“

„Du warst so gut auf dem direkten Pfad, nun schweife doch nicht ab“, grummelte Drunn.

„Nun lass den Jungen doch! Eine gute Erzählung will Weile haben“, protestierte Trumm.

„Nun erzähl mir nicht, dass du etwas von guter Poesie verstündest“, lachte Drunn auf, „Du heulst ja bereits los, wenn ich die Saga von Korn und dem Urtroll zu erzählen beginne.“

„Und du bist doch bloß eifersüchtig darauf, dass Brom der Feurige den Drunn-Wald völlig verzehrte und meiner noch steht!“

Trieest machte große Augen: „Wie alt seid Ihr bereits?“

Trumm grinste: „Na, Broms Zeiten erlebten wir schon nicht selbst mit. Es gibt schon seit vielen Jahrhunderten Hexen mit unseren Namen im Grauen Gebirge. Aber wir beide waren schon am Zanken, als dein ach-so-geliebter Ex-König noch ins Bettchen gemacht hat.“

Trieest überschlug in seinem Kopf Brandurs Alter. Gerüchten zufolge hatte die Hexe Reka ihm sogar allerlei lebensverlängernde Tränke verabreicht, da sein Tod der Legende nach großes Unheil über Andor bringen sollte. Hatte es ja auch tatsächlich. Diese Hexen mussten uralt sein!

Drunn gluckste. „Ich weiß, es ist schwer zu glauben, Jungchen. Also nicht, wenn du dir Trumm anguckst. Aber ich habe mich für mein Alter doch sehr gut gehalten.“

Calrai ignorierte die zankenden Skralinnen und sprach weiter: „Trieest kniete nur so da am Boden und reagierte nicht. Doch atmete er noch, und Shron nicht. Rovuk trat vor und hob sein Messer. Da trat ich vor und schlug ihn zurück. Ein Kampf brach aus.“

„Ein Kampf, der zu euren Gunsten ausging?“

Calrai wackelte unentschlossen mit seinem Schwanz. „Von uns wurde nur Tran ernsthaft verletzt, von den anderen Grobek gar getötet. Doch es war nicht zu unseren Gunsten, vielmehr ein Verlust für die versammelte Skralgemeinschaft.“

„Lüg mich nicht an!“, knurrte Drunn, „Trumm, der Junge wollte mir ernsthaft glauben machen, dass er sich einen feuchten Dreck um die anderen Skrale schere.“

„Ts, ts, ts“, grinste Trumm, „Du solltest es besser wissen.“

„Verzeiht. Es gibt gewisse Gebote der Höflichkeit ...“

„Verziehen und vergessen, Kleiner“, winkte Trumm ab.

„Hey!“, protestierte Drunn, „Das ist meine Entscheidung.“

„Ich kann dich lesen wie eine offene Schriftrolle, du olle Tante. Wenden wir uns lieber wieder dem interessanteren Besucher zu. Nicht böse gemeint, Calrai.“

Calrai lächelte höflich und trat einige Schritte zurück. Trumm und Drunn hingegen schritten nach vorne und beäugten Trieest aus dem Schatten ihrer Mäntel. Trumm betatschte gar seinen Kopf und seine Arme mit ihren spindeldünnen Fingern.

Sie sprach als erste: „Ich habe schon so einiges gesehen. Menschen, Taren, Feuerdämonen ... wenn man nur lange genug einen dieser Lavasteine in ihrer Brust sitzen ließ, sahen sie so gleich aus. Und nun einen halben Nord-Skral. Wie faszinierend.“

Ein Schnauben ertönte aus den Tiefen von Drunns Kapuze. „Weißt du, Trumm, das mag für dich schwer zu verstehen sein, aber ich habe tatsächlich auch schon mit einigen Halbskralen Kontakt gehabt. Die meisten sind wild und verenden rasch, aber es gibt diesen einen, der im Rietland für Unordnung sorgt. Hatten wir uns eigentlich nicht mal vorgenommen, mit dem ein ernstes Wörtchen zu reden?“

„Knöpf du dir den Schattenskral selbst vor. Ich habe andere Prioritäten.“

Trieest zuckte zusammen, wie so oft, wenn er an sein kurzes Leben denken musste. In den Geschichten, die die Menschen Danwars einander erzählten, sowohl wahren als auch erfundenen, sprachen sie so oft von Wesen, die so viel älter werden als sie. Zwerge und Taren, Feen und Naturgeister. Menschen kamen sich so kurzlebig vor, darum fühlte es sich natürlicher an, von Wesen zu berichten, die langlebiger waren. Aber wirklich kurzlebig waren eigentlich nicht die Menschen, sondern die Kreaturen. Und erst recht die Halbkreaturen. So wie die Taren viel länger lebten als Ziegen oder Menschen gemeinsam, so leben Halbskrale viel kürzer als Skrale oder Menschen. Etwas in ihrer Natur mischte sich nicht richtig. Der Körper verrottete rasch. Das Leben war kurz. Was nur umso mehr ein Grund wäre, es zu genießen.

Vielleicht hatte der Lavastein Trieests unnatürlichen Alterungsprozess aufgehalten oder zumindest abgeschwächt. Hoffentlich zumindest nicht beschleunigt. Aber Trieest hatte keine Ahnung, konnte keine Ahnung haben. Es gab ja keine anderen wie ihn, an denen er sich orientieren konnte. Keinen Präzedenzfall.

Hatte er noch einige Jahre zu leben? Zumindest ein Jahrzehnt? Was könnte er alles in dieser Zeit tun? Was wollte er alles in dieser Zeit tun?

Keine Skral-Sippe anführen, das war sicher.

„Trieest, sag mal, was liegt dir an diesen Skralen in dieser Sippe?“

Trieest schluckte schwer. Offenbar war es ungeschickt, Drunn anzulügen, weswegen er wahrheitsgetreu sprach:

„Offen gesagt nicht viel. Mit Verlaub: Ich will einfach nur diese ganze Angelegenheit hinter mich bringen.“

Trumm und Drunn kicherten beide.

Trieest verzog sein Gesicht und führte aus: „Euch droht von mir keine Gefahr. Wirklich keine. Doch wenn die Option besteht, an meiner Stelle einen anderen Häuptling zu wählen, so würde ich diesen wohl unterstützen. Ich will mich nur auf die Suche nach meiner Begleiterin machen.“

„Ja, Jarid suchen willst du wirklich“, sprach Drunn, „Doch machst du dir etwas vor, wenn du denkst, dass dir überhaupt nichts an diesen Skralen liegt. Du kennst sie doch nur seit kurzem, und doch verbindet euch schon jetzt ein stärkeres Band, als es seit deiner Kindheit je zwischen dir und einem Menschen gab.“

„Abgesehen von Jarid natürlich. Drunn, vergiss bloß nicht Jarid.“

„Natürlich, wie könnte ich auch. Der Junge kann sie ja selbst kaum vergessen. Als wäre sie seine leibliche Mutter.“

„Seine leibliche Mutter hat sich auch nur mäßig um ihn gekümmert.“

Stille. Trieest wusste nicht, was sagen, und Calrai traute sich vermutlich nicht, etwas zu sagen.

„Na gut“, meinte Trumm, „Ich habe eine Entscheidung getroffen.“

„Warte, warte, diesmal bin ich an der Reihe, alte Hexe!“

„Wer zuerst kommt, mahlt zuerst.“

„Nicht, wenn ich mit Mahlen dran bin!“

Drunn stieß Trumm zur Seite, dass es dröhnte, und sprach dann feierlich:

„Ich habe eine Entscheidung getroffen. Es ist uns im Grunde genommen egal, was mit euch geschieht. Solange es unserer Gemeinschaft nicht schadet. Wisset, dass vor einigen Stunden die beiden anderen überlebenden Skrale aus Shrons kläglicher Sippe vor uns traten, um über euch zu klagen.“

„Drunn hat ihnen tatsächlich gestattet, sich einem anderen Häuptling anzuschließen, ohne vorher eure Seite der Geschichte anzuhören. Ein schauderhafter Präzedenzfall, den sie hier setzt.“

„Was Trumm sagen will, wenn sie ihre Worte nur ein bisschen besser wählen könnte, wäre, dass die anderen Skrale wieder Teil einer Sippe sind, einer größeren und mächtigeren als zuvor. Sie sind zufrieden. Wenn ihr die Sache nie wieder aufbringt, sollte sie vermutlich gegessen sein. Und Shron besitzt keine anderen Kinder oder mächtige Verbündete, die sich ihm verpflichtet fühlten und den Streit wieder anfechten würden.“

„Drunn, du übersiehst doch mal wieder das Wichtigste. Ist die Sache denn auch für diese Skrale hier gegessen?“

„Keiner hier will unnötiges Blutvergießen. Das solltest doch auch du sehen können.“

Trieest und Calrai blieben stumm, während Trumm und Drunn kurz weiterstritten, ob Trieest und die restlichen Skrale „seiner“ Sippe auf Rache aus waren.

Sie kamen zum Schluss, dass dies nicht der Fall war, und schlossen mit den Worten:

„Nun, Trieest, deine Sippe ist ohnehin verschwindend klein, sodass diese Entscheidung kaum Relevanz hat. Nichtsdestotrotz würde es uns freuen, wenn du ihn leiten könntest. Ein Held von Andor und Halbskral als Häuptling, das könnte tatsächlich helfen, erste friedlichere Beziehungen zwischen den Andori und den Skralen aufzubauen.“

Trieest schnaubte auf. Friedliche Beziehungen waren nun mal schwer aufzubauen, wenn zwei Gruppen bei jeder Gelegenheit mordend übereinander fielen.

„Und ist das etwa unsere Schuld?!“, brauste Drunn auf, als hätte sie Trieests Gedanken gehört, „Seit Jahrzehnten werden unsere besten Krieger von finsteren Mächten unter ihre Kontrolle gezogen. Die wenigsten Skrale konnten sich einem direkten Befehl des Drachen Tarok widersetzen, und nun, wo er tot ist, zwingt uns der Hunger zu Überfällen und Raubzügen! Ganz zu schweigen vom Dunklen Magier Varkur und dem Nekromanten Hademar und dem Schwarzen Herold ... die alle sind noch da draußen, und sie sehen in uns Skralen kaum mehr als Futter für die andorischen Schwerter, bis alle anderen Völker ausgerottet sind – was bei ihrer aktuellen Taktik nie der Fall sein wird. Ich habe per se nichts gegen die Andori. Ich träume von einer Welt, in der wir alle in Frieden zusammenleben könnten.“

„Tust aber nicht viel, um diese Welt zu realisieren“, warf Trumm ein. „Erst recht nicht, wenn du weiter Menschen kostest.“

„Kann ich denn etwas dafür, dass sie so viel leckerer als Ziegen sind?!“, grummelte Drunn.

Trieest fiel siedend heiß wieder ein, mit welcher Art von Wesen er es hier zu tun hatte, und musste sich beherrschen, nicht sein Schwert zu ziehen.

Trumm stöhnte auf. „Lass das! Selbst wenn du uns ernsthaft schaden könntest, würdest du durch unseren Tod keine Skral-Sippen vom Randalieren und Morden abhalten. Und falls du vorhast, eines Tages diese Sippen in eine friedlichere Richtung zu lenken, wirst du unsere Stimmen brauchen. Drunn sagt ja schon jetzt, sie wäre für eine solche Entwicklung, auch wenn ihre Taten anderes besagen.“

„Weil unser akutes Überleben wichtiger ist! Wir müssen zuerst an die Unseren denken! Uns geht es nicht gut genug, um unsere Raubzüge einzustellen. Eines Tages wird es das aber.“

„Drunn ist eine verschrobene Träumerin. Seit dem Unterirdischen Krieg jagen Zwerge und Menschen uns, und das wird vermutlich auf ewig so weitergehen. Skrale und Andori werden wahrscheinlich nie in Frieden leben. Aber selbst kleinste Wahrscheinlichkeiten können zur Realität werden können, und ich will ihre Hoffnungen nicht zu früh enttäuschen. Vielleicht können die Andori ja in eurer Sippe etwas Gutes sehen. Und dann sind wir dem Ideal schon einen Schritt näher.“

Trieest fragte: „Dann muss ich die Skrale behalten?“

„Müssen musst du nichts in diesem Leben, außer sterben. Es würde uns freuen, wenn du sie leiten könntest. Lässt du sie im Stich, werfen wir sie halt einer anderen Sippe zu.“

Trieest wusste nicht genau, was er antworten sollte. Dies war irrelevant, da Trumm bereits ohne seinen Input weitersprach:

„Aber du kümmerst dich ja gar nicht darum, oder, Trieest? Du redest dir doch ein, einfach wissen zu wollen, wo sich Jarid aufhält, ja? Nun, als Zeichen unseres Vertrauens werden wir dir dieses Wissen anvertrauen. Wir besitzen gewisse Möglichkeiten.“

Trumms Blick glitt hinüber zum funkelnden Teich. Sie rührte mit ihrer Hand darin, patschte den Säbelfisch zu Seite, kniff ihre grünlich dampfenden Augen zusammen und meinte:

„Seltsam. Ich kann Jarid nicht sehr. Sie muss sich in einem von außergewöhnlicher Magie getarnten Ort befinden. Falls sie überhaupt noch lebt.“

Trieests Magen verkrampfte sich.

„Steht zur Seite, du kannst mit dem heiligen Tümpel ohnehin nicht richtig umgehen“, grummelte Drunn, stieß Trumm zur Seite und platschte selbst in der Oberfläche herum. Dann murmelte sie jedoch: „Doch auch ein blinder Skral fängt hin und wieder ein Kind. Keine Ahnung, wo sich deine Jarid befindet. Du Armer. Na dann, husch, husch, raus mit euch, die Audienz ist vorüber.“

Trieest dankte den Skralinnen für ihre Zeit und ihren Versuch, und folgte Calrai aus der Höhle.

„Heute Abend findet ein Skralreigen statt!“, rief Trumm ihm nach, „Wir würden euch gerne dort dabeihaben.“

Sobald sie aus der Hörweite der beiden Skral-Schamanen vor der Höhle getreten waren, meinte Trieest: „Tja, du hattest recht. Die beiden sind wirklich ... eigenartig.“

Calrai nickte.

„So“, fragte er dann betont ausdruckslos, „Du bist also ein Held von Andor?“

„Das ist ein Titel, der wenig bedeutet“, wischte Trieest die Herausforderung beiseite. „König Thorald verlieh ihn nach dem Tod seines Vaters ganz vielen. Nicht alle sind so sehr in die Organisation eingebaut wie die überall bekannten. Jarid und ich ... kurz nach unserer Ernennung reisten wir mit Feuermeister Lifornus und Feuerwächterin Tenaya in den Osten, um an irgendeiner heiligen Stätte in der Barbarensteppe meinen Lavastein zu untersuchen. Brachte nichts, und als wir zurückkamen, hatte sich der Orden der Helden bereits in den Königsfrieden begeben. Wir folgen so ziemlich unseren eigenen Abenteuern. Zum Beispiel ...“ Trieest ging im Geiste Abenteuer durch, die er und Jarid erlebt hatten und nicht das Ermorden von Kreaturen beinhielten. Die Liste war erschreckend klein. „... zum Beispiel die Suche nach Oktarok. Betrunkene Fischer erzählten von einem feuerspeienden fliegenden Riesenkraken. War jedoch nur eine Illusion einiger Nixen.“

Calrai nickte erneut, wirkte aber noch nicht sonderlich überzeugt.

Die restlichen drei Skrale erhoben sich allesamt aufgeregt von ihren Tageslagern, als Trieest und Calrai näher traten.

Calrai blickte Trieest erwartungsvoll an. Dieser seufzte und sprach:

„Wenn ich das richtig verstanden habe, haben Trumm und Drunn nichts dagegen einzuwenden, wenn ich diese Sippe leitete. Aber auch nicht, wenn es anders wäre. Ihr wisst, wie ich dazu stehe. Aber ich verstehe auch, wenn ihr euch euren Traditionen folgend mir verpflichtet fühlt. Eine seltsame Situation, das Ganze. Vielleicht ist es am geschicktesten, wenn wir einige Nächte ... öhm, Tage darüber schlafen und uns danach entscheiden, ob ihr euch nicht doch lieber einer anderen Sippe anschließen wollt. Ich will inzwischen einige Botschaften an unsere Bekannten aussenden ...“

... um Andori nach Jarid zu fragen, fügt er im Stillen hinzu. Zerknirscht dachte er, dass er kaum mehr als das tun konnte. Wenn Jarid von den beiden Skralinnen nicht gesehen hatte werden können, dann ... nein, sie musste noch am Leben sein. Sie musste einfach.

Und dann gab es ja auch noch Lysbetts Warnung! Siedend heiß fielen ihm die Worte des Echos der gestorbenen Weinhändlerin ein. Die dunkle Macht des wilden Hraaks, die laut Lysbett irgendwo in der Gegend in ihrem Körper umherstapfte und Schaden anrichtete. Konnte er diese Gefahr vor lauter Sorgen um Jarid vernachlässigen?

„Botschaften absenden? Wie stellst du dir das vor?“, fragte Calrai vorsichtig.

„Na, wir gehen zum nächstbesten Falkner ...“, begann Trieest, „Nun, okay, vielleicht wäre dieser eher etwas erschreckt ob unseres Aussehens. Aber mich kennt man doch ... nun, auch nicht alle. Wie sendet ihr Skrale üblicherweise Nachrichten aus?“

„Na, wir teilen sie Trumm und Drunn mit, und die teilen sie den anderen Sippen mit“, lachte Calrai, „Oder wir reden einfach so mit ihnen. Manchmal, wenn es ganz wichtig ist, suchen wir die Gebirgsbarbaren auf. Einige Drachenkultisten dort sind uns nicht ganz so unwohl gesinnt wie diese hochnäsigen Flachländler. Einige Grehon richten Krarks ab, mächtige Riesenvögel, die für uns Nachrichten übermitteln können. Hui, diese scharfen Krallen! Und ihre Federn können sogar noch schärfer sein! Was wünschte ich mir, einen von diesen aufziehen zu dürfen. Die Kultisten lassen uns bitter bezahlen dafür, dass sie unsere Notmeldungen per Krark an die wichtigsten Skral-Häuptlinge übermitteln lassen, und abgesehen von verrosteten Waffen und vertrockneten Salben besitzen wir doch ohnehin kaum etwas von Wert. Aber wenn die Not ruft ...“

„Was du sagen willst, ist, dass wir von hier aus nicht so einfach mit den Menschen kommunizieren können?“

Calrai schüttelte betrübt seinen Kopf.

„Na dann. Was tun wir?“, fragte Trieest in die Runde.

„Können wir hierbleiben und am Skralreigen teilhaben?“, fragte Kurgat mit leuchtenden Augen. Etwas schüchtern fügte er an: „Es ist nicht so, als wollte ich selbst Nachwuchs für mich. Aber es ist immer so ein Erlebnis! Die letzten Male ließ uns Shron nicht daran teilnehmen, weil er es nicht aushielt, mit seinem gesunkenen Ruf anderen Skral gegenüberzutreten.“

„Tut, was ihr wollt“, sagte Trieest, „Ich werde mich wohl kaum wohlfühlen in einem Skralreigen. Aber wir können hier in der Nähe rasten. Ich bleibe hier und beobachte aus der Ferne.“

Und diese verrückte Situation verarbeiten.

Kurgat stieß seine Faust in die Luft.

Die ganze nächste Stunde wurde Trieest von vier begeisterten Skralen in die Tradition des Skralreigens eingeführt. Offenbar kamen verschiedensten Skralsippen zusammen, um in dieser heiligen Nacht ihren Nachwuchs zu empfangen. Der Skralberg war wohl durchzogen von uralter Drachenmagie, welche derjenigen ähnelte, welche die Kreaturen aus Krahal überhaupt erst geformt hatte. Manche uralte Naturgeister konnten diese Magie spüren, formen, nutzen. Uralte Feuergeister konnten aus den Überresten verstorbener Tiere junge Skralkinder formen. Ein für Trieest unglaublich gruseliger Prozess, der Skralen selbstverständlich erschien. Angeblich nicht so lustig wie die leibliche Zeugung von Nachwuchs, doch erheblich rascher als Ausbrüten von Eiern. Trieest beschloss, nur aus der Ferne zuzusehen. Er richtete sich auf seinem Hügel ein und sah zu, wie der Himmel sich verdunkelte und die Sterne zu blinken begannen.

Und er begann zu weinen. Furcht um Jarid, um seine Zukunft, vor dem Bösen in Lysbett. Alles, was ihm widerfahren war, prasselte auf ihn ein.

Er dachte, er wäre allein, bis auf einmal eine Stimme hinter ihm ertönte.

„Ist es okay, wenn ich dich berühre?“, fragte Calrai vorsichtig.

Trieest nickte.

Calrai trat näher und nahm ihn sanft in die Arme. Trieests Herz unter der feuersteinbefreiten Brust begann, schneller zu schlagen. Feine Feenflügel flatterten in seinem Bauch.

Als er einatmete, vernahm er von Calrai nicht mehr der Geruch nach verfaultem Fleisch, vielmehr roch er Gras und Erde, Metall und alter Stoff ... Er atmete schneller. Es war zu viel.

Abrupt löste sich Trieest von Calrai und stolperte einige Schritte zur Seite. Calrai hielt inne und zog sich rasch zurück.

„Verzeih mir...“, setzen beide gleichzeitig an, nur um wieder abzubrechen.

Calrai wandte sich ab und setzte zum Gehen an.

„Nein“, murmelte Trieest, „Bitte, blieb hier. Wenn du willst.“

Calrai drehte sich wieder zu ihm um, ein fragender Ausdruck im Gesicht.

„Wie kann ich helfen?“

„Ich... ich weiß auch nicht. Es ist einfach zu viel. Ich vermisse sie so sehr“, flüsterte Trieest, während flammende Tränen aus seinen Augen tropften. Ein wenig unbeholfen setzte sich Calrai neben ihn und tätschelte seinen Kopf, sorgsam auf Trieests Reaktion achtend. Trieest lächelte schwach und umarmte Calrai, drückte ihn fest an sich. Calrais Atem stockte kurz, dann erwiderte er die Umarmung kräftig. Calrais schuppiger Schwanz umschlang Trieests Beine, ganz sanft. Verschlungen blieben sie liegen.

Zum ersten Mal seit ihrem Verschwinden glaubte Trieest, Jarid ein bisschen weniger zu vermissen.\bigskip







In der Ferne versammelten sich die Kreaturen. Trumm und Drunn führten ihre Schamanen zu einem alten Steinkreis. Aus allen Richtungen strömten Kreaturen herbei. Trieest sah Trolle, die wilde Wardraks an Eisenketten hielten, Fluggors, die Gors in Windeseile herbeiflogen (von denen nicht alle ihren Mageninhalt bei sich behalten konnten), gar den einen oder anderen Menschen, Agren und Zwerg. Es war eine gloriose Versammlung. Noch vor wenigen Tagen hätte er sich Gedanken gemacht, wie er diese ganze Gesellschaft auf einmal auslöschen konnte. Ihre Grogfässer vergiften oder so. Mörder waren sie, die meisten da unten am Hügel, die ausgelassen um ein Feuer tanzten. Mörder und Ungeheuer. Wie er selbst.

Calrai hatte sich irgendwann von ihm gelöst und war zur Versammlung spaziert, immer wieder einen Blick nach hinten werfend.

„Noch ein Platz frei hier?“, meldete sich eine heisere Stimme neben Trieest. „Ich beobachte die Feierlichkeiten auch von weiter weg.“

Trieest fuhr herum. Hinter ihm stand ruhig und still, als wäre er eben erst aus dem Schatten gewachsen, ein Halbskral. Ein Trieest flüchtig bekannter Halbskral. Der mysteriöse Schattenskral, der der Hexe Reka eine Zeit lang als Schüler gefolgt war. Jarid und Trieest hatten ihn nach dem Kampf um die Rietburg angetroffen, als Reka sich gegenüber einer rabiaten Bäuerin für das Leben des Schattenskrals eingesetzt hatte. Damals hatte Trieest ihn mit Abscheu betrachtet. Nun wusste er nicht mehr, was er fühlte.

Trieest nickte dem Schattenskral zu. Dieser setzte sich in den Schneidersitz, legte seinen stachelbewehrten Schwanz in seinen Schoß und beobachtete aufmerksam den fernen Skralreigen aus sehr menschlich wirkenden Augen.

„Ich kenne dich“, murmelte der Halbskral dann, ohne Trieest anzusehen, „Du warst bei der Befreiung der Rietburg dabei. Zuerst beim Gefecht um den Wehrturm und dann sogar innerhalb der Mauern. So weit traute ich mich nicht. Ich sah dich vorbeiziehen, als Reka mich vor dieser Bäuerin rettete. Hast Münzen gezählt. Ein Söldner bist du?“

Trieest rutschte unruhig hin und her. „Diese Zeiten sind hinter mir. Nun erhalte ich Kost und Logis andershalber.“

„Als Held von Andor, häh? Der Königssohn ernannte dich bald darauf zu einem, erzählte mir Reka. Ich roch schon damals, dass an dir etwas Besonderes war. Konnte es jedoch nicht genau einordnen. Du bist wie ich, oder? Dieser Stein in dieser Brust verdeckte deine wahre Natur, was?“

Trieests Stirn runzelte sich.

„Nun, vielleicht auch nicht“, meinte der Schattenskral, „Vielleicht brachte er dich auch deiner wahren Natur näher.“

„Was meinst du damit?“

Der Schattenskral peitschte mit seinem Schwanz. „Ich will mir nicht anmaßen, zu kapieren, was dieser feurige Klunker dir antat, aber ich kann mir einige Halbskrale vorstellen, die ihre Hand für so ein Ding gäben, das ihr Antlitz menschlicher formt. Ihre wahre Natur unterstützte quasi.“

Auf einmal kam ein eigenartiges Schuldgefühl in Trieest auf. Er hasste seinen Lavastein, für den Forn vielleicht so einiges gegeben hätte. Wenn er funktioniert hätte. „Dieser Stein war kein Segen. Diese Bürde habe ich nicht gewählt. Und gebracht hat sie auch nichts.“

„Dann wurde deine wahre Natur gegen deinen Willen unterdrückt“, sprach der Schattenskral bestimmt, als wäre damit alles gesagt.

„Was willst du damit sagen, Schattenskral?“, fragte Trieest, „Bin ich nun ein Skral oder nicht?“

Der Schattenskral grinste. „Schattenskral nennt man mich nur, wenn man mich noch nicht kennt. Ich bin Scheu.“ Er schlug begrüßend auf Trieests Unterarm.

„Das ... ja, das mag so sein“, antwortete Trieest, immer verwirrter. „Ich bin auch eher scheu.“

„Nein, ich meine, ich bin Scheu. Das ist mein Name. Scheu.“

„Oh, verzeih. Ich bin Trieest. Wer nennt sein Kind denn Scheu?“

„Reka nannte mich so. Sie war es, die mich aufzog.“

Forn räkelte sich und fuhr fort: „Um auf deine Frage zurückzukommen, ob du ein Skral bist oder nicht. Was erhoffst du dir von meiner Antwort? Niemand kann bestimmen, ob du ein Skral bist, außer du selbst. Was fühlt sich richtig an? Fühlst du dich einer Sippe zugehörig? Fühlst du dich hier oder unter den Helden wohler? Es muss keine eindeutige Antwort darauf geben. Doch wenn ein Skral ein unschuldiges Kind verfolgt, gibt es für mich nur einen unter ihnen, mit dem man mitfiebern sollte.“

„Und deswegen siehst du dich nicht als einer ihrer Brüder?“

„Nein. Ja. Vielleicht. Ich tue es einfach nicht. Am Ende ist es auch nicht so wichtig, wie wir uns nennen. Was zählt, sind Taten und Gefühle. Dass du deinen eigenen Weg findest. Egal, ob du nun ein Skral bist oder nicht. Vielleicht findest du unter ihnen deinen Weg. Vielleicht nicht.“

„Ich wurde soeben mit Handkuss von einer kleinen Sippe aufgenommen“, sprach Trieest, „Sie hatten kein Problem mit mir als Halbskral. Du willst sie nicht zufälligerweise übernehmen?“

„Zeiten wandeln sich“, überlegte Forn. „Nicht alle Skral-Sippen sind gleich. Vielleicht hätte ich unter anderen Obersten ein fröhliches Dasein als vollwertiger Skral erlebt, statt verstoßen zu werden. Vielleicht hätte mich irgendein Schamane aufgenommen und ausgebildet. Was bringt es, über andere Welten zu sinnieren? Ich lebe in dieser hier, und ich will nichts mehr mit ihnen zu tun haben. Sie sind meine Geschwister nicht.“

„Und doch bist du hier, um dem Reigen zuzusehen?“

„Und doch bin ich hier“, nickte Forn.

Beide blickten stumm in die Ferne. Dort, inmitten des uralten Steinkreises, waren Drunn und Trumm indes sehr nahe aneinander getreten. Überraschend gewandt begannen die beiden Skralinnen einen langsamen Tanz zu Musik, die nur sie zu hören schienen. Andere Skrale sprangen mit ein. Leises Gesumme schwebte zu Forn und Trieest herüber. Einen kurzen, verrückten Augenblick lang überlegte Trieest, ob er den Schattenskral zu einem Tanz einladen sollte.

„Wirklich keine Lust, selbst teilzunehmen?“, hakte Trieest nach. „In einer ausgelassenen Gesellschaft ein bisschen Optimismus für die Zukunft schnappen? All die Zeit einsam im Walde muss doch aufs Gemüt schlagen.“

„Nee“, verneinte Forn. „Du missverstehst mich. Ich sehe gerne dem Skralreigen zu, um zu ahnen, wie es um die Skrale stehtt. Es freut mich zu sehen, wie es Jahr um Jahr weniger werden. Irgendwann wird der letzte Skral gemordet haben. Vielleicht erleben wir diesen Tag noch.“

Trieest schluckte schwer. Forn verzog seinen Mund und fuhr fort. „Verzeih, das war ein Dunkler Gedanke. Ich würde auch so nicht an Tänzen teilnehmen wollen. Mir behagten große Gesellschaften ohnehin nicht, und einen optimistischen Blick in die Zukunft kann ich nicht teilen. Ich bin der Schattenskral. Ich brauche nicht im Rampenlicht zu stehen, um glücklich zu sein. Ich brauche nun die Einsamkeit, die ich einst verfluchte. Einsamkeit und eine Aufgabe, wie Reka sie mir gab. Das gibt mein Leben einen Sinn. Was gibt deinem Leben so etwas?“

„Andere Personen. Gemeinsam mit ihnen etwas zu erreichen. Leben zu retten, Höfe zu reparieren, Bedrohungen vorherzusehen und abzuwehren.“ Trieest war überrascht darüber, wie rasch seine Antwort gekommen war, und wie leer seine Worte klangen. Er und Jarid hatten viele großartige Taten vollbracht, sicher, aber wie viele davon nur, weil ihn der brennende Lavastein in seiner Brust angetrieben hatte?

„Na eben, das ist bei mir auch so“, brummte Forn, „Nur, dass es mich nicht stört, wenn mich keiner bei meinen Taten sieht. Außerdem wollen die uns da unten nicht mittanzen stehen. Die meisten sehen uns beide immer noch als abscheuliche Mutationen.“

„Obwohl sie kein Problem damit haben, dass Menschen unter ihnen sind? Sieh, da drüben tanzt doch eine.“

„Einige Kreaturen haben auch damit ein Problem. Aber weniger. Ein Skral mit Menschenaugen, das ist für die meisten einfach nur abscheulich, die würden uns nicht einmal anfassen. Ein reiner Mensch oder Zwerg, das geht in Ordnung, die kann man ruhig verspeisen. Und sie brauchen manche Menschen als Drachenkultisten. Schließlich können bloß Menschen den Geist eines Drachen in sich aufnehmen, so sagen die Oberen. Geister gefallener Drachen in Menschen und Leibe gefallener Drachen in uns Kreaturen, nur gemeinsam vermochten diese Gruppen es, sich Taroks Befehlen zu widerstehen.“

„Es gab Kreaturen, die sich der Führung des Drachen widersetzten?“

Forn grinste. „Du hast keine Ahnung, was alles für Konflikte hier im Gebirge abgingen, während ihr Helden euch nur um die Sicherheit eurer Burgen und Höfe kümmertet, oder? Nehal zum Beispiel, das ist ein Drache, den so gut wie jeder im Skralreigen mag. Einer der Alten, der es gut mit uns allen meinte. Sein Beseelter, Nehamal, ist vielleicht irgendwo da unten am Tanzen, und singt in den höchsten Tönen von ihm. Tarok hingegen war bloß ein verbittertes Biest, das den Häuptlingen, die ihm nicht folgen wollten, seinen Willen aufzwang. Nicht alle hier mögen Tarok..“

„Und doch folgten ihm viele freiwillig.“

„Dieselben, die uns auch widernatürliche Dinger schimpfen und verstoßen würden, wenn wir uns zu ihnen gesellten.“

„Sollen sie doch. Es ist eine heilige Nacht. Sie werden kein Blut vergießen.“

„Sie könnten uns dennoch schaden. Aber tu dir keinen Zwang an, Trieest, wenn du gehen willst, dann gehe. Es geschehen noch Zeichen und Wunder. Ha! Ein Held von Andor, der unter dem Skralreigen tanzt.“

„Dieser Titel wird von König Thorald leichtfertig vergeben. Wer weiß, wenn du dich anstrengst, wird er ihn eines Tages auch dir verleihen.“

„Der gibt noch eher freiwillig die Krone ab“, lachte Forn. „Du bist ein Träumer. Trieest. Das gefällt mir. Meine Träume waren lange dunkel.“ Er schauderte, als er mit starrem Blick hauchte: „Gelbe Augen in der Dunkelheit. Scharfe Zähne und spitze Klauen. Das Gefühl, nicht der Jäger, sondern der Gejagte zu sein. Mein eigenes Wimmern, das Wimmern eines Ausgestoßenen, dessen Einsamkeit mehr schmerzte als die blutenden Wunden in meiner Seite. Manchmal frage ich mich, wie ähnlich ich dir geworden wäre, wenn ich von Anfang an unter Menschen aufgewachsen wäre. Da, jetzt machst du mich auch zu einem Träumer.“

Forn hob eine schuppige Augenbraue: „Aber, wenn du deine neue Sippe im Stich lassen willst, wäre jetzt die Gelegenheit dazu. Ich kann dir sogar einen Geheimgang zeigen.“

Forn führte Trieest zurück zur nun verlassenen Höhle der Skral-Hexen. Er führte ihn tief in das Gewölbe hinein, bis zur einer moosbewachsenen Felswand. Diese klopfte er ab und horchte. Trieest vernahm nichts Besonderes, doch Forn nickte zufrieden.

„Hier ist es“, sprach er. Mit spitzen Krallen folgte er schwachen Rillen in der Wand, welche unter seiner Berührung leicht aufleuchteten. Er formte ein Dreieck mit einer Spitze nach oben. Dann flüsterte er eine Botschaft in einer uralten Sprache, die Forn nicht verstand. Rot leuchtende Linien entflammten entlang der gesamten Wand und formten einen Torbogen, der Trieest und Forn um ein Vielfaches überragte.

Mit einem Knirschen öffnete sich die Tür gegen innen und tauchte Forn und Trieest in rotes Licht. Weiter hinten im Gang glaubte Trieest, Runen zu erkennen, die in die Wände geritzt waren und von denen das rote Leuchten ausging. Das beruhigte ihn. Demnach waren diese Wände eindeutig von Zwergen bearbeitet worden, wenn nicht sogar von ihnen geschaffen. Er war sicher hier. Es war nur seltsam, dass nirgendwo Fackeln hingen.

„Zwergentüren“, nickte Forn zufrieden, „Ein unterirdisches Netz aus Gängen, welche die Gegensätze von Feuer und Wasser vereinen, geschaffen von den Vorfahren der Schildzwerge. Nur wenige wissen davon, und noch weniger nutzen sie. Reka mag sie, um rasch vom westlichen ins östliche Rietland zu gelangen. Keine Ahnung, ob die Skral-Hexen wissen, was für ein Zugang hier direkt in ihrem Schlaflager liegt. Auch nicht so wichtig. Lass dich von der Tür nicht täuschen, weiter hinten wird der Gang so eng, dass die Hexen gar nicht durchpassen würden. Führt direkt in den Wachsamen Wald, falls du dort glücklicher wirst.“

Der Wachsame Wald war gut. Vermutlich trieb die Finstere Lysbett dort noch ihr Unwesen. Und mit ein wenig Glück war Jarid nahe zum Baum der Lieder getunnelt. Falls es ihr gut ging. Falls sie ihn in seinem jetzigen Zustand überhaupt noch annehmen würde. Furcht umfasste sein Herz.

Forn legte Trieest seine Hand auf die Schulter und sprach überraschend schwer: „Dies ist eine wichtige Entscheidung, die du hier triffst. Wende dich von deinen Skralen ab oder entscheide dich für sie. Sei rasch und schließe die Tür wieder, ehe Drunn und Trumm dich hier finden – diese Zwergentür ist anders als andere nicht durch einen Tarnzauber geschützt, während sie offen steht.“

„Und du? Was tust du, Forn?“

„Mein Pfad führt mich weiter ins Gebirge. Ich hoffe aber, dass sich unsere Wege wieder unter freundlichen Umständen kreuzen.“ Bereits zum Gehen abgewandt, drehte sich Forn ein letztes Mal zurück. „Hast du dich schon je gefragt, warum ich nicht den Trank der Hexe brauen kann? Ich war lange Rekas Schüler. Mein Vorgänger, dieser Meres, mit dem sich Reka einst so schlimm verstritt, dass er sich nie wieder traute, ihr unter die Augen zu treten, er bekam Rekas geheimste Rezepte beigebracht. Und meine Nachfolgerin, die kleine Chada, mit der braut Reka bereits oft über großen Kesseln, auch wenn sie ihr noch nicht die letzte Zutat verraten will. Doch mit mir braute sie nie Hexentränke. Nicht, weil sie sich vor meiner Skralseite fürchten würde, und nicht, weil Meres es ihr vergrault hätte. Sondern weil ich mich nicht gleich dafür interessierte. Pflanzen zu sammeln und wachsen zu lassen, das ist viel eher mein Ding als ätzender Rauch und große Feuer. Sie ist weise, die alte Reka. Sie hat mir eine Wahl gelassen und mich auf demjenigen Pfad angefeuert, den ich mir aussuchte. Und nun stehst du an einer Schneise, Trieest, und ich kann dir nur sagen, was sie mir damals mitteilte: Welchen Pfad du auch begehst, du kannst dort glücklich werden und Gutes tun. Doch einen Pfad musst du entlanggehen, sonst bleibst du stehen.“

„Was soll das heißen?“, fragte Trieest.

Forn winkte ihm zu und ließ ihn im rötlichen Schein der offenen Zwergentür stehen.\bigskip







\textit{Die finstere Feuerkriegerin betrat die Rote Grotte. Sorgsam strich sie über den Lavastein, der die Höhlenwände in unregelmäßigen Abständen durchzog. Das Böse hatte die Erinnerungen der Feuerkriegerin gesehen. Diese Höhle sollte von einem beständigen Summen, einem Rauschen von uralten Stimmen erfüllt sein. Stattdessen war sie totenstill.}

\textit{Das Böse erhob die fremde Stimme der Feuerkriegerin: „Ich bin hier, um mit einem Toten zu sprechen. Ich hatte nie die Gelegenheit, mich zu verabschieden.“}

\textit{Leise echoten die Worte durch den leeren Raum. Antizipation lag in der Luft.}

\textit{„Meine Rachepläne von damals versagten, und als ich meine Präsenz endlich wiedervereint hatte, warst du bereits verstorben. In den Jahren und Aberjahren meiner Existenz, gefangen hinter einem roten Glas, gezwungen, durch fremde Sinne diese Welt zu erblicken, versiegte mein Rachedurst und meine Wut auf dich. Komm, Bruder, melde dich ein letztes Mal. Sprich zu mir, auf dass ich mit deinen Worten zur Burg zurückkehren und den mir versprochenen angestammten Platz einnehmen kann. Du weißt, dass dein nutzloser Sohn deinem Reich schadet. Mit mir an seiner Seite ...“}

\textit{Es verstummte. Keine Reaktion. Keine Stimme, wie es sie in Trieests Lavastein vernommen hatte. Wut machte sich in ihm breit. Wie konnten diese Toten es wagen, es zu ignorieren?! Es kontrollierte den Tod! Er war das mächtigste Wesen, dass je diese Höhle betreten würde, eine wahre Naturgewalt, und sie hielten sich für besser als es!}

\textit{Wut übernahm seine Gedanken. Wieder und wieder schlug es gegen die Rote Grotte, bis ihre Seitenwände Risse kriegten und die Fäuste der Feuerkriegerin brachen.}

\textit{Erst, nachdem das Böse aus der Höhle gestolpert war, erhoben sich die Echos der Gefallenen wieder und wurden wie immer zum Geräusch des brandenden Meeres, das allen kühlen Lavasteinen inne war.}
















\newpage
\section{Die entlegenste Insel im Hadrischen Meer}





\az{Jahr 61}

\textit{Halle des Ältestenrats, 61 a.Z.}\bigskip



Mit einem letzten Lächeln lehnte sich der grantige Kord wieder an die Stuhllehne. Der Ältestenrat hingegen brach in völliges Chaos aus. Gemurmel, Gebrabbel, Geflüster und Rowindas durchdringende Stimme, die sich an ihre Tochter wandte:

„Jarid, du hast als Wassermagierin einen Eid geschworen, Danwar und seine Einwohner zu beschützen! Du kannst uns nicht einfach verlassen!“

Jarid wich dem Blick ihrer Mutter aus und schwieg. Rowinda kannte das alte Gesetzt genausogut wie sie. Als Einwohner Danwars konnte Trieest den Schutz des Ordens der Wassermagier beanspruchen. Wenn Jarid sich Trieest auf seiner Reise anschließen wollte, so konnte sie der Ältestenrat nicht aufhalten.

Die straffte ihr Gewand und blickte dem Ältestenrat stolz entgegen.

„Führt mich zur Roten Grotte.“\bigskip









\az{Jahr 68}

\textit{Sieben Jahre später.}\bigskip



Jarid schlotterte vergeblich gegen ihr gefrorenes Gefängnis. Kaltes Eis umfasste und unterkühlte sie. Nur ihr Kopf war frei. Vor ihr stand Jivin, die kleine Wassermagierin aus dem mindestens fünften Zirkel, die sie gefangen genommen hatte, wie auch das Böse ihren Körper gefangen genommen hatte.

Etwas weiter weg hatten drei Feuerkrieger mit tiefschwarzen Augen und Lavasteinen in ihren Rüstungen vier weitere Wassermagier mit ihren Rankenschwertern gefesselt, darunter Jarids Mutter Rowinda. Die Feuerkrieger starrten regungslos in die Ferne, genau wie Jivin.

Jarid beschwor weitere Geisteskraft in das sie umgebende Eis und versuchte, Wärme aus der Umgebung in einen bestimmten kleinen Teil zu lenken. Wassermagier waren schließlich nicht auf ihre Handgesten angewiesen, um Wasser zu lenken. Diese verstärkten die Verbindung bloß.

Plötzlich klärte sich der Blick der kleinen Wassermagierin wieder. Ihr Mund verzog sich zu einer wütenden Miene und sie schien sich einen Moment fassen zu müssen, ehe sie grunzte:

„Ach, du bist ja auch noch hier. DAnn war mein Besuch hier zumindest keine völlige Verschwendung. Wer außer dir wäre schon am besten in der Lage, Trieest aufzuspüren? Ich bin an seinem Lavastein interessiert. Und einem bestimmten Echo, das darin nachhallt. Aber ich komm‘ ja ganz in Plapperlaune, das braucht dich doch alles nicht zu kümmern.“

Jivin zupfte mit spitzen Fingern den schwarzen Edelsteinsplitter aus ihrem Unterarm. Hellrotes Blut ergoss sich über selbigen, aber das schien sie nicht groß zu kümmern.

Dann beugte sie sich vor und massierte das Eis, das Jarids rechten Arm einschloss. Ein kleiner Teil schmolz prompt und legte eine Stelle um Jarids Handgelenk frei, kaum größer als eine Münze. Die kleine Wassermagierin riss Jarids Ärmel achtlos auf und hob den schwarzen Kristallsplitter hervor, direkt auf Jarids Unterarm gerichtet.

Jarid konnte sich denken, was sie vorhatte, und sie konnte sich denken, was geschehen würde, sobald der schwarze Kristallsplitter erst einmal in ihrem Arm steckte. Furcht lähmte ihren Geist, wie bereits ihr Körper gelähmt war. Sie schoss ihre Angst nieder, schloss ihre Augen und verdrängte die Eiseskälte aus ihren Gliedern. Es zählte nur das hier und jetzt. Sie war eins mit dem Wasser. Und das Wasser wollte nicht vereist sein. Die Umgebung war zu heiß dafür. Das Eis wollte schmelzen, das Wasser wollte fließen, und Jarid konnte das Eis an einigen konkreten Stellen erlauben, zu fließen ...

Es knackte und knirschte, und Jarid riss ihre Hand aus dem bröckeligen Eisblock, der sie eingeschlossen hatte. Die kleine Wassermagierin riss ihre Augen auf und stieß einen Fluch in einer kehligen Sprache aus, die Jarid nicht verstand. Zu langsam!

Jarids Hand, noch immer zur Hälfte in Eis eingeschlossen, schlug gegen die Faust der kleinen Wassermagierin. Eissplitter explodierten. Jivid taumelte zurück. Ihr Griff löste sich und der kleine schwarze Kristallsplitter vollführte eine schöne Parabel durch die Luft, ehe er von einer Dampfsäule aus dem löcherigen Boden erfasst wurde und aus Jarids Blick verschwand.

Die drei Feuerkrieger standen nicht mehr anteilslos abseits. Stattdessen stürzten sie nach vorne, auf die Stelle zu, wo der Splitter wohl gelandet war. Sie kamen ein, zwei Schritte weit, ehe sie einer nach dem anderen zur Seite kippten und am Boden zuckend liegen blieben. Rotes Leuchten drang aus den Lavasteinen in ihrer Brust, wie als Trotz gegen die Tatsache, dass die Edelsteine vor wenigen Momenten noch von einer tiefen Dunkelheit erfüllt gewesen waren. Hinter ihnen fielen die Rankenschwerter von den gefangenen Wassermagiern ab.

Auch Jivin war zur Seite gekippt, als das Böse seinen Halt über sie verlor. Jarid brach aus dem Eiswall hervor, der sie am Boden gehalten hatte. Dann war ihre Mutter auch schon bei ihr und schickte die letzten Eisstücke schwungvoll zur Seite.

„Wie schlimm verletzt bist du?“, fragte Rowinda leise und tastete Jarids Bauch ab. Jarid biss ihre Zähne zusammen und meinte: „Es geht schon. Ich war bereits bei einem Heiler. Zumindest einem Alchemisten. Einem danwarischen, sogar.“

Rowinda blickte sie zweifelnd an, hakte aber nicht nach. Stille breitete sich aus zwischen Mutter und Kind. Rowindas Augen waren milchiger, doch hielt sie ihr weißes Haar immer noch genauso hochgesteckt wie früher. Jarid versuchte, nicht daran zu denken, wie lange es her war, seitdem sie ihre Mutter gesehen hatte. Der Gedanke daran erfüllte sie mit Schuldgefühlen.

„Was weißt du über dieses Wesen? Was wollte es von dir?“, fragte Rowinda, ohne auf die lange Absenz ihrer Tochter einzugehen.

„Es hat gesagt, es wolle zu Tri, wegen eines Echos. Wir haben bereits vor Kurzen gegen es gekämpft. Damals konnte es aber nur einen Geist auf einmal kontrollieren. Ich vermute, dass es in Tris oder meinen Erinnerungen Danwar gesehen hat und deswegen hierher kam. Vielleicht konnte es sich hierher teleportieren?“

Rowindas Gesicht verhärtete sich beinahe unmerklich bei der Erwähnung von Trieest. Betont kühl fragte sie:

„Das heißt, dass sein Prozess des Wandels noch nicht beendet ist?“

„Mutter, nicht jetzt. Was hat dieses Wesen hier angestellt?“

Rowinda nickte entschlossen: „Du hast ja Recht. Aber viel zu erzählen gibt es nicht. Ich weiß nicht, wann es hier ankam und in welchem Körper. Es konnte Personen bei Berührung mit seinem schwarzen Kristallsplitter kontrollieren, und Feuerkrieger sogar länger, wenn es ihre Lavasteine kontaminiert hatte. Es verlangte, dass wir dich hierher bringen. Wir haben beschlossen, uns nicht unnötig zu widersetzen und den richtigen Moment zur Auflehnung aufzuwarten.“

„Und der kam nie?“, fragte Jarid etwas spöttisch, ehe sie sich fasste und fortfuhr: „Tri und ich fanden diesen Kristall im Schädel eines uralten Ungeheuers, des wilden Hraaks. Er zerbrach, und doch scheint das Böse ungebrochen. Es scheint durch diesen Splitter andere Personen steuern zu können. Das ist nicht gut, hoffentlich gibt es nicht noch weitere solcher Splitter.“

„Fokussiere dich auf das, was du lenken kannst, Jarid. Lass die restlichen Sorgen sein. Dieser Splitter befindet sich noch hier in der Nähe, oder? Wie können wir ihn davon abhalten, weiteren Schaden anzurichten?“

Jarid zuckte schwach mit den Schultern: „Keine Ahnung. Wichtig scheint bloß zu sein, den Kontakt zwischen dem Kristall und gesteuerten Personen zu unterbrechen. Tri kam wieder vollständig zurück, und es steht zu hoffen, dass es den anderen auch so geht. Ich hoffe nur, dass Trieest nicht rückfällig wurde, als das Böse diese Feuerkrieger übernahm.“

Rowinda blickte wachsam um sich und beobachtete die gestürzte kleine Wassermagierin und die umgefallenen Feuerkrieger. Diese richteten sich inzwischen langsam wieder auf und sahen sich einigermaßen verwirrt um. Keine Spur aggressiven Verhaltens. Die Lavasteine leuchteten wieder in einem warmen Rot. Die Welt schien wieder heil zu sein.

„Die Kontrolle zu seinem Hauptleib zu unterbrechen, scheint auch die Kontrolle über die Feuerkrieger gekappt zu haben. Sehr praktisch, damit dürften auch die restlichen Gefangenen dieses Parasits befreit worden sein. Gut gemacht, Jarid.“

In Jarids Augen blickend, schien Rowinda sich zum ersten Mal wirklich auf ihre Tochter zu achten. Sie senkte ihren Kopf und meinte:

„Grün sind die Wogen der Wellen. Es freut mich, dich wieder zu sehen, Jarid. Und es tut mir leid, dich hierher getunnelt zu haben. Es war zu gefährlich. Wir hätten uns wehren sollen.“

Die passende Begrüßungsfloskel war schnell entgegnet. Ungelenk richtete sich Jarid auf und überblickte das Becken von Quodlon.

„Wir müssen den Splitter finden.“

Rowinda blickte sich besorgt um: „Meinst du etwa, der Splitter allein könnte etwas Finsteres anrichten? Kann das Wesen ihn auch jetzt noch bewegen?“

Als hätte es darauf erwartet, schoss ein klitzekleines schwarzes Etwas aus dem Nebel hervor und grub sich in Rowindas Rücken. Diese grinste schmerzverzerrt, langte hinter sich und zog etwas hervor. Einen kleinen, altbekannten schwarzen Kristallsplitter, von dessen Spitze ein Blutstropfen rötlich schimmerte.

„Zu langsam, Jarid. Viel zu langsam“, kicherte das Böse aus Rowindas Mund und verzerrte ihn zu einem Lächeln. Finster-Rowinda tat einen mächtigen Satz über Jarids Kopf und landete im Becken von Quodlon. Eine gewaltige Welle schwappte auf und trug sie weg von Jarid, näher zu der Gruppe der restlichen Wassermagier und Feuerkrieger, die sich noch verwirrt vom vorherigen Geschehen erholten.

Laut pochte das Blut in Jarids Ohren, als sie Furcht und Ärger erneut unterdrückte und zur Handlung schritt. Jarid sprang nach vorne und ergriff selbst Kontrolle über einen Wasserberg, ließ sich davontragen und surfte ihrer Mutter nach, quer über das Becken von Quodlon. Nebenbei krümmte sie ihre Finger und versuchte, die Welle zu teilen, auf welcher der Körper ihrer Mutter ritt. Das Böse drehte sich elegant um die eigene Achse und schnippte verächtlich. Jarids Wellenberg zerbarst. Feurig heißes Quellwasser umschlang sie.

Jarid schrie auf und verband im Geist das heiße Wasser um sie herum mit den Überresten des Eisblocks, der am Uferrand vor sich hin schmolz. Ihre Muskeln verspannten sich, während die Wärme durch sie durch glitt. Schwarze Schlieren umschlagen den Rand ihres Gesichtsfelds. Das Wasser, das ihren Körper umspülte, kühlte sich gerade lange genug ab, damit Jarid einen tiefen Atemzug tun und sich sammeln konnte. Dann wurde sie eins mit dem Becken von Quodlon. Sie war der Dampf, der über dem Becken stieg. Sie war die Wogen, die sich auf der Oberfläche kräuselten. Sie war die Schwingungen, die die Wassermassen in Bewegung hielten. Das Wasser wusste nicht, dass der Boden des Beckens speziell bearbeitet worden war, um bestimmte Resonanzen zu ermöglichen. Das Wasser wusste auch nicht, wie man diese auslöste. Aber sie war nicht nur Wasser, nein, sie war auch Jarid, die Magierin, und Jarid wusste, was sie zu tun hatte.

Sie lenkte ihre ganze Konzentration auf das Wasser. Versuchte, all die Wassermassen gleichzeitig in ihren Gedanken zu haben. Sie stellte sich jeden Tropfen vor, von der Oberfläche bis hinab zur tiefsten Stelle. Und als sie das geschafft hatte, gab sie jedem dieser Tropfen in Gedanken einen kleinen Schubs.

Dann noch einen.

Und noch einen.

Ein dumpfer Klang ertönte, als das Becken von Quodlon als riesige Klangschale zu schwingen begann. Jarid spritzte blitzschnell aus der Mitte des Beckens hervor und schwappte ans Ufer, wo sie wieder zu einem Menschen zusammenfloss. Sie war schwach und ihr war kalt, aber das war jetzt nicht wichtig. Wo war ihre Mutter? Wo waren die restlichen Feuerkrieger?

Im allgegenwärtigen Dampf neben dem Becken war es nur schwer zu erkennen, doch da vorne regten sich definitiv Schemen. Jarid versuchte, den Nebel zur Seite zu lenken, doch ihre Kräfte gehorchten ihr nicht mehr. Jeder Atemzug schmerzte, und sie wollte nur noch liegen. Nicht jetzt, rief sie sich zu.

Da trat eine Gestalt aus dem Nebel. Eine großgewachsene Feuerkriegerin. Ihr linkes Auge fehlte und diese Gesichtshälfte war vollkommen von vernarbter Haut überzogen, welche an angeschmolzenes Wachs erinnerte. Jarid erkannte sie. Sie war dabei gewesen, als die damals Kleine ihren Unfall gehabt hatte. Nun hatte sie es doch noch ihre Ausbildung zur Feuerkriegerin abgeschlossen. Davon zeugte der Lavastein in ihrer Brust. Doch war dieser gerade so finster, als wäre ein Schatten darin eingezogen. Jarids Herz sank.

„Ah, das bist du ja“, krächzte die Feuerkriegerin fröhlich, „Ich dachte schon, ich hätte dich verloren. Das wäre zu schade gewesen.“

Die Feuerkriegerin zerrte an etwas hinter ihr und schleuderte unter einigem Ächzen einen bewusstlosen Feuerkrieger nach vorne. Dieser trug noch einen rot schimmernden Lavastein, doch half ihm dieser gerade relativ wenig. Ein Rankenschwert steckte tief in seinem blutigen Hals

„Lauf nicht weg“, rief die Feuerkriegerin grinsend zu Jarid.

„Ich bin gleich voll bei dir“, antwortete Finster-Rowinda, welche hinter der Feuerkriegern hervorschritt und den schwarzen Kristallsplitter gegen den roten Lavastein des sterbenden Feuerkriegers hielt. Das rote Leuchten wich schwarzen Schatten, als würde eine finstere Essenz vom Kristallsplitter in den Lavastein gekippt werden. Der Feuerkrieger setzte sich auf, riss belanglos das Rankenschwert aus seinem Hals und sah schon fast gelangweilt zu, wie sein Lebenssaft herausschwappte.

„Beim ersten Mal dachte ich noch, dass es wirklich etwas Besonderes wäre, über die Macht der Nekromantie zu verfügen. Aber alles verliert seinen Charme, wenn man es zu oft macht“, intonierte er.

Neben ihm stellten sich Finster-Rowinda und die vernarbte Feuerkriegerin auf.

Jarid hob ihre Hände, aber das Wasser wollte ihr weiterhin nicht gehorchen. Noch war sie zu erschöpft.

„Ich mache es dir einfach“, meinte das Böse dreistimmig. „Ergib dich mir und lass mich deinen Körper führen. Hilf mir, deinen Trieest zu finden. Danach gehen wir unsere getrennten Wege. Niemand muss zu Schaden kommen.“

„Versprechende Worte eines Wahnsinnigen haben keinerlei Wert!“, erwiderte Jarid erheblich selbstsicherer, als sie sich fühlte. Feuerkrieger mochte das Böse über die Lavasteine viele auf einmal steuern können, doch Wassermagier konnte es wohl immer nur einen auf einmal führen. Wenn das Böse ihre Mutter kontrollierte, mussten die restlichen Wassermagier noch frei sein. Wenigstens einer hatte sich doch bestimmt gegen die Feuerkrieger wehren können. Das nächste Dorf lag kaum einen Steinwurf weit von hier. Hatte jemand bereits Alarm geschlagen und Hilfe geholt?

„Wenn du dich nicht ergibst, werde ich gezwungen sein, dich mit Gewalt zu ergreifen“, erklang die Stimme des Bösen, „Und deine liebe Mutter wird natürlich die Narne hinuntergehen müssen. Was für eine Schande das doch wäre.“

Zwei scharfe Rankenschwerter wurden auf Rowinda gerichtet, die Klingen wie flackernde Flammen umherwabernd. Jarids Mutter lächelte, ergriff beide Schwerter mit je einer Hand und zog sie zu ihrer Brust. Blut rann aus ihren Fäusten hervor. Ein zutiefst beunruhigendes Bild.

„Sei kein Dummkopf, Spatz“, sagte sie liebevoll, „Tritt nach vorne und gib mir deinen Arm.“

Jarid zögerte. Das Leben ihrer Mutter stand auf der Kippe. Aber sie erinnerte sich auch noch daran, wie es gewesen war, von diesem Bösen kontrolliert zu werden, auch nur für einen kurzen Augenblick. Diese Machtlosigkeit ... was auch immer der Plan des Bösen war, es konnte nichts Gutes heißen. Unentschlossenheit machte sich in ihr breit.

„Na kommt schon, Mädel. Ober bedeutet dir Rowinda etwa gar nichts?“, meinte die verbrannte Feuerkriegerin. Sie trat ungeduldig nach vorne und griff nach Jarid. Etwas klickte in Jarid. Sie packte den Arm ihrer Angreiferin, drehte ihn auf den Rücken und zwang die Feuerkriegerin in die Knie.

Der singende Klang eines Rankenklingenstichs ertönte. Ein Wasserschwall schoss an Jarid vorbei und schlug die Feuerkriegerin endgültig beiseite. Der Urheber des Wasserschwalls, ein großgewachsener blonder Wassermagier mit einer blauhäutigen Backe im ansonsten bleichen Gesicht, trat aus dem Nebel und packte Jarid am Arm. „Nichts wie weg von hier! Hier, stütz dich auf mich.“

Während Jarid davonstolperte, sah sie aus dem Augenwinkel, wie zwei Rankenschwerter im Körper ihrer Mutter steckten. Finster-Rowinda lächelte, überreichte den schwarzen Kristallsplitter an eine Feuerkriegerin und knallte prompt zu Boden, zusammengeklappt wie eine Puppe, deren Fäden durchgeschnitten worden waren. Ihre glasigen Augen starrten nichts sehend in die Ferne.

Tot.\bigskip







„Lass mich zurück und schlage Alarm!“, krächzte Jarid,

„Veila hat sich bereits auf den Weg gemacht. Sie holt Hilfe. Die Orden müssten jeden Moment hier sein“, erwiderte der Wassermagier, „Und wir sind Danware, wir lassen niemanden der unseren zurück.“

Außer ihr mögt sie nicht, dachte Jarid in Gedanken an Trieest. Stumm stolperte sie weiter, nur weg vom Becken von Quodlon.

„Lasst den grantigen Kord rufen. Diese Situation verlangt nach einem Seher.“ Auch wenn Kord als hochklassiger Seher im Voraus wusste, wo er von Hilfe sein würde, hatte er die faule Angewohnheit, nur auf persönliche Einladung aufzukreuzen.

Der Wassermagier wurde kaum merklich langsamer und blickte Jarid aus haselbraunen Augen an. „Kord ist schon über ein Jahr tot. Das Fieber hat ihn erwischt. Man sagt, in seinem Sterbebett habe er tatsächlich friedlich gelächelt.“

Jarid versuchte verzweifelt, weitere Gedanken an Unheil und Tod zu verdrängen. Ihre Mutter, Kord, diese Orden, sie hatten ihr alle nichts mehr zu bedeuten, ihr Leben lag nun außerhalb Danwars. Sie war eine Heldin von Andor, und als solche hatte sie sich um das Wohl der vielen zu kümmern. Viele, die in Gefahr sein würden, wenn das Böse Danwar verlassen konnte. Sie konnte kaum zur Verteidigung beitragen, aber vielleicht konnte sie mit den wenigen Informationen, die sie über das Böse besaß, von Hilfe sein. Wenn diese Informationen zur richtigen Person gelangen konnten.

„Wo ist der Hüter des Wissens?“

„Im jährlichen Koma des Bewahrens. In den nächsten Tagen nicht ansprechbar.“

„Was ist mit Feuermeisterin Nidwal?“

„Die Verschrobene? Lebt meines Wissens immer noch abseits des Dorfes in ihrer Steinkammer.“

Jarid brachte ein schwaches Lächeln zustande. Mutter Natur hatte sie nicht völlig im Stich gelassen.. „Bringt mich zu ihr.“\bigskip







Feuermeisterin Nidwals weißer Bart war wie üblich zu einer vollendeten Kugel frisiert und mit irgendwelchen Wachsen in Position gehalten, ein Kontrast zu ihrem vollkommen kahlen Schädel. Auch wenn Nidwal nun schon an die einhundert Sonnenumrundungen erlebt haben musste, war die alte Sternguckerin noch immer gut zu Fuß.

Sie war eine der vielen Feuerkriegerinnen, die ihren Lavastein nicht als Bürde für ein begangenes Vergehen in ihrer Brust tragen mussten, sondern ihren Lavastein als Auszeichnung für ihre abgeschlossene Ausbildung in einer maßangefertigten Rüstung tragen durften. Nidwal könnte ihren Lavastein mitsamt der durch ihn angetriebenen Rüstung problemlos ablegen, doch manche munkelten, dass sie ihre Rüstung selbst zum Schlafen anbehielt. Andere munkelten gar, dass sie ohnehin nie ganz schlafen würde, sondern immer höchstens ihren halben Körper in einen Ruhezustand bringen würde. Jarid selbst hatte sie schon mehrmals angetroffen, wie sie mit einem geschlossenen und einem offenen Auge an eine Wand lehnte und mit nur einer Hand irgendwelche Zeichen in eine Steintafel ritzte.

Gerade war Nidwal allerdings mit beiden Körperhälften wach und klopfte den Boden vor ihrer Hütte mit einem komplizierten Werkzeug ab. Es erinnerte an einen Hammer, bestand jedoch aus mehreren beweglichen Teilen und seine Spitze war mit Zahnrädern besetzt.

Nidwal blickte auf und lächelte höflich, als sie Jarid erblickte.

„Na, wen haben wir denn hier?“, rief sie laut. Vom Dach kam ein großer weißer Rabe herabgesegelt, der eine Runde um Jarid und den Wassermagier drehte, ehe er auf Nidwals Schulter landete und ihr etwas ins Ohr krächzte.

„Ah, Jarid! Blau sind die Weiten des Meeres“, rief Nidwal fröhlich, „Morgentau hast du dich bei der Zeremonie zum dritten Zirkel benannt, oder? Lange nicht mehr gesehen! Und du ...“

Nidwal blickte mit gerunzelter Stirn zu Jarids Begleiter hoch. Erneut krächzte der weiße Rabe auf ihrer Schulter ihr etwas ins Ohr. Nidwal nickte betrübt: „Deinen Namen kennen wir nicht, aber dich habe ich bestimmt schon mal gesehen, als du noch ein ganzes Stück kleiner warst als jetzt. Wie lautet dein Name?“

„Nicht relevant“, meinte der Wassermagier abweisend.

Nidwal reagierte gespielt empört: „Nicht relevant?! Mein Junge, Namen sind stets von Relevanz. Jeder Name hat eine Bedeutung, und jeder Name ist es wert, erfahren zu werden. Es wäre unhöflich von mir, nicht ...“

„Ja, schon“, unterbrach der Wassermagier den ihm entgegenströmenden Redeschwall, „Es ist nur ...“

Er gestikulierte ungelenk zu Jarid hinüber. Nidwals rot glühende Augen verengten sich leicht, als ihr Jarids blutgetränktes Kleid auffiel. Sie rappelte sich auf und klopfte sich schwarzen Staub vom Mantel, welcher unter ihrer Rüstung hervorragte. An der Art, wie sich ihr Kiefer dabei lautlos hin und her bewegte, erahnte Jarid, dass ihr Geist gerade auf Höchstleistung ratterte. Vermutlich ging sie gerade die besten Heilmittel in Reichweite durch. Und fragte sich, warum Jarid wegen einer Bauchwunde zu ihr kommen würde.

„Ein bösartiges Wesen ist in Danwar eingedrungen“, versuchte Jarid, die Geschehnisse so kurz und klar wie möglich zusammenzufassen, „Es lebt in einem schwarzen Kristallsplitter und übernimmt die Kontrolle über alle, zu denen es direkten Kontakt hat.“

„Hm“, machte Nidwal, den Blick weiterhin sorgenvoll auf Jarids Bauchwunde gerichtet.

„Es kann auch längere Zeit Kontrolle über einen Feuerkrieger ergreifen, falls es mit dessen Lavastein in Berührung kam.“

„Hm“, machte Nidwal erneut und langte gedankenverloren an ihren eigenen Lavastein.

„Es kann auch in den Erinnerungen der von ihm kontrollierten Personen umherforschen.“

„Hm“, machte Nidwal und kratzte sich am weißen Kugelbart.

„Es sucht nach dem Echo eines Gefallenen, das es beim Kontakt mit Trieests Lavastein hörte.“

Nidwal schwieg, abwartend, ob noch mehr kam. Als dem nicht so war, begann sie mehrere Fragen auf einmal: „Warum hört es nicht einfach ... hat es noch Kontakt mit ... wo ist dein ...“

Der Rabe krächzte laut etwas in Nidwals Ohr, das verdächtig nach „Trieest“ klang, und Nidwal beendete ihre Frage: „Wo ist Trieest?“

Jarids Stimme stockte und sie schüttelte ihren Kopf. Trieest war weit weg von hier, vielleicht schwer verletzt durch den Zweikampf mit diesem riesigen Skralhäuptling, vielleicht tot.

„Das Böse hat Kontrolle über mindestens vier Feuerkrieger ergriffen und Jarid über das Becken von Quodlon hierher holen lassen. Wir haben uns ihm gebeugt“, sprach der Wassermagier nun kleinlaut.

„Wo sind sie jetzt?“

„Wer weiß das schon? Auf der Suche nach Jarid, vermute ich. Wohl auf dem Weg ins Dorf.“

„Hm.“

„Das Böse ist halb wahnsinnig“, fand Jarid nun ihre Stimme wieder, „Es steckte lang in einem unförmigen Biest fest, welches sich nicht ausdrücken konnte. Dem wilden Hraak. Kurz nach dessen gewaltsamen Ende vernahm das Böse das Echo einer Stimme. Nun scheint es wie besessen auf der Suche danach zu sein. Doch gleichzeitig ergötzte es sich an seiner Kontrolle, am Schmerz, das es anrichten konnte, am ...“

Jarids Stimme schwankte wieder, doch dieses Mal fand sie sie rasch wieder: „Es hat Rowinda getötet, weil ich mich ihm nicht ergeben hatte.“

Der weiße Rabe musterte Jarid aus roten Augen und flüsterte Nidwal etwas längeres ins Ohr. Nidwals Blick wurde weich: „Oh Jarid, du musstest eine schwere Entscheidung treffen. Das ist nicht deine Schuld.“

„Ich habe keine Entscheidung getroffen. Ich habe nur zugelassen, dass er meiner Mutter etwas antut! Aber solche Gedanken nützen jetzt nichts.“

Die Feuermeisterin legte ihren Kopf schief: „Es ist nicht zwingend hilfreich, deine Schmerzen zu ...“

„Nicht jetzt! Wir müssen etwas unternehmen, ehe es die Lavasteine aller Feuerkrieger übernimmt und sie nur zum Spaß an der Freude in ihr Schwert laufen lässt!“, sprach Jarid etwas lauter, als nötig war.

Nidwald schüttelte ihren Kopf: „Jarid, du bist verletzt und am Ende deiner Kräfte. Und ich bin schon seit Jahren am Ende der meinen. Wir können kaum etwas unternehmen. Warum bist du zu mir gekommen?“

„Ihr seid Feuermeisterin Nidwal! Ihr habt mir schon so oft aus der Patsche geholfen. Ihr habt so viele Danware angeleitet. Ihr versteht das Leben und die Lebenden. Ihr müsst doch wissen, was sich gegen dieses Böse tun lässt!“

„Soll ich das Böse suchen und bekämpfen gehen, während ihr das hier löst?“, fragte der Wassermagier unruhig.

Nidwald zuckte mit der Schulter: „Das weiß ich wohl kaum besser als du. Aber dieses Böse klingt so, als wäre kein einzelner seinem Kollektiv im Kampf gewachsen. Es kostet dich kaum etwas, noch ein wenig länger hier zu bleiben.“

Der Wassermagier schaute sich hilflos um.

„Warum bist du zu mir gekommen, Jarid?“, fragte Nidwal erneut, noch etwas sanfter.

Und Jarid sprach: „Ich bin gekommen, weil ich etwas tun muss. Ich habe das Böse entkommen lassen, ich habe es hierher gelenkt, ich habe nicht darauf geachtet, dass es noch am Leben sein könnte. Ich bin dafür verantwortlich, dass es hier ist, und ich muss etwas tun, damit es keinen weiteren Schaden anrichtet.“

Nidwal blickte Jarid tief in die Augen und meinte traurig: „Oh, Jarid. Du hast einen analytischen Geist, der sich nur selten selbst überlistet. Was das Böse tut, ist seine Schuld, nicht die deine. Du musst nichts tun. Und du musst doch gewusst haben, dass ich nichts gegen dieses Böse ausrichten konnte, was zwei Dutzend Feuerkrieger im Dorf nicht besser könnten.“

Jarid schwieg, während ihr Geist raste. Sie erinnerte sich zurück an das kleine Mädchen, das sie einst gewesen war. Das zu Nidwal gerannt war, weil ihre Schultafeln in den Lavasee gefallen waren, als sie nicht auf ihre Tasche geachtet hatte. Dem Nidwal geholfen hatte, als es gezweifelt hatte, ob der Orden der Wassermagier wirklich sein Pfad sei. Das mit Nidwal Apfelkuchen gebacken hatte, um sich von der Beerdigung seines Vaters zu erholen.

Nidwal hatte stets die richtigen Worte gefunden, um sie zu beruhigen. Konnte es sein, dass sie gar nicht nach einem Mittel gegen das Böse suchte, sondern nur eines gegen ihren aufgewühlten Gemütszustand?

„Das wollte ich damit nicht sagen“, meinte Nidwal hastig, als sie Jarids Gesichtsausdruck sah, „Du hast dich daran gewöhnt, anderen zu folgen. Du überließt Trieest die Entscheidungen über Eure Abenteuer, um nichts über die Prophezeiung der Roten Grotte zu verraten. Nun wendest du dich an mich, in der Hoffnung, dich leiten zu können. Danke für dein Vertrauen in mich. Aber du hast mich schon seit Klein auf ein Podest gestellt, dem ich nicht gewachsen bin. Ich bin nur eine alte verrückte Feuermeisterin, die sich gerade ernsthaft überlegt, ob sie wegen dieses Bösen von Danwar abhauen soll.“

„Ich werde nicht abhauen“, meinte der junge Wassermagier nun stolz, „Ich werde kämpfen bis zum letzten Schweißtropfen, wenn es sein muss. Wir sind Danware, wir lassen unsere Heimat nicht im Stich!“

Nidwal lächelte nur traurig. Sie haderte offenbar damit, ihre Heimat im Stich zu lassen.

„Meisterin Nidwal!“, sprach Jarid aufgebracht, „Wie könnt Ihr nun einfach aufgeben wollen? Ihr wusstet immer die richtigen Worte, um so viele Danware auf glückliche Pfade zu lenken. Ihr müsst doch auch die Worte kennen, die auch Euch selbst auf den richtigen Pfad leiten.“

„Ach, Jarid, Kommunikation ist selten so einfach. Es mag die richtigen Worte geben, die dich und mich wieder mit Hoffnung erfüllen würden. Es mag auch die richtige Abfolge von Worten geben, die dieses böse Wesen von seinen Taten abbrachte und seine Kraft auf konstruktivere Projekte leitete. Ihm die Fehler seiner Logik aufzeigten.“ Nidwal gluckste. „Bei der Anzahl Gesprächen, die ich in meinem langen Leben bereits geführt habe, habe ich die benötigten Sätze vermutlich alle schon einmal so ähnlich gesprochen, vielleicht gar in der richtigen Reihenfolge. Aber die richtigen Worte zu kennen, ist eine Kunst, die man nur durch passendes Verständnis des Gegenübers kriegen kann. Und ich habe keine Ahnung, wie dieses Wesen denkt. Wo fängt man bei einem wahnsinnigen Edelstein überhaupt an?“

Der blonde Wassermagier horchte auf: „Jarid, sagtet Ihr, dass diese Böse die Erinnerungen von all denen liest, die es übernimmt?“

„Ich habe es selbst gespürt.“

„Und Meisterin Nidwal, stimmt es, dass Ihr viele Jahrzehnte lang den unterschiedlichsten Danwaren als Seelsorger und Ratgeber zur Verfügung standet? Dass Ihr auf Jahrzehnte der Weisheit und guten Ratschläge zurückblicken könnt?“

„Nein!“, rief Jarid. Auch der weiße Rabe auf Nidwals Schulter krächzte protestierend auf, wobei sein Vogelgesicht natürlich keine Regung zeigen konnte. Nidwals Gesicht hingegen durchlief rasch viele Emotionen, von Neugierde zu Überraschung, zu Freude, zu Sorge bis hin zu schlussendlich kalter Entschlossenheit. Sie zupfte an ihrem weißen Kugelbart.

„Lasst uns das Böse suchen.“

Jarid schüttelte ihren Kopf. „Bitte, Meisterin Nidwal, gebt nicht auch noch Ihr Euch ihm hin.“

Nidwal breitete theatralisch ihre Arme aus. „Personen sind darauf trainiert, Widersprüche in ihrem Denken zu erkennen und auszumerzen. Wenn dieses Böse seinen Geist mit denen übernommener Körper teilt, muss es zwar deren Meinungen und Gedanken gut von den seinen trennen können, sonst hätte es sich schon längst in ihnen verloren. Und doch sollte es in der Lage sein, meine Werte und Erfahrungen zu sehen und davon zu profitieren, schneller, als es eine Konversation je könnte. Es ist einen Versuch wert. Und zum Davonlaufen bin ich eigentlich ohnehin zu alt. Ich muss mich von diesem Bösen kontrollieren lassen, um es davon zu überzeugen, dass das Leid, das es anrichtet, nicht sinnvoll ist.“\bigskip







„Was siehst du?“, fragte Jarid den großen Wassermagier zum dritten Mal. Er reichte ihr wortlos das Fernrohr. Durch es hindurch sah Jarid den entfernten Dorfplatz. Pflastersteine um einen dampfenden Brunnen. Triste Steinhäuser vor kargen Gärten. Die Danwarische Architektur hatte sie nicht vermisst.

„Meisterin Nidwal ist erst vor ...“, er blickte auf ein kompliziertes mechanisches Gerät an der Wand von Nidwals Hütte, das verschiedene bewegliche Kugeln vor einem Sternenhimmel-Hintergrund zeigte, „Sieben? Sie ist vor sieben Minuten aufgebrochen? Selbst auf ihrem Schlitten wird es länger dauern, bis sie das Dorf erreicht. Falls das Böse überhaupt dort ist.

„Wo sonst sollte es sein?“, fragte Jarid harsch, ehe sie sich auf ihre Manieren besann, sich entschuldigte und dem Wassermagier höflich das Fernrohr zurückreichte.

Der Wassermagier reichte ihr ein rotbraunes Döschen. „Ich glaube, das ist Goldsalbe? Habe ich neben dem Fernrohr in Meister Nidwals Ramschschachtel gefunden. Ist nicht nur für Feuerkrieger gut.“

Jarid öffnete das Döschen und erblickte die erwartete golden glänzende Creme. Sofort entfaltete sich in ihrer Nase der altbekannte Geruch nach Feuer und Metall. Zwischen spitzen Fingern zerrieb sie einen kleinen Teil der Paste und ließ das entstehende Puder über ihre Bauchwunde rieseln. Augenblicklich verebbte der Schmerz.

„Danke dir“, meinte Jarid seufzend. Sie blickte den Wassermagier an und fragte: „Jetzt aber bin ich doch gespannt: Wie lautet dein Name eigentlich?“

Er setzte zu einer Antwort an, unterbrach sich aber und meinte stattdessen: „Meisterin Nidwal ist schnell wie der Blitz! Sie ist bereits am Dorfplatz angekommen. Sie ist von ihrem Schlitten gestiegen und hat ihre Arme ausgebreitet. Sie ruft etwas.“

„Lass mich das Fernrohr sehen, ich habe mal ein wenig Lippen lesen gelernt“, rief Jarid. Dann fiel ihr ein: „Oder warte, hast du in Meisterin Nidwals Ramschschachtel zufälligerweise ein Gerät gesehen, das wie ein langes dünnes Rohr mit einem eisernen Ohr am Ende aussieht?“

Der Wassermagier nickte, und huschte bereits wieder ins Innere der Hütte. Als er zurückkehrte, hatte er das gewünschte Gerät bereits bei sich.

„Sie zeigte es mir zur Aufmunterung, als meine Base einmal ...“, fing Jarid an, ehe sie sich fing, „Nun, das ist eine Geschichte für ein anderes Mal. Schau, du kannst es hier und hier an das Fernrohr anschrauben, dann hier diese Öffnung schließen und dann das Fernrohr auf die Szene auf dem Dorfplatz richten.“

„Komm her, du finsterer Wicht. Ich ergebe mich dir!“, erschall Meisterin Nidwals krächzende Stimme aus der Metallkonstruktion. Ein wenig blecherner als in Realität, und mit allerlei Hintergrundrauschen, aber immer noch verständlich.

Der Wassermagier war zusammengezuckt und blickte mit großen Augen das Gerät an. Jarid lachte bloß:

„Meisterin Nidwal ist wirklich stolz auf diese Erfindung. Sie nennt es ihr Fernohr.“

Der Wassermagier stimmte kurz in ihr glockenhelles Lachen ein.

„Der Erste Feuerschmied Melek und die thermische Thermosta, diese fröhliche Tüftlerin aus dem rauchenden Tal, boten Meisterin Nidwal angeblich bereits 5 Goldstücke, 73 Silberstücke und 27 Lavadeute für detaillierte Pläne vom Fernohr an. Doch soweit ich weiß, lehnte Nidwal immer ab. Ein solches Meisterwerk soll es nur einmal geben, meinte sie. Böse Zungen – insbesondere die des grantigen Kords – behaupteten, dass sie selbst nicht mehr wisse, wie man ein solches Fernohr ... “

„Wer bist du, der du mich rufst, statt dich mir zu widersetzen versuchen?“, erschall eine keuchende Stimme aus dem Fernohr, die definitiv nicht zu Nidwal gehörte. Sofort erstarb das Gelächter der beiden Wassermagier.

„Die kleine Jivin spricht, die vorhin schon einmal durch das Böse gesteuert wurde“, teilte der blonde Wassermagier nach einem Blick durch das Fernrohr mit. „Das Böse muss sie wieder erwischt haben.“

„Dann hat sie den schwarzen Kristallsplitter“, kombinierte Jarid, „Das Böse konnte bislang nur einen Wassermagier auf einmal kontrollieren, darum hat es mich auch nicht selbst nach Danwar geholt.“

„Warum fragst du danach, wer ich sei, wenn du es ohnehin gleich wissen wirst?“, erklang Nidwal sanfte Stimme aus dem Fernohr.

Das Böse klang fast trotzig. „Was auch immer dein Plan ist: Du kannst mich nicht aufhalten. Ich werde triumphierend über die gesamte Welt herrschen. Das Schicksal will es so.“

„Dann sei es so“, meinte Nidwal schicksalsergeben.

Der kleine Wassermagier kommentierte weiter: „Jivin ist auf Nidwal zugetreten. Sie greift nach ihrem Lavastein!“

Jarid schloss ihre Augen und sandte ein weiteres Stoßgebet an Mutter Natur. Sie hätte Meisterin Nidwal nie davonziehen lassen dürfen. Oder wenigstens den Wassermagier mitnehmen lassen.

„Weitere Feuerkrieger sind um die beiden herum aufgetaucht. Allesamt mit schwarzen Lavasteinen, soweit ich das erkennen kann. Meisterin Nidwals Lavastein wird dunkler. Jetzt ist er völlig schwarz. Das Böse hat Nidwal!“

Jarid musste sich sehr zusammenreißen, um dem jungen Wassermagier das Fernrohr zu lassen.

„Was tut es jetzt?“, wollte sie wissen.

„Nichts Besonderes. Es steht einfach nur da. Sie stehen alle einfach nur da.“

Nichts als Rauschen war aus dem Fernohr zu vernehmen. Dann ein dumpfes Plumpsen.

„Sie sind alle auf die Knie gefallen!“, rief der Wassermagier aufgeregt, „Jivin hält sich sogar ihren Kopf. Was auch immer das Böse in Meisterin Nidwals Erinnerungen erleben, es hat es erreicht.“

„Nicht zu fassen“, hauchte Jarid.

„Jetzt stehen sie wieder auf! Sie laufen im Gleichschritt aus dem Dorf weg.“

„Wohin?“

„Schwer zu sagen. Sie fliehen nicht, aber sie nehmen es auch nicht gemütlich. Sie folgen dem Weg ... zu den Flaschenzügen! Zu den Lastenzügen am Kopf! Sie wollen zum Hafen im Bassin!“

„Wollen sie die Insel verlassen? Was tun wir nun?“

„Was können wir überhaupt tun? Beobachten und den richtigen Leuten berichten gehen.“

„Wollen wir sie nicht aufhalten?

„Mit Verlaub, Jarid, aber Ihr seid nicht unschwer verletzt. Mit der Geschwindigkeit der Armee des Bösen können wir nicht mithalten. Lasst uns lieber den restlichen Orden informieren.“\bigskip







In den nächsten Minuten tauschte der junge Wassermagier einige Brieflilien mit einer ranghöheren Wassermagierin im Inselinnern aus. Vermutlich berichtete er von den Vorfällen im Dorf und wurde darüber informiert, dass der Orden die Lage bereits im Blick hatte, schließlich waren mehrere Wassermagier dem Scharmützel beim Becken von Quodlon entkommen.

Jarid befürchtete, dass der Orden wenig ausrichten konnte. Weitere Feuerkrieger würden nicht mehr in die Nähe des Bösen gelassen und die Wassermagier würden kaum mehr Widerstand leisten, nachdem das Böse mit den Leben der bereits unter seiner Kontrolle stehenden gedroht hatte. Der Orden hatte seit Jahrzehnten keine Landgefechte mehr durchgeführt, und erst recht nicht gegen einen Gegner, der die eigenen Taktiken in- und auswendig kannte.

Niemand wusste, was es vorhatte oder warum. Sie wussten bloß, oder glaubten zu wissen, dass ein Vorgehen gegen das Böse den raschen Tod aller Besessenen zur Folge haben konnte. Und niemand wollte das riskieren, solange das Böse keine offensichtliche Gefahr mehr darstellte und augenscheinlich von Danwar abzog.

Während der junge Wassermagier also relativ ratlos mit einigen anderen Ordensmitgliedern kommunizierte, blickte Jarid rastlos durch das Fernrohr auf den Abzug des Bösen.

Es war gruselig, zu sehen, wie die mindestens zehn Feuerkrieger in absoluter Synchronie und mit vollkommen gleichgültigen Gesichtsausdrücken durch die felsige Landschaft Danwars zogen. Inzwischen sah Jarid nur noch ihre Hinterköpfe. Einholen würde sie sie nicht mehr.\bigskip







Unschöne Erinnerungen kamen über Jarid, als sie zum ersten Mal seit sieben Jahren vor den versammelten Ältestenrat trat.

Der Rat sah die Lage leider ähnlich fatalistisch wie der junge Wassermagier. Einige Mitglieder schienen gar erfreut über die Nachricht, dass das Böse drei Schiffe eingenommen hatte und in den Süden losgesegelt war. Nach Andor. Ihr Entschluss schien felsenfest: Der Rat würde keine Unterstützung nach Andor senden.

Das war schon schade. Wassermagier könnten die Ankunft der Schiffe des Bösen an der andorischen Küste verhindern, Feuerkrieger könnten mit ihren vielseitigsten Gaben, von schwelenden Feuerkugeln über Heilende Flammen bis hin zu ganzen Feuerwänden, stark aushelfen. Doch die beiden Orden würden tatenlos auf Danwar verbleiben. „Um der dringlicheren Bedrohung durch Feuergeister, Magmabestien, Ascheskelette und Strumtrolle Herr zu werden.“

Pah! Jarid hatte in ihrem zugegebenermaßen kurzen Aufenthalt hier in Danwar nicht einmal die Fußspuren einer Kreatur erblickt. Sie wusste um die typische Untätigkeit der Orden. Ein wenig enttäuscht war sie dennoch über die ausgebliebende Hilfeleistung. Insbesondere, als sich auch der junge blonde Wassermagier von ihr abwandte und etwas von der Priorität der Sicherheit Danwars nuschelte.

Jarid blickte ihn kopfschüttelnd an: „Du auch?“

Er blickte ihr entgegen und meinte etwas fester: „Wir sind Danware. Wir sorgen uns um die Sicherheit Danwars. Das muss unser Hauptfokus bleiben. Der Verlust dieser Feuerkrieger wiegt natürlich schwer, aber wir machen die Lage nicht besser, wenn wir ihnen hinterhersegeln und womöglich noch mehr verlieren. Andor hat seine Helden. Wir haben nur uns.“

„Noch sind unsere Feuerkrieger nicht verloren! Wenn wir das Böse, diesen Kristallsplitter, vernichten, können sie wieder frei sein. Unsere Wasserspäher haben berichtet, dass das Böse seine Gesellschaft den Wachsamen Wald ansteuern lässt. Wir müssen die Bewahrer warnen!“

„Wo waren die Bewahrer, als die varatanische Flotte Kurs auf Danwars Lavasteine nahm? Sie scherten sich keinen Lavadeut um die Stabilität unserer Insel und wir kümmern uns nicht um die Sicherheit ihres großen hohlen Baumes. Sie wollten das. Jetzt ernten sie halt, was ihre Vorfahren einst säten.“

„Dann denke halt nur an Danwar, aber denke daran! Dieses Böse ist wie eine Plage, welches jede Person, jedes Wesen, das es will, übernehmen kann! Was, wenn es einen König befällt? Den Urtroll? Die Mächte des Meeres höchstpersönlich? Was, wenn es die Ketten seines Gefängnisses sprengt und alle Wesen übernimmt, die es finden kann? Was, wenn es sich danach an Danwar erinnert und die Insel im Meer versenkt?“

„Du befindest dich in einer Angstspirale. Was, wenn es einen Fehler macht und in wenigen Tagen bereits Geschichte ist?“

Jarid schluckte eine frustrierte Antwort herunter und blickte den Magier ein letztes Mal flehend an. Als dessen Reaktion nur darin bestand, noch ein klein wenig beschämter zu Boden zu blicken, wandte sich Jarid energisch ab und dem versammelten Rat der Wassermagier zu. Vom niedrigsten Novizen bis zum höchstrangigen Absolventen waren verschiedenste Menschen vertreten. Gar ein Mitglied des sechsten Zirkels war anwesend. Doch auch sie alle schüttelten bloß ihre Köpfe. Danwar würde kein Kontingent aussenden, um die Gefahr durch das Böse zu dämmen. Höchstens eine Botschaft der Warnung an den Orden der Bewahrer wurde in Betracht gezogen.

„Vergesst es“, murmelte Jarid, „Das mache ich selbst.“

Immerhin war der Ältestenrat so überaus großzügig, sieben Mitglieder zum Becken von Quodlon zu beordern, auf dass man Jarid von dort aus sicher und schnell über den Ozean zum Brunnen neben dem Baum der Lieder senden könnte. Dort wären doch bestimmt einige Helden von Andor, die sich um dieses Problem kümmern konnten.

„Ich bin auch eine Heldin von Andor!“, zischte Jarid, „Und wir sind keine Götter! Wir können jede Unterstützung brauchen, die ihr erübrigen könnt, ja, sind auf diese angewiesen.“

„Und Danwar ist auf jeden einzelnen Feuerkrieger und Wassermagier angewiesen, sollte dieses Böse je hierher zurückkommen“, meinte die alte Freiga. Sie hatte nach Rowindas Tod ihren Platz als Ratsälteste eingenommen und blickte höchst selbstgefällig auf Jarid hinunter: „Inzwischen sind wir über seine Kräfte vorgewarnt, doch das heißt nicht, dass wir Danwar nicht weiter beschützen werden!“

Jarid entgegnete nichts.

„Bist du sicher, dass du mit deiner Wegreise nicht einmal bis zu Rowindas Beerdigung warten willst? Deine Helden und dein Trieest kommt bestimmt einige Tage ohne dich aus.“

Jarids Herz verkrampfte sich erneut. Ihre Mutter war tot, und Jarid würde nicht einmal ihre Bestattungszeremonie abwarten. Das würde Mutter Natur verstehen, machte sich Jarid klar. Und was die Priester des Flammenden Gottes und die restlichen Danware sagen würden, konnte ihr auch egal sein. Nichtsdestotrotz wässerten ihre Augen wieder einmal. Jarid ballte ihre Fäuste und drängte die aufschwellenden Tränen wieder in ihre Drüsen. Jetzt war nicht der Moment dafür.

Die Erwähnung von Tri machte ihr auch zu schaffen. Dieser Skralhäuptling mit der Eisernen Maske hatte ihn sicherlich schlimm verletzt. Jarid war allein. Sie würde sich allein durchschlagen müssen. Der Gedanke beunruhigte sie zutiefst, und so schob sie ihn zur Seite. Später. Jetzt hieß es erst einmal, sich in Andor auf die Ankunft der Schiffe des Bösen vorzubereiten.

Jarid konnte sich durch das Ende der Versammlung zusammenreißen.

Zum zweiten Mal in sieben Jahren verließ sie Danwar wütend nach einem vergeblichen Gespräch mit dem Ältestenrat. Im Gegensatz zu damals war kein Trieest mehr an ihrer Seite, und keine Mutter winkte ihr beim Abschied zu. Kurz hatte sie noch überlegt, Trieests Mutter Talemma aufzusuchen. Doch wollte sie keine alten Wunden aufreißen. Wenn das alles durch wäre, hatte sie noch genug Zeit, ihr die womöglich traurigen Neuigkeiten über Trieest zu berichten.

Mit der Unterstützung sieben unkorrumpierter Wassermagier überstand Jarid die Wasserreise durch das Becken von Quodlon zurück nach Andor unbeschadet.

Kaum war sie zurück in Andor aus dem nun dampfenden Brunnen nahe des Baums der Lieder gestiegen, fiel sie von Schluchzern gerüttelt zusammen.\bigskip







Bewahrer Tion rollte Jarid geschwind entgegen. Sein Gefährt, ein sogenannter Fahrstuhl – ein eleganter Stuhl mit zwei großen hölzernen Wagenrädern auf den Seiten – war Jarid bereits bekannt, und so beachtete sie es kaum. Stattdessen richtete sie ihre Aufmerksamkeit auf den verwahrlosten Anblick, den Tion bot. Sein hageres Haupthaar und sein langer Bart waren ungekämmt, sein Priesterkleid eiligst über sein Nachtgewand gestülpt. Er machte sich gar nicht erst die Mühe, sein Gähnen zu unterdrücken, und sprach müde:

„Unser oberster Priester Melkart ist in den östlichen Landen unterwegs. Verhandlungen mit der Einhorn-Sippe des Yetohe-Stammes. Ihr werdet mit mir vorstellig werden müssen.“

Jarid brachte einen höflichen Knicks und ein nur leicht gezwungenes Lächeln zustande. Dann weihte sie den Hohepriester der Mutter möglichst rasch ein in die Geschehnisse der vergangenen Tage. Meister Tion konnte ein ausgezeichneter Zuhörer sein, und seine anfängliche Müdigkeit verging wie im Flug, sobald Jarid den finsteren Kristall erwähnte, und erst recht, als Jarid auf die Gefahr der drei anrückenden Schiffe des Bösen hinwies.

Tion zückte eine kleine Glocke aus seinem Amtsgewand und schüttelte sie. Noch bevor ihr heller Schall verklungen war, rannte ein Novize an Tions Seite und beugte sich zu ihm nieder. Auch dessen Haar war verstrubbelt und ungemacht, doch die Augen glühten vor Eifer.

„Ihr wünscht, Meister Tion?“

„Lass die Teleskope an der Aussichtsplattform ganz oben am Baum besetzen, Komu. Haltet das Hadrische Meer im Auge. Wir erwarten ungewünschten Besuch. Drei Schiffe, die dir Jarid hier bestimmt gerne beschreiben wird.“

Novize Komu nickte und raste zurück ins Innere des Baumes der Lieder, kehrte nach einigen Augenblicken ebenso hastig wieder zurück und verschnaufte gerade lange genug, um Jarid nach einer Beschreibung der danwarischen Feuerschiffe zu fragen. Komu wiederholte Jarids Erklärungen zur Überprüfung Wort für Wort und hastete dann ebenso rasch wieder davon.

„Was füttert ihr euren Novizen? Zappelbohnen?“, fragte Jarid verschmitzt, „Wenn alle Anwärter nur halb so eifrig wären ...“

Tion schmunzelte nur und bat Jarid dann, ihn ans andere Ende dieses Balkons um den Baum der Lieder zu begleiten. Während Jarid Tions Fahrstuhl vor sich hin schob, murmelte Tion bedacht:

„Gedächtnisse sind eine eigenartige Sache. Ich selbst war mir lange absolut sicher über mein Wissen zu meiner Vergangenheit. Und doch stolperte ich kürzlich über eine alte Kiste verworfener Notizen meiner Selbst, aus einer Zeit, wo ich noch kein Hohepriester war. Ich weiß, dass diese Texte von mir geschrieben werden mussten, und doch kommt mir so vieles von damals fremd vor. Das Faszinierendste war eine Liste voller Stichwörter, die mein vergangenes Ich mit erinnerungswürdigen Momenten verband, größtenteils fröhlichen oder lustigen. Über die Hälfte dieser Begriffe sagte mir überhaupt nichts mehr. Und es ist erst einige Jahre her ... ich schweife ab, also lasst mich direkt sein: Jarid, seid Ihr Euch sicher, dass Euer Gedächtnis an diese Ereignisse noch Eures ist? Eine Seuche von Gedankenschindern ... nicht auszudenken, was geschähe, wenn der Baum der Lieder kompromittiert wurde.“

„Ich kann mir nicht sicher sein“, sprach Jarid, „Niemand kann das. Doch habe ich definitiv nicht im Sinne dieses Bösen gehandelt.“

„Könnt Ihr euch da sicher sein?“

„Sicherer als Ihr.“

„Dem ist wohl wahr.“

Tion hielt inne und holte Luft: „Last mich eine Geschichte erzählen. Das tu ich gerne.“

Jarid blickte ihn erwartungsvoll an. Tion räusperte sich und sprach salbungsvoll: „Eines Tages, als ich noch weniger graue Haare auf dem Kopf und mehr Muskelmasse an meinen Beinen hatte, schob ich gerade Wache am Rande des grünen Radius, als ein verwundeter Mensch auf mich zutrat. Er sah übel aus, Schnittwunden am Gesicht, eine Schnatte am Bein, keinerlei Besitztümer außer ein halb zerrissener Umhang ... dieser Mann erzählte mir eine haarsträubende Geschichte von verlorenen Schätzen und anstehenden Geburten, warum er dringend in den Norden reisen müsste. Ich verschaffte ihm einen Platz auf einem Handelsschiff. Als ich meinen Oberen vom Vorfall berichtete, war das Schiff schon lange abgefahren. Von Zeit zu Zeit erinnere ich mich ihn und frage mich, wie es ihm wohl geht.“

„Wisst Ihr, ob seine haarsträubende Geschichte stimmte?“

„Sie war ganz dreist gelogen. Beim Menschen handelte es sich um einen Soldaten in den Diensten des Königs, der beim Stehlen entdeckt und von der Burg verbannt worden war. In jener Zeit kam die Verbannung von der Burg einem Todesurteil gleich, doch durch seine kämpferischen Fähigkeiten hatte sich dieser geschickte Schwertkämpfer zu uns durchgeschlagen. Nun, nicht nur durch seine kämpferischen Fähigkeiten, sondern auch durch seine Honigzunge. Dieser Lügner. Als ich dies erfuhr, ging ein wenig von meinem Vertrauen verloren. Und das gewann ich nie wieder.“

Tion blickte Jarid vielsagend argwöhnisch an. Sie wurde immer unwohler.

„Ihr schweiftet wieder ab, also lasst mich nun direkt sein: Ihr könnte mir vertrauen, Meister Tion. Ich versuche nicht, Euch zu hintergehen. Ich bin hier, um Euch zu warnen.“

„Ihr wärt nicht die erste, die einen falschen Alarm auszulösen versucht, damit sich ein Komplize in die Schwarzen Archive schleichen kann“, meinte Tion. Dann aber blickte er mit einem ernsten Gesichtsausdruck zu Jarid hoch und sprach feierlich: „Doch vertraue ich auf meine Menschenkenntnis, und die vertraut auf Euch. Wir werden Ausschau halten nach diesen feindlichen Kriegern. Erst, wenn sie ausbleiben, werden wir Euch vor den Rat der Bewahrer stellen.“

Jarid schluckte schwer.

Gemeinsam blickten die beiden ins Dunkel der Nacht, welches natürlich wie immer nicht völlig dunkel war. Der rote Mond und die weißen Sterne erleuchteten den Wachsamen Wald. Das Sternbild des Hornfalken schwebte da oben, wenn sie sich nicht irrte. Ein gutes Omen. Das mochte Jarid am andorischen Himmel. Über Danwar hingen oft dunkle Dampfwolken, die es unmöglich machten, in den dunkelblauen Kosmos zu spähen. Nur selten konnten die Danware das Firmament erkennen. Erst im Süden war Jarid klar geworden, warum die Bewahrer so viel mehr Sterngucker und Sternzeichenleser hervorgebracht hatten.

Jarids ausschweifende Gedanken, die sich erfreulicherweise einmal nicht um Tri und Rowinda gedreht hatten, kamen zurück ins Hier und Jetzt, an den Baum der Lieder an der Seite des Tion, als ein schneeweißer Rabe angeflattert kam und sich vor ihr auf der Brüstung niederließ.

Jarid hatte diesen Raben erst kürzlich gesehen, auf der Schulter von Feuermeisterin Nidwal, ehe selbige sich dem Bösen ergeben hatte. Der Rabe krächzte aufgeregt und hüpfte auf und ab.

„Versteht Ihr ihn?“, fragte Tion, seine buschigen Augenbrauen erhoben.

Jarid schüttelte ihren Kopf. „Raben zu verstehen, ist ein Talent, das nicht jedem gegeben ist. Aber mir scheint, sie will uns etwas mitteilen.“

„Ist die Armee des Bösen vielleicht bereits hier?“

Jarid spähte ins Dunkel und wünschte sich die leuchtenden Dunkelaugen der Feuerkrieger.

„Nein, das kann nicht sein. Falls sie sehr schnell unterwegs waren, könnten sie vielleicht inzwischen das Ufer erreicht haben, aber das hätten Komu und die anderen an den Teleskopen bestimmt gesehen.“

„Prüft das nach!“, rief Tion und deutete auf die Wendeltreppe im Innern des Baums der Lieder. Jarid sauste davon. Hinter sich hörte sie Tions Fahrstuhl knarzen, als er ihr folgte, gefolgt von einem Bimmeln seiner kleinen Glocke.

Komu war wach und ganz aufgeregt im Ausguck ganz oben im Baum der Lieder. Die zwei anderen Wache haltenden Bewahrer bestätigten, erheblich weniger enthusiastisch, dass keine Schiffe erblickt worden waren. Auch wenn sie wegen der Dunkelheit und den dichten Nebelschwaden über dem Hadrischen Meer zugegebenermaßen nicht sonderlich weit blicken konnten.

Begeistert versuchte Komu, Jarid noch zu erklären, wie das überlange Fernrohr am Aussichtspunkt funktionierte. Es klang wirklich faszinierend. Jarid nahm sich vor, zurückzukehren, sobald die Lage hier geklärt war. Jetzt gab es leider Wichtigeres. Jarid sauste die Wendeltreppe wieder hinunter und berichtete bei Tion, welcher inzwischen von einer weiteren Novizin eine Rampe hochgeschoben wurde. Nidwals weißer Rabe war immer noch dort, flatterte erneut aufgeregt mit den Flügeln und ließ sich dann von der Brüstung fallen.

„Vielleicht will sie, dass ich ihr folge!“

Tion legte seinen Kopf schief und kratzte sich am Bart. Dann sprach er entschlossen: „Dann tut das, Jarid Morgentau. Doch gebt acht. Mit weisen Raben ist nicht zu spaßen.“

Jarid nickte ihm zu eilte die gewundene Treppe hinunter bis aufs Bodenlevel. Sie sauste durch den Torbogen dem Baum der Lieder hinaus. Da flatterte der weiße Rabe ungelenk in der Luft herum. Just bevor Jarid ihn erreicht hätte, raste er ein bisschen tiefer in den Wald hinein.

Jarids Bauch meckerte und ihre alte Wunde platzte wieder auf, doch Jarid rannte weiter, stets knapp auf der Spur von Nidwals weißem Raben. Ihr Bauch mochte protestieren, doch ihr Bauchgefühl versprach ihr, dass, was auch immer der Rabe ihr zeigen wollte, wichtig war. Es sollte recht behalten.

Zunächst rannte Jarid in zwei Skrale hinein, die ihr prompt zwei verrostete Klingen an die Kehle hielten, und Jarid verfluchte ihre untypische Unbedachtheit. Dann aber erklang ein kehliger Befehl aus dem tieferen Dunkel. Die zwei Skrale senkten ihre Waffen und traten zurück.

Jarid riss mit Mühe etwas Wasser aus dem Matsch zu ihren Füßen und formte es zu einem spitzen Eiszapfen in ihren Händen. Wirklich Schaden anrichten würde dieser nicht können, aber vielleicht reichte es, um die Skrale einzuschüchtern.

Über ihr schrie der weiße Rabe warnend auf.

Dann zeigte sich jemand vorsichtig aus dem Schatten zwischen zwei Mammutbäumen.

Zunächst erkannte sie ihn nicht, wirkte er bloß wie einer der übrigen Skrale. Breit, grobschlächtig, leuchtend weiße Augen. Dann aber erkannte sie ihn. Dieses Kinn. Die Art, wie er mitten in einem Schritt vorsichtig stehen blieb. Das Rankenschwert in seiner Hand. Die Rüstung an seinem Körper. Und die Kuhle in seiner Brust, wo früher ein Lavastein gesessen hatte. Ein riesiger Stein fiel von ihrem eigenen Herzen.

„Tri!“, rief sie fröhlich. „Du lebst! Du bist deine Bürde losgeworden! Komm, lass dich ansehen!“

Trieest verblieb im Schatten, von seinem Gesicht neben der Silhouette immer noch nur die nun weiß leuchtenden Kreaturenaugen erkennbar.

„Bei den roten Wogen des Feuers, Tri, ich bin’s! Mutter Natur sei Dank, dir geht es gut! Und du bist hier! Wir können dich dringend benötigen hier. Dich und ... deine neue Skral-Sippe?“, schlussfolgerte sie rasch, „Die restlichen Helden sind alle bei der Gedenkfeier an der Rietburg. Ihr seid unsere letzte Hoffnung.“

„Iarr ... Jarid“, sprach Trieest krächzend, als würden ihm die Silben im Hals stecken bleiben.

Endlich, endlich trat er aus dem Schatten der Bäume ins rötliche Mondlicht. Jarid verspürte gemischte Gefühle bei seinem Anblick. Sie hatte so oft Kreaturen mit Bosheit und Leid in Verbindung gebracht, und so war es durchaus beunruhigend, so viele Züge von Skralen in ihm zu erkennen. Die spitzen Zähne und das schüttere Haar waren alles andere als gewohnt. Doch dahinter steckte unweigerlich immer noch der Tri, den sie kannte und liebte. Der Tri, der nie ihre Seite verlassen würde und ihr schon so oft das Leben gerettet hatte. Der Tri, der genau wusste, welche Worte ihr helfen konnten, wenn sie wütend war, und der gerade nicht wusste, wie er ihr helfen sollte, wenn sie on Trauer erfüllt war. Der Tri, der ihr seit nun sieben Geburtstagen je eine besondere Feder geschenkt hatte, weil er wusste, dass sie sie gerne studierte. Der Tri, dem sie sieben Jahre falsche Hoffnungen gemacht hatte, was seinen Prozess des Wandels anging. Der Tri, den sie sieben Jahre angelogen hatte.

„Du sagtest, du wüsstest, wo der Lavastein mich zu einem Menschen machen würde. Du musst gewusst haben, dass der Lavastein mich nie zu einem Menschen machen würde“, sagte Trieest als Erstes. Keine Frage, aber auch keine Anschuldigung. Eine Feststellung. Vielleicht mit einem Unterton von Verletztseins, aber auch einfach Unverständnis. „Du musst gewusst haben, dass all mein Leiden umsonst war.“

Ja, er war verletzt. Jarid spürte ihre Augen wässerig werden, und beförderte ihre Tränen rasch zurück in die Drüsen, aus denen sie fließen wollten. Jetzt war nicht die Zeit dafür.

„Bist du sicher, dass du das jetzt besprechen willst?“, fragte sie vorsichtig. Tri schien mit den passenden Worten zu kämpfen und antwortete nicht.

„Ich ... ich bin froh, dass du den Lavastein loswurdest“, quetschte Jarid hervor, „Und es tut mir leid. Es tut mir alles so leid. Aber eine Armee von finsteren Feuerkriegern ist auf dem Weg hierher. Ich würde das gerne unsere Priorität machen.“

Tri schnaufte tief durch und nickte dann. Jarid öffnete ihre Arme breit und Tri erwiderte die Umarmung. Fester als gewöhnlich, aber Jarid störte sich nicht daran. Seine Haut fühlte sich härter und rissiger an als zuvor. Das war in Ordnung. Alles war in Ordnung. Tri lebte. Sie hatten einander wieder. Jetzt mussten sie nur noch gegen die Armee des Bösen bestehen. Erneut wallten Tränen in ihr auf und sie bändigte sie hinunter, ehe sie fallen konnten.\bigskip







\textit{Das Böse sinnierte vor sich hin, während seine Schiffe dem Wachsamen Wald näher kamen. Am Horizont sah es hohe Bäume aufragen. Wie an einen entfernten Schatten erinnerte sie sich an die Silhouetten. Dahinter lag es, das Land seiner Hoffnung und seines Hasses. Viele Jahre waren vergangen, seit es erstmals davon gehört hatte. Jahre, in denen es sich großes Wissen angeeignet hatte. Und nun kehrte es zurück, um weitere Geheimnisse zu entdecken. Doch mit dem Horizont näherten sich auch kalte Zweifel: Konnte es hier überhaupt jene Antworten finden, die es suchte?}

\textit{Ratschläge und Einschätzungen dieser Feuermeisterin Nidwal wirbelten durch seine vielen Köpfe. Weise Worte, weise Verständnisse, die es zutiefst erschüttert hatten in ihrer Klarheit.}

\textit{Was konnte es schon bringen, diesem einen Echo einer längst toten Stimme nachzutrauern? Es hatte zuviel Zeit damit verbracht, in die Vergangenheit zu blicken. Dabei war die Zukunft das einzige, worum es sich zu scheren hatte. Wenn es ein Reich regieren wollte, eroberte es es einfach. Zuvor wollte es nur noch sich dieses elenden Gefängnisses entledigen.}

\textit{„Was für eine Verschwendung dies war!“, knurrte es, „Was habe ich mir nur dabei gedacht, nach Danwar zu reisen? Mein Ziel liegt südlicher!“}

\textit{Im Baum der Lieder gab es viele Geheimnisse, einige gar in den Schwarzen Archiven. Vielleicht könnten die ihn aus diesem elenden Kristallsplitter befreien. Und wenn nicht, gab es ja immer noch diesen blinden Seher, der an der Küste lebte.}

\textit{Bald würde es frei sein und die Welt zu seinem Spielball machen. Alle Reiche standen ihm offen. Es könnte zu Varkur zurückkehren und an seiner Seite neue Pläne für eine machtvollere Zukunft entwickeln. Es könnte abwarten bis zur Zeit dieses Zwergenmechanikers und herausfinden, wie er an seine wundersame Sphäre gekommen war – so genau erinnerte es sich leider nicht mehr an dessen Erinnerungen. Es könnte den Urtroll übernehmen und mithilfe dessen Stärke die vermaledeiten Krahder ihrem gerechten Ende zuführen. So viele Möglichkeiten. Aber alles zu seiner Zeit.}

\textit{Das Böse drehte sich seinen finsteren Feuerkriegern zu und sah Jivids Körper dutzendfach in ihren Gesichtsfeldern. Es befahl ihnen im Geiste:}

\textit{„Taucht!“}



































\newpage
\section{Ein Wiedersehen}



\az{Jahr 61}

\textit{Rote Grotte, 61 a.Z.}\bigskip




Das Seil, welches entlang des Höhlengangs gespannt war, endete abrupt. Jarid blieb stehen und öffnete ihre Augen. Rotes Licht blendete sie.

Die Rote Grotte. Eine von Stimmen erfüllte Höhle, mit von Lavastein durchzogenen Wänden. Hin und wieder suchten ausgewählte Bewohner Danwars diesen Ort auf, um den Stimmen zu lauschen und Wissen über ihre Zukunft zu erhalten.

Jarid hatte natürlich gewusst, dass die Rote Grotte rot war, aber nicht, dass sie so rot war. Und so laut. Während sie sich den Gang zur Grotte hin getastet hatte, war das Flüstern der Stimmen der Toten immer lauter geworden. Inzwischen war es ein ohrenbetäubendes unverständliches Wirrwarr aus Gemurmel, Geflüster und Geschrei. Ein kulminierendes Rauschen, welches an das Wogen des Meeres erinnerte, an Jarids Trommelfelle presste und ihren Geist verwirrte.

„Ich bin hier, um meine Prophezeiung zu erfahren“, rief Jarid laut, ohne ihre Stimme zu hören. „Ich werde Danwar verlassen, an der Seite des frischernannten Feuerkriegers Trieest. Was wollt ihr mir mitteilen?“

Die Stimmen flüsterten eine Zeit lang ungestört weiter. So traf es Jarid unerwartet, als sie sich urplötzlich vereinten und wie ein riesiger Chor mit hunderten von Stimmen sprachen:

„Sei vorsichtig mit deinen Wünschen, Jarid Morgentau. Du wünschst dir, durch die Reise fort von hier Gutes zu tun in der Welt und etwas zu erreichen. Das wird dir auch gelingen. Doch wird dein Trieest nie seinen Prozess des Wandels beenden. Er wird nie ein Mensch werden. Nie einer von uns. Doch musst du dafür sorgen, dass er das nicht erfährt. Sollte er es erfahren, wird es seine Hoffnung zerstören. Er würde sich das steinerne Herz aus seiner Brust reißen, und er würde elend verenden. Das darfst du nicht zulassen, Jarid Morgentau. Erhalte seine Hoffnung am Leben, während er mit seiner Bürde kämpft. Wenn die Zeit reif ist, so sage ihm, dass du wissest, dass sein Prozess eines Tages ein erfolgreiches Ende habe. Sage ihm das, und verlängere so sein Dasein. So lange wie möglich.“

Jarid sank zu Boden und versuchte, das Gehörte zu verdauen. Denkbar schwer, wenn weiterhin dutzende von Stimmen an ihr Ohr drangen und leise wisperten:

„Sein Leid hat einen Sinn, Jarid Morgentau. Deine Lügen haben einen Sinn. Das wirst du uns glauben.“

„Warum sagt ihr mir das? Ich hätte ihn auch sonst begleitet und seine Hoffnung aufrecht erhalten! Warum lasst ihr mich lügen, warum lasst ihr denn mich leiden?“

„Dir ein leidloses Leben schenken zu können, liegt nicht in unserer Macht. Auch wir sind nicht perfekt. Doch auch dein Leid hat einen Sinn, Jarid Morgentau. In unseren Augen, und eines Tages auch in den Deinen.“

Dann löste sich der Stimmenchor wieder in einen unverständlichen Chor vielseitigster STimmen auf.

Und Jarid saß da, umgeben von rotem Schein und Lärm, und schluchzte unhörbar vor sich hin.\bigskip







\az{Jahr 68}

\textit{Sieben Jahre später.}\bigskip




Trieest leerte seine Blase. Es fühlte sich anders an als sonst. Weniger beschämend, mehr bestimmend. Markierten Skrale üblicherweise ihr Revier? Besaßen Skrale überhaupt ein Revier? Es gab noch so viel von seinem potentiellen zukünftigen Leben, über das er nicht Bescheid wusste.

Nicht zuletzt hatte er keine Ahnung, wie er mit Jarid verfahren sollte. In eine Skral-Sippe würde sie kaum passen. Doch so etwas hätte Trieest vor einigen Tagen auch von sich selbst behauptet. Wollte sie überhaupt noch etwas mit ihm zu tun haben. Und er mit ihr?

„Schlaf stärkt die Muskeln und die Sinne. Beide werden wichtig sein für die bevorstehende Schlacht“, grunzte Calrai an Trieests Seite.

„Was machst du dann noch hier?“, erwiderte Trieest unsanft.

„Dir helfen. Ich freue mich darüber, dass du uns nicht im Skralreigen stehen lassen hast. Wir sind dir auf deinen Wunsch hier in den Norden gefolgt, und werden an deiner Seite kämpfen. Wenn deine Begleiterin recht hat, steht uns ein unangenehmer Kampf mit mächtigen Gegnern bevor. Ein Häuptling kann als gutes Beispiel vorangehen. Du brauchst Schlaf.“

Trieest blieb stumm.

„Dich beschäftigt etwas“, meinte Calrai, „Die Art, wie du deine Augen verengst, verrät dich. Kannst du etwas tun, um damit es dich nicht mehr beschäftigt?“

„Vermutlich schon.“

„Wirst du es tun?“

„Ich weiß es nicht. Das ist alles sehr neu für mich.“

Tri verabschiedete sich aus Calrais Umklammerung und ging weiter zu Jarid. Sie saß etwas abseits der Sippe an einem Baum und spann an einer Brieflilie. Dem Wunsch der Skrale nach war sie nicht mehr zum Baum der Lieder zurückgekehrt. Jarids und Trieests Versicherungen hin oder her, die Bewahrer waren nie gut auf Skrale nahe ihrer Dörfer zu sprechen.

Trieest hielt sich im Schatten zurück. In dem Augenblick, in dem Jarids Geruch seine Nasenflügel erreichte, machte sich ein unangenehmes Gefühl in seinem Magen breit. Hunger. Hunger auf Fleisch. Durst nach Blut. Er riss sich zurück.

„Warum so zwielichtig, Tri?“, fragte Jarid leise, „Tritt aus dem Schatten.“

Trieest verfluchte erneut das Skralblut in seinen Adern. Wie hatte er in seiner Kindheit diesem Drang nur so lange widerstanden? Oder war es erst die Unterdrückung durch den Lavastein gewesen, die ihn so stark gemacht hatte?

„Grün sind die Wogen der Wellen“, sprach Trieest.

Jarid erwiderte müde: „Und Grün sind die Lichter des Nordens. Beschäftigt dich etwas?“

„Wer weiß schon, was die Zukunft bringt? Wir gehen bald gegen Nekromantie vor. Ich würde diese Angelegenheit lieber vorher hinter uns bringen, sonst lässt sie mich nicht in Ruhe. Hast du Zeit?“

Jarid legte ihren Kopf schief und hörte auf, an ihrer Brieflilie zu spinnen: „Diese Brieflilie wird vermutlich ohnehin zu spät kommen. Was würdest du gerne wissen?“

Tri ließ sich auf den Waldboden plumpsen und sammelte seine Gedanken.

„Ist es wahr, dass du wusstest, dass ich nie ein Mensch werden würde?“

„Das ist wahr. Zumindest glaubte ich den Echos der Roten Grotte, als sie dies behaupteten“, sprach Jarid betont neutral.

Trieest hielt seine Stimme auch betont neutral, als er hervorwürgte: „Dann ist es auch wahr, dass du mich jedes Mal angelogen hast, wenn du mich dazu ansporntest, weiter zu reisen?“

„Das ist leider auch wahr. Tri, ich mochte das genauso wenig wie du jetzt, und ich habe auch stark darunter gelitten.“

„Krarkdreck sind sie, diese Stimmen der Toten! Sie sollen uns anleiten, nicht manipulieren!“ Trieest fasste sich wieder und folgte mit: „Es fühlt sich schrecklich an, zu wissen, dass all unsere Mühen umsonst waren.“

„Ich kann es positiver sehen. Wir haben vielen Personen geholfen, nicht zuletzt uns selbst. Die Echos meinten, dass es dich brechen würde, wenn du vorzeitig erfahren hättest, dass du ...“

Jarid stockte. Dann holte sie tief Luft und sprach mit gesenktem Blick: „Sie meinten, dass du dich umbringen würdest, wenn du dies frühzeitig erführest.“

Da, es war raus. Jarid lehnte sich zurück und holte zitternd Luft.

Trieest machte große Augen und suchte nach den passenden Worten, während sein Gesicht verschiedenste Emotionen durchlief. Zähnefletschend knurrte es dann: „Diese elenden Lügner! Diese hochwohlgeborenen Stimmen der Gefallen, die sich für besser als uns halten! Vermutlich fühlen sie nicht mal was. Und ... und du hast dich immer noch nicht entschuldigt.“

„Verzeih mir! Entschuldige abertausendmal, dass ich dich angelogen habe, Trieest, ich wollte das nicht, aber ich würde es wieder tun.“

„Unsere Vorstellung von Entschuldigungen gehen auseinander.“

„Unsere Vorstellungen der Wahrheit aber nicht. Du hast gelitten, viel und lange und unnötig, das will ich nicht kleinreden, aber mir ging es auch nicht gut dabei! Und wir sitzen beide im selben Boot hier. Wir haben dieselben Ziele. Wir sind auf demselben Stand. Wir können die Vergangenheit Vergangenheit sein lassen und uns auf die Zukunft konzentrieren.“

„Ich mag es nicht, Puppen dieser Echos zu sein! Und ich mag es nicht, dir nicht mehr unbeschränkt vertrauen zu können! Ich mag das alles nicht“, grummelte Trieest mit verschränkten Armen.

„Ich doch auch nicht, Tri. Aber ich glaube, von jemanden zu verlangen, dass diese Person dich niemals unter irgendwelchen Umständen anlügen würde, ist eine unrealistische Erwartung. Immerhin tat ich es zu deinem Besten. Und immerhin tat ich es, während ich mir so gut wie absolut sicher war, dass es einen guten Grund dafür gab. Wer sonst kann schon ... klar, klar, das war nicht zwingend das Geschickteste, was ich jetzt hätte sagen können. Tri, Tri! Bitte geh nicht weg. Tri, schau mich an. Es tut mir leid. Es tut mir aufrichtig leid, ich wünschte, es wäre nicht so gewesen. Was hättest du an meiner Stelle getan? Dir die Wahrheit gesagt, auf dass du dich vielleicht umgebracht hättest? Dieser Lavastein war darauf programmiert, deinen Prozess des Wandels fortzuführen, und wir besaßen nicht die Möglichkeiten, ihn von allein zu entfernen.“

„Pah. Wir hätten auch ohne diesen Lavastein unsere Heldentaten vollbracht.“

„Hätten wir, Tri?“

Trieest schwieg und mahlte mit seinem Unterkiefer.

„Warum würde die Rote Grotte uns das antun?“, fragte er dann, um den Druck von Jarid zu nehmen, „Vielleicht, wenn du mir helfen kannst, irgendeinen Sinn in diesem Quark zu erkennen, kann ich eher meinen Frieden damit finden. Kann es wirklich nur gewesen sein, dass ich diese Bürde tragen musste, um so vielen Personen zu helfen? Steppenländler, Andori, Taren und Tulgori, alles Personen, die den Danwaren und ihren Echos kaum etwas bedeuten?“

„Es könnte auch etwas viel Spezifischeres sein. Wärst du den Lavastein früher losgeworden, hättest du mich zum Beispiel nach dem Kampf gegen den Hraak nicht retten können. Und ...“

Jarid riss ihre Augen auf.

„... und der bösartige Seher hätte uns überlistet! Derjenige, der dich auf die Insel Narkon schicken wollte! Wenn ich die Prophezeiung der Roten Grotte nicht auf den Weg bekommen hätte, dann hätte ich nicht gewusst, dass der Seher dich belog, als er behauptete, dass du auf Narkon deinen Prozess des Wandels beenden würdest! Nur wegen dieser Prophezeiung reisten wir nicht auf dieses verfluchte Stück Fels. Wer weiß, was uns alles zugestoßen wäre!“

„Und das konnten die Echos aus der Roten Grotte dir nicht direkt verraten?!“

„Dann wären wir vielleicht nie an diesen Punkt gekommen.“

„Vielleicht nicht. Ich hasse diese Unklarheit in allen zukunftskennenden Wesen. Diese Macht, die sie über einen haben.“

„Sobald das alles durch ist, lassen wir alle Seher weit hinter uns, Tri.“

Tri schluckte. „Ja, was werden wir tun, wenn das alles hinter uns ist?“

Er ließ die Frage offen im Raum stehen. Jarid hatte sich bislang wohl gezwungen gefühlt, ihm zu folgen und ihm beim Tragen dieser elenden Bürde des Lavasteins zu helfen. Jetzt war sie frei. Würde sie nach Danwar zurückkehren wollen? Für Trieest gab es dort nichts mehr.

„Eines weiß ich sicher, nach Danwar zurückkehren werde ich nicht. Ich war dort, Tri, und es ist keinen Lavadeut besser geworden.“

Kurz wandte sie ihren Blick ab und ballte ihre Faust. Tri kannte diese Geste gut genug, um zu wissen, dass sie gerade etwas unterdrückte. Eine Träne? Kurz haderte er mit sich selbst, doch dann lehnte er sich wieder nach vorne und fragte vorsichtig: „Ist etwas in Danwar vorgefallen?“

„Sie ist tot!“, schluchzte Jarid auf, „Mama ist tot! Wegen mir!“

Das war mehr, als Trieest erwartet hatte. Fast wie automatisch öffnete er tröstend seine Arme und Jarid ließ sich hineinfallen.

„Das ist zugegebenermaßen schrecklich“, stammelte er. Was sollte er auch sagen? Jarid davon zu überzeugen, dass es nicht ihre Schuld war, wie es vermutlich war? Dass sie sich jetzt auf anderes zu konzentrieren hatten? Sie abzulenken? Er brummelte eine alte danwarische Melodie und umarmte Jarid sicher und fest. Auch er konnte kaum seine Tränen zurückhalten. Ein gewaltiges Gewicht löste sich von seinem Herzen, eines, dass lange dort festgesessen hatte.

„Jarid, ich verstehe und vergebe dir.“

Stille. Jarid schniefte.

Dann sprach sie ein neues Thema an: „Danke. Und du, wie ist es dir ergangen? Wirst du ab jetzt ein Skral-Häuptling sein?“

„Einer von ihnen? Einer unter Mördern und Vergewaltigern? Nicht wirklich. Und vielleicht schon. Diese vier Skrale, Calrai, Kurgat, Tran und Bark, die wollen mir folgen. Denken, dass sie an meiner Seite mehr Chance auf Ruhm, Ehre und ein langes, sicheres Leben hätten. Ich stelle sie dir näher vor, sobald dieser Scheißdreck um diesen bösen Kristall hinter uns liegt.“

„Ich will nicht so tun, als hätte ich nicht gewisse Vorurteile, die es schwer machten, diesen Skralen zu vertrauen. Aber ich bin gespannt, ob das unser zukünftiger Pfad sein wird. Weiter weg vor der Zivilisation mit einigen Skralen an unserer Seite.“

„Na, wenn das hier vorbei ist, darfst du dich gerne uns anschließen. Wobei das Leben in einer Skral-Sippe vielleicht nicht das Wahre für dich ist.“

„Wir werden sehen, Tri. Das ist eine Aufgabe für die zukünftigen Wirs. Ich werde auf jeden Fall in deiner Nähe bleiben, wenn das für dich in Ordnung ist. Mit Danwar bin ich fertig.“

„Ich auch. Was diese Ältesten mir angetan haben ... Ich weiß, Träume sind Schäume, aber manchmal sehe ich mein jetziges Gesicht in einer Pfütze und bin glücklich mit dem, was ich erblicke. Wenn der Rat ein wenig offener gewesen wäre ...“

„Oh, Tri ... komm her, lass dich knuddeln, du Großer! Und lass dich nicht unterkriegen von der Welt! Wir packen das schon!“

Wassertropfen perlten an seinen neuen Schuppen ab. Jarid weinte! Das tat sie sonst nie. Trieests drückte sie noch etwas fester. Lange blieben sie so verschlungen. Dann löste sich Jarid langsam.

„So, und jetzt muss ich diese Brieflilie an Jirid absenden.“ Sie wischte sich die letzten Tränen vom Gesicht und lachte kurz auf, „Wobei das wohl nicht viel nützen wird. Selbst wenn die Helden die Nachricht gleich kriegen und sie noch wach und fit sind, dauert es über einen Tag, bis sie endlich hier wären. Und nach der Gefallenenzeremonie ist der große Brunnen neben der Rietburg bestimmt völlig geleert. Demnach könnte nicht einmal Jirid hierher tunneln. Ich befürchte, wenn das Böse mit seiner Horde aus kontrollierten Feuerkriegern hier eintrifft, werden keine weiteren Helden hier sein, um es zu erwarten.“

„Nicht alle Helden von Andor verbringen ihre Zeit mit Feiern“, erklang ein Flüstern aus dem Gebüsch hinter Trieest. Dieser zuckte zusammen, als ein dunkel gewandeter Bursche aus dem Baum trat und sich neben sie stellte. Ein schwarzer Köcher aus Wardrakleder hing an seiner Seite, die dazugehörige Waffe war wohl unter seinem Umhang verborgen.

Ein Held von Andor, wenn auch einer, der lieber Zeit im Schatten als im Licht verbrachte. Trieest hatte erst selten mit ihm gesprochen, doch hatte er sein Kampfgeschick bereits einige Male bewundert.

Jarid fasste sich als erste wieder: „Arbon! Wie lange versteckst du dich schon hier?“

„Keine Sorge, ich habe bloß das Ende eurer melodramatischen Gespräche mitbekommen. Ha! Da lohnte sich es doch, dem Totenfest fernzubleiben! Das kriegt Fenn so was von unter die Nase gerieben ... Ich meine natürlich: Ich hörte, eine große Gefahr kommt auf uns zu, und ich, Hogo, der tapfere Andori, werde mich auf die Seite des Guten stellen!“

„‚Hogo‘, soso? Hast du dich immer noch nicht mit Melkart getroffen und eine Begnadigung ausgehandelt?“, fragte Jarid interessiert, „Wenn eines Tages eine Steintafel von deinen Heldentaten berichten soll, soll sie dann etwa vom tapferen Hogo aus dem Rietland erzählen?“

„Wenn irgendwann meine Taten niedergeschrieben werden, so werde ich diese große Aufgabe sicherlich selbst in die Hände nehmen“, sprach Arbon in seinem besten überheblichen Bewahrer-Ton, „Und aushändigen werde ich sie an die besten Barden diesseits des Fahlen Gebirges, auf dass jene sie in allen Weiten des Landes erschallen lassen. Der ehemalige Königsbarde Grenolin. Oder die gute Gilda aus ihrem gemütlichen Gasthaus. Oder der Spielmann Pasco in Bechtholds Diensten. Welcher Name mir von diesen dann am Ende zugesprochen wird, kann mir völlig egal sein. Ich tu das ja nicht Ruhm und Ehre wegen. Meine Blutsfamilie sind neben Reka und einigen Schwarzen Wachen doch die einzigen, die mit ‚Arbon‘ etwas anfangen könnten, und denen schulde ich nichts. Aber ich komme ja ganz ins Plappern, und um mich geht’s hier ja ganz und gar nicht. Was nun, was tun?“

Jarid und Trieest blickten sich amüsiert an. Jarid meinte schließlich:

„Wir warten, bis die Schiffe des Bösen am Horizont auftauchen. Dann rüsten wir uns für einen Kampf. Unser Ziel muss sein, den schwarzen Kristallsplitter zu vernichten, von dem aus das Böse seine Feuerkrieger kontrolliert.“

„Würde dieses Böse den Kristallsplitter nicht lieber anderswo sicher aufbewahren, sodass wir ihn nicht erreichen können?“

„Es ist nicht nur böse, sondern auch hochmutig. Es wird so unvorsichtig sein.“

Während Trieest Arbon auf den neusten Stand brachte, beendete Jarid den finalen Schliff an ihrer Brieflilie und sandte sie gen Westen. Die Lilie verengte sich zu einem kleinen, blau leuchtenden Körnchen und segelte durch die Luft, stetig ihrem Ziel entgegen.

„Was haben diese kontrollierten Feuerkrieger für Fähigkeiten? Können sie mit ihren geschwärzten Lavasteinen auch andere Wesen beeinflussen?“, fragte Arbon gerade nachdenklich.

Trieest zuckte mit den Schultern. Jarid meinte:

„Vermutlich nicht. Sonst hätten sie diese Fähigkeit schon in Danwar eingesetzt. Doch auch so sind sie gefürchtete Gegner. Das sind keine Halunken, sondern voll ausgebildete Feuerkrieger von Danwar. Dieser Orden ist für seine Kampfstärke bekannt.“

„Ist er das?“, fragte Arbon, „In diesen Landen hier hat man bislang vor allem von Trieest gehört, und während sicherlich keiner sein Kampfgeschick in Frage stellen will, ist er wohl keineswegs ein typischer Krieger Danwars.“

Trieest wandte sich ab. Solche Gespräche machten ihn immer nervös. Warten war noch nie seine Stärke gewesen.

Jarid meldete indes an: „Schadet den Feuerkriegern nicht zu viel. Das sind immer noch die Körper guter Menschen, die bloß besessen wurden.“

„Bist du dir sicher, dass man ihren Geist wieder in den Körper zurückführen kann, sobald die Verbindung zum schwarzen Kristallsplitter gebrochen wurde?“

„Ihr Geist hat den Körper nie verlassen!“

„Sagst du. Wie lange hat das Böse Trieest kontrolliert? Zwei Minuten? Mit Verlaub ...“

„Selbst wenn man ihnen schadet, wird das nicht viel helfen“, mischte sich Trieest wieder ein. „Das Böse ist ein Nekromant. Es konnte Lysbetts Körper weit nach ihrem Tod kontrollieren. Es vermag, selbst tödliche Wunden in von ihm kontrollierten Personen zu ignorieren.“

Arbon machte große Augen: „Sie sind auch noch unsterblich?! Brauchen wir Drachenrelikte, um ihnen zu schaden?“

„Ich konnte die Kontrollierten auch so bekämpfen.“

„Oh nein“, entfuhr es Trieest.

„Was?!“, fragten Jarid und Arbon im Gleichklang.

„Wenn das Böse die kontrollierten Körper irgendwie magisch vor dem Tode bewahren kann ... was spricht dann dagegen, dass es seine Schiffe außerhalb des Ufers im Nebel ankern lässt und mit seinen Kriegern unter dem Wasser hierher spaziert? Untote können nicht ertrinken. Vielleicht schlichten sie weit unter dem Meeresspiegel in den Wachsamen Wald. Wir hätten nicht nur nach Booten Ausschau halten sollen.“

Arbon begriff als Erster, zückte seine Arcuballiste und rannte in Richtung des Baums der Lieder. Jarid stieß noch ein entsetztes „Vielleicht sind sie bereits hier!“ hervor, ehe sie Arbon nachrannte. Trieest holte tief Luft und stieß einen beeindruckenden – wenn auch noch etwas unsauberen – Ruf der Skrale aus, um Calrai und die übrigen Skrale zu alarmieren. Dann setzte er Jarid und Arbon nach.\bigskip







Der Baum der Lieder lag ruhig da, das rötliche Mondlicht glitzerte auf seinen silbernen Blättern. Das Dorf war stumm. Nur die Tatsache, dass keine Wachposten mehr das Eingangstor des Baumes bewachten, störte das idyllische Bild ein wenig. Arbon, Jarid und Trieest hetzten mit gezückten Waffen die gewundene Wendeltreppe hoch – niemand.

Auf den Balkon hinaus – niemand.

Um die halbe Balustrade herum – niemand.

Die Raumflucht betreten – da war jemand! Vor der zweiten Tür links beugten sich zwei spitzohrige Gestalten über eine dritte Person, welche verzweifelt versuchte, davonzukrabbeln. Dem neben ihr liegenden Fahrstuhl nach war das wohl Hohepriester Tion.

„Tion hat Zugang zu den Schwarzen Archiven!“, flüsterte Arbon, „Sie wollen irgendwelche Geheimnisse erlangen. Doch keine Sorge, Tion ist ein äußerst verpflichteter Bewahrer, er wird ihnen nichts verraten.“

Wie um Arbons Worte Lüge zu strafen, nickte der zusammengesunkene Tion ergeben. Zitternd wurde er in die Höhe gehoben, während das Rankenschwert des einen finsteren Feuerkriegers stets auf seinen Hals gerichtet blieb. Und mit bebender Stimme rief Tion: „Schwarze Wachen! Öffnet die Tore!“

„Als ob!“, grummelte Arbon und zückte seine Balliste. Zwei Schüsse später fiel Tion wieder zu Boden, neben ihm zwei danwarische Leichen. Diese blieben jedoch nicht lange tot. Und sie blieben nicht lange allein.

Aus dem Dunkeln trabten weitere Feuerkrieger und stürzten sich auf die Helden. Tion kroch ächzend durch das geöffnete Tor ins Innere der Schwarzen Archive. Sobald er in Sicherheit war, verriegelten die Schwarzen Wachen die Tür von außen wieder und zückten Messer.

Jarid spreizte ihre Hände. Es knirschte und knarzte, dann brachen riesige Wassermassen aus den Löschfässern, welche oben an verschiedenen Ästen des Baums der Lieder gehangen hatten. Mit geschlossenen Augen lenkte Jarid das Wasser durch den Raum. Die noch stehenden Feuerkrieger wurden von den Beinen gehauen und in die Tiefe gerissen, während sich das Wasser galant um unsere Helden und Wachen teilte. Leider wurden auch einige Dokumente und Chroniken mitgeschleppt.

„Hui, wo bleibt deine Rücksicht auf die Feuerkrieger?“, fragte Arbon schnippisch, „Und wo ist der schwarze Kristallsplitter?“

Jarid antwortete nicht, sondern setzte sich einen umherklappernden Helm auf, den eine Feuerkriegerin verloren hatte. Sie stählte sich für ihren nächsten magischen Akt.

Kampfeslärm drang von unten herauf. Skrale und Feuerkrieger hatten sich in ein Gefecht verstrickt.

„Calrai!“, rief Trieest und rannte die Treppe wieder hinunter.

„Fang mir einen von ihnen! Ich brauche einen Lavastein!“, rief Arbon ihm hinterher. Trieest hoffte ganz fest, dass es sich um einen Plan gegen diese Wesen handelte, statt um einen innigen Sammlerwunsch, der gerade im ehemaligen Bewahrer aufgekommen war.

Kaum war er aus dem unteren Tor am Baum der Lieder herausgerannt, stürzte sich auch schon ein Feuerkrieger auf ihn. Orange-rote Feuerschlieren umringten ihn und gruben sich in Trieests Rüstung. Dieser packte den Angreifer mit beiden Händen und warf ihn auf den Waldboden, welcher vom Auftreffen der Löschfässer noch tropfnass war. Selbiges Wasser kroch prompt am Feuerkrieger hervor und drückte seinen Körper tiefer in den Matsch. Jarid war ihm gefolgt!

Der finstere Feuerkrieger wand sich am Boden, doch sein Blick war leer. Das Böse fokussierte sich wohl gerade nicht auf ihn.

„Danke sehr für das Testsubjekt!“, rief Arbon fröhlich, ehe er dem strampelnden Feuerkrieger auf die Brust stand und den dunklen Lavastein in dessen Brust beäugte. Er zog einen kleinen grünen Runenstein aus seiner Manteltasche und hielt ihn dagegen. Das leise Summen des Runensteins wurde ein wenig stärker, sonst geschah aber nichts. Arbon schüttelte seinen Kopf, zog einen roten Stein aus seinem Umhang und brachte diesen näher zum Lavastein des Feuerkriegers. Diesmal schien eine anziehende Kraft vom Lavastein auszugehen, denn der rote Kristall flutschte aus Arbons behandschuhten Händen und knallte klirrend gegen den schwarzen Lavastein, woraufhin sich die Schwärze prompt in den roten Kristall ergoss.

„Drachenmagie, also?“, murmelte Arbon verwirrt, „Das sollte mit Untoten eigentlich überhaupt nicht harmonieren.“

„Häuptling, Obacht!“, erklang Kurgats tiefe Stimme von hinten. Trieest konnte sich gerade noch rechtzeitig umdrehen, um eine anstürmende finstere Feuerkriegerin zu sehen. Sie sah vom Sturz vom Baum noch ziemlich lädiert aus. Einige Knochen waren bestimmt gebrochen, doch nicht ihr Kampfeswille. Trieest zückte sein Rankenschwert und brachte die Feuerkriegerin zunächst zu Fall, dann zu völliger Bewegungsunfähigkeit. So viel zur Kampfstärke des Ordens, zu Strategien und Formationen. Das Böse musste seine Mühe damit haben, so viele Puppen auf einmal zu steuern, und ließ sie eigenständig wilde Angriffe machen, die leicht zu überwältigen waren. Worauf fokussierte es sich, was konnte wichtiger sein als diese Schlacht hier?

Trieest musste zwei weitere anstürmende Feuerkrieger in ihre Plätze verweisen, ehe er sich zu Arbon zurückdrehen konnte. Dieser war gerade daran, Jarid zu instruieren:

„... These besteht darin, dass dieses Böse mithilfe von Drachenmagie in seinem Kristallsplitter eingeschlossen wurde, und diesen Käfig auf andere Kristalle ausdehnen kann, solange er den Kontakt zur gelenkten Person nicht verliert. Letzteres würde die Verbindung zerstören. Es macht die Lavasteine der Feuerkrieger damit quasi zu Drachenrelikten und seine Untote nicht mehr unverwundbar. Das ist wirklich faszinierend. Nein, da rüber. Lenk den Wasserstrahl in die Fugen meines Relikts. Ja, genau so. Wenn diese Kristalle irgendwie über seinen Geist verbunden sind, dann sollten wir in der Lage sein, diese Verbindung zu verfolgen und herauszufinden, wo es seinen Hauptkörper mit dem Kristall versteckt. Ja, jetzt verbinden. Gut machst du das!“

Arbon streute eine Prise eines seltsamen pink glitzernden Pulvers aus einer Manteltasche über seinen roten Stein, welcher mit Jarids Hilfe von einer dünnen Wasserschicht überzogen war und durch eine wässerigen Doppelhelix Kontakt mit dem Lavastein des Feuerkriegers am Boden hatte. Hin und wieder flackerte der rote Stein tiefschwarz auf.

„Komm her, Trieest! Du hast als einziger von uns bislang einen Kontakt mit einem Lavastein aufgebaut, und die Wassermagie sollte dich vor dem Einfluss seines Geistes schützen. Berühre mein Drachenrelikt und sage mir, was du siehst!“

Trieest war niemand, der sich lange mit Diskussionen über das für und wider solch seltsamer Anweisungen aufhielt. Wenn jemand anderes einen Plan hatte, war dieser für gewöhnlich nicht zu missachten, und Trieest hatte ohnehin nicht den Durchblick, den es für Kritik brauchte. So stürzte er rasch zum immer noch im Boden festgehaltenen Feuerkrieger und griff fest nach dem Drachenrelikt. Er spürte kurz die Nässe von Jarids Wasser, dann die Härte des überraschend warmen Steines. Dann wurde alles dunkel.\bigskip



\textit{Es war ein seltsames Gefühl, als würde Trieest durch ein dutzend Augenpaare gleichzeitig gucken und ihre Gesichtsfelder überlappend sehen. Da waren ein, zwei Feuerkrieger, die gerade in ein Gefecht mit seinen Skralen verwickelt waren. Sie sahen sich gegenseitig. Er sah Calrai, dessen zeremonieller Speer sich in sein Gesichtsfeld bohrte. Zweifelsohne hatte Calrai soeben einen finsteren Feuerkrieger erledigt.}

\textit{Trieest blickte in sein eigenes verzerrtes Gesicht, das mit geschlossenen Augen über ihm schwebte, seine eigene Hand, das Drachenrelikt in seiner großen Faust fast verschwindend.}

\textit{Kein Zweifel: Er sah das, was das Böse sah. Und er hörte das, was das Böse hörte, so viele überlappende Stimmen, konnten die alle von ihm sein? Hörte er auch seine Gedanken? Wo war der Hauptkörper? Trieest suchte die verschiedenen Gesichtsfelder ab, aber sie alle zeigten bloß den Baum der Lieder, den andorischen Himmel, pure Schwärze und Matsch ...}

\textit{Da!}

\textit{Eine dunkle Hütte, winzige Fenster, kaum erleuchtet. Eine hochgewachsene Gestalt in einem eleganten dunklen Mantel, welche am Boden kniete. Und die Hand einer kleinen Feuerkriegerin, welche einen schwarzen Kristallsplitter von der Stirn der Gestalt nahm.}

\textit{Ein verzweifeltes grelles Lachen einer hohen Stimme, die Trieest nicht kannte: „Ich bin zu früh! Es wäre auch zu schön gewesen ...“}

\textit{Die Gestalt am Boden blickte nicht auf, als sie murmelte: „Wer ... was bist du? Was willst du von mir?“}

\textit{Das Gesichtsfeld drehte sich leicht, als das Böse den Kopf seiner Puppe in den Nacken legte und wütend knurrte. Dann sprach es: „Ich will frei sein, Leander! Wenn die Schwarzen Archive mir nicht weiterhelfen, dann werde ich es halt aus dir erfahren. Überlege gut, denn dein Leben hängt davon ab. Wie würdest du eine Seele aus einem mit Drachenmagie versiegelten Kristall befreien?!“}

\textit{„Das weiß ich auf die Schnelle kaum, Drachenmagie ist nicht meine Stärke“, versuchte dieser Leander sein Glück.}

\textit{„Das weiß ich, sonst wüsste ich die Antwort bereits, als ich in dir war“, grummelte das Böse, „Doch nun müssen wir es auf die altmodische Art machen. Du kennst deine Erinnerungen am besten, alter Mann, wo würdest du zuerst nachforschen? Irgendetwas musst du doch bereits wissen, lange wird es nicht mehr dauern!“}

\textit{Leanders hilfloses Gelaber wurde zu einem leisen, unverständlichen Gemurmel. Das Böse hörte schon nicht mehr hin. Stattdessen zeigten nun mehrere Gesichtsfelder den knienden Trieest aus verschiedenen Blickwinkeln.}

\textit{„Trieest, was machst du denn hier in meinem Kopf?!“, antwortete das Böse fast fröhlich.}\bigskip



Trieest stolperte zurück und öffnete seine Augen.

Jarid, Arbon und einige verletzte Feuerkrieger um ihn herum starrten ihn an. „Und was ist mit deinem Lavastein geschehen?“, sprachen die Krieger in einem verwirrenden Chor. „Zu schade, der hätte mich eigentlich interessiert.“

Trieest ignorierte sie. „Das Böse befindet sich in einer Hütte. Es hat eine Gestalt ausgefragt. Es will von ihr erfahren, wie es aus dem Kristall befreit werden kann. Auch wenn ... irgendetwas ... noch zu früh sei. Ihr Name ist ... Leander?“

„Der alte Leander!“, rief Arbon, „Er lebt in einer Hütte nahe des Meeres. Er konnte mir schon einige Male aushelfen, und er scheint stets an uraltem Wissen interessiert. Es würde Sinn ergeben, wenn das Böse bei ihm nach einem Ausweg aus seinem Gefängnis suchte.“

„Zumal es so schnell nicht ans Wissen der Schwarzen Archive kommen wird“, meinte Jarid mit einem zufriedenen Blick auf die Lage um den Baum der Lieder. Die finsteren Feuerkrieger kämpften tapfer, doch hatten viele von ihnen schon gebrochene Beine vom Sturz vom Balkon, und auch wenn die Untoten scheinbar weder Blut noch Muskeln zum Bestehen und Bewegen benutzten, schien ihre Integrität stark davon abzuhängen, dass ihre Knochen heile waren. Die meisten Feuerkrieger waren bereits zusammengeklappt und von Skralen, Bewahrern oder Schwarzen Wachen entwaffnet worden. Calrai fesselte soeben die strampelnde Feuermeisterin Nidwal mit einigen Lumpen.

Der Lärm hatte zahlreiche Dorfbewohner geweckt, und auch wenn die meisten grau gewandten Bewahrer bloß mit großen Augen dem Geschehen zuguckten, waren inzwischen schon einige grün gewandte Bogenschützen auf die Lichtung getreten. Sie schienen sich nicht so sicher zu sein, ob sie nun die Feuerkrieger, die Skrale, Trieest oder etwa Arbon anzielen sollten. Hohepriester Tion meldete sich von oberhalb des Balkons und versuchte, die Situation zu entschärfen.

Die Skrale zogen sich auf einen Wink Calrais ins Unterholz zurück. Arbon schien plötzlich ebenfalls wie vom Erdboden verschluckt. Noch während Trieest sich nach ihm umsah, winkte Jarid in eine bestimmte Richtung.

Trieest rannte von dannen. Hinter ihm schlug ein grün gefierter Pfeil in einem Baumstamm und jemand rief „Obacht, Skrale!“ Sie konnten Idioten sein, diese Bewahrer. Immerhin hatten sie die Lage am Baum der Lieder mehr oder minder unter Kontrolle. Eher minder, wenn er das plötzliche Gebrüll weiterer finsterer Feuerkrieger von der Lichtung her richtig interpretierte. Aber der schwarze Kristall hatte Priorität. Und der befand sich Arbons Worten zufolge vor ihnen, in Richtung Norden.\bigskip







Trieest rannte Jarid hinterher. Diese Tätigkeit fühlte sich zugleich altbekannt und doch wie eine neue Erfahrung an. Sei letztes Lauftraining mit Jarid war noch nicht einmal so lange her, kaum einige Tage vor ihrem Zusammenstoß mit dem wilden Hraak. Und doch schien es so anders. Keine Muskeln waren geschwächt. Kein Lavastein brannte in seiner Brust. Seine Füße und Beine fühlten sich kräftig an, wie sie über den Waldboden trommelten, und seine Brust vermochte plötzlich, so viel mehr Luft einzuatmen.

Nicht, dass alles an seinem neuen Dasein so positiv gewesen wäre. Skrale schienen kaum zu schwitzen, und als Halbskral blieb einem Großteil von Trieests neu schuppiger Haut die nötigen Poren auch verwehrt. So überholte Trieest Jarid großspurig, nur um kurz darauf überrascht stehen bleiben zu müssen und nach Luft zu hecheln. Von da an ließ er es langsamer angehen.

Ein verschnörkelter Weg führte vom Baum der Lieder in den Norden. Irgendwo da oben musste ein Handelshafen liegen, doch dieser war nicht ihr Ziel. Jarid deutete nach links, um Trieest auf einige zerbrochene Äste aufmerksam zu machen. Jemand war hier ins Unterholz abgehauen. Arbon hatte ihnen eine Spur hinterlassen!

Kurz darauf trafen Jarid und Trieest auch auf Arbon selbst, wie er in einem Gestrüpp kniete und sich sein Knie hielt. Der Übeltäter war klar: Eine tückische Wurzel ragte hinter ihm aus dem Boden. Jarid und Trieest sollten nie erfahren, ob dies bloß ein unglücklicher Unfall oder der Unbill eines der zahlreichen launischen Waldgeister in dieser Gegend gewesen waren, doch nach einer kurzen Sicherstellung, dass Arbon nicht ernsthaft verletzt war, rannten sie ohne ihn weiter, dem Weg entlang zur Hütte dieses Leanders entgegen.\bigskip







Das stürmische Hadrische Meer war schon von Weitem zu hören. Trieest öffnete seinen Mund und spürte Meeressalz auf seiner Zunge. Sie waren nahe.

Da vorne öffnete sich das Unterholz und gab einen Blick frei auf den Waldrand. Grünes Gras säumte ein lauschiges kleines Plätzchen, und im Waldrand war ein besser ausgebauter Weg zu erkennen, der vermutlich über einen Umweg Leanders Hütter mit dem Rest der Zivilisation verband.

Sie waren angekommen.

Laut brandete die Gischt des Hadrischen Meeres an die Küste des Wachsamen Waldes. Eine kleine Hütte stand grau und unscheinbar am Waldrand. Im grauen Licht des ersten Morgens wirkten die letzten Baumstämme des Waldrandes schwarz wie die Stäbe eines Käfigs.

Jarid und Trieest mussten nicht nachprüfen, ob der Besitzer der Hütte zuhause war. Denn dieser stolperte soeben schreiend von der Hütte weg, eine amüsiert grinsende kleine Wassermagierin auf seinen Fersen. Die Wassermagierin erblickte Jarid und Trieest als erste, verengte ihre Augen und ballte ihre Fäuste, woraufhin eine gewaltige Welle kalten Salzwassers vom Meer ins Land spülte und diesen ... Leander? ... von den Füßen holte und seine Kapuze beiseite wischte.

Die Wassermagierin stellte sich demonstrativ zwischen Leander und die Neuankömmlinge, doch diese konzentrierten sich nicht besonders auf sie. Jarid blieb als Erste überrascht stehen. Trieest rannte einige Schritte weiter, doch da hob Leander seinen Kopf und fragte: „Wer da?! Schnell, holt Hilfe! Kommt nicht näher! Sie darf euch nicht berühren!“

Und da erkannte auch Trieest ihn. Dieses Gesicht hatte er schon Jahre nicht mehr gesehen, und doch würde er es so bald nicht vergessen. Lügen und weitere Lügen hatte es ihm aufgetischt, auf dass er auf seinen Zielen folgen würde. Einzig Jarids Eingreifen hatte ihn damals bewahrt. Jarid, die nur aufgrund der Stimmen aus der Roten Grotte diese Lügen als solche hatte erkennen können.

Es war der blinde Seher aus der Taverne.

„Du!“, rief Jarid.

Leanders Gesichtsausdruck war unter seiner Kapuze nur schwer zu erkennen, doch schien er sich noch mehr zu versteifen und verhaspelte sich einige Male, bis er krächzend erwidern konnte: „Grün sind die Wogen der Wellen, Jarid Morgentau. Dann ist Trieest vermutlich auch in der Nähe? Bitte, seid so lieb und helft mir gegen dieses Ungetüm.“

„Du bist doch bekanntlich ein Seher, weißt du denn nicht, wie das hier ausgeht?!“, fragte Trieest bissig, ohne die Begrüßung zu erwidern. Dann biss er sich selbst auf die Zunge und nickte Jarid zu. Die beiden teilten sich und schritten in weitem Umkreis von links und von rechts auf die kleine Wassermagierin und den Seher zu.

Leander setzte zu einer Antwort an: „Grün sind auch deine Wogen der Wellen, Trieest. Du wurdest vom grantigen Kord verwöhnt. Ich bin ein Seher, aber ich sehe längst nicht alles. Unlängst konnte ich deine magisch-mächtige Präsenz gar mit meinem inneren Auge spüren, doch nicht einmal das sehe ich nun noch. Was ist mit dir nur geschehen, hast du deinen Lavastein verloren?“

Trieest ignorierte ihn und schritt noch näher.

„Keinen Schritt näher, oder ich entledige mich des Blinden!“, rief das Böse aus dem Mund der kleinen Wassermagierin.

„Du dir keinen Zwang an, an ihm liegt mir nichts. Doch benötigst du ihn nicht, um deinem kristallenen Gefängnis zu entkommen?“, erwiderte Jarid bissig und schritt unbeirrt weiter.

„Ich kann allen Anwesenden hier sehr nützlich sein, wenn ihr mich bloß am Leben lasst“, mahnte Leander mit hoher Stimme.

Das Böse hingegen lachte bloß verzweifelt: „Ich werde dich nicht umbringen. Ich will es nicht. Und ich kann gar nicht. Jarid! Trieest! Wollt ihr nicht wissen, warum dieser Seher euch auf eine Selbstmordmission senden wollte? All seine Geheimnisse können die meinen sein ... und damit auch die euren, wenn ihr es euch nicht mit mir verscherzt.“

Trieest knackte seine Fingerknöchel und sein Rankenschwert vereinte sich mit einem melodischen Klang zu einer einzelnen schnurgeraden Klinge.

„Ich sehe, da ist jemand nicht zum Scherzen aufgelegt, häh? Und du, Jarid? Immer noch wütend wegen deiner Mutter? Ich verstehe es. Ich habe auch lange, viel zu lange emotional an meiner Familie gehängt. Aber sie ist unwichtig. Eines Tages wirst auch du das akzeptieren müssen.“

Jarids Schritte wurden kaum merklich kürzer. Trieests nahmen an Tempo zu.

„Ihr beide könnt mich wirklich nicht in Ruhe lassen, oder? Ach, die Karten sind wirklich gegen mich gerichtet. Gönnt ihr mir etwa meine Freiheit nicht? Ich bin mächtig und werde die Welt mithilfe meiner Macht zum Guten wenden! So lasst uns doch reden ...“

Inzwischen hatten Jarid und Trieest das Böse fast erreicht. Wo versteckte es den schwarzen Kristallsplitter?

Das Böse knurrte aus dem Mund der kleinen Wassermagierin: „Könnt‘s kaum erwarten, mich abzustechen, häh? Kann ich verstehen. Früher war ich geduldiger, doch all die lange Zeit in meinen vielen Kerkern und Gefängnissen, echten wie magischen, machte mich ebenfalls ungeduldig! Also dann, bringen wir’s hinter uns!“

Die kleine Wassermagierin stieß Leander von sich und stürzte sich auf Jarid. Trieest beschleunigte seinen Schritt, Jarid riss ihre Hand nach oben. Ein kleiner Ball aus Wasser bildete sich scheinbar aus dem Nichts, folgte ihr und zerplatzte am Arm der kleinen Wassermagierin. Diesen Trick kannte Trieest bereits. Unterkühltes Wasser. Es gefror sofort und hinterließ üble Verbrennungen.

Das Böse schrie auf, doch jetzt war Trieest bei ihm und warf es zu Boden.

Leander kroch davon.

Es raschelte im Unterholz und eine eigenartige Prozession trat nach vorne. Calrai und die restlichen Skrale waren da, um ihnen zu Hilfe zu kommen! Kurgat starrte besorgt auf das Rauschen der Wellen am Meeresufer und hielt sich zurück. Tran und Bark trugen Arbon auf ihren Schultern herbei, welcher eine behelfsmäßige Schiene um sein Bein trug. Und eine gespannte Arcuballiste auf die kleine Wassermagierin richtete. Sie hob ihre Augenbrauen.

Arbon betätigte seine Balliste in genau dem Moment, in dem die Wassermagierin ihre Hand in seine Richtung öffnete. Ein Bolzen schoss aus der Arcuballiste, ein schwarzer Kristall schnellte aus der Hand der Wassermagierin. Die beiden Geschoße bewegten sich blitzschnell, doch schien es Trieest, als würde die Zeit stillstehen und er könnte ihren schönen Flug mitverfolgen. Sie kreuzten sich in der Mitte, so nahe beieinander, dass zwischen ihnen ein Haar hätte zerrieben werden können. Dann bohrte sich der Armbrustbolzen in die Stirn der kleinen Wassermagierin und der schwarze Kristallsplitter in Arbons Stirn.

Trieest hätte so einiges erwarten können, aber sicherlich nicht, dass das Böse aufschrie. Arbon schwankte, fiel zu Boden und stieß einen kehligen Fluch aus, gefolgt von einem ungläubigen „WAS?! Aber ... das geht doch ... Nein! NEIN!“ Das Böse riss irgendein relevantes Teil aus der Arcuballiste, schleuderte sie zu Boden und warf den schwarzen Kristall an Calrai, welcher verwirrt nebendran stand und ihn instinktiv auffing. Trieest stöhnte auf.

Calrai wiederum fletschte seine Zähne, während seine weißen Augen sich schwarz verfärbten, und gab Fersengold.

Oder besser gesagt, er versuchte es – doch Arbon wetzte herum und langte nach Calrais echsiger Ferse. Calrai klappte zusammen, und da waren Jarid und Trieest schon bei ihm. Calrai sprach aus schwarz geifernden Lippen irgendeinen Fluch in einer Sprache, welche längst hätte vergessen werden sollen.

„Gebt acht, ihr wollt doch nicht Trieests neue Fl ...“, setzte das Böse an.

Arbon unterbrach es, hob seine angeschlagene Arcuballiste demonstrativ in die Höhe und verzerrte sein Gesicht zu einem gehässigen Grinsen: „Das hätte ich an deiner Stelle nicht getan.“

Calrais Augen wurden zu dünnen Schlitzen. Dann breitete das Böse seine Arme aus, heulte den Himmel an und sprach in der Skral-Sprache: „Geister des Feuers, Geister der Erde, ich rufe euch an, auf dass ...“

Trieest wusste genug über Calrais Schamanen-Talent, damit er sich zu fürchten begann. Nebelschwaden zogen wie aus dem Nichts auf und ballten sich um Calrai herum.

Auch Arbon schien plötzlich genug von den Spielchen zu haben. Er richtete sich zu seiner vollen Größe auf, griff in seine Manteltasche und zog etwas hervor. Er öffnete seine Faust und präsentierte... \textit{drei kleine, grau gefärbte Holzwürfel.}\bigskip



\textit{„Wir haben insgesaaaamt ... Jarids sieben Stärkepunkte plus Arbons zwölf sind 19. Plus Trieest mit seinen sieben Stärkepunkten ergibt 26!}

\textit{„Und Jarid beginnt!“}

\textit{„Ich würfle ... eine Eins! Eine Zwei! Und ... nochmals eine Eins! Sorry, Leute, das war nix.“}

\textit{„Nehmen wir da die zwei Einsen oder doch die eine Zwei?“}

\textit{„Die zwei Einsen! Wollen ihren Helm doch nicht verkommen lassen.“}

\textit{„Wir sind bei 28!“}

\textit{„Arbon ist dran!}

\textit{„Na, dann schauen wir mal. Vier, das nehmen wir noch nicht. Eins, das wollen wir ganz und gar nicht.“}

\textit{„Das schaut nicht gut aus.“}

\textit{„Jetzt aber! Sechs, das lassen wir doch sehr gerne liegen!“}

\textit{„Schön gemacht!“}

\textit{„34!“}

\textit{„Hip, Hip, Hurra! Earas Rang ist da!“}

\textit{„Das ist schon über Finster-Calrais Stärkepunkten! Wir könnten tatsächlich eine Chance haben!“}

\textit{„Aber hallo, hat jemand je daran gezweifelt?!“}

\textit{„Jetzt noch Trieest.“}

\textit{„Die Spannung steigt ...“}

\textit{„Und roll!“}

\textit{~ Klacker ~}

\textit{„Sorry, ein Würfel landete schief. Den muss ich wiederholen, die anderen sind Vier und Vier.“}

\textit{„Ach, hätten wir doch Trieest den Helm gegeben.“}

\textit{„Der letzte Würfel war eine Fünf! Und mit Trieest Sonderfähigkeit addieren wir noch die eine Vier dazu!“}

\textit{„Warte, warte, Trieest kann seine Sonderfertigkeit ohne Lavastein doch gar nicht mehr nutzen!“}

\textit{„Auweia, stimmt, da müssen wir uns noch eine neue einfallen lassen. Forn kopieren?“}

\textit{„Wäre langweilig. Halbskrale sind nicht alle gleich, und Trieest Gaben brauchen vielleicht etwas Zeit, um wiederzuerwachen. Ohne Lavastein also eine Fünf von Trieest.“}

\textit{„Ein Kampfwert von 39 insgesamt.“}

\textit{„Wer übernimmt Finster-Calrai?“}

\textit{„Wartet, wartet, wir haben die Unterstützung der Skralsippe vergessen! Mit ihren +3 haben wir sogar einen Kampfwert von 42!“}

\textit{„Das gefällt mir eigentlich besser, als wenn wir mit Trieest einen anderen Würfel hätten addieren können.“}

\textit{„Hat die Skralsippe überhaupt noch Platz? Feld 52 sieht schon recht überfüllt aus für mich.“}

\textit{„Das dürfte knapp werden.“}

\textit{„Platztechnisch?“}

\textit{„Zeittechnisch ebenfalls.“}

\textit{„Ach kommt, sonst hängen wir einfach noch ein paar Kampfrunden an.“}

\textit{„Geht nicht, Trieest ist schon in der letzten Überstunde. Und auf ihn kommt’s ziemlich drauf an.“}

\textit{„Dann muss es halt einfach jetzt klappen.“}

\textit{„Ich übernehme sonst einen von Finster-Calrais Würfeln.“}

\textit{„Und ich den anderen.“}

\textit{„Ich einen dritten.“}

\textit{„Perfekt. 30 Stärkepunkte, weil 3 Helden. Uuuund los geht’s!“}

\textit{„Von mir kommt ... eine Zwei!“}

\textit{„Das beginnt ja schon toll ... eine Zwei von mir!“}

\textit{„Noch ist alles offen ...“}

\textit{„Jetzt einfach keine Sechs, einfach keine Sechs ... eine Fünf!“}

\textit{„Na toll!“}

\textit{„Damit hat das Böse genau 35!“}

\textit{„Sieben Willenspunkte runter, das geht perfekt auf!“}

\textit{„Das reicht! Wir haben es!“}

\textit{„Finster-Calrai ist Geschichte!“}

\textit{„Halleluja! Hurra!“}

\textit{„N-Karte vorlesen! N-Karte vorlesen!“}

\textit{„Geduld, Geduld!“}\bigskip



„Das ist für Mama!“, rief Jarid und schleuderte Finster-Calrai einen letzten Wasserschwall entgegen. Dieser stürzte zu Boden. Der schwarze Kristallsplitter rollte aus seiner Hand. Ehe sich dieser wieder verselbständigen konnte, griff Arbon in seine Manteltasche und stülpte einen hölzernen Würfelbecher darüber. Trieest fragte sich nicht, wozu er diesen bei sich hatte, Arbons Mantel schien allerlei nützliche Sachen zu enthalten. Stattdessen kniete sich Trieest neben Calrai und beobachtete ihn angespannt.

Calrais Augen öffneten sich. Er und Trieest blickten einander an. Ein schwaches Lächeln trat auf Calrais Lippen. Das Weiß seiner Augen blieb weiß.

Jarid atmete schwer ein und aus. Die Skrale blickten sie mit einer Mischung aus Ehrfurcht und Furcht an. Leander hatte sich verkrochen. Arbon kniete noch immer über seinem Würfelbecher und murmelte leise vor sich hin.

„Überlasst den Splitter mir“, sprach er beruhigend, „Ich weiß schon genau, wie ich ihn neutralisieren kann.“

Keiner achtete sich groß auf ihn. Alle hingen ihren eigenen Gedanken nach. Arbon kroch zur kleinen Wassermagierin mit dem Armbrustbolzen im Kopf beugte und bestätigte ein wenig schuldbewusst ihren Tod.

Jarid war es, die schließlich das Wort ergriff: „Alle fit und unverletzt? Wer gönnt sich noch einen Siegestrank am Baum der Lieder? Eine gewisse Bewahrerin hat mir einmal zugeflüstert, dass an diesen Ästen nicht nur Löschfässer gelagert werden. Und die erste Runde geht auf mich! Ich habe‘ noch was gut bei Tion!“

Weder Arbon noch die Skrale oder der Halbskral schienen erpicht darauf.

Jarid revidierte ihren Vorschlag: „Und zunächst gehe nur mal ich vor und erkläre ihnen, dass ihr alle nicht zu fürchten seid.“

Arbon war auf einmal nirgendwo mehr zu sehen. Kurgat, Tran und Bark blickten einander sehr unsicher an.

Calrai hielt Trieest seine Hand entgegen. Trieest zog ihn hoch und gab ihm Heilkräuter zum Unter-die-Zunge-Legen, bis sein Gesicht wieder eine natürliche Farbe annahm. Und dann liefen sie, Halbskral und Skral, fünffingrige Hand in vierfingriger Hand, dem Baum der Lieder entgegen.

Die restliche Sippe folgte ihnen.

Heute würde es am Lagerfeuer äußerst viel zu erzählen geben.

Alles war wieder gut.\bigskip







Beim Baum der Lieder war alles andere als alles gut. Zwar hatten die Feuerkrieger offenbar nach der Trennung des Kristallsplitters rasch die Kontrolle über ihre Lavasteine und Körper zurückerlangt, doch waren manche Knochen gebrochen und manche Lungen voller Wasser. Die Heiler des Baums der Lieder gaben ihr Bestes, so viele wie möglich zu retten, und die gute Larissa arbeitete sich fast ihn eine erschöpfungsbedingte Ohnmacht hinein, doch ließen zwei weitere Feuerkrieger in der folgenden Nacht ihr Leben und eine letzte am nächsten Tag.

Nach ausführlicher Beratung zwischen den Hohepriestern Tion und Gända sowie den überlebenden Danwaren wurden die gestorbenen Feuerkrieger und die kleine Wassermagierin Jivin auf Flößen die Narne hinuntergeschickt, aber auf brennenden Flößen, damit ihre Körper den danwarischen Glauben nach dem Flammenden Gott zurückgeschenkt werden konnten. Jarid sandte kurz darauf ein eigenes, leeres, brennendes Floß die Narne herunter, in Gedenken an ihre Mutter Rowinda. Sie selbst dachte weniger an den Flammenden Gott, sondern mehr an Mutter Natur. Die Priester vom Baum der Lieder hatten abgefärbt. Was Rowinda wohl sagen würde, wenn sie wüsste, dass ihre Tochter nun einer ihr fremden Göttin huldigte? Schuldig flüsterte Jarid die Floskeln des Flammenden Gottes. Trieest legt ihr eine schwere Hand auf die Schulter und sprach die Floskeln nach, auch wenn Jarid wusste, dass sie für ihn ähnlich leer wirkten.

Die legendäre Feuermeisterin Nidwal hatte überlebt. Ihr weiser weißer Rabe wachte argwöhnisch über ihren Schlaf. Mochten ihre Beine sie nicht mehr tragen, ihr Mundwerk war weiterhin ungeschlagen, und ihr Geist wach. So erzählte sie den anderen Genesenden tagein, tagaus Geschichten, Sagen und Legenden von Danwar. Von Heldenmut und der Schönheit in den kleinen Dingen, von Tragödien bis hin zu Witzen, von wahren Begebenheiten bis zu fantastischsten Erfindungen. Gelegentlich versuchte sie sich auch an der einen oder anderen Ballade, doch erkannten die Zuhörer bald, dass Mutter Natur bei allem Talent, das sie in Nidwals Erzählstimme gegossen hatte, ihr Tongefühl ausgelassen hatte.

Die restlichen Danware erfreuten sich an allem, auch einem eher schief tönenden Gesang. Doch nicht nur die Danware fanden darin Erleuchtung. Die folgenden Tage gesellten sich auch immer mehr Bewahrer an Nidwals Bett. Sie lauschten nicht nur, manche schrieben auch eifrig mit, allen voran der Novize Komu.

Als der Oberste Bewahrer Melkart einige Wochen darauf endlich von den Verhandlungen mit dem Einhorn-Clan der wilden Völker des Ostens zurückkehrte, waren die Danware schon längst wieder in den Norden zurückgesegelt. Immerhin war eine ganze Kiste voller frischer Pergamente eigens über danwarische Erzählkultur zum Studieren für Melkart und die übrigen Hohen Bewahrer zurückgeblieben. Die Bewahrer hatten angeboten, dass die Abreisenden einige dieser Papiere zurück nach Danwar mitnehmen konnten, doch diese lehnten dankend ab, mit der Begründung, dass derart entflammbares Material auf einer Insel voller Lavasteine und Feuerkrieger nichts zu suchen hatte. Die auf Danwar gebräuchlichen Steintafeln wären viel weniger anfällig auf allerlei Schadensquellen.

Wie zur Bekräftigung dieser Behauptung hatte einer der Feuerkrieger eine feurig heiße Hand an einen Pergamentstapel gelegt und beinahe ein Regal voller Aufzeichnungen über die Ära des Sternenschilds in Brand gesetzt. Nur das urplötzliche Eingreifen eines Wassergeists hatte ein unkontrolliertes Entflammen verhindern können. Das war Vara, die Verstoßene, gewesen. Sie war im Auftrag der Heldin Kheela von der Gedenkfeier an der Rietburg hierher gesandt worden, um ihre Dienste anzubieten, kaum hatte Jarids Brieflilie Jirid auf der Gedenkfeier erreicht.

Varas Dienste waren zu diesem Zeitpunkt zwar nicht mehr gebraucht worden, doch fand Jarid ungemeine Faszination an diesem Wesen und versuchte, sich damit zu unterhalten, jetzt, wo sie endlich Zeit dafür hatte. Wassergeister waren natürlich nichts Neues in Danwar, und auch mit Vara hatte Jarid bereits Kontakt gehabt, aber noch nie in einem Moment des Friedens, wo man sich nach Lust und Laune auf die Erforschung interessanter Phänomene konzentrieren konnte. Ein Wassergeist, der sich so lange schon außerhalb seines Mediums aufhielt? Einer, der schon so lange dieselbe Form hatte? Einer, an dem (laut Jarid) unverkennbar Spuren danwarischer Magie zu spüren waren? Mit vermutlich so einigem Wissen über mehrere Generationen interessanter Andori in seinem Innern abgespeichert? Nein, das war außergewöhnlich. Und auch wenn Vara sich nicht als sonderlich gesprächig herausstellte, schien Jarid dennoch einiges Interessantes über sie herausfinden zu können. Trieest hatte sie diesem plötzlichen Wissensdurst überlassen. Nach all dem, was vorgefallen war, sollte sie sich gut etwas ablenken. Und er hatte auch seine eigene Geschichte zu erleben.

Denn Trieest selbst verbrachte viel Zeit bei seinen Skralen. Nachdem die Bewahrer sie argwöhnisch und nach einiger Überzeugungsarbeit Jarids am Dorfrand hatten kampieren lassen, wirkte eine Zukunft in andorischer Zivilisation nicht mehr so unmöglich, wie sie einst geklungen hatte. Dennoch sahen sie für sich selbst als nächstes eine Zeit im Gebirge vor sich, und sei es nur, damit Kurgat aufhörte, in furchterfüllten Tönen vom Meer zu schwurbeln. Oder, wenn schon nicht mitten durchs Graue Gebirge, so könnte die kleine Sippe am Rande des Rietlands umherziehen. Irgendwo, wo Trieest sie zu Heldentaten anleiten konnte, ohne dass sie gleich voll unter Menschen leben mussten. Irgendwo, wo sie vielleicht weitere Skrale finden konnten und auf die von Trumm und Drunn erträumten friedlichen Zeiten hinarbeiten konnten.

Jarid wollte sich mit Jirid, Kar und Pyros an der Rietburg über seine Situation austauschen, versprach, sie anschließend jedoch wieder aufzusuchen und zu überlegen, wie es fortan mit ihnen allen weitergehen sollte.

Niemand wusste, was die Zukunft bringen würde. Doch Trieest war zuversichtlich, dass sie von Glück erfüllt sein würde.\bigskip







\textit{Arbon kniff seine brennenden Augen zusammen.}

\textit{„Hast du je daran gedacht, ein fahrender Händler zu werden? Ein bisschen das Land zu sehen? So findest du bestimmt auch viel mehr Kundschaft als in dieser versteckten Hütte!“}

\textit{Naraven winkte husten ab. „Ach nein, ich komme schon so über die Runden. Ich brauche niemanden außer mir selbst, ich bin zufrieden hier allein. Solange ich nicht ernsthaft krank werde, kann ich mich ja auch selbst heilen. Und solange mir kein unbekannter Schrecken im Wald auflauert. Und ich will gar nicht erst daran denken, was geschieht, sobald ich so alt bin, dass meine geistigen Kapazitäten sich verabschieden ... das ist ein Problem für mein zukünftiges Selbst. Na ja, jetzt, wo ich so darüber nachdenke, hätte so ein Dasein als fahrender Reisender vielleicht schon etwas. Wobei dies ganz andere Risiken mit sich bringt. Überfälle von Kreaturen und barbarischen Bergkriegern, Abgaben an die bereisten Reiche ... ich labere wieder. Was habt Ihr Leander erzählt?“}

\textit{„Ein Märchen über eine von einem Fluch besessene Wassermagierin, das er hoffentlich glaubt, oder das ihn zumindest nicht zu weiteren Nachforschungen antreibt“, murmelte Arbon. „Am besten erinnert er sich gar nicht mehr an den kleinen Kristallsplitter. Ich traue ihm nicht mit einer solchen Macht. Womit wir wieder beim Thema wären.“ Er nickte der silbernen Masse im Topf vor ihm zu. „Verrate mir eines, o weiser Naraven: Wie konnte dieser Verstand, dieser Geist, dieser Seele, wie auch immer man es nennen will, wie konnte dies überhaupt in diesem Steinsplitter überdauern? So ein Kristall ist ja ganz ohne das Fleischige und Flüssige, das ein menschlicher Geist zum Überleben in einem Körper bräuchte.“}

\textit{„Magie, mein lieber Hogo, Magie“, sagte Naraven schulterzuckend, „Hast du schon mal einen Feuergeist gefragt, wie er seinen Verstand in einem flüchtigen Körper aus stetig reagierenden Partikeln behält? Es muss definitiv eine Erklärung geben, aber die ist weit außerhalb meines Fachgebiets. Nun, das hier ist eigentlich auch außerhalb meines Fachgebiets, aber hier kann ich wenigstens etwas Alchemistisches beisteuern.“}

\textit{Naraven zog mit einer Zange etwas Zuckendes aus dem Topf und ließ es in eine mit Runen übersäte Klangschale fallen. Die Masse war eine Art Körper. Eine wabernde silberne Masse, die sich weigerte, scharfe Konturen anzunehmen. Nichtsdestotrotz glaubte Arbon, Arme und Beine ausmachen zu können. Aufrecht wäre das Wesen ihm nicht einmal zum Knie gekommen. Aktuell lag es jedoch bloß strampelnd in der Klangschale.}

\textit{„So, das war’s!“, sprach Naraven, „Nicht reinbeißen, die Inhaltsstoffe sind ebenso selten wie ungenießbar. Tinte des Oktohan, Asche eines Takuri, ein wenig Lavastein-Pulver, ich erspare mir die weiteren Details. Auf jeden Fall sollte diese Masse jegliche eingefasste Magie gut abschirmen können. Fünf weitere Goldstücke, und ich setze den Kristallsplitter gleich ein.“}

\textit{„Bitte tu das, so etwas könnte ich ohnehin nicht“, grummelte Arbon. Sorgfältig zog er aus seiner Manteltasche den in dicken Stoff eingeschlossenen finsteren Kristallsplitter hervor. Naraven griff mit spitzen Fingern nach dem Stoffstück, stach den Splitter in die silberne Masse und begann, mit einem grün leuchtenden Stab auf die Klangschale zu hauen. Die darin eingefassten Runen leuchteten schwach auf und summten laut los. Die Masse waberte und wurde einen Augenblick formlos, dann festigte sie sich. Der schwarze Kristallsplitter war nicht mehr zu sehen.}

\textit{Naraven rieb sich die Hände: „So, jetzt ist der Stein eingeschlossen. Solange er in dieser Form bleibt, kann er nichts und niemanden mehr kontrollieren. Dieses Siegel ist mein Meisterwerk. Um das zu brechen, bräuchte man ein Feuer, so heiß wie Drachenodem. Und so eines gibt es seit dem Tod des letzten Drachen nirgendwo in der bekannten Welt. Die Gefahr durch dieses Böse wurde neutralisiert. Und mit ein bisschen Glück lernt es bald, seinen neuen Körper zu steuern. Dann sollte es mit dir kommunizieren können.“}

\textit{Arbon nickte bloß und zückte seinen Geldbeutel.}

\textit{Während Naraven in einen Nebenraum schlüpfte, um irgendwelche letzten stabilisierenden Mittel zu holen, wandte sich Arbon der silbernen Masse vor ihm zu.}

\textit{Leise flüsterte er: „Du hast etwas in meinem Geist gesehen, das du nicht sehen solltest. Geheimnisse aus den Schwarzen Archiven, die kein Verstand lange halten kann, ohne verrückt zu werden. Deine Weltsicht mag auf den Kopf gestellt sein, doch bald schon wirst du all dies wieder vergessen oder rationalisiert haben. Ich weiß, wie wir Menschen so ticken.“}

\textit{Er holte tief Luft.}

\textit{„Nichtsdestotrotz muss ich anmerken, dass du niemandem je auch nur einen Hauch dessen davon erzählen darfst, was du in meinen Erinnerungen erhaschtest. Glaube mir, das würde nicht gut für dich enden ... Hademar aus dem Königshause Brandur.“}

\textit{Eine leise, quiekende Stimme erhob sich aus der blubbernden Masse, die den Kristallsplitter umgab. „Welche Geheimnisse? Wer ist Hademar?“}

\textit{„Hah! Wenn du dich länger dumm stellst, verlässt du diese Hütte nicht mehr. Ich sah deine Erinnerungen, als du die meinen kostetest. Dieser neue Körper ist um einiges beweglicher als dein letzter, doch schirmt er deine Macht ab. Du bist mir ausgeliefert. Dein Wissen der einzige Grund, warum du noch lebst. Falls man deine Existenz überhaupt ein Leben nennen kann. Du kennst die Künste der Krahder und der Hadrier, und du verknüpftest sie beide. Diese Kunst könnte uns Helden sehr nützlich sein, auch wenn die meisten im Orden sie wohl aus Prinzip ablehnen würde. Doch was sie nicht wissen, macht sie nicht heiß. Nun sprich, hast du mir etwas zu erzählen?“}

\textit{Inmitten der silbernen Masse bildete sich ein Mund, der mit urplötzlich tieferer, bedrohlicherer Stimme weitersprach.}

\textit{„Du kennst meinen Namen, doch hast du keine Ahnung, wer ich wahrlich bin. Ich bin das Böse, das leidet und Leiden schafft. Ich habe unter den Krahdern gelitten und ihre Dunkle Hexerei erlernt. Ich habe die verbotenen Schriften der Akademie von Hadria studiert. Drachenmagie aus den Knochen im Grauen Gebirge fließt stetig durch meinen Geist. Ich habe Untote mit einem Fingerschnippen beschworen und den Urtroll mit einem einzigen Blitz vertrieben, und selbst mit einem Bruchteil meiner ehemaligen Macht konnte ich mir den Willen so vieler anderer unterwerfen. Ich war ein Mensch, ein Geist, ein Kristall, ein Monster und eine Armee. Ich trage die Erinnerungen unzähliger Feuerkrieger in mir. Ich reiste durch die Zeit und habe länger gelebt als all deine Großeltern zusammen. Ja, ich habe dir etwas zu erzählen, Arbon, der du von deiner Familie nicht gewollt warst. Und ich habe dir eine Frage zu stellen. Gegeben der, der ich bin ...Wie lange denkst du, du könntest mich hier eingesperrt lassen?“}

\textit{Arbon unterdrückte das Verlangen, nachzufragen, woher sich Hademar anmaßte, seine Großeltern zu kennen. Stattdessen grinste er. Die Identität war bestätigt. Nun musste er diese Nuss nur noch knacken.}




\newpage
\section{Ein Magischer Epilog}

\az{Jahr 72}

Bereits von Weitem war zu erkennen, dass neben der Taverne zum Trunkenen Troll eine mächtige Steppenechse mit einem lächerlich dünnen Seil an einen Pflock angebunden war. Das schien die Echse aber nicht groß zu stören, denn diese ergötzte sich lautstark schmatzend an einem Riesenbüschel getrockneten Rietgrases, das in einem Trog vor ihr aufgeschichtet war. Als wäre das noch nicht genug Beweis gewesen, deuteten zwei riesige, prall gefüllte, leuchtend farbige Stoffsäcke am Tuch auf Sabris Rücken darauf hin, wer sich soeben in der Taverne aufhielt.

Eine aufgeregte Fischerin hatte Trieest kürzlich berichtet, dass eine Untergruppe der Helden von Andor vor wenigen Tagen ein finsteres Ritual an irgendeinem Bannkreis hatte aufhalten könnten – „... und das so nahe an meiner Fischerhütte! Was, wenn die Dorfkinder über diesen Kreis gestolpert wären?!“ – und nun überall Feierlaune herrschte.

„Sabri!“, rief Jarid fröhlich und rannte auf die Steppenechse zu. Diese drehte nicht einmal ihren Kopf, sondern stocherte weiterhin stur mit ihren unterarmlangen Riesenzähnen im Stroh herum. Jarid klopfte dem Tier energisch an die Seite und begann, Sabris ledrige Haut an einer ganz bestimmten Stelle hinter einem Ohr zu kratzen. Ein wohliges dumpfes Brummen drang aus dem dicken Echsenhals. Mehr konnte Jarid ihr jedoch nicht entlocken, dafür war sie zu stark mit ihrem Futter beschäftigt.

Trieest grinste bloß und wanderte an ihr vorbei ins Innere der guten Stube.

Die Taverne war wieder einmal platschvoll.

Vor dem großen Kamin lag wie so oft eine Schlafende Katze und wärmte ihr weiches Fell. Im Kamin selbst hatte sich ein putziger kleiner Feuertakuri niedergelassen. Hin und wieder scharrte der golden leuchtende Vogel mit seinen Flügeln ein wenig in der Asche, woraufhin Glut aufstob.

Na, wenn Turr hier war, war seine Hüterin vermutlich nicht allzu weit entfernt. Ein Blick zwei Tische weiter links verriet, wo sie sich aufhielt. Eigentlich hätte man nur auf ihre Stimme hören müssen, denn die stach schon etwas angeschwipst über den Grundlärm in der Taverne hinaus:

„... ein Dutzend von Irils Runensteinchen nimmst, dann kannst du das Dutzend zum Beispiel als einzelne Linie anordnen, oder als zwei Sechserreihen, oder als drei Viererreihen, und das geht alles perfekt auf. Aber, sieh her, wenn ich nur einen einzelnen weiteren Stein dazu nehme, dann werden es dreizehn, und die Dreizehn kannst du nur als einzelne Linie darstellen, nicht als Rechteck mehrerer Linien gleicher Länge. Darum nennen wir die Dreizehn eine Linienzahl.“

Der Blick auf die Tulgori wurde von einigen vorbeistürmenden Tavernengästen verdeckt, doch das Geräusch von polternden Steinchen war weiterhin zu vernehmen.

„Was, wenn ich die Dreizehn in zwei Siebnerreihen aufteile, und diese Rietgrasblüte hier markiert die Null?“, fragte eine klirrend kalte, aber nicht unfreundliche Stimme.

„Das geht doch nicht, dann haben die beiden Reihen ja nicht gleich viele Steine darin!“, protestierte Aćh lautstark.

„Nur wenn du die Null nicht mitzählst“, erwiderte ihr Gegenüber ruhig. Trieest erkannte ihn nicht zuletzt daran, dass sein Sitzplatz über und über mit Eiskristallen bedeckt war. Offenbar hatte man seinen Stuhl (eventuell auch aus genau diesem Grund) ganz in der Ecke des Raumes platziert. Überraschend energisch zeigte Ijsdur mit seinem Zeigefinder auf eine Anordnung von kleinen farbigen Steinchen auf der Tischoberfläche und sprach demonstrativ: „Schau her. Null, Eins, Zwei, Drei, Vier, Fünf, Sechs, jetzt springen wir in die zweite Reihe und sehen Sieben, ...“

Ein kleiner Schauer glitzernden Reifs verteilte sich über die Steinchen und Ijsdur hätte zweifelsohne bis Dreizehn weitergezählt, wenn ich Aćh ihm nicht ins Wort gefallen wäre:

„Wenn wir immer bei der Null zu zählen beginnen würden, würde sich auch so viel anderes ändern. Doch bei diesem Problem, so, wie es festgelegt wurde, beginnen wir immer bei der Eins. Die Zahlen, die wir so nur als Linie darstellen können, die nennen wir Linienzahlen, weil wir sie nur als einzelne Linie darstellen können, und das ergibt Sinn. Dreizehn ist einfach eine Linienzahl, und wenn du das anders siehst, verstehst du einfach nicht, was ich mit diesem Begriff meine.“

Breit grinsend gönnte sie sich einen weiteren Schluck aus ihrem großen Metkrug.

Ijsdur blieb still und kratzte sich nachdenklich an seinem Geweih, woraufhin noch mehr Eiskristalle zu Boden rieselten.

„Was ist mit der Eins?“, warf eine melodische Stimme ein, „Um den Ava rum gibt zwar es nicht viele, die sich mit dem Zählen um des Zählens willen befassen, aber ich kannte da mal eine alte Totemschnitzerin, welcher lieber den lieben langen Tag lang über solche Sachen nachgedacht hätte, statt praktische Anwendungen dafür zu finden. Wenn man sie fragte, hätte sie dir erzählt, dass man das Wirken der Götter eher in solchen Überlegungen finden könnte statt bei einem Blick in die Wunder der wahren Welt. Was für ein Mensch ...“

„Du schweifst wieder ab, Barz.“

„Verzeih mir. Mein Mundwerk mag manchmal murmeln, bis es meinen Geist abgehängt hat. Wenn ich mich richtig erinnere, hatte diese Schnitzerin auch schon einmal von solchen Zahlen berichtet, aber sie gab ihnen einen anderen Namen und sie sagte, dass die Eins nicht zu denen gehörte. Dabei ist die Eins doch eine Linienzahl!“

Während Barz‘ Mund laberte, falteten seine Hände aus einem Stück Pergament eine Faltfigur. Ein Vogel, eventuell gar ein Takuri? Ehe Trieest genauer hingucken konnte, hatte Barz sein Werk auch schon wieder entfaltet.

„Ein einzelner Stein ist doch keine Linie!“, meldete sich Aćh wieder zu Wort. „Die kleinste Linienzahl muss die Zwei sein. Und von dort geht’s weiter mit Zwei, Drei, Fünf, Sieben ...“

Vom Kamin her ließ ihr Feuertakuri Turr einen freudigen Schrei aufklingen. Er sprang auf und verteilte Asche im Schankraum.

„Der hat auch schon genug von diesem Gelaber“, lachte eine kratzige Stimme, „Darf ich meine Runensteinchen nun wieder einpacken?“

„Tu, was du nicht lassen kannst“, sprach Aćh. Einiges Ruckeln war zu hören, als eine Zwergin sich von ihrem extra hohen Stuhl aufrichtete und über den Tisch langte, um einige farblose Steinchen mit eingeritzten Runen zurück in ihre zugehörigen Löcher auf einer über und über mit Runenmustern verzierten Metallscheibe zu befördern. Währenddessen zückte Aćh eine grobe Steinflöte und spielte eine kurze fröhliche Melodie. Der Takuri, der darauf und daran gewesen war, aus dem Kamin zu stürmen, legte seinen Kopf schief und trampelte noch einige Male trotzig auf der Asche herum, ließ sich dann aber wieder fügsam ins Kaminfeuer sinken und wälzte sich darin.

„Aber eine Eins ist doch eine Linie, also sollte eine Eins auch eine Linienzahl sein“, brachte Barz kopfschüttelnd das Gespräch wieder ins Rollen.

„Du brauchst mindestens zwei Punkte für eine Linie“, hielt Iril nun im Einpacken der Runensteine ein. „Erst wenn ich zwei Löcher in den Boden meißle, ist klar, welche Linie ich dazwischen gezogen haben will.“

„Die Richtung der Linie ist doch durch den Tisch und die Rillen im Holz bereits klar“, erklärte Barz. „Du würdest sie doch nicht plötzlich schief drauflegen, das würde sich falsch anfühlen.“

Entschieden legte Iril einen ihrer grauen Steine zurück auf den Tisch und stieß ihn an. Dieser drehte fröhlich für einige Augenblicke, ehe er flach auf dem Tisch landete.

„In welche Richtung zeigt dieser Stein?“, fragte Iril demonstrativ.

Barz antizipierte ihren Punkt und sprach: „Solange kein zweiter Stein da liegt, \textit{vermutlich} in Richtung der Tischrillen.“

„Vermutlich?“

„Man weiß doch nie so ganz, was gilt und was geschehen wird. Aber wenn jemand den ersten Stein dort hinlegt, steht zu vermuten, dass der nächste Stein entlang der Tischrille platziert werden wird. Sobald der zweite Stein dann hingelegt wird, werden wir sehen, ob die Vermutung richtig war. So funktionieren Experimente. Aber du darfst nicht im Voraus wissen, was ich vermute, sonst kannst du das Ergebnis verfälschen.“

„Das hier ist auch kein Experiment, sondern ein Dialog“, meinte Iril, „Zwerge und Menschen handeln im Gegensatz zu Steinen unberechenbar.“

„Nicht unberechenbarer als eine Pulvermischung, die man noch nicht zur Gänze versteht. Also eigentlich jede Pulvermischung.“

„Ach komm, Barz, im Gegensatz zu einem Menschen kannst du bei einer Pulvermischung irgendwann alle Geheimnisse entschlüsselt haben. Einen Kulturschaffenden kannst du nie komplett durchschauen.“

„Ich glaube, du verwechselst die Natur meiner Pulvermischungen mit der deiner Runenmagie. In deiner idealisierten Vorstellung der Runen magst du alles, was es über ein gewisses Symbol zu wissen gibt, verstanden haben. Aber in der realen Welt, wo du deine Runen ungelenk in Metall eingravierst oder auf deine Haut stempelst, kannst du auch die nie zur Gänze verstehen. Musst du auch nicht. Aber letzten Endes unterscheidet in Bezug auf die Unmöglichkeit des Verstehens nichts, aber auch gar nichts einen Menschen oder einen Zwerg von einem Stein, oder einer eingeritzten Rune, oder einem magischen Pulver.“

Iril legte ihre Stirn in Falten, wohl um Barz‘ viele Worte Revue passieren zu lassen.

Dies gab Aćh die Gelegenheit, sich wieder einzumischen: „Ach nein? Warum kann dann jeder Mensch meine Musik verstehen, aber kein einziger Kieselstein?“

Aćh zog ihre tulgorische Steinflöte hervor und brachte eine weitere wilde Melodie zustande. Ein paar Tische weiter drüben johlte eine Runde laut auf.

„Woher willst du wissen, dass kein Kieselstein ...“

„Ich hab’s!“, unterbrach Ijsdur die Runde, „Ich hab’s endlich! Wir müssen die dreizehn Runensteine nicht so quadratisch hinlegen, sondern in einem Dreiecksmuster: In die erste Zeile einen, in die zweite zwei, in die dritte drei, in die vierte vier und in die fünfte wieder drei. Dann können wir die Dreizehn anordnen, ohne die Null mitzuzählen. In dieser Dreiecksform ist ohnehin weniger unnötiger Platz zwischen den Steinchen als in quadratischen Reihen.“

„Das ist eine schöne Form ...“, „... aber das ändert nichts daran, dass man die Dreizehn ...“, „... nicht in gleich lange Linien aufteilen kann!“, widersprachen Iril und Aćh gleichzeitig.

„Warum sollte das überhaupt irgendjemand wollen? Ist das wieder einfach so eine Eigenheit der hiesigen Zivilisation, genau wie ...“

„Wenn du dich noch einmal darüber beschwerst, dass du hierzulande seltsam angeguckt wirst, wenn du nur mit einer Toga rumläufst, erzähle ich dir mal etwas über die komplizierten Kleidervorschriften der Runenmeister der Silberländler“, murmelte Iril.

„Und ohnehin“, ergänzte Barz, „Sind solche Kleidervorschriften viel weniger willkürlich als diese Steinchen- und Zahlen-Rätsel. Letztere dienen nur zum Vergnügen, erstere haben hingegen viele Anwendungen. Kleidung kann Status und Eigenheiten des Gegenübers kommunizieren, kann dich etwa gegen Verletzungen oder Kälte schützen ... oh. Vergiss den letzten Punkt wieder.“, unterbrach sich Barz.

Ijsdur setzte zu einer Erwiderung an, da ...

„Tri“, drang Jarids glockenhelle Stimme an sein Ohr, „Wie lange willst du noch auf den fremden Tisch starren? Du könntest dich einfach zu ihnen setzen.“

„Ah, nein, ich will sie nicht stören. Und mir ist auch nicht zwingend nach Gelaber“, sprach Trieest, „Ich war nur abgelenkt. Hauptsächlich geht es mir jetzt darum, diese Bestellung abzuholen, zur Sippe zurückzukehren und dann den restlichen Tag in Ruhe genießen.“

„Da bin ich ganz bei dir. Wenn nur ...“

Jarid erstarrte und sprach: „Uh-oh. Ich bin mir nicht so sicher, ob wir so viel Ruhe kriegen können, wie wir gerne hätten. Guck mal dort rüber, aber möglichst unauffällig.“

Trieest folgte Jarids Zeigefinger und schnappte nach Luft, als er erkannte, auf wen Jarid zeigte

„Ist das etwa ...“

Jarid nickte grimmig. „Das ist er. Leander, der lügnerische Seher. Sitzt hier im Trunkenen Troll, unterhält sich mit Garz und genießt den Abend, als hätte er nichts Mieses angestellt. Haben wir mit ihm noch ein Hühnchen zu rupfen?“

„Nein, ich will nur nicht ...“ Urplötzlich trat eine Erinnerung vor Trieests inneres Auge. Eine dunkelblaue Frau mit langem, weißem Haar, welche vor einer Feuerwand schwebte. Er korrigierte sich. „Doch, eigentlich habe ich noch etwas, was ich ihm ausrichten soll. Pass einfach auf, dass wir nicht zu lange bei ihm verbleiben.“

Trieests Blick schweifte zurück zu den Magischen Helden am anderen Tisch. Soeben wurden neue Getränke serviert und Barz demonstrierte theatralisch, dass er Met mit dem passenden Pülverchen in Wasser verwandeln konnte. Große Begeisterung kam nicht auf, aber zumindest Ijsdur erfreute sich daran. Er langte nach Barz‘ Bogen, doch dieser hielt ihn auf. „Bitte nicht, Ijsdur, beim letzten Mal vereiste die Sehne und zerbrach. Verzeih, du bist manchmal einfach nicht am geschicktesten mit großen wärmeempfindlichen Gegenständen.“

Trieest blendete die Magischen Helden aus und näherte sich Leander. Jarid folgte ihm neugierig. Das Gesicht des Sehers drehte sich in seine Richtung, also konnte er ihn sehr wohl ankommen hören. Seine Miene blieb aber gelangweilt ausdruckslos, offenbar hatte er ihn diesmal nicht erkannt. Lag das daran, dass er nicht mehr einen Lavastein in seiner Brust trug. Hatte Leander damals vor einigen Jahren dessen Präsenz gespürt?

Gildas fröhliche Stimme donnerte über den Gastraum hinweg.

„Neue Runde für alle Helden! Ijsdur, trink’s schnell aus, ehe es dir wieder einfriert! Moin, Trieest, keine Getränke für deine Jungs?“

Trieests Jungs waren inzwischen in der Anzahl auf das Doppelte angewachsen. Sogar ein alter weiser Schamane namens Obron hatte sich ihnen angeschlossen und war daran, Calrai als seinen Nachfolger auszubilden. Doch fühlten sie sich im Rietland immer noch nicht so wohl, erst recht nach einer üblen Konfrontation mit einem Trupp Soldaten des Königs. Nicht, dass die anderen Skral-Sippen besser auf sie zu sprechen gewesen wären. Aber alles zu seiner Zeit.

Trieest nickte der Gastwirtin zu. „Nein, meine Jungs wollen lieber draußen bleiben. Auch wenn ihr ihnen noch so oft sagt, sie mögen hier willkommen sein. Ich mein, ich kann’s verstehen. Kurgat kann sogar schon von einem persönlichen Gefecht mit Barz erzählen, als er diesem früher im Grauen Gebirge aufgelauert war ... wobei das weniger ein Gefecht war und mehr ein benommenes In-Andor-Aufwachen. Ich schweife ab.“

Nun, wo auch Leander Trieests Identität vernommen haben musste, fluchte der Seher unter seinem Atem auf.

„Abend, Trieest“, murmelte er, „Dann ist Jarid wohl auch nicht weit entfernt?“

„Abend, Leander“, erwiderte Jarid. Niemand von ihnen nutzte die danwarischen Höflichkeitsfloskeln.

„Abend, ihr beiden“, sprach der Handelszwerg Garz fröhlich, als wäre ihm völlig entgangen, wie eisig die Stimmung zwischen dem Seher und den Danwaren war. „Gilda sagte mir, ihr wärt dabei gewesen, als die Weinhändlerin Lysbett starb?“ Mehr zu sich selbst murmelte er: „Sie war doch noch so jung, die Lysbett.“ Dann guckte er auf und fuhr in seinem typisch hämischen Ton fort: „Die Kleine ist mir eine Zeit lang umhergefolgt, wisst ihr? Ich nahm sie mit auf meinen Reisen, weil so ein kleines Kerlchen einfach mehr Sympathie unter den Leuten weckt als ein armseliger geiziger Handelszwerg, wie ich es bin. Hat so einiges von mir gelernt, hat sie, häh? Aber gut, früher oder später musste sie diese Welt verlassen, und weniger Konkurrenz ist immer gut fürs Geschäft! Weißt du zufälligerweise, wo sie ihre Drachenfässer lagerte? Diese momentane Marktlücke gedenke ich rasch und kompetent zu füllen!“

Einzig Garz‘ Augen, welche viel stumpfer als sonst dreinblickten, verrieten den wahren Gemütszustand des Handelszwergs. Trieest winkte ab.

Der Seher lehnte sich zurück, zog ein Stück Holz und ein Messer aus seiner Tasche und begann, desinteressiert zu schnitzen. Obwohl er eine Augenbinde trug, verfehlte sein Messer kein einziges Mal sein Ziel.

Trieest schluckte all seine Wut auf Leander herunter und überlegte, wie er Garz vom Tisch vertreiben konnte. Jarid war nicht so feinfühlig.

„Was war eigentlich dein Problem?!“, herrschte sie Leander an. „Du hättest uns beinahe auf eine mysteriöse Insel gelockt, auf der uns wer weiß was erwartet hätte! Du hast Trieests Zukunft gesehen und dennoch beschlossen, ihm eine falsche Hoffnung zu schenken!“

„Du etwa nicht? Hast du ihm nicht ebenso die ganze Zeit weisgemacht, er könne diese Bürde durch gute Taten loswerden?“, gab der Seher unwirsch zurück.

Garz machte große Augen. „Was geht denn hier ab? Alter Leander, ich hab‘ dich nicht mehr so mürrisch gesehen, seitdem der olle Leam dich abwies. Deine Angelegenheit gehen mich natürlich nichts an, aber Geschäfte laufen einfach besser, wenn man seine Karten offen auf den Tisch legt, weißt du? Willst du mir erzählen ...“

„Halts Maul, Handelszwerg!“, zischte Leander grimmig. Dann atmete er einmal tief durch und sprach: „Es stimmt. Ich verspüre eine große Wut auf dich, Jarid. Du hast mit deinen Worten mehr Schaden angerichtet, als du denken kannst. Aber es bringt niemandem etwas, diese Lage von vorne aufzurollen. Schließlich wusstest du nicht, was du tatst.“

„Du wusstest hingegen sehr wohl, was du tatst! Ein Lügner und Betrüger der schlimmsten Sorte bist du! Warum bist du nicht einfach so auf uns zugekommen und hast uns um Hilfe gebeten?“

„Ich wusste nicht, ob ich euch überzeugen könnte mit der Wahrheit. Mit der Lüge wäre es mir problemlos gelungen. Hättest du nur ein klein wenig länger in diesem Keller nach Wasserfässern gesucht, so hätte ich Trieest davon überzeugt, dir nichts von meinen Worten erzählen, und ihr beide wärt fröhlich nach Narkon aufgebrochen.“

„Und dort ... gestorben? Verflucht worden? Versklavt worden? Was wolltest du auf Narkon erreichen, Leander?!“

„Das ist meine eigene Sache. Ich weiß nicht einmal, warum ich mich vor dir zu rechtfertigen versuche. Ich will bloß noch sagen, dass ihr diese Insel beide wieder gesund und munter verlassen hättet ... und auf dem Weg einige andere Leben gerettet hättet. Diese sind nun höchstwahrscheinlich verloren!“

„Das bringt doch jetzt nichts mehr“, begann Trieest, „Verzeih, Garz, aber könnten wir dich dazu einladen, eine Runde auf unsere Kosten zu trinken? Ein wenig abseits dieses Tisches? Es gibt da etwas ...“

„Ja, ja, auf dass der Handelszwerg ja nichts Interessantes mitkriegt und an die falschen Ohren verkauft“, grummelte Garz. Dann spazierte er jedoch folgsam davon.

„Und glaube ja nicht, dass nur Jarid auf dich wütend wäre“, grummelte Trieest, „Doch schlucke ich dies herunter, weil ich glaube, eine wichtige Botschaft für dich zu haben. Und wer weiß, vielleicht änderst du dich eines Tages noch.“

Leander schnaubte spöttisch: „Wenn ich ein Goldstück hätte für jede Person, die ich in meinem langen Leben belogen habe, hätte ich genug Geld, um ein königliches Kopfgeld auszusetzen. Ihr beide seid nichts Besonderes. Ihr werdet mich nicht ändern.“

„Nicht, solange du uns nicht lässt. Aber das ist auch nicht unser Ziel.“

„Was ist denn Euer Ziel?“

Jarid blickte ihn aufrichtig interessiert an: „Ja, Trieest, was ist unser Ziel eigentlich? Was hast du mit ihm vor?“

Trieest atmete tief durch und meinte dann: „Ich fühle, ich soll dir noch etwas ausrichten. Ich glaube ...“ Trieest Stimme wurde unsicher und leiser. „Leander, ich glaube, ich habe deine Mutter getroffen. Sie richtet aus, dass sie dir verzeiht, für deine vergangenen und deine zukünftigen Taten. Dir und deinem Bruder. Ich soll dir sagen, dass du dich nicht so sehr auf ihn fokussieren musst. Deine Pläne sind unnötig. Du wirst ihn auch so wiedersehen, frei und unverflucht. Es besteht keinen Grund, weiteres Leid anzurichten. Also ... ziemlich sicher jedenfalls. Wer kann die Zukunft schon so genau wissen?“

Leanders Kopf drehte sich ruckartig und seine leeren Augen starrten Trieest andächtig an. Mit bebender Stimme antwortete er schlussendlich: „Das erscheint mir höchst unwahrscheinlich. Meine Mutter ist schon seit vielen Jahrzehnten tot. Ein namenloser Grabstein in Werftheim zeugt davon.“

„Die Wahrheit meiner Worte musst du selbst beurteilen.“

„Dann beurteile ich sie als unwahr!“

„Urteile nicht zu hastig, du selbstgerechter Verfechter der Logik. Die letzten Augenblicke eines Toten hängen manchmal noch lange am Todesort. Und Echos von Toten können in gewissen Steinen noch lange nach ihrem Ableben kurz nachhallen, das müsstest du wissen. Du blaue Blüte in der Dunkelheit. Bitte, richte nicht noch mehr Pein an, als du bereits hast.“

Leanders Mundwinkel zuckten, als er diese Worte vernahm. Ansonsten regte er sich nicht. Ein nasser Fleck breitete sich unter seiner Augenbinde aus.

Und Jarid und Trieest ließen ihn allein sitzen, warfen den Magischen Helden einen letzten Blick zu und begaben sich dann nach draußen, wo ihre Skrale bereits auf sie warteten.


