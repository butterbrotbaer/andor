\begin{chapterbox}
    \chapter{Der verschwundene Feuertakuri (2022)}
    \label{Der verschwundene Feuertakuri (2022)}
    \az{Jahr 63}

    \begin{center}
        Teil II der Magischen Abenteuer
        
        Fortsetzung von \hypref{Der Steppennomade, der große See und das große Feuer (2022)}
    \end{center}
    
    Takuri-Hüterin Aćh hat einen seltsamen Tag. Ein fremdsprachiger Nomade aus einem fernen Land erscheint wie aus dem Nichts an ihrem Nestbaum in den westlichen Ausläufern des Kuolema-Gebirges. Zeitgleich verschwindet ihr liebster Feuertakuri Turr. Sie kann ihn nicht einmal mehr mit ihrer Steinflöte zurückrufen. Hängen die beiden Vorfälle zusammen? Kann vielleicht der vielwissende Hüter der Zeit in der Metropole Agarb am anderen Ende von Tulgor weiterhelfen?
\end{chapterbox}


\section{Prolog}

\az{Jahr 62}

Die Stimmung am Nestbaum war angespannt. Schon seit Tagen lag Saarćhan, die älteste und riesigste aller hier angesiedelten Feuertakuri, hustend und keuchend in ihrem gewaltigen Nest. Sie lag im Sterben, und sie würde bald in einem riesigen Feuerwirbel ihr Leben beenden.

Der Lebenszyklus größerer Takuri dauerte erheblich länger als der von kleineren, und es war eine wahre Seltenheit, dass derart riesige Takuri ihr Leben beendeten. Sämtliche Hüter sahen solchen Momenten immer mit Spannung entgegen. Manchmal, ganz selten, kam es vor, dass ein Takuri im Feuer verging, ohne ein kleines Küken in der Asche zu hinterlassen. Ob dieser Takuri dann wirklich vom Angesicht dieser Welt verschwunden war, oder ob das Küken sich schlicht in diesem Augenblick an einen anderen Ort teleportiert hatte, war schwer zu sagen. Die Unsicherheit blieb jedenfalls. Und so warteten die Hüter allesamt Saarćhans letzten Atemzug ab. Doch bedeutete dies nicht, dass sie deswegen die Pflege der dutzenden anderen aufmüpfigen Feuervögel am Nestbaum vernachlässigen durften.

Takuri-Hüterin Aćh kümmerte sich beispielsweise gerade darum, die Gewichtsverläufe dreier schwächerer Takuri über die letzten Wochen zusammenzusammeln und möglichst übersichtlich auf einem Stück feuerfesten Pergaments gegeneinander zu plotten. Ihre umherschweifenden Gedanken wurden allerdings abrupt unterbrochen.

„Aćh! Aaaaćh! Hüterin Aćh!“, erklang eine hohe Stimme. Das klang nach einem Temm. Aćh verdächtigte Yrbstschly. Was für ein Name! Als hätte die Kleine eine mit Absicht besonders schwer auszusprechende Zeichenkombination gewählt.

Yrbstschly war im Gegensatz zu den meisten anderen hier ansässigen Temm nicht am Bau von Takuri-Spiegeln oder ähnlich magisch aufwendigen Aufgaben beteiligt, sondern hatte sich dem profanen Orden der Hüter angeschlossen. Mit ihren für eine Temm mickrigen 53 Lebensjahren war sie ziemlich jung und relativ unerfahren. Sie hatte die Tendenz, stets um Hilfe zu rufen, sobald ihr auch nur eine Brotkrume zu viel zu Boden fiel. Und stets rief sie nach Aćh. Wohl weil die beiden im selben Zimmer ruhten. Es war nicht so, dass Aćh sich nicht freute, aushelfen zu können. Aber manchmal konnte es auch einfach etwas viel werden.

„Aaaaćh! Hilfe!“, erklang erneut der hohe Ruf. Aćh seufzte, legte ihre Schreibfeder beiseite und eilte aus ihrem Zimmer hinaus.

Ihr und Yrbstschlys gemeinsames Zimmer lag in einem eleganten Gebäudekomplex, der auf den Wurzeln des Nestbaums der Takuri erbaut worden war. Der Nestbaum der Takuri wiederum war ein gigantischer Baum aus purem Gestein, der sich dutzende Meter in den Himmel erhob. Er stand am Übergang von den Gebirgsausläufern in die rote Steppe Tulgors. Man könnte denken, jemand hätte einen gigantischen Mammutbaum versteinert. Dass der Baum direkt aus einem Berg gehauen worden war, schien den hier wohnenden Hütern unwahrscheinlich, denn sie schafften es heutzutage nicht mal mit ihren besten Pickeln, mehr als einige Kratzer in die Oberfläche des Nestbaums zu schlagen. Allzu oft zerbrachen stattdessen die Pickel. Darum hatten die Hüter auch keine Behausungen und Treppen im Baum selbst gebaut. Stattdessen führten aufwendige Konstruktionen aus Plattformen, Kränen und dergleichen um den Baum herum und erlaubten den Hütern, die darin eingelassenen Nisthöhlen zu erreichen.

Diese Baumhöhlen waren ebenso wie der riesige Nestbaum selbst schon so lange hier gewesen, wie die schriftlichen Chroniken der Hüter zurückreichten. Und die Takuri hatten schon damals diesen Baum so sehr geliebt, dass der Orden der Takuri-Hüter fast keine andere Wahl gehabt hatte, als sich hier niederzulassen.

Auch heute umkreisten dutzende Feuertakuri glücklich den riesigen Steinbaum und regneten Funken auf die Gebäude der Hüter auf den Wurzeln des Baums herab. Die aufmüpfigen Feuervögel waren die gute Seele des Nestbaums. Ohne sie gäbe es den gesamten hier ansässigen Orden ebenso wenig wie die Spiegelschmiede eine halbe Meile weiter nördlich. Nicht, dass Aćh sich groß auf die wunderschönen Vögel achten konnte. Denn kaum war Aćh ins Freie getreten, lief sie schnurstracks in eine herbeieilende Temm hinein.

Es war tatsächlich Yrbstschly, die nach ihr gerufen hatte. Die Kleine begann prompt zu sprechen: „Endlich erreiche ich dich! Aćh, es ist schrecklich. Ich hatte Turr zu den Minenarbeitern eskortiert...“

Aćh seufzte. Die Hüter sollten eigentlich keine Takuri allein zur Mine bringen. Die Sprengungen in den Minenstollen waren sehr kontrolliert und eine von einem übereifrigen Feuertakuri entzündete Lunte konnte unschöne Konsequenzen haben. Aćh setzte einen tadelnden Blick auf, wurde jedoch von Yrbstschly unterbrochen, ehe sie überhaupt etwas sagen konnte.

„Beschwere dich bitte nicht. Ich habe deine Oma gefragt, und sie meinte, ausnahmsweise sei es in Ordnung. Die Minenmeisterin höchstpersönlich brauchte dringend einen Takuri, um mit seiner Hilfe die nach einem Bergbeben verstellten Spiegel in den Stollen neu zu kalibrieren. Bei dieser ganzen Bürokratie, wenn sie einen offiziellen Antrag gestellt hätte, dann wäre Turr erst in einer Woche bei ihnen gewesen. Die Menge an Mera, die dadurch verloren gegangen wäre...“

„...und Turr ist klein und vergleichsweise brav, darum nahmst du ihn mit“, beendete Aćh den Satz, „Ich verstehe. Was ist geschehen?“

Angesichts der Tatsache, dass Yrbstschly sich nicht mehr in Begleitung eines kleinen Takuri befand, bot sich eine gewisse üble Vermutung an. Immerhin sah Aćh in den letzten Sonnenstrahlen der untergehenden Sonne keinen Rauch von den Stolleneingängen an den Hängen des Kuolema-Gebirges aufsteigen. Zu einer Explosion schien es nicht gekommen zu sein. Doch wo war Turr nun?

„Ich tat wie geboten und brachte Turr tief in die Stollen. Hatte gehörig Sufarsaft zur Sicherheit mitgenommen, und sein liebstes Knabberseil. Doch als kurz bevor wir den ersten verstellten Spiegel erreicht hätten, hat sich Turr einfach so in Luft aufgelöst!“, berichtete Yrbstschly schlotternd und bestätigte so Aćhs Vermutung. Sie kniete sich zur Temm nieder.

„Etwas erzählst du mir noch nicht, oder? Warum verängstigt dich sein Verschwinden derart?“

Feuertakuri waren nun mal bekannt dafür, dass sie sowohl Zeit als auch Raum bezwingen konnten. Die Zeit, indem sie nach jedem Tod ein neues Küken wurden. Den Raum, indem sie in einem Flammenbausch verschwinden und an einem anderen Ort wieder auftauchen konnten. Diese Teleportationseigenschaften waren schließlich der Grund, warum diese Niststätte hier auch als Spiegelproduktionsstätte fungierte: Aus der Asche der Feuertakuri wurden in der Spiegelschmiede eine halbe Meile weiter nördlich wertvolle Spiegel mit geheimnisvollen Kräften der Teleportation gefertigt. Insofern sollte ein verschwundener Feuertakuri eine Hüterin nicht derart in Furcht versetzen, ja, nicht einmal überraschen.

Yrbstschly biss sich auf die Lippen, atmete dreimal tief durch und plapperte dann los:

„Nun, Hüterin Aćh, Turr ist nicht einfach nur in einem Flammenbausch verschwunden. Nein, plötzlich ging so ein bläuliches Glitzern über ihn, und glaube mir, ich habe ein leises Zittern im Boden und ein fernes Donnern vernommen, und eine Stimme, die etwas schrie.“

Yrbstschly räusperte sich und sprach mit einer theatralischen, volltönenden, tiefen Stimme einige Aćh völlig fremd erscheinende Worte: „Narbi gicein, prento varafera tarkaran nuo dragos! Dirqo on te bini earim!“

Mit ihrer gewöhnlichen, hohen Stimme fuhr Yrbstschly fort:

„Und dann verschwand Turr in einem Flammenwirbel. Einem blauen! Das ist unnatürlich, sage ich dir, unnatürlich! Ich habe mich so schnell wie möglich hierher zurück gemacht, aber der Weg aus den Minenstollen ins Freie war ein langer, und dann auch noch den Hang hinunter... es sind mehrere Stunden seit Turrs Verschwinden vergangen. Und meine Sorgen wachsen mit jeder Minute. Bläuliches Glitzern!“

Aćh kratzt sich am Kopf. Das war tatsächlich äußerst mysteriös. Nun war sie auch beunruhigt.

Aćh zog ihre tulgorische Steinflöte aus einer Seitentasche hervor. Die Flöte hatte vier Löcher und ähnelte mit etwas Fantasie einem sitzenden Takuri. Das Äußere ihrer Steinflöte hatte Aćh selbst zu bearbeiten gehabt, als sie sie erhalten hatte. Damals, als sie höchstoffiziell zu einer Takuri-Hüterin ernannt worden war.

Die Traditionen von Tulgor sagten, dass zum Trainieren und Führen eines Takuri außergewöhnliche Fähigkeiten nötig waren, die man durch eine lange und harte Ausbildung erwerben musste. Ein Takuri-Hüter zu sein war eine große Ehre. Und zum Abschluss ihrer Ausbildung erhielten neue Hüter eben ihre eigenen tulgorischen Steinflöten, um auch über lange Strecken mit den ihnen anvertrauten Feuervögeln kommunizieren zu können.

Das Äußere der Flöte mochte von Aćh selbst geschlagen worden sein und nicht von der größten Handwerkskunst zeugen. Das Innere der Flöte hingegen war sehr exakt gefertigt worden, von einer Temm, die beinahe ihr ganzes langes Leben hier am Nestbaum verbracht und die Kunst des Steinformens perfektioniert hatte. Sie wusste genau, wie man die Gänge durch den Stein zu ziehen hatte und wo man die Löcher hervorkommen lassen musste, damit das fertige Produkt die gewünschten Melodien hervorbringen konnte. Und in solchen Situationen war Aćh äußerst froh darüber.

Aćh setzte die Flöte an ihre Lippen und blies sanft hinein. Ein langsam anschwellender Ton, der dem Heulen des Windes durch die Schluchten des Kuolema-Gebirges ähnelte. Dann ließ sie ihre Finger geübt über die Löcher springen. Eine kurze, wilde Melodie, die abrupt endete. Turrs Melodie. Wie ein Name war sie, und vielleicht war sie gar effektiver als Turrs tatsächlicher Name. Manchmal weigerte sich der kleine Schlingel doch aus Prinzip, auf „Turr“ zu hören.

Aćh wiederholte Turrs Melodie noch einige Male, jeweils lauter und besorgter, und hängte einige dissonante Töne an, die Dringlichkeit und Sorge ausdrückten.

Kein Turr erschien. War er zu weit weg, um die Musik zu hören? War ihm vielleicht gar etwas zugestoßen? Yrbstschly betrachtete Aćh mit von Sorgen faltigem Gesicht. Aćh setzte eine betont gelassene Miene auf und verkündete: „Na, dann bleibt uns nur eines übrig: Ab ins Aschelager mit uns.“

Temm waren erheblich kleiner als Menschen und konnten im Rennen kaum mit ihnen mithalten. Darum ließ Aćh Yrbstschly auf ihre Schultern klettern. Gemeinsam eilten sie so ins Aschelager.

Das Aschelager war gemeinsam mit einigen anderen Gebäuden der Hüter und Spiegelbauer am Fuße des Nistbaums gebaut worden. Lange, hohe Regale mit verschiedensten Kisten und Schachteln bedeckten die kreisrunden Wände. Zielstrebig eilte Aćh zu einer Holzdose mit der Aufschrift „Turr“. Sie leerte ein Stückchen von Turrs Asche in eine speziell dafür eingebaute Einbuchtung an der Steinflöte. Ein wenig der Asche stieb in ihre Nase, die unangenehm zu kribbeln begann.

Aćh kannte die Anzeichen ihrer Niesreizattacken gut genug, um sich rechtzeitig abzuwenden und zu verhindern, die Aschen dutzender Takuri aus ihren Schachteln zu niesen. Leider hatte sie nicht bedacht, dass aktuell eine Temm auf ihren Schultern saß. Bei ihrem rasanten Umdrehen streifte Yrbstschly das Regal und rempelte Turrs Aschedose zu Boden.

Nachdem das letzte „Hatschi“ Aćhs Kopf durchgeschüttelt hatte, erkannte sie Turrs Dose umgedreht am Boden, während ein Windhauch den größten Teil seiner Asche effizient zur Tür hinauswehte. Und auch die Einbuchtung ihrer Steinflöte war wieder aschefrei. Aćh unterdrückte einen Fluch. Zerbrochenen Spiegeln soll man bekanntlich nicht nachtrauern, sondern stattdessen neue anfertigen.

Yrbstschly kletterte von Aćhs Rücken und hob die fast leere Dose von Turrs Asche auf. Mit großen Augen blickte sie zu Aćh hoch:

„Oh nein, oh nein, das wollte ich nicht! Verzeih mir, Hüterin Aćh. Was sollen wir nun nur...“

„Das ist nicht deine Schuld. Ich war das Trampeltier hier. Hast du dir wehgetan?“

Yrbstschly betastete ihren Glatzkopf sorgfältig und schüttelte selbigen dann. Aćh atmete auf.

„Dann ist doch alles gut. Ich habe noch genug Asche übrig, um ihn zu rufen.“

Sie griff an ihre bronzene Halskette und löste das linkste Element daran. Eine kleine Dose mit einer Prise von Turrs Asche darin. Sie hatte doch gewusst, dass dies eines Tages nützlich sein könnte!

„Ich habe kein gutes Gefühl dabei, Aćh“, murmelte Yrbstschly.

„Noch können wir hoffen, dass wir nicht mehr brauchen werden“, versuchte Aćh, optimistisch zu bleiben. Entschieden kippte sie die Dosis in ihre Steinflöte. Dann lief sie nach draußen und winkte den ersten vorbeifliegenden Takuri zu sich. Sie kraulte diesen am Nacken, bis er wohlig gurrend einige Funken aus seinem Gefieder verstreute.

Die Funken genügten, um die Turrs Asche in der Steinflöte zu entzünden. Rauch stieg auf und immer mehr Ascheflocken wurden mit leisem Plopp an den fremden Ort gesogen, an dem sich Turr aufhielt. Wo auch immer das sein mochte. Die Verbindung stand.

Erneut blies Aćh in ihre Flöte und spielte Turrs Melodie, betont besorgt und dringlich.

Kaum waren die letzten Flötentöne leise verklungen, entzündete sich aus dem Nichts ein Flammenwirbel in der Luft über Aćhs Kopf.

Ein kleiner Feuervogel fiel aus dem Feuerbausch herunter, prallte ungelenk auf Aćhs Kopf, krallte sich in ihre langen Haare und rollte in ihre Arme. Leise gurrend rieb der Takuri seinen Kopf an Aćhs Schulter. Turr. Aćh atmete erleichtert auf. Mit ihm hatte sie schon eine besondere Bindung gehabt, als sie noch ein kleines Kind gewesen war. Vorsichtig streichelte Aćh die feinen Federn an Turrs Nacken. Flammen flackerten über Turrs Gefieder, und Aćh zog ihre Hand rasch wieder zurück.

Das flackernde Gefieder und der Blick in Turrs Augen verrieten Aćh, dass er aufgeregt war, ja, gar ängstlich. Yrbstschly hatte gesagt, ein seltsames blaues Glitzern habe ihn überzogen, ehe er verschwunden war? Magie? Aber nicht Magie eines Temm, das hätte Yrbstschly spüren können. Ein Höhlenwicht? Ein menschlicher Magier?

Hin und wieder tauchten Pilgerer von nah und fern am Nestbaum auf, die sich etwas von den Takuri erhofften. Asche, Federn, Knochen, Zutaten für irgendwelche Tränke oder Pulver. Solange die Takuri fürs Stiften dieser besonderen Ingredienzien nicht leiden mussten, folgten die Hüter den Wünschen meistens. Gegen entsprechendes, angemessenes Entgelt, verstand sich. Konnte es sein, dass ein solcher Hexer hinter Turrs Verschwinden steckte?

Aćh und Yrbstschly informierten ihre Vorgesetzten über den Vorfall und den Unfall mit Turrs Aschedose. Diese ließen die gesamte Umgebung des Nestbaums nach einem Übeltäter absuchen, ohne Erfolg. Turr wurde beruhigt und flackerte eine Stunde nach seiner Rückkehr kaum mehr auf, nur noch einige Funken sprühten hin und wieder von seinen Schwanzfeder.

Dennoch machte sich Aćh nicht gleich wieder daran, neue Asche für Turrs Aschelager zu sammeln. Die Sammelprozedur war anstrengend und nicht immer angenehm für die Takuri, wie auch das Teleportieren kräftezehrend für sie war. Nachdem Turr zweimal eine unbekannte, potenziell sehr große Strecke gesprungen war, wollte sie ihn sicher nicht überstrapazieren. So bald würden sie die Asche ja hoffentlich nicht brauchen.\bigskip

***\bigskip

Barz starrte verwirrt auf die Schulter des großen Lifornus. Noch vor wenigen Sekunden hatte dort ein kleines brennendes Vögelchen gesessen. Nun, eine spontane Stichflamme später, war von diesem Vogel nichts mehr zu sehen bis auf ein kleines Häufchen Asche auf dem edlen Gewand des Zauberers. Leise Flötentöne schwangen einen Augenblick lang noch in der Luft, dann waren auch sie verklungen.

„Du hast ihn gesehen, oder?“, rief Lifornus beinahe flehend, „Mein Ritual funktioniert!“

Barz schluckte eine bissige Bemerkung über Lifornus‘ Wagemut runter und nickte stumm. Ein waschechter Phoenix. Es gab sie tatsächlich.

Während der große Lifornus stolz seinen spitzen Zaubererhut richtete und sich den Schweiß von der braunen Stirn wischte, streckte Barz seine Hand nach den Brandflecken auf Lifornus‘ Gewand aus.

„Dürfte ich vielleicht seine Asche einsammeln?“


\newpage
\section{Turr verschwindet und ein Fremder erscheint}

\az{Jahr 63}

Zwei Monate später\bigskip



„Aćh! Aaaaaćh! Hüterin Aćh! Da sitzt ein Mann einfach so im Nest der Takuri!“, erklang eine hohe Stimme aus einer Nesthöhle über ihr. Klang nach einem Temm. Aćh verdächtigte Yrbstschly. Sie unterbrach das Putzen der Ascheflecken am Boden vor ihr, streckte ihren Kopf aus der Nisthöhle und blickte am Stamm des Nestbaums in die Höhe.

„Was ist denn jetzt?“, rief sie hoch.

„Ein Fremder! Da sitzt ein Fremder in Turrs Nest!“, kam die Antwort abrupt.

Die Stimme klang wirklich verängstigt. Aćh ließ ihr Putzzeug fallen und schwang sich zwischen Plattformen am Nestbaum in die Höhe, bis sie die Baumhöhle erreichte, aus der eine Temm eifrig winkte. Es war tatsächlich Yrbstschly. Und Yrbstschly war nicht allein.

Ein in einen langen braunen Mantel gekleideter Mann stand zwischen verstreutem Nestmaterial weiter innen in der Nisthöhle und beäugte Yrbstschly vorsichtig. Seine eine Hand hatte der Fremde beschwichtigend in die Höhe gehoben, die andere langte an einen mit vielen kleinen Säckchen behängten Gürtel. Die Temm hatte wiederum ihre Hände gehoben und ließ ein grünliches magisches Glühen von ihnen ausgehen.

Absurderweise war der erste Gedanke, der Aćh durch den Kopf schoss, ob man den Mantel des Fremden wirklich einen Mantel nennen konnte. Die Schulterpartie erinnerte sie eher an mehrere übereinandergeschichtete ornamentierte Tücher. Ein bisschen wie die die Kleidung der Nomaden aus der Wilden Wüste, die sie damals gesehen hatte, als ihre Mutter sie auf eine diplomatische Reise ans andere Ende des Landes mitgenommen hatte.

Am Rücken des Fremden, unterhalb seines Köchers, hing ein schon leicht zerfleddertes Fell im selben Braun wie sein Mantel. Nebst den mysteriösen Säckchen an seinen Gurten waren auch zahlreiche Taschen in Mantel eingelassen, vorne saß gar eine große Bauchtasche, und weiter hinten in der Höhle sah Aćh einen weiteren beladenen Rucksack herumliegen. Der Mann war vollkommen überladen. So konnte man doch nicht umherreisen.

Wie war er nur hierher gelangt? Sie befanden sich einer Höhle im Nestbaum der Takuri, dutzende Meter über dem Boden. Bestimmte hätte jemand gesehen, wenn dieser Fremde das Gerüst erklommen hätte. Sie blickte sich um, sah jedoch nirgends einen Takuri-Spiegel liegen. Nun, es war nicht relevant. Zunächst einmal war wichtig, sicherzustellen, dass vom Neuankömmling keine Gefahr ausging. Er hatte einen Bogen umgeschnallt und trug einen Köcher auf seinem Rücken. Die Pfeile darin könnte er im Notfall auch im Nahkampf nutzen.

Aćh zückte ihr goldenes Schwert und hielt es dem Fremden vor die Kehle, auf dass er nicht auf dumme Ideen komme.

Laut fragte sie: „Wer bist du? Woher kommst du? Was suchst du hier?“

Der Fremde kniff seine Augen zusammen, verzog seine Miene, wartete einen Augenblick und begann dann, mit seinen Händen zu gestikulieren. War er stumm?

Nein, stellte sich heraus. Denn nachdem Aćh ihn eine Zeit lang unverständig angeguckt hatte, ließ der Fremde ernüchtert seine Schultern sinken, legte seine Hand an seine Brust – Handfläche nach außen – und sprach: „Barz.“

Wie eine Begrüßung.

Aćh konnte den Akzent nicht einordnen, ebenso wenig die Sprache, und erst recht nicht die Herkunft des Fremden. Aber feindlich eingestellt wirkte er nicht. Sie senkte ihr Schwert wieder, legte ihre eigene Hand auf ihre Brust und sprach ihm nach:

„Barz.“

Sie versuchte, das fremde Wort zu wiederholen, doch ihre Zunge stolperte über den Zischlaut am Ende. Der Fremde schüttelte freundlich lächelnd seinen Kopf und wiederholte: „Barz. Barz.“

Dabei zeigte er zunächst auf sich, ehe er seine Nase rümpfte und dann auf Aćh zeigte. Jetzt war Aćh noch verwirrter. Sie rümpfte ebenfalls ihre Nase und sprach: „Barzbarz. Barzbarz Barz.“

Der Zischlaut machte ihr immer noch zu schaffen, doch glaubte sie, die fremde Floskel immer besser imitieren zu können. Der Fremde grinste plötzlich breit und gluckste auf. Aćh wünschte sich, ihre Mutter wäre hier. Die war schon immer in diplomatischen Angelegenheiten besser gewesen. Aber nein, Nelímar musste ausgerechnet jetzt auf diplomatischer Mission in der Metropole Agarb sein.

Nun wurde das Gesicht des Fremden plötzlich wieder ernst und er starrte Aćh in die Augen. Dann tat er einige Schritte innerhalb der Höhle und zeigte nacheinander auf alle möglichen Dinge. Dazu gab er Laute von sich, die vermutlich Wörter in seiner Muttersprache waren – was für eine faszinierende fremde Sprache das war, mit einem besonderen Singsang in der Stimme und so wenigen harten Lauten. Yrbstschly ließ ihr bedrohliches magisches Leuchten wieder anschwellen, als der Fremde auf sie zeigte, ansonsten rührte sie sich nicht vom Fleck.

Am Ende zeigte der Fremde auf einen vor der Höhle vorbeirauschenden Takuri – „Fönićs“ – und dann noch einmal auf sich selbst – „Barz.“

Ach so, das war ein Name! Blut schoss in ihre Wangen. Etwas beschämt wiederholte Aćh Barz‘ Prozedere. Auch sie zeigte auf den Takuri vor der Baumhöhle – „Takuri“ und auf Barz – „Barz.“ Dann endete sie mit ihrem eigenen Namen – „Aćh.“

„Agk“, wiederholte Barz. Aćh grinste. Nun war es an ihr, den Kopf wegen einer Aussprache zu schütteln.

Noch ehe Aćh Yrbstschly vorstellen konnte, trat Barz auf einmal ihren Schritt näher. Rasch hatte Aćh ihr goldenes Schwert wieder gehoben, und Barz stand wieder stockstill da.

Da vernahm Aćh ein Geräusch, das verdächtig nach dem Knacken einer Eierschale klang.

Das war seltsam. In Turrs Nest sollten gar keine Eier liegen. Ohnehin war es unglaublich selten, dass ein Takuri ein Ei legte, aktuell gab es am gesamten Nestbaum wohl nur eines. Was war hier los? An Yrbstschlys und Barz‘ verwirrten Blicken erkannte Aćh, dass sie nicht die Einzige war, die das Knackgeräusch nicht einordnen konnte.

Dann blickte Barz an sich herunter und Aćh folgte seinem Blick. Sie glaubte, eine Bewegung in der Tragetasche vor seinem Bauch zu erkennen. Konnte es sein...

Erschrocken und etwas hilflos blickte Barz wieder zurück zu Aćh. Und sie glaubte zu verstehen.

Mit einem vorsichtigen Seitenblick auf Yrbstschly steckte Aćh ihr goldenes Schwert weg und gestikulierte das Abziehen der Tragetasche. In Barz‘ Augen blitzte Erleichterung. Er stellte seine Tragetasche ab, öffnete den Verschluss und zog vorsichtig, ganz vorsichtig, ein knackendes Ei heraus.

Das Ei war wirklich schön, selbst im Vergleich zu den leuchtenden Takuri-Eiern, die Aćh hier am Nestbaum schon gesehen hatte. Seine Oberfläche war relativ glatt und dunkel. In der schimmernden Schale spiegelte sich das in die Baumhöhle fallende Licht. Keine Ahnung, zu welcher Spezies es gehörte, und was dieser Barz damit vorhatte. So groß, wie es war, konnte er es kaum mit einer Hand halten. Unwahrscheinlich, dass er noch weitere bei sich trug.

Aćhs Faszination wurde abrupt von ihrem praktisch orientierteren Geist überdeckt. Das Ei schlüpfte! Was brauchte es? Was brauchte Barz? Sie scharrte ein wenig vom feuerfesten Nistmaterial der Takuri zusammen und gebot Barz, sein Ei dorthin zu legen.

Sorgfältig, wie man ein Neugeborenes in sein Bettchen legte, platzierte Barz das Ei im Nest und trat einige Schritte zurück. Dann blickte er fragend zu Aćh.

„Was? Was brauchst du?“, fragte sie, in der Hoffnung, ihr Tonfall könne irgendwie kommunizieren, was ihre Worte allein nicht konnten.

„Kann sich das Tier von selbst befreien oder müssen wir ihm helfen?“, mischte sich nun auch Yrbstschly ein. Aćh nutzte den Moment, um auf Yrbstschly zu zeigen und ihren Namen auszusprechen, da sie zuvor deren Vorstellung vergessen hatte. Barz kümmerte sich nicht sonderlich darum, sondern blickte sie weiterhin verständnislos an. Aćh versuchte, das Öffnen einer Eierschale mit ihren Händen darzustellen, wurde jedoch von einem lauten Knacken der tatsächlichen Eierschale unterbrochen.

Eine regelmäßige Reihe langer, dünner Nadeln stach synchron durch die Schale des Eis und ließ einen Teil davon abblättern. Barz fiel auf die Knie und beobachtete ganz genau, was da vorging. Er hielt jedoch weiterhin seine Distanz.

Erneut stach die Nadelreihe durch die Schale und blätterte einen Teil davon ab. Gute Güte, waren das Zähne? Aćh erhaschte einen ersten Blick auf einen Ausschnitt des Wesens, welches hier das Licht der Welt erblickte. Graue, faltige Haut. Ein winziges Auge umgeben von einer gräulichen Fleischmasse. Bei die sieben Feuern des Himmels, was war das nur für ein Wesen?!

Dann platzte die Schale komplett auf und etwas purzelte aus dem Ei heraus.

Ein überdimensionierter Kopf, der über einen dicken, mehrgliedrigen Hals in einen massigen Körper überging. Dünne Beinchen, ein langer Schwanz und ein überlanger Unterkiefer. Und dann erst die spitzen Zähne. Ein Raubtier? Eine abartige Echse? Bei ihrem Anblick schüttelte es Aćh innerlich.

Doch als sie Barz‘ Gesicht sah, erkannte sie darin den liebevollen Blick, den Aćh Turr und den anderen Takuri schenkte. Diese groteske Echse bedeutete ihm viel.

Aktuell ging es der Echse alles andere als gut. Ihr überproportionierter Magen warf Wellen und ein leises Ächzen war daraus zu vernehmen, aber keine Atmung. Barz stürzte zur Echse und drehte sie auf ihren Bauch, auf ihren Rücken, im Kreise, klopfte sie ab, tat alles Mögliche, ohne dass es ihr besser ginge. Noch immer röchelte sie mehr, als dass sie Luft holte. Hilflos blickte Barz um sich.

Aćh hatte noch nie ein Lebewesen belebt, geschweige denn eine Echse, von der sie nicht einmal wusste, wo in ihrem Körper sich Herzen befinden könnten. Aber sie war schon einmal dabei gewesen, als eine Temm vom Nestbaum gestürzt war und ihre Großmutter erste Hilfe geleistet hatte. Sie versuchte, die unangenehmen Erinnerungen zurückzuholen. So klein war der gefallene Körper gewesen. Was hatte ihre Großmutter schon wieder gemacht?

Barz tat etwas, was Aćh ganz und gar nicht gut erschien: Er schüttelte die Echse verzweifelt. Jetzt fühlte sie, dass sie in Aktion treten sollte. Sie stürzte nach vorne, hielt Barz‘ Hände fest und blickte ihn fragend an. In Barz‘ Augen schimmerten Tränen, und er nickte hastig. Das war für den Moment eine ausreichende Erlaubnis.

Aćh nahm ihm die Echse ab und hielt sie vorsichtig in ihren Händen. Ihre Haut fühlte sich viel rauer an, als sie erwartet hätte. Und sie war so zerbrechlich, so filigran. Aćh tastete die Seite der Echse ab und spürte Muskeln, weichere Teile, dann etwas, das Rippen sein könnten. Einen Brustkorb? Sie legte ihre Hände an eine gewisse Stelle und wartete einen kurzen Augenblick, ob ihr wirklich nichts Besseres einfiel. Dann drückte sie zu. Und nochmal. Und nochmal. Immer und immer wieder. Und auf einmal begannen die dürren Beinchen der Echse zu strampeln. Ein jämmerliches Geschrei entfloh ihrem winzigen Mund. Dann wurde es still. Ihre Brust hob und senkte sich, zwar unregelmäßig, aber immerhin.

„Es atmet, es atmet!“, rief Aćh, „Doch es scheint schwach. Sehr schwach.“

Barz konnte natürlich kein Wort davon verstanden haben, aber plötzlich war er wieder neben ihr. Er zückte ein kleines helles Säcklein aus seinem Mantel, öffnete es und zog eine Prise glitzernden Pulvers hervor, welches er der Echse ins Gesicht pustete. Ein magisches Glitzern breitete sich über die Echse aus und ihr Strampeln wurde wieder kräftiger. Ihre Seite hob und senkte sich regelmäßig.

Barz‘ sorgenvolles Gesicht wurde zu einer Miene der Freude. Er stieß seine Faust in die Luft und setzte an, Aćh zu umarmen, ehe er innehielt und ihr nur dankbar zuzwinkerte. Dann nahm er die Echse an sich und streichelte ihren massigen Kopf. Ihre lange Zunge leckte sein Gesicht.

„Sabri“, sprach Barz, und zeigte auf die Echse, „Sabri.“

Aćh fragte sich, ob das ein persönlicher Name oder der Name der Spezies war. Dann fiel ihr auf, dass es aktuell komplett irrelevant war. Und auch so bleiben würde, bis sie je auf andere Vertreter dieser Spezies stoßen sollte.

„Sabri“, sprach Aćh nach.

Sie sah den Fremden anderen Augen als noch vor ein paar Minuten. Doch blieb da immer noch ein ungutes Gefühl. Dieses magische Glitzern, das die Echse bei ihrer Heilung überzogen hatte... Damals, vor zwei Monaten, als Turr verschwunden war, hatte Yrbstschly auch etwas von einem magischen Glitzern gesagt. Heute war der Fremde augenscheinlich einfach so in Turrs Nest aufgetaucht. Und wo befand sich Turr?!

Ein noch schlechteres Gefühl breitete sich in Aćhs Magen aus. Was, wenn Turr wieder verschwunden war? Sie hätte Turrs Dose im Aschelager seit dem Missgeschick inzwischen auffüllen sollen. Doch hatte sie das immer weiter hinausgeschoben. Mist! Sie hatte keine Möglichkeit mehr, ihn zu rufen!

Sie befahl sich selbst im Stillen, Ruhe zu bewahren. Beinahe beiläufig zog sie ihre Steinflöte hervor und spielte Turrs Melodie, doch leider erschien der Takuri nicht. Doppelmist.

Naja, nun, wo die akute Notlage der Echse geklärt schien, war es an ihnen, zu ihren Vorgesetzten zu gehen und um Rat zu fragen.\bigskip







Aćh geleitete Barz aus dem Nest und vom Baum runter. Yrbstschly krabbelte auf Aćhs Schultern und warf dem Fremden hin und wieder misstrauische Blicke zu. Doch Barz kümmerte sich nicht groß um sie. Es war ein Wunder, dass er lange genug vom Blick auf die Echse in seinen Armen abließ, um seine Umgebung zu erkennen.

Zunächst warf er einen beunruhigten Blick auf die filigran wirkende Holzkonstruktion, die sich um den steinernen Nestbaum wand und es den Hütern erlaubte, rasch die verschiedenen Nester zu erreichen.

Während Aćh ihn auf die andere Seite des Baums führte, auf dass sie den dortigen Aufzug erreichen konnten, öffnete Barz staunend seinen Mund.

Aćh konnte es ihm nicht verübeln. Für sie war der Blick auf die steilen Felswände des Kuolema-Gebirges ein alltäglicher, und selbst sie war stets beeindruckt davon. Bis hoch in die Wolken ragten die Gipfel des Gebirges, dessen Name ja buchstäblich „Berge in den Wolken“ bedeutete. Das Gestein des Kuolema war in vergleichsweise hellen, fahlen Tönen gehalten. So konnte man kaum erkennen, wo Stein in Schnee und Eis überging, und wo in wabernde Wolken.

An manchen Tagen reichten die Wolken so tief, dass sie bis an den Nestbaum der Takuri reichte. Dur Nestbaum stand stolz und steinern auf den westlichen Ausläufern des Kuolema, gerade an der Grenze zur roten Steppe Tulgors.

Heute aber standen die Wolken hoch, und der Blick auf den Großteil des Gebirges war klar. Nur die Gipfel waren nicht zu sehen. Die Gipfel waren nie zu sehen. Manch einer mochte behaupten, es gäbe nicht einmal welche, und man könnte ihn nicht widerlegen. Aćh vermutete aber eher, dass ein magischer Fluch die Wolken am Auflösen hinderte.

Hier und da waren an den Hängen des Kuolema auch dunkle Eingänge in tiefe Höhlensysteme und Stollen zu erkennen. Darin arbeiteten die Minenarbeiter und entrangen dem gnadenlosen Gestein seine gut gehüteten Geheimnisse. Sie förderten neben Gold, Silber und Eisen auch uralte magische Mera-Steine. Geschickt in die Stollen gelenkte Strahlen roten Mondlichts ließen die Mera-Steine in den Stollenwänden aufglühen, woraufhin sie grob aus dem Felsen geschlagen werden konnten. Die Brocken wurden dann weiter inland über komplizierte Prozesse ganz aus dem Gestein befreit, geschleift und etwa in mächtige magische Amulette und Knochenhelme eingesetzt.

Aćh sah, wie Barz‘ Blick zu dem riesigen Gebäude eine halbe Meile weiter nördlich schweiften, aus welchem Dampf aufstieg.

Die Spiegelschmiede.

Dort wurden neben gewöhnlichen Spiegeln hin und wieder aus der Asche von Feuertakuri wertvolle Takuri-Spiegel hergestellt. Auch diese wurden später weiter ins Landinnere Tulgors geliefert, wo sie elegante Rahmen verpasst kriegten und an hohe Fürsten verkauft wurden.

Während die drei das Gerüst am Nestbaum herunterkletterten, ließ Yrbstschly weiterhin hin und wieder ein bedrohlich wirken sollendes grünliches Glühen über ihre Handflächen wabern, doch wirkte Barz davon eher interessiert als beunruhigt. Insbesondere musterte er fasziniert den goldenen Sand, der aus dem grünlich glühenden Ball der Magie herunterrieselte. Doch vor allem kümmerte er sich um das kleine Echsenwesen, das in seiner Bauchtausche ruhte und leise wimmerte.

Aćh vermutete nicht, dass aktuell eine Gefahr von ihm ausging, ließ ihre Hand aber zur Sicherheit demonstrativ auf dem Griff ihres goldenen Schwerts ruhen. Mit ihrem Daumen fuhr sie über die gemusterte Mondsichel, die als Parierstange diente. Der Mond, insbesondere sein rotes Licht, waren für die Tulgori von hoher Wichtigkeit, sowohl für uralte Traditionen als auch fürs Fördern von Mera-Steinen. Insofern ergab es natürlich Sinn, dass ein zeremonielles Objekt wie dieses goldene Schwert eine Mondsichel enthielt.

Hier am Nestbaum der Takuri wurden solche goldenen Schwerter, Speere und ähnliche Waffen eigentlich nur zu zeremoniellen Zwecken benutzt. Sie waren aus gewöhnlichem Metall gefertigt und nur mit einer dünnen Schicht Gold überzogen, die in einem tatsächlichen Kampf zerbröckelen würde. Doch aktuell waren die goldenen Überzüge der Waffen der Hüter allesamt intakt, und die in die zeremoniellen Knochenhelme eingelassenen Mera-Steine funkelten allesamt in mattem Blau – ein Zeichen, dass sie schon seit langem nicht mehr zum Blutvergießen genutzt worden waren. Die intakte goldene Farbe und die blauen Steine zeugten von den friedlichen Bedingungen hier am Nestbaum. Selbst die Höhlenwichte aus den nahegelegenen Höhlengängen im Kuolema stritten sich höchstens mit den Bergleuten in den Stollen und ließen die Takuri-Hüter in Ruhe.

Dennoch hatte Aćh viel Zeit damit verbracht, die Schwertkunst zu erlernen, um in einem Ernstfall zur Verteidigung des Baumes beitragen zu können. Und weil die Schwertübungen etwas Entspannendes, Anmutiges hatten, wie eine Art Tanz.

Echte Konflikte waren alles andere als das, sondern dreckig und unangenehm, davor hatte sie ihre Großmutter oft genug gewarnt. Oma Òkôkó hatte noch die blutigen Konflikte mit den marodierenden Banditenbanden aus dem Südwesten miterlebt. Sie hatte danach geschworen, nie wieder ein Schwert zu führen. Den Schwur hatte sie bislang halten können. Nun war sie eine der Obersten des Ordens der Hüter und eine gute Ratgeberin. Sie würde wissen, was mit Barz zu tun war.\bigskip







Òkôkó runzelte ihre Stirn und murmelte leise, sie wünschte sich, Aćhs Mutter wäre hier. Als Diplomatin kam Nelímar viel umher in den umliegenden Gebieten, und sie kannte viele Dialekte. Vielleicht hätte sie auch den von diesem Barz‘ erkennen können.

Und ein solches Tier wie diese Sabri hatte Òkôkó noch nie gesehen.

Sie kratzte sich am Kinn.

„Und dein Turr ist immer noch verschwunden?“

Aćh nickte angespannt: „Ich vermute, dass sich der Vorfall von vor zwei Monaten wiederholt hat. Und dass Barz irgendwie involviert ist. Auch wenn er jetzt gerade nicht so wirkt, als würde er einer Fliege etwas zu Leid tun wollen.“

Aćh deutete hinüber zu Barz, welcher gerade mit seiner kleinen Echse umhertollte und ihnen beiden keine Aufmerksamkeit schenkte.

Oma Òkôkó legte ihren Kopf schief: „Nun, es gibt einen großen Unterschied dazwischen, Schaden anzurichten und Schaden anzurichten zu wollen.“

„Verzeih mir“, nickte Aćh, „Ich hätte inzwischen neue Asche von Turr sammeln sollen. Nur wegen mir können wir ihn nun nicht zurückrufen.“

„Ach so? Hätte wirklich kein anderer Hüter die Asche sammeln können?“, meinte Òkôkó schnippisch.

„Das meinte ich nicht. Turr wurde mir anvertraut. Seine Sicherheit und Pflege sollte meine Verantwortung sein. Wenn ihm etwas zustößt, ja, vielleicht gar etwas zugestoßen ist, dann geht das auf meine Kappe.“

„Wir werden klarkommen, was auch immer genau hier los ist. Konnten wir bislang stets“, sprach Òkôkó zuversichtlich, „Doch was wollen wir nun tun? Was tun, was tun?“

„Wir könnten Turr suchen gehen, doch die Welt ist klein für einen Takuri und groß für uns. Ich fürchte mich um ihn. Er hat bislang noch selten von selbst zum Nestbaum zurückgefunden, wenn er einmal zu weit wegflog.“

„Wir bräuchten etwas, das ihn zu uns führt. Oder uns zu ihm führt. Versuche, Kontakt mit diesem Barz aufzunehmen. Finde heraus, ob er weiß, wo sich Turr aufhält.“

Aćh warf einen Blick zurück zu Barz. Dieser hatte aufgehört, mit seiner Echse zu spielen, und musterte fasziniert die vielen eleganten goldenen Reifen um Oma Òkôkós Hals.

„Ich werde mein Bestes geben, doch bin ich nicht zuversichtlich“, sagte Aćh, „Aber wer sonst könnte wissen, wie wir Turr wiederfinden können?“

Oma Òkôkó schnippte mit ihren Fingern: „Der Hüter der Zeit würde es wissen.“

„Ja, der Hüter der Zeit würde es wissen“, nickte Aćh nachdenklich, „Und Mama ist gerade in der Metropole Agarb unterwegs! Wir können ihr einen Falken schicken. Und dann könnte sie für uns mit dem Hüter konversieren, sofern er sich die Zeit nehmen kann.“

„Natürlich, sende den Falken an deine Mutter. Doch Angelegenheiten mit dem Hüter der Zeit sollte man meiner Erfahrung nach lieber persönlich klären. Vier Nächte wollen wir darüber schlafen. Falls Turr bis dann noch nicht zurück ist, schicke ich dich auf dem nächsten Hängeschiff nach Agarb. Wenn Nelímar bis dahin ein bisschen ihre Verbindungen zu den Stadtwachen spielen lässt, befindest du dich zwei, drei Tage später bereits in einer Unterredung mit ihm.“

„Oma, vier ganze Nächte? Turr kann in dieser Zeit alles Mögliche geschehen.“

„Er ist ein Takuri, er wird es überleben. Egal, was für ein Leid ihm zustößt, er kann es wieder vergessen, solange wir ihn zeitig finden. Doch wollen wir auch den Hüter nicht unbedacht stören. Mir scheint, vier Tage des Wartens sind eine angemessene Zeit.“

„Zwei Nächte!“

„Drei.“

„Wie du willst, Oma. Nach drei Nächten brechen wir zum Hüter auf, falls Turr bis dahin nicht zurück ist.“

Aćh neigte ihren Kopf und wandte sich zum Gehen.

„Ah, und kümmere dich um unseren seltsamen Neuankömmling“, meinte Òkôkó beiläufig, „Er hat vermutlich vielleicht mit Turrs Verschwinden zu tun. Wenn er dir irgendetwas verraten kann... nun, wahrscheinlich ist es am geschicktesten, mit ihm zu sprechen zu versuchen. Oder noch besser, nimm ihn gleich mit, wenn du zum Hüter der Zeit pilgerst. Der Hüter würde wissen, was mit ihm anzufangen ist.“

Aćh nickte ebenfalls und winkte Barz zu sich. Barz, der zuvor gerade eine Zeit lang Òkôkós schnell sprechenden Mund angestarrt hatte, vermutlich ohne auch nur ein Wort zu verstehen, eilte geschwind zu ihr. Wenigstens schien er Aćhs Handzeichen zu verstehen.

Bösartig wirkte er auch nicht. Stattdessen folgte er Aćh brav wieder ins Freie, während er mit einer Hand Sabri in seiner Bauchtasche streichelte.

Tja, jetzt hatte sie den Fremden am Hals, bis sie Turr ausfindig machen konnten. Das könnte interessant werden.\bigskip







Aćh hatte Barz rasch in ihrem und Yrbstschlys gemeinsamen Zimmer ein behelfsmäßiges Bett eingerichtet und ihm im hiesigen Chaos etwas Platz freigeräumt, auf dass seinen riesigen Rucksack und einige seiner vielen Taschen ablegen konnte. Nebendran richtete Aćh ein Nest für die kleine Echse Sabri ein, die Barz aber nicht aus seiner Nähe lassen wollte. Immer wieder schielte er vorsichtig zu den Takuri, die von Zeit zu Zeit um den Nestbaum schwirrten. Insbesondere wenn sie Funken sprühten oder ihr Gefieder aufflammte, kniff er beunruhigt seine Augen zusammen. Und immer wieder blickte er auch vorsichtig auf Aćhs Arme, denn unter ihren Armschonern lugte die eine oder andere Brandnarbe hervor. Das Hüten der Takuri war nun mal kein Kinderspiel, insbesondere das Hüten von jungen, wilden, rasch aufgeregten Exemplaren.

Aćh ahnte, dass das viele Feuer ihm Sorge bereitete, und führte ihn darum als nächstes zur Schutzanlage auf der anderen Seite des Nestbaums. Dort könnte er seine Haare und seine Kleidung mit Sufar einreiben. Sufar war eine klare, süßlich riechende Substanz, die einen zumindest eine Zeit lang vor dem Feuer der Takuri schützte.

Barz betrachtete mit äußerst großem Interesse, wie Aćh ihm die Feuerresistenz von Sufar demonstrierte, indem sie ihre Hand ins Sufarbecken tauchte und danach schmerzlos in einen Kessel voller glimmenden Takuri-Kots langte.

Als Aćh ihren eigenen Umhang demonstrativ ins Sufarbecken eintauchte und Barz gebot, dasselbe zu tun, schien dieser jedoch etwas dagegen zu haben, seinen Mantel einzutauchen. Wild gestikulierend öffnete er verschiedene Manteltaschen und daran befestigte Säckchen und zeigte ihr verschiedenste Pülverchen, die er darin aufbewahrte. War er ein Kräutersammler? Würde das Sufar diesen schaden? Aćh entschied, ihn gewähren zu lassen. Er wusste am besten, was sein Mantel brauchte, und nun, wo er das Sufarbecken kannte, könnte er sich in Zukunft selbst damit schützen, wenn er es denn wollen würde.

Die beiden verließen die Schutzstation wieder. Aćh nickte im Vorbeigehen einer alten Temm zu, die mit einem komplizierten Seilmechanismus das Sufar im Becken durchmischte und so vor dem Kristallisieren abhielt. Die Temm nickte zurück und blickte Barz unverhohlen neugierig an. Aćh zuckte nur die Schultern. Beim Abendessen würde es noch genug Zeit geben, Barz vorzustellen.

Die nächsten paar Stunden verbrachte Aćh mit ihrer üblichen Tagesroutine. Takuri füttern und bespaßen, Größen und Gewichte notieren, Dreck wegräumen, alles eigentlich wie gewohnt. Doch konnte sie sich kaum darauf konzentrieren, denn ihre Gedanken schweiften immer wieder zu Turr. Dem Takuri, zu dem sie von allen die tiefste Verbindung aufgebaut hatte. Dem Takuri, der nun schon zum zweiten Mal verschwunden war. Oh, wenn sie nur nicht seine Aschedose verschüttet hätte. Wo befand Turr sich wohl? Ging es ihm gut?

Und stets war Barz dabei, der Aćh auf den Fuß folgte und äußerst interessiert alles begutachtete, was sie tat und ließ, und sei es das Wegschütten feurigen Kots im stinkenden Abfalllager.\bigskip







Während des Abendessens war Barz wie erwartet die Attraktion der Woche, wenn nicht gar des Monats. Hüter um Hüter und Arbeiter um Arbeiter tröpfelte zur Essenszeit in die große Esshalle, wo ein ungewöhnlich großwüchsiger Temm aus einer noch größeren Suppenschüssel schöpfte. Sobald Barz entdeckt wurde, wie er gemeinsam mit Aćh am Rand des Saals saß und seine Suppe löffelte, bewegten sich Hüter um Hüter und Arbeiter um Arbeiter in seine Nähe und versuchten zu erhaschen, worüber er und Aćh sich unterhielten. Irgendetwas, was ihnen vielleicht verraten könnte, wer der mysteriöse Fremde mit der rundlichen Echse war, ohne dass sie ihre Privatsphäre ganz offen brächen.

Leider konnten sich Aćh und Barz überhaupt nicht unterhalten. Aćh versuchte Barz zu fragen, wer er sei, woher er komme und ob er etwas mit Turrs Verschwinden zu tun habe.

Barz hingegen sprach Worte, die ihr nichts sagten, und gestikulierte auf verschiedenste Dinge, mit denen sie wiederum nichts anfangen konnte.

Am Ende beschränkte sich Aćh darauf, auf verschiedene, einfache Dinge zu zeigen und ihre tulgorischen Namen zu nennen, während Barz die Namen wiederholte. Eine gemeinsame Sprachbasis musste etabliert werden, ehe sie sich auch nur halbwegs unterhalten konnten. Im Laufe des Essens gelang es Barz bereits, „Suppe gut“ und „viel Salz“ zu sagen.

Nach dem Abendessen wurden wie üblich Musikinstrumente hervorgeholt, von Flöten über Trommeln bis hin zu doppelhalsigen Streichinstrumenten, die sich über einen komplizierten Mechanismus selbst bezupften. Dann begannen die Feiern und Tanzereien des Abends. Barz schaute mit großen Augen zu, wie die Hüter und restlichen Arbeiter über den Tanzboden wirbelten, während über ihnen am Himmel Feuertakuri vorbeirauschten und fröhliche Laute von sich gaben.

Der Rhythmus der Tänze wurde schneller und einige Personen wagten es, zu Barz zu treten und ihn zum Tanz aufzufordern. Schon beim ersten Mal erhob er sich bereitwillig und ließ sich anleiten. Rasch wurde klar, dass er weder ein Gefühl für den Rhythmus noch für die Bewegungsabläufe hatte, doch ließ er sich davon nicht einschüchtern. Schon bald stolperte er leider über seinen Aufforderer, und obwohl dieser beim Weghumpeln mit zusammengebissenen Zähnen betonte, dass alles in Ordnung und er nicht böse sei, winkte Barz den restlichen Abend alle weiteren Tanzaufforderungen höflich ab. Er verließ die Versammlung, ehe die Musik verklang.

Aćh folgte ihm.\bigskip






Barz lief einige Minuten vom Nestbaum und den Feiern weg. Er beobachtete aufmerksam seine Umgebung, wiederum streng beobachtet von Aćh.

Im Westen führte ein Abhang von den westlichen Ausläufern des Kuolema-Gebirges weit in die rote Steppe Tulgors hinaus. Die Steppe breitete sich von hier bis zum Horizont aus, soweit das Auge reichte. Unterbrochen wurde das allgegenwärtige Rot des Steppengrases nur von der goldenen Farbe der weiten Kornfelder der Bauern, die hier und da angebaut waren, besonders in der Nähe des Gebirges. Hier und da konnte man auch kleine Häuser mit strohgedeckten Dächern und schwarzen Schornsteinen erspähen. Weiter im Norden gab es größere Häuseransammlungen, höhere Burgen und elegantere Brückenbauten geschickterer Baumeister über wildere Flüsse, doch waren diese alle von hier aus höchstens als weit entfernte Silhouetten zu erahnen.

Doch im Gegensatz zu all dem schien sich Barz vor allem für die Position der Sonne zu interessieren. Er suchte aus einer seiner unzähligen Taschen einen Fetzen Papier hervor und versuchte vergebens, mit einem dicken Kohlestift darauf etwas zu notieren.

Nachdem Aćh ihm einen besseren Stift und eine Unterlage besorgt hatte, gelang es Barz tatsächlich, eine rudimentäre Karte zu zeichnen. Rote Steppe, Baum, Gebirge. Sonnenverlauf (es dauerte einige Versuche, bis Aćh diese gekritzelte Kugel und den Pfeil mit der soeben hinter dem Kuolema verschwundenen Sonne verband), Osten und Westen. Die Zeichen, die Barz auf die Karte kritzelte, sagten Aćh nichts, doch mit den Bildern konnte sie etwas anfangen.

Dann zückte Barz ein neues Papier und zeichnete eine zweite rudimentäre Karte. See, Haus im See, Steppe drumrum. Sonnenverlauf, Osten und Westen. Große Wassermassen im Norden, Berge im Süden und welche im Westen, dahinter noch einmal eine Steppe, dahinter noch einmal Berge. Barz tippte wiederholt auf die Siedlung im See und auf sich selbst, dann auf die Karte des Nestbaums und auf Aćh.

Okay, damit konnten sie arbeiten. Die grobe Karte von Barz‘ Heimat sagte Aćh zwar noch nichts, sie war aber sicher ein guter Start.

Aćh ergänzte noch einige Details auf der Karte des Nestbaums. Von hier aus mochte die rote Steppe Tulgors schier endlos scheinen, doch war sie das in Realität natürlich nicht der Fall. Im Südwesten der Steppe lag die Wilde Wüste, welche sich seit Jahrhunderten langsam, aber stetig immer mehr der Steppe einverleibte. Im Süden lag ein weiteres Gebirge. Im Norden die Steilklippen des Ozeans. Und auch die Grenzen Tulgors vermochte Aćh einzuzeichnen.

Das Land Tulgor beanspruchte relativ klare Grenzen für sich. Aller Boden vom Kuolema bis zum Ende der Steppe galt als Reich des weisen Rats der Fürsten. Was weiter draußen lag, kümmerte sie ebenso wenig wie der weite Ozean hinter den steilen Klippen im Norden oder die Reiche jenseits der Berge und der Wüste.

Ob die Tulgori hier in diesen Landen entstanden oder aus Steppe, Wüste, Bergen oder Meer angereist waren, das war eine Frage, die sich nicht so leicht beantworten ließ. Die ersten Schriften, die die Erforscher der Geschichte hatten aufspüren können, stammten bereits aus einer Zeit, als Menschen schon seit mehreren Generationen in den hiesigen Dörfern lebten und sich nicht groß um ihre Herkunft scherten.

Die wenigen Entdecker, die in der niedergeschriebenen Vergangenheit in fremde Lande aufgebrochen waren, berichteten bloß von Riesen. Hinter den Bergen im Osten gab es welche mit großen Hörnern und langen Kinnen. Hinter den Bergen im Süden gab es welche mit kleinen Köpfen, die sich mit Skeletten schmückten. In der Wilden Wüste im Westen gab es welche mit langen Stoßzähnen und Bauchtaschen in ihrer Haut. Und der Ozean im Norden wimmelte ohnehin nur so von gehörnten Riesen, ungeheuren Kreaturen und Gefahren. So waren die meisten Tulgori zufrieden, sich in ihrem fruchtbaren, riesenlosen Fleckchen der Welt niederzulassen und das Schicksal durch weite Reisen nicht allzu sehr herauszufordern.

Aćh ergänzte auf Barz‘ Karte die Metropole Agarb an der Grenze zwischen Wüste und Steppe, die wohl bekannteste Stadt von ganz Tulgor. Den Ozean im Norden. Die breite Gebirgskette im Süden. Dieser Ozean und das Gebirge schienen auch mit der großen Wassermasse und den südlichen Bergen von Barz‘ Karte zusammenzupassen. Zufall, oder lag da mehr dahinter? Lag Barz’ Heimat etwa direkt im Osten oder im Westen von Tulgor? Oder vielleicht gleich weit in beiden Richtungen entfernt? Stammte er von einem Ort gerade gegenüberliegend auf der großen Weltenkugel?

Barz selbst schien die Antwort auf diese Fragen auch nicht zu kennen, denn er vertauschte mehrmals demonstrativ verwirrt die beiden Landkarten, um zu zeigen, dass er ihre relative Anordnung nicht kannte. Anschließend faltete er erstaunlich geschickt zwei kleine Figürchen aus Pergamentfetzen, einen Menschen und einen Vogel. Er setzte den Vogel auf die westlichere Steppe auf seiner eigenen Karte und tippte auf sich selbst, während er den Menschen auf der Karte seines Sees und seiner Steppe setzte. Dann verschob er seine Figur von seinem See auf die Nestbaum-Karte und machte: „Puff.“

Es dauerte noch einige Versuche, bis es ihm und Aćh gelang, einander ihre Ideen zu übermitteln.

Offenbar hatte Barz sich irgendwie hierher teleportiert. Möglicherweise hatte es mit den magischen Pulvern zu tun, die er in Säckchen bei sich trug und von denen er immer wieder welche vorzeigte. Es schien Aćh, als wäre sein Sprung hierher nicht das gewünschte Ergebnis gewesen. Seine wirkliche Absicht konnte sie zum jetzigen Zeitpunkt noch nicht erahnen. Ziemlich sicher war, dass Barz hinter der Verschwinden Turrs steckte. Wobei Barz zu glauben schien, dass Turr sich vor seinem Verschwinden bereits in der Nähe von Barz‘ Heimat befunden habe, in der westlicheren der beiden Steppen. „Andor“, sprach Barz immer, wenn er darauf zeigte.

Zudem schien Barz zu glauben, dass sich Turr nun in Barz‘ Heimat befand. Nun, selbst wenn dies stimmte, so würde dies nur so lange stimmen, bis Turr sich zurückzuteleportieren versuchte, und mangels leitender Flötenklänge womöglich irgendwo weit entfernt vom Nestbaum landete.

Sie hatten keine Ahnung, wo Turr sich aufhielt.

Aćh schüttelte resigniert ihren Kopf. Wie konnte man nur so fahrlässig sein?! Barz hatte Turr unbedacht von hier weggerissen und sich selbst an einen Ort versetzt, den er nicht kannte und an dem er nicht sein wollte. Und Aćh hatte Turrs Asche vor zwei Monaten verschüttet und konnte ihn somit nicht mehr zurückrufen. Eklig, eklig, eklig, diese Angelegenheit.

Als nächstes versuchte sie, Barz zu fragen, ob der Vorfall mit Turr vor zwei Monaten auch sein Werk gewesen war. Sie hantierte mit Turrs Papierfigur auf den Karten herum und spielte ihre Flöte vor – diese Klänge schienen Barz irgendwie bekannt vorzukommen, seine Augen zeigten jedenfalls Erkennen – doch scheiterte die Kommunikation schon daran, dass es Aćh nicht gelang, das Konzept von „Vergangenheit“ zu kommunizieren. Wie sollte das auch gehen, die herkömmlichen Monduhren Tulgors schienen Barz überhaupt nichts zu sagen. Wie man in seiner Heimat wohl die Zeit maß?

Resigniert gab Aćh irgendwann auf und brachte Barz zurück zu seinem provisorischen Bett.

Zeit, „Gute Nacht“ auf Tulgorisch zu lernen. Auch diese Worte schien er sich rasch merken zu können. Er antwortete mit einem Gruß in seiner eigenen Sprache, den Aćh sich wiederum einprägte.

Was für ein Tag!\bigskip







Die nächsten beiden Tage zogen überraschend rasch vorüber. Barz folgte Aćh nicht mehr auf Schritt und Tritt, zeigte sich aber immer noch äußerst interessiert an den Vorgängen rund um den Nestbaum der Takuri. Zum Glück für die beiden war er relativ gut darin, sich neue Begriffe zu merken, und schon bald wurden seine und Aćhs limitierte Konversationen weniger zu einem Spiel der Pantomime und des Zeichnungserratens, sondern ein Spiel der Assoziationen simpler Wortkombinationen.

Barz zeigte Aćh, wie man aus Papierfetzen kleine Papiermenschen, -temm und -vögel zu falten vermochte, und Aćh führte Barz in ein simples, doch in Tulgor sehr beliebtes Spielchen ein, in dem es darum ging, über Würfelergebnisse unter Bechern zu flunkern. Nebenbei war dies auch eine gute Gelegenheit für Barz, die tulgorischen Zahlenbezeichnungen zu repetieren. Auch wenn Aćh manchmal den Verdacht hatte, Barz würde die Würfel irgendwie mit magischen Kräften beeinflussen. Zu oft fühlte Aćh sich bei seinen Ergebnissen von den Gesetzen der Wahrscheinlichkeit hintergangen.

Yrbstschly, die Temm, hielt sich in den überraschend fern von den beiden. Insbesondere war ihr Barz‘ Echse Sabri nicht ganz geheuer, nachdem diese einmal nach ihren Beinen geschnappt hatte.

Umgekehrt war Barz hingegen überaus interessiert an Yrbstschly und den sonstigen Temm. Insbesondere schien er fasziniert am sandartigen goldenen Pulver, das bei vielen magischen Akten der Temm als Nebenprodukt freigesetzt wurde. Aćh ertappte ihn einmal dabei, wie er ein wenig Sand der Temm vom Boden zusammenkratzte und in eines seiner vielen Pulversäckchen füllte.

Ohnehin legte Barz einige ungewöhnliche Verhaltensweisen an den Tag. So demonstrierte er etwa auch großes Interesse am feuerfesten Nistmaterial der Takuri, welches von den Tulgori aus einer Pflanze hergestellt wurde, aus deren Fasern auch feuer- und reißfeste Kleidungsstücke und sogar Fischernetze hergestellt wurden. Eines Nachmittags fand Aćh ihn neben seinem Bett, wie er ein stark riechendes Pulver über eine Probe des Nistmaterials streute und daneben Notizen in fremden Zeichen auf einem Stück Papier machte. Aćh hatte nicht einmal mitgekriegt, dass Barz ein Stück vom Nistmaterial hatte mitgehen lassen.\bigskip







Ein anderes Mal fand Aćh Barz unansprechbar im Schneidersitz in seinem Bett sitzend vor, eine bräunliche Pulverpaste um seine Nase geschmiert. Mit geschlossenen Augen saß er da, reagierte nicht auf Aćhs Rufe und murmelte etwas Unverständliches vor sich hin. Etwas erschrocken und vermutlich unnötigerweise besorgt, rüttelte Aćh an seiner Schulter, bis Barz wieder seine Augen öffnete. Leichte Tränen glitzerten darin, als er sie erneut anstarrte.

Mit seiner Hand umfasste er wie unbewusst eine Halskette, an der zwei klobige Ringe hingen. Einer aus dunklem Holz und einer aus hellem Stein. Während Barz traurig vor sich hinblickte, zeigte Aćh fragend auf die Ringkette. Barz überlegte lange, lief dann zum Fenster, zeigte auf Aćhs Großmutter, welche in einiger Entfernung gerade einen jungen Hüter fürs Verschütten von Sufar schalt, und machte eine umarmende Geste.

Familie?

Familie.

Barz vermisste seine Familie? Baute er durch dieses bräunliche Pulver eine Verbindung zu ihnen auf, und Aćh hatte ihn dabei unterbrochen? Oder half es ihm dabei, sich an sie zu erinnern? Aćh war verwirrt. Konnte Barz sich etwa nicht einfach wieder dorthin zurückteleportieren, woher er gekommen war? Brauchte er Turr dafür? Brauchte er mehr magische Materialien, als er bei sich trug? Steckte er nun in Tulgor fest?

Die Fragen wurden nicht weniger.\bigskip







Aćh hatte einen Durchbruch, als es ihr gelang, Barz die Worte für „Vergangenheit“ zu übermitteln und ihn in ihr Zeitrechnungssystem einzuweihen. Dann endlich konnte sie auf die Ereignisse um Turr vor zwei Monaten zu sprechen kommen. Barz zeigte sich geständig, auch bereits in diese Ereignisse verwickelt zu sein, bastelte aber demonstrativ eine andere Papierfigur mit einem hohen spitzen Hut – oder einer besonders extravaganten Frisur? – mit der er auf den Landkarten etwas darstellte. Der Hinweis schien klar. Für das damalige Verschwinden war eine andere Person verantwortlich gewesen.\bigskip







Einmal wurde Aćh mitten in der Nacht geweckt, weil ein Takuri ihr die kokelnden Überreste einer Steppenmaus aufs Bett erbrach. Aćh verdrehte nur ihre Augen und schlurfte müde zur Putzstation. Barz hingegen machte große Augen und wälzte sich die restliche Nacht lang unruhig auf seinem behelfsmäßigen Bett herum, wobei er immer wieder mal aufstand und vorsichtig nach seiner kleinen Echse Sabri sah.

Als sie am nächsten Tag den Raum verließen und die noch friedlich ruhende Echse zurückließen, versuchte Barz gar, einige lose Holzstücke vor den Eingang zu packen. Vermutlich wollte er verhindern, dass ein Takuri dort eindrang.

Wann würde ihm wohl auffallen, dass es keine Absperrungen oder Zäune um den Nestbaum gab, obwohl die nicht allzu fern liegende rote Steppe Tulgors nur allzu leicht von einem übereifrigen Takuri entzündet werden könnte? Dass die Hüter sich nicht mal die Mühe machten, ihre Futtervorräte vor den Takuri zu verschließen? Die Takuri waren Meister der Teleportation, und die einzige bestätigte Möglichkeit, Takuri an einem Ort zu behalten, bestand darin, diesen Ort so spannend zu gestalten, dass sie von selbst dort bleiben wollten. Darum boten die Hüter den Takuri so viele Streicheleinheiten und Spielmöglichkeiten an. Und darum musste sich Barz keine großen Sorgen machen, dass ein Takuri zu Sabri eindringen würde. Eine schlafende Echse war für die Takuri wohl viel langweiliger, als einander über den Himmel zu jagen und sich gegenseitig Knabberseile abspenstig zu machen.

Das sollte nicht heißen, dass die Hüter gar keine Möglichkeiten hatten, die Takuri im Zaum zu halten. Einmal auf einige Hüter geprägt, hörten die meisten Takuri auf den Klang derer tulgorischer Steinflöten. Und natürlich waren ausgeklügelte Löschanlagen am Rande der Steppe eingerichtet worden, auf dass man etwaige durch jagende Takuri ausgelöste Steppenbrände durchs Öffnen einiger passenden Schleusen im Keim ersticken konnte.

Manchmal fragte sich Aćh wirklich, was die Takuri vor den Hütern gemacht hatten. Wenn sie schon damals so brandfreudig wie heute gewesen waren, hätte die Steppe wohl kaum überlebt. War vielleicht ein sehr mächtiger Temm so freundlich gewesen, ihren großen Nestbaum in Stein zu verwandeln, auf dass er von ihnen nicht stetig niedergebrannt wurde? War dies der Grund, dass der Nestbaum so hart war, dass Pickel und Bohrer an ihm zerbrachen, und dass die Takuri-Hüter ihre Kletterkonstruktionen außerhalb des Baumes hatten erbauen müssen? Fragen über Fragen...\bigskip






Zwischendurch suchten Aćh und Barz auch den besten Kartographen der Minenarbeiter auf, doch auch auf einer professionell hergestellten Karte Tulgors konnte Barz keine Verbindung zu seiner Heimat machen.

Während sie schon in der Mine waren, versuchte Aćh erneut ihr Glück, Turr mit der tulgorischen Steinflöte zu rufen. Leider keine Antwort.\bigskip







Dann, auf einmal, waren die drei Nächte des Wartens auch schon um. Aćhs Mutter Nelímar hatte einen Falken mit einer positiven Nachricht zurückgesandt. Wenn nicht irgendeine abrupte Fürstenkonferenz angesagt würde, konnte sie ihnen eine Audienz beim Hüter verschaffen. Sie konnten aufbrechen. Von den westlichen Ausläufern des Kuolema, wo sich der Nestbaum der Takuri befand, tief ins Land Tulgor hinein.

Auf, nach Agarb!

Auf, zum Hüter der Zeit!

Er würde ihnen sagen können, was sie wissen wollten. Sofern er sie für würdig befände. Nicht, dass er gemein oder hochnäsig wäre, im Gegenteil, er wurde immer als sehr fröhlich und freundlich beschrieben. Doch war seine Zeit nun mal sehr kostbar, und für jeden verschwundenen Schlüsselbund, dessen Position er seinem Besitzer verriet, verstarb möglicherweise ein erfolgreicher Heiler, der in seinem Leben noch zahlreiche Gebrechen hätte heilen können.

Aćh gruselte es etwas, darüber nachzudenken, über wie viel Macht der Hüter der Zeit verfügte. Macht korrumpierte, das sagte Òkôkó immer, und doch schien sie dem Hüter der Zeit nicht zum kleinen Kopfe zu steigen.

Barz, der in den letzten Tagen einmal die Funktionsweise der Takuri-Spiegel hatte aus der Ferne beobachten können, blickte fragend hinüber zur Spiegelschmiede, als Aćh ihn schnurstracks daran vorbeiführte. Dachte er vielleicht, dass sie per Spiegel reisen durften?

Aćh lächelte schief und winkte ab. Für die Tulgori waren Takuri-Spiegel von großem Wert. Selbst ihre Scherben bargen noch besondere Kräfte und ihr Besitz wurde, wie die Benutzung der Spiegel, strengstens reguliert durch die großen Spiegelmagnaten. Per Takuri-Spiegel zu reisen war viel zu teuer für eine einfache Takuri-Hüterin wie Aćh, die für ihre Arbeit am Nestbaum mehr in Kost und Logis denn in Sold entlohnt wurde.

Aćh und Barz würden stattdessen ein Hängeschiff nehmen, dessen Strecke Tulgor vom Osten bis in den Westen durchzog, von ihrem jetzigen Aufenthaltsort bis zur prächtigen Stadt Agarb. Nicht so schnell wie die Spiegel, doch immer noch schneller als die Reise per Kutsche oder Trampeltier. Nur wie würde sie das Barz alles erklären können?

„Geld? Gold?”, fragte Aćh Barz, und hielt ihm einige Goldstücke aus ihrer Tasche entgegen, die sie bei den Zwischenstopps für Verpflegung und Unterkunft ausgeben werden würden. Vielleicht auch für einige Souvenirs.

Barz schüttelte seinen Kopf. Kannte man überhaupt das Konzept von Goldstücken, dort, woher er kam? Oder gar Währungen im Allgemeinen?

Aćh steckte das Gold wieder ein und zeigte auf den gewundenen Weg, der vom Nestbaum der Takuri die bergigen Ausläufer hinunterführte und in ein kleines Dörfchen am Rande der roten Steppe führte.

Dort lebten und ruhten die meisten Minenarbeiter, die von dort aus jeweils für mehrere Tage in die Berge zogen und dort Stollen tief ins Innere des Kuolema-Gebirges gruben.

Die Spiegelschmiede neben dem Nestbaum der Takuri produzierte auch zahlreiche gewöhnliche Spiegel, und diese wurden oftmals eingesetzt, um das Licht des roten Mondes tiefer in die Stollen zu lenken, wenn die Felswände mal dafür nicht glatt genug geschliffen werden konnten.

Heute war es wieder einmal soweit. Der rote Mond stand am Tageshimmel und sein roter Schein wurde über mehrere Prismen vom restlichen Tageslicht getrennt und in die Stollen in die Tiefen des Berges geleitet, wo er Mera-Steine erstrahlen ließ.

Doch das brauchte Aćh und Barz nicht zu kümmern. Aćh streckte ihre Hand aus und lenkte Barz‘ Blick vom gewundenen Weg und vom kleinen Dörfchen am Fuße des Berges zum hohen Halteturm der Hängeschiffe, und von dort über eine regelmäßige Reihe hoher Masten, die vom Halteturm bis weit in die Steppe führten, so weit das Auge reichte. Zwischen den Masten waren stabile Transportseile gespannt, und in weiter Ferne war an einer Stelle ein Konstrukt mit großen weißen Segeln zu erkennen, welches über die Transportseile stetig näher glitt.

Ein Hängeschiff.

Tagein, tagaus, fuhren die Hängeschiffe die Transportseile entlang und transportierten aus den Minen geförderte Mera-Steine und in der Spiegelschmiede produzierte Spiegel vom Gebirge tief in Tulgor hinein, wo sie etwa von reichen Nomaden aus der Wilden Wüste gekauft wurden. Takuri-Spiegel bargen geheimnisvolle Kräfte und waren den Tulgori sehr wertvoll. Und selbst unreine, noch halb im Felsbrocken verborgene Mera-Steine waren als profane Gegenstände in den Sippen der wilden Wüste ungemein gefragt, ob als Stärkebeweis, Willensbeweis oder einfach nur als wertvolle Handelsgegenstände. Ganz zu schweigen von ihrer Beliebtheit als magisch potente Mittel. Jeder Hexer im Lande wollte wenigstens einige wenige dieser Steine bei sich haben, um mit ihnen magische Ströme zu leiten. Man munkelte, in den südlichen Bergen des Kuolema lebe ein alter Magier, der mithilfe einer horrenden Menge an Mera-Steinen jegliche Person ausfindig machen konnte. Dieser könnte Aćhs nächstes Ziel sein, falls der Hüter der Zeit ihnen mit Turr nicht aushelfen konnte.

Auf jeden Fall sorgte die enorme Beliebtheit der Mera-Steine und Spiegel im Lande dafür, dass viele Hängeschiffe die Transportseile entlangfuhren. Für ein geringes Entgelt lieferten sie auch Gäste von Dorf zu Dorf, von Stadt zu Stadt, von Burg zu Burg. Oder von den westlichen Ausläufern des Kuolema bis hin zur Metropole Agarb am anderen Ende Tulgors, wo der Hüter der Zeit residierte. Und dieser würde ihnen hoffentlich verraten können, wo sich Turr aufhielt.

Auweia. Aćhs Kopf schmerzte schon jetzt, wenn sie daran dachte, Barz auch nur einen Ansatz davon hauptsächlich mit Karten und Gesten erklären zu wollen.\bigskip







Die Zeit war gut getroffen. Noch während Aćh und Barz die Treppen zur Spitze des Halteturms erklommen – Barz wunderte sich eine Zeit lang über die kleinen Zwischenstufen, die jeweils ganz links der Treppe angebracht waren, bis er eine Temm beim Aufstieg überholte und wissend auflächelte –, sahen sie über sich ein Hängeschiff einfahren. Auf der obersten Plattform des Halteturms angekommen, reihten sie sich hinter den wenigen anderen Fahrgästen ein und spazierten über einen wackeligen Steg aufs Hängeschiff, während unter ihnen geschäftige Arbeiter Esswaren für die Minenarbeiter und Takuri-Hüter aus- sowie Mera-Steine und elegante Spiegel einluden.

Ein gut gebauter junger Mann mit einem prächtigen dunklen Haarschopf stellte sich wichtigtuend vor Aćh und Barz. Er überragte Aćh um einen ganzen Kopf und streckte ihr gebietend einen zweiteiligen Metallgegenstand hin, den er mehrmals klacken ließ. Einen Locher.

Während der Tulgori Aćhs und Barz’ Fahrkarten lochte, ließ der Mann seine helle Stimme über die anstehenden Fahrgäste erschallen: „Willkommen auf dem Hängeschiff ARCTOR, meine sehr verehrten Herrschaften, Damenschaften und allerlei anderen Mitglieder der Gesellschaft. Mein Name ist Ijs, und da der übliche Kapitän an diesem sonnigen Tag leider mit Magenbeschwerden an sein Bett gefesselt ist, werde ich auf der heutigen Reise nicht nur Ihr liebster Schaffner, sondern auch Ihr liebster stellvertretender Kapitän sein. Gleichzeitig werde ich mich auch als Aussichtsleiter versuchen. Ich kenne die Gegend hier wie meine Westentasche und vermag zu jedem der auf Ihrer Fahrt zu sehenden Bauwerken mindestens drei lustige Fakten abzugeben. Ganz zu schweigen von meinen Kenntnissen über die verschiedensten Brücken, die es aus von hier aus zu sehen geben wird. Freut euch!“

Aćh bedachte den enthusiastischen Schaffner und stellvertretenden Kapitän mit einem betont gelangweilten Blick. Barz hing ihm hingegen fasziniert an den sich schnell bewegenden Lippen, ohne dass er wohl auch nur das geringste Wort verstand. Aćh bugsierte Barz am Kapitän vorbei ins Innere. So jung, wie er aussah, konnte Aćh sich nicht einmal seiner Volljährigkeit sicher sein, geschweige denn seiner Kompetenz. Wobei, wenn die Schiffsgesellschaft einen so jungen Mann bereits mit dem Kommando über ein Schiff überließ, sprach das durchaus für ihn.

„Nicht von der gesprächigen Sorte, sind wir?”, fragte Ijs mit einem übertriebenen Schmollmund.

Aćh winkte mit einem höflichen Lächeln ab. Sie kannte vielleicht gar mehr lustige Fakten über die Bauwerke dieses Landes als er. Und Barz konnte sie beide ohnehin kaum verstehen.

Ijs las ihre Stimmung richtig und wandte sich den nächsten Fahrgästen zu, doch nicht ohne Aćh und ihrem Begleiter freundlich zuzuzwinkern.

Barz blickte immer wieder strahlend in die Ferne. Die Aussicht war wirklich wunderschön. Hin und wieder starrte Barz aber auch unsicher auf den doch ziemlich dünnen, schwankenden Holzboden, auf der er stand, und auf die filigran aussehende Konstruktion von Holz, Seilen und Metallbeschlägen, mit der das Hängeschiff an den gewaltigen Transportkabeln über ihnen hing. Aćh bemerkte den starken Griff, mit dem Barz sich an die Reling des Hängeschiffes klammerte. Beruhigend legte sie ihm eine Hand auf die Schulter. Sie hatte auch einige Reisen mit den Hängeschiffen gebraucht, ehe sie ihnen hatte vertrauen können. Doch dem ersten Eindruck zum Trotz war das System sicher entwickelt worden. Komplikationen auf Reisen waren selten und Abstürze noch seltener.

Dann stellte sich Ijs hinter die Kommandozentrale in der Mitte des Schiffs und klappte einige Hebel um. Gewaltige Segel entrollten sich links, rechts und unterhalb des Hängeschiffs. Ijs warf einen Kontrollblick darauf und hielt seinen Finger in die Luft. Aćh deutete sein Grinsen als gutes Zeichen. Der Wind stand günstig. Sie würden vorerst keinen zusätzlichen Antrieb nötig haben. Und keine einzige Wolke stand am Himmel. Nass würden sie auch nicht werden.

„Nehmt bitte alle Platz, die Anfahrt kann etwas schaukeln“, rief Ijs. Die wenigen Gäste, die noch nicht auf den ordentlichen Sitzreihen Platz genommen hatten, folgten seiner Anleitung. Barz folgte Aćhs Beispiel.

Ijs betätigte zwei weitere Hebel und drehte an einer Kurbel. Die Blockaden an den Rollen am Transportseil über ihnen lösten sich.

Und das Hängeschiff setzte sich ruckelnd in Bewegung.\bigskip



***\bigskip


Die Sonne ging auf über der Roten Pyramide. Der Hüter der Zeit rollte sich von seinem Schlafkissen, befestigte seinen mächtigen Turban auf seinem Kopf und blickte von der Spitze aus die vielen Treppenstufen der Pyramide hinab. An ihrem Fuße hatte sich bereits eine kleine Traube von Besuchern versammelt, doch der Hüter sah sie nicht wirklich. Während er gedankenverloren sein Frühstücksbrot verzehrte und zur Zahnpflege auf einem Kauast herumknabberte, starrten seine Augen ins Leere. Sein Geist forstete durch Jahre von Eindrücken, Bildern und Gesprächsfetzen, die er noch nicht erlebt hatte. Sein Herz raste vor Aufregung, wie schon seit Jahren nicht mehr. Nur noch wenige Tage. Dann würde er seine Worte ganz mit Bedacht wählen müssen. Die Zeitlinie stand auf der Kippe. Und nicht nur die von Tulgor. Die gesamte Welt war in Gefahr, in eine Eiswüste verwandelt zu werden.





\newpage
\section{Aćh und Barz reisen zur roten Pyramide}



Aćh mochte das Reisen sehr. Nichts gegen ihre Zeit am Nestbaum der Takuri, das Arbeiten mit den Feuervögeln war eine sehr lohnenswerte Erfahrung, doch wer konnte schon etwas gegen eine gelegentliche Tour durch die verschiedensten Orte des Landes haben? Wenn es diesmal doch nur unter glücklicheren Umständen hätte stattfinden können.

An Barz’ begeistertem Blick auf alles Temm- und Menschenmögliche, an dem sie vorbeifuhren, von dem elegantesten Kuppelbau eines hohen Lerninstituts bis hin zum simplen Design der öffentlichen Toiletten, sah Aćh wieder einmal, wie großartig die tulgorische Zivilisation doch sein konnte. Manchmal war das leicht zu vergessen in ihrem relativ abgeschiedenen Alltag am Nestbaum. Hier wurde sie auf eine schöne Weise daran erinnert.

Nun gut, eigentlich sah Barz auch alles andere mit Begeisterung an. Wilde Steppenblumen, deren blaue Blüten durch das rote Steppengras tief unter dem Hängeschiff stachen. Sechsbeinige Katzen, die auf der Suche nach Steppenmäusen durchs Gras huschten. Fledervögel, die in Schwärmen über den blauen Himmel zogen.

Schiffe konnten Menschen entlang von Flüssen transportieren, doch nur wenige Flüsse durchquerten die Steppe, und diese waren oft sehr kurvig.

Einst mussten die Urahnen der Tulgori Waren über Transportseile quer durch Tulgor geschickt haben. Diese Seile, gedreht aus derselben hochstabilen Faser, die auch für das feuerfeste Nestmaterial der Takuri verantwortlich war, waren auch heute noch zwischen hohen Masten über der roten Steppe gespannt. Und einst musste jemand mit einem besonders stabilen Schiff auf die Idee gekommen sein, selbiges entlang dieser Masten zu schicken. Die Hängeschiffe, große, bauchige Holz- und Metallkonstrukte mit weiten, in alle Richtungen abgespreizten Segeln, die den Transportseilen entlang über die Steppe huschten, wirkten auf den ersten Blick wie eine realitätsfremde, quirlige Idee, die einem uralten Märchenbuch entsprungen war. Doch die klugen Köpfe und Baumeister Tulgors hatten sie seit ihrem ersten Auftreten stetig verbessert. Heute waren sie ausgestattet mit modernsten Bautechniken und einer Prise magischen Unterstützung, und somit komfortabel, sicher, und vor allen Dingen eines: Schnell.

Bei günstigstem Wind dauerte die Reise von den westlichen Kuolema-Ausläufern zur Metropole Agarb mit den Hängeschiffen nur zwei Tage. Die Nacht würden Aćh und Barz in einer Taverne in Thelot verbringen, einer florierenden Stadt an der Handelsstrecke, wo sich die Transportseillinie der Hängeschiffe mit einem breiten Fluss kreuzten. Dieser Fluss floss aus dem südlichen Tulgor bis hierher und führte von hier aus bis weit in den Norden, wo die rote Steppe in grüne Felder überging. Dort verzweigte der Fluss sich und fiel an der felsigen Steilküste Tulgors in einem prächtigen Wasserfall in die Tiefe.

Aćh war schon einmal dort im Norden gewesen, als ihre Mutter Nelímar im Namen des hohen Rats der Fürsten mit einigen Schafshirten an der Küste hatte verhandeln müsse. Irgendein Landnutzungsdisput, den heute niemand mehr interessierte. Um die grünen Felder des Nordens, wo vielfältige Feldfrüchte und Obstbäume gediehen, lebten heuer nur noch Bauern, Fischer, Schmiede, Baumeister und Weber, relativ auf sich allein gestellt. Der hohe Rat der Fürsten hatte am Ende lieber andere Felder weiter westlich für sich beansprucht, und sein Einflussgebiet ohnehin eher in die Gebiete direkt außerhalb seiner Burgen und Städte zurückgezogen. Städte wie der Touristenmagnet Thelot, an dem Aćh und Barz die Nacht verbringen würden. Zahlreiche gewöhnliche Schiffe und gut ausgebaute Wege führten Passagiere von Thelot in die weiter Inlands liegenden Dörfchen, die alle ihren eigenen Charme hatten. Doch Aćh und Barz waren nicht als Touristen hierhergekommen.\bigskip







Kapitän Ijs verschloss den Zugang zum Hängeschiff hinter den letzten Passagieren mit einem überkompliziert aussehenden Schlüssel, winkte ihnen zum Abschied noch einmal zu und bewegte sich dann in Richtung einer Unterkunft für Hängeschiffmatrosen.

Barz hatte schon kurz nach ihrer Ankunft bereits einen seiner vielen Rucksäcke abgezogen, geöffnet und Aćh ein dünnes Stück Stoff gezeigt, welches wohl als Schlafsack fungieren sollte. Fragend deute er auf ein Stück Grünfläche neben einer hohen Brücke. Aćh lachte und gestikulierte ihm, er solle sein Schlafmaterial wieder wegpacken. Diese Nacht würden sie in einer Wirtschaft in Thelot verbringen. Das Konzept einer Wirtschaft schien Barz eher unbekannt zu sein.

Die Taverne zum Tauchenden Takuri war nicht das edelste Gasthaus, das Thelot zu bieten hatte, doch war sie sicher, gut besucht und das Essen ausgesprochen lecker. Der Name rührte von einer alten Legende her. Angeblich hatte es einst (und eigentlich immer noch) einen Takuri gegeben, der das Wasser geliebt hatte wie kein anderer, und immer wieder in den großen Ozean im Norden vorgedrungen war, trotz aller bösartiger Kreaturen, Seeriesen, Riesenkraken und sonstiger Ungetüme, die sich darin verbargen.

Eines Tages hatte Olrećh, ein gewiefter tulgorischer Safthändler, bei einer spätabendlichen Saftlieferung die falsche Richtung eingeschlagen und war mitsamt seiner gesamten Wagenladung in den Fluss gefallen. Da sei der Tauchende Takuri aufgetaucht und habe ihn aus dem Wasser gezogen. Olrećh sei überglücklich gewesen über sein Überleben und habe dieses Erlebnis als Schicksalssignal gewertet, dass er nicht für ein Leben als fahrender Safthändler geschaffen sei. So habe Olrećh am nächsten Tag mit der Hilfe einiger Fischer die Überreste seines versunkenen Wagens geborgen und daraus einen Stand gezimmert, wo er seinen gloriosen Saft verkaufte. Und der war so beliebt, dass der Stand bald zu einer kleinen Hütte wurde. Viele Reisende kamen vorbei und erzählten, was hier und dort im Lande geschah. Und so wurde bald aus den Überresten des alten Karrens die Taverne zum Tauchenden Takuri.\bigskip







Aćh und Barz konnten sich in einem großen Zimmer im oberen Stock der Taverne Platz einrichten und die Nacht dort verbringen. Es war ein Massenverschlag, doch die anderen Gäste waren angenehm still. Stiller als der Schankraum unter ihren Füßen, aus dem noch bis tief in die Nacht grölende Gesänge erklangen.

Auf einem der vielen Bette lag eine mysteriöse Person in einem grünen Gewand. Sie mampfte geistesabwesend an einer blauen Staude mit kleinen runden Beeren. Blaubachbeeren? Eine Sternkrautblüte war elegant in ihrem langen goldenen Haar platziert. Als sie Aćhs Blick bemerkte, schenkte sie ihr ein Lächeln und sprach schmatzend: „Ach, mich musst du nicht beachten. Ich war nie hier.“ Aćh beachtete sie nicht weiter. Als ihr Blick das nächste Mal auf dieses Bett fiel, war jenes menschenleer.

Barz kniete sich neben seinem Bett hin, öffnete einen Beutel mit einem braunen Pulver und rieb sich eine Prise davon um die Nase, ehe er sich seufzend aufs Bett sinken ließ. Seine Hand umklammerte wieder die zwei Ringe an seiner Halskette, die Barz selbst zum Schlafen nicht abzog.

Am nächsten Morgen erwachte Aćh zum sanften Klingeln einer Weckglocke. Barz lag nicht mehr in seinem Bett, sein Gepäck war bereits ordentlich neben seinem Bett zusammengepackt. Aćh fand ihren Begleiter draußen vor der Wirtschaft, wie er seine Echse Sabri dazu zu bewegen versuchte, ein Büschel Gras zu essen. Das war ja faszinierend, zuvor hatte Aćh nur gesehen, wie Barz sie mit totem Takuri-Futter zu füttern versucht hatte. War die Echse etwa ein Allesfresser?

Frohen Mutes, wenn auch mit der üblichen Stille zwischen ihnen, brachen Aćh, Barz und Sabri auf zum Halteturm eines Hängeschiffs nach Agarb.

An Bord des Schiffs erwartete sie bereits ein breit grinsender stellvertretender Kapitän Ijs.

„Noch seid ihr mich nicht los! Wie es sich herausstellt, suchen die in der Metropole einen exzellenten Schiffslenker und ich will mich bewerben. Nun führe ich dieses Hängeschiff bis nach Agarb weiter.“

„Na, da können wir uns aber glücklich schätzen“, grinste Aćh.\bigskip







Zwei Drittel der Reise von Thelot nach Agarb verliefen relativ ereignislos. Barz blickte immer wieder begeistert in die Umgebung hinaus und sorgenvoll zu Sabri, die in seiner Transporttasche schlief. Aćh wurde plötzlich siedend heiß bewusst, dass sie nie nachgeforscht hatte, ob Tiere auf den Hängeschiffen erlaubt waren, und blickte verstohlen zu Ijs herüber.

Nachdem der sich aber überhaupt nicht um Sabri kümmerte, beruhigte sich Aćh wieder, zückte ein Rätselpergament, welches sie sich an einem Touristenstand besorgt hatte, und machte sich daran, es zu lösen.

Immer wieder fiel ihr eine andere Passagierin auf. Eine junge Temm, die interessiert auf Aćhs Rätselpergament schielte. Als Aćh ihr anbot, sie könne auch einen Blick darauf werfen, wandte sich die Temm jedoch wortlos ab und verzog sich hinter einen übergroßen Koffer, der wohl zu ihr gehörte.

Hin und wieder zückte Aćh auch ihre Takuri-Flöte und spielte hoffnungsvoll Turrs Lockmelodie, leider wie üblich ohne Folgen.

Doch dann, am späteren Nachmittag, zogen plötzlich Wolken auf. Unnatürlich dunkle. Und unnatürlich schnelle. Barz und Aćh warfen einander besorgte Blicke zu. Ijs blickte auf einen sich wild drehenden Sturmhahn und schraubte mit gerunzelter Stirn an einem Hebel, der das Segel des Hängeschiffs etwas einklappen ließ. Die restlichen Passagiere rückten enger aneinander.

Dann ruckelte das Hängeschiff, stockte und stoppte gar vollständig. Gemurmel machte sich breit, während Ijs vergeblich an zwei anderen Hebeln herumrüttelte.

„Mir scheint, dass unser Schiffsmast irgendwie blockiert wurde. Verzeiht die Unannehmlichkeiten und macht Euch keine Sorgen, wir werden die Lage bald gelöst haben“, versprach er den Reisenden.

Aćh warf einen Blick über die Reling des Hängeschiffs. Zwischen dem schaukelnden Hängeschiff und dem Boden lagen mehrere Dutzend Meter freier Fall. Der rote Steppenboden darunter war viel weiter entfernt als üblicherweise. Es war eigentlich ein Vorteil der Hängeschiffe, dass sie auch über hügelige Gefilde reisen konnten, wo eine Kutsche einen großen Umweg schlagen müsste. Doch dass das Schiff nun ausgerechnet über einem tiefen Tal hatten sie stecken bleiben müssen, gefiel ihr ganz und gar nicht.

Der anziehende Sturm, der ihr durch die Haare rauschte, stärkte ihr Vertrauen in die Sicherheit ihres Gefährts auch nicht. Sie fluchte.

Kapitän Ijs lief zu einem komplexen Kasten und tippte in bestimmten Mustern auf einen kleinen Schalter. Aćh kannte die Muster. Der Code des Seefahrers Morseus. Damit konnten Nachrichten rasch entlang der Transportseile geschickt werden, wenn ein Hängeschiff in Problemen steckte. Und ebenso rasch würde eine Antwort aus Agarb zurück zu Ijs gelangen.

Ijs richtete sich auf und versuchte offensichtlich, Professionalität auszustrahlen. Laut proklamierte er: „Bitte bewahren Sie Ruhe. Verstärkung aus Agarb ist auf dem Weg hierher und wird uns hier abholen. Und selbstverständlich werden Sie allesamt für die Unannehmlichkeiten kompensiert werden.“

Die Sturmwolken wurden dunkler und dichter. Die ersten Regentropfen fielen aufs Holzdeck. Passagiere zogen ihre Kapuzen hervor und murmelten etwas davon, dass es doch noch vor so kurzem nach strahlend schönem Wetter ausgesehen hatte.

„Arb“, zupfte Barz an Aćhs Ärmel und zeigte beunruhigt auf den sich verdunkelnden Himmel, „Arboduk.“

Aćh blickte ihn verständnislos an. Barz zückte einen Pergamentfetzen und skizzierte eine Gewitterwolke. Dann zeichnete er ein wütendes Gesicht darin.

„Arboduk.“

„Ein bösartiges Gewitter?“

„Leben. Arb.“

„Ein lebendiges Gewitt... oh. Oh. Uh-oh. Ein Sturmgeist? Bist du dir sicher?“

Ein Blitz zuckte über den wolkenverhangenen Himmel. Wind wehte über das steckengebliebene Hängeschiffdeck und riss Barz den Papierfetzen aus der Hand.

Barz nickte bestürzt. Er zeigte auf verschiedene Stellen im Himmel, wo Aćh nur dieselben grauen Wolken sehen konnte. Nun, sie konnte ihm wohl vertrauen, dass er sich mit Sturmgeistern besser auskannte als sie. Naturgeister waren in Tulgor früher häufiger anzutreffen gewesen als heute, und so kannte Aćh sie vor allem als Akteure in alten tulgorischen Märchen, die sie auch heute noch gerne verschlang. Schwer zu verstehen waren die Naturgeister, und noch schwerer zu besänftigen, wenn sie etwas aufgescheucht hatte. Wenn das hier wirklich ein wütender Sturmgeist war, war ihre Lage alles andere als rosig.

Aćh quetschte sich nach vorne zu Kapitän Ijs durch und raunte ihm zu: „Was könntet Ihr tun, wenn ein wütender Sturmgeist kurz davor wäre, zu wüten?“

Ijs riss seine Augen auf: „Seid Ihr sicher?“

In diesem Moment zuckte eine weitere Serie von Blitzen über den Himmel. Diesmal konnte auch Aćh darin zwei wütende Augen erkennen. Der Donner folgte beinahe augenblicklich auf das helle Licht und brüllte etwas über das Hängeschiff, das mit etwas Fantasie nach „trOpfEn prAssEln AUf fEUchtE ErdE“ klang. Sprach der Geist? Was hatte dies zu bedeuten?

Ijs wich einen Augenblick verschreckt zurück. Aćh erinnerte sich daran, wie jung er doch noch sein musste, und zweifelte daran, ob sie ihn hätte behelligen sollen. Doch dann wurde sie eines Besseren belehrt, als Ijs sich räusperte, sich aufrecht hinstellte und gestikulierte: „In den Laderaum! Alle, rasch! Bringt Euch unter Deck in Sicherheit!“

Mit einem Zug an einem Kontrollhebel fielen die Segel des Hängeschiffs vollends in sich zusammen. Mit einem zweiten Zug eines Hebels öffnete Ijs am Heck eine versteckte Luke in den Laderaum voller Mera-Steine, wertvoller Spiegel und dergleichen teurer Ware.

„Es dürfte etwas eng werden, aber dafür seid ihr sicher vor der Witterung“, rief Ijs und scheuchte die meisten Passagiere zur Luke. An Aćh gewandt fuhr er fort: „Hüterin vom Nestbaum, wisst Ihr, wie man einen Takuri-Spiegel bedient? Wir haben einen an Bord. Damit könnten wir im Notfall vielleicht von hier verschwinden.“

Aćh schüttelte traurig ihren Kopf: „Die sind sehr wertvoll und nicht unser Fachgebiet. Ich habe ja noch nicht mal je einen berühren dürfen.“

Der Regen wurde ebenso stärker wie der Wind. Aćhs Stimme wurde übertönt von einem tiefen Grummeln des Sturmgeists: „blItz sprIngt zwIschEn dUnklEn wOlkEn!“

Seine Stimme klang zugleich nach Donnergrollen und einem protestierenden Menschen.

Barz bewegte sich nicht zum Heck des Schiffs, wo der Eingang zum trockenen Lagerraum wartete, sondern packte Sabri sicher in seine Bauchtasche und eilte zum Bug. Er warf einen Blick in die Tiefe unterhalb des Hängeschiffs und klammerte sich an die Reling. Dann griff er in eines seiner Pulversäckchen. Er warf eine Prise eines dunkelblauen Pulvers in die Luft, welches sich im magisch glitzernd im umherwirbelnden Wind verteilte. Das Heulen des Windes ließ etwas nach.

„Was tut er da?!“, fragte Ijs Aćh, die letzten Passagiere unter Deck winkend.

„Keine Ahnung, aber besser, als den Sturmgeist auszusitzen, dürfte es allemal sein“, meinte Aćh und rannte ebenfalls nach vorne zu Barz. Ihr goldenes Schwert zückte sie nicht. Wenn Barz‘ Bogen nichts dagegen ausrichten konnte, konnte ein Schwert das wohl erst recht nicht.

Ijs rannte ihr kopfschüttelnd hinterher. „Kommt nicht in Frage, dass der Kapitän sich unter Deck versteckt, während einige Passagiere einen Sturmgeist weiter reizen!“

„Wart Ihr nicht bloß ‚stellvertretender Kapitän‘?“

„Ist diese Unterscheidung jetzt wirklich von Relevanz?!“

„Sie ist, wenn Ihr wegen falschem Stolz etwas Unüberlegtes tut.“

Da blieb Ijs still.

„Was tust du da?“, fragte Aćh Barz, welcher immer noch Pulver in die Sturmluft warf, trotz des peitschenden Regens und der immer heftiger zuckenden Blitze.

„Gute Nacht. Gute Nacht Sturmgeist. Pulver nicht gut“, suchte jener nach Worten.

„Du willst ihn zum Schlafen bringen? Oder zumindest beruhigen?“

„In den Märchen war es auch immer eine bestimmte Tat oder ein Umstand, der die Sturmgeister aufregte“, meldete sich der stellvertretende Kapitän Ijs zu Wort, „Und man konnte sie dadurch besänftigen. Rasch! Was regt einen Sturmgeist auf?“

„Das ist die Seele eines Sturmwindes. Sie ist ungestüm und unberechenbar. Wer kann schon wissen, was genau in ihrem Geist vorgeht? Was sie aufregen könnte?!“, rief Aćh gegen den prasselnden Regen.

„Pulver? Magie? Böse?“, schrie Barz einige ihm bekannte tulgorische Worte in den Wind. Er konnte sich nicht genau ausdrücken, doch war das wohl auch nicht wichtig. Sie hatten keine Ahnung, was den Sturmgeist aufgeregt hatte. Und was auch immer Barz‘ Pulver tun sollten, sie taten es nicht. Oder nicht genug.

„Mag er vielleicht keine Takuri-Spiegel oder Mera-Steine?“, wandte sich Aćh an Ijs.

„Die vergangenen Fuhren wurden nie angegriffen!“, protestierte Ijs, „Wenn schon, dann sollten sich lieber Euer Freund seiner Pulver entledigen.“

„Zufall?“, warf Barz nun ein.

Aćh kauerte sich auf Deck.

„Ich sehe ihn nun auch!“, rief Ijs und zeigte in den Himmel. An einer Stelle hatten sich die Wolken zu einem besonders dunklen Fleck zusammengezogen, und die ununterbrochen umherzuckenden Blitze legten stets dieselben Bahnen zurück. Eine lange Blitzspur, die wie ein großer Mund wirkte, und darüber zwei kurze, die wie zwei Augen waren. Ein Gesicht im Sturm. Es schien wütend.

„Was willst du von uns?!“, schrie Aćh ins Gesicht des Sturmgeistes. Der Wind heulte nichts Verständliches, doch verstärkte er sich urplötzlich, zerrte an ihrem flatternden roten Umhang, riss sie von den Füssen und ließ sie einmal quer übers Deck schliddern, bis sie unschön gegen eine Sitzbank krachte.

Aćh blieb einen Augenblick benommen liegen und kauerte sich gegen das nasse Holz, während sie ihre Gedanken zu ordnen versuchte. Dann fiel ihr Blick auf einen großen Koffer an der Reling, der von den fliehenden Passagieren nicht mit unter Deck genommen worden war. Und auf die kleine Temm, welche sich dahinter verstecken versuchte.

„Hier rüber!“, rief sie Ijs und Barz zu und hangelte sich an der Reling entlang zur kleinen Temm.

„Hallo“, begrüßte Aćh die Temm vorsichtig.

Die Temm antwortete nicht und starrte entschieden an Aćh vorbei. Diese erkannte ein Namensschild, das mit einem Band an der Kleidung der Temm befestigt war und im Wind flatterte. „Trortra“ stand in großen Buchstaben darauf.

„Hallo, Trortra“, sprach Aćh. „Komm unter Deck, hier oben ist es gefährlich.“ Trortra die Temm antwortete immer noch nicht und drehte nun sogar ihren Kopf von Aćh weg. Doch Aćh hatte die Furcht in ihrem Gesicht bereits gesehen.

Hinter ihnen heulte der Sturmgeist noch heftiger aus. Der Koffer, hinter dem sich Trortra versteckte, wackelte gefährlich. Ja, das gesamte Hängeschiff wurde immer heftiger durchgeschüttelt.

„Verzeihung, Trortra, doch hier bist du nicht sicher“, sprach Aćh und packte die Temm mit beiden Händen, um sie in Richtung des sichereren Laderaums zu befördern. Da strampelte Trortra plötzlich auf, gab protestierende Laute von sich und streckte ihre kleinen Arme nach ihrem riesigen Koffer aus. Irrte sich Aćh, oder ertönten aus dem Koffer ebenfalls leise Laute?

„Bring sie unter Deck!“, rief Aćh Ijs zu und übergab ihm Trortra.

„Was macht sie denn noch hier oben?!“, antwortete der Kapitän. Er übernahm die protestierende Trortra von Aćh und hechtete mit ihr zur trockenen Luke.

Indes wandte sich Aćh ihrem mysteriösen Koffer zu. Durch einen verschwommenen Vorhang des vom Himmel runtergeschleuderten Wassers hindurch erkannte sie mehrere Verschlüsse an der Seite des Koffers, als bestünde er aus mehreren übereinander gestellten Teilen, die einzeln geöffnet werden konnten.

Aćh zog den Verschluss am obersten Kofferteil beiseite und enthüllte... einen Käfig. In dessen hinterster Ecke kauerte eine kleine, braunfellige Gestalt, die mit einem Schwanz nach Aćh stieß. Ein Schwanz, dessen harte Spitze dunkel schimmerte. Ein junger Skorpionfuchs aus der Wilden Wüste. Äußerst giftig, und sicher nicht ohne Deklaration transportierbar. Eine Skofu-Schmugglerin? Aćh blickte Trortra anschuldigend hinterher, doch die Temm war bereits von Ijs unter Deck befördert worden.

Konnte es sein, dass der Sturmgeist sich über den Skofu ärgerte? Hatten Sturmgeister etwas gegen Gifte? Oder gegen die typischerweise wasserlose Wüste, die einen Regensturm sicherlich schreckte?

Aćh verschloss das oberste Kofferfach wieder fest und zog das nächste auf. Sie erblickte einen giftigen Sumpffrosch in einem Glasgefäß. Das wurde ja immer heiterer. War die Temm eine Gifthändlerin? Arbeitete sie für eine?

Aćh zog den Vorhang vor einem weiteren Kofferteil zur Seite. Kaum hatte sie das sich darin befindende wieselähnliche Wesen mit blau blitzenden Schuppen gesehen, warf der blauschuppige Wühler auch schon eine Schar Stachelhaare von seinem Kopf in Richtung von Aćhs Hand. Rasch zog Aćh selbige zurück und verdeckte den Blick auf das Wesen wieder, doch hatte sie genug gesehen. Ein Ferun von den Feldern im Norden Tulgors. An der dortigen Steilküste waren sie äußerst ungern gesehen, doch hier schien offenbar jemand Interesse an ihm zu haben. Aćh ächzte, während sie einige harte Stachelhaare aus ihrer Hand klaubte. Nicht ohne Grund hatte man diese Spezies den Kopfstachlern zugeordnet.

Eins musste sie der Schmugglerin lassen, sie schien über ein äußerst breites Angebot zu verfügen.

Ijs winkte ihr von der Luke zum Laderaum her zu, und Aćh schlidderte ihm den Koffer übers rutschige Hängeschiffdeck entgegen. Er verfehlte einige Sitzbänke nur knapp.

„Pack den Koffer sicher unter Deck! Da sind Tiere drin! Aber pass auf, sie sind giftig!“

„Was?“, rief Ijs durch das Heulen des Sturmgeists zurück.

„Gifttiere!“

„Was?“

„Gift!“

„Saft?!“

„Gift!!!“

Ijs schüttelte seinen Kopf unverständig und reichte den Koffer einem anderen Fahrgast ins Hängeschiff hinein. Das war wohl in Ordnung, es würde schon niemand auf die Idee kommen, den Koffer unbedacht zu öffnen.

Nach dem Fund von Trortra wollte Aćh doppelt und dreifach sicher gehen, dass sie keinen anderen Fahrgast auf Deck übersehen hatten. So kämpfte sie sich den Wogen des Windes entgegen die Sitzreihen entlang und suchte nach weiteren versteckenden Fahrgästen. Sie fand niemanden. Alle waren sicher unter Deck. Alle außer ihr, Barz und Ijs.

Nicht, dass es den anderen besser ergehen würde, wenn das Hängeschiff sich vom Transportseil lösen und in die Tiefe stürzen sollte. Schon jetzt knarzten die Verbindungsmasten gefährlich stark. Und noch immer steckte das Hängeschiff über einem tiefen Steppental fest. Ihr Fall würde lang dauern und schmerzhaft enden. Wie lange konnten die Masten des Hängeschiffs halten?

Barz blickte ebenfalls besorgt hoch auf die knarzenden Masten. Er zückte ein grünes Pulversäckchen, rieb eine Prise daraus zwischen zwei Fingerspitzen, bis es knisterte, und schleuderte die Pulvermenge in die Höhe, an einen der beiden Schiffsmasten des Hängeschiffs.

Magisches Knattern ertönte und ein hellgrünes Glühen breitete sich von der Zielstelle aus. Dann gab es einen Ruck, und der Mast stand auf einmal stockstill in der Luft fest. Rund um die grünlich dampfende Stelle, an der das Pulver den Mast getroffen hatte, wirkte es, als fielen die Regentropfen langsamer. Ja, manche Tropfen ganz nah am glühenden Mast schienen gar völlig stillzustehen. Egal, wie stark der Sturmgeist am Rest des Schiffs oder an den Transportseilen ruckelte, der grün glühende Teil des Masts wich keinen Fingerbreit. Er stand buchstäblich in der Zeit eingefroren in der Luft fest. Der Sturm tobte, und doch hing ihr Schiff plötzlich wieder ruhig da.

Mit einem triumphierenden Grinsen auf dem Gesicht drehte sich Barz zu Aćh zurück. Sie wollte gerade ihre Daumen hochstrecken, da knarzte es unschön hinter Barz. Ein langer gezackter Spalt zog sich durch den unteren Teil des gebannten Masts, wo kein magisches Glühen herrschte. Er war angebrochen! Obwohl der Mast selbst noch still in der Luft stand, schaukelte das Hängeschiff darunter wieder im Sturm. Aćh und Barz fielen auf ihre Knie. Plötzlich war Kapitän Ijs wieder auf Deck, packte Aćh bei den Schultern und schrie ihr entgegen: „Dein Begleiter soll nicht den Mast bannen, so bricht er nur das Schiff auseinander! Aber wenn er stattdessen dieses magische Pulver auf ...“

„Ja, ja, sag es einfach ihm selbst!“, rief Aćh zurück und schubste den Kapitän weiter in Richtung Barz. Sie selbst huschte zur Luke zurück und warf einen Blick unter Deck.

Große Mengen der Ladefläche war mit Holzkisten bedeckt, welche sicher einmal schön aufeinander gestapelt gewesen waren. Nun herrschte das reine Chaos, doch war dies ebenfalls insignifikant. Regen tröpftelte durchs Deck hindurch und befeuchtigte die Passagiere, doch war dies auch nicht wichtig. Wichtig war, dass sie allesamt in relativer Sicherheit und unverletzt waren.

Auch Trortra, die Temm, befand sich im Laderaum, und versteckte sich mit einem schuldbewussten Blick hinter ihrem großen Koffer. Aćh stellte erleichtert fest, dass der Koffer immer noch verschlossen war. Aćh winkte ihr zu. Trortra schüttelte ihren Kopf.

Aćh blickte von der Luke zurück zum regnerischen Äußeren des Hängeschiffs. Regen und Hagel peitschte wie eh und je über das Deck und ließen das Holz bedrohlich knarzen. Ijs und Barz schrien einander einige Dutzend Schritte von der Luke an. Ijs deutete in eine bestimmte Richtung. Barz griff in sein grünes Pulversäcklein, zog eine weitere Prise Bannpulver hervor, setzte zum Wurf an und...

Eine weitere Blitzsalve jagte über den düsteren Himmel. Donner grollte. Der Sturmgeist riss sein gewaltiges Maul auf und brüllte: „wInd jAgt gEtIEr dUrchs stEppEngrAs!“

Ein unvergleichlicher Windstoß fuhr auf das Hängeschiff herunter. Er riss Sitzbänke beiseite. Ijs schlidderte gegen einen Mast und sank bewusstlos zu Boden. Sein langes im Sturm wehendes Haar war das einzige, das sich noch an ihm rührte.

Doch schien sich der Windstoß hauptsächlich auf Barz konzentriert zu haben. Der Nomade wurde über Bord geschleudert. Aćh schrie auf.

Der Sturmgeist riss Barz in die Höhe. Barz‘ Mund öffnete sich zu einem Schrei, den nur er selbst über den Sturm hinweg hören konnte. Sein Mantel bauschte sich auf, Taschen und Säckchen öffneten sich, verschiedenfarbig glitzernde Pulver verteilten sich durch die Luft und wirbelten in anmutigen Mustern durch die Regenschleier. Hier und dort knisterte und knatterte es, ein Glühen oder eine magische Flamme erwachte, nur um gleich wieder zu erlöschen.

Es sah wunderschön aus.

Dann hatte der Sturmgeist genug von Barz. Der Nomade wurde in hohem Bogen vom Hängeschiff weggeschleudert. Im freien Fall stürzte er der roten Steppe im Tal tief unter ihnen entgegen.

Aćh zögerte einen Augenblick. Dafür, dass Barz fürs Turr Verschwinden verantwortlich war, hatte sie ihn in den letzten Tagen überraschend ins Herz geschlossen. Doch hieß dies nicht, dass wegen ihm leichtfertig ihr Leben wegwerfen würde.

In diesem Augenblick wurde Barz wieder von einem Windstoß erfasst und in die Höhe geweht. Er jagte schreiend am Hängeschiff vorbei, und etwas löste sich von ihm. Eine kleine, graue, halslose Echse rutschte aus Barz‘ Bauchtasche, an seinen nach ihr greifenden Fingern vorbei, und rauschte selbst schreiend durch die Luft. Eine Echse, der Aćh schon einmal das Leben gerettet hatte.

Etwas in ihrem Kopf klickte, und ohne weiter zu überlegen zückte sie ihre Steinflöte und sprang über die Reling des Hängeschiffs.

Regen und Hagel schlugen ihr von allen Richtungen zugleich ins Gesicht, als sie im freien Fall auf Sabri zuhielt.

Sabris ledrigen Körper bekam sie sanft zu fassen.

Aćh drehte sich umher und suchte die Sturmschwaden unter ihr nach Barz ab. Sie sah seinen rasch fallenden Körper tief unter ihr dem Boden entgegenstürzen. Es wirkte nicht so, als könne sie ihn noch rechtzeitig erreichen. Und hatte er etwa seinen Körper zu einer Kugel zusammengerollt? Nein, damit würde er doch nur noch schneller fallen!

Da hörte Aćh über den Sturm hinweg ein bekanntes magisches Knattern. Ein grünliches Glühen breitete sich über Barz‘ zu einer Kugel zusammengerollten Körper aus. Sein Fall wurde verlangsamt, bis Barz mitten in der Luft erstarrt hing. Bewegungslos hing er da, mitten im Fall eingefroren in der Zeit.

Sein Bannpulver.

Barz hatte sich selbst gebannt und so seinen Fall aufgehalten.

Kaum war Aćh bewusst geworden, dass Barz‘ Rettung nicht mehr ihre höchste Priorität war und sie mit der kleinen Sabri aufs Hängeschiff zurückkehren konnte, fiel ihr auf, dass dies keineswegs so einfach sein würde wie gedacht. Regen und Wind zausten an ihren Haaren und ihrem langen Umhang, über sich hörte sie Passagiere schreien und den Sturmgeist heulen, unter ihr knatterte Barz‘ Bannpulver, und in ihrer Hand krächzte eine hilflose kleine Echse.

Aćh ließ alle Geräusche auf sich einfließen und zu einem dumpfen Hintergrundrauschen werden. Sie atmete tief durch, einmal, zweimal, bis sie ihren meditativen Mittelpunkt gefunden hatte. Dann handelte sie. Rasch, aber nicht unüberlegt.

Aćh griff an ihren Hals und löste das mittlere Element ihrer bronzenen Halskette auf. Im Behältnis glitzerte eine Prise Takuri-Asche. Asche von Saarćhan, der ältesten und riesigsten aller Feuertakuri, die schon seit Wochen am Nestbaum im Sterben lag. Nun musste Aćh hoffen, dass sie nicht in den zwei Tagen seit ihrer Abreise ihr Leben gelassen hatte und zu einem Küken geworden war. Und dass sie noch kräftig genug für einen letzten Teleport war.

Mit der einen Hand zog sie ihre Steinflöte hervor, mit der anderen kippte sie die kleine Dosis von Saarćhans Asche in die speziell dafür angefertigte Einbuchtung der Flöte. Jetzt musste es rasch gehen, ehe der Sturmgeist die Asche in alle Himmelsrichtungen verstreute. Oder ehe Aćh auf dem immer näher kommenden Boden einschlug. Aćh griff nach einer der beiden zeremoniellen Takuri-Federn, die ihren Umhang zierten, löste sie und brach sie mit einer Hand in zwei Stücke. Ihr Umhang, nicht mehr länger an ihrer Rüstung gehalten, flatterte davon. Die gebrochene Feder hingegen glühte leuchtend auf und sprühte Funken. Aćh presste sie in die Steinflöteneinbuchtung.

Die Funken genügten, um Saarćhans Asche in der Steinflöte zu entzünden. Rauch stieg auf und immer mehr Ascheflocken wurden mit leisem Plopp an den weit entfernten Ort gesogen, an dem sich Saarćhan aufhielt. Die Verbindung stand.

Verzweifelt blies Aćh in die Flöte und spielte eine Lockmelodie, betont besorgt und dringlich. Sie hängte auch einige dissonante Töne an, die ihre Not und Lebensgefahr ausdrückten.

Die letzten Flötentöne waren noch nicht einmal verklungen, da entzündete sich aus dem Nichts ein riesiger Flammenwirbel in der Luft unter Aćh. Ein dumpfes Tröten erklang. Mächtige Schwingen, jede beinahe so lang wie Aćh selbst, entrollten sich aus dem Wirbel. Die Flügel begannen, dem Wetter entgegenzuschlagen. Wind zupfte an ihren Federn und Regen prasselte auf ihr Feuer ein, doch die riesige Takuri ließ sich davon nicht beeindrucken. Saarćhan war gekommen!

Aćh stürzte der Takuri entgegen, die sich ihrerseits fallen ließ, um ihre Geschwindigkeit an Aćhs anzugleichen. Aćh prallte ungelenk auf Saarćhans Rücken, krallte sich an ihren Hals und zupfte Federn aus. Jedes Stück entblößter Haut brannte und schmerzte, denn Saarćhan stand buchstäblich in Flammen und Aćh hatte keinerlei schützendes Sufar aufgetragen. Aber es würde gehen müssen.

„Danke, danke, Saarćhan, alte Heldin“, rief Aćh durch zusammengebissene Zähne hindurch.

„Kurjo, Saarćhan!“, wies sie die Takuri an und deutete auf das Hängeschiff über ihnen. „Kurjo!“

Wie zur Bestätigung stieß Saarćhan ein melodisches Kreischen aus, dann schlug sie mit ihren mächtigen Schwingen und erhob sich in die Luft. Das golden schimmernde Vogelwesen brauchte nur wenige Flügelschläge, um die Höhe des Hängeschiffs zu erreichen. Langsam umschwirrte sie das Schiff, während der Sturm ihre brennenden Federn zerzauste.

Oben angekommen, rollte sich Aćh vom Rücken der uralten Takuri aufs nasse Holzdeck des Hängeschiffes. Ihre verbrannte Haut protestierte, doch Aćh zwang sich, aufzustehen. Noch war es nicht vorbei. Sie suchte den Himmel nach dem Sturmgeist ab, und sah sein blitzendes Gesicht an dem magisch gebannten Mast vorbeizucken.

Aćh legte die Hände wie einen Trichter an den Mund und rief dem Tier ein neues Kommando zu: „Saarćhan?! Advaria! Advaria meza!“

Dann zeigte sie auf das Gesicht des Sturmgeists.

Saarćhan hob kurz den Kopf, als wolle sie zeigen, dass sie verstanden hatte. Majestätisch schwang sie sich höher in die Luft und jagte direkt auf den Sturmgeist zu. Sie breitete ihre gewaltigen Schwingen weit aus und ließ sie dann mit aller Kraft vor ihrem Körper zusammenklatschen. Im selben Moment schien ein großer Ball feurigen Lichts vor der Takuri zu explodieren! Der gewaltige Feuerball hüllte den Sturmgeist ein und war so hell, dass Aćh geblendet wegsehen musste.

Der Sturmgeist schrie auf und donnerte in seiner tiefen Stimme: „schAttEn flÜchtEn vOr vErschlIngEndEm flAmmEnmEEr!“

Als die Flammen sich legten, plumpste etwas vom Himmel und knallte hart aufs Holzdeck. Etwas belämmert vom Aufprall hockte dort nun ein kleines, piepsendes Küken, das mit großen Kugelaugen in die Welt schaute.

„Oh, Saarćhan, bist du verletzt?“, rief Aćh.

Als die Takuri ihren Namen hörte, watschelte sie eilig zu Aćh. Diese blickte ängstlich nach oben. Falls der Sturmgeist von Saarćhans Feuerball nicht vertrieben worden war, hatte sie wahrlich keine Ahnung, was sie nun noch gegen ihn tun könnten.

Doch es hatte gereicht.

Die dunklen Wolken wichen wieder strahlend blauem Himmel, als hätte dieser Vorfall nie stattgefunden.

Der Sturmgeist war davongezogen, so schnell wie er erschienen war.

Aćh sah mit der Sonne im Rücken in die rote Steppe hinaus und erblickte einen wunderschönen doppelten Regenbogen.

Die Passagiere trauten sich langsam wieder aufs Deck hervor. Selbst Kapitän Ijs, der unschön gegen den Mast gestürzt war, hob benommen seinen Kopf und murmelte: „Was ist geschehen?“

Aćh rannte zu ihm und tastete seinen Kopf und Körper nach Verletzungen ab. Als er zudem meinte, dass ihm weder schlecht sei noch dass er verschwommen sähe oder sich ihm die Welt drehe, atmete Aćh erleichtert auf.

„Das war unglaublich!“, rief sie aus und schlug ihm begeistert auf die Schulter. Diese Aufregung! Die Spannung! So etwas habe ich schon seit langem nicht mehr gefühlt!“

„Ooookay“, murmelte Ijs und rubbelte seine Schulter, „Schön, dass du etwas Schönes draus ziehen konntest. Das Hängeschiff steckt immer noch fest. Und wo befindet sich dein pulveriger Kumpel? Wir werden hier ausharren müssen, bis Verstärkung aus Agarb eintrifft. Mann, was hat den Sturmgeist nur so in Rage versetzt?“

Ijs blickte unruhig über die Reling des Hängeschiffs hinaus, ob der Sturmgeist nicht plötzlich doch noch zurückkehren wollte.

„Wer kann das schon so genau sagen? Wir haben ihn vertrieben! Wir sind mächtig!“

Aćh jubelte glücklich. Sie balancierte ein Takuri-Küken und eine junge Steppenechse auf sich. Dann blickte sie über Bord und begann sich Sorgen darum zu machen, was geschehen würde, wenn Barz‘ Bannpulver zu wirken aufhörte.\bigskip







Tatsächlich hing Barz die gesamte Zeit bis zum nächsten Sonnenaufgang in der Schwebe zwischen dem Hängeschiff und dem Erdboden. Er befand sich nur einige Meter über dem Steppengras. sein Bannpulver hatte gerade noch rechtzeitig zu wirken begonnen.

In der Zwischenzeit erreichten andere Schiffe von Agarb aus die Stelle, wo die ARCTOR feststeckte. Die restlichen Passagiere wurden mit langen Leitern aus dem Hängeschiff in die Steppe gerettet und Ijs versprach ihnen allen einen Drink in Agarb auf seine Rechnung. Zwei äußerst offiziell und äußerst grimmig dreinblickende Speerträger warteten bereits auf den Boden und versprachen „weitgehende Untersuchungen“ dieses Unglücks. Die Ladung des Hängeschiffs wurde mit einem mobilen Kran in ein zweites eingetroffenes Hängeschiff umgeladen und weiterverschifft. Und ein großes Luftkissen wurde unter den eingefrorenen Barz geschoben, für wenn er wieder zu fallen begänne.

Die sehr offiziell wirkenden Speerträger stellten sich als stolze Ordnungshüter heraus, die umherliefen und Anwesende nach dem Vorfall ausfragten, während sie sich Notizen in langen Schriftrollen machten.

Schon bald erschienen über weitere Hängeschiffe von Agarb auch die ersten sonstigen Personen. Verwandte und Bekannte von Reisenden, Helfende und Schaulustige.

Die kleine Temm, Trortra, wurde abgeholt von einer in einen langen Mantel gehüllten Temm, die sie schluchzend umarmte und herzte.

„Ach du liebe Güte, was ist denn hier geschehen? Geht es dir gut? Tut dir etwas weh?“

Aćh fiel aber auch auf, wie rasch die abholende Temm einen größtmöglichen Abstand zwischen die Ordnungshüter und Trortra brachte. Und als Aćh sich einige Minuten später erneut nach ihnen umsah, waren die beiden plötzlich wie vom Erdboden verschluckt. Trortras Koffer voller seltener und giftiger Tiere stand hingegen immer noch da und wurde kurz darauf von den Ordnungshütern konfisziert.

Nachdem alle Zeugenberichte des Vorfalls aufgenommen worden waren und nachdem Aćh berichtete, dass sie leider nicht wusste, ob das Bannpulver den Schiffsmast und Barz von selbst wieder freigeben würde, war Aćh die Einzige, die noch bei Barz blieb. Die restlichen Passagiere zogen mit einem anderen Hängeschiff davon. Ijs klopfte ihr zum Abschied auf die Schulter und versprach, dass er ihr diese Heldentat nie vergessen würde.

„Mein Vater Saro meint immer, dass man keine gute Tat unbelohnt lassen soll. Ich verschaffe euch beiden Gratisfahrten auf den Hängeschiffen auf Lebenszeit“, meinte Ijs zwinkernd, während er mit den restlichen Passagieren davonzog. Aus der Ferne hörte Aćh noch, wie er erneut alle, die es wollten, auf eine Runde Bier in Agarb einlud.

Eine Ordnungshüterin blieb eine längere Zeit bei Aćh und unterhielt sich mit ihr über dies und das, bis auch sie sich entschuldigte und meinte, sie müsse nach Agarb zurück. Wer könne schon wissen, wie sich dieser Bann lösen ließe? Mit etwas Pech säßen sie noch eine ganze Woche hier, ohne dass etwas geschah. Aćh versprach ihr, dass sie sich spätestens beim Sonnenuntergang von der Szene lösen würde.

Zwischendurch tauchten einige Techniker der Hängeschiffgesellschaft auf, untersuchten das von Barz magisch gebannte Schiff, das immer noch regungslos an seinem Transportseil hing, kratzten sich an ihren Kinnen und reisten dann wieder zurück.

Und Aćh saß einfach da, betrachtete die Umgebung, streichelte das Saarćhan-Küken und Sabri und hoffte darauf, dass Barz‘ Bannpulver sich irgendwann in Luft auflösen würde. Sie konnte nicht einfach von hier weg. Sie könnte zwar den Hüter der Zeit auch ohne ihn aufsuchen, doch wer würde sich dann um Barz kümmern?\bigskip







Nelímar, Aćhs Mutter, reiste kurz vor Sonnenuntergang in einem Hängeschiff an. Sie brachte Verbandszeug, Decken und einen Riesenkorb voller Vorräte mit sich. Ihre sonst stets makellose Frisur saß etwas schief und ihr langes festliches Gewand in elegantem Schnitt war vorne falsch geschnürt. Hastig sprach sie: „Ich bin sofort aufgebrochen, als ich davon gehört hatte! Du Arme! Ihr alle Arme! Einen wütenden Sturmgeist, das gab es schon seit Jahren nicht mehr.“

Sie schüttelte ihren Kopf und machte sich sofort daran, die Brandspuren auf den Oberarmen ihrer Tochter zu verarzten.

„Du kommst mit mir, junge Dame!“, meinte Nelímar, „Ich lasse zwei meiner Sekretäre antreten und Wache halten. Sie werden berichten, wenn sich bei diesem Barz etwas tun sollte, und könnten ihn danach sofort zu uns führen. Falls er morgen immer noch hier feststeckt, lasse ich einen Druiden rufen, der sich mit solchen Zeitanomalien auskennt. Zunächst einmal bringe ich dich jedenfalls in die Diplomatengemächer, päpple dich auf und gucke, ob wir trotz der heiklen politischen Lage eine Audienz beim Hüter der Zeit bekommen können. Wie klingt das?“

Abgesehen von der „heiklen politischen Lage“ klang das großartig, musste Aćh zugeben. Sie protestierte noch etwas, sie wolle bei Barz bleiben, sah dann aber ein, dass sie hier kaum mehr erreichen konnte als zwei erheblich wachere Menschen.

So wurde Aćh per Hängeschiff nach Agarb gebracht und von Nelímar in ein Gästezimmer im Diplomatenquartier geführt. Sie konnte ihre durchnässte zeremonielle Rüstung in trockene, vorgewärmte Diplomatenkleider eintauschen. Und kaum war sie unter weiche, schwere Decken geschlüpft, fiel sie auch bereits in einen tiefen, traumlosen Schlaf.\bigskip







Aćh erwachte erst wieder zur Mittagszeit, als die Sonne schon hoch am Himmel stand. Ihre Gedanken liefen nur langsam wieder an. Das war doch nicht ihre Decke... das war doch nicht ihr Bett... das war auch nicht ihr Zimmer... der Sturmgeist! Agarb! Barz!

Erschreckt aus dem Bett strampelnd, stolperte Aćh beinahe über das Saarćhan-Küken und die kleine Steppenechse Sabri, die sich balgten. Oder besser gesagt: Saarćhan hopste Sabri nach und versuchte, auf ihren Rücken zu klettern, während die Echse sich wälzte und das Küken abzuschütteln versuchte.

So fasziniert vom Blick auf die beiden Kleinen war Aćh gewesen, dass sie beinahe übersehen hätte, dass Barz im nächstgelegenen Bett schnarchte. Sein durchnässter Mantel hing an einem eleganten Stuhl daneben. Offenbar hatte man auch ihm ein modisches langes Diplomatenkleid angeboten. Offenbar hatte sich sein Bannpulver doch noch gelöst. Oder war gelöst worden.

Am Frühstückstisch – nun, wohl eher zum Mittagsmahl – erzählte ihre Mutter ihr, was ihre Sekretäre miterlebt hatten.

Im Licht der ersten Sonnenstrahlen über dem Horizont hatte sich der grüne Schimmer des Bannpulvers leuchtend weiß verfärbt und vollständig verflüchtigt. Barz’ Fall hatte sich wieder beschleunigt und er war sanft ins Luftkissen geplumpst. Prompt habe er nach Sturmgeist, Schiff und Sabri gefragt, aber Kommunikation hatte sich als schwierig herausgestellt. Am Ende war er ihnen folgsam, doch nervös, nach Agarb gefolgt und sei überglücklich gewesen, als er erfahren hatte, dass Aćh seine Sabri gerettet hatte.

Sie hatten nicht einmal einen Druiden den Bann untersuchen und lösen lassen müssen.\bigskip






Bald trat der wache Barz auch schon selbst aus seinem Zimmer. In seinen Händen spielte er mit einem goldenen Armreif und zwei goldenen Ohrringen, die vermutlich zu seinem Gewand passen sollten. Stach man sich denn keine Ohrlöcher in seiner Heimat?

Interessiert betrachtete Barz die Augengläser auf Nelímars Nase, durch die sie irgendein diplomatisches Dokument musterte. Dann, noch etwas wackelig auf den Beinen, trat Barz an Nelímar vorbei und stellte sich vor Aćh.

Nur noch drei kleine Pulversäcklein hatte er mit etwas Seil an Knöpfen seines Diplomatengewands befestigt: Ein grünes, ein dunkelbraunes und ein weißes. Waren dies alle Pulver, die der Sturmgeist gestern nicht vernichtet hatte? Das grüne kannte sie bereits, das war dasjenige Pulver, das den Mast des Hängeschiffs und auch Barz‘ Fall selbst gebannt hatte. Die anderen beiden waren ihr noch unbekannt.

Barz ignorierte die leckeren Speisen auf dem Tisch, zückte das weiße Pulversäcklein und legte es sanft in Aćhs Hände.

„Danke“, sagte Barz, „Danke Sabri. Und Verzeihung. Verzeihung Takuri. Turr. Unglück nicht wollen. Turr finden. Helfen Turr finden. Verzeihung.“

Das waren vermutlich die meisten tulgorischen Worte, die Barz je auf einmal gesprochen hatte.

Ein Dank dafür, Sabri gerettet zu haben. Eine Entschuldigung dafür, Turrs Verschwinden ausgelöst zu haben. In Form... eines seiner wenigen übrigen Pulversäcklein? Aćh öffnete das Säcklein und nieste, als ein bisschen goldener Pulverstaub in ihre Nase aufstieg.

Barz schwenkte seine Hand zu seinem grünen Pulver und meinte: „Bannpulver Gefahr. Schwierig. Viele Menschen Pulver nicht gut. Nicht das da. Das gut. Gleichgewicht Magie.“

„Ich... öhm... danke. Danke, Barz, für dieses großzügige Geschenk. Ein magisches Pulver. Wie edel. Und mach dir keine Sorgen. Wir werden Turr finden.“

„Du böse ich. Verstehen. Gut. Turr finden.“

Was sollte das nun heißen?

„Was ist das für Pulver?“, fragte Aćh, auch um das Thema zu wechseln.

Barz legte seinen Kopf schief und schien nach Worten zu suchen. Nun, das war absolut verständlich, es hätte Aćh gewundert, wenn er es einfach so hätte ausdrücken können. Barz kratzte seinen Barz, zückte ein Stück Papier und skizzierte rasch ein Bild eines Wesens, das Aćh unbekannt war. Halb Fischschwanz, halb Strichmännlein. Ein Gestaltwandler? Vermutlich lebte es in Barz‘ Heimat.

„Nixe“, sprach Barz, und zeigte auf das Wesen.

„Nixe“, wiederholte Aćh.

„Nixe“, wiederholte Barz.

„Nixe“, wiederholte Aćh. Sie war sich nicht sicher, wie gut ihre Aussprache war, ebenso wie Barz sich vermutlich nicht bewusst war, wie gut die seine war. Sie mussten einander nur verstehen können. Der Rest würde sich mit der Zeit ergeben. Falls sie tatsächlich mehr als diese paar Tage miteinander zu tun haben würden.

Das war also die Nixe.

Wenn Aćh die Zeichnung richtig interpretierte, sonderte dieses Wesen offenbar Staub ab, den man einsammeln konnte und... nun, so richtig appetitlich wirkte das nun wahrlich nicht. Nixenstaub?

Barz gestikulierte eine Wurfbewegung und sprach: „Pulver. Stark Feind. Du schwach. Du stark. Muss man mögen.“

Er nickte mit einem breiten Grinsen. Aćh grinste ebenfalls, ein wenig aus Reflex. Sie mochte nicht ganz verstehen, was das Pulver, aber „wirf gegen starke Feinde zum Schwächen“ reichte ihr für den Moment. Wobei, wie stark konnte der Nixenstaub schon sein, wenn Barz es ihr einfach so anvertraute? Vielleicht war es viel mehr eine symbolische Geste.

Egal. Es war an der Zeit, beim Hüter der Zeit eine Audienz zu ersuchen.\bigskip




Die Metropole Agarb war ein riesiger Flickenteppich von Behausungen und Grünflächen, umzogen von einer hohen dunkelroten Mauer, die die Steppe draußen hielt.

Verschiedene Kulturen stießen hier aufeinander, von den Steppenwanderern über verschiedene soziale Schichten von Stadtbewohnern bis hin zu Karawanenführer aus der Wilden Wüste weiter westlich.

Inmitten des kunterbunten Sammelsuriums, das die Metropole Agarb war, stand eine hohe Steinpyramide, ziegelrot und uralt. Sie war eine der vielen architektonischen Besonderheiten der Metropole. Einer der Gründe, warum viele Interessenten der Architektur von nah und fern hierher strömten, um diese Besonderheiten zu studieren, die Baumeister der neueren Werke zu treffen und allgemein von ihnen zu lernen. Und diese Pyramide war der Sitz des Hüters der Zeit.

Sie hatte keine Spitze, sondern war oben flach. Vor einigen Jahrhunderten hatte der Rat der Fürsten von Tulgor beschlossen, dort eine Sitzsäule darauf einzurichten, eigens für den tulgorischen Hüter der Zeit. Dieses hohe Amt war extra erstellt worden, um einen gewissen, damals noch sehr jungen Temm offiziell in die bürokratische Anlage der Großstadt einzuordnen. Denn der Rat der Fürsten war nicht zufrieden gewesen damit, dass jemand einen derart großen Einfluss auf das Stadtgeschehen ausübte, ohne einen offiziellen Platz in ihrer politischen Hierarchie zu haben.

Heute waren nur noch sehr wenige der damaligen Fürsten am Leben, und auch den Hüter der Zeit zierten viel mehr Falten als damals. Doch noch immer ruhte er den Großteil seiner Zeit meditierend auf der Spitze der Säule der Roten Pyramide, und noch immer übten seine Ratschläge einen enormen Einfluss auf das politische Geschehen in der Metropole, ja, in ganz Tulgor aus.

Der Hüter der Zeit hatte seit Geburt eine besondere Gabe gehabt. Sein Geist erinnerte sich nicht an das, was bereits geschehen war, sondern an das, was erst noch kam. Und so ersuchte jeder, von Fürsten im Zoff bis hin zu Kindern, deren ihr Kätzchen davongesprungen war, eine Audienz beim Hüter. Und der Hüter widmete jedem, der bis zur Spitze der Roten Pyramide kletterte, einen angemessenen Moment, um ihnen auszuhelfen.

Freilich, schon seit Jahrzehnten konnte nicht einfach so jeder Dahergelaufene die Pyramide erklettern. Die Zeit des Hüters war schließlich kostbar, und so kümmerten sich Stadtwachen darum, sicherzustellen, dass Würdige als Erste die Pyramide erklimmen durften. Leider bedeutete das in Realität, dass die Reichen und Mächtigen für ihre teils kleinlichen Bedürfnisse zuerst zum Hüter vorgelassen wurden als eine Bettlerin, für deren Familie es um Leben und Tod gehen mochte.

Als Protestreaktion auf diese Ungerechtigkeiten gab es nun schon seit einigen Jahrzehnten den Brauch, zum Fuße der Roten Pyramide zu pilgern, aber sich nicht bei den Stadtwachen für Zulass zu bewerben, sondern schlicht einige Tage beim Fuße der Pyramide zu ruhen, in der Hoffnung, der Hüter der Zeit möge einen von selbst zur Spitze rufen. Denn wessen Urteil als das des Hüters selbst könnte schon besser abschätzen, wer seiner Zeit am würdigsten wäre?

So kam es, dass der Hüter immer häufiger jemanden vom Fuße der Pyramide nach oben rief, um seinen Rat zu empfangen. Doch wenn man sich mit den Stadtwachen gut verstand, waren diese immer noch eine sicherere Möglichkeit, vor den Hüter treten zu dürfen.

Grundsätzlich hatten es Diplomaten wie Nelímar auch wirklich gut mit den Stadtwachen. Leider war aktuell gerade einiges an politischem Geschehen im Gange. Der Rat der Fürsten hatte vorgestern gleich zwei Mitglieder an ein heißes Fieber verloren, und nun buhlte halb Tulgor um ihre Positionen.

Soeben war Fürstin Regilda, ein hochrangiges Ratsmitglied, mit einer ganzen Kolonne an Bediensteten aus ihrem Schloss in der roten Steppe zur Roten Pyramide angereist. In Anbetracht ihrer vergangenen Besuche war abzuschätzen, dass ihr Aufenthalt den Hüter der Zeit für mindestens zwei volle Tage in Anspruch nehmen würde.

Aćh hätte es natürlich nicht viel ausgemacht, zwei Tage zu warten, doch bis dahin würden höchstwahrscheinlich schon die nächsten Ratsmitglieder für eine Unterredung eingetroffen sein. Und wenn Ratsmitglieder den Hüter für sich beanspruchten, rief er so gut wie nie Bittsteller vom Fuße der Pyramide zu sich hoch. Die Lage sah nicht gut aus für ihr Anliegen, Turrs Aufenthaltsort zu erfahren

Nichtsdestotrotz führte Nelímar Aćh und Barz an diesem Nachmittag zur Roten Pyramide. Nachdem sie sich einige Souvenirs gekauft hatten, darunter eine kleine Statue des buckligen Hüters und ein kleines Holzmodell der Roten Pyramide, das man wie ein Puzzle auseinandernehmen und – unter größerer Denkleistung – wieder zusammensetzen konnte, machten sie sich auf den Weg zum Vorplatz am Fuße des Gebildes.

Wie zu erwarten hatte Fürstin Regildas Ankunft die meisten Hilfesucher abgeschreckt. Nur wenige Tulgori hatten sich auf dem großen Vorplatz verteilt, aßen vor Marktständen und spielten Kartenspiele. Keiner sah hingebungsvoll zur Spitze der Pyramide hoch, wie es sonst oft zu sehen war. Hin und wieder blickte jemand nervös zu den golden eingekleideten Bediensteten der Fürstin, die vor der Treppe Wache standen.

Und da war sie, Fürstin Regilda höchstpersönlich. Die Goldene Zentaurin. Soeben kämpfte sie ihren schweren Pferdeunterkörper unter größter Mühe die Treppe zur Pyramidenspitze hoch. Immerhin musste sie nicht getragen werden. Warum der Hüter der Zeit wohl all jene, die ihn besuchten, die Pyramide erklimmen ließ, selbst wenn es für sie eine Tortur darstellte? Manch ein armer Tropf hatte in der Vergangenheit gar hochgetragen werden müssen.

Die Sonne glänzte auf Fürstin Regildas Körper und Kleidung. Beide sahen aus, als wären sie in ein Bad aus Goldfarbe getaucht worden. Und wie üblich wirkte es, als sähe Regilda Aćh direkt in die Augen, ihr aus dieser Entfernung nur verschwommen erkennbarer Kopf in einem unnatürlichen Winkel verdreht. Die Fürstin hatte eine unheimliche Präsenz, so viel stand fest.

Barz blickte verschreckt drein, während er die Fürstin betrachtete.

„Augen. Sehen. Ich. Ich.”, sprach er flüsternd. Auch er musste die Fürstin so sehen, als würde sie ihm stets direkt ins Gesicht blicken.

„Illusion”, schüttelte Aćh ihren Kopf beruhigend, „Es sieht für alle nur so aus, als würde sie sie direkt ansehen. Weiß der Himmel, wie oder warum sie das tut, aber es ist nur eine Illusion.”

In diesem Augenblick ertönten hastige Schritte von der Spitze der Pyramide. Eine ältere Frau – vermutlich hatte sie dem Hüter vorhin seinen Nachmittagssnack gebracht – eilte die Treppenstufen hinunter, hielt vor der goldenen Fürstin an und verneigte sich tief. Von unten an der Pyramide war es unmöglich zu vernehmen, worüber die beiden sich unterhielten. Hier und dort sah Aćh Wartende von Spiel und Speise aufsehen, um dem eigenartigen Geschehen einen verstohlenen Blick zuzuwerfen.

Dann öffnete die Fürstin ihren Kopf. Ihre tiefe Stimme hallte laut über die Ebene, als sie sprach:

„Sind hier eine Aćh und ein Barz anwesend?! Wer auch immer ihr sein mögt, tretet vor und sprecht mit dem Hüter der Zeit! Und macht gefälligst schnell!“

Aćhs Herz machte einen Purzelbaum und rutschte in ihre Magengegend zugleich. Fürstin Regilda hatte sie angesprochen! Fürstin Regilda klang verärgert! Fürstin Regilda wurde vom Hüter der Zeit vertagt, damit er mit ihr und Barz sprechen konnte!

Sie packte den verwirrt dreinblickenden Barz am Ärmel und eilte nach vorne, auf dass sie so rasch wie möglich die Pyramide erklimmen könnten. Sie wagte es nicht, einen Blick auf die Fürstin zu werfen, als die beiden an ihr vorbeikletterten.

Der Hüter der Zeit hatte sie zu sich gerufen.\bigskip

***\bigskip

Ein kleines Vögelchen lag zitternd auf einer meterdicken Eisschicht und fror. Hier herrschte überall Schatten. In diese Schluchten fiel nie direktes Sonnenlicht, unter anderem wegen der dicken Wolkendecke, die stets den Blick auf den Himmel verdeckte. Die hellste Lichtquelle waren die leuchtend goldenen Stichflammen, welche hin und wieder aus dem schwachen Körper des Vogels hervorbrachen und einen Teil des Eises schmolzen, auf dem er lag. Doch dadurch sank der unterkühlte Takuri nur noch ein bisschen tiefer in seine Eiswasserlache ein.

Leise Schritte waren zu vernehmen. Sie verstummten, als sich eine Gestalt über die Lache beugte. Eine klirrende, weibliche Stimme ohne jegliche Gefühlsregung bohrte sich in den Kopf des Feuervogels:

„Na, was haben wir denn hier?“









\newpage
\section{Der Hüter der Zeit weist zum Ewigen Eis}

Aćh hatte den Hüter der Zeit noch nie persönlich gesehen, bloß von ihm gehört. Auf den ersten Blick sah er wie ein ganz gewöhnlicher Temm aus, ein kleines buckliges Männchen halt. Er trug einen hohen Turban, der seinen ohnehin schon überproportionierten Kopf noch viel größer erschienen lässt. Darunter lugten zwei verschmitzte Augen hervor. Der Blick aus diesen Augen wirkte auf Aćh, als würde der Hüter direkt in ihre Seele blicken.

„Wir sind uns noch nicht begegnet und doch kenne ich euch bereits“, sagte der Hüter mit krächzender Stimme. „Man nennt mich den Hüter der Zeit und dieser Aufgabe gehe ich nach. Und ich spüre, dass ich durch euch beide die Zeitlinie ganz besonders hüten kann.“

Aćh machte große Augen. Sie und Barz, sie beide waren besonders relevant für den Verlauf der Zeitlinie?!

Der Hüter der Zeit, vielleicht im Glauben, Aćh wolle seiner Aussage widersprechen, fuhr fort: „Da wirst du mir einfach glauben müssen. Ich nehme die Zeit anders war als ihr euch vorstellen könnt.“

Das hatte Aćh bereits gewusst und gedacht. Nichts läge ihr ferner, als dem Hüter nicht zu glauben. Dass er das nicht erkannte, bedeutete, dass er nicht allwissend war, ja, nicht einmal unfehlbar. Aber das hatte sie eigentlich schon gewusst. Der Hüter der Zeit erinnerte sich an die Zukunft, und sorgte dafür, dass sich die Zeitlinie möglichst wünschenswert entwickelte. Nicht mehr, nicht weniger.

„Verzeiht, dass wir Eure Zeit in Anspruch nehmen, und danke, dass Ihr...“, setzte Aćh an, doch der Hüter unterbrach sie.

„Verzeih mir, aber diese Unterredung wird kürzer dauern als diejenige mit der edlen Fürstin, und ist von ähnlicher Relevanz. Lassen wir das höfische Geplapper sein und fallen wir mit der Tür ins Haus. Was ist Euer Belangen?“

Aćh legte ihren Kopf schief: „Wisst Ihr das nicht bereits?“

Der Hüter gluckste: „Nun, selbst wenn mein Geist sich meiner Antwort bereits bewusst wäre, müsste die Frage nicht immer noch gestellt werden, damit ich mich an sie erinnern konnte, nicht?“

Was sollte denn das nun heißen? Nun, der Appell zur Eile fiel nicht auf taube Ohren. So sprach Aćh:

„Nun, wir sind auf der Suche nach meinen verschwundenen Feuertakuri, Turr. Barz hier... komm her, Barz! ... Barz hier hat sich irgendwie nach Tulgor teleportiert, von einem fernen Land namens... Andor? Seit seiner Ankunft hier ist Turr verschwunden, und wir wollten um Rat fragen kommen. Wo hält er sich auf?“

Ganz entgegen seiner ursprünglichen Bitte, ihre Unterredung kurz zu halten, lehnte sich der Hüter der Zeit zurück, schloss seine Augen und bewegte seine Lippen leise.

„Andor... Andor... ja, das sagt mir etwas. Andor, das Königreich hinter dem Kuolema. Ja, ich war sogar schon mal da.“

Aćh stockte: „Aber der Kuolema ist unüberquerbar. Die ganzen Geister und Dämonen. Keiner...“

„Ich sagte auch nichts davon, dass ich über die Berge gereist wäre. Die Dämonen der Berge wurden vor Urzeiten von den Temm ins Gestein zurückgetrieben und in die Täler des ewigen Eises verbannt. Wir wissen es besser, als dass wir es wagen würden, dort einzudringen. Doch es gibt andere Pfade zwischen Tulgor und Andor. Uralte, brüchige Stollen, die tief unter dem Kuolema hindurchführen. Die Wege wandeln sich stetig, und immer mehr schließen sich, doch hin und wieder öffnet sich eine Gelegenheit zur Durchreise. Früher schloss ich mich einst den reisenden Temm an. Wir unterquerten das Kuolema-Gebirge und gelangten in dieses fremde Land namens Andor. Ich half, wo ich konnte, und ich holte einen kleinen Feuertakuri ab, der sich im Land verirrt hatte. Wenn ich mich nicht irre, könnte dies gar derjenige Takuri sein, wegen dessen Verschollenheit ihr mich hier und heute aufsuchen kamt. Auf jeden Fall kehrte ich mit dem Takuri wieder hierhin zurück, und nahm meine Rolle als Hüter der Zeit auf der roten Pyramide wieder auf. Eine kurze Zeit danach erschütterte ein Beben den Kuolema und verschloss den Gang, den ich begangen hatte. Einer meiner Begleiter, Wrort, steckt sich heute immer noch in Andor fest. Doch bleiben solche unterirdischen Gänge nie bis in alle Ewigkeit verschlossen.“

Der Hüter der Zeit zwinkerte Aćh zu, welcher aktuell keine Antwort in den Sinn kommen wollte. Zu viele Informationen auf einmal hatte der Hüter ihr zu vermitteln versucht.

So wandte sich der Hüter der Zeit Barz zu, der wie üblich mit leerem Blick nebenan gestanden und an seiner neuen tulgorischen Kleidung herumgefingert hatte.

„Ronda tatt Hieron“, sprach der Hüter der Zeit. Während Aćh die Stirn runzelte und die fremden Worte zu verstehen versuchte, glänzten Barz‘ Augen auf.\bigskip







„Du sprichst die Sprache der Bewahrer?!“, rief Barz freudig auf. Er war kein Meister der andorischen Sprache, doch waren die Bewahrer die engsten Handelspartner der Barbaren und so wurde ihre Sprache jedem reisenden Nomaden zumindest im Ansatz beigebracht.

Während kaum ein Bewahrer sich darum kümmerte, die Sprache der Barbarenstämme zu erlernen, die für sie nur die „wilden Völker des Ostens“ waren. Barz wusste, dass das Wort der Bewahrer für die Barbarenvölker eine negative Konnotation hatte. Und dass „barbarisch” dort beinahe eine Beleidigung war und für „roh“, „grausam“ und „unkultiviert“ stand. Barz hatte schon seit jeher seinen Kopf schütteln müssen über diese Tatsache. Er war ein stolzer Barbar. Ihre Kultur mochte anders sein als die der Bewahrer. Auf Thakkum wurde nur selten mit Gold gehandelt, öfters in Gütern und Dienstleistungen, was Handel mit den Bewahrern schwierig machten. Und die Barbaren mochten nicht so tief in die Vergangenheit zurückblicken können wie die Bewahrer. Doch sagte dies doch nichts aus über ihre Kultiviertheit, ja, ihren Wert als Personen.

Doch all dies brauchte ihn nun wahrlich nichts zu kümmern. Er konnte sich mit diesem mysteriösen Hüter der Zeit unterhalten. Endlich! Er räusperte sich eilig. Seine Sprache war ganz eingerostet in den letzten Tagen.

Barz hatte schon vor einigen Tagen am Stand der Sonne ablesen können, wo Osten und Westen, Norden und Süden lagen. Nun konnte dieser bucklige Wichtel ihm vielleicht auch verraten, in welcher Richtung Andor lag. Wo Nabib sich aufhielt. Und wie Barz in seine Heimat zurückkehren könnte.

Der Wichtel grinste und sprach schnell: „Ich spreche die Sprache der Bewahrer vermutlich besser als du. Man nennt mich den Hüter der Zeit, weil ich einen besonderen Blick auf die Zeitlinie habe. Ich erinnere mich nicht an das, was bereits geschehen ist, sondern an das, was erst noch kommt. Ich weiß, weil man es mir sagen wir, dass ich mich schon mal in Gesellschaft der Bewahrer befand, einige Jahrzehnte ist es her. Und ich sehe, dass ich mich erneut in der Gesellschaft der Bewahrer befinden werde.“

Barz schluckte seine Freude darüber hinunter, sich endlich wieder mit jemandem austauschen zu können. Aćh hatte ihm trotz der limitierten Kommunikationsmöglichkeiten eintrichtern können, was für eine wichtige Person dieser Hüter war. Und wenn er sich tatsächlich an die Zukunft zu erinnern vermochte, dann war das auch berechtigt. Ohne seine Gedanken lange zu sortieren, schwafelte Barz dutzende Fragen herunter:

„Oh, die Götter seien gepriesen! Wo sind wir hier? Wo liegt die große Steppe meiner Heimat von hier aus? Wo liegt Andor? Wo liegen die Grenzen Eures Wissens? Wie geht es Nabib? Werde ich meine Familie wiedersehen?“

„Immer mit der Ruhe!“, lachte der Hüter der Zeit, „Dann gucken wir uns doch mal diese Fragen eine nach der anderen an. Wo sind wir hier? Wir befinden uns hier in einem Land, das von seinen Einwohnern Tulgor genannt wird. Und es wird auch von den Andori so genannt werden, wenn sie erst einmal von seiner Existenz erfahren. Andor liegt im Osten von hier, gleich hinter dem riesigen Gebirge, deren Gipfel stets in den Wolken liegen. Und deine große Steppe liegt noch weitaus weiter östlich.“

Der Hüter der Zeit schloss seine Augen und runzelte seine Stirn, als versuche er angestrengt, sich an etwas zu erinnern.

„Ich glaube, die Andori nannten dieses Gebirge ‚Fahles Gebirge‘. Sagt dir das etwas?“

Das sagte Barz tatsächlich etwas. In den Briefen aus Andor hatten die invadierenden Barbaren etwas von den unglaublich hohen Bergen im Westen berichtet, dem Fahlen Gebirge eben. Und Barz erinnerte sich auch an die Geschichte, die dieser Feuerzauberer aus dem fernen Hadria ihm damals in Thakkum erzählt hatte:

„Dieser Phoenix war angeblich vor Generationen von einer Einsiedlerin gefunden worden, welche versucht hatte, ein unüberwindbares hohes Gebirge im Westen zu überwinden, dessen Gipfel stets in Wolken gehüllt waren. Manche munkelten gar, das Gebirge sei unendlich hoch, obwohl diese These anhand der endlichen Länge des Schattens des Gebirges natürlich unzweifelhaft widerlegt werden kann.“

Damit war es klar. Das Kuolema-Gebirge war das hohe Gebirge im Westen Andors. Als Barz versucht hatte, sich nach Andor zu teleportieren, hatte er sein Ziel um einiges überschossen. Und das vermutlich, weil dieser Takuri, Turr, gar nicht aus Andor gekommen war, sondern eben vom Nestbaum! Nun ergab alles Sinn.

Doch, ehe Barz sich vollständig darüber freuen konnte, dass er nun wusste, wo seine Heimat lag, fuhr der Hüter schon mit gerunzelter Stirn fort:

„Und wie geht es Nabib? Wer ist denn... Nabib... Nabib... lass mich überlegen... ja, ja, ihr beide werdet euch wiedersehen, nachdem du Andor erreicht hast. Es geht ihm gut. Er war krank? Eine Wunde. Ich glaube, da wird eine entzündete Wunde sein. Oder da war eine. Doch sie heilt gut, er verbrachte nur einige Zeit lang ohne Bewusstsein.“

„Ich werde ihn wiedersehen? Wann? Wie?“

„Oh, mit Zeitangaben war ich noch nie gut“, brummelte der Hüter der Zeit und kratzte sich an seiner breiten Nase.

„Viele Monate werden vergehen... Ein... zwei... ja, mehr als zwei Jahre.“

Der Hüter der Zeit streckte seine Hand mit zweieinhalb von vier Fingern ausgetreckt vor.

„Zwischen zwei und drei Jahren, dann wird etwas geschehen, was dir erlaubt, nach Andor zu reisen. Dort wird dich Nabib wiederfinden.“

Die Erleichterung ob Nabibs Überleben wich einem dumpfen Schrecken.

Zwei bis drei Jahre, bis er Nabib wiedersehen konnte?! Zwei bis drei Jahre, die er noch hier verbringen müsste?! Nicht, dass er es in Tulgor nicht mochte, aber das war einfach zu viel Zeit weit weg von seinen Geliebten und Bekannten. Barz langte sich an den Kopf und sank zu Boden.

Wie hatte er nur so unüberlegt sein können?! Mit einem Experimentierpulver umherzureisen, statt einfach mit den restlichen Barbaren nach Andor auszuziehen! Mit Mächten zu spielen, von denen er weniger verstand. Mit einer Pulvermischung, die er nun nicht mehr reproduzieren konnte, und die ihm vielleicht nicht mal nützte, selbst wenn er es könnte. Denn es gab keine Takuri mehr in Andor, zu denen er sich teleportieren könnte. Er saß hier fest.\bigskip







Aćh sah Barz zu Boden sinken. Besorgt schielte sie zu ihm.

„Was habt Ihr ihm gesagt?“

„Das ist wohl an ihm, dir mitzuteilen, wenn er es denn will“, sprach der Hüter entschuldigend.

Verständnisvoll nickte Aćh und fragte:

„Was ist denn nun mit Turr? Was ist mit ihm geschehen? Können wir ihn finden? Wo können wir ihn finden?“

Sich am Kopf kratzend überlegte der Hüter der Zeit eine Zeit lang mit geschlossenen Augen. Dann hüpfte sein Kopf heftig auf und ab, als er endlich die erlösenden Worte sprach:

„Ja, ich glaube, mich zu erinnern. Du hast mir davon erzählt, wie ihr Turr wiedergefunden habt. Der Takuri verirrte sich wohl auf seinem Rückweg von Barz‘ Heimat nach Tulgor. Er landete dort, wo kein feuriges Wesen je sein sollte. Ihr beide werdet ihn finden in einer Schlucht hoch oben im Kuolema-Gebirge. Im ewigen Eis gleich hinter dem uralten Felsentor. Ihr Takuri-Hüter solltet irgendwo Aufzeichnungen haben darüber, wo sich das befindet.“

Aćh verpasste die Implikation, dass sie und der Hüter der Zeit sich eines Tages wiedersehen und ausgiebig unterhalten würden. Eine Vielzahl von Emotionen durchströmte sie, sowohl Freude über die Information als auch vor allem Sorge um Turr. Der Arme saß nun schon seit mehreren Tagen im ewigen Eis fest. Welche Tortur hatte er erlitten? Wie oft war er bereits erfroren?

Barz am Arm packend sprach Aćh: „Wir müssen zurück! Wir müssen sofort zurück – ins Kuolema-Gebirge!“\bigskip







Die Rückreise von Agarb zum Nestbaum verlief ereignislos. Kein Sturmgeist überfiel das Hängeschiff. Das Wetter war großartig. Das Essen war gut. Die Aussicht war bombastisch.

So rasch wie möglich hatte Aćh Falken ausgesandt, damit die anderen Hüter bei ihrer Ankunft bereits alle bekannten Notizen und Karten zu den Schluchten des ewigen Eises und dem uralten Felsentor zusammengesucht hatten. Diesen Aufzeichnungen nach befand sich das Felsentor in der Nähe eines Waldes, am oberen Ende eines geschlängelten Pfades durch ein Meer aus Steinen den Berg hinauf. Diesem Pfad würden sie folgen.

Das Küken von Saarćhan wurde gemeinsam mit der kleinen Sabri am Nestbaum in Obhut gegeben. Barz trennte sich nur äußerst schweren Herzens von seiner Echse, doch meinte er, dass die Kälte der Berge der Kleinen alles andere als guttun würde. Sie würde ohnehin nur in eine Winterstarre fallen, wenn sie sie mitnähmen.

Aćh war zwar nicht überrascht, doch erfreut darüber, dass Barz sie bereitwillig auf ihrer Reise in den Kuolema begleiten würde. Klar, Turrs Verschwinden war seine Schuld, doch hätte er sich geradesogut von dieser Verantwortung drücken können wie alle anderen Takuri-Hüter am Nestbaum.

„Wenn der Hüter der Zeit meinte, dass ihr beide den Takuri zurückholen werdet, dann werdet ihr beide das tun müssen“, hatte Oma Òkôkó bestimmend gesprochen, „Der Kuolema ist tödlich, und wir können nicht für die Sicherheit weiterer Begleiter garantieren. Doch unsere Wünsche und Hoffnungen werden bei euch sein. Und die beste Bergsteigerausrüstung diesseits des Ozeans.“

Der Weg begann, wie so oft im Gebirge, mit einem weiten offenen Pfad, der mit zunehmender Höhe schmaler wurde. Es folgten steile Passagen und schmale Schluchten. Die beiden kamen gut voran und die Laune war den Umständen entsprechend prächtig. Je höher sie kamen, desto kühler und windiger wurde es. Jetzt waren sie froh um die schneefeste Ausrüstung, die sie am Nestbaum erhalten hatten. Aćh und Barz schlossen ihre Kleidung, so gut es ging, und zogen die Kapuzen ihrer Umhänge über ihre Köpfe.

Im Laufe des Aufstiegs nahm der Wind noch zu und die Temperatur sank stark ab. Sie waren an der Schneegrenze angekommen. Jetzt hieß es noch vorsichtiger sein, denn der Pfad war unter der Schneedecke nur zu erahnen. Während einer Rast setzte Schneefall ein. Mit noch enger zusammengebundenen Kapuzen bahnten sie sich ihren weiteren Weg durch den knietiefen Schnee, bis sie endlich das Felsentor erreichten.

Das legendäre Felsentor, hinter dem der Legende nach die Eis-Dämonen und Geister des Gebirges ihr Unheil trieben.

Bereits aus der Ferne sah das Tor gigantisch aus. Wie ein Torbogen spannte es sich über den Weg.

„Schau her, wir haben es geschafft!“, rief Aćh freudig und zeigte auf das Felsentor. Barz schüttelte nur seinen Kopf und deutete an eine Stelle zwischen ihrem Pfad und dem Tor, wo selbst durch das dichte Schneetreiben hindurch eine dunklere Stelle zu erkennen war.

Ein tiefer, gezackter, breiter Spalt trennte Aćh und Barz von dem Felsentor, hinter dem sich Turr laut dem Hüter der Zeit befand. So ein Spalt war auf keiner Karte eingezeichnet gewesen. Würden sie ihn überqueren können?

„Was tun?“, fragte Barz. Er zückte ein Seil und blickte Aćh fragend an.

„Wo willst du das befestigen?“

„Nicht. Springen. Tod. Gefahr.“

Aćh nickte ergeben.

„Gehen wir mal näher heran. Vielleicht finden wir irgendeinen Ort, an dem wir hinübersteigen können. Schlimmstenfalls nutzen wir dein Bannpulver, um ein bisschen Schnee in der Luft schweben zu lassen und darüberzusteigen...“

„Bannpulver nicht viel. Nicht so machen. Rutschen. Fallen. Tod.“

„Es funktioniert nicht so? Tja, du musst es wissen. Lass uns dennoch mal näher treten.“

Aćh und Barz schritten durch den tiefen Schnee so nahe an die Schlucht vor dem Felsentor, wie sie konnten. Von nahe sah das Tor noch beeindruckender aus. Es war nicht natürlichen Ursprungs. Auch wenn es auf den ersten Blick wie eine zufällige Felsenformation wirken mochte, erkannte Aćh von nahe zwei mächtige Säulen, die das Tor stützten.

Durch das Tor hindurch erblickten sie eine riesige glatte Eisfläche dahinter. Schimmerndes Weiß und kaltes Blau, soweit das Auge reichte. Das ewige Eis. Einst war es vielleicht ein See in einem komplexen System von tiefen Schluchten gewesen, doch nun war es eine legendäre Eismasse, die sich quer durch den Kuolema zog.

Während Jahrhunderte alte, verbleichte Karten noch angaben, dass das ewige Eis am Grunde des großflächigen Tals und tief hinter dem Felsentor lag, reichte die Eisfläche nun bis direkt ans Felsentor. Als hätte sie sich in den vergangenen Jahrhunderten stetig die Talhänge entlang ausgebreitet und wäre nur vom Felsentor aufgehalten worden.

Jahr um Jahr zogen wagemutige Jugendliche aus, um Wege durch die Berge zu finden. Niemand, der das ewige Eis erreicht hatte und weitergezogen war, kehrte je wieder zurück, um davon zu berichten. Und wenn die ersten Sonnenstrahlen des Frühlings den Schnee an den felsigen Hängen schmolzen, gab ihnen das Gebirge die Toten zurück. Oft sprachen die Alten von Geistern und Dämonen, die in den Bergen hausten. Und auch Aćh, die eigentlich nicht viel von solchen Gruselgeschichten hielt, langte beinahe unbewusst an ihren Schwertgriff.

„Da!“, rief Barz, aufs Felsentor zeigend. Seine nackte Hand war bereits bläulich angelaufen. Er hätte Handschuhe anziehen sollen.

Aćh kniff ihre Augen zusammen und hatte ziemlich Mühe damit, durch die Schwaden fallender Schneeflocken hindurch zu erspähen, was Barz‘ Augen erkannt hatten.

Vor einer der beiden imposanten Säulen des Felsentors stand eine über und über mit Schnee und Eis bedeckte Statue einer Frau mit einem Schwert. Wie ein Wachposten wirkte sie. Es hätte Aćh nicht gewundert, wenn sie sich aus der Säule gelöst hätte und mit gezücktem Schwert auf sie und Barz zugekommen wäre.

Da löste sich die Frau aus der Säule, schüttelte ein wenig Schnee von sich und trat zwei Schritte zur Seite, doch stand sie immer noch auf der anderen Seite des Felsentors. Ihr Schwert hielt sie hingegen locker an ihrer Seite. Es war ebenso eisblau wie ihr Körper und ihre Kleidung. Was für ein Material war das?

Die Frau war tatsächlich keine Statue. Aber auch kein Mensch. Sie wirkte, aus wäre sie aus purem Schnee geformt. Aus ihren pupillenlosen Augen glühte ein weißes Licht. Und unter ihrem langen Haarschopf lugten zwei kurze Geweihe hervor.

„Ängstigt euch nicht, ich will euch kein Unheil“, sprach die Frau mit einer klirrenden Stimme. Sie zückte aus dem Schnee hinter sich einen mächtigen Eisblitz und warf ihn in die Schlucht zwischen ihr und den beiden Neuankömmlingen. Unter lautem Getöse wie von berstendem Eis wuchs unter Aćhs und Barz‘ staunenden Blicken ein Steg aus Eis. Ihr Weg zum Felsentor.

„Seid ihr hier, um diesen Feuervogel abzuholen?“, fragte die seltsame Frau

Aćh nickte, immer noch misstrauisch.

„Dann kommt und nehmt das elende Vieh mit!“, rief die Schneefrau, „Es schmilzt uns schon seit Tagen ein Loch ins ewige Eis!“

„Wer sein?“, fragte Barz nun, der von Aćh unbemerkt seinen Bogen gezückt hatte und ihn locker an seiner Seite hielt.

„Hui, jetzt macht mir mal keine Angst!“, rief die Schneefrau überspitzt, „Wir Eis-Dämonen halten nicht so viel aus, wie ihr wagemutigen Menschen immer meint.“

„Wer bist du?“, wiederholte Aćh Barz‘ Frage.

„Mein Name ist Nesdora, und ich bin eine einsame Bewohnerin des ewigen Eises. Eine der wenigen. Und möglicherweise die Einzige, die noch bei klarem Verstand ist. Die lange Zeit im Eis tut den wenigsten langfristig gut. Auch mir wird es nicht auf ewig guttun.“

Aćh ging in Gedanken die Geschichten über die Eis-Dämonen des Kuolema durch, die sie schon seit ihrer Kindheit vernommen hatte. Sie glaubte, sich daran zu erinnern, dass vor einem Jahrzehnt eine Tulgori namens Nes im Gebirge verschollen gegangen war. Und dass im Gegensatz zu den Leichen ihrer Begleiter die von Nes nie gefunden worden war.

Vorsichtig marschierte Aćh über die Eisbrücke auf die Fremde zu. Hinter ihr hörte sie Barz zunächst einen protestierenden Laut produzieren, ehe er ihr folgte.

Vorsichtig trat Aćh näher zum Felsentor. Nun konnte sie auch die kleinen Runenkritzeleien erkennen, die auf den Torsäulen zu sehen waren. Faszinierend.

„Sieh her, da hinten sitzt er, der elende Feuervogel“, zischte Nesdora und zeigte aufs ewige Eis hinaus. Kaum hundert Meter vom Felsentor entfernt war eine kleine Senke im Eis zu erkennen, in der etwas brannte.

Ja, das musste Turr sein! Von hier aus konnte sie nur eine kleine Gestalt in einer Wasserlache erkennen, die vor sich hin flackerte, doch passte dies so gut zur Prophezeiung des Hüters der Zeit. Der Arme! Nun war seine Rettung nur noch Schritte entfernt. Aćh setzte an, zu ihm zu rennen.

„Wartet!“, rief Nesdora, „Wenn ihr einfach so durch das Felsentor zu schreiten versucht, werdet ihr unsägliche Schmerzen erleiden. Es wurde einst von den Temm geschaffen, um unsere beiden Völker zu trennen. Hier, zieht diese Eiskristall-Ketten an. Sie gewähren euch Durchgang.“

Nesdora zog wie aus dem Nichts zwei durchscheinende Halsketten hinter ihrem Rücken hervor. An beiden hingen jeweils drei rautenförmige Eiskristalle, ähnlich zu den kleinen Aschedöschen, die Aćh an ihrer eigenen Halskette trug. Und als Aćhs Blick Nesdoras Hals streifte, erkannte sie, dass der dünne Umgang der Eis-Dämonin von einer identischen Eiskristallkette zusammengehalten wurde.

Nesdora holte aus und warf ihnen beiden je eine Eiskristallkette zu. Während Aćh die ihre mit einem Handschuh fing, fasste Barz die seine geschickt in einer nackten Hand auf.

„Huch! Kalt. Viel kalt.“, beschwerte er sich.

Nesdora verdrehte die Augen, doch wirkte die Geste irgendwie unnatürlich. Oder zumindest ungewohnt.

Mit klirrender Stimme fuhr sie fort: „Zieht die Ketten über euren Nacken, so, dass sie direkten Kontakt mit eurer Haut haben. Sonst hält die Magie im Tor euch für Eindringlinge und hält euch zurück. Oder Schlimmeres. Ich habe es zum Glück noch nie miterlebt.“

Barz tat wie geboten, öffnete seinen Wintermantel so weit, dass sein Hals entblößt wurde, und legte sich seine Eiskristallkette um. Oder besser gesagt, er versuchte es. Er trug bereits eine Kette mit zwei klobigen Ringen um seinen Hals, und diese kollidierten nun mit den Eiskristallen von Nesdoras Kette, welche deswegen nicht schön anliegen wollte. Frustriert riss Barz sich die Ringkette von seinem Hals und ließ selbige achtlos zu Boden fallen.

„Du auch“, mahnte Nesdora Aćh. Diese hatte bislang ihre Kette nur in ihrem Handschuh gehalten und tat nichts dergleichen. Ihr war die Situation nicht geheuer, auch wenn sie nicht in Worte zu fassen vermochte, warum.

Aćh warf einen Blick auf Nesdora, welche sie aus ausdruckslosen Augen anstarrte. Zurück zu Barz, welcher die Eiskristallkette an seinem Hals gerade rückte. Zurück zum Felsentor, dessen Säulen von zahlreichen Runen übersät waren. Zu Turr, welcher kaum hundert Meter davon entfernt am Boden saß und litt.

Dann schloss sie ihre Augen, und schritt energisch durchs Felsentor.

Keine unsäglichen Schmerzen überfielen sie.

Keine Magie hielt sie davon ab, hindurchzumarschieren.

Obwohl ihre Kette keinen Kontakt zu ihrer Haut hatte, hatte das Tor sie nicht aufgehalten.

Nesdora hatte gelogen.

Anschuldigend blickte Aćh die Eis-Dämonin an und schleuderte ihre Eiskristallkette demonstrativ zu Boden.

„Okay, das sieht jetzt böse aus...”, begann Nesdora, beschwichtigend ihre Hände hebend.

„Barz, zieh deine Kette aus!”, bellte Aćh, den Blick weiterhin auf Nesdora gerichtet.

Nesdora versuchte es erneut: „Es gibt eine ganz logische Erklärung. Ich habe euch nicht ganz angelogen. Und ich will euch nichts Böses.”

„Barz, deine Kette!”, rief Aćh erneut. Barz stand weiterhin reglos auf der anderen Seite des Felsentors und blickte sie an. Dann legte er seinen Kopf schief und sprach in gebrochenem Tulgor ganz unschuldig: „Nicht abnehmen wollen.”

Jetzt hatte Aćh aber genug. Sie langte zum goldenen Schwert, welches immer noch in seiner Scheide an ihrer Seite steckte.

Gleichzeitig geschahen zwei Dinge: Zum einen langte Barz hastig nach seinem Pulvergürtel. Zum anderen sprang Nesdora Aćh an und hinderte sie mit einem eiskalten Griff am Ziehen ihres goldenen Schwertes.

Nesdora hauchte aus. Eisiger Dampf schwallte aus ihrem Mund auf Aćhs Kleidung herunter und überzog sie mit einem Reif aus Forst. Aćh schrie verärgert auf, doch war ihre Kleidung natürlich am Nestbaum mit feuerfestem Sufarsaft überzogen worden. Was gegen das Feuer der Takuri isolierte, half auch gegen die Eiskälte des Kuolema.

Von Nesdoras Kälteangriff unbetroffen, ließ sich Aćh nach vorne fallen. Nesdora, welche darauf nicht vorbereitet gewesen war, stürzte in den Schnee und ließ Aćh los. Diese verpasste ihr einen Faustschlag und stolperte von ihr weg.

Erneut machte sich Aćh daran, ihr goldenes Schwert zu ziehen. Mit Blick auf Barz bemerkte sie jedoch, dass ihr kaum die Zeit dafür blieb. Barz hatte seinen grünen Bannpulversack ergriffen und rieb nun eine Prise des gefährlichen Materials in seiner Handfläche warm. Dabei blickte er starr auf Aćh. Das konnte kein gutes Zeichen sein. Gleich würde er das Pulver werfen.

Mit einem mächtigen Schritt hopste Aćh zurück durchs Felsentor, weg von Nesdora und hin zu Barz, und packte dessen Faust mit ihren Handschuhen.

„Lass los, Barz“, redete sie ruhig auf ihn ein, „Ich bin es, Aćh! Ich bin nicht dein Feind. Diese Dämonin hat etwas mit dir gemacht, doch du kannst stärker sein als ihre Zauber!“

„Kenne dich. Aćh, Takuri-Hüterin“, keuchte Barz, „Nicht wünschen Kampf. Eis Hilfe. Siantari Hilfe.“

„Barz! Lass das Pulver fallen! Was soll das überhaupt heißen?“

„Bald. Bald verstehen.“

Einen Augenblick lang rangelten Aćh und Barz noch um das Bannpulver in seiner Faust. Im Geschüttel lösten sich einige Körnchen und gelangten in Aćhs Nase. Aćh kannte die Anzeichen ihrer Niesreizattacken gut genug, um zu wissen, was nun folgte, doch wusste nicht, wie sie hier und jetzt darauf reagieren sollte.

Nieser um Nieser schüttelte ihren Körper durch, und da sie Barz‘ Hand festzuhalten versuchte, auch diese. Aćh fühlte, wie ihr Barz‘ Pulverfaust entglitt, ja, wie sie selbige gar durch den Schwung eines Niesers schwungvoll irgendwohin stieß. Sie stolperte ängstlich zurück, während sie das inzwischen vertraute magische Knattern des sich entfaltenden Bannpulvers hörte.

Nachdem das letzte „Hatschi“ Aćhs Kopf durchgeschüttelt hatte, öffnete sie ihre Augen, in der Hoffnung, das Bannpulver möge sie selbst verfehlt haben.

Es hatte.

Vor ihre drehte und wand sich Barz und versuchte, vom Fleck zu gelangen. Doch seine Faust, welche vorhin eine Prise Bannpulver gehalten hatte, steckte wie angegossen in der Luft fest, die Finger weit gespreizt, von einem grünlichen Glühen umgeben. Egal, wie sehr sich Barz bemühte und den Rest seines Körpers bewegte, seine gebannte Faust regte sich kein bisschen und hielt ihn zurück.

Freilich konnte sich Aćh nicht lange darüber freuen. Barz war ihr noch nahe genug, dass er sie am Umgang packen und nach vorne ziehen konnte.

Ihr Kämpferinstinkt erwachte. Aćh ließ sich von Barz‘ Schwung mitziehen und stieß sich gerade rechtzeitig vom Boden ab, sodass ein gewöhnlicher Gegner von ihr mitgerissen und zu Boden geworfen worden wäre. Leider war Barz‘ gebannte Faust kein gewöhnlicher Gegner.

Sie stieß sich im richtigen Moment vom Boden ab und rollte über Barz hinweg, doch statt dass dieser zu Boden ging, ertönte ein grausiges Knirschgeräusch und Aćh wurde allein in den Schnee geschleudert.

Als sie hinter sich blickte, drehte sich ihr Magen um. Barz‘ Faust, weiterhin vom grünlichen Glühen des Bannpulvers umgeben, steckte weiterhin in der genau selben Position in der Luft fest. Der Rest von Barz‘ Körper befand sich nun allerdings in einem sehr unnatürlichen Winkel dazu. Barz‘ Unterarm sah aus, als hätte er neben seinem Ellbogen plötzlich ein zweites Gelenk entwickelt.

Es schmerzte Aćh bereits, Barz’ gebrochenen Arm so zu sehen. Dieser biss hingegen nur seine klappernden Zähne zusammen und ließ keinen Schmerzenslaut vernehmen.

Jetzt hatte Aćh endlich die Zeit, ihr goldenes Schwert zücken und auf Barz richten. Nicht, dass dies nun noch viel nützen würde. Sie blickte zurück zum Felsentor und erkannte mit Freude, dass Nesdora auf der anderen Seite stand und ihr bloß einen eisigen Blick zuwarf. Offenbar konnte die Eis-Dämonin das Felsentor nicht durchschreiten. Turr mochte noch nicht gerettet sein, doch die akute Gefahr war gebannt.

Da ertönte ein Ächzen hinter ihr. Aćh drehte sich zurück und erblickte, wie Barz mit seiner gesunden Hand einen violetten Stein in seinen Mund schob und darauf biss. Weiß der Himmel, woher er diesen Stein gezogen hatte. Es knackte, und als Barz seinen Mund wieder öffnete und den Stein ausspuckte, war dieser in mehrere spitze Splitter zerfallen. Von ihnen allen ging ein immer heller werdendes Licht aus, welches Aćh in den Augen schmerzte. Und wo diese Lichtstrahlen auf das hellgrüne Leuchten des Bannpulvers trafen, verschwand das grünliche Glimmen in der Luft. Das Bannpulver löste sich!

Barz‘ gebrochener Arm wurde nicht länger in der Luft gehalten, sondern fiel an seine Seite. Er war wieder frei. Und schon wieder griff er in ein Pulversäcklein. Was sollte sie tun? Sie konnte Barz doch nicht einfach abstechen!

Da dachte Aćh zurück an den Nixenstaub, das Barz ihr geschenkt hatte. Gegen einen starken Feind solle sie es einsetzen, um ihn zu schwächen, hatte Barz gesagt. Die Gelegenheit schien unglaublich passend. Sein eigenes Geschenk würde den von irgendeiner dämonischen Magie belegten Nomaden schwächen, auf dass Aćh ihm hoffentlich helfen konnte. Wie poetisch.

Sie griff an ihren Wintermantel und zückte das weiße Nixenstaub-Säcklein. Die Schnur war schnell geöffnet, und ebenso eine Prise des Pulvers ergriffen.

Barz blieb einfach nur stehen und blickte sie regungslos an.

Aćh pustete ihm den Nixenstaub ins Gesicht.

Barz grinste.

Urplötzlich knackte es in Aćhs Arm. Unsägliche Schmerzen strahlten ihn entlang. Das goldene Schwert entglitt ihr und stürzte – wie Aćh – selbst in den Schnee. Verkrampft hielt sie ihren Unterarm an ihren Körper, während es darin knirschte und schmerzte, als pulverisierten sich all ihre Armknochen. Ihr Gesichtsfeld verengte sich. Als sie zurück zu Barz blickte, erkannte sie, dass dessen gebrochener Arm auf einmal wieder ganz gerade und gesund aussah.

Bei den sieben Feuern des Himmels, was war soeben geschehen?!

Barz fackelte nicht lange, sondern stürzte sich auf Aćh und drückte sie zu Boden. Er ergriff Aćhs gesunden Arm, zerrte ihren Handschuh zur Seite und führte ihre Finger mit seinen eiskalten Händen zu seinem Hals. Dann presste er Aćhs Hand auf seine Eiskristallkette.

Sie war eiskalt.

Und noch so viel mehr.\bigskip







Alle Wut und alle Sorgen, ja, sogar aller Schmerz aus ihrem gebrochenen Arm flossen aus Aćhdora wie aus einem löchrigen Sieb. Barzdur blickte ihr in die Augen. Sie verstanden einander.

Aćhdora fühlte, wie starke Magie aus den Eiskristallen von Barzdurs Kette seinen und ihren Körper einhüllten. Die magischen Kristalle waren der Ursprung von Barzdurs aktuellen Gedanken und seines Willens. Und nun waren sie auch der Ursprung von Aćhdoras Gedanken und Willen.

Plötzlich wurde Aćhdora bewusst, dass sie beobachtet wurde. Aus weiter Ferne spürte sie einen forschenden Geist auf sie blicken. Wie etwas Fremdes ihre Erinnerungen durchforschte und selbst welche mit ihr teilte.

Vor ihrem inneren Auge sah sie das Antlitz einer Frau, weiß und durchscheinend, langhaarig und gehörnt. Die Eis-Dämonin sprach, ohne ihren Mund zu öffnen, und ihre klirrende Stimme bohrte sich tief in Aćhdoras Geist.

„Ich bin Siantari. In den tiefen Schluchten des Kuolema scheint niemals die Sonne auf das ewige Eis, und dort ist mein Reich.“

Ihr wurde bewusst, als hätte sie es schon lange gewusst und sich erst jetzt wieder daran erinnert, dass Siantari nun schon seit beinahe 500 Jahren die Dämonin des ewigen Eises war, die über die schattigen Täler hinter dem Felsentor herrschte und das ewige Eis nährte. Und ihr wurde bewusst, dass das schlecht war. Sie wurde wie das ewige Eis im Kuolema eingesperrt, während es sich doch eigentlich über die gesamte Welt ausbreiten sollte. Glücklicherweise gab es nun sie und Barzdur.

Ja, Aćhdora und Barzdur befanden sich auf der anderen Seite des Felsentors. Sie waren frei. Wenn sie einige Stunden warteten, würden die Eiskristallketten auch ihre Körper erfrieren lassen und ihnen ein Leben als Eis-Dämonen schenken. 500 Jahre lang könnten sie so bestehen, bis auch ihre Schneekörper nicht mehr halten würden und sie einen Nachfolger für ihre Ketten zu finden hätten.

Aber das war in weiter Zukunft. Siantari hatte ein dringlicheres Problem. Ihr eigener Schneekörper war schon beinahe 500 Jahre alt. Lange würde er nicht mehr halten. Sie musste einen Nachfolger finden. Und Siantari wusste, dass die abenteuerlustigen Tulgori, die immer wieder dieses Gebirge zu erklimmen versuchen, nicht als Nachfolger taugten. Schwächliche Jungen, die meinten, das Eis der Berge würde sie zu Männern formen…

Glücklicherweise war es Siantari als erster Eis-Dämonin des Kuolema gelungen, Eiskristallketten zu formen und ihre Kräfte so auf weitere Menschen zu übertragen, ohne selbst zu vergehen. Leider hielt der schwächere Geist dieser neuen Eis-Dämonen die Einsamkeit des Gebirges kaum aus, und mit der Zeit wurden sie alle von einem Wahn befallen.

Doch Aćhdora und Barzdur befanden sich im Gegensatz zu den anderen Durs und Doras außerhalb des Felsentors. Sie konnten nach Tulgor zurückkehren und einen würdigen Nachfolger für Siantari finden. Vielleicht gar einen Temm, der wusste, wie man das Felsentor öffnete. Dann wäre es endlich soweit. Das Felsentor könnte sich öffnen. Siantari könnte wieder frei sein. Das ewige Eis könnte sich weiter ausbreiten. Die gesamten umliegenden Lande könnten in eine Eiswüste verwandelt werden.

Ja, Aćhdora und Barzdur waren bereit, in Siantaris Dienst zu treten. Vielleicht, wer weiß, würde ihnen außerhalb des Felsentors auch das Schicksal aller anderen Durs und Doras des Gebirges erspart blieben. Und selbst wenn sie eines Tages der Wahn befallen sollte, würde es das wert gewesen sein.

Alles war gut.\bigskip







In diesem Augenblick wurde Aćh an ihrem langen roten Umhang nach hinten gerissen. Ihr Gesichtsfeld verfärbte sich schneeweiß, als sie über den verschneiten Boden rutschte und ihre Kapuze sich mit kaltem Nass füllte. Ihr Geist wurde wieder klar. Leider kehrten auch die pochenden Schmerzen in ihrem Arm zurück. Sie hatte Mühe, die Eindrücke der letzten Augenblicke einzusortieren. Und die Zeit drängte.

Nesdora rupfte weiterhin an Aćhs Umhang und näherte sich ihr langsam. Die Eis-Dämonin hatte die zweite Eiskristallkette in den Händen, die für Aćh bestimmt war. Offenbar hatte sie sie eingesammelt und hatte nun vor, sie Aćh aufzuzwingen. Doch hatte sie wohl nicht erkannt, dass Barz Aćh bereits erwischt hatte, und ihr nun ungewollt einen Moment der Klarheit geschenkt. Einen Moment, um sich zu wehren. Ihre Gedanken rasten.

Aćh lag gerade mitten unter dem Felsentor, ein bösartiger Barz zu ihrer Linken und eine nicht nette Nesdora zu ihrer Rechten. Doch war sie nicht allein. Sie hatte Turr! Noch mochte der Takuri keine Ahnung haben, dass sie hier war, doch würde er ihr sicherlich zu Hilfe kommen, wenn sie ihn rief. Warum hatte sie nicht früher an ihn gedacht?!

„Herjo! Turr, Herjo!“, brüllte Aćh aus voller Kehle. Irgendwie schaffte sie es, ihr Schwert zu packen und von Barz zurückzustolpern. Sie wankte durchs Felsentor auf Nesdoras Seite. Zum ewigen Eis hin. Und zu Turr. Mit Nesdora konnte sie es vielleicht selbst in diesem geschwächten Zustand aufnehmen, wenn sie nur die andere Eiskristallkette zerstören konnte. Um Barz würde sie sich später kümmern müssen.

Ein lautes Knattern ertönte hinter ihr. Als sie zurückblickte, sah sie, wie dutzender weiße Lichtkugeln vor Barz herumwirbelten und ihn zurückdrängten, während er unter großer Anstrengung vergeblich versuchte, das Felsentor zu durchschreiten. Wie vermutet ließ das Tor keine Eis-Dämonen durch, auch keine Angehenden.

Das hieß, dass sie nur noch mit Nesdora zu kämpfen hatte. Und sie war nicht allein.

Ein melodisches Kreischen kündigte Turrs Ankunft an. Der Takuri war schwach, und er schlidderte mehr übers Eis, als dass er darüber flog, doch bewegte er sich stetig auf Aćh und Nesdora zu.

Aćh und Nesdora hatten beide ihre jeweiligen Schwerter erhoben und hielten sie vorsichtig vor sich. Aćh versuchte, sich an die Kunst des Schwerttanzes zurückzuerinnern, doch sie war erschöpft und ihr dominanter Arm war gebrochen. So musste sie sich mit einigen uneleganten Schwüngen ihres Schwerts begnügen. Hauptsache, sie hielten Nesdora noch einige Momente von sich selbst weg. Einige Momente, bis...

Jetzt!

„Turr?! Advaria! Advaria meza!”

Aćh wartete nicht ab, zu sehen, ob Turr sie verstanden hatte. Sie drehte sich ab. Ein lautes Klatschen von goldenen Flügeln war zu hören. Eine Hitzewelle rauschte über Aćh hinweg. Als sie sich wieder zurückdrehte, drehte Turr stolz Kreise über einer am Boden zusammengekugelten, reglosen Eis-Dämonin. Turr sah größer und kräftiger aus als vor dem Kampf. Stichflammen flackerten über sein wunderschönes Gefieder.

Nesdora war vorerst ausgeknockt. Zeit, sich um Barz zu kümmern.

„Turr?! Kurjo!“, rief Aćh, und zeigte mit ihrem gesunden Arm auf Barz. Dieser hatte aufgehört, sich durchs Felsentor zu kämpfen versuchen, und starrte ausdruckslos auf Aćh. Sein Problem hatte mit diesen Eiskristallketten zu tun. Da war doch vernünftig, anzunehmen, dass man mit genügend Hitze Abhilfe schaffen konnte.

Turr ließ nicht lange auf sich warten, sondern stieß einen melodischen Ruf aus, breitete seine Schwingen aus und hielt rasend schnell auf Barz zu. Aćh humpelte ein wenig langsamer hinterher.

Doch auch Barz war nicht untätig. In Windeseile hatte er seinen Bogen erhoben, einen Pfeil aus seinem Köcher gezückt und angelegt. Dann zielte er. Aćh ließ sich zu Boden fallen und duckte sich hinter eine Schneewehe. Dann blickte sie panisch wieder auf, als ihr bewusst wurde, dass vermutlich nicht sie das Ziel war.

„Turr! Herjo! Nepom...“

Ein lautes Kreischen unterbrach ihren Ruf. Barz hatte seine Bogensehne losgelassen und den Pfeil abgeschossen. Turr, der nur noch weniger Meter von ihm entfernt gewesen war und sich wie ein Seeadler mit ausgestreckten Krallen auf ihn gestürzt hatte, wurde von Barz‘ Pfeil mitten in der Brust getroffen. Der Feuervogel schrie. Sein Schrei wurde von einem Knistern und Brodeln überdeckt. Rund um den Pfeil in seiner Brust brachen Flammen hervor und verzehrten seinen Körper. Doch während Turr starb und zu Asche verglühte, stürzte sein sich wild umherdrehender Körper weiterhin auf Barz zu.

Der Steppennomade blinzelte nicht einmal, während der sterbende Takuri ihm mit Höchstgeschwindigkeit ins Gesicht klatschte und es in einen Feuerball tauchte.

Aćh zwang ihren protestierenden Körper, sich aufzurichten und zu Barz zu stolpern. Ihren gebrochenen Arm an die Seite gepresst, humpelte sie so schnell sie konnte durchs Felsentor auf ihn zu.

Barz hatte seinen Bogen losgelassen und war zu Boden gestürzt. Sein Bart war angekokelt und seine Miene eine Maske der Furcht. Neben ihm lag ein halb geschmolzener Pfeil und ein unversehrtes, zitterndes Takuri-Küken in einem Häufchen Takuri-Asche, welches langsam ein Loch in den Schnee schmolz.

Aćh kniete auf Barz Brust nieder und hinderte den zappelnden Nomaden so am Aufstehen. Barz‘ Wintermantel war oben weiterhin so weit geöffnet, dass sie gut auf seinen Hals sehen konnte. Es sah alles andere als gut aus. Die Eiskristallkette klebte weiterhin wie angeleimt auf seiner Haut, und schlimmer noch, rund um die Kette hatte sich sein Hals bläulich-weiß verfärbt, als wäre er aus Eis geformt.

„Weg!“, hauchte Barz schwach, „Feuer... nicht Feuer... Takuri... weg!“

Sein Knie traf Aćhs Rücken, doch so leicht ließ sie sich nicht von ihm schubsen. Aćh langte neben Barz, unter das Turr-Küken, und ergriff mit ihrem Handschuh eine Handvoll glühender Takuri-Asche. Dann presste sie den feurigen Mischmasch auf Barz‘ Hals. Barz schrie auf und strampelte noch heftiger, doch sie ließ sie nicht ab.

Endlich wurde Barz still. Etwas knirschte. Als Aćh ihre Hand von seinem Hals ließ, war von der Eiskristallkette nichts mehr übrig außer dahinschmelzende Eissplitter. Barz‘ Hals sah übel aus, ein Mischmasch aus eisblauen Flecken und rötlichen Verbrennungen, doch seine Augen blickten wacher drein als zuvor.

„Geht es wieder?“, fragte Aćh vorsichtig. Barz murmelte etwas Unverständliches. Dass er sich nicht mehr wehrte, wertete Aćh als gutes Zeichen. Ihre eigene Ergebenheit gegenüber Siantari hatte ja auch aufgehört, sobald sie keinen direkten Kontakt mehr zur Eiskristallkette gehabt hatte. Alles war wieder gut.

Ächzend rollte Aćh von Barz runter, heulte auf vor lauter Schmerzen in ihrem gebrochenen Arm und tastete nach dem kleinen Turr. Mit großen Kugelaugen blickte das Küken sie an und wimmerte leise. Aćh steckte ihn in eine Tasche ihres Mantels, streichelte sein Gefieder und redete ihm gut zu.

„Gefahr!“, rief Barz.

Aćh blickte seinem ausgestreckten Finger entlang und erblickte Nesdora, die Eis-Dämonin, welche sich wieder aufgerichtet hatte und auf sie beide zuhumpelte. In einer Hand hielt sie ihr bläulich schimmerndes Eisschwert, die andere war leer.

Noch befanden sie sich auf der anderen Seite des Felsentors. Nesdora konnte sie technisch gesehen nicht erreichen. Sollten sie versuchen zu fliehen? Da manifestierte Nesdora wie aus dem Nichts einen gewaltigen Eisblitz in ihrer leeren Hand und hob diese, als wolle sie den Blitz auf Aćh und Barz schleudern. Davor konnten sie nicht wegrennen.

Doch das war auch nicht nötig. Barz, Wut in seinen Augen, griff in eine seiner Manteltaschen und zückte einen dunkelbraunen Beutel. Noch während Nesdora zum Wurf ausholte, öffnete er es und pustete Nesdora eine Wolke von etwas Goldenem entgegen, was für Aćh verdächtig nach zusammengescharrtem Sand der Temm aussah. Wie viel Sand hatte Barz am Nestbaum gesammelt?!

Auf jeden Fall schien es zu wirken, denn die Eis-Dämonin nieste, stolperte und rutschte auf der glatten Fläche des ewigen Eises nach vorne. Ihr Eisblitz fiel ihr aus der Hand und verdampfte, als hätte er nie existiert. Ihr Eisschwert schlidderte durchs Felsentor zu Aćh. Diese fackelte nicht lange, sondern packte das Schwert mit ihrer nichtdominanten Hand, schritt durchs Felsentor nach vorne und ließ das Schwert auf die Eis-Dämonin niedergehen.

Es schnitt glatt durch die Dämonin, wie ein warmes Messer durch Butter. Nur an der Eiskristallkette um ihren Hals blieb es kurz stecken, doch nach einem Ruck zersplitterte auch die Kette. Während der Kopf der Dämonin zur Seite rollte und mit dem Geweih an einer Säule des Felsentors stecken blieb, löste sich der Körper der Eis-Dämonin langsam in viele einzelne Eiskristalle auf. Fast schien es, als versinke sie im ewigen Eis.

Aćh verharrte noch einen Augenblick, um sicherzugehen, dass Nesdora sich nicht irgendwo um sich erneut manifestierte. Wer konnte schon so genau wissen, über welche Fähigkeiten diese Dämonen verfügte. Dann traf sie die Realisation.

„Ich... ich habe sie getötet“, flüsterte sie, „Sie... sie war doch nur...“

Barz hob Aćhs Schwert auf und verpasste Nesdoras Kopf einen verächtlichen Tritt, sodass auch dieser aufs ewige Eis rollte und sich darin auflöste. Er betastete mit einer Hand sorgfältig seinen wunden Hals, während er ihr mit der anderen Aćhs goldenes Schwert überreichte. Noch immer war dessen Oberfläche unzerkratzt.

Aćh fuhr mit einem zitternden Daumen über die Mondsichel über dem Schwertgriff und schob das Schwert dann wieder in seine Scheide.

Sie schaute ein weiteres Mal sorgsam nach dem kleinen Turr-Küken und stolperte dann aus dem Felsentor hinaus, dicht gefolgt von Barz. Nur weg von hier.

Als sie einen vorsichtigen Blick zurück auf die ewige Eisfläche warf, erstarrte sie. Eine großgewachsene Gestalt kam rasch auf sie zu. Sie lief nicht über das ewige Eis, sondern schwebte in einem Wirbel aus Nebel und Schnee, die Arme auf beiden Seiten ausgestreckt. Noch war nur ihre Silhouette am Horizont zu erkennen. Doch kam die Eis-Dämonin rasend schnell näher. Bald würde man ihren Kopf genau erkennen können, und Aćh hätte sich stark gewundert, wenn dort kein Geweih zu sehen wäre. Und ein Blick, so kalt, dass man meinte, sofort zu Eis zu erstarren.

Siantari. Die oberste Dämonin des ewigen Eises.

Eine Eiseskälte befiel ihre Glieder. In ihr breitete sich ein Gefühl zunehmender Erstarrung aus.

Ein Ruck fuhr durch Aćhs Körper, als Barz sie packte, umdrehte und weiterzog.

„Turr hier. Gehen, gehen, gehen, jetzt!“, intonierte er panisch.

Mit zusammengebissenen Zähnen drehte sich Aćh um und stolperte ebenfalls den Weg die Berge hinab.

Als sie einen letzten Blick zurückwarf, sah sie mit Erleichterung, dass auch Siantari vom Felsentor zurückgehalten wurde. Noch immer hatte diese ihre Arme ausgebreitet und gebot dem Sturm um sie herum, zu wüten. Eine Wolke aus Schnee und Kälte jagte den Berg hinab auf Aćh und Barz zu und hatte sie rasch eingehüllt. Hätte Barz nicht schon so Aćh gestützt, hätten sie einander verloren.

Dann brachen sie aus dem Sturm heraus ins Freie. Noch immer war es bitterkalt, doch jagte ihnen kein peitschender Wind mehr ins Gesicht. Siantaris Macht schien ihre Grenzen zu haben. Sie hatten es geschafft.\bigskip







Etwas langsamer setzen Aćh und Barz ihren Weg zum Nestbaum zurück fort. Erleichtert, dass Aćhdora und Barzdur nun doch keine Realität würden.

„Verzeihung“, war alles, was Barz hervorbrachte, immer und immer wieder, „Verzeihung.“

Er tastete immer wieder nach seinem Hals und nach seinen Pulversäckchen, schien mit dem Ergebnis allerdings nicht zufrieden. Insbesondere hatte er keinen weiteren Nixenstaub mehr übrig.

„Nicht Nixenstaub geben sollen. Doch Gefahr. Viel Gefahr. Pulver Gefahr. Dumm. Dumm! Verzeihung.“

Aćh sah schon kommen, dass Barz sie nie wieder in die Nähe seiner Pulver lassen kommen würde. Nun, eigentlich störte sie das auch nicht zwingend. Ohnehin war nun vor allem wichtig, dass sie beide zum Nestbaum zurückkehrten, ehe sie erfroren. Und ehe ihnen die Nahrungsvorräte ausgingen.

Einmal blieb Barz abrupt stehen, als er nach seinem verletzten Hals langte und wohl erkannte, dass seine Ringkette immer noch fehlte. Die war nun wohl in Siantaris Gewahrsam. Er verzog sein Gesicht, schritt dann aber hastig weiter.

So humpelten Aćh und Barz den Berg hinunter. Die untergehende Sonne warf nur noch ein schwaches Licht auf das Aćh so wohlbekannte Land Tulgor. Sie betrachtete Turr in der Dämmerung. Noch immer kuschelte er sich schwach an ihre Seite, und nur ein leichtes Glimmen schimmerte über sein Gefieder. Nichtsdestotrotz wusste Aćh, dass das Leuchten seiner Federn ihr Licht sein würde in den dunklen Stunden, die vor ihnen lagen.

Bis zurück an den Nestbaum der Takuri schafften sie es vor dem Einbruch der Nacht nicht, und so suchten sie Schutz im erstbesten verlassenen Mineneingang. Barz fand eine kleine Holzkonstruktion, welche früher wohl mal einen Spiegel in sich verankert gehabt hatte, um das Licht des roten Mondes tiefer ins Gebirge zu lenken.

Mithilfe von Turrs Federn gelang es Aćh, das Konstrukt in Brand zu setzen. Dann ließ sie sich ächzend zu Boden sinken und umklammerte ihren gebrochenen Arm.

Barz blickte sie schuldbewusst an.

„Ich Wächter. Du Ruhe. Zukunft Baum Heiler“, sprach er ruhig, auch wenn Aćh nicht ganz klar war, ob er eher sie oder sich selbst beruhigen wollte. Es war egal. Schon bald hatte sich ihr unterkühlter Körper am Lagerfeuer etwas aufwärmen können, und trotz der stechenden Schmerzen in ihrem Arm sank sie bald darauf in einen unruhigen Schlaf.\bigskip







„Nicht Gute Nacht! Nicht Gute Nacht! Gefahr!“, erklang Barz Stimme viel zu laut. Aćh schlug ihre Augen auf. Sie fröstelte. Ihr Arm pochte höllisch. Das Lagerfeuer war erloschen und ihre Decke war zur Seite gerutscht. Barz stand ein wenig von ihr entfernt und blickte angespannt ins Dunkel der Mine hinein. Laute Geräusche waren von dort zu hören, ein Schlurfen und Stampfen, ein Brabbeln und Lärmen. Lebewesen. Waren das Arpachen, Sporne, etwa gar Lumiwürmer? Das wäre alles NICHT gut!

Aćh stemmte sich in die Höhe und zückte ihre Steinflöte. Eine rasche Melodie und das Kommando „Kurjo!“ genügte, und schon schwirrte der kleine Turr den Gang entlang, tiefer in die Mine hinein.

Im schwachen goldenen Licht, das von seinem Gefieder ausging, erkannte Aćh eine Vielzahl bleicher Gestalten in rudimentärer Kleidung. Unter ihren langen Kapuzen lugten breitnasige Gesichter hervor. Höhlenwichte! Mindestens vier! Sie grabschten mit klammen Klauen nach dem vorbeifliegenden Turr, bekamen ihn aber nicht zu fassen.

„Menschenfresser! Mensch essen!“, rief Aćh.

Barz verstand, machte große Augen, zückte seinen Bogen und legte einen Pfeil an.

„Weg! Husch!“ rief er den Höhlenwichten zu. Turr war inzwischen wieder bei Aćh angelangt, weswegen die Wichte aktuell nur an ihren glänzenden Augen im Schatten zu erkennen waren.

Ein Schrei erklang hinter Aćh. Sie wirbelte herum und erblickte, wie Barz von zwei weiteren Höhlenwichten angesprungen und umgeworfen wurde. Vor dem Mineneingang befanden sich weitere Wichte. Ihr Ausgang war versperrt!

Während Barz mit den beiden Wichten auf ihm rang, schlidderte sein Bogen hinüber zu Aćh. Doch selbst mit zwei gesunden Armen hätte sie diesen nicht beherrscht. Fluchend zückte sie ihr goldenes Schwert und fuchtelte damit in Richtung der vier Höhlenwichte tiefer in der Mine. Diese sollten bloß nicht näherkommen.

Ein Höhlenwicht saß frech auf Barz‘ Brust, während ein anderer ihm nach den Augen klaubte. Barz wehrte sich vehement, doch ohne großen Erfolg. Die Wichte waren überraschend kräftig für ihr Aussehen.

Aćh humpelte näher und zog ihr goldenes Schwert dem einen Höhlenwicht über den Schädel. Dieser klappte zusammen, rollte von Barz runter und ein schwaches magisches Glitzern breitete sich über seinen Körper aus.

Ehe Aćh sich dem zweiten Höhlenwicht auf Barz zuwenden konnte, kündigte ein Schlurfen und wirres Brabbeln hinter ihr von der Ankunft der restlichen Höhlenwichte. Sie wandte sich ihnen zu und schwang ihr Schwert, trotz ihres protestierenden Arms.

Sie blickte Turr an. War er schon stark genug, wieder einen Angriff auszuführen?

Ehe sie Turr rufen konnte, wurde auch sie von kalten, feuchten Händen gepackt. Ein Höhlenwicht rang Aćh zu Boden. Sie schrie, sowohl wegen der Schmerzen als auch, um die Wichte zurückzuscheuchen. Mit aller Kraft versuchte sie, sich dem Griff zu entziehen. Erfolglos.

Ein einzelner Höhlenwicht schob sein grimmig grinsendes grün-graues Gesicht vor Aćhs und brabbelte etwas vor sich hin, das bei genauem Hinhören Wortfetzen sein könnten. Dann biss der Wicht ihr in die Nase.

Neben sich hörte sie Barz erneut verzweifelt aufschreien.

Sie waren geliefert.

Da erklang eine tonlose aber nicht unfreundliche Stimme in einem fremden Akzent: „Das Besetzen unserer Minen ist in diesen Landen nicht gern gesehen. Und nächtliche Überfälle erst recht nicht. Ich muss euch bitten, euch unverzüglich zu ergeben, Unruhestifter!“

Ein grünlicher Blitz zuckte durch den dunklen Stollen. Aćhs Gesichtsfeld wurde von blendend hellem Grün geblendet. Ein Hitzeschwall rauschte über sie hinweg. Der Höhlenwicht kreischte auf und ließ von ihrer Nase ab. Als ihre Sicht langsam wieder zurückkehrte, erkannte Aćh die Silhouette eines Menschen vor dem Höhleneingang. In seiner Hand glühte etwas Grünliches, von dem immer wieder Strahlen in Richtung der ins Dunkel der Mine zurückweichenden Höhlenwichte ausschlugen. Nur der von Aćh ausgeknockte Höhlenwicht blieb liegen, sowie der über Barz zappelnde, der sich immer noch labernd an sein Opfer klammerte.

Grünes Feuer.

Magisches Feuer.

Ein Hexer.

Der Hexer griff an seinen Gürtel, zückte ein Säcklein, öffnete es und streute ein hellblaues Pulver daraus über den strampelnden Höhlenwicht.

Der Höhlenwicht wankte und stöhnte, rollte von Barz runter und wandelte seine Gestalt, während magisches grünes Feuer rund um ihn herum aus dem Boden brach und sich um seine Gliedmaßen wand. Als das Feuer verschwand, lag da nicht mehr ein Höhlenwicht, sondern eine hässliche kleine Schuppenkreatur mit zwei Köpfen und Flossen anstelle von Gliedmaßen. Sie erinnerte an einen Sumpffisch. Und offenbar brauchte sie wie ein Fisch Wasser zum Atmen. Der verwandelte Höhlenwicht japste einige Male vergeblich nach Luft. Dann wurde sein Körper totenstill.

Barz hörte auch auf zu strampeln und hielt sich stöhnend seinen Magen. Zugleich blickte er jedoch staunend zum Hexer und dessen blauen Pulversäckchen hoch.

„In der Steppe Tulgors wächst ein besonderes Kraut, welches im Licht der aufgehenden Sonne in hellem Blau erstrahlt. Dessen getrocknete Blüten können zu einem Pulver zermahlen werden, welches neuen Mut verleiht und Feinde verwandeln kann“, erklärte der Hexer mit einer gelassenen Selbstverständlichkeit. Als wäre dies in der aktuellen Lage die wichtigste Information.

„Doch müssen wir Acht geben. Der Einsatz des Pulvers kann auch dafür sorgen, dass mehr Kreaturen auftauchen. Hexerei ist nun mal nicht komplett kontrollierbar, und jeder Einsatz kann ungewollte Konsequenzen zur Folge haben.“

Dann blickte der Hexer zum kleinen Takuri und meinte: „Nun, der wäre natürlich noch praktischer. Aus der Asche von Feuertakuri lässt sich ein Pulver erstellen, welches Kreaturen völlig verschwinden lassen kann. Zumindest eine Zeit lang.“

Barz blickte noch interessierter zu ihm hoch, auch wenn er wohl nur die Wörter „Takuri“ und natürlich „Pulver“ verstanden hatte.

Der Hexer blickte hinter sich und rief laut: „He, Leute, kommt euch ansehen, was ich gefunden habe.“

Rufe von anderen Menschen außerhalb der Mine wurden laut.

Dann streckte der Hexer Aćh und Barz seine Hände entgegen und half ihnen auf.

„Los, nichts wie ins Freie mit uns. Dieser Gegend wimmelt ohnehin nur so von Wichten. Wir haben schon vor Wochen den Stollen hier aufgegeben. Wir graben jetzt weiter westlich.“

„Danke. Vielen Dank, o Fremder“, war alles, was die beiden als Antwort herausbrachten.

„Ich bin doch kein Fremder“, grinste der Fremde schwach, „Mein Name ist Haamun und ich bin meines Zeichens Minenarbeiter in den hiesigen Mera-Minen.“

„Aćh, Takuri-Hüterin vom Nestbaum.“

„Letzteres hätte ich mir doch fast denken können, wenn ich mir den Takuri so ansehe.“

„Das ergibt natürlich Sinn! Nun, was tut Ihr hier, ‚Haamun‘? ‚Licht des Morgens‘, hä? Da können wir doch von Glück reden, dass Ihr nicht erst zum Sonnenaufgang erschienen seid, sonst wären wir beide wohl Geschichte.“

Haamun gluckste leise: „Das war kein Glück und auch kein Zufall. Zufälle sind sehr unwahrscheinlich. Nein, wir haben euer Lagerfeuer von weiter unten aus gesehen und gedacht, dass das Wichte anlocken könnte. Einfach so in einem Minengang zu nächtigen, ohne auf Gefahren zu achten, ist schon fahrlässig.“

Sein tadelnder Ton verschwand, als Haamun Aćhs gebrochenen Arm sah.

„Was suchtet ihr überhaupt hier? Alles in Ordnung mit euch?“

„Es könnte schlimmer sein“, meinte Aćh, „Wir suchten diesen Takuri, und wollen nun nur noch zurück an den Nestbaum. Das hier ist übrigens Barz, ein Fremder aus einem weit entfernten Land. Er spricht unsere Sprache kaum.“

Haamun nickte Barz höflich zu. Dieser starrte weiterhin fasziniert auf das Pulversäcklein an Haamuns Gürtel.

Während Haamun sie ins Freie geleitete, stießen weitere Minenarbeiter zu ihnen, die wissen wollten, was diese hirnverbrannten Reisenden denn in dieser Mine zu suchen hatten. Als der Suchtrupp vollständig war, begaben sie sich alle gemeinsam zum Lager der Minenarbeiter weiter westlich. Dort wurden Aćh und Barz verpflegt und verarztet, ehe sich alle Arbeiter um ein Lagerfeuer versammelten. Der Morgen war bald da, und manchmal war kein Schlaf bekanntlich besser als gar kein Schlaf.

Haamun entzündete das Lagerfeuer mit magischen Mitteln, doch brauchte er einige Versuche. Immer wieder schnippte er seine Hand gegen die Scheite und murmelte etwas vor sich hin, bis endlich ein grüner Funke übersprang. Erschöpfte sich seine Kraft mit der Zeit?

Barz rückte näher an Haamun und fragte interessiert: „Alle Minenarbeiter. Magie?“

Allgemeines Gelächter von den Arbeitern war zu vernehmen.

„Ich wünschte, es wäre so!“, rief eine Arbeiterin, „Stell dir vor, wir alle könnten derartig Kräfte führen. Aber unser Haamun hier ist etwas Besonderes. Seine Fähigkeiten sind so vielseitig wie unberechenbar. Und das ist nicht alles. Aus einem fernen fremden Land will er kommen, ja, gar den Kuolema allein überquert haben! Erzähl ihnen davon, Haamun!“

Haamun schüttelte seinen Kopf und meinte leise: „Das ist eine Geschichte für ein andermal. Ich denke nicht gerne daran zurück. Damals war ich noch nicht Haamun. Und ihr glaubt mir doch ohnehin nicht.“

„Vielleicht glauben wir dir schon“, warf Aćh ein, „Wir haben kürzlich einiges erlebt, was aus Legenden kommen könnte. Eine Eis-Dämonin wollte uns rekrutieren und Sturmgeist hat unser Hängeschiff angegriffen.“

Haamun grinste schief: „Ah, mit diesen Geistern kann ich auch nichts anfangen. Sie mögen meine Hexerei nicht. Glaubt mir, wegen ihres langen Lebens liegt ihnen Vergebung sehr lange fern.“

„Barz hier will auch aus einem weit entfernten Land stammen. Und wir haben soeben den Kuolema erklommen und das ewige Eis hinter dem Felsentor mit eigenen Augen gesehen. Wir würden dir vermutlich einige verrückte Geschichten glauben.“

„Ihr habt das Felsentor gesehen?“, fragte Haamun erstaunt, „Ich spürte damals bei meiner Überquerung des ewigen Eises einen dunklen Dämon, der mich aus seinem Versteck beobachtete. Und ich spürte mächtige Magie über dem Felsentor liegen, als ich daran vorbeischritt. Es schien irgendetwas dahinter einzusperren. Und die Wolken, die das ewige Eis stets in Schatten hüllen, werden von einer Magie festgehalten, an der die Spuren der Drachen hing. Ich war froh darüber, das ewige Eis hinter mir gelassen zu haben. Magische Flammen an meiner Seite hin oder her, diese Schluchten jagen mir einen Schauer ein.“

Barz murmelte etwas Unverständliches vor sich hin. Dann, als wäre ihm plötzlich ein Licht aufgegangen, wandte er sich erneut an Haamun: „Du sagen was Land du kommen?”

Haamuns vom grünlichen Licht gruselig beleuchtetes Grinsen erstarb, als der Hexer tonlos flüsterte: „Ich stamme aus Andor.”

Barz klatschte triumphierend in seine Hände.

Aćh zog ihre Augenbraue hoch.

Turr gurrte.\bigskip

***\bigskip

Der Mond stand hoch am Himmel und schien sein silbriges Licht über Silberland. Im Norden der Insel heulten Nordskrale in die Nacht hinein, doch Iril war so stark auf ihre Arbeit fokussiert, dass sie die Rufe nicht wahrnahm. Die Runenschülerin saß vor dem Eingang zur Silbermine und blickte durch eine vor ihr linkes Auge geklemmte Lupe auf ihr Werk. Sorgfältig ritzte sie feinste Runenfolgen auf den Rand des silbernen Rundschilds in ihrem Schoß. Hin und wieder blickte sie vom Schild auf, um ihre Zeichnungen mit einem der dutzend Notizblätter abzugleichen, welche sie um sich herum verteilt hatte.

Jemand klopfte ihr auf die Schulter.

„Iril, komm rein. Heute wirst du nicht mehr fertig. Kreatok hat seine Schilde auch nicht in einer Nacht erschaffen. Selbst die besten Meister brauchen ihren Schlaf.“

Es war ihre alte Runenmeisterin. Jede noch so kleine Fläche ihrer Haut war mit einer Runentätowierung überzogen. Ihr kahl rasierter Schädel glänzte im Licht der Nacht in allen Regenbogenfarben.

„Noch nicht“, protestierte Iril, „Nur noch den dritten Runenreif, dann sind die Gravierungen abgeschlossen.“

„Wie du meinst“, grinste die Meisterin, „Lass mich dir wenigstens einen heißen Cocoa bringen.“

Während sie sich von Iril entfernte, spielte sie gedankenverloren mit dem grünlich schimmernden Hammer an ihrem Gürtel.






\newpage
\section{Epiloge}


„Hey, Ijs, kommt dein älterer Bruder nicht mit?“

„Eforas? Ne, der hat Schiss. Mein Vater hat ihm zu viele Gruselgeschichten über den Kuolema erzählt.“

„Der olle Saro soll mal nicht so tun. Wie steht es um dich?“

„Ich?! Ich habe mich noch nie vor einer Herausforderung gefürchtet! Außerdem, hast du nicht von Aćh und Barz gehört?“

„Von wem?“

„Die beiden haben sich hoch ins Gebirge gewagt und sind völlig unversehrt, ja, gar mit einem Takuri an ihrer Seite, zurückgekommen.“

„Diese Hüterin und dieser Fremde, der sich bei den Bergleuten verdingt? Ja, natürlich habe ich von ihnen gehört.“

„‚Unversehrt‘? Hatte nicht einer von ihnen seinen Arm verloren?“

„Gebrochen, nicht verloren. Das macht einen Riesenunterschied.“

„Und, glaubt ihr ihre Geschichte? Dass sie wirklich Eisdämonen getroffen haben?“

„Ja, logo. Die sind ehrenhafte Helden, die lügen nicht. Die beiden waren auf der Fahrt zwischen Thelot und Agarb dabei, als unser Hängeschiff vom Sturmgeist angegriffen worden. Haben uns gerettet. Heldenhaft, sage ich euch.“

„Na bitte! Dann ist es ja definitiv möglich, bis ins Gebirge vorzudringen und zurückzukommen.“

„Aber nur weil die beiden es schaffen, heißt das nicht, dass wir das auch können.“

„Ijs, kriegst du etwa kalte Füße?“

„Niemals.“\bigskip







Drei Tage und eine Lawine später\bigskip



Ijs schlidderte ein weiteres Mal auf der riesigen Eisfläche aus. Diesmal blieb er auf dem ewigen Eis liegen. Er hatte einfach nicht mehr die Kraft dazu, auch nur einen seiner gefühlslosen Finger zu heben. Wirre Gedanken und Erinnerungen schwirrten durch seinen unterkühlten Kopf. Bilder des Todes, die kein fühlendes Wesen je sehen müssen sollte. Dazwischen hallten Gesprächsfetzen aus einer gefühlt Jahre zurückliegenden Vergangenheit durch seinen Kopf. Sein Vater Saro, der ihn mit gerunzelter Stirn und Furcht in den Augen vor dem Kuolema warnte. Sein älterer Bruder Eforas, der mit gesenktem Blick meinte, dass er auf Ijs‘ wagemutigen Ausflug lieber nicht mitkommen würde.

Kaltes, glattes, ewiges Eis presste gegen Ijs‘ Backe, die so kurz vor dem Erfrieren war, dass sie nicht einmal mehr kribbelte. Ijs spürte keinen Hunger mehr, und keinen Schmerz, nur noch eine wohlige Wärme, die sich in seinem Körper ausbreitete. Er war der letzte Überlebende, da war er sich sicher. Er hatte seine Freunde in den Tod geführt. Da war es doch nur fair, dass er ihnen nun in den Tod folgte. Ijs hatte gehört, dass man vor seinem Tod sein Leben noch einmal vorbeiziehen sah, doch nichts dergleichen geschah hier. Es wurde einfach nur alles dunkler, und immer dunkler.

„Junge!“, erklang eine eisige Stimme, die Ijs nicht näher einordnen konnte. War sie eine Illusion? Realität?

Ein eiskalter Griff packte Ijs und drehte ihn unsanft auf den Rücken. Weiße Schemen schwirrten in seinem Gesichtsfeld. Dann, plötzlich, erkannte er verschwommen ein runzliges schneeweißes Gesicht mit einem langen verfilzten Bart, das sich auf ihn niederbeugte. Es war kein menschliches Gesicht. Die stahlblauen Augen zeigten keine Emotionen, und die abgebrochenen Überreste eines Geweihs ragten links und rechts aus dem Schädel seines Gegenübers hervor.

„Junge!“, wiederholte der Eisdämon mit rasselnder Stimme, „Du sollst hier nicht sterben. Das wäre eine Verschwendung. Du kannst Tari dienen. Dein Körper ist noch frisch. Ganze fünfhundert Jahre lang kannst du noch Tari dienen. Erst danach sollst du diese Welt verlassen, und zwar, indem du einen Nachfolger wählst. Indem du all deine Kraft an einen Nachkommen übergibst. So, wie ich es hier mit dir tue. Ich, Beriandur, der hiermit dein Vater werde.“

Ijs verstand die Worte seines Gegenübers nicht mehr, aber das musste er auch nicht. Das würde er später tun.

Beriandur der Eisdämon grabschte nach Ijs‘ Hand und führte dessen klamme Finger zu etwas Scharfem, Kantigem, Eiskalten. Wie Ijs‘ Hand die Eiskristallkette streifte, so begann sein Geist, sich zu klären. All die Furcht und das Trauma fielen von ihm ab. Tiefe Ruhe erfüllte ihn ebenso wie fremde Eindrücke und Erinnerungen, die nicht die seinen waren.

„Ijs heißt du also. Willkommen, mein Sohn Ijsdur, im ewigen Eis“, erklang erneut die klirrende Stimme des Eisdämons. Ijs‘ Blick fokussierte sich. Nun konnte er wieder den wolkenverhangenen Himmel über sich erkennen, und die Schneeflocken, die ihm entgegengewirbelt wurden. Und den Eisdämon, der sich über ihn beugte und ihn mit ausdruckslosem Blick betrachtete.

Urplötzlich ergriff der Eisdämon die Eiskristallkette, die um seinen eigenen Hals hing, und riss sie gewaltsam von seinem schneeigen Körper. Er krümmte sich und ächzte, während sich von den zurückgelassenen Einbuchtungen der Eiskristalle in seinem Hals Splitter und Spalte in alle Richtungen seinen Körper entlang ausbreiteten. Ijs wollte zurückweichen, doch sein unterkühlter Körper weigerte sich zu gehorchen.

Erleichterung machte sich auf dem Gesicht des Eisdämons breit.

„Ich bin alt, mein Sohn. Mein Geist ist nicht mehr klar. Ich fürchtete schon, dass ich keinen würdigen Nachfolger mehr finden würde, ehe dieser Körper vergehe. Doch dem war nicht so. Nun kann ich in Frieden diese Welt verlassen.“

Mit diesen Worten riss der bärtige Eisdämon Ijs‘ Mantel auf und presste ihm die Eiskristallkette aufs Schlüsselbein. Ijs hörte etwas knacken und knirschen, während sich eine Taubheit von seinem Hals aus über seinen ganzen Körper ausbreitete.

„Willkommen, Ijsdur, zu Deinem zukünftigen Dasein in den Diensten Taris.“

Das waren die letzten Worte des Eisdämons. Er trat einige Schritte zurück, stolperte dann und verschmolz mit der Eisfläche, versank buchstäblich darin.

Ijsdur versuchte vergeblich, die fremden Eindrücke und Erinnerungen in seinem Geist zu sortieren und zu vorstehen. Dann, langsam, gehorchte ihm sein Körper wieder. Er hob seine zitternden Hände, doch waren diese schneeweiß geworden, als wäre er schon längst tot.

Langsam erhob er sich. Es schien, als ob er dem ewigen Eis des Kuolema-Gebirges entwuchs. Er schaute auf die Gestalt, die vor ihm lag. Die Gestalt seines "Vaters", des Eisdämons Beriandur, die mehr und mehr vor seinen Augen verschwamm und mit der endlosen Eisfläche eins wurde, bis nur noch einige wenige Eiskristalle übrig blieben.

Als er an sein eigenes Spiegelbild erblickte, waren an seinen Schläfen kleine Geweihe gewachsen. Und seine Ohren hatten sich zugespitzt.

Doch fürchtete er sich nicht.

Denn Ijsdur verstand nun, warum er hier war.

Dies war nun seine Heimat.

Und es war gut so.\bigskip





\az{Jahr 65}



Zwei Jahre später\bigskip



Ein Kopf, der den Unterirdischen Krieg mit eigenen, blutroten Augen miterlebt hatte, sank tief in den blutgetränken Boden Andors ein. Er schmeckte Matsch, Dreck und Blut. Ein letztes Ächzen drang aus dem riesigen Maul.

Damit schied Tarok, der letzte Drache, für immer aus dem Leben. Seine Flamme des Zorns erlosch.

Südlich davon, tief, tief unter den Bergen des Grauen Gebirges, wogte in einer gewaltigen Höhle ein Geflecht aus Düsternis umher. In der Mitte ragte ein riesiger Baum empor. Schimmernde Edelsteine ragten hier und da aus seiner längst versteinerten Oberfläche. Doch klar erkennbar unter dem durchscheinenden hellen Gestein floss in seinen dicken Adern stetig rotes Blut durch den Baum. Schon seit Jahrtausenden, seit der Erschaffung durch die uralten Drachen, hatte der Baum des Blutes hier in Krahal gestanden, den Drachen Leben gespendet und ihnen Kraft und Energie verliehen.

Nun stockte der Herzschlag des Baumes. Das rote Blut gerann. Ein Glimmen lief über die Rinde des Baums und verwandelte helles, lebendiges Gestein in tote, schwarze Asche. Der Baum verödete innert Minuten. Die Schreie tausender Drachenseelen erklangen ein letztes Mal und verstummten dann für immer.

Einen Augenblick lang, regte sich nichts mehr, als wäre die Zeit selbst in Krahal stehen geblieben.

Dann, urplötzlich, wurde Krahal zutiefst erschüttert. Das Geflecht der Düsternis wirbelte wild umher und zerriss die erstarrten Überreste des blutigen Baumes. Krahal verschlang sich selbst. Die gewaltige Höhle stürzte in sich zusammen.

Das Splittern von Stein hallte in den Bergtälern wider und übertönte die Musik der Toten in den Schluchten des Grauen Gebirges. Felsen barsten und Lawinen krachten. Erschütterungen gingen vom Epizentrum aus. Sie zogen sich in Wellen durchs Graue Gebirge, jagten durch Fels und Stein, zwischen Schluchten und durch die unterirdischen Gänge der Schildzwerge.

An der Ruine der Winterburg stürzte eine weitere Mauer unter lautem Gekrache in sich zusammen.

Rhona und Grone, die Stammesältesten der Agren, fielen in eine stille Umarmung, während ihre Höhle um sie herum bebte.

Schildzwerge fielen auf ihre Knie und sandten Stoßgebete aus, während ihre Umgebung zitterte und die Tiefminen Flammen spuckten.

Der Mechaniker Kjall raste in seinem kleinen Museum umher und versuchte unter Einsatz eines seiner Tiefminen-Golems verzweifelt, seine Schätze – Fackeln von Cavern, schöne Schilde, ein zersplittertes Hadrisches Stundenglas – auf ihren angestammten Sockeln zu behalten.

Der Riesenkrake Irlok zuckte unruhig umher, während gewaltige Wellen ans Ufer des Geheimen Sees brandeten und Stalaktiten von der Decke ins Wasser klatschten.

Die von Krahal ausgehenden Beben wogten auch bis ins Kuolema-Gebirge.

Tulgorische Höhlenwichte sprangen beiseite, als ihre Gänge sich teilten.

Ein tiefer Spalt tat sich im ewigen Eis auf.

Siantari, die oberste Dämonin des ewigen Eises, breitete ihre Arme aus und rief eine wirbelnde Wolke aus Schnee und Eis zu sich. Der Wirbel hob sie in die Höhe und ließ sie einige Meter über der zitternden Eisfläche schweben, weg von den Vibrationen der Erde. Ihre eisblauen Augen starrten regungslos auf das Felsentor, das ihr nun schon seit 49 Dekaden den Zugang zur restlichen Welt verwehrte.

Das Felsentor erzitterte wie der gesamte Rest des Kuolema-Gebirges. Staub rieselte herab und Splitter bröckelten. Doch die Magie der Temm, die das stabile Tor vor Jahrtausenden vor dem ersten Dämon des ewigen Eises verschlossen hatten, ließ nicht nach.

Dann war es vorbei. Die Welt, das Gebirge, das ewige Eis, sie alle standen wieder ruhig da. Krahal war nicht mehr. Shan, die Schattenhexe, blickte verwirrt um sich, als sie die gewaltige Seelenkraft Krahals tief unter der Erde auf einmal nicht mehr fühlen konnte.

Doch das Felsentor vor dem ewigen Eis des Kuolema hatte standgehalten. Die Eisdämonen waren immer noch in der Schlucht dahinter gefangen.

„Fećht!“, fluchte Siantari. So schnell, wie der Ärger in ihr aufgeschwellt war, wurde er auch schon wieder von der allgegenwärtigen Ruhe des Eises überdeckt. Schon lange hatte sie nicht mehr so stark gehofft. Wer nicht hoffte, konnte auch nicht enttäuscht werden. Und dennoch war Siantari nun enttäuscht, dass das Felsentor dem Beben standgehalten hatte.

Dafür hatte, zunächst unbemerkt von Siantari, etwas Anderes dem Einsturz Krahals nicht standgehalten. Mochten die Temm vor Jahrtausenden ihre eigene Magiequelle für das Felsentor genutzt haben, so war die unnatürliche Wolkenkuppel, die sie über die Gipfel des Kuolema gelegt hatten, durch uralte Drachenmagie gespeist worden. Denn auch die Drachen der Urzeiten hatten ein Interesse daran gehabt, dass die Eisdämonen im Fahlen Gebirge und ihre tiefe Schlucht abgeschirmt vom Rest der Welt blieben.

Nun, wo Krahal eingestürzt und die magischen Ströme der Drachen versiegt waren, lichteten sich die Wolken am Gipfel des Kuolema langsam im beißenden Wind. Der Himmel über der ewigen Eisfläche war wieder zu sehen. Zum ersten Mal in beinahe fünf Jahrhunderten erblickte Siantari wieder das Licht der Sonne. Ungewohnt prickelte es auf ihrer eisigen Haut.\bigskip







Es gab auch noch einen anderen, der verstand, warum sich die Wolken von den Gipfeln des Kuolema verzogen. Haamun, der Hexer aus Andor. Er hatte bei seiner Überquerung des Gebirges die Magie der Temm gespürt, die das Felsentor für die Eisdämonen undurchdringlich machte. Und er hatte die Drachenmagie gespürt, die die Wolken auf den Gipfeln des Kuolema-Gebirges gehalten hatte.

Haamun trat vor seine Hütte und blickte zum Kuolema hinauf. Nun waren die schneebedeckten Gipfel der Berge sichtbar geworden. Der Himmel jenseits des Gebirges erglühte in rotem Licht. Da wusste Haamun, dass der letzte Drache besiegt worden war. Und wenn der Drache angegriffen hatte, so hieß das, dass der alte König tot war. Und dass es an der Zeit war, nach Hause zurückzukehren.

Haamun wurde von einem Hilferuf aus seinen Gedanken gerissen.

„Haamun!“, erklang die Stimme einer Minenarbeiterin, die aus einem Stollen zu rennen kam, „Haamun! Die Erde hat gebebt. Der Stollen hat nachgegeben. Seram hat eine Platzwunde am Kopf. Kannst du...“

Haamun hatte bereits eine Handvoll Heilkräuter aus der Steppe ergriffen und rannte los. In den Jahren, seit er hierhergekommen war, hatte er viele der seltenen und oft unbekannten Kräuter der Steppe gesammelt, Tränke gebraut und die verschiedensten Salben, Pulver und Gifte hergestellt. Wenn es einen Notfall gab, war er zur Stelle, um zu helfen. Selbst wenn den Geholfenen am Ende etwas Unvorhergesehenes geschah, wenn sie seit der Heilung grüne Haut hatten oder nur noch im Versmaß sprechen konnten... nun, solange Haamuns Hexerkunst ihr Leben rettete, war es das weitaus wert.

„Der Stollen ist nicht zusammengestürzt“, behauptete Seram trotzig, während Haamun seinen Kopf verarztete, „Im Gegenteil! Aus dem Felsen hat sich ein alter Stollen wieder befreit. Während ihr alle euch vor niederrieselnden Kieseln versteckt habt, habe ich gesehen, wie der Berg sich mir gegenüber öffnete! Einen uralten Gang mit Wänden voller Runen und ähnlicher Kritzeleien sah ich. Das Beben hat einen der uralten Gänge der Temm freigelegt!“

Haamun hielt inne. Er hatte schon seit langem gespürt, dass die Stollen unterhalb des Kuolema-Gebirges sein Weg zurück nach Andor sein könnten. Nun, mit dem Tod des alten Königs und des Drachen, begann eine neue Ära für Andor. Eine, an der Meres wieder teilhaben konnte. Und ausgerechnet jetzt wurde ein uralter Temm-Stollen freigelegt? Alle Spielsteine fielen an ihre Stelle. Dieser Gang würde sie nach Andor führen.

Es war an der Zeit. Nun musste er rasch handeln. Mitreisende auftreiben für die Rückkehr in seine Heimat. Haamun selbst, seine Tränke und seine Geschichten waren beliebt bei einigen Bauern nahe des Gebirges, insbesondere den jüngeren. Vielleicht würden sich einige davon ihm anschließen. Und in den nahe gelegenen Dörfern konnte er sicherlich einige Händler, Baumeister und Kartographen zusammentreiben, deren Neugier stärker war als die Furcht vor den Geistern der Berge. Mit einer ganzen Gesellschaft würde er in Andor aufkreuzen und in den unruhigen Zeiten nach dem Tod von König und Drache aushelfen. Sie würden ihm verzeihen und den längst vergangenen Zwist beilegen. Er konnte wieder Meres sein.\bigskip







Aćh war gerade unterwegs in der Metropole Agarb, als sie ein Falke mit einer Nachricht von Barz erreichte.\bigskip



„Hochverehrte Hüterin Aćh,



Ein Beben hat am Ende eines Stollens der Bergarbeiter eine Verbindung zu einem uralten Gang der Temm freigelegt. Haamun, der Hexer, ist sich absolut sicher, dass er uns nach Andor führen kann. Er trommelt eine Gruppe Reisender zusammen, die sich für das Königreich auf der anderen Seite des Kuolema interessieren. Und der Gang scheint gerade groß genug für Sabri zu sein. Ich habe vor, mich ihm anzuschließen. Bist du dabei? Wir brechen in wenigen Tagen auf.



Beste Grüße

Barz“\bigskip



Wie weit Barz‘ Sprachkenntnisse doch in den letzten Jahren gewachsen waren!

Aćh schluckte schwer und ließ ihre Optionen Revue passieren. Endlich war es soweit. Seit dem Gespräch mit dem Hüter der Zeit damals hatte Barz gewusst, dass es einen Weg für ihn nach Andor und zurück zu seiner Familie geben würde. Und in den Jahren seit damals, in denen sie und Barz sich näher gekommen waren, war mehrmals die Frage aufgekommen, ob sie ihn begleiten würde.

Sie mochte ihre Arbeit am Nestbaum der Takuri, und sie mochte ihre Freunde und Bekannte dort. Doch sie mochte auch Barz, und sie mochte Abenteuer. Sturmgeister und Eisdämonen, auf so etwas stieß man nicht, wenn man sein Leben lang am selben Baum verbrachte. Andor, das fremde Land, konnte bestimmt von diplomatischem Kontakt mit Tulgor profitieren, und als Tochter einer Diplomatin war Aćh für eine solche Rolle vermutlich besser geeignet als ein Hexer und seine zusammengewürfelte Reisegruppe.

Noch während Aćh ihre Gedanken sortierte, klopfte es an ihrer Tür. Als sie sie öffnete, blickte sie überrascht in das Gesicht ihrer Mutter... und in die verschmitzten Augen des Hüters der Zeit, der auf ihrem Rücken saß. Seinen Turban hatte er gegen eine braune Kutte eingetauscht.

„Na, schon reisefertig gemacht?“, krächzte der Temm grinsend, „Ich schließe mich dir gleich an, wenn das für dich in Ordnung ist. Schließlich will ich auch mit nach Andor!“\bigskip







Gemeinsam hatte Aćh mit dem Takuri das Land durchquert und vor wenigen Tagen das Gebirge erreicht. Die Sonne war hinter den schneebedeckten Bergspitzen untergegangen, und ein roter Mond stand am Himmel. Nun sah der Takuri Aćh noch einmal an, und in einem kurzen Augenblick rotglühenden Feuers beendete er sein Leben, so wie er es schon so oft beendet hatte. Als kleines Küken watschelte er in Aćhs Arme und kuschelte sich an sie.

Der Hüter der Zeit betrachtete Turr mit zusammengekniffenen Augen:

„Ist alles in Ordnung mit dem Takuri? Er ist nun schon zum zweiten Mal in zwei Wochen verglüht.“

Aćh schüttelte ihren Kopf.

„Nein, irgendetwas ist besonders an ihm. Seit seinem Verschwinden und seiner Zeit im ewigen Eis altert er viel zu schnell. Manchmal durchlebt er seinen Zyklus in wenigen Tagen!“

„Vielleicht finden wir ja in Andor etwas, das ihm helfen kann“, meinte Barz optimistisch, und setzte sich neben die beiden ans Lagerfeuer, „Und daran zu leiden scheint er auch nicht.“

Mit einem nachdenklichen Blick setzte Barz nach: „Ist das eigentlich nach jedem Zyklus immer noch derselbe Takuri?“

Aćh grinste. Sie hatte sich eine solche Frage schon oft gestellt. War ein Takuri nach seiner Regeneration noch derselbe Takuri? Wie sollte man sie nur beantworten? Der neue Takuri würde nie identisch aussehen wie der alte, aber dennoch würde der neue Takuri dem alten erheblich ähnlicher sehen als allen anderen. Vielleicht mit einer anderen Federanordnung oder sonstigen Kleinigkeiten. Auch wenn das Küken noch auf den alten Namen reagieren würde, stellte sich die Frage, ob esdieselben Erinnerungen wie der frühere Takuri trug. Aćh hatte oft genug mit einem Töpfchentraining von vorne anfangen müssen. Gewisse Eigenheiten, bestimmte Verhaltensmuster, blieben dennoch nach jeder Regeneration, soweit sie sagen konnte. Wobei ihre Einschätzung natürlich vielen Vorurteilen unterlag. Wie wahrheitsgetreu sie war, war schwer einzuschätzen.

Aćh beschloss, Barz eine Gegenfrage zu stellen: „Was würdest du sagen: Wenn du am Morgen aufwachst, bist du dann derselbe Mensch, der eingeschlafen ist?”

„Natürlich”, meint Barz, „Wenn wir uns nach jedem Schlaf als andere Person sehen würden, gäbe es ja kaum einen Grund, uns groß um die nächsten Tage zu sorgen. Wir würden doch einfach unseren einzelnen Tag mit kurzweiligen Genüssen verschwenden und die Menschheit hätte sich gar nie erst zu einer Gesellschaft entwickelt.”

Das war nicht die Sorte Antwort, die Aćh erwartet hatte.

„Pfff. Da kann ich ja nicht einmal eine Analogie zur Regeneration des Takuri ziehen. Außerdem verzichten wir doch für unsere Nachkommen auf gewisse Genüsse, könnten wir das dann nicht auch hier tun? Und dass etwas ohne den Glauben an etwas nicht entstanden wäre, sagt ohnehin nichts darüber aus, wie wahr dieses etwas wirklich ist, oder etwa nicht?”

Barz legte den Kopf schief: „Da magst du Recht haben. Ich weiß, dass du diesen Takuri hier nach jeder Regeneration immer noch Turr nennst. Macht ihr das normalerweise immer so, oder gebt ihr den Älteren jeweils neue Namen, wenn sie sich regenerieren?”

„Wir sehen von Fall zu Fall, ob wir einen neuen Namen verleihen wollen. Oftmals ist es praktischer, einfach dieselben Namen zu verwenden. Manchmal ist ihre Persönlichkeit nach einem Zyklus aber einfach zu anders, und dann wählen manche Hüter neue Namen. Dann gibt es oft eine Namenszeremonie, wie wenn einer von uns den seinen wechselt.“

„Tulgori wechseln ihre Namen?”

„Ihr Barbaren etwa nicht?“

„Natürlich nicht!“

„Hast du denn in all deiner Zeit hier keine Namenszeremonie mitgekriegt? Heißt du als Kind etwa gleich wie als Jugendlicher? Wie als Volljähriger? Wie als alter Erwachsener?”

„Natürlich heiße ich immer gleich.”

„Aber ein Kind kann doch noch keinen Namen wählen. Das Kind kennt die Bedeutung ja gar nicht.”

„Das muss es auch nicht. Die Eltern wählen ja für das Kind.“

„Und was, wenn die Eltern sich nicht einig sind?“

„Dann diskutieren sie das halt aus.“

„Und was, wenn das Kind den Namen später nicht mag?“

„Jeder, den ich kenne, mag seinen Namen. Irgendwie.“

„Was, wenn nicht? Ich mochte meinen Geburtsnamen nicht.“

„Du wusstest ja auch, dass du ihn ändern kannst. Wir gewöhnten uns dran.“

„Na eben.“

„Dann weiß man ja nicht mal, von wem man redet, wenn Leute später immer anders heißen.“

„So oft ändert sich der Name ja auch wieder nicht. Und man gewöhnt sich daran.“

Hinter Barz grunzte es, als seine Echse Sabri ihn endlich erreichte (sie hatte sich in der letzten Stunde unglaublich gemächlich auf ihn zubewegt) und sich ächzend auf den Boden sinken ließ. In den letzten Jahren war sie beachtlich gewachsen, sodass sie ihm im Stehen nun schon bis zur Schulter reichte. Sie war über und über beladen mit verschiedenen Pulversäcklein. Barz‘ Vorräte mochten beim Kampf gegen den Sturmgeist erheblich reduziert worden sein, doch hatte er in der Steppe Tulgors und in zahlreichen Gesprächen mit dem Hexer Haamun seine Pulversammlung teils aufstocken (Mondbeeren für sein Bannpulver schienen beispielsweise buchstäblich überall zu wachsen) und teils gar alternative Pulvermischungen zusammenstellen können.

Haamun und sein Trupp waren gleich zwei Tage nach dem großen Beben, dem Verschwinden der Wolken über den Gipfeln und dem Freilegen des Temm-Stollens in der Mine verschwunden, um die Lage mit eigenen Augen auszukundschaften. Schon am ersten Abend waren sie mit glänzenden Augen und einer Handvoll wertvoller Edelsteine in seiner Tasche wieder beim Stolleneingang aufgetaucht und berichteten von einer „lohnenswerten Sackgasse“.

Nun, eine ganze Woche nach dem großen Beben, war die Kompanie schon zum dritten Mal an den Startpunkt zurückgekehrt. Ihre gute Laune hatte sich gelegt und Haamun verbrachte viel Zeit damit, mit einer Art Wünschelrute in seinen Händen im Eingangsbereich verschiedener Stollenabzweigungen umherzulaufen. Die unterirdischen Gänge der Temm hatten sich als wahres Labyrinth herausgestellt, welches nur mit der nötigen Magiekunde navigiert werden konnte. Und Haamun rang mit seinem Stolz.

Eforas, einer seiner engsten Vertrauten, unterhielt sich soeben etwas abseits des Lagerfeuers mit ihm. Man konnte von außen nicht viel vernehmen, doch Aćh glaubte eindeutig, ein „Du sagtest, du würdest den Weg durch den Berg kennen!“ von Eforas zu vernehmen, sowie eine zischende Antwort von Haamun.

Ein Raunen ging durch die Dutzend Händler, Baumeister und Kartographen, die ebenfalls in der Nähe saßen.

Aćh wisperte zum Hüter der Zeit: „Ich weiß, dass diese Temm-Gänge viele Verzweigungen besitzen, aber du bist ein Temm und kannst noch dazu die Zukunft sehen! Kannst nicht einfach du uns durch die Gänge führen?“

„Wer kann schon wissen, was genau ich alles tun kann?“, meinte der Hüter mit einem verlegenen Grinsen, „Aber das würde Haamun wohl nicht gefallen. Bald schon wird er von selbst den richtigen Einfall haben. Und ganz abgesehen davon mag ich die Gesellschaft hier.“

Aćh und Barz blickten sich an und grinsten. In Barz‘ Blick war aber auch eine lange unterdrückte Sehnsucht zu erkennen. Gedankenverloren griff er an seinen Hals und spielte mit einer goldenen Kette, die Aćh ihm vor einem Jahr geschenkt hatte. Barz‘ Ringkette war, soweit sie wussten, noch immer vor dem Felsentor der Eisdämonen eingeschneit.\bigskip







„Was... was ist das?“, stammelte Haamun leise.

„Es ist wunderschön“, murmelte Aćh.

Haamun hatte endlich – so hofften alle – die richtigen Abzweigungen gefunden. Schon seit einigen Tagen waren sie nun die Stollen unter dem Berg entlanggewandert, ohne sich in Sackgassen zu verirren. So langsam wünschten sich alle Reisenden, wieder das Licht der Sonne zu sehen und den Wind in den Haaren zu spüren. Doch der Anblick, der sich ihnen hier unter der Erde bot, war beinahe ein Ausgleich dafür.

Sie hatten eine Höhle unter der Erde erreicht, aus deren Wände dutzende Edel- und Mera-Steine ragten. Ein leises Summen erfüllte den Raum, und wie überall sonst waren auch hier zahlreiche Runen am Boden zu sehen. Turr entfaltete seine Flügel und erhob sich zum ersten Mal in einigen Tagen zu einem kurzen Rundflug in der geräumigen Höhle. Er stieß einen Freudenschrei aus. Seine Schwanzfedern glommen glücklich.

Barz klopfte der dem Trupp langsam nachtrottenden Sabri auf den Rücken. Sie folgte dem Angebot bereitwillig und ließ sich zu Boden plumpsen. Schon bald ertönte ein Schnarchen aus ihrem breiten Mund.

Der Hüter der Zeit kniff seine Augen zusammen und runzelte seine Stirn. Nach einiger Zeit öffnete er sie wieder und sprach:

„Tut mir leid, werte Reisende. Ich habe keine Ahnung, um was für eine magische Höhle es sich hier handelt. Ich werde dies mein ganzes Leben lang nicht in Erfahrung bringen. Und ihr demnach wohl ebenfalls nicht.“

Murmeln war unter den Reisenden zu hören. Dann sprach Haamun leise und monton:

„Es wäre schon fast schade, dieses schöne Bild zu zerstören und die Mera-Steine mitzunehmen.“

Eforas nickte: „Fast.“

Energisch trat er nach vorne an die Wand und untersuchte einen Mera-Stein, der darin verankert war. Wie üblich sah es aus, als wäre der farbige Stein mit dem umliegenden Felsen verschmolzen.

„Sitzt fest wie angeleimt.“

Eforas zückte einen Pickel und ließ ihn auf den Felsen rund um den Stein niederfahren. Der Pickel zerbrach mit einem hellen Klingen. Es erinnerte Aćh an daran, wie die Werkzeuge der Minenarbeiter auch am Nestbaum der Takuri zerschellten.

„Hier wollte jemand nicht, dass diese Schätze entwendet werden“, meinte Eforas schicksalsergeben.

„Schaut mal her!“, rief Barz da. Er hatte in den Rucksack auf Sabris Rücken gelangt, den die Reisenden mit allerlei Schätzen und Gaben beladen hatten, und ein rautenförmiges Felsstück herausgezogen. Aus dem Felsstück ragte ein schwach grünlich leuchtender Mera-Stein. In dieser Form wurden die Steine üblicherweise weiter versandt, um weiter Inlands komplett aus dem Felsen geschält zu werden. Ein komplexer Prozess, der den Minenarbeitern abgenommen wurde.

Barz hob den grünen Mera-Stein in die Höhe und brachte ihn in die Nähe eines in die Höhlenwand eingelassenen grünen Mera-Steins. Je näher Barz trat, desto heller leuchteten die beiden magischen Steine, und ein immer lauteres Summen war zu vernehmen. Während Barz die Distanz zwischen den Mera-Steinen verringerte, wurde das Summen zudem immer dumpfer und tiefer, bis die Reisenden es nicht mehr hören konnten, doch die Vibrationen immer noch in ihrem Körper zu fühlen glaubten.

Dann machte es „klick“, als die beiden Mera-Steine einander berührten. Das helle Leuchten erlosch – und der Mera-Stein aus der Wand rutschte herunter und kullerte über den Boden, als hätte ihn nie etwas in der Wand gehalten.

Da waren die Reisenden aber außer Rand und Band. Innert Minuten hatten sie alle Mera-Steine mithilfe anderer Mera-Steine aus der Wand gelöst und eingepackt. Nachdem jemand feststellte, dass sich die Edelsteine ebenfalls so lösen ließen, dauerte es sogar noch kürzer, bis auch diese allesamt ordentlich in einem Sack versammelt waren. Außer die ein, zwei Steinchen, die vielleicht jemand lieber in seine eigene Hosentasche hatte wandern lassen. Sabri wurde eine weitere Tragetasche voller Mera-Steine übergehängt. Die Echse ächzte, wehrte sich aber nicht.

In vielen Augen blitzte Begeisterung über den unerwarteten Fund. Der Hüter der Zeit blickte hingegen nur nachdenklich drein. Und auch Haamun fiel nicht in Freudenschreie ein. Ausgerechnet Meres machte sich Gedanken darüber, ob das Lösen der Mera-Steine ungewollte Konsequenzen haben könnte.\bigskip







Siantari tigerte hinter dem Felsentor über das ewige Eis. Hin und her, hin und her. Ihre Augen waren starr auf den Himmel gerichtet. Die Wolken hatten sich verzogen, und die Sterne zogen ihren Blick magisch an. Die klirrende Kälte, die Siantari unerbittlich entgegenschlug, konnte ihr nichts anhaben. Sie war die frostigen Tage und Nächte im Kuolema-Gebirge gewohnt.

So stand sie da und blickte in den weiten Kosmos, weiß auf Blau. Ein Kosmos, in dem sie und ihre Eisdämonen doch nur ein so kleiner, unwichtiger Teil waren. Noch. Denn eines Tages würde das ewige Eis über alles unter dem Himmelszelt gewachsen sein. Und Siantari würde dabei geholfen haben.

Sie verspürte Hoffnung. Ja, die Sterne erfüllten sie mit Hoffnung.

Die uralte Drachenmagie, die die Gipfel des Kuolema stets in Wolken verhüllt hatte, war verklungen. Alles verfiel mit der Zeit. Auch uralte Magien. Und so konnte es nicht mehr unendlich lange dauern, bis auch ihr Felsentor geöffnet wurde und sie oder einer ihrer Nachfolger das ewige Eis verlassen konnte.

Etwas regte sich in ihrem Hinterkopf. Eine Wahrnehmung, klein und fein, und doch so unglaublich relevant.

Sie fühlte, dass der Fremde zurückgekehrt war. Der Hexer mit den grünen Flammen, der das ewige Eis vor einigen Jahren, die für Siantari kaum mehr als ein Wimpernschlag her waren, vom Osten her überquert hatte. Doch überquerte er den Kuolema nicht. Er ging unter den Bergen hindurch, und er war nicht allein. Und sie trugen Mera-Steine bei sich, diese Narren!

Siantari spürte, wie sich etwas grundlegend änderte. Damals, als sie zur Eisdämonin geworden war, hatte sie auf einmal eine Erkenntnis ergriffen. Dass sie nie wieder durchs Felsentor gehen konnte. Dass sie nun eine Gefangene hinter dem Felsentor war, dazu verdammt, 500 Jahre lang das ewige Eis zu nähren, um am Ende einen neuen Eisdämon zu erschaffen und dann für immer im Eis zu versinken. Diese Erkenntnis, magisch auferlegt durch das Felsentor und alle anderen uralten Vorrichtungen der Temm, die die Eisdämonen von allen Seiten in diesen Schluchten einsperrten... noch vor wenigen Augenblicken war diese Erkenntnis Siantari noch bewusst gewesen, wie eine leise Last, die ihr seit Jahrhunderten auf den Schultern lag. Nun war sie einfach nicht mehr da. Sie fühlte sich so unbeschwert. Der Bann war verschwunden.

Misstrauisch schritt Siantari auf das Felsentor zu. Konnte es sein, dass das Tor geöffnet worden war? Es sah noch genau gleich aus wie zuvor. Doch fühlte es sich ganz anders an.

Triumphierend glitt Siantari nach vorne und marschierte durch das Felsentor. Keine uralte Magie hinderte sie daran. Die Ströme aus dem Innern des Bergs waren versiegt.

Der Fremde und seine Begleiter mussten die Quelle des Felsentors gefunden haben. Und ihre Mera-Steine hatten den Mechanismus ausgehebelt. Sie hatten das Felsentor geöffnet. Siantaris Felsentor war offen. Sie war frei!

Ein kurzes Lächeln huschte über Siantaris Lippen.

Und es kam noch besser. Die Reisenden tief unter dem Gebirge trugen weiterhin Mera-Steine mit sich. Siantari konnte die Steine und ihre Lage spüren, so wie sie den gesamten Berg spüren konnte. Schließlich war das ihr Berg. Das waren ihre Steine. Und so würde sie stets wissen, wohin diese Reisenden reisten. Sie konnte ihnen folgen. Sie würden ihr den Weg durch den Berg zeigen. Der Weg in ein fremdes Land. Sollte sie ihnen folgen?

Siantari blickte sich um. Links oder rechts? Wollte sie zurück nach Tulgor, in das Land ihrer Kindheit, oder lieber in den Osten, wohin diese Narren reisten und woher dieser mysteriöse Hexer gekommen war?

Leise Wehmut übermannte Siantari, als sie an ihre Heimat dachte. Ein überaus menschliches Gefühl, wie sie überrascht feststellte. Wollte sie dorthin zurückkehren? Sehen, was aus der Hütte ihrer Familie geworden war?

Nein, sprach sie zu sich. Tulgor bedeutete ihr nichts. Zeit, neue Wege zu beschreiten. Zeit, sich der Zukunft zuzuwenden. Zeit, in das Land im Osten aufzubrechen.

Siantari erhob sich und blickte ein letztes Mal über das ewige Eis des Kuolema. Hier und dort glaubte sie, ferne Silhouetten anderer Eisdämonen ausmachen zu können. Einsame, verwirrte Gestalten.

Siantari war es als erster Dämonin des ewigen Eises gelungen, Eiskristallketten zu fertigen, die ihre Gabe und ihre Mission an andere weitergeben können. Hin und wieder hatte sie aus verirrten Bergsteigern Durs und Doras geschaffen. Sie hatte sie einsetzen wollen, um aus dem Felsentor auszubrechen. Doch hatte die lange Zeit im ewigen Eis den menschlicheren Geistern der Durs und Doras nicht gutgetan. Und nun brauchte Siantari brauchte sie nicht mehr. Das ewige Eis konnte sie auch allein bis in die umliegenden Länder ausbreiten.

„Meine Kinder“, erhob sie ihre kalte Stimme, und das Echo ihrer Worte schallte über die riesige Eisfläche, „Ihr seid nun endlich frei, und frei sollt ihr nun sein. Ich entlasse euch aus meinen Diensten. Geht, wohin euch eure Schicksale ziehen. Verbreitet das ewige Eis. Wenn es so sein soll, werden wir uns wiedersehen.“

Keine Reaktion kam aus dem ewigen Eis. Siantari wusste, dass ihre Worte angekommen waren. Doch wie viele der Eisdämonen hier besaßen überhaupt noch genug Willenskraft, um sich zu irgendeiner Handlung aufzurappeln? Sie war allein. Aber das war in Ordnung. Sie war schon so lange allein gewesen. Sie wandte sich nach Osten und verließ das ewige Eis und das Fahle Gebirge. Zeit, alles Land in eine Eiswüste zu verwandeln.

\begin{center}
    Weiter geht es in \hypref{Runen im Schnee (2022)}.
\end{center}






