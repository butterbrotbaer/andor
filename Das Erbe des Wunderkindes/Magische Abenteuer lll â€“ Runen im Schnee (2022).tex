
\begin{chapterbox}
    \chapter{Runen im Schnee (2022)}
    \label{Runen im Schnee (2022)}
    \az{Jahr 65}

    \begin{center}
        Teil lll der Magischen Abenteuer
        
        Fortsetzung von \hypref{Der verschwundene Feuertakuri (2022)}
    \end{center}
    
    Eine Runenmeisterin aus Silberhall reist nach dem Tod ihrer Lehrmeisterin zurück in ihre ehemalige Heimat Cavern. Ein Eis-Dämon verlässt nach der Öffnung des Felsentors das ewige Eis. Eine Takuri-Hüterin und ein Steppennomade begleiten eine tulgorische Reisetruppe bei der Unterquerung des gefährlichen Kuolema-Gebirges. Und ein Herold, der unlängst den Tod seines Meisters zu betrauern hatte, findet in einem uralten Kult Verbündete.
\end{chapterbox}







\section{Epilog}

\az{Jahr 563}

Die Bewahrer vom Baum der Lieder schrieben das Jahr 563 nach andorischer Zeitrechnung.

Die Rietburg und die umliegenden Dörfer waren zur großen Rietstadt zusammengewachsen.

Die Ewige Kälte Hadrias war grünem Gras und sprießenden Blumen gewichen.

Das Graue Gebirge war einige Meter höher als zur Zeit der ersten Helden.

Der große See Ava hatte sich beinahe bis zum Hadrischen Meer ausgebreitet.

Die Rote Steppe Tulgors war größtenteils von der Wilden Wüste des Westens eingenommen worden. Nur ganz nahe am Kuolema-Gebirge wuchs das wilde Steppengras noch ungehindert.

Eine einsame Gestalt lief durch diese tulgorische Steppe, eine Spur aus Eis und Schnee hinter sich herziehend. Ein langer Bart fiel auf eine nackte Brust. Bis auf Stiefel, einen Rock und einen Gürtel mit einigen daran befestigten Artefakten war die Person unbekleidet. Ihre Kleidung war so blauweiß wie ihre Haut und ihr langes Haar. So blauweiß wie das Eis.

Dies war ein Eis-Dämon. Lange schon hatte man in dieser Gegend keinen mehr gesehen. Sein Name war Ijsdur. Schon einige Male war er gerufen worden. „Hilf uns, o Dämon des ewigen Eises. Komm herunter von deinem hohen Berg. Rette das Leben von diesen und jenen. Schenke uns Armen mehr Zeit in dieser unfairen Welt.“

Dabei besaß Ijsdur gar nicht die Fähigkeit, weitere lebensrettende Eiskristallketten zu schaffen. Er wusste nicht, wie er neue Durs und Doras erschaffen sollte, ohne selbst daran zugrunde zu gehen.

Diesmal war er dennoch gekommen. Denn es war an der Zeit. Seine fünf Jahrhunderte waren beinahe um.

Ijsdur besuchte ein Dorf am Fuße des Kuolema-Gebirges. Er fand die gesuchte Hütte, trat durch eine Tür und dann noch eine, und dann erblickte er sie. Eine gebrechliche, todkranke junge Tulgori, die in ihrem Bettchen lag und müde vor sich hin hustete. Ihre Familie saß ums Bett herum und schien nicht wirklich zu wissen, wie sie reagieren sollte. Manche gruselten sich vor dem Eis-Dämon. Ein Vater wich Ijsdurs Blick gekonnt aus. Ein anderer weinte ungehemmt. Ein dritter lächelte. Dieser wirkte gar ... hoffnungsvoll?

Sie waren nicht wichtig. Wichtig war nur die kleine Tulgori, die erneut schwach hustete. Nalle hieß sie. Schwach leuchtende Flecken überzogen ihre dunkle Haut. Die Druiden hatten ihr gesagt, dass sie kaum mehr als ein, zwei Wochen zu leben hätte. Der Blick aus Nalles Augen war wachsam und berechnend, als sie Ijsdur musterte.

„Danke“, sprach sie heiser.

„Danke mir lieber noch nicht“, meinte Ijsdur grimmig, ehe er sich eines Besseren besann und ein gezwungenes Lächeln aufsetze.

Nalle verabschiedete sich von allen Anwesenden. Noch mehr Tränen brachen aus. Ijsdur verzog keine Miene und wandte sich höflich ab. Er streckte seine Arme aus und ließ sich eine Decke darüberlegen, damit die Kleine auf dem Weg nicht allzu sehr frieren würde. Dann ergriff er Nalle mit beiden Armen. Sie war zu schwach, um allein zu laufen.

Langsam trug Ijsdur sie die Hänge des Kuolema-Gebirges hoch. Ihr Pfad schlängelte sich durch ein Meer aus Steinen den Berg hinauf.

In der Ferne sah Ijsdur einen brennenden Takuri über den Himmel fliegen, einen Flammenschweif hinter sich herziehend. Ijsdur drehte sich so, dass Nalle den Feuervogel sehen konnte.

„Hast du schon je einen Feuertakuri von nahe gesehen?“

Nalle schüttelte ihren Kopf und Schneeflocken von ihren Haaren. „Ich habe entfernte Familie unter den Hütern. Aber die trauen sich nur noch ohne Feuervögel in die Nähe des trockenen Steppengrases.“

„Hm. Ich habe schon einige gesehen. Sogar auf dem Weg hierher. Ich besuchte den Nestbaum der Feuertakuri in den westlichen Ausläufern des Kuolema-Gebirges, obwohl die mich dort nicht wirklich mögen. Beruht auch auf Gegenseitigkeit. Die Takuri sind keine Freunde des Eises. Und wir Eis-Dämonen sind keine Freunde von Feuervögeln. Einen gab es mal, den ich mochte. Turr. Nun, eigentlich gibt es ihn immer noch. Er trägt inzwischen einen neuen Namen. Und er erkennt mich nicht mehr. Aber es ist immer noch derselbe Vogel. Irgendwie. Faszinierend. Ich wollte mich von ihm verabschieden.“

Seine Stimme stockte, während er gedankenverloren mit einem schwarzen Pulversack an seinem Gürtel spielte, in welchem etwas knisterte. Warum redete er überhaupt? Und warum so viel? Wollte er seine letzten Stunden des Lebens noch auskosten?

Nachdenklich blickte er in die Ferne. Die rote Sonne ging weit im Westen hinter der endlos scheinenden Steppe unter.

Es wurde dunkel.

„Angst“, flüsterte Nalle leise.

„Oh. Das ist natürlich. Das ist verständlich. Das ist vollkommen in Ordnung“, sagte Ijsdur. „Bald wird sich dein Leben für immer ändern. Aber du wirst nicht mehr leiden. All diese hässlichen Gefühle, die in deinem Kopf herumschwirren, werden sich lindern. Du wirst freier denken können. Du wirst es genießen.“

Nalle grinste schwach: „Du sprichst lustig. Deine Betonung. So altmodisch.“

Dann wurde sie wieder ernst: „Wie ist es? Wie ist es, ein Eis-Dämon zu sein?“

„Willst du es bereits fühlen?“

Ijsdur setzte Nalle sanft ab und zeigte ihr die Kette magischer Kristalle, welche um seinem Hals hing und fest mit seiner Brust verschmolzen war.

„Wenn du willst, kannst du die Eiskristalle anfassen. Sie werden deine Furcht, Sorgen und Schmerzen etwas lindern, wenn das für dich angenehmer ist. Dann weißt du, wie es sich anfühlen wird.“

Nalle überlegte kurz und streckte ihre Hand aus: „Werden die Kristalle auch meine Freude nehmen? Meine Faszination?“

„Sie werden alle gedämpft werden, deine Emotionen“, gab Ijsdur zu, „Doch deine Interessen werden bleiben. Du wirst keine gefühlslose Maschine werden, nur ... kälter.“

„Emotionen sind wichtig“, sprach Nalle fest, „Sie drängen uns zu großen Taten voran und geben uns Kraft.“

Vermutlich hatte ihre Familie ihr dies noch vor Kurzem vorgepredigt. Ijsdur fühlte sich nicht kompetent genug, um sich über den Sinn oder Unsinn von Gefühlen zu unterhalten. Dennoch dachte er laut nach: „Sie mögen uns zu großen Taten antreiben, aber manchmal verschleiern sie auch unsere Sinne und hindern uns daran, unser Potential auszuschöpfen. Wir erkennen sie in anderen Wesen wieder. Und wenn sie gedämpft sind, glauben viele fälschlicherweise, einen nicht mehr zu erkennen. Aber ...“

„Danke für das Angebot, Ijsdur. Doch ich würde lieber noch warten mit der Kette.“

„Das ist auch vollkommen in Ordnung.“

Ijsdur hob Nalle wieder in die Höhe und transportierte sie weiter den Berg hinauf. Sie durchquerten das Felsentor, welches Jahrtausende lang die Eis-Dämonen in ihrem Tal festgehalten hatte. Inzwischen war es schon beinahe ein halbes Jahrtausend lang wieder offen. In dieser Zeit hatte sich das ewige Eis, diese legendäre Eisfläche, aus dem Tal nicht nur bis zum Felsentor hin, sondern auch über das Tor hinweg und ein kleines Stück den Berghang hinunter ausgebreitet.

Ijsdur wollte mehr, als nur das ewige Eis zu berühren. Er wollte nicht riskieren, dass die Weitergabe der Eiskristallkette versagte. Er wollte ins Zentrum des Tals, wo die Kraft des Eises am größten war.

Lange schritten er und Nalle über die Eisfläche zum Mittelpunkt des ewigen Eises. Sie passierten eingeschneite Eissäulen in verschiedensten Formen. Manche sahen gar menschlich aus. Doch gab es kein Leben in ihnen. Sie waren das Werk von vergangenen Eis-Dämonen, welche lange im Ewigen Eis verharrt hatten und verrottet waren. Das Eis hatte sich ihrem stetig schwächer werdenden Willen gebeugt und diese Formen angenommen. Nun zeugten sie nur noch vom wirren Wirken derjenigen, die schon längst vergangen waren. Ijsdur schenkte ihnen keine Aufmerksamkeit. Nalle hingegen drückte sich ängstlich tiefer in Ijsdurs schützende Arme.

Trotz der Decke begann sie zu zittern. Das war eigentlich auch gut so. Schließlich musste ihr todkranker Körper erfrieren, damit sie sich bald als unversehrte Eis-Dämonin wieder erheben konnte. Doch fühlte Ijsdur mit ihr und ihrem Leiden in der klirrenden Kälte. Er bot ihr noch einmal an, dass sie seine Eiskristallkette bereits berühren dürfe, um weniger leiden zu müssen.

Nalle blieb eine Zeit lang stumm. Dann fragte sie stattdessen leise: „Magst du... magst du mir eine Geschichte erzählen?“

Ijsdur blieb überrascht stehen.

„Was für eine Geschichte?“

„Irgendeine. Du bist so alt, du musst doch Unmengen kennen.“

„Dem sollte so sein“, sinnierte Ijsdur, „Ich habe unzählige Reiche bereist. Erlebt, wie Kinder und Kindeskinder von Freunden ihre Plätze einnahmen oder ihre eigenen Wege gingen. Ein ganzer Hort von Geschenken und geschichtsträchtigen Gegenständen lagert auf dem höchsten Gipfel des Kuolema-Gebirges, wo ich mir einst eine letzte Behausung suchte, nachdem die Welt sich zu schnell weiterdrehte für mein uraltes Ich. Doch meine Erinnerungen an all die vergangenen Geschichten sind nicht mehr, was sie einst waren. Lass mich meine Gedanken sammeln und eine Erzählung finden, an der du Freude finden könntest.“

„Was ist mit Choranat? Ich habe gehört ...“

„Nicht Choranat“, sprach Ijsdur bestimmt, „Die Geschichte von Choranat ist nichts für Kinderohren. Dieser Bannkreis hat genug Schaden angerichtet.“

Nalle riss ihre Augen interessiert auf: „Oooh, ist das ein Geheimnis? Ich liebe Geheimnisse.“

Ijsdur stolperte. Einer seiner Mundwinkel zuckte. Die erste wirkliche Gesichtsregung, die Nalle in seiner sonst steinernen Miene erkennen konnte. „Du erinnerst mich an jemanden, der ich vor einer langen Zeit kannte.“

Ijsdurs Hand glitt an seinen Gürtel. Er trug nicht viel bei sich – musste er doch im ewigen Eis kaum Nahrung zu sich nehmen und sich auch nicht mit Kleidung vor der Witterung schützen – doch dieser von Runenmustern übersäte Hammer, der an seiner Hüfte hing, war schon seit mehreren Jahrhunderten sein stummer Begleiter.

„Magst du mir von ihm erzählen, an den ich dich erinnere?“, fragte Nalle neugierig.

„Von ihr. Es war eine sie.“

Ijsdur seufzte und begann, zu erzählen. Seine sonst tonlose Stimme wurde weich. Langsam tröpfelten die Worte aus seinem Mund, dann immer schneller, bis er wie ein Wasserfall sprudelte. Er war ungeübt darin, Geschichten zu erzählen. Doch hatte er es einst gerne getan. Er konnte es wieder probieren.

„Diese Geschichte spielt zu einer Zeit, als ich noch ein ganz frischer Eis-Dämon war. Kaum zwei Jahre lang war ich ziellos übers ewige Eis gewandert und hatte mich an meine Existenz gewöhnt. Dann, eines Tages, wurde das uralte Felsentor geöffnet, das das ewige Eis seit Jahrtausenden in unserem schattigen Tal hoch oben im Kuolema-Gebirge festgehalten hatte. Wir Eis-Dämonen konnten erstmals wieder in die weite Welt hinaus reisen. Siantari, die Herrin des ewigen Eises, die uns geschaffen hatte, erlöste uns und ließ uns frei sein. Vergleichsweise frei.

Doch nur wenige Eis-Dämonen waren noch genug bei Verstand nach ihren hunderten von Jahren im Schnee und Eis. Ich glaube, ich war der erste, der die Berghänge heruntertorkelte. Vielleicht auch der einzige.

Jedenfalls stolperte ich, Schnee und Eis hinter mir herziehend, nach Tulgor. Ins Dorf, das einst Ijs‘ Heimat gewesen war.“

Ijsdur stockte kurz und überlegte, wie viel er Nalle zumuten konnte. Er entschloss sich für eine Beschönigung der Geschehnisse. Er musste sie ja nicht anlügen. Nur nicht die ganze Geschichte erzählen.

„Mein Vater – mein leiblicher Vater Saro, nicht der Eis-Dämon, der mir seine Kette übergab – war der erste und einzige Tulgori, der mich wiedererkannte.

Viele Tulgori, vor allem abenteuerlustige Jungen, wagten sich Jahr für Jahr ins Gebirge. Niemand kehrte je wieder, und wenn die ersten Sonnenstrahlen des Frühlings den Schnee an den felsigen Hängen schmolzen, gab ihnen das Gebirge die Toten zurück. Doch Ijs‘ Körper war nie gefunden worden. Mein Vater und mein Bruder hatten nach mir gesucht und unter großer Trauer die Überreste meiner leichtsinnigen Begleiter gefunden, doch nicht mich. Und so hatten sie stets heimlich gehofft, Ijs könnte irgendwie überlebt haben, ja, einen Weg über das Gebirge gefunden zu haben.

Dem war nicht so.

Und mein Vater Saro war nicht glücklich, als er mich wiedersah. Leblos erschien ich ihm. Er zitterte stets in meiner Nähe. Er kam nicht damit klar, dass überall um mich herum Schneeflocken herumschwirrten, Küche und Schlafzimmer bedeckten. Und als ich ihm sagte, dass ich sein Sohn sei, da spürte ich, dass er mir nicht glaubte. Dass er mich verabscheute.

Da wandte ich mich ab zu gehen. Ich wusste, dass ich eigentlich Trauer, Schmerz oder dergleichen fühlen sollte. Aber in mir war nur eine Kälte. Eine Kälte, die meinen Vater von mir abwandte. Dies war das erste Mal, dass ich mir wünschte, wieder ein Mensch zu sein.

Ich will es nicht beschönigen. Solche Wünsche nach Menschlichkeit wirst du auch von Zeit zu Zeit haben, Nalle. Aber sie werden vorbeigehen. Und vermutlich mit der Zeit immer seltener werden. Im Gegensatz zu meinem Vater wissen die deinen, was dich erwartet. Hoffentlich reagieren sie besser auf dich.

Jedenfalls teilte mein Vater mir mit, dass mein Bruder nicht mehr in Tulgor verweilte. Eforas war gemeinsam mit einigen anderen unter dem Kuolema in Richtung eines fremden Landes namens Andor aufgebrochen. In Richtung des legendären Königreichs, aus dem einst ein gewisser Haamun angereist war, und wohin ein gewisser Barz hatte reisen wollen.

Sie waren erst vor wenigen Tagen aufgebrochen. Da fragte ich mich, ob es vielleicht einen Zusammenhang gab zwischen dem Einsturz des Felsentors und dem Aufbruch der Reisenden. Und ohne besseren Plan für mein zukünftiges Leben, ohne Platz in Tulgor, reiste ich los, ihnen nach, unter dem Berg hindurch. Andor machte mich neugierig. Und vielleicht würde mein Bruder mit mir besser klarkommen als mein Vater. Zu Beginn würde er das zwar nicht tun. Doch mit der Zeit würde er sich lockern.

So unterquerte ich das Kuolema-Gebirge. Ich konnte mich sogar besser durch die labyrinthischen Gänge der Temm orientieren als die Reisengruppe selbst, welche immer wieder in Sackgassen landeten. Ich konnte Abkürzungen finden, ja, mithilfe meiner Eismagie gar eigene Abkürzungen schaffen.

So überholte ich unwissentlich die Reisegruppe von Eforas und Haamun. Und ich erreichte als allererster das fremde Königreich.

Zu einer ähnlichen Zeit – oder vielleicht eher einige Wochen zuvor – war eine gewisse Iril kurz davor gewesen, den Wachsamen Wald zu erreichen.“

„Ist Iril die Person, der einst dein Hammer gehörte?“, unterbrach ihn Nalle.

„Genau so ist es“, lächelte Ijsdur. Er orientiere sich wieder neu und räusperte sich. Dann intonierte er mit klirrender Stimme weiter:

„Das stolze Schiff, die BALENA, glitt über die stürmische See. Sie war von Hadria in See gestochen, hatte Iril und einige Silbergüter der Silberzwerge von Silberland aufgegabelt und hatte inzwischen Werftheim passiert. Die Mannschaft hatte aufgeatmet. Die Windrichtung änderte sich bei den Nebelinseln ständig und nie konnte man vorhersehen, ob das Wetter halten würde. Doch wenn man einmal Werftheim passiert hatte, konnte man so gut wie sicher sein, dass die transportierten Waren ihren Zielhafen bald erreichen konnten. Und was hatten sie nicht alles für Waren geladen! Rindsleder aus Sturmtal, Werkzeuge aus Werftheim, Silberarbeiten aus Silberhall und allerlei ähnliche Produkte, die den Seehandel im und um das Hadrische Meer so lukrativ machten.

Die Wellen des Hadrischen Meeres schlugen gegen einen vollkommen überfüllten Bug. Die Gischt trug Meeressalz und den Geruch nach nassem Holz tief in Irils Nase. Sie prustete.

Am Horizont sah Iril hohe Bäume aufragen. Wie an einen entfernten Schatten erinnerte sie sich an die Silhouetten. Dahinter lag es, das Land ihrer Kindheit. Viele Jahre waren vergangen, seit sie von hier aufgebrochen war. Jahre, in denen sie sich großes Wissen angeeignet hatte. Und nun kehrte sie zurück, um Geheimnisse zu entdecken, derer sie sich jetzt erst gewahr wurde. Doch mit dem Horizont näherten sich auch kalte Zweifel: Würden die anderen Zwerge sie überhaupt willkommen heißen?

Sie hatte damals ihre Familie in Cavern zurückgelassen, ja, sich gar mit ihr zerstritten. Viele der Altvorderen hatten es nicht mit Freude vernommen, dass sich einige Schildzwerge nach der Entdeckung der Silbervorkommen unter dem Silberberg am südlichsten Zipfel Silberlands niedergelassen, die Silbermine Silberhall gegründet und den Silberschild dort oben behalten hatten. Diese Sehnsucht nach Neuem hatte so einige Schildzwerge angesteckt, was von den nicht Sehnsüchtigen nicht mit Wohlwollen quittiert wurde. Schließlich hatte dieselbe Sehnsucht den vorherigen Fürst Hallwort und dessen Gefolge auf Jari Dorrs Handelskogge – möge dieser Dieb in der Unterwelt verrotten – in den Norden gelockt und ihn sein Leben gekostet.

Nichtsdestotrotz hatte Aufbruchsstimmung in der Luft gelegen und in den folgenden Jahren hatten sich viele Schildzwerge aufgemacht, um Hallworts Vorbild zu folgen und sich den Silberzwergen anzuschließen. Manche hatten gar daran geglaubt, dass es sich hierbei um einen göttlichen Fingerzeig gehandelt hätte, der die Zwerge in den Norden riefe.

Iril war aus anderen Gründen nach Silberland gereist. Und nun, Jahrzehnte später, war Iril darauf und daran, ihre alte Heimat wiederzusehen. Nicht nur deswegen ging es ihr gerade speiübel.

Du musst bedenken, Nalle, dass sie eine Zwergin war. Zu meiner Zeit lebten in Tulgor noch keine Zwerge, aber du kennst bestimmt einige, oder?“

Nalle nickte stumm.

„Man möge möglicherweise meinen, dass die Körper von Menschen und Zwergen aufgrund ihres derart ähnlichen Aussehens auch ganz gleich funktionierten, aber dem ist nicht ganz so. Bei Zwergenkörpern geht vieles irgendwie langsamer voran. Sie wachsen und altern in anderem Tempo. Verletzungen heilen gemächlicher. Krankheiten brauchen länger zum Abklingen, aber auch zum Ausbrechen. Die erheblich erhöhte Lebenserwartung im Vergleich zu Menschen ist da keine Überraschung. Manche behaupten, die sturen Zwerge bräuchten auch länger, um aus alten Denkmustern auszubrechen und Neues zu lernen. Und ihre Schwangerschaften dauern signifikant länger und sind seltener.

So ganz nebenbei gesagt, gibt es natürlich auch bei den typischerweise seltenen Schwangerschaften Ausnahmen. Ich kannte da einen gewissen Zwergenfürsten, dessen Eltern über ein Dutzend Kinder in die Welt gesetzt hatten.

Auf jeden Fall dauert bei Zwergen auch der Zyklus signifikant länger und ihre Zeit des Blutes kann besonders heftig sein. Wurde mir gesagt. Erinnerungen daran wurden auch schon direkt mit mir geteilt. Sagen wir einfach, es ging Iril gerade speiübel, während ihr Schiff auf die Küste des Wachsamen Waldes zuhielt.

Geräuschvoll übergab Iril sich über die Reling und warf den sie belustigt begutachtenden Matrosen böse Blicke zu. Zu ihren Krämpfen gesellten sich stechende Kopfschmerzen. In Silberhall hätte Iril frisch gebrauten Mondkrauttee schlucken und ihre Füße hochlegen können. Doch hier, auf diesem Schiff, blieb ihr kaum etwas anderes übrig, als die Zähne zusammenzubeißen und zu hoffen, dass die Fahrt nach Andor möglichst bald und möglichst ereignislos zu Ende gehe.

Leider war sie nicht so glücklich.

Ein markerschütternder Schrei hallte durch den Wachsamen Wald.“

Nalle blickte Ijsdur erwartungsvoll an. Dieser legte theatralisch den Kopf in den Nacken und schrie in die dunkle Nacht hinaus.

„DRACHEE!“



\newpage
\section{Der Zorn des Drachen}

\az{Jahr 65}


„DRACHEE!“

Der verzweifelte Schrei hallte durch den Wachsamen Wald und weit ins Hadrische Meer hinaus. Die Nixen, welche bis eben noch dem Schiff nachgeschwommen waren und der blassen Iril mitleidige Gute-Besserungs-Wünsche zugesprochen hatten, waren auf einmal nirgendwo mehr zu erblicken. Vermutlich hatten sie sich in ihre tief unter der Oberfläche liegenden Behausungen zurückgezogen.

Keine Minute zu früh.

Denn auf einmal waren das Rauschen gewaltiger Schwingen und ein fernes Brüllen zu vernehmen.

„Ein Drache?!“, wiederholte Iril ungläubig, „Einen Drachenangriff gab’s doch nicht mehr seit dem Unterirdischen Krieg! Der ist Jahrhunderte her!“

Damit hatte sie fast recht. Sie konnte das nicht wissen, denn die wenigen, unvollständigen Informationen zum letzten aller Drachen waren verstreut über verbotene Schriftrollen in den Schwarzen Archiven und alte Lieder von Tavernenwirtinnen, die Iril nicht kannte. Doch ein einziger Drache hatte den Unterirdischen Krieg und das nachfolgende Massensterben überlebt.

Tarok. Er hatte sich während Jahrhunderten in seiner Höhle im Grauen Gebirge verschanzt und in unruhigen Träumen seinen Zorn gepflegt. Hin und wieder hatte Tarok nichtsahnende Reisende überfallen, doch seit einem legendären Kampf gegen den zukünftigen König Brandur von Andor hatte er auch das kaum mehr getan. Er hatte sich gefürchtet, geschlafen und gelitten. Während ein, zwei Gelegenheiten, die hier nicht näher behandelt werden sollen, hatte Tarok sich aus unruhigen Träumen reißen lassen und in die Lüfte erhoben, um Feuer und Glut vom Himmel regnen zu lassen und Bewohner der Berge und Steppe an seinem Leid Anteil haben zu lassen. Und um diese widerlichen Krahder in ihre Grenzen zu verwesen.

So kam es, dass hin und wieder in der Bevölkerung das Gerücht umging, dass jemand einen leibhaftigen Drachen gesehen hätte.

Doch davon abgesehen hatte Tarok sich während Jahrzehnten still in seiner Höhle versteckt und Kraft aus Krahal gezogen. Die Schildzwerge und ihre Drachenfallen hatte er gemieden. Brandur hatte er gemieden. Andor hatte er gemieden. Ruhig darauf wartend, dass König Brandur das Zeitliche segnete. Er war ein Drache, theoretisch unsterblich, sofern er nicht dem Fluch des Steins anfiel. Brandur war nur ein Mensch. Mochte die Hexe Reka sein Leben mit Tränkchen und Mitteln unnatürlich in die Länge ziehen: Früher oder später würde die Zeit seinen Erzfeind für ihn erledigen.

Aber das musste nicht heißen, dass Tarok nicht nachhelfen konnte.

Nun war es soweit gewesen. Taroks Kreaturen hatten unter dem Befehl seines Schwarzen Herolds die Rietburg eingenommen und Brandur tödlich verwundet. Während die Helden um die Befreiung der Rietburg gekämpft hatten, war der alte König seinen Verletzungen erlegen. Und Tarok war erwacht. Noch in derselben Stunde hatte er sich aus seiner Knochengrube erhoben und war schnurstracks zum Angriff auf Andor übergegangen.

Auf seinem Weg zur verhassten Rietburg hatte er einen Abstecher zum Baum der Lieder gemacht und die Magie des Landes in sich aufgezogen. Nun, mit erheblich gewachsener Macht, erhob sich der Drache mit mächtigen Flügelschlägen aus dem Wachsamen Wald in die Höhe.

Und wurde unter anderem von einer gefährlich nahe segelnden Handelskogge erspäht.

Iril kniff ihre Augen zusammen. Täuschte sie sich, oder saß tatsächlich ein Reiter auf dem Rücken des Drachen? Zwischen zwei Rückenstacheln eingekeilt, ein langes Schwert hoch in die Luft gereckt, den langen schwarzen Umhang hinter sich flatternd.

Irils Aufmerksamkeit wurde vom mysteriösen Reiter weggerissen, als der Drache sich einmal im Kreis drehte, während er einen gewaltigen Flammenstrahl spie.

Bäume und Büsche entflammten. Beißender Rauch stieg auf vom Wachsamen Wald. Und Iril wurde ihrer prekären Lage gewahr. Sie befand sich auf einem brüchigen Kahn vor einem brennenden Wald, mit einer riesigen Echse hoch über ihr, die sie jeden Augenblick erspähen und für Grillspaß befinden konnte.

„Wir müssen weg von hier, echt schnell!“, rief Kapitän Lunor und kurbelte wild am Steuer herum. Lunor war ein stämmiger, muskulöser Mann, an dessen Körper alle Stellen, deren Anblick man schicklicher Weise Anderen zumuten konnte – und vermutlich auch einige unschickliche Stellen – mit Tätowierungen bedeckt waren. Diese Tradition hatte Kapitän Lunor an seine Mutter angelehnt, der legendären Kapitänin Mondrianne, die der Legende nach einst gar ein Handelsschiff aus Oktohans Schlund gerettet hatte. Im Gegensatz zu Irils Runentattoos hatten Mondriannes und Lunors Tätowierungen eher persönliche als magische Bedeutungen. Und im Gegensatz zu Iril hatten sie keine magische Kraft nötig. Mit purer Muskelkraft zog Lunor am großen Steuer und blinzelte in die Gischt. Die BALENA legte sich gefährlich schief und drehte sich langsam vom Ufer ab, in Richtung des sicheren Nordens. Iril klammerte sich an der Reling fest, um nicht übers Deck zu rutschen. Doch der Wind war nicht mit ihnen. Nur langsam kamen sie voran.

Die meisten Reisenden suchten unter Deck Schutz. Nicht so Iril. Sie öffnete ihre Reisetasche und kramte daraus eine bestimmte kleine Metallscheibe hervor, kaum größer als ihre Handfläche. Es war keine komplexe Runenfolge, die sie jetzt brauchte, aber eine sehr mächtige. Sie griff an ihren Gürtel.

Es fühlte immer noch falsch an, Burmrits Runenhammer selbst zu führen. Sorgfältig strich Iril über die gravierte Oberfläche und über das stumpfe Ende des Hammers. Ein leises Summen ertönte aus seinem metallenen Inneren. Er war bereit, magische Ströme weiterzuleiten. Iril fasste den Runenhammer fest und ließ ihn mit der flachen Seite auf die Runenscheibe niederfahren. Ein Geräusch wie von einem Gong ertönte, gefolgt von einem fernen Donnern. Runenscheibe, Hammer und Irils Augen glühten gelblich auf. Ihr ganzer Körper kribbelte wohlig.

Der Wind nahm zu und bauschte die Segel des kleinen Schiffs. Die BALENA nahm Fahrt auf, so schnell wie möglich weg vom Ufer mit dem wütenden Drachen. Die beiden Seekrieger, die nebst Kapitän Lunor als einzige auf Deck geblieben waren, jubelten und klatschten. Doch dann furchte die eine plötzlich ihre Stirn und schrie an Iril zurück: „Abbremsen! Abbremsen! Da vorne ist der Hirschhuf! Klippe voraus!“

Iril drehte die Runenscheibe in ihren Händen, verzweifelt versuchend, den Wind noch rechtzeitig abzulenken.

Kapitän Lunor drehte ebenso verzweifelt am Steuer und fluchte gegen den aufziehenden Sturm.

Mit voller Fahrt raste das Schiff in die Klippe hinein.

Ein hässliches Knirschen ertönte, als der Rumpf des Schiffes sich öffnete und Wasser hineinströmte.

Mit einem Ruck wurde Iril von der Reling gerissen und taumelte übers Deck.

„Sporndreck!“, fluchte sie. Schuldgefühle übermannten sie. Doch sollte sie sich lieber später damit befassen, wenn sie nicht mehr auf einem sinkenden Schiff stand.

Die Seekriegerin fasste sich als erste und schrie: „Hier sind wir nicht sicher. Wenn ein Drache hier ist, sind seine bösartigen Kreaturen nicht fern! Und das Meer wimmelt nur so von grausamen Meereskreaturen!“

„Wo sollen wir hin?! An Land wütet der Drache!“

Iril warf einen wachsamen Blick ans Land rüber. Der Drache schwebte nicht mehr über dem Wachsamen Wald, sondern hatte sich weiter gen Westen bewegt. Der dunkel gewandte Reiter auf seinem Rücken zeigte mit seinem langen Schwert nach unten. Der lange Hals des Drachen folgte. Und wie ein Seeadler stieß der mächtige Drache auf irgendetwas im Westen herab.

Iril öffnete erneut ihre Reisetasche und zog eine andere Runenscheibe hervor. Diese hier bestand aus einer handgroßen gläsernen Linse, die in einem mit Runen übersäten Steinring eingefasst war. Iril hielt ihren Runenhammer darunter und konzentrierte sich. Licht strömte hervor, schillerte in allen Regenbogenfarben und ließ in der Runenlinse ein verzerrtes, bewegtes Bild erscheinen.

Iril blickte aus der Vogelperspektive auf das Geschehen herab. Ein großer Schatten fiel auf das Rietland. Das goldene Rietgras zerfiel unter einem Flammenstrahl zu schwarzer Asche. Der Drache stieß auf einen alten Wachturm hinab.

Iril kniff ihre Augen zusammen und versuchte, in der Schemen dieser Runenscheibe mehr Einzelheiten zu erkennen.

Das Gemäuer, das sie erblickte, musste der legendäre alte Wehrturm sein. Vor Urzeiten war er einst von den Schildzwergen erbaut worden, doch seitdem er einst in einem Kampf gegen einen riesigen Trollfürsten eingerissen worden war, munkelte man, dass ein Fluch auf ihm läge. Schon so oft hatte jemand versucht, ihn wieder aufzubauen, und so oft war jemand daran gescheitert. Jetzt stand der Turm wieder. Vermutlich hatten mutige Andori ihre Kämpfer mit dem Wiederaufbau des Turms unterstützt, so gut es ging. Iril wäre jede Wette eingegangen, dass er in einigen Monaten schon wieder eine Ruine wäre.

Iril konzentrierte sich stärker. Das magische Bild innerhalb der Runenlinse zoomte hinein und verdeutlichte sich.

Vor dem Wehrturm hatte sich eine Gruppe Krieger zusammengeschart. Kleine Gestalten umringten einen Reiter auf einem schwarzen Pferd. Sie flohen nicht, sondern schienen ihre Waffen bereitzuhalten. Ganz oben auf dem Wehrturm hatte sich gar ein Bogenschütze eingerichtet und verschoss wirkungslos Pfeile auf den auf sie zustürzenden Drachen.

Tapfere Idioten.

Schade um sie.

Der verzerrte Drache in der Runenscheiben-Vision riss seinen Mund zu einem Schrei auf. Kurz darauf schallte ein leises Echo seines animalischen Schlachtrufes an Irils Ohren.

Sie zuckte zusammen, als Kapitän Lunor direkt neben ihr zu erkennen gab, dass er die Scheibe ebenfalls beobachtet hatte. Laut rief er: „Der Drache ist abgelenkt. Das Risiko an Land ist echt geringer, als wenn wir im sinkenden Schiff bleiben. Hier sind wir wie ein Signalfeuer für Nerax. Und für Schlimmeres. Echt, wir müssen an Land rüber. Macht euch bereit“

Mit Blick auf Iril fügte er an: „Kannst du schwimmen?!“ Iril nickte hastig. Technisch gesehen konnte sie schwimmen. Sie hatte es nur schon lange nicht mehr getan. So schwer konnte es ja nicht sein, sich an die Züge zu erinnern.

Während Lunor unter Deck rannte und seine kleine Tochter an Bord rettete, packte Iril ihre Runenscheibe zurück in ihre luftdicht verschließende Tasche und setzte an, ihre Schuhe und sonstige schwere Kleidung zurückzulassen. Da tauchten plötzlich zwei Nixen vor ihr aus dem Wasser auf. Sie blubberten und heulten etwas. Unter Wasser hätte es bestimmt schön geklungen und der Klang wäre weit gedrungen, doch hier, bei der Geräuschkulisse eines Drachenkampfs im fernen Hintergrund, hätte Iril sie wohl nicht einmal verstanden, wenn sie Nixisch sprechen könnte.

Natürlich gäbe es entsprechende Runentattoos, die Iril aktivieren könnte, um die Nixen zu verstehen. Magie war ein wundervolles Allzweckmittel, mithilfe dessen man überraschend schnell Strukturen in gesprochenen Worten erkennen und fremde Sprachen entschlüsseln konnte. Doch soweit musste Iril gar nicht greifen. Die freundlich lächelnden Gesichter der Nixen waren verständlich genug, sodass Iril sich die Bedeutung der fremden Worte vorstellen konnte. „Wir können euch helfen“, sagten sie leise.

Die beiden Seekrieger schlugen das Angebot der Nixen zugunsten weniger schwimmfähiger Passagiere aus und sprangen elegant ins Meer. Es geschah nicht selten, dass ein Schiff in einen Sturm geriet (oder schlimmeres) und seine Passagiere dann irgendwo an Land gespült wurde. Seekrieger mussten das Schwimmen beherrschen, und ihre Schulterplatten waren aus leichtem, auftreibendem Material statt schwerem Metall. Kompliziert war vielmehr, dass der eine Seekrieger während der Überfahrt einen strampelnden Streifenmarder an sich drückte und über Wasser zu halten versuchte[1]. Aber das Schiffsmaskottchen zurückzulassen kam nicht in Frage.

Lunor weigerte sich, seiner Tochter einer Nixe zu übergeben. „Ich habe echt schon seit meiner Kindheit der See getrotzt! Ich kümmere mich selbst um die Sicherheit meines Töchterleins.“

Er führte seine Tochter sanft in die wogende See und schwamm mit kräftigen Zügen in Richtung Ufer.

Die restlichen Passagiere ließen sich bereitwillig von den Nixen an Land bringen. Iril beobachtete, wie die zwei Seekrieger ans stürmische Ufer traten und sofort ihre langen Naginata bereithielten, als würden diese Stangenwaffen etwas gegen einen angreifenden Drachen ausrichten können.

Dann war Iril an der Reihe. Das Wasser des Hadrischen Meers war kühl und stürmisch. Salz ließ Irils Augen brennen, während sie sich verzweifelt an den glitschigen Körper ihrer Nixe zu klammern versuchte. Sie hustete und prustete und schluckte Seetang. Dann war es vorbei.

Iril schleppte sich an Land und ließ sich aufs sandige Ufer sinken

So hatte sie sich ihre Rückkehr nach Andor nicht vorgestellt.

„Danke.“

Die Nixe blubberte ihr eine unverständliche Abschiedsnachricht zu und sprang dann zurück in die tosende See, um weitere Reisende aus dem Wrack zu retten. Iril wrang ihre nassen Haare aus und betrachtete das Schiffswrack auf dem Hirschhuf, der Klippe vor dem Wachsamen Wald. Sie versuchte nicht an die vielen Waren zu denken, welche nun dort nun entweder verrotten oder von Meereskreaturen gestohlen werden würden. Immerhin waren alle Schiffsbrüchigen in Sicherheit.

Da ertönte ein Schrei direkt neben ihr. Sie wirbelte herum. Kapitän Lunor saß klitschnass neben seiner Tochter. Selbige lag zusammengesunken im seichten Ufergebiet und rang erfolglos nach Luft. Wasser strömte aus ihren Augen und ihrem Mund. Lunor schüttelte seine Tochter panisch. Das half natürlich nicht.

„Lasst mich!“, rief Iril. Sie kniete sich hastig neben die Kleine. Der Runenhammer glitt zu Boden. Iril öffnete ihre Reisetasche und durchsuchte ihre Vorräte an kleinen metallenen Scheiben. Manche Runenfolge war so vielseitig nützlich, dass sie sie permanent auf einer ihrer Scheiben eingraviert hatte und mit sich führte. Iril fand die gesuchte Scheibe, hob sie in die Höhe und ließ sie mit aller Macht auf ihren Runenhammer donnern. Die Scheibe flackerte bläulich auf. Dünne Linien darauf begannen zu schimmern und verschmolzen rasch zu Symbolen. Die Scheibe wurde warm. Ebenso pulsierten Irils Augen und der Runenhammer in blauen Tönen. Ihre gesamte Haut kitzelte wohlig.

Der Sturmwind ebbte ab.

Überall um Iril herum stiegen Wassertropfen aus dem matschigen Boden in die Höhe. Iril selbst fühlte ebenfalls den vertrauten Zug des Wassers in ihrem Körper, auch wenn sie selbst natürlich zu schwer war, um davon in die Höhe gehoben zu werden. Iril bewegte die leuchtende Runenscheibe zur hustenden Tochter und zog sie langsam von ihrem Magen bis zu ihrem Gesicht. Weitere Wasserschwalle brachen aus der Kleinen hervor und sammelten sich als wabbelige Wassermasse um die Schiebe. Ihr Gesicht rot an, ebenso Irils Hand, die die Scheibe hielt. Dann erbrach sich die Kleine und sank schwach in die Arme ihres Vaters. Angeschlagen, doch wieder angemessen atmend.

„Danke. Echt“, hauchte Lunor, seine Tochter eng umarmend.

„Kein Problem. Wasserrunen sind meine Spezialität. Könnte mit der Hintergrundgeschichte dieses Hammers zusammenhänge ... aber nein, die ist jetzt nicht wichtig“, murmelte Iril. Sie tippte so rasch wie möglich auf die Scheibe und brach den Anziehungseffekt ab, ehe sie aus Versehen die Kotze der Kleinen ansaugte.

Erst jetzt konnte Iril sich wirklich auf die Umgebung konzentrieren. Die Nixen hatten sie zu den Anlegestegen der Bewahrer gebracht. An diesem Hafen wurde sonst oft emsig gehandelt. Nicht so heute.

„Hierher. Da lodert es!“

„Eimerkette bilden! Dalli, dalli, der Wald wartet nicht!“

Waldbewohner eilen umher und versuchen, die Brände im Wachsamen Wald einzudämmen.

Der über die restlichen Bäume hinausragende Wipfel des gewaltigen Baums der Lieder war angekokelt. Löschfässer wurden von seinen obersten Ästen gelöst und verschwanden hinter den tiefer liegenden Baumwipfeln, gefolgt von Platschgeräuschen.

Einige Bogenschützen in der Kleidung der Farbe des Sommerlaubes erreichten die leeren Anlegestege. Manche hielten Bögen im Anschlag, andere trugen große Rucksäcke. Sie begleiteten weitere Bewahrer in grauen Gewändern. Ein Bewahrer, in ein weißes Kleid gekleidet, rollte auf einem hölzernen Stuhl mit breiten Rädern an der Seite über den Waldpfad zum Hafen und balancierte mühevoll einen Stapel breiter Bücher und Schriftrollen-Kisten auf seinem Schoß, so hoch, dass er selbst kaum darüber hinwegblicken konnte. Sein kleiner Kopf guckte auf der Seite des Bücherstapels hervor und sank enttäuscht, als er den leeren Hafen erblickte.

„Was nun, Hoher Priester Tion?“, fragte ihn eine Begleiterin.

Tion kratzte sich am Bart. „Unschön. Wenn in den nächsten Tagen keine Schiffe von hier abfahren, müssen wir umgehend kehrt machen und anderswo Asyl suchen. In die Tiefen Caverns oder in die Lande der wilden Völker des Ostens. Nur weit weg vom wütenden Drachen.“

Iril sollte später herausfinden, dass die Bewahrer hier am Hafen auf Boote gehofft hatten. Der oberste Priester der Bewahrer hatte sie weise aufgefordert, das Land zu verlassen. Sie hatten den Auftrag, so viele Pergamente wie möglich vom Baum der Lieder mitzunehmen und nach Sturmtal aufzubrechen. Von dort sollten sie ein Schiff nehmen und weit, weit fortsegeln. Denn Tarok, der Drache, würde den Baum der Lieder bestimmt nicht verschonen, wenn er erst einmal die Rietburg dem Erdboden gleichgemacht hatte.

Und nun lagen nicht einmal mehr Boote zum Fliehen an den Stegen.

Die Bewahrer waren nicht als einzige über die fehlenden Schiffe im Hafen enttäuscht. Garz, ein im gesamten Norden berühmt-berüchtigter Handelszwerg mit einem komödiantisch dicken Rucksack auf dem Rucksack, blickte fassungslos den leeren Steg entlang und murmelte: „Mein Schiff nach Hadria sollte doch schon längst hier am Hafen vor Anker liegen. Bei allen Kreaturen der Tiefe, wo ist mein Schiff nach Hadria?!“

„Auf der Klippe liegt es. Ein Wrack ist es geworden!“, sprach Kapitän Lunor missmutig, immer noch seine Tochter umarmend, „Und es war nicht mal mein eigenes Schiff. Syenna wird mich umbringen.“

Das holte Iril in den Moment zurück.

„Verzeiht mir. Das ist alles meine Schuld“, sprach sie hastig, „Ich weiß nicht, wie ich dafür aufkommen soll. Ich bin nicht reich, habe aber noch einige Goldmünzen ...“

„Nein“, widersprach Lunor, „Es mag dein magischer Wind gewesen sein, der uns auf den Hirschhuf auflaufen ließ. Aber es war auch mein Segel. Und mein Steuer. Ich werde dafür geradestehen. Du hast meine Tochter gerettet. Ich würde sagen, wir sind echt mehr als quitt.“

Iril nickte, doch das Schuldgefühl in ihrem Inneren ließ sie nicht los. Sie war es gewesen, die Lunors Tochter überhaupt erst in Gefahr gebracht hatte. Niemand hätte ihr verziehen, wenn die Kleine ihr Leben verloren hätte.

„Hört ihr das auch?“, unterbrach sie der eine Seekrieger, der immer noch seinen tropfnassen Streifenmarder an sich drückte und streichelte.

„Was meinst du, Stinner?“, fragte die andere Seekriegerin, „Ich höre nichts.“

„Eben.“

Bislang war die ganze Zeit wie in weiter Ferne ein Echo zu vernehmen gewesen. Ein Brüllen und Fauchen, das den Boden leicht erzittern ließ. Der Lärm eines Drachenkampfs. Doch nun war es plötzlich still geworden.

Einen Augenblick lang standen alle Anwesenden wie erstarrt da.

Urplötzlich rumpelte die Erde. Heftig. Vögel stiegen von den Bäumen auf. Laub, Äste, Eichhörnchen und kletternde Streifenmarder wurden gleichermaßen zu Boden geworfen. Sturmwellen platschten mannshoch ans Ufer. In der Ferne hörte man Felsen bersten und Lawinen krachen.

Dann war das Erdbeben auch schon wieder vorbei. Der Wachsame Wald lag wieder ruhig da. Und dennoch konnte Iril das Gefühl nicht abschütteln, dass etwas grundlegend anders war als noch vor wenigen Minuten.

„Was war das? Seht ihr irgendetwas?“, rief eine Bewahrerin zwei anderen zu, welche mit Ferngläsern in den Himmel starrten.

„Kein Drache in Sicht“, kam die Antwort.

„Ich glaube echt, der Drache ist tot!“, jubelte Lunor ein wenig optimistisch. Die Seekrieger stimmten mit ein. Die Bewahrer vom Baum der Lieder schienen noch skeptisch.

Der Hohe Bewahrer Tion gab den Befehl, man solle ihn und die wertvollen Schriften zurück zum Baum der Lieder befördern. Hier am Hafen waren sie nur ein Ziel für Kreaturen. Und hoffentlich würde bald schon ein Falke mit Neuigkeiten eintreffen.

Iril schloss sich ihnen an.

Sie bot Tion auch an, einen Teil seines gewaltigen Bücherstapels abzunehmen, doch der Hohe Bewahrer winkte entschieden ab. Die Schriften wären sowohl sehr wertvoll als auch potenziell gefährlich und sollten somit lieber nicht von einer dahergelaufenen Kriegerin transportiert werden. Iril versuchte, das nicht persönlich zu nehmen.

Als Tion dann allerdings über eine zu große Astwurzel rollte und sein Schriftenstapel beinahe in sich zusammenfiel, konnte er sich immerhin dazu herablassen, Iril die oberste Schachtel voller Schriftrollen tragen zu lassen.

So reiste die Truppe weiter zum Baum der Lieder.\bigskip







Ein riesiger schwarzer Fleck zeigte an, wo unlängst ein gewaltiger Drache auf dem Baum der Lieder gesessen und den uralten Mammutbaum mit Klauen bearbeitet hatte.

Die Lichtung war übersät von angekokelten Riesenästen und den Leichen von Menschen und Zwergen, Gors und Skralen. Die sonst rötlich-pink schimmernden Schuppen der Kreaturen waren von schwarzen Spuren übersät und die sonst weißen Augen schimmerten tiefschwarz. Hatte der Drache sie mit einem dunkelmagischen Blitz gestärkt?

Auch wenn die aktuelle Gefahr durch das Feuer gebannt sein mochte, herrschte an der Lichtung wie am Hafen emsiges Treiben. Lösch- und Bergungsaktionen wurden durchgeführt. Verletzte sammelten sich an feuchten Stellen und blickten furchtsam in den Himmel.

„Doro, überprüfe bitte, ob bereits Falken eingetroffen sind“, sprach der Hohe Bewahrer Tion zu einer seiner Begleiterinnen, kaum hatten sie den Rand des Dorfes erreicht. „Und magst du auch noch beim Ausguck nach Neuigkeiten vorbeischauen?“

„Bis ganz nach oben?! Oh, Mutter, wie ich mir wünschte, dass es Abkürzungen für so etwas gäbe“, sprach Doro, eilte dann jedoch folgsam davon. Tion drehte sich um, wies einen hammerschwingenden Begleiter auf einen seiner Meinung nach höchst einsturzgefährdeten Ast hin und sprach weiter: „Und Giftknödel, kannst du bitte ... Giftknödel, wo steckst du?!“

Der angesprochene junge Bewahrer eilte zu Tion, um weitere Anweisungen entgegenzunehmen. Allzu dringlich konnten sie jedoch nicht sein. Denn anstatt fortzueilen, nickte Giftknödel nur und kehrte danach wieder an Irils Seite zurück. Er warf weitere interessierte Blicke auf Irils leuchtenden Hammer, wie er es schon auf dem ganzen Weg zum Baum der Lieder getan hatte.

„Das ist ein Runenhammer“, meldete Iril kurz angebunden, „Der Runenhammer von Golja.“

„Runenmagie?“

„Genau! Ihr Bewahrer wisst bestimmt so einiges darüber.“

„Aber nicht über einen solchen Hammer. Die Lücken in unseren Archiven schrumpfen stetig und werden doch nicht weniger. Welch mannigfaltige Fähigkeiten verleiht der Hammer dir?“

„Ich habe lange studiert, um verschiedenste Aspekte der Macht der Runen zu meistern. Der Hammer ist nur ein kleiner Teil davon.“

„Noch bin ich nur ein Novize, doch bald, in weniger als zwei Jahren, werde ich ein vollwertiger Adept des Bewahrerordens werden und mitbestimmen dürfen, welche Berichte ich verfolgen will. Darf ich dich dann aufsuchen und mehr erfahren, eine Lücke unserer Legenden schließen?“

„Wenn du mich dann noch findest, dann sicher.“

„Wo dürfte ich dich denn erwarten?“

„Wenn ich das wüsste. Meine Zukunft ist ungewiss. Cavern. Andor. Irgendwo, wo man mich brauchen kann.“

„Das engt es nicht wirklich ein.“

„Das ist wohl wahr. Vielleicht halte ich einfach nach dir Ausschau, wenn es mich in Zukunft wieder an den Baum der Lieder lenken sollte.“

Die beiden wurden vom emsigen Geschehen auf der Lichtung abgelenkt.

Eine krächzende Stimme tönte über das Stimmengewirr hinaus. „Trinkt das. Der Schleim mag euch grausen, im Nachhinein werdet ihr mir aber danken. Nein, Larissa, jetzt vergeuden wir unsere Zeit sicher nicht damit, übers Geld zu sprechen. Die Verletzten brauchen eine Heilerin.“ Die Urheberin dieser Worte, eine grau gewandte Bucklige, eilte im Dorf umher und verteilte orange leuchtende Tränke, die gegen die Hitze helfen sollen. Auch wenn sie scheußlich schmeckten und ihre Konsistenz an schleimigen Schaum erinnerte, sollte man sie so rasch wie möglich runterschlucken. Eine Bewahrerin in einem langen weißen Gewand – Heilerin Larissa – eilte der Alten hinterher und zischte etwas über zu hohe Preise.

Dann erklang Geschrei aus dem Unterholz. Ein Flötenspieler, dessen verschmutzte Kleidung mit einem hübschen Rautenmuster überdeckt war, unterhielt sich ein wenig abseits im Unterholz aufgebracht mit einer blauhäutigen Kapuzengestalt mit Augenbinde. Gerade intonierte der mutmaßliche Barde: „... lügnerischer Habicht, du, der mir die Sonne vom Himmel zu holen versprach, wenn ich nur deinen Wünschen Folge leiste. Worin auch immer deine weiteren Pläne bestehen mögen, zähle mich absent von ihnen, du kleingeistiger Scharlatan niederster Sorte!“

Die erheblich leiser gesprochene Antwort des Blinden entging Iril, denn in diesem Augenblick kehrte auch schon wieder die von Tion ausgesandte Bewahrerin zurück, ein weiteres Ordensmitglied im Schlepptau. Jenes versuchte vergeblich, eine goldene Mitte zu finden zwischen dem möglichst raschen Laufen zu Tion und der Schonung eines jungen Falken in seinen Armen.

„Keine Neuigkeit von der Rietburg. Gända meldet sich nicht. Tapta meint, dass der einzige kürzlich eingetroffene Falke aus dem Osten kam und keine Nachricht, sondern bloß einen angeknacksten Flügel mit sich mitbrachte.“

„So ist es“, bestätigte Tapta, weiterhin den angeschlagenen Falken stützend, „Keine neue Nachricht über den Drachenangriff oder dieses Erdbeben. Keine Informationen von Gända. Immerhin hat der Ausguck seit letzterem keinen fliegenden Drachen mehr erblickt. Doch muss das nichts heißen. Wir können mit den besten Fernrohren nicht ins Rietland blicken. Der Nebel steht heute hoch über den Wipfeln des Wachsamen Waldes. Es könnte weiterhin weise zu sein, unsere wichtigsten Schriften in Sicherheit zu bringen.“

Dann fiel sein Blick auf Giftknödel, und er führte an: „Aber zumindest steht dein Feigenbaum noch, Phlegon. Kein Funken hat ihn erreicht!“

„Immerhin so viel“, entspannte sich Giftknödel, „Dann darf ich wohl annehmen, dass es um unsere Familie und Freunde auch nicht schlechter steht?“

Tapta nickte. Die zwei jungen Bewahrer schienen weitersprechen zu wollen, doch verstummten sie.

Stille breitete sich allgemein auf der Lichtung vor dem Baum der Lieder aus, als eine herrische Gestalt durch die Portale des Baums trat.

Gekleidet war der Priester in ein edles weißes Gewand mit goldenen Verzierungen, welches einige Rußflecken trug. Die langen braunen Haare trug er offen und waren wild zerzaust. Nichtsdestotrotz strahlte er eine ehrwürdige Aura aus.

Iril hatte schon von ihm gehört.

Der Oberste Priester Melkart.

Der Anführer des Bewahrerordens ließ eine Versammlung einberufen. Als erstes machte er sich daran, zwei Mitglieder seines Ordens zu finden, welche mutig genug waren, ins möglicherweise drachenverseuchte Rietland aufzubrechen.

Eigentlich verließen die Bewahrer so selten wie möglich den grünen Radius, sondern warteten darauf, dass Besucher aus der Umgebung ihre Berichte hierherbrachten. Dies war eine Ausnahme. Ein Drachenangriff war ein unglaublich ungewöhnliches Ereignis. Wieder einmal hatte die Geschichte Andors einen Wendepunkt erreicht. Und es war die Aufgabe der Bewahrer, diese Geschehnisse niederzuschreiben. Damit große Geschichten nicht zu Verschollenen Legenden wurde. Melkart wollte das Risiko nicht eingehen, Berichte aus erster Hand zu verpassen, weil Verletzte bald aus dieser Welt scheiden könnten.

So erlaubte der Oberste Priester zwei jungen Bewahrerinnen, zwei fortgeschrittenen Adeptinnen des Bewahrerordens namens Sanja und Jorna, den grünen Radius zu verlassen. Aufgeregt reisten die beiden bald los. Sie sollten in den Westen aufbrechen und herausfinden, ob der Drache tatsächlich gefallen war. Falls dem so sei, sollen sie Berichte der Überlebenden zu sammeln. Falls nicht ... möge die Mutter ihnen gnädig sein.\bigskip







Sanja und Jornas Ziel, das westliche Rietland, war nicht Irils Ziel. Der Pfad unserer Runenmeisterin führte sie weiter nach Cavern. Dorthin hatte sie eigentlich schon von Beginn an aufbrechen wollen.

Auf dem Weg durch den Wald bis zum nördlichen Mineneingang wurde sie von mehreren Falken überflogen. Die Nachrichtenvögel verbreiteten unzweifelhaft Nachrichten über die aktuelle Lage. Iril wünschte sich, sie zu lesen.

Der nördliche Mineneingang wirkte noch unversehrt. Einzig die große Tanne, deren tief hängende Äste die Pforte verbergen sollte, war ein wenig angekokelt.

Zwei gut gerüstete Zwergenwachen standen davor und schienen ihre Gedanken gerade woanders zu haben. Anstatt ihre Waffen zu heben und Iril nach ihrem Belang fragten, blickten sie wie erstarrt in die Ferne. Der eine, mit einem helleren Hautton, streichelte gedankenverloren eine elegante Eule auf seinem Arm.

„Werte Wachen!“, rief Iril beim Nähertreten, „Mein Name ist Iril von Silberhall, und ich bin hier ...“

„Keine Zeit für lange Vorstellungen. Ich bin Bort“, unterbrach sie der eulenlose Wächter, und zeigte auf seinen eulentragenden Kumpanen, „Und der hier ist Mart. Sprecht, habt Ihr Neuigkeiten vernommen?“

„Ich sah den Drachen aufsteigen und über dem alten Wehrturm niedergehen. Das ist alles.“

Bort ließ seinen Kopf sinken.

„Was ist überhaupt los? Wie ist es möglich, dass nach all dieser Zeit ein Dr...“, setzte Iril an.

„Der König der Andori ist tot“, sprach Mart niedergeschlagen, ohne Iril ausreden zu lassen. „Die Lage ist ernst. Der Prinz von Andor wurde verschleppt. Die Rietburg wurde angegriffen und eingenommen, während die Helden von Andor sich hier in unserer Mine um die Bedürfnisse unseres Fürsten kümmerten. Während die Helden schon wieder den Dunklen Magier aus unserer Mine vertreiben mussten, als wären wir Schildzwerge nicht mehr in der Lage, unsere heiligen Hallen zu Beschützen. Es ist eine Schande. Bei Boords Bart, ich mache mir solche Sorgen.“

„Sei beruhigt, dieser Drache kriegt uns schon nicht klein“, grinste Bort schwach, „Und ganz allein waren die Helden ja auch nicht. Manche unseres Volkes haben bei der Befreiung der Mine und nun auch bei der Befreiung der Rietburg geholfen. Wie schon beim ersten Mal, als die Rietburg erobert wurde. Deine Schwester ist dabei. Brolaf ist ausgerückt. Und natürlich der heroische Kram. Ich hoffe, dass es ihnen gut geht. Nein, ich glaube sogar, dass es ihnen gut geht.“

Mart nickte bedrückt und sah alles andere als überzeugt aus. Bort umarmte ihn beruhigend und streichelte Marts Haarschopf. Iril beachteten die beiden nicht weiter. So viel zu den berüchtigten Sicherheitskontrollen der Schildzwerge. Aber angesichts der aktuellen Umstände war das durchaus zu verstehen, dass sie nicht jeden durchreisenden Zwerg filzten.\bigskip







Iril zog es weiter ins Innere der Mine. Mancherorts herrschte Chaos. Zwerge rannten herum, manche kampfbereit, andere im Nachthemd, als hätte sie das Erdbeben direkt aus ihren Schlafgemächern gerissen. Ein übellauniger Zwerg mit rostbraunem Schnurrbart fluchte beim eingestürzten Tiefen Markt über frische Risse an der Stollendecke. Eine stämmige Kriegerin mit einer unpraktisch großen Kampfaxt und golden glänzenden Zwergenstiefeln raste an Iril vorbei, dicht gefolgt von einigen Zwergenkindern, die sie prompt in eine sichere Nische eines Höhlengangs scheuchte. Zwerge rasten hin und her, begutachteten die Schäden des Bebens und stützten Gänge mit komplizierten Metallstützen ab.

Keiner achtete auf Iril.

Einmal trampelte gar eine Gruppe Gors an ihr vorbei, ohne sich um sie zu kümmern, weitergescheucht durch einen Skral mit einer Augenklappe. Iril langte nach ihrem Hammer, doch der Skral warf ihr kaum mehr als einen raschen Blick zu, ehe er etwas in seiner Sprache fluchte und wieder davonsprang. Nichts außer der Gestank nach Blut und Fäulnis verriet, dass sie hier gewesen waren.

Eine elegante Eule überholte Iril auf ihrem Weg tiefer in die Minen. An ihrem Bein hing eine Nachricht. Neue Neuigkeiten der Wächter Bort und Mart? Ob der Drache wohl wirklich gefallen war?

Iril führte noch einige Handvoll getrockneter Algen aus der Zucht Silberhalls in ihrer zum Glück wassersicheren Reisetasche mit sich. Sie pflanzte sich in eine steinerne Nische, belegte ein Stück würziges Silberbrot mit den Algen und bedachte mampfend ihre nächsten Schritte, während sie ihren krampfenden Bauch zu ignorieren versuchte.

Ihr ursprünglicher Plan hatte schlicht darin bestanden, nach Cavern zurückzukehren, den Kontakt mit ihren Freunden oder ihrer Familie wieder aufzunehmen und dann weiterzugucken, was aus ihrem weiteren Leben werden sollte. Viele Geheimnisse der Runen hatte sie im Norden gelüftet und die erworbenen Fähigkeiten in Silberhall bereits zur Genüge einsetzen können. Doch hatte sie es als moralische Pflicht empfunden, nicht nur ihre außergewöhnlichen Kenntnisse zu mehren und die silberzwergischen Kenntnisse über die Runenmagie weiterzubringen. Nun war die Zeit für einen neuen Lebensabschnitt gekommen. Einen Neuanfang. Nun, wo Irils Runenmeisterin verstorben war und ihr ihren unbezahlbaren Runenhammer vermacht hatte. Nun, wo Iril vernommen hatte, wie in Cavern unzählige Kreaturen ihr Unheil suchten und sich ein Dunkler Magier hier versteckt hatte – gar zweimal in dem letzten Halbdutzend Jahren – hatte sie gedacht, dass ihre Hilfe hier nötiger war als im Norden. Diese Meinung hatten nicht viele geteilt. Und so hatte sie sich schweren Herzens von ihren Freunden bei den Silberzwergen getrennt und war auf Lunors Schiff, der BALENA, in den Süden gestochen.

Jetzt, wo sie hier war, schien es gar möglich, dass die Bewohner des Rietlands ihre Hilfe nötiger hatten als die Schildzwerge. Doch noch bestand ihre erste Priorität darin, Bekannte zu finden. Und dies stellte sich als schwerer hinaus als gedacht.

Iril suchte die Behausungen ihrer Familie nahe der Tiefminen auf. Schon dies kostete sie einige Zeit, denn sie war das Orientieren in diesen teils engen, verwinkelten Gängen nicht mehr gewohnt. In Silberhall waren die Gänge breit, rechtwinklig und blankpoliert. Hier in Cavern, insbesondere vor den Tiefminen, waren die Gänge eng, unförmig und von Ruß überzogen. Auch hier wuselte es nur so von Zwergen, die einander mit Reparaturen aushalfen und die wildesten Gerüchte über die Geschehnisse der Außenwelt herumsprachen. Niemand erkannte Iril, ja, niemand schien überhaupt ihren Namen oder die ihrer Familie zu erkennen. Oder sie wollten sich gerade einfach nicht die Mühe machen, darüber nachzudenken, wo es dringlichere Angelegenheiten hab.

Iril bedauerte wieder einmal ihre Entscheidung, mit ihrer Familie zu brechen. Sie hatten einander nicht mehr viel zu sagen gehabt nach ihren vielen Streiten über Silberhall und die Zukunft des Zwergenreichs. Was hatten sie auch so stur sein müssen! Iril war kein Kind mehr gewesen. Sie hatte ein Recht darauf gehabt, ihren eigenen Pfad einzuschlagen. Die Tiefminen hinter sich zu lassen.

Und nun war sie wieder zurück. Iril hatte die Höhle erreicht, in der sie ihre Kindheit lang gelebt hatte. Der Eingang war von einer schweren Steintür verschlossen, die mit massiven Metallbeschlägen verziert war. Dank eines ausgeklügelten Systems konnte man die schwere Tür dennoch ganz sachte beiseiteschieben. Sofern man den richtigen Schlüssel besaß. Iril tat das nicht. Sanft fuhr sie mit ihrer Hand über die rußige Tür und schluckte einen Kloß in ihrer Kehle herunter.

„Verlaufen?“, fragte eine heisere Stimme hinter ihr. Die Stimme gehörte zu einem Zwerg mit angegrautem blondem Haar, der achtsam einen großen Kessel mit irgendeiner köstlich dampfenden Suppe auf den Boden stellte. Er trug einen Schulterpanzer mit einem schwarzen Zwergenseil darüber. Dies wies ihn als Tiefminen-Arbeiter aus. Er war einer der Entbehrlichen, die für die hohen Fürsten Edelsteine aus den feurig heißen Untiefen der Erde bergen „durften“. Kam ja nicht in Frage, dass ein hochwohlgeborener Schildzwerg mit einer langen Ahnenlinie sein Leben in den brüchigen Lebensadern Caverns aufs Spiel setzte.

Solche Ungerechtigkeiten gab es in Silberhall nicht. Zumindest noch nicht. Die Mine war nicht alt genug, als dass sich Zwerge mit langen Ahnenlinien darauf beruhen könnten und verlangen, dass ihre Untertanen ihnen mehr Reichtum verschafften.

„Hast du dich verlaufen?“, wiederholte der Zwerg etwas lauter.

Iril zuckte zusammen. „Verzeiht, ich wollte nicht stören. Wohnt Ihr hier?“

„Nein, nein, wir wohnen zwei Gänge weiter in Richtung Tiefe“, winkte der Zwerg lächelnd ab, „Aber ich kenne die Familie, die hier haust. Und du siehst ein wenig verloren aus, wenn ich das anmerken darf.“

„Dem ist wohl so. Ich bin Iril. Ich liebte hier einmal, Jahrzehnte ist es her.“

„Drak. Freut mich, deine Bekanntschaft zu machen. Wenn du hier wohntest, dann hast du früher einmal in den Tiefminen gearbeitet, nicht wahr?“

„So ist es. Wie wohl alle aus meiner Ahnenlinie.“

„Na, na, wir Zwerge sind auch erst von irgendwoher nach Cavern gezogen. Sag, bist du vor oder nach der Ankunft der Flüchtigen aus Krahd von hier weggezogen?“

„Nachher. Silberhall gab es bei der Ankunft der Andori ja noch gar nicht.“

„Ah, du bist eine derjenigen, die sich von Silberlands Versprechen verlocken ließen?“ Drak verzog sein runzeliges Gesicht kurz.

„Schuldig im Sinne der Anklage. Doch nahm mein Leben in Silberhall eine andere Richtung, als ich ursprünglich gedacht hatte.“

Irils Hand ruhte auf ihrem Runenhammer. Drak zuckte er mit den Schultern und meinte: „Tja, was geschehen ist, ist geschehen. Wenn du vor der Gründung Silberhalls noch hier lebtest, habe ich dich vermutlich doch schon getroffen, auch wenn du dich nicht mehr daran erinnern magst.“

„Ich bin das Kind von Graiah und Perith.“

„Ah! Stimmt, die hatten doch vor langer Zeit eine Tochter, die ausgewandert ist. Warst du das, die uns immer die Stollenwände vollgekritzelt hat?!“ Drak kicherte. Dann wurde er ernst. „Tut mir zutiefst leid, Iril, das mit deiner Familie.“ Drak senkte seinen Blick und stampfte zweimal kurz auf, eine Geste des Respekts.

„Was ist mit meiner Familie?“, fragte Iril argwöhnisch.

„Weißt du das nicht?“

„Ich habe schon seit Jahren nicht mehr von ihnen gehört.“

„Oh. Auweia. Ich bin wohl nicht die beste Person, um dir das zu verraten. Doch ist es oft besser, die schlechten Nachrichten rasch zu überbringen, statt lange um den heißen Brei herumzureden. Willst du, dass ich es dir erzähle?“

Iril nickte, während ein drückender Kloß der Furcht sich in ihrem Innern ausbreitete.

„Na dann ...“, seufzte Drak, „Die Sache mit Iolith hat deine Eltern hart getroffen. Perith wurde vom Fieber der Traurigkeit erwischt und erlag ihm. Graiah mochte sich durchaus gegen den Kummer zu wehren, doch hielt sie es kaum mehr in dieser Höhle aus. Oft zog sie sich von ihren Bekannten zurück und stapfte tagelang ziellos in den verlassenen Gängen umher. Eine Horde Gors überraschte sie. Einen dieser scheußlichen Drachendiener riss sie noch mit ins Reich der Toten. Die restlichen ... nun ... wir fanden nur noch Blutspuren ... und ...“

Drak schüttelte seinen Kopf und schloss: „Ich labere schon wieder zu viel. Es tut mir leid, Iril. Du hast hier keine Familie mehr. Deine Eltern sind beide dahingeschieden. Die Mutter der Erde hat sie zu sich geholt.“ Erneut stampfte Drak zweimal auf.

Iril brachte nicht einmal eine kohärente Antwort zustande. Ihr Herz pochte und sank zugleich in ihre Magengegend, welche wild zu rumoren begann. Ihr wurde übel und schwindelig. Ihre Eltern sollten ihr nicht viel bedeuten, nachdem sie so lange nicht mehr mit ihnen interagiert hatten. Und doch erfüllte sie die Nachricht ihres Todes mit einem Gruseln.

Iril lehnte sich an die Höhlenmauer und versuchte, das Gehörte zu verarbeiten. Gefühle wallten in ihr auf, die sie nicht richtig zuzuordnen vermochte. Bilder von Graiah und Perith, einander anschreiend. Einander umarmend. Sie ins Bettlein bringend und sanft zudeckend. Auch sie hatte der Tod sich geholt. Zuerst Burmrit, und nun ihre Familie. Iril ließ einige Tränen fließen. Nichtsdestotrotz sprach sie fest:

„Was ... was war diese Sache mit Iolith, die alles auslöste? Wer oder was ist Iolith?“

Drak machte große Augen und schluckte.

„Oh ... öhm ... wie lange warst du denn nicht mehr in Kontakt mit Cavern? Iolith ist ... deine Schwester.“

Das war zu viel für Iril. Zum Glück lehnte sie bereits an der Stollenwand, denn jetzt musste sie sich erst einmal hinsetzen und tief durchatmen.\bigskip







Drak war so lieb, die überwältigte Iril zum Abendessen einzuladen. Er bot ihr sogar an, sie am nächsten Tag zum Grab ihrer Eltern zu führen und ihr mehr von Iolith zu erzählen. Aber zunächst einmal knurrte sein Magen. Und er wollte mehr über die wilden Neuigkeiten aus dem Land erfahren. Auf zu seiner Familie.

Darks Wohnraum war wie diejenigen der meisten Tiefminen-Arbeiter grob in die Wand des eines breiten Querstollens gehauen. Seine Familie hatte daraus ein wirklich heimeliges Reich gemacht. Zahlreiche schmucke Laternen hingen von der Decke und erhellten auch die letzte Ecke des Raumes. Grünlich schimmernde Pflanzen waren in eigens dafür geschaffenen Löchern in den Wänden eingelassen und präsentierten ihre vielfarbigen Blüten. Elegante Wandteppiche schmückten den blanken Stein der Wände. Sie zeigten Menschen und Zwerge im und unter dem Gebirge, beim Fördern von Edelsteinen und Zurücktreiben von Kreaturen. Ein Teppich zeigte gar einen schwarzen Drachen, der von einem hochgewachsenen Helden mit einem blauen Schild konfrontiert wurde. Die sternförmigen Runen auf dem Schild kennzeichnete diesen eindeutig als den Sternenschild, den ersten der vier mächtigen Schilde aus uralter Zeit. Warum Drak wohl einen Menschen auf seinem Wandteppich abbildete?

Iril konnte sich nicht lange auf die Verzierung des Wohnraums konzentrieren, ehe sie auch schon zu Tisch gerufen wurde.

Draks Familie war außergewöhnlich groß, besonders für Zwerge. Zehn Kinder drängelten sich neben Iril, Drak und dessen Gemahlin Bairen an einem überlangen Esstisch, der nahtlos in den Steinboden überging. Wobei der Begriff „Kinder“ irreführend sein könnte. Viele der anwesenden Zwerge waren bereits lange erwachen. Groß, stark und rußverschmiert von der Arbeit in den Tiefminen. Einige sahen auch etwas herausgeputzter aus und legten klappernde Rüstungen ab, ehe sie sich zu Tisch begaben. Nur ein, zwei Anwesende waren wahrlich noch Kinder. Der jüngste musste gar noch mit Lätzchen essen und verschmierte vergnügt sowohl seinen eigenen kleinen Bart als auch das Gesicht einer älteren Schwester, die neben ihm saß, mit dem Essen, das eigentlich in seinen Magen gehörte.

Zu essen gab es den köstlichsten Waldpilz-Eintopf, den Iril je die Ehre hatte zu verspeisen. Doch konnte sie sich irgendwie nicht darauf konzentrieren. Zu sehr kehrten ihre Gedanken immer wieder zu dem zurück, was Drak ihr erzählt hatte. Wie ein Geschwür schlichen sich die Gedanken zu ihrer verlassenen Familie in ihren Geist und hinderten sie daran, das geniale Gericht zur Gänze zu genießen.

Sie hatte eine Schwester. Iolith hieß sie. Geboren war sie offenbar kurz, nachdem Iril nach Silberhall gezogen war. Niemand wusste, wo sie sich im Moment aufhielt. Iolith hatte sich ähnlich wie Iril mit ihren Eltern zerstritten und ihren eigenen Weg gesucht. Scheinbar hatte sie sich im Grauen Gebirge in eine Agren verliebt, welche die Schildzwerge und ihre unterirdischen Grabungen als Schandtaten an der Natur ansah. Ganz ungeachtet dessen, wie sehr die Zwerge bei der Erschaffung ihrer berühmten Bauten stets bemüht waren, im Einklang mit den natürlichen Felsformationen zu sein, wie zu Stein gewordene Echos der Natur.

Graiah und Perith hatten ihre zweite Tochter nicht auch noch verlieren wollen, und leider dadurch umso mehr davongetrieben. Eines Tages war an Ioliths Stelle ein Abschiedsbrief auf ihrem Kopfkissen gelegen. Keiner wusste genau, wo sie nun war. Keiner wollte es mehr so genau wissen.

Ein weiteres Geheimnis, das Iril eines Tages ergründen würde.

Sie liebte schließlich Geheimnisse.\bigskip







Mitten ins Abendessen platzte eine aufgeregte Zwergin, welche ein aufgerissenes Stück Pergament schwang. Sie blieb verheißungsvoll grinsend im Eingang stehen.

„Habt ihr es schon gehört?“, ertönte ihre schallende Stimme. Alle Augen richteten sich auf sie.

„Spann uns nicht auf die Folter, Marun!“, rief einer der vielen Brüder und schlabberte seine Schüssel Waldpilzeintopf mit einem Schluck leer.

„Ein Goldstück darauf, dass der Drache tot ist!“, warf eine Schwester ein.

„Wie geht es Kram?“, fragte Drak angespannt. „Wie steht es um meinen Jungen?“

Iril hatte den Namen Kram schon gehört und konnte ihn nach einiger Überlegung einordnen. Ein Zwerg aus den Tiefminen und Held von Andor. Einer derjenigen, die den Dunklen Magier Varkur in den Tiefen der Mine aufgestöbert hatten, und somit einer der Gründe, weswegen Iril überhaupt erst hierher aufgebrochen war. Doch erst jetzt machte sie die langsam die Verbindung zwischen dieser Familie und Kram. Welch Zufall, dass sie ausgerechnet hier gelandet war. Oder war es etwa mehr als das? Konnte es sein, dass dieselbe Hilfsbereitschaft, mit der Draks Familie Iril zu sich eingeladen hatte, auch dieselbe war, weswegen Kram einer der wenigen Schildzwerge war, die sich um das Schicksal der Menschen kümmerten? Dank der er überhaupt zu einem Helden von Andor geworden war?

Einen Augenblick lang herrschte noch Stille. Dann wurden Irils Gedanken unterbrochen, als Marun freudig herausplatzte: „Der Drache ist tot! Kram hat ihn erlegt! Und gleich noch den verschleppten Prinzen gerettet! Heldenhaft. Es geht ihm gut. Sie feiern draußen im Rietland.“

Einen weiteren Augenblick lang herrschte wieder Stille. Dann brach der gesamte Tisch in Jubel aus. Die Anwesenden klopften einander auf den Rücken, jubelten, und prosteten einander mit Rachenputzer zu.

Schon leicht angetrunken erhob sich Draks Gemahlin Bairen und rief: „Ein Hoch auf unseren Sohn! Auf dass er bald wieder gesund zu uns zurückkehrt und uns von dieser Heldentat berichten kann!“. Marun rannte weiter, wohl, um die nächste Wohnung über die fröhliche Neuigkeit zu informieren. Drak rannte indes in den Lagerraum und holte ein Holzfass heraus, welches mit drei X angeschrieben war.

„Drachenfass Rachenputzer. Enorm starker Zwergenschnaps. Das trinken wir sonst nur zu Geburten und Beerdigungen. Doch welch bessere Gelegenheit gäbe es, das hier anzuzapfen, als einen Sieg über einen verfluchten Drachen zu feiern! Möge er der letzte gewesen sein!“

Erneut ertönte allgemeiner Jubel. Iril stimmte mit ein. Doch fühlte sie auch ein Stechen in ihrem Herzen. So fühlte es sich in einer Familie an. Und eine Familie hatte sie hier nicht mehr. Cavern war nicht mehr ihr Zuhause.






\newpage
\section{Die Spur der Drachenkultisten}

Nach dem feierlichen Abendessen verteilten sich die Anwesenden wieder auf die umliegenden Gänge, um sich weiter umzusehen, wo Hilfe vonnöten war. Glücklicherweise hatte das Erdbeben keinen direkten Stolleneinsturz verursacht. Die wenigen kritischen Ritze wurden von Baumeistern rasch gestützt und geflickt. Verletzte gab es kaum welche. Die ersten Invaliden von der Befreiung der Rietburg und dem Kampf gegen den Drachen würden erst in einigen Tagen hier eintreffen. Und so nutzten die Schildzwerge ihre Zeit zum vollen aus.

Ein unbeschreibliches Fest folgte auf den Sieg über den Drachen. Erwachsene tanzten wild in den Stollen und die Kinder lachten und sangen. Die letzten Tage hatte sie alle viel gekostet. Doch nun wurde gefeiert!

Met wurde ausgeschenkt, guter Kuchen aus den Vorratsstollen geholt, Festbänke aufgestellt und komplexe alchemistische Funkenkörper durch den Schlauchgang Barathrum in die Höhe geschossen, wo sie in verschiedenste farbenfrohe Muster explodierten.

Iril bedanke sich ausgiebig bei Draks Familie für deren Gastfreundschaft und zog weiter. Sie glaubte inzwischen kaum mehr daran, sich wieder mit alten Freunden aus längst vergangenen Zeiten auszutauschen.

So zog es sie an die verlasseneren Orte Caverns. Sie hatte Feste noch nie die famoseste Freizeitbeschäftigung gefunden.

Neugierde war aber ihr Ding. Und als sich eine in einen dunklen Mantel gekleidete Zwergin geheimnisvoll an der Ecke vorbeischlich, in welche Iril sich verkrochen hatte, keimte Irils unbändigbare Neugierde wieder auf. Sie liebte schließlich Geheimnisse.

Die Zwergin mit dem dunklen Mantel blickte sich mehrmals um, als ob sie fürchtete, beobachtet zu werden. In ihrer Hand hielt sie einen rötlichen Edelstein, um den jemand ein kompliziertes metallenes Gerüst geformt hatte. Solche Schätze sah man in den Tiefminen selten, und noch seltener wurden sie über legale Händel erworben. Eine Diebin, die die Gunst der Stunde und das Chaos der Feierlichkeiten ausnutzte? Iril sah ein Glimmen in den dunklen Augen der Fremden und kniff schnell die eigenen zusammen. Doch schien die mysteriöse Zwergin Iril im Schatten nicht zu erkennen und schlich weiter den Gang entlang.

Kurz debattierte Iril die Moral davon, jemandem zu folgen, der offensichtlich nicht verfolgt werden wollte. Die Chance, dass es hier ein Unrecht aufzudecken gab. Die potenziellen Gefahren, die am Ende des Gangs auf sie lauern könnten. Und das unwiderstehliche Drängen ihrer Neugierde, die kein Geheimnis unerforscht lassen wollen.

Rasch war Iril auf den Beinen, zückte ihren – aktuell nicht leuchtenden – Runenhammer und huschte der mysteriösen Zwergin leise hinterher. Hinter ihr donnerten die Festmusik der Zwerge, der Klang dutzender blecherner Blasinstrumente dutzendfach verstärkt durch das Echo der dünnen Gänge. Allzu große Sorgen, von der mysteriösen Kuttenträgerin gehört zu werden, musste sie sich nicht machen.

He, diese Gänge kannte sie ja, die mündeten wieder in die nördlichen Oberhallen Caverns!

Die Zwergin im dunklen Mantel eilte an der prunkvollen Halle der vier Schilde vorbei und zwischen zwei prunkvollen Säulen hindurch, bis sie vor einer protzigen, mit Blattgold verkleideten Tür stehen blieb, durch die selbst ein Troll gepasst hätte. Sie läutete an einem goldenen Glöckchen und wartete, bis die Tür sich, von unsichtbaren Mechanismen gezogen, von selbst öffnete.

Rhythmisches Hämmern erklang dahinter, und rötliche Schimmer gleißten hervor. Rauch lag in der Luft.

Nach einem letzten Umschauen, bei dem der geheimnisvollen Zwergin wieder einmal die sich in die Schatten Caverns drückende Gwundernase Iril entging, verschwand sie hinter der Tür. Hinter ihr verschloss sich das gewaltige Tor wieder wie von selbst. Ein darüber hängender golden verzierter Hammer verriet Iril, was für ein Raum dahinter lag.

Dies war die Schmiede von Hildorf, dem Schmiedemeister, welcher – ganz bescheiden – seine Arbeitsstelle direkt neben der legendären Halle der vier Schilde eingerichtet hatte. Seine Spezialität war die Fertigung magischer Artefakte. Magischer Artefakte wie dem Drachenrelikt, das diese mysteriöse Zwergin in den Händen gehalten hatte. Artefakte aus Roteisen, dem Blutstein-Erz, welches seit Jahrhunderten vom Blut der Drachen durchtränkt war.

Die Macht dieses roten Metalls im Stein war ein Geheimnis. Ein Geheimnis der Tiefminen. Wie alle Geheimnisse, von denen sie gehört hatte, hatte Iril auch dieses einst zu knacken versucht. Erfolglos. Selbst in den Tiefminen beherrschten nur wenige Zwerge dieses Metall. Der Roteisenstein, oder auch Blutstein, wie ihn manche nannten, war von dem roten Erz durchsetzt und wurde von den Schildzwergen oft als Baustoff für das Errichten von Sockeln, Mauern und Türmen verwendet. Doch nicht viele vermochten das Roteisen aus dem Stein herauszulösen und in Form zu gießen. Wenige wussten überhaupt, woher der Roteisenstein seinen Namen hatte. Und so gut wie niemand wusste, wie man damit magische Artefakte schaffen konnte. Hildorf behauptete gerne, dass er der Einzige sei, der diese Drachenmagie handhaben konnte. Es stimmte zumindest, dass kaum jemand seiner Schmiedekunst das Wasser reichen konnte. Nicht, seitdem der legendäre Schmiedemeister Xoll Hammeraxt, dessen Ruf sogar zu Iril nach Silberhall gedrungen war, das Zeitliche gesegnet hatte.

Ein wütender Aufruf erklang aus dem Innern von Hildorfs Schmiede.

Iril konnte sich nicht davon abhalten, näherzutreten und zu lauschen.

„Der Drache ist tot?!“ Eine tiefe Stimme, vermutlich Hildorf selbst. Wut und Unglauben schwang in seinem Tonfall mit.

„Die Nachricht war eindeutig. Tarok der Gewaltige ist gefallen“, antwortete eine zweite Stimme, vermutlich die der mysteriösen Zwergin.

„Wer hat es gewagt?!“

„Die Helden von Andor, so sagt man.“

„Welch Unrecht! Tarok der Gewaltige wollte doch nur ...“

„Egal. Wir müssen rasch handeln! Ehe sie seinen Leichnam schänden können.“

„Natürlich. Was braucht Ihr, Serafimma? Weitere Drachenrelikte, um den Uralten mehr Beseelte zu verschaffen? Mehr Geisterfeuerzeuge für die Dunklen Messen? Zusätzliche Waffen für den kommenden Konflikt? Rüstungen, die Feuer und Pfeil gleichermaßen fernhalten?“

„Nichts von alledem, Hildorf. Wir brauchen dich. Die Drachenkultisten scharen so viele Anhänger wie möglich aus und ziehen in den Westen, zum Leichnam. Mit genügend Anhängern sind wir eine bedrohliche Präsenz. Dort nützt du uns mehr, als wenn du dich wieder tagelang hier einpferchst und weitere Artefakte herstellst, die wir – seien wir ehrlich – nicht mehr wirklich nötig haben.“

Nun klang Hildorf widerspenstig. „Ich bin doch nur ein Schmied. Ein guter zweifelsohne, doch nicht mehr. Ich kann keinen Kriegshammer führen.“

„Mit etwas Glück wirst du das auch nicht müssen. Wir wollen nur genügend viele sein, dass man uns nicht ignorieren kann. Los jetzt, pack‘ deine Sachen.“

„Aber ... meine Gehilfin ...“

„Branna wird einige Tage lang ohne dich auskommen können. Auf, mein Freund, auf mit dir! Schamanin Sagramak vom Barbaren-Stamm der Jpaxo wartet bereits vor dem südlichen Tor. Als von Taroks nächstem Verwandten Beseelte steht es ihr zu, das Ritual zu vollziehen.“

„Ich weiß, ich weiß. Du vergisst, dass ich es war, der überhaupt erst das Drachenrelikt schuf, welches nun von Sagraks Seele erfüllt ist.“

„Na bitte! Die Drachen verdanken dir bereits so viel. Da ist es doch ein Klacks, für sie einige Schritte ins Rietland zu tun.“

Hildorf grummelte etwas, das nach verdächtig wie „Einige würden sagen, mein Dienst an die Flammen wäre bereits erfüllt“ klang.

„Was war das?!“, antwortete die Zwergin – Serafimma – schnippisch, „Wenn du willst, kannst du es ja mit den Drachen aushandeln, wenn sie hinter der Pforte des Todes über dich urteilen. Ich weiß, was ich an deiner Stelle tun würde.“

Hildorf grummelte weiter vor sich hin. Den Geräuschen nach machte er sich aber auch zeitgleich daran, seine Sachen zu packen.

Iril duckte sich in den Schatten eines Erkers im Gang, während Serafimma die Schmiede verließ und erneut an ihr vorbeischlich, diesmal in Begleitung von Hildorf.

Der Meisterschmied trug einen gewaltigen Rucksack, an dem einige grünlich schimmernde Tränke klirrten. Anstelle einer linken Hand befand sich dort eine eiserne Prothese, an der eine eckige Laterne befestigt war. In seinen für zwergische Verhältnisse äußerst kurzen, hellbraunen Bart waren Dutzende kleine und zweifellos unverschämt kostbare Edelsteine eingeflochten.

Irils Gedanken rasten. Drachenkultisten! Anhänger der kriegerischen Bestien aus den Urzeiten. Hier, mitten im Herzen von Cavern! Und Schmiedemeister Hildorf war einer von ihnen!

Iril hatte Gerüchte über die Verbreitung dieses Kults von Drachenanbetern gehört, diese jedoch nicht glauben wollen. Hatten sie denn alle die grausamen Taten der Drachen während des Unterirdischen Kriegs vergessen?!

Ihre Gedanken ratterten. Was sollte sie tun? Gab es Obrigkeiten, die man in solchen Fällen informieren sollte? Nein, leben und glauben sollte ja jeder Zwerg, was er wollte, solange er damit niemandem schadete. Und die Drachenkultisten schadeten niemandem mit ihrer Existenz. Doch schienen sie irgendetwas mit Taroks Leichnam zu planen. Iril musste mehr herausfinden.

Sie schlich Hildorf und Serafimma nach. Inzwischen waren die beiden schon so weit fortgeschritten, dass Iril ihnen nicht mehr strikt folgen konnte. Doch hatte sie belauscht, dass eine Schamanin der Barbaren vor dem südlichen Ausgang wartete. Und selbst so betrunken, wie die meisten restlichen Zwerge waren, konnte Iril hin und wieder jemandem finden, der ihr den schnellsten Weg zum südlichen Ausgang weisen konnte.\bigskip







Bis Iril den südlichen Ausgang Caverns erreichte, hatten die meisten Feierlichkeiten längst ein Ende gefunden.

Am Minenportal stieß Iril auf zwei Zwergenwachen, welche mürrisch aufs Rietland blickten. Es herrschte Nacht. Der Mond war nicht zu erkennen, musste er sich doch gerade auf der anderen Seite des Erdenkörpers aufhalten. Doch nicht nur die Fackeln aus der Tiefe und die Sterne am Himmel spendeten Licht. Weit draußen im Rietland auf der anderen Seite der Narne leuchteten weitere Fackeln und Lagerfeuer. Feierten die Andori?

Ein weiteres Licht fiel Iril auf. Ein bisschen weiter inland, doch noch immer vor der Marktbrücke, beleuchtete ein flackerndes Lagerfeuer eine mächtige einsame Eiche. Die legendäre Zwergeneiche von Fürstin Brala.

Iril hinterließ bei den Wachen am südlichen Ausgang eine knappe Nachricht an Draks Familie, in der sie sie über die Lage aufklärte. Zudem kaufte sie sich einen Falken.

Dann machte Iril sich auf in die Dunkelheit und auf zur Zwergeneiche, wo sie Hildorf und die Drachenkultisten vermutete. In genügender Entfernung blieb sie hinter einem Hügel liegen und spähte zur Zwergeneiche.

Wie erwartet, erkannte sie einige aufgespannte Zelte und einige finstere Gestalten um ein Lagerfeuer sitzen. Zwerge und Menschen und sogar einen Skral. Am Rande der Gruppe, nur noch schwach beleuchtet, saß ein gewaltiger rötlicher Raubvogel zusammengesunken auf einem Stein, den Kopf unter einen riesigen Flügel gesteckt. Hildorf, den Schmiedemeister, erkannte Iril am genau anderen Ende des Lagerfeuers, so weit wie möglich vom Raubvogel entfernt, wie er am Boden saß und mit Serafimma ein Kartenspiel spielte. Die meisten anderen sichtbaren Anwesenden schliefen auf behelfsmäßigen Betten. In den Zelten hielten sich vermutlich noch mehr Drachenkultisten auf.

Das mussten über ein Dutzend Personen sein, vielleicht sogar zwei. Iril drehte sich der Magen um. Was hatten sie vor?! Sie sandte ihren braven Falken mit einer entsprechenden Notiz zurück an die Wachen am südlichen Eingang – wohl wissend, dass diese auch kaum etwas damit anfangen könnten – und machte sich daran, sich ein provisorisches Nachtlager zu bereiten.

Und Iril fiel in einen erschöpften, traumlosen Schlaf.\bigskip







Iril wurde von einem kurzen Schrei geweckt. Hastig richtete sie sich auf, während die Erinnerungen an die vergangenen Tage langsam auf sie einprasselten. Ihr Falke war zurückgekehrt und ruhte neben ihrem dürftigen Kissen, auf einem Bein, den Kopf unnatürlich verdreht und in sein Gefieder gesteckt. Dem Falken schien dies nichts auszumachen.

Der Himmel über ihr erglühte in rotem Licht, als wäre das Blut des gefallenen Drachen in ihn aufgestiegen und hätte dem Licht der aufgehenden Sonne gefärbt.

Mondbeeren glitzerten im vom Tau nassen Rietgras. Der Sonnenaufgang konnte noch nicht lange her sein.

Vorsichtig blinzelte Iril über die Kuppe des Hügels, hinter dem sie geruht hatte. Vor der Zwergeneiche war immer noch eine Horde an Drachenkultisten versammelt. Die meisten Zelte waren bereits abgebaut. Das Lagerfeuer war erloschen. Hastig huschten die Anwesenden umher und machten sich reisefertig. Der Riesenvogel von gestern war nirgends zu sehen.

Dafür erklang ein weiterer kurzer Schrei. Und diesmal fand Iril die Urheberin.

Am Rande des Lagers der Drachenkultisten, im Schatten der Zwergeneiche, saß eine Frau im Schneidersitz. Sie trug eine elegante silberne Rüstung mit vielen Verzierungen. Neben ihr lag eine lange Stangenwaffe mit einer wie ein Schneckenhaus verdrehten Spitze. Wirkte eher zeremoniell als kampftauglich. Und dann erst der prunkvolle silberne Helm, der neben der Frau lag und scheinbar metallene Flügel als Verzierung trug. Nicht zwingend zweckdienlich, doch zweifelsohne imposant.

Iril musterte das ferne Gesicht der Kultistin. Deren Augen waren geschlossen. Ihre Nase sah schief aus, als wäre sie gebrochen. Ihr Mund öffnete und schloss sich, ohne Worte zu bilden.

Da! Erneut schrie sie kurz auf. Diesmal klang es weniger überrascht und mehr verärgert.

Zwei weitere Kultisten saßen neben ihr, ein Mann in einem dunkelblauen Gewand und ein Wesen in einem dunklen Kapuzenmantel, das nicht näher erkennbar war, doch der Form der Kapuze nach lange hohe Hörner trug. Es redete leise auf die Kultistin ein. Iril spitzte ihre Ohren, vernahm jedoch bloß Wortfetzen:

„Welche Wahrnehmung ..., Sagramak? ... sich der Zustand ... Tarok?“

Die Kultistin, Sagramak, schlug flatternd ihre Augen auf und rief:

„Es ist der Prinz! Der elende Prinz beansprucht den Leichnam für sich! Seine Garde umkreist ihn und lässt niemanden hin. Sie wollen ihn zerteilen! Rasch, rasch, macht euch zum Aufbruch bereit!“

Ihren Worten wurde rasch Folge geleistet. Iril packte ebenfalls ihr Schlafzeug zusammen und erledigte ihr morgendliches Geschäft, während sie über die Worte der Drachenkultistin nachsinnierte. Als sie wieder über den Hügel blickte, war der Tross der Drachenanbeter bereits aufgebrochen und daran, die Marktbrücke ins westliche Rietland zu überqueren.

Iril eilte ihnen hinterher. Einmal verharrte sie ängstlich, als sie sah, wie in der Ferne der Riesenvogel von gestern vom Himmel stürzte. Gute Güte, das war ein waschechter Krark! Rotbraune, rasiermesserscharfe Federn. Ein gewaltiger gebogener Schnabel, der selbst in Felsen hineinhacken konnte. Unterarmlange, gebogene Krallen, fast so scharf wie die Schneide eines Schwerts. Iril schauderte es.

Der Krark ließ sich vor Schamanin Sagramak nieder und biss ihr spielerisch in die Nase, während sie seinen Hals streichelte. Hatte sie etwa so ihre Nase vernarbt? Nun, der Krark könnte erklären, wie Sagramak vorhin Geschehnisse im Rietland beobachtet hatte. Konnte sie vielleicht mit diesem Krark eine geistige Verbindung aufbauen und durch seine Augen sehen?

Der Tross der Kultisten reiste tiefer ins Rietland. Nach einer kurzen Wartezeit folgte Iril ihnen verdeckt. Der Krark schien sich nicht groß um seine Umgebung (oder etwaige Verfolger) zu kümmern, sondern stolzierte stolz an Sagramaks Seite. Das sah sehr seltsam aus. Aber Iril hatte es lieber so, als dass er am Himmel seine Runden gedreht und sie möglicherweise erspäht hätte. Mit diesen Bestien war nicht zu spaßen, falls sie der Hunger überkam.

Außerdem erlaubte dies Iril, ihren Falken sorglos erneut mit einer Nachricht an die Schildzwerge zurückzuschicken.

Von der Marktbrücke aus erspähte Iril bereits den gewaltigen schwarzen Drachenkadaver Taroks. Der Drache lag zusammengesunken neben den wieder eingestürzten alten Wehrturm, sein langer Hals verbogen, die durchlöcherten Flügel zerknittert, die Schnauze tief in der matschigen Erde vergraben. Offenbar war der Alte Wehrturm im Todeskampf der Riesenechse schon wieder eingestürzt. Das wievielte Mal war das nun schon, dass der Turm von Grund auf neu hatte erbaut werden müssen? Unglaublich, als läge ein Fluch darauf.

Doch nicht nur der Alte Wehrturm hatte unter den Angriffen von Drachen und Kreaturen der letzten Woche gelitten. Das ganze Rietland wirkte verwüstet.

Brandspuren durchzogen das goldene Rietgras. Letzte Rauchwolken hingen tief über dem Land. Bauernhäuser waren eingestürzt. Der Freie Markt, eine Ansammlung diverser Stände verschiedenster Händler nicht unweit des alten Wehrturms, lag in Trümmern. Die Händler waren geflohen. Dies war nicht nur das Werk des Drachen, sondern auch das seiner dunklen Kreaturen. Sie hatten wirklich alles gegeben für diesen Angriff auf das Rietland.

Iril erkannte einige grün gewandte Soldaten der Andori in der Ferne. Einige trugen Flaggen, auf denen das Wappen Andors wehte. Eine Sternblume auf leuchtend rotem Grund. Dies war die legendäre Rietgarde Andors.

Doch machte sich die Rietgarde nicht daran, die Schäden zu beheben. Nein, sie standen rund um den Leichnam der Riesenechse und richteten lange Speere drohend in Richtung der ankommenden Drachenkultisten.

In ihrer Mitte saß ein hünenhafter Mann auf einem mächtigen Rappen und fuchtelte mit seinem Schwert herum.

Es war, wie Sagramak es gesehen hatte. Prinz Thorald von Andor beanspruchte den Leichnam Taroks für sich und hinderte die Drachenkultisten daran, ihn zu erreichen.

Beim Näherkommen erkannte Iril tiefe Furchen in der Flanke des Leichnams. Offenbar hatte Sagramak auch damit Recht gehabt, dass die Ritter des Königs den Drachen zu zerteilen begonnen hatten. Damit hatten sie aber aufgehört. Nun war ihre Priorität, diesen Zwist zwischen Drachenanbetern und dem Prinzen zu ihren Gunsten aufzulösen.\bigskip







„Zurück! Zurück, befehle ich euch! Ich, euer Prinz!“, rief ein offensichtlich aufgewühlter Prinz Thorald. Die nähergetretene Menge an Drachenkultisten folgte nur widerwillig und breitete sich vor Taroks Leichnam aus, während die Rietgarde sie weiterhin mit Speeren auf Abstand hielten. Es waren augenscheinlich mehr Kultisten als Soldaten anwesend, doch waren die Soldaten besser bewaffnet. Die meisten Kultisten wirkten kaum kampftauglich, eher wie eine zusammengewürfelte Sammlung von Bauern und Jägern, welche nervös und wütend blickten.

Rufe von beiden Seiten wurden laut.

„Mörder!“

„Weichet zurück!“

„Lasst den Körper sein!“

„Lasst doch selbst den Körper sein!“

„Was wollt ihr mit ihm?“

„Zurück, sagte ich!“

Armond, Anführer der Rietgarde, schrie nicht. Er trat einige Schritte vor seine restlichen Krieger und legte demonstrativ seinen Speer nieder. Die Menschen direkt vor ihm blickten nervös zu ihm auf. Armond erhob seine salbungsvolle Stimme über die vor ihm versammelte Menge.

„Gibt es jemanden, der Euch anführt? Jemanden, mit dem man verhandeln könnte?“

„Sehen sie so aus, als wollten sie verhandeln?!“, zischte ihm Prinz Thorald ängstlich zu.

„Mit Verlaub, was für eine andere Möglichkeit gibt es, eine Eskalation zu verhindern?“, murrte ein junger Gardist.

„So einige, Malin, so einige“, sagte Armond, „Doch keine ehrenvollen. Wir sind die Krieger des Königs. Uns bindet ein Eid, für Frieden und Freiheit einzustehen. Wer wären wir, wenn wir ihn so leichtfertig brechen würden?“

„Ich bin eine Anführerin mit Verhandlungsmacht“, meldete eine volltönende Stimme aus der Menge. Die Kultisten wichen zur Seite und machten Platz für ... „Ich bin Sagramak, eine Schamanin der Jpaxo. Ich spreche für die hier Anwesenden. Ebenso wie für die Seelen der Drachen. Ich bin beseelt vom Drachen Sagrak, und als Taroks nächstem Nachkommen steht es ihm zu, über seinen Leichnam zu urteilen. Und damit mir.“

Sagramak trat stolz zu Armond, ihre elegante Rüstung mit jedem Schritt scheppernd. Auf dem Kopf trug sie ihren imposanten Drachenhelm. An ihrer Seite hielt sie ihre lange Stangenwaffe, die sie analog zu Armond demonstrativ zu Boden legte. Als sie sich dem Anführer der Rietgarde näherte, öffnete ihr gewaltiger Krark in der Menge hinter ihr kurz demonstrativ seine Flügel und krächzte in den Himmel. Thorald erbleichte.

Sagramak wandte sich ihm zu: „Mit welchem Recht beansprucht Ihr Mörder den Leichnam des letzten Drachen für euch? Nur, weil ihr der Nachkomme desjenigen seid, der sich vor einigen Jahrzehnten stolz selbst zum König von Andor erklärte?“

„Wagt es nicht, von meinem Vater zu lästern“, zischte Thorald, „Er wurde fair gekrönt und ihr Flussländler freutet euch darüber, dass euch jemand vor den wilden Trollen beschützte!“

„Ich stamme nicht aus den Flusslanden, sondern aus den Gebirgsausläufern“, korrigierte Sagramak, „Und wir hatten keinerlei Probleme mit den Kreaturen des Drachen, bis euer Vater dessen Zorn auf sich zog und sein Wille sie zum Angriff eurer Lager stürmte. Unsere Verbindung zu ihm ist größer als eure. Wir haben ein Anrecht auf seinen Körper.“

Thorald wollte etwas entgegnen, wurde jedoch von Armond zurückgehalten.

Waffenlos traten Armond und Sagramak einander gegenüber, während die Kultisten und Krieger einander weiterhin drohend anblickten. Nicht alle natürlich. Iril sah, wie der junge Mann, der vorhin gesprochen hatte – Malin hieß er – seinen Speer demonstrativ locker an seiner Seite hielt, statt damit in Richtung der Kultisten zu fuchteln. Und in der Menge der Kultisten gab es so einige, darunter den Schmiedemeister Hildorf, die dreinblickten, als würden sie am liebsten im Boden versinken. Keinerlei Kreaturen. Wohin die sich wohl verzogen hatten?

Iril glaubte für einen Augenblick, in der Menge der Drachenkultisten auch eine dunkel gewandte Gestalt mit einer seltsamen Maske zu sehen. Der mysteriöse Drachenreiter mit dem langen Schwert? Hatte er überlebt? Doch als sie wieder blinzelte, war die Gestalt verschwunden.

Sie richtete ihre Aufmerksamkeit wieder aufs Geschehen vorne.

Iril vernahm nicht, was Armond und Sagramak leise miteinander diskutierten, doch verstand sie den grundlegenden Konflikt. Die Drachenkultisten wollten den Leichnam des Drachen, wohl aus religiösen Gründen. Und die Andori wollten ihnen den Drachen nicht überlassen. Nicht, dass der Leichnam ihr Eigentum gewesen wäre oder dass sie grundsätzlich verhandlungsunwillig gegenüber ihnen nicht freundlich gesinnten Völkern wären. Doch waren frisches Drachenblut und Drachenknochen äußerst potente magische Mittel. Und Tarok war ein äußerst mächtiger Drache gewesen, vielleicht der größte und mächtigste von allen. Es war nicht auszudenken, was für Schandtaten ein finsterer Magier damit anrichten könnte. Wer konnte schon garantieren, dass sich unter den Drachenkultisten kein finsterer Magier befand?

Iril konnte den Hass der Andori auf den Drachen gut verstehen. Man sollte seine Schuppen und Zähne zu Staub zermahlen, sein Fleisch verbrennen und seine Knochen ins Meer werfen. Seine Überreste vernichten und verstreuen, damit nie wieder jemand von diesem Dämon heimgesucht werden konnte.

Doch konnte Iril auch verstehen, dass die Drachenkultisten da andere Präferenzen hatten.

Iril sah einen andorischen Ritter heimlichtuerisch einen Nachrichtenfalken lossenden, während Armond und Sagramak weiterverhandelten.

Die Andori schindeten Zeit. Doch wohin wurde der Falke gesandt? könnte kommen, um diesen Konflikt noch zu ihren Gunsten zu drehen?

Eine ganze Menge Personen, wie sich herausstellte.

Als erstes tauchte ein Paar auf, das auf einem großen braunen Pferd mit einem weißen Fleck auf der Schnauze angeritten kam. Die hinten sitzende Person, eine grün gewandte Bogenschützin, sprang bereits elegant vom Reittier, während die vorne sitzende Person, ein blau gewandter Krieger, das Pferd vorsichtig drosselte.

„Was geht hier vor sich?“, verlangte die Bogenschützin zu wissen. Entschlossenheit blitzte aus ihren dunklen grünen Augen auf.

Die beiden Neuankömmlinge traten zu Armond und ließen sich von ihm die Lage erklären. Iril erkannte, dass ihre beiden Umhänge vorne von einer Sternblumen-Brosche zusammengehalten wurden. Das Zeichen des Königshauses Brandur. Dies waren Helden von Andor. Diejenigen, die den Drachen geschlagen hatten.

„Vielleicht habt ihr unsere Namen schon gehört. Ich bin Chada. Das ist Thorn“, rief die junge Bogenschützin nun an die Menge, „Wir stellen uns grundsätzlich gegen alle, die die Freiheit anderer berauben wollen. Doch ist es auch so, dass gewisse Geheimnisse zu groß und zu gefährlich sind, um sie allen zugänglich zu machen. Die Bewahrer wissen das. Und ich weiß, dass Ihr das wisst. Im Namen des Friedens muss ich Euch alle bitten, zurückzutreten. Der Körper dieses Drachen ist ein mächtiges magisches Mittel. Er soll die Narne hinuntergeschwemmt werden und sich im Hadrischen Meer verteilen, wo niemand ihn für finstere Zwecke missbrauchen kann. Selbst er hat verdient, das Ewige Glück zu erreichen.“

„Bockmist!“, rief eine Frau mit einer Zipfelmütze aus der Menge der Kultisten hervor, „Taroks Körper steht uns zu. Ihr, die ihr schon den letzten Avatar der Götter ermordet habt, wollt uns nun nicht einmal die Gnade seiner Bestattung gewährend!“

Chada runzelte ihre Stirn. „Tarok war ein zorniger Mörder, der schon seit Jahrzehnten seine Kreaturen zum Krieg gegen die Unschuldigen dieses Landes aussandte! Seine Kreaturen haben den edlen Brandur getötet! Wie hirnverbrannt ...“

Thorn trat hinter sie und legte ihr besänftigend eine Hand auf die Schulter. Chada verstummte. Dann erhob Thorn selbst seine freundliche Stimme und sprach beruhigend auf die murrende Menge ein. Es wirkte kaum. Doch Thorn ließ sich davon nicht beirren.

Langsam bahnte Iril sich einen Pfad durch die Menge der Kultisten. Wie vermutet, achtete niemand groß auf sie. Keiner der Anwesenden konnte wissen, wer alles Teil dieser Menge war, und wer sich nur hindurchschlich.

Ein Aufschrei ertönte, als Sagramak ihre Stangenwaffe wieder ergriff und in die Menge zurückstolzierte. Offensichtlich hatten ihre Verhandlungen mit Armond wenig gebracht. Wütend watete die Schamanin an Iril durch die Ansammlung der Kultisten. Iril senkte ihren Blick, doch auch Sagramak schien in Iril nichts Unübliches zu sehen.

Mit der Zeit erschienen immer mehr Helden von Andor und stellten sich zwischen die Ritter der Rietgrade und die Kultisten. Manche, wie ein massiger Krieger mit grauem Fellumhang, taten dies eher widerstrebend und mürrisch. Andere, wie eine Zauberin in schlichtem braunem Mantel, taten dies sehr gefasst und ernst. Die Zauberin steckte gar ihren Stab in den Boden und breitete theatralisch ihre Arme aus, während sie einzelne Personen in der Menge anblickte, welche daraufhin oft verschreckt umherblickten.

Einer der andorischen Soldaten schrie auf und stach mit seinem Speer gegen einen Mann in dunkler Kutte, der ihm zu nahe gekommen war. Ein neben ihm stehender Held – ein gehörntes Wesen mit einem noch längeren Speer – legte dem Krieger besänftigend eine Hand auf die Schulter und wies ihn zurück. Dieser zischte eine hastige Antwort, die Iril nicht verstand.

Da sah Iril Hass in den Augen einer Kultistin direkt vor sich aufblitzen. Eine Zwergin in glänzender Rüstung, vielleicht gar Serafimma selbst. Die Kultistin grummelte etwas unter ihrem Atem, bahnte sich einen Weg durch die Menge und schwang auf einmal einen gewaltigen Hammer auf einen jungen Soldaten, der erschrocken zurückwich – und da war Iril plötzlich vorgetreten und knallte ihren eigenen Hammer gegen den der Angreiferin. Der Runenhammer glühte rötlich auf und hinterließ eine nicht unbeachtliche Delle in der Waffe der Zwergin. Diese wich vor Irils entschlossenem Blick zurück. Oder vielleicht auch davor, wie Irils Augen dabei magisch aufleuchteten. Der Soldat – Manus hieß er, wie sich herausstellte – bedankte sich überschwänglich bei Iril.

Nun war Iril auch Teil der Kette zwischen Kultisten und dem toten Drachen. Auweia.

Iril überlegte sich, sich zurück in die Menge zu mischen. Nicht, dass die Helden und Krieger noch dachten, dass sie eine Kultistin wäre, die sich auf den Leichnam zu schleichen versuchte. Dann jedoch blieb sie stehen. Hier konnte sie helfen.

Die ernste Zauberin blickte Iril tief in die Augen. Iril schreckte zurück, als in ihrem Kopf eine leise Stimme ertönte: „Danke. Ich spüre die Tapferkeit in dir. Es ist wichtig, weiteres Blutvergießen zu verhindern.“

Iril hatte noch niemals eine so schöne Stimme gehört, ernst und leise und gleichzeitig so klar und rein. Doch hatte sie nicht die Zeit, sich mehr mit ihrer Urheberin zu befassen. Denn von irgendwoher kam ein Stiefel zu fliegen und traf Iril in die Seite. Sie rieb sich ihre Rippen.

„Wenn noch länger gewartet wird, eskaliert diese Lage! Warum wird der Drache nicht einfach verbrannt?“, fragte Iril.

„Auch ihm steht das ewige Glück zu“, murmelte Chada neben ihr erneut bestimmt.

„Und es ist eine Qual, ein so großes Feuer sicher zu entfachen“, fügte eine Kriegerin Andors blinzelnd hinzu, „Ganz abgesehen davon, dass Drachenfleisch lässt sich nicht so leicht verbrennen lässt. Das bleibt selbst im stärksten Schmiedefeuer roh. Ich habe gehört, jemand wollte es schon einmal für isolierende Rüstungen nutzen. Hat aber nicht viel genutzt.“

Keiner wusste, was mit dem riesigen Leichnam anzufangen sei. Schlussendlich befahl Thorald, zum ursprünglichen Plan zurückzukehren. Taroks Leiche wurde er zerteilt und Stück für Stück die Narne heruntergespült – auf dass selbst diese verbitterte Echse vielleicht das ewige Glück finden könne.

Helden und andorische Krieger wechselten sich ab mit verschiedenen Aufgaben. Und davon gab es wahrlich viele. Aufgebrachte Kultisten friedlich vom Körper zurückzuhalten, den gewaltigen Körper mit langen Sägen zu zerteilen, die blutigen Stücke auf Karren zu laden, die Karren sicher zur Narne zu begleiten, den stinkenden Inhalt dort wieder abzuladen und fortzuspülen ...

Die Wassergeister der Narne erhoben sich immer wieder aus dem Wasser und schleuderten wortlose Flüche gegen die Andori. Prinz Thorald rief ihnen zu, dass, wenn sie nicht zufrieden waren mit Taroks Leichnam in ihrem reißenden Fluss, sie ihn doch lieber schnell ins offene Meer bringen sollten.

Es dauerte lange. Mehrere Tage. Und es benötigte eine Menge an heldenhaften Helfern.

Iril sah den legendären Zwerg Kram aus der Ferne, wie er mit einer eleganten Zwergenaxt messerscharfe Schuppen vom toten Tarok löste und auf Karren zum Abtransport stapelte.

Iril sah einen Menschen mit rotem Haarschopf, breitschultrigen Körperbau und einem Raben auf der Schulter, der einige harsche Worte in einer fremden Sprache mit Sagramak austauschte.

Iril sah eine Hüterin in weißem Gewand, die einen Wassergeist umherdirigierte. Dieser Wassergeist raste in Windeseile umher. Wo immer Spannungen auftraten, war er vor Ort, um Hitzköpfe abzukühlen. Und als der lange Hals des toten Drachen sich auf einmal aufblähte und Lava absonderte, war der Wassergeist zur Stelle, um ein ausbrechendes Feuer zu verhindern.

Gemeinsam konnten die Helden von Andor schier jedes Hindernis überwinden.\bigskip







Nach wenigen Tagen schien die Lage sich größtenteils beruhigt zu haben.

Von Tarok war fast nur noch ein grobes, stinkendes Gerippe übrig, welches nun ebenfalls Stück um Stück abtransportiert wurde.

Die meisten Kultisten waren abgezogen. Einige wenige hatten in der Nähe des alten Wehrturms ein Lager aufgeschlagen, saßen in Kreisen herum und riefen wilde Worte in längst vergessenen Sprachen in den Himmel.

Das Interessanteste, was in den letzten Tagen geschehen war, war eine Gruppe von drei Jugendlichen gewesen, welche sich in die Narne geworfen hatten, um einige Knochen des Drachen herauszufischen. Sie hatten Bekanntschaft mit der reißenden Narne und den willensstarken Wassergeistern gemacht und waren erfolglos wieder herausgefischt worden. Nun saßen sie von Decken umgeben um ein Lagerfeuer und zitterten.

Die Krieger der Rietburg waren erschöpft und freuten sich aufs sich abzeichnende Ende der Knochenarbeit. Prinz Thorald hatte sich schon länger nicht mehr blicken lassen, vermutlich, weil er schon lange wieder in seinem weichen Bett in der Rietburg weilte.

Tarok war von einer imposanten Kreatur zunächst zu einem stinkenden Knochengerippe geworden, dessen letzten Überreste kaum als Teile eines Körpers erkennbar waren.

Iril saß auf einem Pferdekarren, der eine Pranke des Drachen in Richtung Narne transportierte, und unterhielt sich mit Chada und dem schlanken Wachmann Manus. Letzterer ließ sich soeben begeistert darüber aus, wie die Rietgarde jegliche Bedrohung von der Burg abwehren könnte, wenn sie denn nur einen Seher fänden und auf ihre Seite zögen. Chada selbst schien in eigene Gedanken versunken. Iril hatte nur am Rande mitgekriegt, wie sie sich kürzlich mit der Hohen Priesterin Gända gestritten hatte und letztere zurück in den Wachsamen Wald abgedampft war.

Drei Personen Begleitschutz pro Karren. Das war die Regel. Und an der Abladestation an der Narne wurden noch stärkere Sicherheitsvorkehrungen getroffen.

An diesem Nachmittag zeigte sich sogar Prinz Thorald wieder. Stolz galoppierte er auf seinem edlen Rappen an den entfernt voneinander durchs Rietland gezogenen Karren vorbei und sprach den Andori Mut und Tapferkeit zu. Manus unterdrückte sich ein Kichern, als Thorald zum dritten Mal an ihrem Karren vorbeirannte und ihnen Zugang zum besten Tropfen aus seinem Weinkeller versprach.

Die Lage war ruhig.

Zu ruhig.

Die Pferde schnaubten.

Ohne Vorwarnung teilten sich die Wolken am Himmel und ein Schatten stürzte sich auf ihren Karren. Ein gewaltiger Krark mit Flügelspannweite von mehrfacher Mannslänge warf sich auf die Pferde und riss sie beiseite. Zwei in Kutten gekleidete Skrale sprangen aus dem hohen Rietgras auf den Weg und gaben den Reittieren den Rest. Das panische Wiehern erstarb abrupt. Die echsenhaften Kreaturen bleckten ihre Zähne und bedachten die Kutscher mit milchigen Augen.

Der riesige Raubvogel drehte sich zum Kutschbock herum und starrte den drei dort stationierten Personen entgegen. Dann schwang er sich über die beiden Skrale und die drei Kutscher hinweg und grub seine Klauen in die Überreste von Taroks Pranke, die auf dem Karren transportiert wurde. Mit mächtigen Schlägen seiner Schwingen versuchte der Krark, abzuheben.

Manus war als erster auf den Beinen, sprang auf die Drachenpranke und stocherte mit einem Schwert nach dem Krark.

Chada wirbelte so schnell herum, dass ihr langer Zopf beinahe Irils Kopf getroffen hätte, wenn er nur ein klein wenig höher gelegen hätte. Chada legte einen Pfeil in ihren treuen Bogen Audax, während Iril magische Kraft in ihrem Runenhammer sammelte. Iril schleuderte die magische Waffe auf den linken Skral und warf ihn zu Boden. Ein Pfeil surrte an ihr vorbei und bohrte sich tief in die Brust des Skrals. Chada warf Iril ein aufmunterndes Grinsen zu.

Ein menschlicher Schrei ertönte und verklang ebenso abrupt. Iril fuhr herum. Der gekrümmte Schnabel der Krarks hatte sich in Manus‘ Nacken verbissen und dem tapferen Soldaten innert Augenblicken ein blutiges Ende bereitet.

Iril versuchte, sich wieder umzudrehen, sich dem letzten Skral zuzuwenden, doch ihr Körper gehorchte ihr nicht. Wie gebannt starrte sie auf Manus‘ Leichnam, versuchte, zu verarbeiten, was soeben aus diesem freundlichen Gesellen geworden war.

Wie in Zeitlupe nahm sie wahr, dass der Krark seine Flügel ausbreitete, seine Krallen tief in Taroks tote Pranke grub und nun ungestört in den Himmel abhob. Unter ihm schwankte ein fleischiges Stück von Taroks Pranke.

Ein Pfeil drang tief in den Schädel des Krarks ein. Der Krark krächzte und ächzte gequält. Leblos krachte das Vieh wieder auf den Boden zurück und begrub den Karren unter sich.

Pferdegetrappel ertönte. Prinz Thorald und sein edler Rappen waren zur Rettung gekommen. Thoralds Lanze riss den letzten Skral zu Boden. Die Hufe seines Reittiers gaben ihm den Rest.

Chada fand Manus‘ Hand, fühlte nach seinem Puls und ließ enttäuscht ihren Kopf sinken. Bei den Pferden machte sie sich nicht einmal mehr die Mühe. Sie waren offensichtlich aus dieser Welt geschieden.

Chada legte ihre Hand auf Irils Schulter und blickte ihr in die Augen.

„Geht es dir gut? Bist du verletzt?“, fragte ihre helle Stimme.

Gute Güte, sie schien so jung, und doch so unbekümmert ob des soeben Geschehenen. Iril schüttelte ihren Kopf und sortierte ihre Gedanken.

„Alles in Ordnung“, brachte sie hervor. Das genügte Chada fürs Erste. Die Heldin richtete sich auf und überprüfte, dass die beiden Skrale auch wirklich das Zeitliche gesegnet hatten.

Thorald stieg von seinem Pferd ab und warf Chada einen selbstgerechten Blick zu.

„Immer wieder gerne bereit, zu helfen“, sprach er.

Traurig blickte er in die Ferne, sein Mantel im Wind wehend.

„Der arme Manus. Er konnte mir so einiges beibringen in seiner Zeit. Wäre ich nur früher hier gewesen ...“

„... hättest du den Riesenvogel etwa mit deiner Lanze abzustechen versucht?“

Thorald murmelte etwas vor sich hin und fuhr sich durch die Haare. Anschließend sprach er mit Bedacht: „Diese Kultisten hofften, sich dieses Stück von Taroks Leichnam mit Gewalt verschaffen zu können, und dies kostete einer der unseren sein Leben. Das darf nicht sein. Wir lassen uns von ihnen nichts vorschreiben. Uns nicht einschüchtern. Lasst dieses Stück Drachenfleisch in die Rietburg bringen. Wir werden sie in unserer Schatzkammer einschließen. Als Trophäe für den erschlagenen Krark – und den erschlagenen Drachen. Als Andenken an Manus – und alle anderen, die in den letzten Tagen ihr Leben gaben. Und vor allem als Zeichen, dass die Kultisten diese Pranke niemals haben können werden!“

„Was?!“, fragte Chada ungläubig. „Das ist nichts als ein vermoderndes Stück Fleisch, das wir gerade eben loswerden sollten, damit es nicht die falsche Aufmerksamkeit zieht. Hat die Schatzkammer überhaupt Platz dafür?!“

Thorald knurrte: „Ich bin der Regent von Andor. Der zukünftige König. Wenn ich will ...“

„Natürlich, wenn du willst, wird es so sein. Aber was bringt dir das? Diese Tatze ist tot und stinkt. Keiner ...“

„Na gut, na gut“, murmelte Thorald, „Der Geruch könnte tatsächlich ein Problem werden. Lasst uns einfach ...“

Thorald stach mit seinem Dolch in das Fleischstück hinein, rümpfte seine Nase und angelte einige Fußknöchelchen des Drachen hervor. Für Tarok mochten sie winzig gewesen sein, doch war jedes einzelne größer als Thoralds Hand.

„... lasst uns lieber nur einige Drachenknochen mitnehmen. Als Andenken, Trophäe und Zeichen, dass wir uns nicht von den Kultisten einschüchtern lassen.“

„Thorald, bitte, halte ein und überlege dir noch einmal gut, worin du dich hier verrennst ...“

Iril ließ die beiden sein, sprang vom Kutschbock, hielt ihre Nase zu und sammelte ihren Hammer von der Fratze des linken Skrals ein. Sie versuchte, die tote Kreatur nicht allzu genau anzusehen.

Mit Nase und Augen abgewandt, nahm dafür ihr Gehör etwas wahr. Ein Rascheln im Rietgras fiel ihr auf.

Vorsichtig trat Iril nach vorne, hob ihren Hammer, und ...

... blickte tränennassen Augen entgegen. Versteckt im hohen Rietgras lag Schamanin Sagramak. Ihre glänzende Rüstung hatte sie gegen ein Lumpengewand eingetauscht, doch diese gebrochene Nase erkannte Iril problemlos.

Sagramak bewegte ihren Mund, als suche sie nach Worten. Ihre entsetzten Augen schwirrten zwischen der vor ihr stehenden Iril und dem toten Krark hin und her. „Nein ... nein, das hätte alles nicht so ...“

„Hierher!“, rief Iril zu Chada und Thorald.

Hastig rappelte sich Sagramak auf und versuchte, ihre Fassung zu wahren. Eine Waffe zog sie allerdings nicht. Stattdessen jaulte sie: „Dies hätte nicht so kommen sollen. Ihr hättet uns einfach Zugang zu Taroks Körper geben sollen!“

„Und nun, wo ihr es nicht konntet, versuchtet ihr es mit Mord und Totschlag?“, erklang Chadas zornige Stimme.

„Euer Krieger ...“, ächzte Sagramak, „Ich wollte nicht ... die Instinkte des Krarks ... es tut mir so leid.“

„Das kommt davon, wenn man zur Axt statt zum Schild greift“, sprach Thorald geschwollen.

„Wir mussten zu solchen Mitteln greifen!“, rief Sagramak. „Was sollen wir denn sonst tun, um unsere Stimme hörbar zu machen?!“

„Wir hören euch doch schon“, rief Thorald wenig hilfreich, „Aber eure Belange kümmern uns wenig!“

Frustriert fauchte Sagramak auf, wirbelte herum und verschwand überraschend, gar unnatürlich rasch tiefer im hohen Rietgras.

Chada hielt Audax im Anschlag und zielte. Und zögerte.

„Na, mach schon!“, fuhr Thorald sie an, „Schieß!“

Chadas blinzelte unentschlossen, doch ihre Hand blieb ruhig.

Iril dachte zurück an die mit einzelnen Pfeilen getroffenen Skral und Krark. Ihr fuhr es schaurig den Rücken herunter. Sie hatte gehört, dass die Bewahrer ihren Bogenschützen beibrachten, Tiere mit nur einem einzigen Schuss zu erlegen, damit sie keine unnötigen Qualen litten. Man wollte sie lieber nicht zum Gegner haben.

Entschieden ließ Chada ihren Bogen sinken. „Sie ist keine Gefahr. Taroks Körper ist die Gefahr, und was man damit alle für finstere Rituale auslösen könnte.“

Thorald schnaubte auf und blickte zurück zu seinem Pferd, unzweifelhaft kalkulierend, ob die durchs Rietgras davonsprintende Sagramak den Aufwand einer Verfolgung wert war. Schließlich winkte auch er ab. „Die sehen wir so bald nicht wieder. Kümmern wir uns lieber darum, den Karren zu reparieren. Und den Toten einen würdigen Abschied zu bereiten.“

Er blickte Chada und Iril vielsagend an und galoppierte dann einfach davon. Vielleicht kam es ihm nicht einmal in den Sinn, dass er selbst hätte aushelfen können.

Iril schluckte schwer und begann damit, Manus‘ Leichnam vom Karren zu heben.\bigskip







Noch am selben Tag war es vorbei.

Der restliche Abtransport von Taroks Leichnam ging ohne Probleme vonstatten. Eine Priesterin aus der Kapelle der Rietburg trat an die Narne und sprach einige friedliche Worte, während die letzten Überreste der Riesenechse ins Hadrische Meer trieben. Wassergeister plätscherten unverständliche Worte der Warnung gegenüber den Kultisten, die der Narne zu nahe kamen – und gegen die Krieger, die ihren schönen Fluss mit finsteren Knochen, Fleisch und Blut besudelt hatten.

Manus Körper wurde aufgebahrt. Krieger zollten ihm Respekt. Seine Familie bereitete den Totenritus vor.

Soldaten stellten sicher, dass die großen Blutlachen vor dem alten Wehrturm aus dem Rietgras gewaschen waren. Dann traten sie zurück und ließen die trauernden Kultisten nähertreten. Viele von ihnen sanken zu Boden. Manche weinten, manche suchten nach Überresten des Drachen, und sei es nur, um ein Stück einer Schuppe oder eine Phiole verschmutzten Drachenblutes finden.

Die meisten Helden von Andor blickten einander unsicher an. Dies war definitiv nicht so, wie sie sich das Nachspiel eines heroischen Drachenkampfes vorgestellt hatten. Und einige schienen unsicher, ob sie auf der richtigen Seite dieses Konflikts gestanden hatten.

Sei dem, wie es sei, es war vorbei. Taroks Leichnam war fort. Keiner konnte den Körper mehr für etwaige finstere Zwecke nutzen. Die Helden konnten wieder abziehen. Manus‘ Familie konnte um ihn trauern.

Die Hüterin der Flusslande teilte den restlichen Helden mit, dass sie hierbleiben würde. Unter den Drachenkultisten hatten sich auch ein, zwei Flussländer befunden. Mit ihnen würde sie demnächst ein Wörtchen sprechen.

Die restlichen Helden verstreuten sich wieder in alle Himmelsrichtungen.

„Kommst du mit?“, fragte eine tiefe Stimme hinter Iril. Diese erschrak, als sie im Sprechen den berühmten Kram aus den Tiefminen erkannte, fasste sich aber auch wieder rasch.

„Wie meint Ihr?“

„Bitte, du kannst das Du nutzen. Ich bin Kram. Danke für deine Unterstützung. Kommst du nicht auch von Cavern? Ich breche dorthin auf. Du könntest dich mir anschließen.“

Iril fasste sich ein Herz.

„Ich weiß nicht. Ich stamme eigentlich von Silberhall. Ich fühle mich in Cavern nicht mehr wirklich zuhause.“

„Das verstehe ich“, murmelte Kram, „Ich bin mir auch nicht mehr sicher, wo mein Herz mehr liegt. Doch ändert das nichts daran, dass es eine wahre Freude ist, bei meiner Familie in Cavern zu sein.“

„Familie, ja“, schluckte Iril, und dachte an Iolith. „Die habe ich nicht mehr hier. Bitte, richten S ... richte Du einen Gruß an deine Familie aus. Es sind feine Gesellen. Aber in Cavern hält mich nichts mehr.“

Kram kratzte sich am behelmten Kopf, nickte dann aber und wandte sich in Richtung Süden. An seiner Seite stapfte der große Wolfskrieger.

„Na, auch auf zum Trunkenen Troll, Kram?“

„Nee, Papa kommt bestimmt schon um vor Sorge.“

„Dein Pech. Ich gönne mir jetzt mindestens eine Woche Auszeit in Gildas Taverne. Danach können wir uns Gedanken machen über den Wiederaufbau all dessen, das vernichtet wurde.“

Iril sah die beiden Helden abziehen. Sie wusste nicht, wo sie hinsollte. Doch wusste sie, dass es sie nicht nach Cavern zog. Beim Gedanken, dorthin zurückzukehren und vor den Behausungen ihrer toten Familie herumzulungern, drehte sich ihr Magen um. Nein, in Cavern gab es nichts für sie. Zeit, zu schauen, ob sie den Andori helfen konnte.\bigskip







Iril folgte einer Gruppe abziehender andorischer Krieger in Richtung Rietburg. Nachdem sie sich im Debakel mit den Kultisten auf die Seite der Helden gestellt und eine edle Rede über Manus‘ tapferen Einsatz geschwungen hatte, erfuhr sie kaum Misstrauen der Rietgarde. Die Krieger des Königs waren ohnehin größtenteils viel zu erschöpft, um sich groß um sie zu kümmern.

Die Türme der Rietburg ragten hoch in den Himmel. Auch über diesem Gemäuer hing eine dunkle Rauchwolke vom letzten, doch das ewige Feuer vor den hohen Toren flackerte in hellem orange. Ein Zeichen, dass die Gefahr sich gelegt hatte. Dass die Andori in das befreite Gemäuer zurückkehren und sich an die Reparaturen machen konnten.

Iril erlebte am Rande mit, wie Prinz Thorald mit viel Theatralik den Sack mit Taroks letzten Fußknochen in den Thronsaal brachte.

„Der bringt die Knochen jetzt in seine Schatzkammer, wo auch der Bruderschild verstaubt. Einer der vier mächtigen Schilde aus uralter Zeit. Die Helden von Andor fanden ihr vor einigen Jahren wieder. Nun verrottet er jedoch die größte Zeit in Thoralds Prunksaal“, sprach eine tiefe Stimme.

Iril blickte sich um. Neben ihr stand ein Zwerg in voller Plattenrüstung, der sich auf einen langen Hammer stützte und das Geschehen mürrisch beobachtete. Seine kahle Stirn glänze im Sonnenlicht. Sein langer Bart war wohl einst rötlich gewesen, nur jedoch eher braun und grau vor lauter Dreck. Einer der Schildzwerge, die bei der Befreiung der Rietburg mitgeholfen hatte?

Iril, die sehr wohl gewusst hatte, was der Bruderschild war, blieb stumm. Dies nahm der mürrische Zwerg zum Anlass, fortzufahren: „Ich könnte es zumindest respektieren, wenn die Helden den Bruderschild für gute Zwecke einsetzen würden. Wenn er aber ohnehin nur in der Privatsammlung eines Prinzen Staub sammelt, könnte man ihn geradesogut denjenigen zurückgeben, die ihn wahrlich verdient hatten. Den Schildzwergen. Den Nachkommen Kreatoks. Nicht wahr?“

Iril dachte zurück an den Silberschild, denjenigen mächtigen Schild aus Kreatoks und Nehals Sammlung, der schon seit Jahrzehnten in Silberhall Staub sammelte. Sturmschild nannte man ihn auch, da er es seinem Träger erlaubte, sich die Winde untertan zu machen. Einige der darauf zu findenden Runen hatten den Runenmeistern gereicht, um solche Effekte in kleineren Skalen zu replizieren. Doch davor, die Sturmwinde über dem gesamten Hadrischen Meer zu befehligen, wie es Träger des Sturmschilds konnten, träumten die Runenmeister nur. Erst recht, solange Arkteron, der Herr der Stürme, in den wogenden Meereswellen lauerte.

Iril hatte den Silberschild nur einmal aus der Ferne gesehen. Schon seit längerem war er nicht mehr aus der Schatzkammer geholt worden. Die Silberzwerge, die Werftheimer und die Taren hatten Rat gehalten und beschlossen, den Schild lieber nicht zu demonstrativ präsentieren. Nicht, dass die Mächte des Meeres sich durch dieses höchste Gut zwergischer Schmiedekunst provoziert fühlten. Eines Tages würden würdige Träger vielleicht in der Lage sein, den Schild zum Wohle des Nordens einzusetzen. Doch bis dahin hielt man ihn lieber versteckt.

Iril schüttelte ihren Kopf. In die Politik der mächtigen Schilde wollte sie sich nicht einmischen. Die meisten Geheimnisse dieser Schilde waren bereits enthüllt worden. Und ob sich seit Kreatoks Tod je wieder wirklich würdige Träger dafür finden sollten, stand in den Sternen geschrieben.

Der mürrische Zwerg vor Iril schien ihre mangelnde Antwort richtig zu interpretieren und deutete zum Themenwechsel auf ein nahegelegenes Hausdach.

„So hätten wir mit dem Drachen umgehen sollen“, murmelte er.

Es dauerte einen Augenblick, bis Iril erkannte, was ihre Augen wahrnahmen. Ein gewaltiges geschupptes Wesen lag quer über dem Dach des Palas‘, mit einigen Pfeilen quer aus seiner Kehle ragend. Rückenstacheln länger als ein Arm. Hörner länger als ein ausgewachsener Mensch. Ein doppelt so langer Schwanz. Eine dampfende, blau glühende Flüssigkeit tropfte aus einem vielzahnigen Mund. Iril schauderte es bei seinem Anblick. Welch finstere Kreaturen hatte dieser Drache befehligt?!

Leitern waren rund um das Gebäude angebracht worden. Um das gewaltige Wesen standen und saßen viele Andori auf dem Strohdach und entfernten Fleischstücke aus dem Leichnam. Andere reichten die Fleischstücke an den Boden, wo sie auf verschiedenen Feuern landeten.

Iril fiel besonders ein kleiner Wichtel in einem Kapuzenmantel auf, der am Boden stand und mit großen Gesten etwas herumdirigierte. Umso überraschter war Iril, als sie sah, dass auf das Winken und Wedeln des Wichtels ganze Fleischstücke der geschuppten toten Kreatur von grünem Licht erfüllt wurden und sanft zu Boden schwebten. Der Wichtel grinste fröhlich bei der Arbeit. Glitzernder Sand rieselte aus seinen Händen, während er weiter herumzauberte.

„Wunderst du dich über das kleine Männlein?“, sprach der mürrische Zwerg. „Das ist Wrort. Er sagt, er sei aus einem fernen Land angereist. Will aber nicht sagen, woher. Magisch begabte Wesen, diese Wichtel. Mir sind sie nicht ganz geheuer. Und das riesige Viech auf dem Dach nennt man einen Mhourl. Schon lange haben wir keine mehr gesehen. Es ist kein gutes Zeichen, dass sie jetzt wieder auftauchen. Ich bin übrigens Lafgar. Sag, kannst du auch sprechen?“

„Wenn man mich lässt“, grinste Iril und stellte sich vor. Lafgar grinste nicht, begrüßte sie jedoch mit einem Faustschlag.

„Warum wird der Mhourl denn erst am Boden verbrannt? Kann man ihn nicht schon dort oben verbrennen?“, fragte Iril neugierig.

„Nicht, ohne die ganzen Dächer anzuzünden. Mit getrocknetem Rietgras wurden sie gedeckt“, lachte Lafgar kopfschüttelnd, „Als wollte jemand, dass sie in vom erstbesten fliegenden Funken in Brand gesteckt werden.“ Er lachte leise beim Gedanken daran.

„Das wäre ein Feuerfest geworden, wenn der Drache bis zur Rietburg gekommen wäre“, brummelte Lafgar weiter. „Aber das hätte mich nicht gefreut. Diese Rieseneidechse entkam dem Tod schon zu lange. Ein hirnloses Biest war das, dafür müsste niemand das Ewige Glück in der Narne suchen. Wie sehr hätte es mich erfreut, wenn man den gewaltigen Fleischvorrat genutzt hätte und auch noch in zwei Jahren davon hätte zehren können.“

„Drachenfleisch kann Körper und Geist vergiften“, meinte eine fremde Stimme, „Es ist gut, dass wir es so rasch wie möglich beseitigten.“

Iril blickte sich um und erblickte einen weiteren Gesprächigen, einen jungen Andori, der sich zu ihnen gesellt hatte. Er trug bereits die Kleidung eines andorischen Kriegers, doch schien das Schwert an seinem Gürtel eher zu Trainingszwecken.

Lafgar und der Andori tauschten einige kreative Beleidigungen des toten Drachen aus. Iril lachte mit ihnen. Dann erinnerte sie sich und fragte interessiert: „Wer ist eigentlich dieser Reiter des Drachen mit dem langen Schwert? Ich glaube, ihn unter den Drachenkultisten gesehen zu haben.“

„Ein Drachenreiter?! Keine Ahnung“, brummte Lafgar, „Man muss doch verrückt sein, um auf diesen Dingern reiten zu wollen.“

Das Gesicht des jungen Andori hellte hingegen auf. Er sprach zur Begrüßung seinen Namen – Peta – und berichtete dann beflissen: „Ihr habt den Drachenreiter unter den Kultisten gesehen? Das ist bedenklich. Wenn auch kein Wunder. Das ist der Schwarze Herold. Eine Sagengestalt. Aber keine erfundene. Ich habe ihn auch schon gesehen. Er unterstützt alles Böse in Andor. Er treibt die Kreaturen des Drachen an.“

„Heißt das, dass die Drachenkultisten die nächsten Bösen der andorischen Geschichte sind, wenn er nun bei ihnen weilt?“

Peta antwortete: „Fest steht, dass diese Drachenkultisten dem Königreich nichts Gutes wollen. Ehrlich, ich kann nicht verstehen, wie man die Drachen anbeten kann.“

„Vielleicht haben sie Angst statt Ehrfurcht“, überlegte Iril, die an die belauschte Konversation von Hildorf dem Meisterschmied zurückdachte. „Wenn man denkt, dass die Drachen einen nach seinem Tode richten werden, würde man durchaus einiges tun, um ihre Gunst zu gewinnen.“

„Verzeiht ihre Taten noch lange nicht.“

„Der Prinzen ist jedoch auch nicht automatisch im Recht. Der Drachenleichnam ist zu gefährlich, um anderen Zugang dazu zu gewähren, und doch lässt Thorald sich dazu hinreißen, einige Knochen in seine Schatzkammer zu bringen?“

„Unser Prinz wird seine guten Gründe gehabt haben“, sprach Peta.

„Pah“, meldete sich nun der mürrische Lafgar wieder zu Wort, „Als ob der Prinz wüsste, was er tut. Mit der Rietgraskrone auf Thoralds Kopf wird das Leben nirgendwo besser werden. Wir können von Glück reden, wenn er keinen Krieg mit den Drachenkultisten anzettelt. Oder einen Aufstand der Flussländler. Oder die Barbaren wieder vertreibt. Er ist vieles, aber kein Diplomat. Und kein feiner Herrscher.“

„So einen feinen wie Brandur wird es nie wieder geben“, sprach Peta andächtig. Lafgar gluckste ungläubig. Ehe er zu einer potenziellen Tirade über den Landräuber Brandur ausbrechen konnte, mischte sich Iril ein:

„Mit Verlaub. Brandur war nur ein König. Ein guter König vielleicht, aber immer noch nur ein König. Ein Mensch, der Fehler macht. Sein Sohn wird auch Fehler machen. Aber er wird dieselben Berater haben, dieselbe Burg, dieselben tapferen Krieger an seiner Seite. Worüber macht ihr euch solche Sorgen?

Peta antwortete leise: „Oh, ihr kennt Thorald noch nicht. Er ist ein ausgezeichneter Reiter und gut im Umgang mit der Lanze, aber das ist auch schon alles. Und ihr versteht nicht, wie beliebt König Brandur hier war. Brandur hat unsere Großeltern im Alleingang als Jugendlicher aus der Sklaverei ins Freie geführt. Todesmutig verschaffte er allein seinem Gefolge einen Weg am Drachen Tarok vorbei. Später verteidigte er die hier Angekommenen, die Andori, in den Trollkriegen immer und immer wieder aufs Neue. Er ist ein Held. Manche sehen ihn als Geschenk von Mutter Natur höchstpersönlich. Schon als kleiner Junge träumte ich davon, einer seiner Krieger zu werden. Als ich ihm das erste Mal gegenüberstand, dachte ich, mein Herz schlüge aus meiner Brust hinaus vor Aufregung. Manche dachten, Brandur könne gar nicht sterben. Er wurde älter und älter, und natürlich auch gebrechlicher, doch sein Wille schien nie gebrochen ...“

„Unsinn, sein Wille ward mehrmals gebrochen.“, unterbrach ihn Lafgar wieder, „Unlängst brauchte es gar ein seltenes Heilkraut, um seine angeschlagene Stimmung zu retten. Und als sein böser Bruder die Agren terrorisierte, bot Brandur ihm einen Platz an seiner Seite an. Sein eigen Blut kümmerte ihn schon seit jeher mehr als alles andere. Erinnert ihr euch noch an die Zeit, als Thorald im Eisschlaf feststeckte und Brandur nur noch vor sich hin trauerte, statt eine Lösung zu suchen? Brandur war anfällig auf Fehler wie wir alle. Und dann erst die Gerüchte, dass er in vor nicht einmal so langer Zeit einigen un- und verheirateten Damen ...“

„Schweig stille“, zischte Peta, „Das ist unser König, über da sprichst!“ Wut glitzerte in seinen Augen.

„Eben nicht mehr“, gab der übellaunige Zwerg zurück.

Peta nickte traurig und wandte sich ab. „Nicht einmal eine ganze Woche ist er von uns gegangen, und schon spricht man schlecht von ihm. Natürlich tat man das auch schon vorher, aber nun, mit Thorald an seiner Stelle ...“

Peta verstummte und schüttelte seinen Kopf. „Nicht verzagen. Wir können noch hoffen. Und unser Bestes geben. Vielleicht, mit etwas Glück und Verstand, werden auch die Jahre von König Thoralds Regentschaft durch Frieden und Freude gezeichnet. Vielleicht sogar mehr als die Jahre von Brandur selbst. Wir können noch hoffen.“

Iril nickte bloß. Sie verabschiedete sich von den beiden und kehrte in den Flüchtlingslagern an der Rietburg ein. Die nächste Zeit würde sie hier verbringen und aushelfen, wo sie konnte.

Anfangen würde sie beim Zerteilen des Mhourls.







\newpage
\section{Der Eis-Dämon}

Über eine Woche war vergangen seit dem Tode Taroks.

Eigentlich hätten viele Bauern wieder ins Land ziehen wollen, doch gab es weiterhin viele Kreaturen, die nach dem Ableben des Drachen ziellos umherzogen. Bauernhöfe ungeschützt wieder aufzubauen, war lebensgefährlich. Daher blieben viele Bauernfamilien vorerst lieber hier, auf der sicheren Rietburg. Auch da gab es zahlreiche Reparaturen von niedergerissenen Türen und beschädigten Türmen zu tätigen. Zahlreiche Verletzte zu pflegen. Und zahlreiche Tote zu verabschieden, auch wenn die Totenzeremonie des gefallenen Königs immer weiter in die Ferne geschoben wurde.

Durchgehend lernte Iril neue Leute kennen. Den Keltermeister der Rietburg, der sich bei jeder Gelegenheit über unerwünschten Pflanzenwuchs in seinem Weinkeller ausließ. Baumeister Mard, der sich um den Säugling seiner verletzten Schwester kümmerte und ihm die eleganteste Wiege diesseits des Ozeans versprach. Torwächterin Felda, die weder lesen noch schreiben konnte, aber jedes einmal gesehene Schriftstück beinahe perfekt replizierte. Meisterbäcker Karmat, dessen luftige Kekse selbst die Kochkunst von Draks Familie in Schatten stellte.

Mit so einigen Anwohnern machte Iril flüchtige Bekanntschaft, doch klickte sie mit niemandem so besonders, wie sie es sich bei ihrer Ankunft in Silberhall direkt an Burmrit geklammert hatte. Diese innere Leere würde nicht so schnell gefüllt werden.

Auch wenn Iril sich stets amüsierte dabei, mit den Hunden der Rietburg spazieren zu gehen. Oder den Barden Grenolin beim Dichten seiner neusten Epen zuzuhören.

Oder einer alten Gelehrten bei der Suche nach den verschollenen Runensteinchen zu suchen.

Wie es sich herausstellte, hatte Tarok bei seinem Angriff auf Andor die Magie des Landes in sich aufgesogen und dabei zahlreiche Runensteine aus allen Ecken des Landes angesaugt und verschluckt – eben auch die der Gelehrten der Rietburg. Es stand zu befürchten, dass sie in Taroks Magen die Narne heruntergespült worden waren.

Im Anschluss daran, der Hohen Gelehrten eine vollständige Runenstein-Sammlung zu beschaffen, hatte Iril der Gelehrten und ihren aufgeweckten Schülern die Macht der Runen präsentieren können.

So, wie die Kinderaugen beim Aufführen von kleinen Tricks mit Irils Runenscheibe schon aufgeleuchtet hatten, schien es gut möglich, dass einige davon eine Zukunft als Runengelehrte in Betracht zogen. Schöne Entwicklung. Der Erforschung der Magie konnte es schließlich kaum zu viele begeisterte Helfer geben. Im Anschluss erzählte die Hohe Gelehrte Iril, was für Rituale man theoretisch mit den Drachenknochen anfangen konnte, die neuerdings in der Schatzkammer der Rietburg eingeschlossen waren. Iril staunte und trauerte auf einmal über die Bedeutung des Schatzes an Drachenmagie, den der Prinz mit Taroks Leichnam ins Hadrische Meer hatte spülen lassen.

In den Lazaretten der Rietburg konnte Iril mit ihrer Runenmagie besonders gut aushelfen. Beim alten Heiler Readem, der Knochen richten und Gliedmaßen amputieren, aber nichts gegen simple Halsschmerzen unternehmen konnte. Bei dessen Assistent Nabib, der die meiste Zeit lang detaillierte Zeichnungen von Knochen und Körpern anfertigte oder studierte. Sie beide blickten immer wieder mit ungläubigem Staunen auf Iril, wenn sie wieder einmal ihre Runenscheibe drehte, zum Glühen brachte, und damit dringend benötigtes sauberes Wasser in einem der nahe gelegenen Brunnen aufsprühte.

Hilfsbedürftige gab es in den Zeiten nach Taroks Zorn zur Genüge, selbst wenn die Stimmung unter den Rietländern eher ausgelassen war und Feiern gefeiert wurden. Iril half aus, wo sie konnte. Sie hätte es wohl auch getan, wenn es sie genervt hätte. Doch wie sie herausfand, erlangte sie ein gutes Gefühl der Erfüllung dabei, die Andori in ihren Wiederaufbauten zu unterstützen. Im Schweiße ihrer Arbeit vergingen die düsteren Gedanken zurück an ihre tote Lehrmeisterin, an ihre toten Eltern, an ihre verschollene Schwester. Bei Tag war sie zufrieden. Nur des Nachts schlichen sich die finsteren Gedanken wieder wie schleimige Schnecken in ihre Träume und hinterließen ihren klebrigen Schleim.

Und wie Iril stets an verschiedensten Winkeln der Rietburg zu Hilfe war, kam es, dass sie als eine der ersten gerufen wurde, als eines Morgens ein Brunnen hinter steilen, hohen, gar unbezwingbaren Westmauer der Rietburg plötzlich vereiste.

„Ich kann’s mir auch nicht erklären“, rief der Junge, der die Nachricht brachte. Jandro hieß er. „Es geht noch Monate, bis der Winter uns erreichen sollte. Doch ist dieser Brunnen und das umliegende Gras bereits jetzt ist mit einer Eisschicht versehen, dicker als in allen Wintern zuvor. Ich konnt’s aus der Ferne sehen. Mein Vater Najuk ist bereits ausgezogen, um näheres zu berichten.“

Ein vereister Brunnen vor der Rietburg? Iril erster Gedanke war, dass dies Bragor gar nicht gefallen würde. In den letzten Tagen hatte Iril sich mit diesem riesenhaften Tarus aus Sturmtal angefreundet, indem sie den Brunnen vor der Westmauer kraft ihrer Runenscheibe immer wieder aufs Neue hatte sprudeln lassen. Bragor hatte seinen schier unstillbaren Durst bewiesen. Hätte Iril nicht irgendwann eingehalten, hätte er wohl getrunken, bis seine Blase geplatzt wäre. Nicht ohne Grund gab es in seiner Heimat das Sprichwort „Ein Schluck Wasser bringt dir Stärke.“

Als Iril dieses geflügelte Wort das erste Mal vernommen hatte, hatte sie es noch mit dem alten Zwergensprichwort „Wasser ist wichtig! Wein ist gut! Met ist besser!“ gekontert, welches es offenbar selbst in die andorischen Kochbücher geschafft hatte. Seitdem sie miterlebt hatte, wie sehr frisches Quellwasser den Tarus erfrischen konnte, zweifelte sie nicht mehr daran, dass manchen auch Wasser völlig ausreichte.

Bragor war wie Iril von einer ihm vertrauten Nebelinsel in den südlichen Kontinent aufgebrochen und hatte sich hier neu zurechtfinden müssen. In den Helden von Andor hatte er jedoch eine neue Familie gefunden. Und manche andorischen Brunnen konnten gar beinahe die Frische der Quellen seiner stürmischen Heimat erreichen. Jedoch eben nur, wenn sie nicht gerade eingefroren waren.

Iril und der junge Jandro eilten auf den Wehrgang der Rietburg und spähten durch ein Fernrohr in die Ferne. Wie Jandro berichtet hatte, war der Brunnen im Südwesten der Burg von einer weißen Wolke bedeckt. Schnee lag um ihn herum. Zu dieser Jahreszeit. In dieser Tiefe. Unmöglich. Unnatürlich. Hexerei?

Iril überlegte sich bereits, ob sie von hier aus ihre Runen dazu befragen oder lieber direkt zum Brunnen ausreisen sollte, da ertönte ein Schrei von weiter unten am Eingangstor.

„Hilfe! Hilfeee! Ich werde verfolgt!“

„Das ist Papas Stimme!“, rief Jandro. Der Andori rannte die Treppe herunter und den Weg zum Tor entlang, wobei er gehörig Staub aufwirbelte.

Iril musterte den schreiend heranrennenden Andori neugierig. Najuk, hatte der Kleine gesagt. Dem „Na“ in seinem Namen nach stammte Najuk vermutlich ursprünglich aus dem Land der drei Brüder im Osten und war mit den einfallenden Sippen des Yetohe-Stammes nach Andor gelangt. Als er nun verschnaufte und hastig eine Antwort hervorstammelte, war in seiner Sprache allerdings nicht der geringste Akzent zu vernehmen.

„Da hat ein fremdes, eisiges Wesen einfach im Brunnen geschlafen. Vermutlich hat es ihn so vereist! Und es verfolgt mich! Guckt!“

Panisch zeigte Najuk auf eine Gestalt, welche nun durchs offene Tor der Rietburg trat und sich interessiert umsah. Schneeweiß war sie, kaum bekleidet, mit zwei kleinen Geweihen links und rechts aus dem Kopf wachsend. Der Schneemann trug eine auffällige Kette mit spitzen Gliedern um den Hals.

Zahlreiche Wachen richteten Speere und Bögen auf das Wesen, allerdings mehr verwirrt als vorsichtig, denn das Ewige Feuer in der Schale vor dem Tor flackerte weiterhin größtenteils orange. Ein Zeichen, dass zumindest aktuell keine große Gefahr für die Rietburg bestand. Die meisten anderen Anwesenden guckten nur neugierig. Helden von Andor schienen keine anwesend. Prinz Thorald ebenso wenig.

Das Wesen hob seine Stimme und sprach einige Worte in einer fremden Sprache. Bestimmt. Entschlossen. Absolut unverständlich.

„So was hat es schon vorhin gebrabbelt“, meinte Najuk schlotternd.

„Ein Eis-Troll!“, reif eine grau gewandte Gestalt aus der Menge der Schlaulustigen. Iril erkannte darin eine Bewahrerin vom Baum der Lieder, wohl hierher gesandt, um die Geschichte von Taroks Tod in die Archive des Baums der Lieder aufzunehmen. Sie wiederholte: „Das ist ein Eis-Troll. Viele Jahrhunderte schon haben wir keinen mehr gesehen. Doch unsere Aufzeichnungen berichten davon, wie die Vorfahren der ersten Bewahrer von Mutter Natur einige Abstecher ins Fahle Gebirge wagten. Und von einem riesigen Eis-Troll überfallen wurden. Die wenigen Überlebenden berichteten von einem riesenhaften Troll, weiß wie der Schnee und das ewige Eis. Er trug ein mächtiges Geweih auf seinem Kopf –und nichts als einen eisigen Lendenschurz, genau wie unser Neuankömmling hier. Ich glaube, er nannte sich ... Rakor? Rokur?“

„Eis-Troll?! Sieht dieser Schneemann für dich etwa wie ein Troll aus, Sanja?“, fragte der Bewahrer neben ihr unsicher.

„Nun, Jorna, vielleicht ist es auch eher ein Eis-Mensch“, gab die Bewahrerin zurück, „Wenn es Eis-Trolle geben kann, ist die Existenz von Eis-Menschen nicht so unwahrscheinlich, oder? Das macht ihn nicht weniger gefährlich.“

Iril blickte den Fremden erneut an.

Er wirkte wirklich eiskalt. Beim Ausatmen quoll Dampf aus seinem Mund, unter seinen eisfarbenen Stiefeln gefror der Boden und er zog eine Spur aus Schnee und Eis hinter sich her.

Erneut rief der Fremde etwas. Seinem klirrenden Tonfall konnte Iril keine Bedeutung entringen. Und die Sprache sagte ihr ohnehin nichts.

Noch nicht.

Nicht ohne Grund hatte sie sich einst eine Übersetzungsrune eintätowieren lassen. Iril fasste sich an den Nacken und ertastete das passende Tattoo. Sie griff den Runenhammer fest und fühlte, wie er Energie aus der Umgebung einsog. Dann presste sie den Hammer mit der spitzen Seite in ihren Nacken und fühlte, wie eine wohlige Wärme sich ihrer rechten Wange entlang ausbreitete und ihr Ohr erhitzte.

„Wer bist du?“, fragte sie den Schneemann. Leider konnte er sie nicht verstehen – so mächtig waren die Übersetzungsrunen dann auch wieder nicht. Aber sie sorgten dafür, dass sie ihn verstehen konnte. Nun musste sie ihn nur wieder zum Sprechen anregen.

„Ich verstehe kein Wort. Doch guckt mich an, ich bin keine Gefahr. Ich trage ja nicht einmal eine Waffe“, sprach der Schneemann mit klirrender Stimme und erhobenen Händen. Die Runen auf Irils Wange flackerten, dann verstand sie. Sie nickte.

„Du kannst mich nicht verstehen, aber ich dich“, sprach Iril ruhig, in der Hoffnung, ihr Tonfall könne irgendwie kommunizieren, was ihre Worte allein nicht konnten. Das letzte Wort des Schneemanns war „Waffe“ gewesen. Demonstrativ zeigte Iril auf einen Speer.

Der Schneemann hielt kurz inne und fragte dann: „Sonne?“

Iril zeigte, ohne zu zögern, hoch in den Himmel, an dem der brennende Feuerball hing.

„Boden“, versuchte der Schneemann es erneut. Iril zeigte auf den Boden.

„Wo Sonne, Mond und Sterne schweigen, denn das längste Licht wird durch den kleinsten Tropfen verwehrt“, sprach der Schneemann nun. Iril schwieg verwirrt. Der Singsang hatte nach einem Rätsel geklungen, oder nach einem Reim. Reime waren besonders interessant durch eine Übersetzungsrune wahrzunehmen, schließlich hörte Iril quasi mit dem einen Ohr den schönen Singsang, wie er im Originaltext klingen sollte, während die Runen ihr die Bedeutung der Worte in ihren Kopf flüsterten, wo sie sich in Bilder und dissonante Klänge auflösten, die ihr Geist verstehen konnte. Und die doch nie genau die ursprüngliche Intention herbeibringen konnte. Dafür waren Sprachen dann doch zu grundlegend verschieden. Manche Gedichte waren einfach unübersetzbar.

„Wolken. Ich meinte Wolken“, sprach der Schneemann etwas resigniert. Iril grinste, zeigte auf einige vorbeischwebende Wolken, und hatte damit dem Schneemann wohl zur Genüge ihre Übersetzungsgabe demonstriert.

„Du verstehst mich?“, fragte der Schneemann. Als ob er es nicht schon längst kapiert hatte.

Iril nickte. Dann hatte sie eine Erkenntnis. Normalerweise brauchten die Silberzwerge besondere Tinte, um die magischen Runen auf oder unter der Haut der Runenträger zu hinterlassen. Schließlich hatte man die schmerzhafte Tradition, stärkende Runen immer neu bis aufs Blut in die Haut zu schneiden, schon seit den Jahrhunderten seit dem Unterirdischen Krieg hinter sich gelassen.

Aber so ein Schneemann bestand doch nur aus Schnee, oder? Da wäre es doch ein leichtes, eine Rune darin zu hinterlassen.

Iril deutete auf ihre eigenen glühenden Runen und dann auf Ijsdurs Brust.

„Ich kann auch dich verstehen lassen. Wenn du mich lässt. Du. Verstehen. Dank. Runen. Wenn ich dir die Rune einzeichnen darf.“

Der Schneemann blickte verwirrt drein. Iril ergriff die Initiative und trat auf ihn zu. Von ihr mit Gesten geboten, kniete er sich zu ihr herunter. Sein kalter Atem kitzelte sie in der Nase. Dann hob sie ihren Hammer und hielt diesen auf Ijsdurs Brust. Sanft führte sie ihn dorthin und zog einen Strich darüber.

Abwesend murmelte sie: „Keine Nippel? Und auch keinen Bauchnabel. Dafür spitze Ohren ... so, als hätte jemand versucht, einen Menschen aus Schnee und Eis zu schaffen, aber keine perfekte Kopie erstellt. Und weiter oben ... ist das ein Geweih? Was bist du nur?“

Der Schneemann antwortete natürlich nicht, da er sie noch nicht verstehen konnte. Er zuckte jedoch zusammen, sobald Irils glühender Hammer mit seiner Spitze seine Brust berührte. Beim Kontakt schmolz ein kleiner Teil des Schnees seines Körpers und tropfte zu Boden. Iril beobachtete vorsichtig die Wunde. Ein klarer, dünner Strich. Kein Blut tropfte daraus hervor, ja, darunter schien nur mehr Schnee zu liegen. Und auch wenn dem Schneemann der Vorgang unangenehm erschien, zuckte er auch nicht zurück wie jemand, dem man die nackte Brust aufgerissen hätte. Iril würde ihn fragen müssen, wie es um sein Schmerzempfinden stand. Bald, sobald sie ihn verstehen konnte.

„Ruhig, ganz ruhig“, sprach Iril beruhigend. Zu ihrer Freude wehrte sich der Schneemann nicht. In Kürze hatte Iril die Runen der Übersetzung auf die schneeige Brust eingeritzt, den Hammer neu aufgeladen, ihn leicht auf die Brust des Schneemanns getippt und die Rune aktiviert.

Grünlicher Schimmer bereitete sich aus und der Schneemann sprach: „Uh, das kitzelt.“

„Das ist ein gutes Zeichen. Das heißt, es funktioniert“, sprach Iril.

Der Schneemann hielt eine Zeit lang überrascht inne und meinte dann: „Tatsächlich. Ich verstehe dich. Was für ein Wunder ist das?“

„Runenmagie“, sagte Iril fröhlich, und tappte stolz auf ihren Hammer, „Runen, die dir erlauben, Strukturen in gesprochenen Worten zu erkennen und sie besser sortieren zu können. Die dir ermöglichen, mich zu verstehen, und alle anderen Anwesenden, auch wenn du unsere Sprache nicht sprechen magst. Ich bin übrigens Iril.“

„Ich bin Ijsdur, der Eis-Dämon“, sprach Ijsdur.

„Willkommen in der Rietburg, Ijsdur.“

„Die Rietburg? Was ist das hier für ein Ort?“

„Die größte Festung außerhalb des Grauen Gebirges. Erbaut von der Schar des legendären Königs Brandur, des Anführers der Angekommenen.“

Zum ersten Mal zeigte Ijsdur eine Regung. Er machte einen verwirrten, doch beinahe freudigen Gesichtseindruck.

„Bran-dur? Ein Eis-Dämon ist euer König?“

„Nein, warum?“, fragte Iril nun verwirrt.

Ijsdurs freudige Miene erstarb. Kopfschüttelnd antwortete er: „Nicht wichtig.“

Die Tore des Palas krachten auf und ein aufgebrachter Prinz Thorald schritt daraus hervor. Saft tropfte von seinem Bart. Trank er bereits so am Morgen? Nichtsdestotrotz blickte der Prinz herrisch um sich und griff zu seinem Schwert, als er Ijsdur erblickte.

„Was ist das hier für ein Klamauk?! Und was ist das für ein Wesen?!“

„Das ist Ijsdur, ein Eis-Dämon. Najuk fand ihn in einem Brunnen.“, flüsterte eine Wächterin ihm zu.

„Was, Eis-Dämon? Dämonen bringen nichts als Ärger. Er soll sich einen anderen Schlafplatz suchen! Oder ins Gebirge zurückkehren, wo er herkommt. Die Rietburg beherbergt schon so zu viele Leute!“

„Wer ist das denn?“, fragte Ijsdur Iril. Da er weiterhin in seiner Muttersprache sprach, blickten ihn die restlichen Anwesenden nur verwirrt an.

„Das ist Thorald, der Herrscher dieses Landes“, murmelte Iril.

„Ich dachte, der Herrscher wäre dieser König Bran-dur.“

Thorald zuckte zusammen, als er Brandurs Namen inmitten der fremden Worte dieses fremden Wesens vernahm.

„Brandur ist kürzlich gestorben“, erklärte Iril.

Obwohl dies zuvor unmöglich geschienen hatte, blickte Thorald noch verwirrter drein, als er diese Antwort hörte. Und trauriger.

„Ich verstehe“, sprach Ijsdur, „Und ich verstehe, dass ich hier nicht erwünscht bin. Ich gehe.“

„Warte“, meinte Iril, „Ich komme mit dir. Ohne mich und meine Übersetzungskünste kannst du anderen Andori vielleicht verstehen, aber sich zu unterhalten dürfte schwer werden.“

„Ich danke dir, Iril.“

Iril fragte die Anwesenden, wo jemand wie Ijsdur sonst noch unterkommen könne. Sie erhielt mehrmals dieselbe Antwort:

„Na, die Taverne zum Trunkenen Troll natürlich!“

„Die Taverne ist wie durch ein Wunder der Zerstörung durch den Drachen und seine Kreaturen entkommen.“

„In Gildas Taverne ist jeder willkommen.“

„Wenn der Eis-Dämon dort kein Zimmer findet, dann nirgendwo.“\bigskip







Still wanderten Iril und Ijsdur durchs Rietland in den Süden, in Richtung dieser berühmten Taverne.

Iril reflektierte über die Stille und darüber, wie angenehm es doch sein konnte, ruhig neben jemand anderes zu laufen, statt irgendwelche oberflächlichen Floskeln auszutauschen, die keinen von ihnen interessierte. Ihre Gedanken kreisten um den Eis-Dämon. Er war wohl am ehesten ein Naturgeist, doch ein ausgesprochen menschlicher. Hatte jemand ihn geschaffen? Hatte er einen Zweck? Im Geiste erstellte sie eine Liste der zehn dringendsten Fragen, die sie Ijsdur fragen würde, sobald er sich hier eingefunden hatte.

Auch wenn es Iril auf der Zunge brannte, seine Geheimnisse zu lüften, so glaubte sie, dass er lieber nicht als erstes mit Fragen gelöchert werden wollte, während er versuchte, sich in einem fremden Land zurechtzufinden.

So wanderten die beiden in Stille durchs Rietland.

„Schönes Wetter. Scheint in diesen Landen immer so oft die Sonne?“, sprach der Eis-Dämon auf einmal.

Iril warf ihm einen verwirrten Gesichtsausdruck zu und gab knapp Antwort: „Öfter als im bewölkten Fahlen Gebirge, würde ich meinen.“

Was wollte Ijsdur? Warum stellte er eine solch seltsame Frage?

Es dauerte kaum wenige Minuten, bis er sich erneut an Iril wandte.

„Was ist da für ein Tier?“, fragte er, seinen Zeigefinger auf ein wildes Pferd richtend. Sein Tonfall klang absolut uninteressiert.

Erneut gab Iril etwas verwirrt Antwort: „Das ist ein Pferd.“

„Kann man es reiten?“

„Ja. Die Rietgarde richtete sie ab.“

Stille.

„Geht es noch lange bis zur Taverne?“

„Nachdem, was ich gehört habe, ja.“

Stille.

„Hinter jeder Person versteckt sich eine interessante Geschichte. Was ist deine?“, sprach Ijsdur kalt. Fast so, als fühle er sich gezwungen, Konversation zu beginnen.

„Ist das so, wie man dir beigebracht hat, Gespräche zu beginnen?“, fragte Iril, „Ich glaube nicht, dass meine Geschichte besonders interessant ist im Vergleich zu deiner.“

„Verzeih, ich bin etwas aus der Übung Doch weißt du, du, solche Fragen sind typischerweise in Konversationen eine Einladung, mehr zu erzählen. Du gibst mir nicht viel Material, mit dem wir arbeiten könnten.“

„Ich weiß“, meinte Iril, „Aber was soll ich schon erzählen?“

„Was auch immer du willst. Konversationen werden begonnen und entwickeln sich danach irgendwie in Richtungen, die beiden Gesprächspartnern gefallen.“

„Sofern beide Gesprächspartner das wollen.“

„Willst du dich lieber nicht unterhalten?“

Iril blieb einen Moment lang stumm.

„Ich weiß es nicht. Vermutlich schon. Ich weiß nur ohnehin schon kaum, wie man gut konversiert, und erst recht nicht mit Fremden, und erst recht nicht mit einem lebendigen Schneemann.“

„Ich bin kein Schneemann“, sprach Ijsdur, „Mich hat niemand gebaut. Würdest du gerne wissen, was Eis-Dämonen wie ich sind? Und woher ich stamme?“

„Ich wäre durchaus neugierig“, meinte Iril, „Ich liebe Rätsel und Geheimnisse.“

„Na, dann hättest du doch fragen können“, gab Ijsdur zurück.

„Das wäre nicht höflich gewesen.“

„Ah.“

Damit war das Eis gebrochen und die Worte begannen zu sprudeln.\bigskip







„Tulgor?“, fragte Iril verwirrt. Sie hatte gedacht, alle bekannten Länder und Reiche zumindest ansatzweise zu kennen.

„Genau, Tulgor. Das Land hinter den Bergen in den Wolken. Wie nennt ihr das hier?“

„‚Bergen in den Wolken‘? Meinst du das Fahle Gebirge?“

„Ja, dieses hohe Gebirge da im Westen. Dahinter liegt unser Land.“

Iril betrachtete die hohen Berge nachdenklich. Sie waren aus hellem, fahlem Gestein, sodass man kaum erkennen, wo Stein in Schnee und Eis überging, und wo in wabernde Wolken. So lange sie zurückdenken konnte, hatten Wolken die Gipfel der Berge verhüllt, sodass lange Zeit niemand gewusst hatte, wie hoch sie wirklich waren. Bis jemand auf die Idee gekommen war, den Schatten des Gebirges zu vermessen. Nun hatten sich die ewigen Wolken gelichtet. Ob da wohl ein Zusammenhang zum kürzlichen Kampf mit und Tod von Tarok bestand?

„Ich wusste gar nicht, dass hinter diesen Bergen noch ein weiteres Reich liegt. Ich dachte, das sei nichts sei als Ödnis. Sind da alle so wie du? Kann ich mir Tulgor als Eiswüste vorstellen?“

„Im Gegenteil!“, lachte Ijsdur, „Tulgor ist ein warmes Land voller wilder Steppen und satter Felder. Das Reich eines friedlichen Volkes. Ich hingegen bin ein Eis-Dämon. Wir kommen nicht direkt aus Tulgor, sondern aus den Bergen davor, die ihr offenbar Fahles Gebirge nennt. Aus dem ewigen Eis. Einer riesigen Eisfläche in einem schattigen Tal hoch oben.“

„Beeindruckend“, murmelte Iril. Das waren eine Menge Informationen auf einmal. „Muss einsam sein im Gebirge, nicht?“

Ijsdur stockte kurz, ehe er fortfuhr. „Ich denke nicht oft über Einsamkeit nach. So gedämpft die Gefühle von Eis-Dämonen im Vergleich zu Menschen sind, so kann fehlende Gesellschaft uns dennoch schmerzen.“

„Warum schlosst ihr euch denn nicht dem Rest der tulgorischen Gesellschaft an? Schmilzt ihr, wenn ihr in zu warmen Gefilden agiert?“

„Ich hoffe nicht“, sprach Ijsdur ernst, „Doch kann ich mir nicht sicher sein. Wir Eis-Dämonen waren für Jahrtausende eingesperrt hinter einem magisch versiegelten Felsentor in unserem schattigen Tal des ewigen Eises. Erst kürzlich wurde das Felsentor geöffnet und wir waren frei.“

Ijsdur blickte hinter sich auf die schwache Schicht aus Schneeflocken, die er auf dem erdigen Weg hinterließ. „Wenn ich mir das so ansehe, mache ich mir keine großen Sorgen ums Schmelzen. Die Magie aus meiner Eiskristallkette ist mächtig und wird vom ewigen Eis gespeist. Dem geht so bald die Kälte nicht aus.“

Iril fröstelte.

Plötzlich schreckte sie ein seltsames Geräusch auf. Es war eine Art Scharren, das aus dem hohen Rietgras direkt hinter ihr kam. Iril blieb stehen und wies Ijsdur an, gleich zu tun.

„Was ist ...“, setzte der Eis-Dämon an. Er erstarrte.

Als Iril sich umdrehte, tauchte wie aus dem Nichts ein Gor auf. Zischend schwang die Bestie ihre riesigen Hornklauen.

„Futter. Futter. Fein.“, zischte der Gor und leckte sich über das Maul. Wie sprachbegabt die Vertreter seiner Spezies doch sein konnten. Und leider waren sie so gut wie nie allein unterwegs.

„Gors!“, brüllte Iril, „Wenn der erste Gor in Sichtweite ist, steht der zweite schon ...“

Ein Heulen ertönte hinter ihr. Sie wagte kaum mehr, als einen kurzen Blick auf Ijsdur zu werfen. Der kurze Blick genügte schon, um zu erkennen, dass ihr Begleiter von besagtem zweiten Gor mitten in die nackte Brust getroffen worden war.

Es sah nicht gut aus! Der Gor schnappte erneut zu und riss dem überrumpelten Ijsdur ein gewaltiges Loch in die Brust. Iril hatte gerade genug Zeit, zu registrieren, dass Ijsdurs Innenleben aus verschiedenen Schneeschichten zu bestehen schien, da sackte der Eis-Dämon auch schon zu Boden. Schneeflocken wirbelte umher und Wasser plätscherte aus seiner Wunde. Doch ehe Iril ihm zu Hilfe kommen konnte, kündigte ein Knurren hinter ihr die nächste Attacke ihres eigenen Gors an.

Einen Herzschlag lang stockte Iril der Atem und das Feuer der Furcht schoss durch ihren Körper. War dies so, wie sich ihre Mutter gleich vor ihrem Tod gefühlt hatte? Stand Iril nun dasselbe Schicksal bevor?

Ehe sie sich groß in ihren eigenen Gedanken verheddern konnte, bewegte sich ihr Körper wie von allein. Jahrelanges Training unter den Mauerbergen hatten sie auf solche Situationen vorbereitet.

Iril duckte sich unter einer unförmigen Hornklaue ihres Gegners hinweg und verbarg sich im hohen Rietgras. Der Gor stak mit seinen langen Klauen haarscharf neben ihr in den Boden. Jetzt hatte sie genug! Iril schwang ihren Runenhammer. Die mit einer Vielzahl von Runen überzogene Waffe folgte Irils Schwung und schmetterte mit der flachen Seite gegen den massigen Bauch des Gors, ehe Iril den Hammer überhaupt magisch aufladen konnte. Der Gor winselte und taumelte einige Schritte zurück. Iril warf sich auf und war bereit, gnadenlos nachzusetzen. Mit gemeinen Gors hatte sie kein Mitleid. Doch der Gor wich dem zweiten Schlag – diesmal hätte ihn die spitze Seite des Runenhammers am Kopf getroffen – haarscharf aus.

Und noch schlimmer: Der Gor packte Iril Hammer mit einer Hornklaue und riss ihn weg von ihr. Mit seiner schieren Kraft konnte sie nicht mithalten. Iril ließ den Hammer los und tappte auf ein Tattoo an ihrem Arm. Eine wohlige Wärme kitzelte ihre Hand, welche rötlich aufleuchtete. Ebenso leuchtete der fortgeschleuderte Hammer, auf welcher nun mitten in der Luft eine Drehung vollzog und zu Iril zurückzufliegen kam. Auf dem Weg zurück in ihre ausgestreckte Hand haute er den Gor nach vorne. So taumelte der Gor zu Boden, sein ungeschützter Nacken offen zugänglich. Ein weiterer Schlag gab der unsäglichen Kreatur den Rest.

Iril wirbelte herum. „Ijsdur, halte dich von seinen Klauen fern und versuche, ihn von hinten anzugreifen! Sein Nacken ...“

Doch, ehe sie fertig sprechen oder gegenüber Ijsdurs Gor gewalttätig werden konnte, hatte Ijsdur sich schon um diesen gekümmert. Stolz und unversehrt ragte der Eis-Dämon vor seinem Gegner auf. Keine Spur mehr der tiefen Wunde, die der Gor in ihn gerissen hatte. Ein helles Licht blitzte aus Ijsdurs ausgestreckter Hand auf und ein längliches, gezacktes Objekt – war das ein riesiger Eiszapfen? – raste auf des Gors garstiges Gesicht zu. Getroffen wurde er zurückgeschleudert, sein ganzer Körper mit Eiskristallen bedeckt. Auch dieser Gor rührte sich nicht mehr.

„Interessante Wesen“, bemerkte Ijsdur, „Sehen sie nicht, dass es in ihrem Interesse wäre, zu fliehen?“

„Elend, diese Kreaturen“, spuckte Iril aus, „Sie leben beinahe überall, wo es Zivilisationen gibt. Sie dienen den Drachen. Ich hatte gehofft, mit dem Tode Taroks wären sie nicht mehr auf die Unschuldigen aus. Aber alte Angewohnheiten sterben schwer und der Hunger auf Fleisch scheint sie immer noch anzutreiben.“

„Von Zeit zu Zeit verirrte sich die eine oder andere Kreatur aufs ewige Eis“, berichtete Ijsdur, „Vielleicht sind gar ein paar von ihnen zu Eis-Dämonen geworden. Aber zumindest diese Gors stehen uns in Geistesgaben nichts nach.“

Iril starrte weiterhin verwundert auf Ijsdurs so unversehrte Brust. „War da nicht zuvor noch eine tödliche Verletzung?“

„Oh, was tödlich ist, hängt immer vom Körper ab. Mein Leib aus Schnee wird wohl nie so kräftig oder stabil sein wie dein Gebilde aus Knochen, Fleisch, und all dem Zeug, doch vermute ich, Verletzungen erheblich rascher und einfacher reparieren zu können. Weg ist natürlich nichts. Bis die Schmerzen nachlassen, könnte es noch eine Weile dauern. Aber Schmerzen kann man gut ignorieren.“

Iril nickte stumm. Ehe sie weiter nachhaken konnte, rasten ihre Gedanken weiter zu Ijsdurs Hand, aus welcher zuvor ein eisiger Blitz den Gor mit einem Schlag überwunden hatte. „Faszinierend, dieser Eisblitz. Was war das? Wie kreierst du die?“

„Ich weiß nicht“, druckste Ijsdur herum.

„Oooh, ist das ein Geheimnis? Ich liebe Geheimnisse.“

„Leider kein Spannendes. Diese Eisblitze lösen sich einfach irgendwie von mir. Wenn ich es will. Und die Kraft dazu habe. Die Eiskristallkette um meinen Hals ist der Pfad zu mannigfaltigen Fähigkeiten.“

Iril fröstelte es erneut, und nicht nur wegen der Nähe zu Ijsdurs eiskaltem Körper. Während die Aufregung des überraschenden Gorkampfs versiegte, haderte sie auf einmal mit dem Gedanken, ob es klug gewesen war, allein an der Seite dieses Fremden ins Land zu ziehen. Nicht, dass sie nicht auf ihre kämpferischen Fähigkeiten oder auf das Urteil des Ewigen Feuers vertraute. Doch ward ihr nun bewusst, dass die Magie des undurchsichtigen Eis-Dämons ihre eigene in Kampfkraft durchaus übersteigen konnte. Dass das Ewige Feuer eigentlich nur Gefahren für die Rietburg sondierte, nicht für kleine Runenmeisterinnen in fremden Landen. Und dass sie noch keine Ahnung von den Motiven des Eis-Dämons hatte.\bigskip







Die Runenmeisterin und der Eis-Dämon schritten weiterhin Seite an Seite in Richtung Taverne, auf ausgetretenen Trampelpfaden zwischen goldenen Rietgras-Feldern voller rosa Rietgras-Blüten. Iril schritt leise voran, mit schnellen Schritten, während ihr Kopf ratterte. Ijsdur glitt langsam voran, wie üblich eine Schicht aus Schnee und Eis hinter sich herziehend – auch wenn Iril glaubte zu sehen, dass die Schicht dünner war als auch schon.

Irgendwann konnte sie sich eine vorsichtige Frage nach Ijsdurs Motiven nicht mehr verkneifen: „Verzeih, dass ich so direkt frage: Was suchst du hier in Andor eigentlich?“

Ijsdur stutzte einen Augenblick und sprach dann so tonlos wie immer: „Wenn ich das wüsste. Endlich bin ich frei von diesem Schattental hinter dem Felsentor. Doch meine Zukunft ist ungewiss. Ich weiß, dass ich in Tulgor keine Zukunft habe. Und dass die vielen einsamen Nächte hoch oben im Kuolema-Gebirge taten mir nicht gut. Ich hoffte, hier auf eine gewisse tulgorische Reisegruppe zu stoßen, doch hörte ich von dir, dass die Tulgori noch gar nicht hier eingetroffen sind.“

„Zumindest ich hätte noch nichts von ihnen gehört“, entschuldigte sich Iril.

„Äußerst eigenartig“, murmelte Ijsdur, „Vielleicht überholte ich sie auf dem Weg unter dem Gebirge hindurch. Die Stollen unter dem Kuolema-Gebirge hindurch sind ein Labyrinth und als Eis-Dämon kann ich einzigartige Abkürzungen schaffen. Wie dem auch sei, so werde ich nun hier auf ihre Ankunft warten. Früher oder später werden sie doch aufkreuzen müssen.“

„Hoffentlich vereisen sie uns nicht ein zweites Mal den Brunnen“, lachte Iril.

„Keine Gefahr“, meinte Ijsdur, „Ganz abgesehen davon, dass keine Eis-Dämonen mit ihnen reisen. Ich wanderte nach dem Austritt aus dem Gebirge einfach ziellos durch die Lande, sah die Rietburg in der Ferne und beschloss, ihr näher zu kommen und in dieser stärkenden Wasserquelle zu nächtigen, um niemanden zu wecken. Ich weiß, dass Menschen ungern aus dem Schlaf gerissen werden. So, wie ich diese Reisegruppe kenne, werden sie sich eher getrauen, einfach an das Tor zu klopfen.“

„Siehst du dich nicht mehr als Menschen?“

„Ich war nie einer. Ich hoffe, dass ich die richtige Entscheidung traf, indem ich hierherkam. Hier könnte ich neue Kontakte knüpfen. Eine neue Existenz aufbauen. Eine Bestimmung finden.“

„Klingt ganz ähnlich wie ich. Auch ich fühle mich zuhause nicht mehr zuhause und suche nun hier nach einer neuen Bestimmung.“

„Dann sitzen wir ja im selben Boot.“ So tonlos wie oft zuvor hängte Ijsdur an: „Es freut mich außerordentlich, dich getroffen zu haben, Iril.“

Iril bedachte die gruselige Macht von Ijsdurs Eisblitzen nach dem Kampf gegen den Gor. Und wie unüberwindbar ein schnell heilender Körper aus Schnee doch sein konnte. Nicht, dass ihre eigene Macht über die Runen nicht auch gruselig sein konnte. Oder dass sie das grundlegende Misstrauen gegenüber Fremden, das so vielen Schildzwergen eigen war, für gut hielt. Doch schüchterte sie Ijsdurs eisige Art immer mehr ein.

Sie erinnerte sich an den bösartigen Eis-Troll, von der diese Bewahrerin in der Rietburg erzählt hatte. Und daran, dass Ijsdur erzählt habe, die seinen während seit Jahrtausenden in diesem Tal des ewigen Eises im Fahlen Gebirge eingesperrt gewesen. Und dann nannte er sich selbst auch noch ‚Eis-Dämon‘. Natürlich war das nur eine ungefähre Übersetzung des originalen Wortes in seiner Sprache. Vermutlich trug es in seiner Heimat keine negative Konnotation.

Erneut bedachte sie den pupillenlosen Blick aus seinen milchigen Augen. Ein Blick, den er sich mit den bösartigen Kreaturen teilte, die dank diesen Augen auch in der tiefsten Finsternis der Nacht unschuldige Opfer erspähen konnten.

Iril packte ihren Hammer ungewollt fester und fragte betont unschuldig: „Darf ich fragen, warum ihr Eis-Dämonen hinter diesem Felsentor eingesperrt wart?“

Ijsdur hielt kurz inne, antwortete dann jedoch wie aus der Balliste geschossen: „Das war ein ungerechtfertigter Fluch. Die willensstarke Kaiserin von Tulgor – damals war Tulgor nämlich noch ein Kaiserreich – fürchtete uns Eis-Dämonen. Ihr Adoptivsohn war in einer bitterkalten Nacht im Gebirge erfroren. Sie hatte uns die Schuld gegeben. Und wie königliche Flüche halt so sind, löste sich unserer erst Jahrtausende später auf.“

Iril nickte. Als Silberzwergin wusste nur allzu gut darüber Bescheid, was königliche Flüche ausrichten konnten. Narkon war auch Jahrzehnte nach Varatans Tod noch immer eine verdammte Falle, voller wahnsinniger Piraten, wilder Hautwandler und wütender Kreaturen. Und niemand wusste, wann sich der Fluch des verstorbenen Seekönigs lichten würde.

Ijsdur betrachtete Iril mit schiefgelegtem Kopf und fügte an: „Keine Sorge. Wir Eis-Dämonen wurden nicht für vergangene Verbrechen eingesperrt oder so. Wir haben den Sohn der Kaiserin nicht retten können, aber ihn auch nicht umgebracht. Wir sind unschuldig. Wir sind keine Gefahr für die Öffentlichkeit. Wir wollen Andor nichts Böses.“

Iril schluckte fest und löste ihren Runenhammer von ihrem Gürtel. Diesen Trick hatte sie schon einmal von einem der hohen Runenmeister eingesetzt gesehen. Sie musste nur überzeugend genug auftreten.

Vorsichtig wählte sie ihre Worte und log: „Ijsdur? Weißt du, dass meine Übersetzungsrunen nicht nur die Bedeutung gesprochener Worte übertragen können? Sie geben mir manchmal auch ein Gefühl der Intention des Sprechers dahinter. Oft nur klein, fein, beinahe vergesslich. Aber Lügen kann ich mit meinen Runen von Meilen her riechen. Und ich rieche Lügen in deinen Worten. Was willst du wirklich hier?“

Iril blieb stehen und beobachtete Ijsdurs Mimik, suchte nach irgendeiner Reaktion, die ihr verraten könnte, ob ihr Misstrauen gerechtfertigt war.

Ijsdur blieb ebenfalls stehen und blickte Iril einige Augenblick lag vollkommen ausdruckslos an.

„Schade“, murmelte er zu sich selbst, „Ich begann gerade erst, dich zu mögen.“

Dann stürzte er sich auf Iril, ohne auch nur mit der Wimper zu zucken.\bigskip







Ijsdur griff nach Irils Armen und ließ nicht los. Sein fester Griff war eiskalt und kribbelte auf Irils tätowierter Haut. Er öffnete seinen Mund und hauchte aus. Eisiger Dampf schwallte auf Iril Kleidung herunter und überzog sie mit einem Reif aus Frost.

„So halte schon still, dann überlebst du das vielleicht“, sprach Ijsdur.

Iril drehte ihre Arme in Richtung seiner Daumen und löste sich mit Mühe von ihm. Erschreckt nahm sie wahr, dass ihre Haut unter seinem Griff schon leicht bläulich angelaufen war. Sie schlug einen Purzelbaum an Ijsdurs strammen Beinen vorbei, während sie seinen Tritten auswich. Definitiv feindlich eingestellt, dieser Dämon. Sie machte ihren Hammer bereit. Ihr ganzer Körper schmerzte und protestierte. Das Herz schlug ihr bis zum Halse. Und es knirschte unschön in ihrer Reisetasche, über die sie soeben gerollt war.

Zwei überraschende Kämpfe gleich hintereinander, das hatte sie schon lange nicht mehr erlebt! Hoffentlich war sie nicht allzu eingerostet.

Ijsdur drehte sich zu ihr um und hob seine eigentlich leere Hand. Etwas knisterte und glomm darin. Ehe er etwa erneut einen Eisblitz schleudern konnte, versenkte Iril ihren Hammer mit der Spitze voran in Ijsdurs Kniekehle. Beim Kontakt entluden sich aufgesparte magische Ströme in farbigen Lichtblitzen. Der Hammer glitt beinahe mühelos durch beide Beine des Eis-Dämons hindurch und pulverisierte sie zu Schneehäufchen. Ijsdur fiel flach auf seinen Bauch. Doch Iril wusste vom Gorkampf noch, dass Ijsdur eine solche Verletzung problemlos davonsteckte. Also setzte sie nach. Ein weiterer Schlag des Hammers hinterließ eine beachtliche Delle in Ijsdurs Rücken. Schnee und Eis spritzten umher.

Iril durfte ihn keinen Eisblitz generieren lassen. In der starken Magie seiner Eiskristallkette schien seine wahre Stärke zu liegen. Also musste sie rasch handeln, so rasch, dass er sie nicht gegen sie einsetzen konnte. Schnell, jetzt!

Zwei weitere Schläge, und Ijsdurs Hände waren Vergangenheit. Temporär. Hoffentlich konnte er nur von dort aus Eisblitze schleudern. Sie ließ von ihm ab, öffnete ihre Reisetasche und sondierte verzweifelt ihre verschiedenen metallenen Runenscheiben.

Ijsdur wand sich neben ihr am Boden. Während sein Körper sich wieder zusammenschichtete, machte sein Kopf eine 180-Grad-Drehung um seinen Hals und starrte Iril durchdringend an. Die Delle, die ihr Hammer in seinem Rücken hinterlassen hatte, begann, sich magisch zu schließen. Ächzend richtete der Eis-Dämon sich wieder auf.

Hektisch durchsuchte Iril ihrer Tasche nach einer bestimmten Runenscheibe.

Gefunden!

Sie sammelte Magie im Hammer und tippte die Scheibe an – silbern glitzernde Lichter flackerten umher – und versenkte sie tief in der Delle in Ijsdurs Rücken. Schnee wuchs darüber, doch darum kümmerte Iril sich nicht groß. Einmal aufgeladen, brauchte die Scheibe kein Sonnen- oder Mondlicht, um ihre Wirkung zu entfalten.

Unter den von den Runenmeistern der Silberzwerge am häufigsten genutzten Runenfolgen gab es eine bestimmte, welche verspukte Höhlengänge von verlorenen Seelen zu reinigen vermochte – denn als erstmals Minen in den Silberberg gegraben worden waren, hatten sich überraschend viele ungewollte Geister aus vergangenen Jahrhunderten gezeigt und die Zwerge zu vertreiben versucht. Seitdem gehörte eine solche Runenscheibe zur Vertreibung fremder geistiger Einflüsse zum Standardrepertoire eines jedes Runenmeisters, auch wenn sie in letzter Zeit nur noch selten mit verirrten Seelen zu kämpfen hatten.

Ijsdur stürzte in sich zusammen, kaum hatte Iril die Scheibe in ihn gesteckt. Also hatte die Runenscheibe ihre gehoffte Wirkung gehabt und Ijsdur den Geist ausgetrieben. Was für ein Glück.

Traurig betrachtete Iril die leblosen Klumpen Schnee vor ihr, der immer noch vage die Form eines menschlichen Leibes hatte. Es wäre wirklich faszinierend gewesen, dieses seltene Phänomen der Natur zu studieren, ihm alle seine Geheimnisse zu entlocken. Doch hatte Ijsdur sie angegriffen und damit seine üblen Absichten für dieses Land kundgetan. Vermutlich war es besser so.

Iril war kurz davor, sich abzuwenden und zurück zur Rietburg aufzubrechen, da bewegte sich der Schneehaufen vor ihr.

Langsam setzte sich Ijsdur auf und hielt seinen Kopf. Seine leicht zerfallene Form festigte sich wieder. Die Eiskristalle an der Kette in seiner Brust glitzerten in allen Regenbogenfarben. Fahle Hände tasteten den Bauch ab, in welchem Iril ihre Geister abwehrende Runenscheibe versenkt hatte. Eisige Augenlider öffnete sich und zeigten pupillenlose Augen.

Starr blickten Iril und Ijsdur einander an. Iril war verwirrt. Falls Ijsdur, wie sie gedacht hatte, quasi eine einen Schneehaufen besessende Seele aus einer anderen Sphäre der Realität wäre, sollte die Scheibe ihn eigentlich in diese andere Sphäre zurückgeschickt haben. Und falls er eine Puppe einer fremden Entität des ewigen Eises gewesen wäre, hätte die Scheibe ihn aus der Kontrolle der fremden Entität befreit. Dass er noch hier war, dass er sich noch bewegte, bewies, dass es sich bei ihm um ein lebendiges, fühlendes Wesen handelte, welches in diese Ebene der Realität gehörte. Wie ein Naturgeist? Doch konnte Iril sich nicht an einen einzigen Naturgeist erinnern, der so eloquent wie Ijsdur gewesen wäre. Welch Wunder barg dieser Eis-Dämon denn noch?

Zu schade, dass sie sie nicht erforschen konnte, wenn ihr ihr eigener Leib und Leben wichtig waren. Iril ließ erneut Magie in ihrem Hammer ansammeln und bereitete sich auf einen mächtigen Schlag vor.

„Nein!“, rief Ijsdur, und hob seine Hand, „Ich danke dir, liebe Iril, von ganzem Herzen, für was auch immer du tatst! Die Stimmen der Vergangenheit sind verklungen. Ich spüre den Willen der Herrin des ewigen Eises nicht mehr. Ich bin frei! Wirklich frei!“

Iril ließ ihren Hammer nicht sinken.

„Ich will dir nichts Böses. Auch diesem Land nicht! Fühle die Wahrheit meiner Worte!“

„Böse und Gut sind sehr relative Begriffe“, knurrte Iril, „Magst du noch spezifizieren? Und erinnere dich daran ...“

„... dass du Lügen spüren kannst, natürlich.“

Ijsdur holte tief Luft – wohl eher theatralisch, denn groß atmen zu müssen schien er nicht – und ratterte herunter:

„Ich will, dass du weder leidest noch stirbst, Iril. Du und alle anderen Lebewesen in Tulgor und diesen Landen. Ich will Siantari nicht mehr bis in alle Ewigkeit dienen. Ich will nicht mehr heimlich diese östlichen Reiche erforschen, um die Schwachstellen dieses Reichs zu erkennen. Ich will nicht mehr die tulgorische Reisegruppe abfangen, ehe sie diese Reiche vor Siantari warnen könnten. Ich will nicht mehr eine Eiswüste über diese Welt verbreiten!“

„Das klingt ja so, als hättest du bis vor wenigen Augenblicken einen relativ ausgeklügelten finsteren Plan gehabt. Oder lese ich da zu viel hinein?“

„Keineswegs“, gab Ijsdur zu, „Wie alle Eis-Dämonen des Kuolema-Gebirges habe ich oft die Stimme Siantaris in meinem Hinterkopf vernommen. Die Stimme der Herrin des ewigen Eises. Und die wirren Rufe ihrer Vorgänger. Dies war ihr Reich. Sie und alle Tari vor ihr waren nicht ohne Grund hinter dem Felsentor eingesperrt worden. Sie wollten die ganze Welt in eine Eiswüste verwandeln. Und ich hätte bis soeben nicht einmal ihre Stimme hören müssen, um mir sicher zu sein, dass es mein Schicksal sei, das ewige Eis in ihrem Namen weiter auszubreiten.“

„Und das ist es nun nicht mehr?“

Ijsdur fiel auf seine Knie. Leise flüsterte er: „In all dem Lärm, in all den Stimmen der Vorgänger, in alle dem konnte ich die Erinnerungen nie richtig lesen. Die Erinnerungen von Ijs, dem im Eis gestorbenen Junge, dessen Körper und Namen ich trage. Ich konnte seine Erinnerungen sondieren und doch deutete ich sie falsch. Unterdrückte, was seine Gefühle mir längst hätten sagen sollen. Deine Runen haben die Stimmer der Tari verstummen lassen. Nun, wo sie stumm sind ...“

Ijsdur schluckte, ehe er fortfuhr.

„Die Jahre im ewigen Eis waren keine schöne Zeit. Ich fühlte mich nicht gut. Verwirrt, rastlos, ziellos. Ijs‘ Jahre in Tulgor hingegen, so verblasst sie in meinen Erinnerungen auch sind, sind so viel schöner. Voller. Angenehmer. Freudiger. Es ist eigenartig. Die Tulgori bedeuten mir nichts. Du bedeutest mir nichts. Ihr alle werdet sterben, lange bevor ich vergehen werde. Und doch ... ich ... ich will Tulgor nicht zu einer Eiswüste werden lassen. Ich will nicht sehen, wie du im Eis erfrierst.“

„Wie beruhigend.“

„Es ist wahr. Eines Tages werde ich sterben. Und ich fühle mich nicht wohl beim Gedanken, das allein zu tun, in einer eisigen Welt. Lebende haben so viel mehr zu bieten.“

Stumm blickten die beiden Kämpfenden einander an.

„Was tun wir nun?“, fragte Ijsdur.

„Was willst du tun?“

„Wenn ich das wüsste. Ich muss zunächst einmal meinen Kopf ins Klare kriegen. Meine Gedanken in eine passende Reihenfolge bringen.“

Stille.

Iril sprach als erste wieder: „Wie wäre es, wenn wir wie vorhin gewollt zu diesem Wirtshaus weiterreisten? Dort können wir weiterschauen, wie er mit dir weitergehen soll.“

Ijsdur nickte.\bigskip







Im Vorbeigehen öffnete Iril ihre vom Kampf lädierte Reisetasche und nahm Inventar auf. Ihre Sammlung an Runenscheiben war größtenteils ganz. Manche der Metallscheiben waren etwas verbogen, aber nichts Schlimmeres. Mit einer Ausnahme.

Der steinerne Runenring mit eingelassenem Glas zur Fernsicht war unrettbar zerbrochen. Zuletzt hatte Iril ihn genutzt, um Taroks Angriff auf die Helden aus der Ferne zu beobachten.

„Einfach nichts schlimmeres“, murmelte sie, um sich selbst zu beruhigen, „Wir können nur hoffen, dass wir in den nächsten Tagen nicht dringend einen Fernblick bräuchten.“

„Wir können hoffen“, bestätigte Ijsdur.

Leise fügte er an: „Es tut mir leid, Iril. Schon zum zweiten Mal heute hast du mir enorm geholfen. Ich verstehe das Konzept von Schuld nicht ganz, doch glaube ich, mich tief in der deinen zu befinden.“

Iril winkte ab.

Ijsdur beugte sich nochmal vor, guckte ihr tief in die Augen und bekräftigte: „Danke, Iril. Einfach nur Danke.“

Er erstarrte für einen kurzen Augenblick. Die um ihn herumwirbelnden Schneeflocken wurden etwas schneller, als er klirrend weitersprach. „Siantari! Ich vergaß sie völlig. Die Herrin des ewigen Eises ist auf dem Weg hierher. Sie verließ als erste von uns das ewige Eis. Und sie hat immer noch vor, die gesamte Welt unter einer dicken Schicht aus Eis verschwinden zu lassen. Wenn wir die hier lebenden Wesen wahren wollen, müssen wir sie vor Siantari warnen.“

„Das ist schon in Ordnung“, meinte Iril, „Wir warnen die Andori einfach vor. Die Helden von Andor sind mächtig. Sie haben einen Drachen niedergerungen. Da werden sie hoffentlich mit einem schnöden Schneeklotz klarkommen können. Nicht persönlich gemeint.“

„Keine Beleidigung angekommen“, versicherte Ijsdur, „Doch unterschätzt Siantari nicht. Sie ist eine formidable Gegnerin. Wenn du nicht ihren Bann über mich gelöst hättest, würde ich aktuell immer noch, ohne mit der Wimper zu zucken, die Ausbreitung des ewigen Eises über das Leben aller Menschen stellen, die ich je kannte.“

„Wie kann es sein, dass Siantari hier noch nicht eingetroffen ist?“, fragte Iril, „Ist sie vielleicht schon hier? Heimlich?“

„Heimlichkeit ist nicht Siantaris Stärke. Sie wollte zunächst, dass ich die Lage sondiere und ihr vielleicht die Möglichkeit einer Täuschung verschaffe, indem ich die von den Eis-Dämonen Wissenden Tulgori abfange. Bevor du meinen Geist abschirmtest, muss sie zumindest im Hinterkopf meine Erfahrungen wahrgenommen haben. Sie weiß, dass diese Bewahrer von Eis-Dämonen gehört haben. Und sie wird annehmen, dass du mich vernichtet hast. Vielleicht versteckt sie sich irgendwo und plant etwas.“

„Was könnte sie planen?“

„Keine Ahnung. Ihre Fähigkeiten sind mannigfaltig, doch haben sie allesamt mit Eis und Schnee zu tun.“

„Also können wir schließen, dass uns keine Gefahr droht, bis der erste Schnee fällt?“, fragte Iril.

„Oder der erste Schnee fällt erheblich früher als üblich. Aber Siantaris bisherige Abwesenheit muss nichts bedeuten. Tulgor liegt erheblich näher Felsentor zum ewigen Eis als Andor. Und der Weg durch die Temm-Pfade unter dem Kuolema-Gebirge hindurch ist erheblich schneller als die mühsame Gratwanderung über die Berge. Insbesondere wenn man wie ich die Abkürzungen zwischen den labyrinthischen Temm-Pfaden nehmen kann, statt wie sie über hohe Gipfel und tiefe Täler zu spazieren, wäre es kein Wunder, wenn ich schneller hier landete, als sie es tat.“

Stille.

Iril brach sie: „Die Tulgori abzufangen, das war ein direkter Befehl von Siantari? Der einzige, dem du folgtest?“

„So ist es.“

„Warum dann so friedlich auf uns zugekommen? Warum versuchen, in der Rietburg Anschluss zu finden, und gleich wieder zu gehen, nachdem man dich nicht herzlich willkommen heißt?“

„Ich wollte die Lage nicht eskalieren lassen. Von euren Verteidigungsmöglichkeiten herausfinden, aber nicht, indem ihr sie auf mich anwendetet.“

„Und du bist dir sicher, dass du dir nicht einfach eine Ausrede suchtest, um nicht sofort mit dem Töten und Morden anzufangen?“

„Ziemlich. Nur mein Ziel bedeutete mir etwas. Die Ausbreitung des ewigen Eises. Der Kampf gegen die ewige Hitze.“

„Warum tötetest du mich dann nicht einfach, nachdem ich misstrauisch geworden war? Ein Eisblitz hätte mich komplett erledigt. Aber nein, du versuchtest stattdessen bloß, mich festzuhalten.“

Ijsdur schwieg eine Zeit lang. „Ich hatte noch nicht genug Zeit zum Überlegen. Ich musste vorsichtig sein. Wenn du verschwändest, direkt nachdem du an meiner Seite ins Rietland zögest, wären Leute misstrauischer geworden. Vielleicht hätte es einen Weg gegeben, dich auf meine Seite zu ziehen. Du hättest eine ausgezeichnete Eis-Dämonin abgegeben. Runenmagie an unserer Seite hätte Siantari stark geholfen.“

Iril kicherte. „Du sagtest, du seist kein Mensch mehr.“

„Ich war nie einer.“

Iril grinste. „Und doch ist es ausgesprochen menschlich, sich für das eigene Verhalten nachträglich halbgare Begründungen auszudenken.“

„Pfff“, machte Ijsdur, „Wenn ich dich aus anderen Gründen verschont hätte, wäre mir das nicht bewusst?“

„Es ist zumindest möglich, dass du dir etwas einredest, damit dein Verhalten nicht deinem Selbstverständnis widerspricht. Vielleicht konntest du doch nicht so rasch zum Morden überwinden, wie du gerne gehabt hättest.“

„Vielleicht“, murmelte Ijsdur, „Du warst die erste Person seit langem, mit der ich interagierte, ohne direkt verachtet zu werden. Und selbst du hattest kein Problem damit, mir zu misstrauen. Du magst im Recht gewesen und ich freue mich, dass du mich von Siantaris Einflüssen befreit hast. Und doch schmerzt es.“

Ijsdur hielt an und blickte Iril mit starrem, pupillenlosem Weiß tief in die Augen.

„Ich bin kein starrer Eisklotz. Ich habe Gefühle. Vertraue mir. Es bringt ja ohnehin nichts, dich anzulügen.“

Iril konnte sich nicht dazu überwinden, Ijsdur hier und jetzt zu beichten, dass sie ihn da angelogen hatte.\bigskip







In stille Gedanken versunken, streiften Iril und Ijsdur weiter in Richtung Taverne. Links und rechts von ihrem Trampelpfad zogen Felder andorischer Bauern vorbei. Die wenigsten waren der Zerstörung durch den Drachen vollständig entkommen. In der nächsten würden sich Andori vor allem aufs Jagen und Sammeln von Nahrung, auf Fischfang und Apfelnusspflücken verlassen müssen.

Dann, endlich, tauchte die Taverne hinter einer Hügelkuppte auf. Rauch stieg von der gemütlichen Stube auf. Aus dem Kamin. Dort, wo Rauch üblicherweise aufsteigen sollte. An der Seite des Gasthauses war ein Anhängeschild zu erkennen, auf das jemand das Antlitz eines Trolls geritzt hatte.

„Die ‚Taverne zum Trunkenen Troll‘“, murmelte Iril leise, „Schöne Alliteration. Was da wohl für eine Geschichte dahinter steckt?“

„Verwirrend“, murmelte Ijsdur vor sich hin, „In meiner Sprache ist dies auch Alliteration. Und in Thelot, einer Stadt in Tulgor, gibt es eine bekannte Taverne, die ebenfalls eine solche Alliteration zum Titel trägt. Die ‚Taverne zum Tauchenden Takuri‘. Kann dies ein Zufall sein?“

„Was?!“ Iril machte große Augen. „Das ist auch in der andorischen Sprache eine ... Wie kann das sein? Und in Werftheim ... ah, das sagt dir natürlich nichts ... das ist eine Nebelinsel ... also, eine Insel im Hadrischen Meer ... also, der Ozean im Norden ... da studierte ich ... verzeih, lass mich von Anfang beginnen:

Der Drang, möglichst alles über die Macht der Runen zu erforschen, führte mich vor einiger Zeit in den Norden. Als Runenmeisterin verschrieb ich mich in Silberhall der Lehre der Runenmagie. Ich studierte nicht nur unter Burmrit, der größten Runenlehrmeisterin der Silberzwerge, sondern auch einige Jahre lang unter einigen menschlichen Lehrmeistern auf einer nahe gelegenen Insel namens Werftheim. Die einst so geschäftige Hafenstadt der Insel leidet unter den stetigen Angriffen der Meereskreaturen, doch noch heute beherbergt sie neben der besten Schiffswerft des Nordens auch exzellente Akademie, an der Physikusse ausgebildet werden, wo man auch einige vergangene Schriften der ersten Runendruiden einsehen kann. Während ich dort lebte, kehrte ich gelegentlich in einer Taverne ein und genoss die berühmten Werftheimer Spezialitäten.

Diese Taverne in Werftheim hatte ebenfalls so einen alliterativen Namen. Die ‚Taverne zur Tanzenden Tare‘. Falls du dort mal vorbeikommst, lass die Gelegenheit nicht aus, dort einzukehren. Nebst einer ausgezeichneten Gerüchteküche zu allem Zwielichtigen, das in den stürmischen Gefilden des Nordens vorgeht, bieten die einem dort auch unglaublich erfrischende Massagen mit erhitzten Runensteinen an. Auch wenn einem Eis-Dämon wohl eher ein eiskaltes Eisbad gefallen würde? Egal. Es wunderte mich schon, mit diesem Gasthaus zum Trunkenen Troll einen weiteren solchen alliterativen Tavernennamen zu hören. Nun, wo du noch von Tulgor erzählt hast ...“

„Das kann kein Zufall sein“, gab Ijsdur zurück, „Zumindest wäre es sehr unwahrscheinlich.“

„Doch was für eine Verbindung könnte dahinterstecken? Gut möglich, dass die Tavernen in Sturmtal und in Andor zum Beispiel denselben Rumhändler haben, und dass die eine ihren Namen von der anderen abguckte. Doch Tulgor?! Die beiden Länder wussten doch vor wenigen Minuten noch nichts voneinander, geschweige denn von den anderen Sprachen. Das macht eine kausale Beziehung unmöglich.“

„Das würde ich nicht zwingend sagen“, sagte Ijsdur, „Wir Tulgori wussten zumindest vage, dass auf der anderen Seite des Gebirges Leute lebten. Aber auch gehörnte Bestien und allerlei andere Gefahren. Wir blieben lieber für uns. Doch der eine oder andere Temm wird bestimmt schon hierhergereist sein.“

„Temm? Dieser Begriff sagt mir nichts.“

„Es sind quasi kleine buckelige Menschen. Sie kleiden sich oft in braune Umhänge. Sehr langlebig. Tendieren zu magischen Begabungen. Aber sie werden wohl kaum gesehen, wenn sie das nicht wollen.“

„Ich glaube, ich könnte schon so einen gesehen haben! Wrort hieß er. Er lungerte nach dem Kampf um die Rietburg herum und half einigen Verletzten. Du sagst, der wäre aus Tulgor gekommen?“

„Nun, er hätte das sicher nicht an die große Glocke gehängt. Doch soweit ich weiß, stammen alle Temm ursprünglich aus Tulgor.“

„Und wie genau weißt du das?“

„Nicht schlechter als du.“

„Vertrauenserweckend.“

„Ironie.“

„Angebrachte. Ah, da wären wir schon bei der Taverne. So schnell kann die Zeit vergehen, wenn man schön konversiert, statt sich gegenseitig abzuschlagen“, meinte Ijsdur, „Solche Unterhaltungen habe ich im ewigen Eis vermisst.“

Iril sagte nichts. Sie lächelte aber.

Dann traten die beiden zur grölenden Menschenansammlung vor der Taverne zum Trunkenen Troll.









\newpage
\section{Alte und Neue Helden}

Vor der Taverne zum Trunkenen Troll saß eine alte Frau mit einigen anderen Bauern an einem Tisch. Als Iril an ihr vorbeilief, zückte die Alte einen Becher mit einigen Münzen darin und hielt ihn Iril klappernd vor die Nase.

„Eine Spende für den Wiederaufbau unserer Kate, die Dame? Wir können uns weder die Werkzeuge noch die Hilfskräfte leisten ... oh, was seid Ihr denn für ein Eisklotz?“

„Das ist Ijsdur. Lasst Euch nicht verschrecken, er ist ganz lieb. Neuerdings.“

Iril ließ eine Münze in den Becher der Bäuerin fallen.

Da ertönte ein Klopfen aus dem Inneren der Taverne. Iril stellte sich auf ihre Zehenspitzen und erblickte durch milchiges Glas hindurch ... Orfen, den Wolfskrieger! Der grauhaarige Held blickte zur alten Bäuerin und sprach dumpf durch das Glas:

„Ich hab‘s dir doch schon oft gesagt, Runga: Du musst kein Geld sammeln. Wir Helden werden Euch mit dem Wiederaufbau helfen. Bald, wenn sich die Dunklen Kreaturen etwas beruhigt haben. Und nachdem wir bei den Schildzwergen ordentliche Unterstützung eingeholt haben. Ich kenne einige der Zwerge gut, eine der Wachen am nördlichen Eingang ist ein alter Gefährte von mir! Es sind raue Zeiten, doch werden sie uns helfen.“

Dann blickte Orfen zu Iril herunter und sprach: „He, dich habe ich doch schon mal gesehen. Warst du nicht beim Gestürm um die Drachenleiche dabei? Na, wie gut ist deine Konstitution? Wenn du mich unter den Tisch trinkst, zahle ich dir die Runde!“

„Du schuldest mir doch bereits Gold für drei Runden“, ertönte eine glockenhelle Stimme. Eine rothaarige Dame mit einem mit Metgläsern überfüllten Tablar in der einen Hand und einem breiten Lächeln auf dem Gesicht tauchte hinter Orfen auf und schnippte dem Wolfskrieger spielerisch ans Ohr. Orfen blickte sie an und seine üblicherweise von mürrischen Falten durchzogene Miene glättete sich. Er flüsterte der Wirtin etwas zu, was Iril im allgemeinen Lärm der Tavernengäste nicht genau wahrnehmen könnte. Die Wirtin lachte und schlug sich mit der freien Hand auf den Schenkel.

Iril und der mit den vielen Leuten etwas überfordert wirkende Ijsdur schritten weiter zur Eingangstür und betraten die warme Taverne. Der leckere Geruch heißer Drachenbohnensuppe schlug ihnen entgegen, akustisch unterlegt mit einem Sammelsurium an kaum verständlichen Gesprächsfetzen, manche davon gesungen. 

% [v23.1:] Gleich neben der Eingangstür saß eine mysteriöse Person in einem grünen Gewand alleine an einem Tisch. Sie trug eine Sternkrautblüte im langen goldenen Haar und mampfte nachdenklich an einer blauen Staude mit kleinen runden Beeren. Blaubachbeeren? Als sie Irils Blick bemerkte, lächelte sie fröhlich auf und sprach mit vollem Mund: „Beachte mich einfach nicht. Ich war nie hier.“ Iril beachtete sie nicht weiter, und als ihr Blick das nächste Mal zu diesem Tisch schweifte, war jener verlassen.

Orfen saß immer noch am Fenster, winkte die beiden zu sich und starrte Ijsdur unverhohlen neugierig an. Da war er nicht der einzige. Ijsdur ließ sich auf dem ihm zugewiesenen Sitz nieder und wirbelte Schneeflocken umher. Die Schneespur hinter ihm schmolz zu kleinen Pfützen.

„Na, was bist du denn für einer?“, fragte Orfen.

„Ich bin ein Eis-Dämon. Ich suche die Helden von Andor. Wir müssen sie warnen.“

„Er ist ein Eis-Dämon. Er sucht die Helden von Andor. Er will sie warnen.“

„Wie passend! Manch einer zählt mich zu den Helden von Andor, auch wenn ich nie Brandurs Brosche trug. Setz dich! Wovor willst du uns denn warnen?“\bigskip







Es war Abend geworden. Iril und Ijsdur saßen weiterhin an Orfens Tisch und genossen Gildas Küche. Wie zu erwarten, hatte die Erwähnung Siantaris Orfen kaum eingeschüchtert. Er versprach, bald den anderen Helden die Warnung zu übermitteln. Sie würden Ausschau halten nach der Eis-Dämonin und ihr eine gehörige Lektion erteilen, wenn sie sich in diese Lande wagen sollte.

Zunächst jedoch gab es Speis und Trank zu genießen.

„Wir hätten ihn verspeisen sollen“, murmelte der Wolfskrieger trunken, „Den Drachen. Es hätte uns gestärkt. Und der Körper wäre nicht nutzlos die Narne heruntergespült worden.“

„Drachenfleisch kann Körper und Geist vergiften“, meinte Iril, „Und Drachenfleisch lässt sich nicht so leicht verbrennen. Das bleibt im stärksten Schmiedefeuer roh. Und rohes Fleisch ist gefährlich.“

„Nicht, wenn man es schnell genug verarbeitet.“

„Da werden wir wohl nicht auf einen grünen Zweig kommen.“

„Willst du lieber über einen anderen grünen Zweig reden?“

„Eigentlich ja. Was könnt Ihr mir über diesen Schwarzen Herold sagen?“, fragte Iril neugierig, „An der Rietburg weigert sich irgendwie jeder, genaueres zu ihm zu verraten. Viele wiesen auf dich, Orfen. Du hast die Andori schließlich erst gerade vor ihm beschützt und ihm einen Denkzettel verpasst. Weißt du mehr über ihn?“

Orfens Gesicht verdüsterte sich. „Lieber nicht. Es tut nicht gut, vom Bösen zu schwafeln.“

„Warum nennt man ihn denn den Schwarzen Herold?“, fragte Ijsdur ungeachtet Orfens Aussage. Iril übersetzte und Orfen fuhr sichtlich irritiert fort: „Nun, öhm, er ist in einen schwarzen Umhang gekleidet, und er ist ein Herold des Unheils. Ein Vorbote von Unwettern und feindlichen Angriffen. Ein Diener des Drachen, der seine Ankunft verkündigte. Der Name ‚Schwarzer Herold‘ hat sich halt so eingebürgert.“

„Und was ist er?“

Orfen seufzte tief. Er blickte kurz an die Decke, murmelte ein Stoßgebet an Mutter Natur und brummelte: „Der Schwarze Herold ist eine Sagengestalt. Vieles erzählt man sich über ihn. Angeblich soll er ein uralter Naturgeist sein, der Mutter Natur verraten hat. Oder ein allzu menschlicher Anhänger des Drachen, der in seinem Namen gegen das Königreich vorgeht. Oder auch nur ein verstorbener Kutscher, dessen Stiefel nie blankgeputzt werden können und der deswegen nie das ewige Glück finden wird. Selbst über die Geschichte seiner hohen eisernen Maske gibt es blutrünstige Geschichten.“

Iril hakte nach. „Wie realistisch sind all diese Thesen? Seit wann tritt der Herold bereits als Feind Andors auf?“

„Mindestens seit den Trollkriegen“, murmelte Orfen, „Vielleicht auch schon früher. Aber wir wissen ja noch nicht einmal, ob diese Schreckgestalt immer dieselbe Person war. Es wäre ja auch möglich, dass zur Zeit der Trollkriege ein Vater diese grausige Maske aufhatte und sie nun sein Sohn trägt. Fest steht nur, dass dieser Herold, was auch immer er ist, nicht auf unserer Seite steht. Und dass man ihn so einfach nicht vertreiben kann. Ich weiß das. Ich selbst habe ihn schließlich besiegt, ihm mein Schwert durch die Rüstung gestoßen und gedacht: So, den sind wir los! Aber im Gegenteil, stattdessen kam er gleich wieder zurück und dirigierte den Drachenangriff von dessen Rücken aus. Und nun gesellte er sich zu den Speichelleckern der Drachen.“

„Habt ihr schon versucht, den Herold auf andere Arten zu beseitigen als ein Schwert durch die Brust?“

Orfen druckste herum: „Nun, vielleicht solltet ihr euch dafür an Korbert und Wilselm wenden. Das waren die beiden anderen Wolfskrieger, die das Königsrudel der Wölfe beschützten. Allesamt andorische Hauptmänner. Korbert und Wilselm haben tapfer in beiden Befreiungen der Rietburg gekämpft und dem Schwarzen Herold auch sonst schon getrotzt. Wilselm hat gar seinen linken Arm an ihn verloren. Ich habe dem Herold hingegen nur einige wenige Male gegenübergestanden. Ich setzte mich längere Zeit nicht mehr aktiv für die Rietlande ein. Ich verliebte mich im Gebirge in eine Agren.“

„Wie meine Schwester“, verknüpfte Iril ihre Hintergrundgeschichten, „Ihr Name ist Iolith. Du hast sie nicht zufälligerweise getroffen?“

„Eine Schildzwergin unter den Agren? Nein, daran hätte ich mich erinnert.“

„Schade. Doch was machst du denn nun hier, so weit weg des Gebirges?“

„Nun, das Königsrudel benötigt schon lange nicht mehr den Schutz von Wolfskriegern, eher im Gegenteil. Und nach dem Tod von ...“

Orfens Faust schloss sich um den Griff seines Schwerts. Iril erkannte erst jetzt, dass es sich dabei um ein gerilltes Trollhorn handelte.

„... nun, mich hält jedenfalls nichts mehr dort oben.“

Iril fühlte mit ihm mit.

Ijsdurs klirrende Stimme meldete sich: „Mein Beileid.“

Iril übersetzte schwer schluckend.

„Jetzt lass mich aber nicht ein alter Trauersack sein“, brummelte Orfen, „Was geschehen ist, ist geschehen, meine große Liebe ist gerächt, und nun können wir alle aufs Neue unser Glück finden. Hier etwa, in dieser tollen Taverne. Dieser Ort ist mir der liebste in ganz Andor, und gute Gesellschaft findet man hier allemal. Lass uns von etwas anderem sprechen. Oder singen. Nur zum Schwarzen Herold vermag ich nicht mehr viel zu sagen.“

Iril überhörte das geflissentlich und überlegte: „Nun, wenn der Schwarze Herold eine verirrte Seele ist, die diese Sphäre noch nicht verlassen kann, könnte ich vielleicht aushelfen. Es gibt da eine bestimmte Runenfolge, die wir Silberzwerge nutzen, um verspukte Höhlengänge zu reinigen. Wenn ihr wüsstet, wie viele Geister in Silberhall umgingen, als wir erstmals Minen unter den Silberberg und die restlichen Mauerberge gruben ... die Insel war verflucht, müsst ihr wissen. Ist es immer noch. Wir dürfen die Mauerberge nicht überqueren, oder wir kehren nie wieder. Und die Berge zu unterqueren, birgt manchmal ähnliche Probleme. Etwas sehr Ähnliches nutzte ich, um Ijsdur von Siantaris Einfluss zu befreien. Was ich sagen will: Eine gewisse Runenfolge kann verlorene Seelen aus dieser Welt verscheuchen. Falls es uns gelingt, den Schwarzen Herold an Ort und Stelle zu halten, kann ich versuchen, ihn aus dieser physischen Welt zu vertreiben.“

Ijsdur widersprach: „Der Schwarze Herold klingt wie eine Naturgewalt. Eine solche kann man nicht vernichten.“

„Alles kann man vernichten mit genügend Macht“, widersprach Iril. Kurzzeitig flackerte ein giftgrüner Schein um den Runenhammer in ihrer Hand.

„Darauf trinke ich!“, meinte Orfen, und hob seinen Metkrug.\bigskip







Ijsdur konnte sich tatsächlich problemlos in der Taverne einquartieren. Iril tat es ihm nach. Die nächsten Tage verbrachten sie in emsigen Gesprächen über Tulgor und Andor, Eis-Dämonen und Zwergen, Siantari und Tarok, und wie man in dieser Gegend am besten aushelfen konnte. Und bald schon war mehr als ein halber Mond vergangen seit der Befreiung der Rietburg und dem Tode Taroks.

Eigentlich sollte nach dem Tod König Brandurs Prinz Thorald der neue König werden. Doch das Volk stand vor dringlicheren Aufgaben. Und so trat die Krönung in den Hintergrund. Thorald hatte die Rietgraskrone noch nicht angenommen und nutzte bislang weiterhin den Titel Prinz. Immer seltener sah man ihn. Gerüchte mehrten sich, dass er wie schon in seiner Prinzenzeit lieber eine gewisse Bäuerin am Fuße des Gebirges aufsuchte, statt sich in der weit entfernten Rietburg um die Regierung sein Königsreich zu kümmern.

Doch nun hatte Thorald zu einer großen Veranstaltung beim Sommerfels am Narnenufer aufgerufen. Fast jeder Bewohner der Rietburg, der laufen konnte, brach dorthin auf – und auch manche, die das nicht konnten.

Orfen hatte Iril und Ijsdur ebenfalls zur großen Versammlung eingeladen. Die beiden waren aus Interesse gekommen. Und auch, weil Iril ahnte, dass die Drachenkultisten der Jpaxo sich zu einem solche Zeitpunkt am ehesten wieder zeigen würden. Achtsam sondierte sie die Menschenmenge.

Schon von weitem war der Trubel an Menschen zu sehen und zu hören, der sich um den Sommerfels herum eingefunden hatte. Manche hatten den gewaltigen Findling gar erklommen. Nahe der Narne war ein behelfsmäßiges Podium aufgebaut worden, auf dem Prinz Thorald stand und unruhig umherlief. Hinter ihm flatterte eine große Flagge im Wind. Sie trug das andorische Wappen, die Sternblume auf rotem Grund.

Rund um das Podium herum sondierten wachsame Wachen die Menge.

Die Menschenmasse blickte indes interessiert hinter Prinz Thorald. Dort lag eine hölzerne Kiste, mit zahlreichen schönen Mustern versehen und einem weißen Totentuch darüber drapiert. Mannsgroß war die Kiste in der Länge. Iril hatte eine Vermutung, welcher Mann darin lag.

„Erinnert ihr euch an das Lied vom Blutstrom?“, begann Prinz Thorald seine schallende Ansprache.

Iril und Ijsdur blickten sich an und schüttelten ihre Köpfe. Die meisten anderen Anwesenden um sie herum nickten hingegen. Gilda die Wirtsfrau und Grenolin der Barde traten fröhlich zwei Schritte näher ans Podest.

„Wir haben nicht die Zeit und Muße, das Lied hier vorzutragen“, sprach Thorald weiter. Grenolin und Gilda ließen enttäuscht ihre Köpfe senken.

„Doch ist die Ballade vom Blutstrom eines der schönsten Werke, welches die Heldentaten desjenigen Menschen ehrt, der schon seit seiner Kindheit für unsere Freiheit kämpfte. Ungeheuer tapfer und mutig war er. Brandur! Ein Anführer! Ein Held! Ein König! Ein Va ...“

Thoralds Stimme stockte. Unruhig blickte er über die Menge. Iril folgte seinem Blick. Unter anderem fiel ihr die Heldin Chada auf, welche ein silbernes Amulett fest umklammerte und starr vor sich hinblickte. Thorn hatte seinen Arm um sie gelegt und kuschelte sich beruhigend an sie.

Thorald fing sich wieder und sprach weiter: „Eine lange Zeit hat mein Vater diesem Königreich gedient. Länger, als wir alle es ihm zugetraut hätten. Er flüchtete mit einer Horde Sklaven aus dem finsteren Reich der Krahder. Er hätte sich Tarok tapfer geopfert, um seiner verlotterten Schar mehr Zeit zu verschaffen, doch stattdessen verpasste er dieser Riesenechse einen Denkzettel, den sie ihr Leben lang nicht mehr vergaß! Er ließ für unser Volk eigenhändig ein sicheres Lager bauen und später eine ganze Burg. Er sorgte für ein sicheres Reich. In den Trollkriegen rettete er das ganze Land mehrfach. Er suchte den Frieden, selbst mit denen, die ihn verachteten. Und nie gab er sein eigen Fleisch und Blut auf, selbst, nachdem es ihn hintergangen hatte.“

Thorald rieb sich gedankenverloren die Rippe. Iril hatte davon gehört, wie Brandurs lange tot geglaubter Bruder Hademar vor nicht einmal einem ganzen Jahr wieder aufgetaucht war und seinen Neffen hatte von einem Schwarzen Ritter abstechen lassen. Thorald schien ihm im Gegensatz zu Brandur definitiv nicht vergeben zu haben.

„In seinen ersten Jahren hier in Andor sprang Brandur bereits in die Narne, um diese unvorsichtigen hadrischen Zauberer vor dem Ertrinken zu retten. Sie schenkten ihm zum Dank das Ewige Feuer. Dessen heilige Schale wurde damals vor einem kleinen Lager errichtet, beschützt nur durch eine schwache Holzpalisade. Heute steht das Ewige Feuer vor der Rietburg, dem prächtigsten und sichersten Bauwerk des bekannten Kontinents!“ – Iril hörte einige Schildzwerge lautstark murren – „Und das Ewige Feuer zeugt noch heute von Brandurs Tapferkeit! Unsere Lieder zeugen davon! Unsere Leben zeugen davon! Wir verdanken ihm mehr, als wir ihm je zurückzahlen könnten.“

In der Menge johlte jemand auf. Thorald nutzte die Gelegenheit, um sich eine Träne aus dem Auge zu wischen.

„Und nun ist er von uns gegangen. Nicht friedlich im Schlafe, sondern im Fieberwahn, hinterhältig verwundet durch Handlanger eines feigen Feuerdrachen! Das ist ungerecht. Doch das Leben ist nun einmal ungerecht. Es gibt nichts mehr zu rächen. Danken wir Mutter Natur dafür, dass sie Brandur so lange unter uns weilen ließ. Dass er dem Tod so oft ein Schnippchen schlagen durfte, wie er konnte. Verfluchen wir seine Mörder. Und lassen wir uns von ihm als Beispiel leiten. Nun ist die Zeit gekommen, dass Vater die Last der Krone absetzen kann. Nun ist die Zeit gekommen, wo Vater meine Mutter wiedersehen kann. Und seinen treuen Harthalt. Und Terba von den Flusslanden. Ja, selbst den viel zu früh von uns gegangenen Janner. All jene, die er in den letzten Jahren von sich gegen lassen musste. Nun ist die Zeit gekommen, wo König Brandur von Andor sich zurücklehnen und das ewige Glück genießen darf.“

Theatralisch schritt Thorald näher ans Ufer der Narne und die kleine Treppe herunter, die an dieser Seite in die Böschung gebaut worden war. Die Gischt des wirbelnden Narnenwassers benetzte ihn. Die unteren Spitzen seines purpurnen Umhangs wurden von den Wassergeistern getränkt, doch kümmerte ihn dies nicht.

Zwei Wachen trugen Brandurs Sarg näher. Die Traditionen Andors verlangten eigentlich, dass Tote am zweiten Sonnenaufgang nach ihrem Tod an ein Floß gebunden und die Narne heruntergeschickt wurden. Doch am zweiten Sonnenaufgang nach Brandurs Tod hatten die Helden noch immer gegen Horden anstürmender Kreaturen und einen wütenden Drachen gekämpft. An ein Totenritual war nicht zu denken gewesen.

Und für Könige galten vielleicht ohnehin andere Regeln. Erst recht für erste Könige eines neuen Reiches, dessen Regeln noch nirgendwo festgeschrieben standen. Statt dass Brandurs Leichnam in ein weißes Totentuch eingeschlagen wäre, war das weiße Tuch über seinen Sarg drapiert worden. Statt auf ein Holzbrett geschnallt zu sein, umgaben sechs Holzflächen die Leiche von allen Seiten. Vielleicht auch, um den Geruch nach Krankheit und Tod zurückzuhalten. Und statt zwei Tagen, hatte Thorald über zwei Wochen mit Brandurs Totenrital auf sich warten lassen. Doch die Zeit des Wartens schien vergessen, als alle anwesenden Andori andächtig ihre Häupter senkten und den Thronfolger anblickten.

Thorald beugte sich nieder und flüsterte eine leise letzte Botschaft an den gefallenen König. Dann gab er dem Holzsarg einen Stoß und starrte melancholisch hinterher, wie er von der Strömung geleitet dem offenen Meer entgegenschaukelte. Diesmal protestieren die Wassergeister der Narne – im Gegensatz zu Tarok – nicht gegen den Toten, sondern schmiegten sich an den Sarg und gaben heulende Geräusche von sich. Trauerten selbst sie um den hohen König?

Iril sondierte ihre Umgebung. Die Erwähnung einer Krone in Thoralds Rede hatte sie aufgeregt nach der Rietgraskrone umgucken lassen. War es möglich, dass sie die Krönung des zweiten Königs von Andor miterleben durfte? Doch die Rietgraskrone war nirgends zu sehen, und eine Krönung war auch nicht die Zeremonie, zu der Thorald als nächstes kommen sollte. Nein, sobald Brandurs Sarg nicht mehr zu sehen war, erklomm Thorald wieder sein behelfsmäßiges Podium und wandte sich der Menge zu. Seine wässrigen Augen straften sein breites Lächeln Lügen. Dennoch gab er sich die größte Mühe, seine tiefe Stimme ruhig zu halten, während er weitersprach.

„Und nun kommen wir zu etwas Fröhlicherem! Beinahe sechs Jahre ist es her, da ernannte mein Vater vier Helden von nah und fern zu Helden von Andor. Chada, Thorn, Eara und Kram. Ihnen wurden die letzten vier von Wulfrons Heldenbroschen anvertraut. Ich sehe die metallenen Sternblumen auch heute noch an ihren Umhängen hängen!“

Erneut brandete Jubel auf. Die vier genannten Helden wurden aus der Menge nach vorne geleitet.

„Doch nicht nur sie haben in den letzten Tagen, ja, in den letzten Jahren Tapferkeit bewiesen. Es mögen keine weiteren Heldenbroschen mehr übrig sein, doch Titel sind unbegrenzt. Es gibt neue Helden von Andor zu ernennen! Die unzertrennlichen Fenn und Hogo, kommt nach vorne! Die tapferen Jarid und Trieest aus dem fernen Dandar, ohne die die Befreiung der Rietburg niemals geglückt wäre! Die ehrenvolle ...“

Während Thorald weitere Namen aufrief, blieb Irils Blick bei dem ungleichen danwarischen Heldenpaar hängen. Eine stolze Wassermagierin in einem eleganten blauen Gewand mit goldenen Mustern. Ein an eine Kreatur erinnernder hünenhafter Krieger mit einem leuchtenden Stein in der Brust. Ihre Geschichte wäre bestimmt interessant zu erfahren.

„... und das waren alle!“, beendete Thorald seine Liste.

Ein Klatschen ging durch die Menge. Iril sah viele lächelnde Gesichter, aber auch einige unglückliche. Ein rundlicher Zwerg mit umgeschnalltem Signalhorn ließ enttäuscht seinen Kopf sinken, während seine Nachbarin ihm tröstend auf die Schulter klopfte. Nicht jeder tapfere Kämpfer hatte die Ehre, zum Helden ernannt zu werden.

Orfen war auch nirgends mehr zu sehen. Nun, so wie Iril ihn kannte, machte es ihm wohl nichts aus, übergangen zu werden.

„Und denkt natürlich nicht, dass die alten Helden leer ausgehen würden!“, rief Thorald freudig,

„Ihr alle werdet nicht nur zu Helden, sondern auch zu Fürsten von Andor ernannt! Möge der Titel euch etwas bedeuten. Ihr seid nun erst recht verantwortlich für unser Volk. Setzt euch für unsere Freiheit ein und für die gleichen Rechte aller. Helft, wo ihr könnt, und man wird euch helfen, wo man kann. Ich lasse euch allen ein Haus in der Rietburg bauen. Mard, der Baumeister, steht schon bereit, um mit euch die Details auszuhandeln. Und fünf...“

„Danke, lieber Thorald, für diese großzügige Geste“, unterbrach ihn Thorn, ehe Thorald den Helden noch mehr Dinge versprechen konnte, die sie gar nicht wollten. Iril vermutete, dass den meisten Helden dieser Titel, „Fürst“, wenig bedeutete, und dass sie den Häuserdeal später unter zwei Augen ein wenig allgemeiner modifizieren würden.

Gesonderte Heldenhäuser in der Rietburg würden wohl den Großteil der Zeit nur leer stehen, und Mutter Natur wusste, wie viele Andori dort schon jetzt in diesen Zeiten der Not kein Dach über dem Kopf fanden. Helden streiften doch lieber frei durchs Land, wohin auch immer sie der Ruf der Hilfsbedürftigen zog. Manche würde es an die Küste im Norden mit Blick auf das Hadrische Meer ziehen, andere ins Rietland im Süden, und wieder andere in den Wachsamen Westerwald. So war es halt mit den Helden.

Iril konnte sich durchaus vorstellen, ein solches Leben zu führen. Ob sie selbst wohl eines Tages diesen Titel erhalten würde? Sie stellte sich vor, wie es wäre, vor dieser gewaltigen Menge auf einem Podest zu stehen, während der Prinz sein Schwert auf ihre Schulter legte und die zeremoniellen Worte sprach ...

„Pst. Sollten wir das melden?“, klirrte es hinter Iril. Irils Gedanken waren von Ijsdur unterbrochen worden, der sich zu ihr heruntergebeugt hatte und ihr mit einem eiskalten Zeigefinger in die Schulter stupste. Verstohlen zeigte er auf die Rietburg. Iril folgte seinem ausgestreckten Arm und sah, wie das leuchtende Ewige Feuer auf dem hohen Hügel in der Ferne aufflackerte. Violette Funken mischten sich in orangenes Feuer.

Von einer vom Freien Markt geflohenen Händlerin hatte Iril letzthin deren letztes Fernrohr erstanden. Nun studierte sie durch dessen grobe Linsen nachdenklich das flackernde Ewige Feuer. Thoralds feierliche Zeremonie zu unterbrechen, um die Helden auf etwaige Gefahren aufmerksam zu machen, war schon keine schöne Manier. Doch wäre es nicht auch fahrlässig, das drohende Feuer nicht zu melden?

Durch die dicken Gläser des Fernrohrs hindurch erhaschte Iril eine Bewegung. Da! Gerangel auf der Burgmauer der Rietburg! Zwei Gestalten im Gefecht! Dann schleuderte die eine Person den anderen Menschen über die Brüstung in die Tiefe. Der Boden war viele Schritte entfernt, einen Sturz würde er niemals überleben.

Oder etwa schon? Die zweite Gestalt sprang waghalsig von der Burgmauer und landete scheinbar unversehrt auf dem harten Felsen, auf dem die Rietburg errichtet worden war. Selbst durch das Fernrohr hindurch war die Gestalt so klein, dass sie kaum mehr als ein Flackern im Gras darstellte, doch bewegte sie sich unzweifelhaft weiter, als hätte sie den gewaltigen Sprung ohne den geringsten Kratzer überstanden.

„Siehst du das auch?“, fragte Iril, „Magie?“

„Ich sehe es auch“, bestätigte Ijsdur, „Vielleicht ist es Magie. Aber ein Mensch ist das sicherlich nicht. Sieh dir die Hörner an.“

Ja, inzwischen glaubte auch Iril, an der von der Burg fliehenden Gestalt zwei lange Hörner zu erkennen. Und sie war schneeweiß.

„Ist das etwa Siantari?“, fragte Iril.

Ijsdur grinste. „Kaum. Wenn schon, hat sie sich in den letzten beiden Wochen sehr stark verändert.“

Iril blickte hektisch zurück zu Thorald, welcher sich soeben räusperte und mit feierlicher Stimme die Heldenernennung zu intonieren begann: „Andori! Wir wurden vom Bösen heimgesucht und haben schreckliche Verluste erlitten. Wir werden das Opfer aller Verstorbenen und Verletzten in Erinnerung behalten. Ihr alle habt euer Leben riskiert. Ihr alle seid Helden. Aber manche haben ganz besonders heldenhaft gehandelt. Ihr Helden, die ihr vor mir steht, habt die Rietburg eigenhändig aus der Hand der Dunkelheit befreit. Ihr habt den letzten Drachen niedergerungen und alle Lande von dieser Pein befreit. Ihr seid diejenigen, die einen Sieg über das Böse ermöglicht haben. Ihr habt immer das Wohl aller Andori im Sinn gehabt. Und von heute an und für immer mögt ihr Helden von Andor sein.“

Er zückte sein zeremonielles Schwert, machte sich auf zum linksten Helden – zu Fenn dem Fährtenleser – und sprach „Bitte knie nieder.“

Fenn tat wie gebeten und senkte seinen Kopf ehrfürchtig vor dem Prinzen.

Ehe Iril ein Weg einfiel, wie sie den Prinzen oder einen Rietgardisten möglichst störungslos auf die unschönen Geschehnisse auf der Burg hinweisen konnte, explodierte auch schon über Ijsdur ein Schauer an Eis- und Schneekristallen. Sobald Ijsdur die Aufmerksamkeit des Publikums und der Helden – und eines verdutzt dreinblickenden Thoralds – hatte, rief er schallend: „Seht! Rietburg! Dieb!“

Wie toll, dass er in der letzten Woche doch schon einige andorische Wörter aufgeschnappt hatte!

Der Dieb mit den langen Hörnern war gerade noch lange genug sichtbar, damit Ijsdur nicht als Lügner durchging. Dann war er nicht mehr zu sehen.

Thorald schickte sofort Boten zur Rietburg, doch bestätigten diese nur den schon gehegten Verdacht einiger Anwesenden.

Die Drachenkultisten hatten zugeschlagen.

Soeben waren Taroks letzte Knochen aus der Schatzkammer der Rietburg geklaut worden.\bigskip







Die Heldenzeremonie wurde rasch zu Ende gebracht, mit einem Versprechen auf eine gewaltige baldige Feier, sobald die aktuelle Situation geklärt wäre.

Thorald ließ die Brücken ins östliche Rietland sperren, da er stark vermutete, dass die Kultisten dorthin fliehen wollten. Sie kamen schließlich bekanntlich aus den nördlichen Ausläufern des Grauen Gebirges, weit im Osten.

Die Helden von Andor, sowohl neu ernannte als auch alte, schwärmten aus und versuchten, die Schuldigen zu finden. Alle hatten ihre eigenen Ansätze. Fenn der Fährtenleser saß in den Schneidersitz und ließ seinen sprechenden Raben Morar ausfliegen. Tenaya, die Wächterin des Feuers, zündete einen tiefschwarzen trockenen Ast an und blickte in die Flammen, als könnten sie ihr etwas verraten. Hinter ihr rollte Flaps der Flederfuchs über den Boden und gab quiekende Laute von sich. Chada, die Bogenschützin, ritt an der Seite von Thorn in den Süden weiter und rief etwas darüber aus, wie toll es jetzt wäre, wenn sich ihr Gezähmter nicht ins Gebirge zurückgezogen hätte. Wobei sie für ‚Gezähmter‘ nicht das übliche Wort nutzte, sondern ein sehr altes. ‚Lonas‘. Faszinierend. Doch das war ein Rätsel für eine spätere Zeit.

Auch Ijsdur und Iril blieben etwas ratlos zurück. Nun gut, eigentlich war nur Ijsdur planlos. Denn Iril hatte einen Plan.

Die beiden zogen sich vom restlichen Klamauk am Sommerfels etwas zurück, bis aus dem Stimmenwirrwarr der tratschenden Andori nur ein leises Rauschen im Hintergrund geworden war.

Iril suchte sich einen Flecken Boden, an dem das goldene Rietgras bereits niedergetrampelt war. Sie langte in ihre Reisetasche und zog eine etwa handgroße kreisrunde Steinscheibe hervor, welche in verschiedene Speichen unterteilt war. Diese Runenscheibe war erheblich dicker und massiver als die zahlreichen dünnen metallenen Runenscheiben, die sie mit sich führte. Ihre Oberfläche war durch zahlreiche Kritzeleien verkratzt, welche einst Runen gewesen waren. Doch die Zeichen waren alle verblasst, kaum erkennbar, und überlagerten sich völlig. Nur am Rand der Scheibe waren deutliche, tiefere Runen zu erkennen, welche in ein Muster aus eckigen Schneckenhäusern und kantigen Wurzeln eingeschlossen waren.

Noch glühten die Runen nicht, weder die in der Scheibe noch die auf dem Hammer noch die unter Irils Armhaut. Einzig die Übersetzungsrunen an Irils Kopf und auf Ijsdurs Brust schillerten aktiv.

Iril setzte sich in den Schneidersitz auf den Boden und holte einen kleinen spitzen Metallstab hervor. Damit begann sie, frische Zeichen in eine Speiche der Runenscheibe zu ritzen. Währenddessen murmelte sie vor sich hin. Als Ijsdur ihr nach fünf Minuten immer noch regungslos zusah, hielt sie inne und erklärte:

„Ich werde die Runen befragen, wo die Geflüchteten gelangt sein könnten. Die Runen erlauben mir manchmal einen Blick auf ferne Geschehnisse. Träger gewisser Blutstropfen zu finden ist besonders leicht. Aber auch wenn wir kein Blut des Flüchtenden haben: Sofern starke Magiequellen involviert sind, wie Drachenknochen, sollte es mir auch möglich sein, sie zu orten. Doch mag dieses Vorgehen eine Zeit lang dauern. Du musst mir nicht die ganze Zeit dabei zusehen.“

„Stehen macht mir nichts aus. ‚Runen befragen‘?“, fragte Ijsdur, „Das klingt ja beinahe so, als könntest du mit ihnen kommunizieren. Können die Runen sprechen?“

Iril gluckste auf: „Das dachte ich zu Beginn meines Studiums auch. Zu lebendig schienen mir die Wunder, die die Runenmeister der Silberzwerge mit ihrer Hilfe schufen.“

Ihr Blick verklärte sich.

„Aber ebenso wenig, wie die Magie selbst lebendig ist, sind es auch die Runen. Es sind nur einfache Zeichen in komplexen Zeichenfolgen, die wir in Objekte ritzen, um mit ihnen magische Ströme zu lenken. Und wahrlich wundervolle Dinge zu erschaffen, die bei genügend großer Komplexität ununterscheidbar von Leben sind, aber eben doch nicht leben.“

Ijsdur sah Iril einen Moment lang stumm an. Dann meinte er: „Nein, das ist nicht wahr. Magie ist doch lebendig. Das ewige Eis ... Siantari ... wann immer wir unsere Kräfte nutzen, dann spüren wir doch, wie ein fremder Geist den unseren streift.“

„Falls hinter diesen Formen der Magie ein Geist steckt, dann ist er zu weit von unserem entfernt, um von uns als solcher verstanden zu werden. Guck dir zum Beispiel diese drei Runen hier an. Die erste saugt ein wenig Magie aus der Umgebung an und leitet sie in den Osten. Die zweite da drüben teilt einen Strom der Magie in drei gleiche Teile. Die beiden linken Teile können wir dann auf diese Art wieder vereinen, um die nächste starke Quelle der Magie zu orten, wie mit einem Kompass. Und den rechten Teil nutzen wir, um die Scheibe selbst schw....“

Iril bemerkte, dass Ijsdur ihren Gedanken schon lange nicht mehr folgte.

„Verzeih mir, ich gerate wieder ins Schwurbeln. Ich meine nur: Ich ritze mit diesem Hammer bestimmte Runen in die Oberfläche, die die allgegenwärtigen magischen Ströme der Energie nutzen, bündeln, allerlei solche Dinge. Und die Runen tun genau das, was wir ihnen so auftragen, verlässlich, folgsam, manchmal für viele Jahrzehnte, ehe sie wieder verblassen. Sehr komplexe Dinge, aber keine intelligenten. Wenn ich nur einen kleinen Fehler in meinen Überlegungen mache, funktioniert nichts mehr so, wie von mir gewollt. Das kann doch kein Leben sein.“

„Kann es sehr wohl“, sprach Ijsdur ruhig, „Die magischen Ströme, die du in diese Muster zwingst, sprudeln nur so von Leben.“

„Ich vermute, dass du eine sehr andere Vorstellung von diesem Begriff hast als ich.“

Iril beendete die Arbeit an der Runenscheibe. Dann zückte sie den grünlich schimmernden Runenhammer und ließ ihn mit Gedöns auf die Steinscheibe niederfahren. Statt dass die Scheibe zersplitterte, wie Ijsdur erwartet hätte, glühten die neu in die Scheibe geritzten Runen grünlich auf.

Ganz kurz.

Dann erloschen die Runen wieder.

Iril stöhnte auf. „Es ist immer ein bisschen ein Glücksspiel, ob es beim ersten Mal funktioniert. Eine einzige falsch geritzte Rune kann das Ergebnis völlig verändern, und wenn man müde ist, häufen sich Flüchtigkeitsfehler nun mal. Wer weiß, vielleicht ist soeben irgendwo im Lande ein klitzekleiner Regenschauer niedergegangen.“

„Du kannst mit diesen Runen Regen rufen?“, fragte Ijsdur. Sein Tonfall blieb kalt, doch Iril kannte ihn inzwischen gut genug, um seine Überraschung zu deuten.

„Ein paar Tropfen vom Himmel holen trifft es eher, aber ja, das können die Runen. Und nicht nur das. Trockene Brunnen auffrischen, Nebelschleier lüften, versiegte Quellen aufs Neue sprudeln lassen ... die Runen vermögen so einiges. Wasser zu lenken ist ein bisschen meine Spezialität. Vielleicht hat es etwas mit meinem Runenhammer zu tun. Der Legende nach wurde dieser als allererstes von einer Nixe geschwungen.“

„Was ist eine Nixe? Dieses Wort erschließt sich mir trotz deiner Übersetzungsrune nicht.“

„Hm. Vielleicht gibt es für dieses Wort gar kein tulgorisches Äquivalent. Denk an einen Fisch-Menschen. Den Oberkörper eines Menschen und der Körper eines Fisches. Oder vielleicht eher einer Seeschlange?“

„Faszinierend. Wie in aller Welt sind entstehen solche Wesen?“

„Nun, wenn eine Nixe und eine andere Nixe einander ganz fest lieb haben ...“

Kopfschüttelnd korrigierte Ijsdur: „Nein, ich meine, die ganze Spezies. Solche Wesen können ja kaum von Fischen oder von Menschen allein abstammen, dafür sind sie einander zu ähnlich. Doch Fische und Menschen können keine gemeinsamen Nachkommen zeugen, wie es beispielsweise Temm und Feen können. War Magie im Spiel? Sagt, sind diese Nixen kompatibel mit Fischen oder mit Menschen?“

„Nun, es könne sich auf jeden Fall Menschen und Nixen ineinander verlieben.“

„Das sagt doch noch nichts aus. Ich könnte mich doch auch in einen Kieselstein verlieben.“

„Was?! Selbst wenn, ein Kieselstein kann dich nicht in irgendeiner Art und Weise zurücklieben, die diesen Namen verdient hätte.“

„Dann lassen wir den Kieselstein doch Kieselstein sein und fragen, ob Menschen wie du ein Kind mit einer Nixe haben könnten.“

Iril kannte die Antwort auf diese Frage nicht, doch hatte sie hier ohnehin etwas richtigzustellen.

„Warte mal. Ich bin gar kein Mensch. Ich bin ein Zwerg.“

Ijsdur hob überrascht eine durchscheinende Augenbraue, sagte aber nichts.

„Was, dachtest du etwa, ich sei ein kleinwüchsiger Mensch mit breiterem Gesicht, größerer Nase, Dunkelsicht und einem ungewöhnlich langen Leben? Ich bin 72 Jahre alt!“

„Du zählst das? Ich habe keine Ahnung, wie alt ich genau bin. Vermutlich weniger als 72 Jahre.“

„Vermutlich. Ich habe schon Runen gelehrt, da wussten die meisten Menschen hier noch nicht mal den Namen ihrer Eltern. Und ich werde vermutlich noch hier sein, nachdem alle heute lebenden Bewohner der Rietburg die Narne hinunter gingen. Außer vielleicht einige der jungen Zwerge, die sich den Andori angeschlossen haben.“

Ijsdur lachte: „Nicht nur die Zwerge leben so lange. Auch ich. Ich bin ein ganz frischer Eis-Dämon. Der größte Teil meines 500 Jahre langen Lebens liegt noch vor mir.“

„Ich nehme alles zurück, was ich über dein kurzes Leben sagte. Ändert aber nichts daran, dass du mich als Nicht-Menschen hättest erkennen können.“

„Ich kenne weder deine Lebensspanne noch deine Dunkelsicht. Und so ungewöhnlich ist dein Aussehen nicht. In Tulgor, meiner Heimat, gibt es auch sehr kleine Menschen mit breiten Nasen.“

„Vielleicht sind das auch einfach Zwerge“, bedachte Iril.

„Faszinierender Gedanke. Was unterscheidet denn einen Zwerg von einem kleinen Menschen?“

„Nun, ein Zwerg wird von Zwergen geboren, ein kleiner Mensch halt von Menschen – die eventuell auch klein sind.“

„Das scheint mir ein wenig ein Takuri-und-Ei-Problem zu sein.“

„Ein was?“

„Egal. Sagt, sind Menschen und Zwerge kompatibel im Sinne davon, wie Braunbär und Schwarzbär es sind? Können sie gemeinsame Kinder kriegen?“

„Ich glaube schon. Aber solche Liaisonen waren lange Zeit verpönt in diesen Landen. Zwerge und Menschen mochten einander eine lange Zeit lang nicht sonderlich.“

„Wirklich, ALLE Menschen und Zwerge mochten einander nicht?“

„Naja, offensichtlich mochten ein paar schon einander, sonst wäre diese Beziehungen nicht verpönt gewesen.“

„Und sind die Kinder, die aus diesen Verbindungen entspringen, Menschen oder Zwerge?“

„Je nachdem. Sie sind halt Teil des Volkes, in dem sie aufwachsen.“

„Also sind diese Begriffe mehr eine Volksbezeichnung?“

„So könnte man es sehen. Interessant, so hatte ich noch nie darüber nachgedacht. Aber ein Mensch, der unter Zwergen aufwächst, ist immer noch kein Zwerg. Er wäre immer noch viel größer als die anderen. Und würde schneller altern.“

„Bist du sicher? Altert JEDER einzelne Mensch schneller als JEDER Zwerg?“

„Nein, natürlich nicht. Der Oberste Bewahrer Melkart ist beispielsweise bekanntlichermaßen schon viel älter, als man einem Menschen seines Aussehens zutrauen würde. Und dann erst Brandur und Reka, die beiden geflohenen ...“

„Soll das heißen, dass sie vielleicht Zwergenblut in sich tragen?“

„Oder dass sie einfach diejenigen Ausnahmen sind, die die Regel bestätigen.“

Iril und Ijsdur blickten einander in Stille an. Ijsdur brach sie wieder.

„Verzeiht, ich habe dich unterbrochen. Du wolltest noch etwas von einer Nixe mit dem Runenhammer erzählen?“

Iril lachte auf: „Tatsache! Wir sind völlig abgeschweift.“

Sie setzte erneut an, die Hintergrundgeschichte ihres Runenhammers zu erzählen. Diesmal war Ijsdur keine Einwürfe ein.

„Die Insel Hadria hoch im Norden ist das Land der Zauberei und der Magie. Dort ist die Magie allgegenwärtig. Sie steigt wie unsichtbarer Dampf aus der Hadrischen Unterwelt auf. Und so werden dort überproportional viele Kinder mit magischen Talenten geboren. Darunter auch manche Nixen. Und diese eine Nixe, deren Name inzwischen verschollen ist, fand einen verborgenen Zugang zur Hadrischen Unterwelt. Sie schwamm mit einem gewöhnlichen Hammer in die Unterwelt hinein und mit einem machtvollen, mit einigen Runen bedeckten magischen Hammer wieder nach ans Tageslicht. Seither birgt dieser Hammer einen Teil Dunkler Magie. Dunkle Magie aus dem weit entfernten Hadria, dem ich nur dank des Zwergenbluts in meinen Adern nicht erliege. Wir Zwerge scheinen irgendwie gehörlos zu sein, was die verführerische Stimme der Dunklen Magie angeht. Noch ein Unterschied mehr zwischen Menschen und Zwergen.

Damals war das Wissen um die Dunkle Magie bei den Zauberern noch nicht vertreten. Oder sie wurde zumindest noch nicht als separates Konzept von der Zauberei erforscht. Sie hatte noch nicht einmal einen Namen. Das kam erst zu Orweyns Zeiten. Ja, lange vor Orweyn war Hadria noch ein blühendes Land ohne ewigen Schnee und ohne nie weichendes Eis. Und die hadrischen Nixen mussten sich noch nicht wie heute mit Blubber gegen das Erfrieren schützen.

Nach dem Tod der Erschafferin des Hammers erhoben immer wieder Zauberer Ansprüche darauf. Irgendwann gaben ihre Nachkommen nach. Der Hammer wurde an Land als besonderes Artefakt studiert. Und dann, eines Tages, fand er seinen Weg in die Minen Caverns. Zu einer bestimmen Runenmeisterin der Schildzwerge namens Golja. Manch einer will glauben, dass es ein gerechter Handel war, doch höchstwahrscheinlich ward der Runenhammer gestohlen aus den Hallen der Zauberer. Vielleicht ist das ein Glück. Denn nach der Entstehung der zwei Zaubererorden wäre er bestimmt in den Eisernen Turm gesperrt worden. So aber konnten die Runenmeister der Schildzwerge von ihm lernen, ihn hegen und pflegen, und für allerlei nützliche Gerätschaften nutzen. Und mit immer mehr Runen versehen. Unser Zwergenblut mag uns vor der anreizenden Stimme der Dunklen Magie schützen, doch es sind die Runen, mithilfe derer wir des Hammers unglaubliche Macht bündeln und steuern können. Wer die Geheimnisse der Runen enthüllt hat, muss kein magisches Talent besitzen, um Magie nutzen zu können. Es ist Tradition, dass der letzte Träger des Runenhammers ihn in seinem Testament einem seiner Schüler vermacht. Und meine Runenmeisterin, Burmrit ...“

Irils Stimme versagte. Doch sie musste nicht mehr sagen, Ijsdur hatte auch so verstanden.

In Stille arbeitete sie weiter an ihrer Runenscheibe.

Schließlich runzelte sie ihre Stirn und musterte die neu gekritzelte Runenfolge. Nachdenklich öffnete sie ihren Reisesack und zog einen breiten silbernen Gürtel hervor, entlang dessen Länge sechzehn Runen abgebildet waren. Einige waren gestickt, andere nur schwach darauf gezeichnet. Manchmal standen noch weitere kleine Kribbel nebendran. Iril überprüfte etwas darauf und glich es mit ihrer Runenscheibe ab. Dann nickte sie und korrigierte zwei Striche auf der Runenscheibe

Ein zweites Mal haute Iril mit dem Hammer auf die bearbeitete Speiche der Runenscheibe. Erneut sprang ein grünlich schimmernder Funke auf die Scheibe über. Diesmal blieben die Runen grünlich leuchtend. Ein leises Summen ertönte.

„Jawohl!“, rief Iril fröhlich auf und klatschte in ihre Hände.

„Die Runen allein reichen meistens nicht, um Magie dauerhaft in einem Objekt zu halten“, erklärte sie, „Es braucht einen Startschuss, der die ersten Ströme durch sie leitet. Mondlicht eignet sich besonders gut für längere Werke. Sogar besser als Sonnenlicht. Ich vermute, dass der Mond ein magische Ströme amplifizierender Reflektor ist. Aber um kurzweilige Werke wie dieses hier mit Strom zu erfüllen, reicht ein solcher Hammer völlig aus.“

Ijsdur nickte nur staunend.

Die leuchtende Runenscheibe erhob sich aus Irils ausgestreckter Hand einige Fingerbreiten in die Höhe und drehte sich zitternd im Kreis. Dann verharrte sie in einer bestimmten Orientierung. Iril nickte.

Sie fasste die Scheibe wieder und schlug sie erneut mit dem Runenhammer, diesmal jedoch an der Seite. Die am Scheibenrand angebrachten Runen begannen, hellblau zu glühen, während diejenigen innerhalb der Speiche erloschen und verblassten. Iril zückte einen rauen Stein aus ihrer Tasche und schmirgelte damit an der bearbeiteten Speiche herum. Der Stein wich dem Schmirgelstein, als wäre sie weich wie Käse. Als Iril den Stein wieder entfernte, war ihr vorheriges Runenwerk ausgebleicht und kein einziger glühender Funken mehr zu erkennen.

„Wiederverwendbare Runenscheibe, mehrspeichig“, grinste Iril, „Geheimtrick meiner Runenmeisterin Burmrit. Sehr praktisch, um neuartige Kombinationen auszuprobieren, ohne eine ganze Metallscheibe zu verbrauchen. Lass uns aufbrechen.“

„Was hast du gerade genau getan?“

„Ich habe die Scheibe auf die unsichtbaren Ströme der Magie sensitiv werden lassen. Insbesondere auf Knochenmagie. Mit etwas Glück zeigte sie soeben in die Richtung der entwendeten letzten Knochenfragmente Taroks. Die Drachen verfügen schließlich über eine unglaubliche Nähe zur Magie, die keinem Menschen oder Zwerg nahekommt. Und die restlichen Überreste von Taroks Körper wurden inzwischen schon weit ins nördliche Meer hinaus geschwemmt, die lenken uns nicht ab.“

„Dann los! Lass uns Drachenknochen jagen!“\bigskip







Der Hüter der Zeit rollte sich von seinem Schlafkissen und räkelte seinen schrumpeligen kleinen Körper. Die andere Temm der Reisetruppe blinzelte ihm fröhliche Aufwach-Grüße zu, doch der Hüter sah sie nicht wirklich. Während er gedankenverloren sein Frühstücksbrot verzehrte und zur Zahnpflege auf einem Kauast herumknabberte, starrten seine Augen ins Leere. Sein Geist forstete durch Jahre von Eindrücken, Bildern und Gesprächsfetzen, die er noch nicht erlebt hatte.

Der Hüter der Zeit erinnerte sich daran, wie einige weiter vorne in diesem unterirdischen Stollen, zwischen ihrem Schlafplatz und Andor, eine riesige Bruthöhle lag. Der Hüter sah das verschwommene Antlitz von Krgur vor sich. Wie der Anführer seine Würmer zur Jagd auf die elenden Eindringlinge in sein Heiligtum anreizte. Üble Sache. Der Hüter befürchtete, dass die Reisegruppe an diesen Kreaturen vorbeimusste, um Andor zeitig zu erreichen. Zumindest fiel ihm nichts anderes ein.

Heimlich richtete der Hüter sich auf und entnahm einen Edelstein aus einem mächtigen Sack, der auf dem Rücken der schlafenden Steppenechse Sabri befestigt war. Dann schlich er sich den Stollen entlang in die nächste Höhle. Stockdunkel war sie, doch der Hüter musste nicht sehen, um die richtige Stelle am Boden zu finden. Er musste nur leise genug sein, damit die schlafenden Kreideskrale schlafend blieben.

Zielgenau langte der Hüter auf den Boden. Uralte Rillen überzogen ihn. An einer bestimmten Stelle kreuzten sich gleich drei Rillen in einer kleinen Kuhle. Sorgfältig platzierte der Hüter den grünen Edelstein dort darin. Er würde sich nicht mehr daran erinnern, den Edelstein hierhin gebracht zu haben. Doch erinnerte er sich daran, in seinen Tagebüchern davon gelesen zu haben. In denjenigen wenigen seiner vergilbten Schriftrollen, die er gar auf diese Reise mitgenommen hatte, um sich die baldigen Geschehnisse kurz nachher noch ins Gedächtnis zu rufen, und sie somit kurz vorher ins Gedächtnis gerufen zu haben.

Fröhlich umhertrippelnd kehrte der Hüter zurück zu den anderen Reisenden. Die meisten schliefen noch. Hexer Haamun murmelte leise im Schlaf, einen großen Sack voller magischer Kräuter und Salben als Kopfkissen nutzend. Steppennomade Barz lag mit dem Rücken an seiner gewaltigen Echse, die mit mehreren großen Säcken und einem wunderschönen, mannshohen Spiegel beladen war. Takuri-Hüterin Aćh schnarchte neben dem goldenen Feuervogel Turr, der seinen Kopf unter sein flammendes Gefieder gesteckt hatte.

Hier, in der Dunkelheit der Stollen unter dem Kuolema-Gebirge, war der Tagesrhythmus der Reisegruppe war völlig durcheinandergeraten. Der Hüter der Zeit freute sich schon darauf, sich endlich wieder unter freiem Himmel zu befinden. Er machte sich daran, sie zu wecken.











\newpage
\section{Der Trupp aus Tulgor}




Iril und Ijsdur folgten der von den Runen angezeigten Richtung, in der Hoffnung, die von den Drachenkultisten gestohlenen Drachenknochen aufzuspüren. Allzu weit konnten sie ja nicht gebracht worden sein. Der Prinz hatte die beiden Brücken ins östliche Rietland und in den Wachsamen Wald absperren lassen. Im Norden lag das Hadrische Meer, im Westen das hohe Kuolema-Gebirge, in allen anderen Richtungen die tückische Narne, und einen Riesenvogel hatten die Kultisten seit dem Vorfall mit Taroks letzten Knochen auch nicht mehr. Die Diebe saßen in der Falle.

Die Runen lenkten Iril und Ijsdur in den Westen, in Richtung des Kuolema-Gebirges. Kurz vor dem Südlichen Wald befragte Iril die Runen erneut. Diese hatten auf einmal ein völlig anderes Ziel im Sinn. Nun zeigten sie weiter in den Süden. Hatten die Drachenkultisten sich in so kurzer Zeit so stark bewegt? Auf welche Knochenmagie außer Taroks Fußknochen könnten die Runen anspringen?

Iril und Ijsdur schlenderten am Rande des südlichen Walds entlang und am Krallenfelsen vorbei tief in den Süden Andors.

Zu ihrer Linken lag der kleine See, aus dem die Narne entsprang. Flugforellen sprangen aus dem See hervor und flatterten kurz mit ihren funkelnden Flügeln, ehe sie wieder im kühlen Nass verschwanden. Er wirkte so friedlich, ganz im Gegensatz zu den scharfen Klippen im Osten, wo die Narne dem See entsprang.

Iril füllte einen Trinkschlauch auf. Ijsdur trank gierig aus dem See. Sein Schneekörper schien an Masse zuzunehmen.

Hinter dem See waren die tiefen Schluchten des Grauen Gebirges zu erkennen, aus deren Quellen wiederum der kleine See gespeist wurde. Doch nicht dorthin zeigte Irils Runenscheibe.

Sondern zu ihrer rechten. Ins Kuolema-Gebirge hinein.

An einen ganz bestimmten Ort.

Am Hang hinauf, nur einige hundert Meter von Iril und Ijsdur entfernt, lag ein kleiner Höhleneingang, kaum groß genug für einen Zwerg.

Solche Gänge gab es zu Dutzenden in den Bergen, sowohl hier im Fahlen Gebirge als auch im Grauen. Uralte, brüchige Stollen, die tief unters Fahle Gebirge führten. Die Wege wandelten sich stetig. Manch ein Eingang schloss sich unter Beben oder tektonischen Verschiebungen, manch ein anderer öffnete sich erst durch eben solche Prozesse. Iril hätte sich unter gewöhnlichen Umständen kaum dort hinein getraut. Zu oft hatten ihre Eltern ihr eingebläut, ja nie in einen unbekannten Berggang zu stapfen. Egal ob Gruftasseln, Hornbären, Arpachen, Sporne,oder auch nur eine verrückte alte Agren sich darin versteckten. Freunde fand man dort kaum.

Ein eiskalter Hauch streifte Irils Schultern, als Ijsdurs Hand – gefolgt von Dutzenden Schneeflocken – an ihr vorbeischwang. Er legte seinen Finger vor die Lippen und deutete auf eine Stelle leicht unterhalb des Höhleneingangs.

Im Schatten eines Baumes, aber immer noch gut erkennbar im Tageslicht, lag ein einsamer Gor. Offenbar waren Gors hin und wieder doch allein unterwegs. Er schlief auf einem behelfsmäßigen Bett aus zusammengescharrten Blättern, leicht zitternd. Auch er spürte die Winde des kommenden Winters.

„Meinst du, die Runen locken uns zu diesem Gor?“, flüsterte Ijsdur.

„Ein einzelner Gor, dessen Knochen als magische Quellen mit Tarok vergleichbar wären? Kaum. Etwas geht in dieser Höhle vor“, murmelte Iril.

Passend dazu flackerte grünes Licht im Höhlengang auf.

„Sind das die Kultisten?“

„Gut möglich. Grüne Flammen deutet auf magisches Feuer hin. Drachenknochen brennen hingegen schwarz-silbern, nicht grün.“

Erneut dachte Iril an alle Schauergeschichten, mit denen sie vom Erkunden solcher Gänge abgehalten worden war. Manchmal lauerten finstere Wesen darin auf ihre Beute. Skrale, so weiß wie die Kreide, mit denen die andorischen Lehrer ihren Schülern Rechnen und Schreiben beizubringen versuchten. Große Trolle, die ihre Höhlen gerne mit den Gerippen ihrer Opfer schmückten. Höhlenwichte, die sich größte Mühe gäben, eines jeden neugierigen Eindringlings die Nase abzubeißen. Es gab Gründe, warum Eltern ihren Kindern verbaten, diese Höhlen auch nur näher zu kommen. Iril ergriff ihren Runenhammer fester. Als erstes würde sie den Gor ausschalten.

Doch, ehe sie dem Gor ein rasches Ende bereiten konnte, trat hinter einer Linde unten am Berghang eine grau gewandte Gestalt hervor.

Es war eine kleine, ältere Frau, auf deren krummen Rücken ein geschnürtes Bündel und ein kleiner Kessel befestigt waren. Sie hielt einen giftgrünen Pilz zwischen langen Fingern, musterte ihn argwöhnisch und ließ ihn in den Kessel fallen. Dann machte sie sich leise murmelnd auf den Weg, den Berghang hinunter.

Als sie Iril und Ijsdur erblickte, winkte sie und hielt einen Finger vor ihre Lippen, während sie mit der anderen Hand auf den schlafenden Gor deutete.

Ijsdur blickte verwirrt zu Iril für Handlungsangaben. Iril zuckte mit den Schultern. Sie wusste nicht so recht, was sie von dieser Frau halten sollte, aber sie hatte merkwürdigerweise das deutliche Gefühl, dass sie ihr vertrauen konnte. Sie hatte sie schon einmal gesehen, während Taroks Angriff. Die Alte hatte Bewahrern Tränke verkauft. Eine Trankbrauerin?

Iril und Ijsdur warteten einige Minuten, bis die Frau den Hang heruntergekraxelt war und sie erreichte. Bei jedem ihrer Schritte klirrte und klimperte ihr Umhang sanft, als trüge sie ein allerlei Flaschen oder Phiolen mit sich. Eine silbern glänzende Schlange schlich ihr nach und wand sich am Bein der Fremden hoch, sobald sie vor Iril und Ijsdur zu stehen gekommen war.

Mit salbungsvoller Stimme flüsterte die Fremde: „Lasst den Gor sein, weckt ihn noch nicht. Ich vermute, dass ihr aus demselben Grund hier seid wie ich?“

Iril setzte zu einer Antwort an, da unterbrach sie die Buckelige schon wieder: „Tatsächlich, seid Ihr! Wie passend. Mehr als ich wisst ihr aber leider auch nicht. Ich bin Reka.“

„Die Kräuterhexe!“, rief Iril. Rekas Künste waren weit über die Grenzen von Andor hinaus bekannt. Ebenso wie ihre eigenwillige Art.

„Genau die bin ich. Wir drei werden etwas Historisches erleben, hier und jetzt. So wurde es mir vor einigen Jahrzehnten prophezeit. Ich bin ja mal gespannt.“

Ijsdur meldete sich zu Wort: „Verzeiht, aber wie könnt Ihr wissen, dass etwas passieren wird, aber nicht, was?“

Reka grinste: „Wahrlich, wie kann ich das wohl wissen? Ich weiß es weniger, als dass ich schlicht jemandes Aussage vertraue, der es wirklich weiß. Seine einzigartige Sicht hat sich noch nicht einmal getäuscht.“

„Das ist eher eine Ausrede als eine Antwort“, merkte Ijsdur an.

„Manch einer würde sagen, dass eine Antwort umso wertvoller ist, wenn sie deine Gedanken an der Hand durch den sich so lichtenden Nebel führt, statt dich direkt an ein Ziel zu teleportieren, welches du dann im Nebel nicht erkennen kannst.“

Ehe Ijsdur zu einer Entgegnung ansetzen konnte, richtete sie ihren Blick auf seine Brust und murmelte: „Faszinierend. Dein Geist verrät nicht leichtfertig seine Geheimnisse. Darf ich fragen, wie ...“

Da erklang ein klirrendes Geräusch. Irils Blick wurde wieder zum schlafenden Gor am Berghang gezogen. Über ihm wellte sich auf einmal die Luft, als könne das Sonnenlicht nicht mehr gerade scheinen.

„Obacht! Ich glaube, es geht los!“, rief Reka freudig.

Auch der schlafende Gor musste das unangenehm laute Klirren und Knirschen über ihm wahrgenommen haben. Erschreckt jaulend sprang er auf. Die drei Zuschauer weiter hangabwärts ignorierte er. Denn in diesem Moment verformte sich der schwirrende Schemen neben ihm und nahm konkrete Gestalt an.

Es war ein gewaltiger, mannshoher Spiegel mit einem wunderschönen Rahmen, der über und über mit glitzernden Steinen versehen war. Grüne, blaue und rote, symmetrisch geschliffen und quadratische Rahmen eingelassen.

Auch nachdem der mysteriöse Spiegel sich komplett manifestiert hatte, schwebte er einige Handbreiten über dem Boden, wie von unsichtbaren Fäden gehalten.

Und wie von unsichtbaren magischen Fäden geführt, wurde der strampelnde Gor in die Höhe gerissen und vor den Spiegel bugsiert.

Von weiter unten hatten die drei Zuschauer nur einen schrägen Blick auf die Spiegeloberfläche. Es wirkte so, als welle sich das Spiegelbild des verschreckten Gors und verforme sich zu einem Menschen. Nicht nur das, auch das Spiegelbild des Berghangs um den Gor verformte und verdunkelte sich. Im Spiegelbild hinter dem gespiegelten Menschen ging irgendein Gefecht ab. Blitze erhellten einen dunklen Stollen. War der Begriff „Spiegel“ überhaupt noch passend? Inzwischen erinnerte das mysteriöse Objekt, das neben dem Gor erschienen war, eher an ein magisches Portal.

Die Person im Spiegel hob prüfend ihre rechte Hand. Daraufhin hob sich die linke Hand des Gors, wie von unsichtbaren Fäden gezogen.

Der Spiegelmensch trat nach vorne und durchbrach die Spiegeloberfläche. Der strampelnde Gor ahmte – wohl unfreiwillig – diese Bewegung exakt nach und glitt in den Spiegel hinein. Ein heller Lichtblitz zuckte auf. Gliedmaßen durchfuhren einander, als wären sie nichts als Illusionen. Dann war der Gor im Spiegel verschwunden. An seiner Stelle stand ein dunkelhäutiger, gehörnter Mann.

Der Spiegel hörte auf zu schweben, fiel zu Boden und zerbarst in hunderte kleiner Scherben.

Der aus dem Spiegel getretene Mann machte einige wackelige Schritte und stürzte ebenfalls zu Boden. Blut tropfte aus einer Wunde in seinem Bauch. Eine giftgrün leuchtende Substanz dampfte auf seinem einen Bein. Ein gehörter Helm mit eingelassenen roten Edelsteinen löste sich von seinem Kopf und kullerte den Hang hinunter. Er war gar nicht gehörnt gewesen!

„Sollen wir ihm ...“, setzte Iril an.

„Ja, sollen wir!“, meinte Reka. Sie wühlte in ihrem Reisesack und zog eine vielblättrige Pflanze heraus. „Geh du, ich bin nicht mehr so rasch auf den Beinen. Gib ihm von diesem Heilkraut. Zwei Blatt unter seine Zunge sollten ihn stabilisieren.“

Iril raste los, den Hang hinauf, zum Verletzten hin. Im Nebenbei bemerkte sie überrascht, dass Ijsdur ihr nicht folgte. Stattdessen blickte der Eis-Dämon starr den Neuankömmling an. Er schien mit sich zu hadern. Faszinierend. Aber ein Geheimnis, das sie später zu knacken hatte. Jetzt musste sie sich um den Verletzten kümmern.

Im Hochlaufen beäugte Iril den gehörnten Helm, der vom Kopf des Verletzten gerollt war. Zeremonieller Natur? Ihr Hammer summte leise auf, als sie den Helm an sich nahm. Definitiv ein magisches Artefakt. Wie der magische Spiegel besetzt mit leuchtenden Edelsteinen und die Hörner schienen aus abgebrochenen Knochen zu bestehen? Faszinierend. In Iril keimte der Verdacht auf, dass ihre Runen sich auf der Suche nach der nächsten Knochemagiequelle leider nicht auf Taroks letzte Knochen fokussiert hatten.

Die beiden Heilkrautblätter waren rasch dem Verletzten übergeben. Iril blickte zurück zu Ijsdur und Reka am Berghang hinunter und zuckte mit den Schultern. Was sollten sie nun tun?

Bevor sie eine Antwort erhalten konnte, lenkte Flackern aus demselben Höhleneingang wie zuvor ihren Blick ab.

Da! Erneut flackerte grünliches Feuer im Höhleneingang des Berghangs auf. Dann explodierte der Stolleneingang in einem Schauer aus grünen Funken und Flammen.

Erde rutschte zur Seite und enthüllte einen größeren Gang, als sie erwartet hatte.

Doch nicht nur das: Die wegrutschende Erde enthüllte auch eine Gesellschaft von vielleicht einem knappen Dutzend verschiedenster Reisender, welche mit zusammengekniffenen Augen ins Sonnenlicht blinzelten. Besonders auffällig unter der Menge war ein Mensch in langem Mantel, um dessen erhobene Hände sich magische grüne Flammen wanden.

Ein Hexer.

„Das war der richtige Weg. Du hast uns gerettet, Haamun!“, rief einer der Reisenden und klopfte dem Hexer anerkennend auf den Rücken. Der Hexer zuckte zusammen und rieb sich ächzend die Stelle.

Die Truppe rannte – oder besser: stolperte – ins Freie. Manch einer blickte ängstlich in den dunklen Stollen zurück. Iril hielt ihrem Hammer bereit. Doch wirkte es nicht so, als würden die Neuankömmlinge eine große Gefahr darstellen. Bis auf vielleicht diesen Hexer, Haamun. Dessen Begleiter mochten vereinzelt Dolche und Schwerter bei sich tragen, doch hatten sie kaum mehr die Koordination, diese geschickt zu führen. Die meisten ließen sich achtlos zu Boden fallen und keuchten nach Luft, betasteten belastete Gliedmaßen und blutige Schnitte. Was war in dieser Höhle vorgefallen? Wie Drachenkultisten wirkte diese Gesellschaft kaum, zumindest konnte Iril keine Drachenrelikte, -figuren oder -knochen erkennen.

Haamun ließ sich nicht sofort zu Boden fallen, sondern setzte zuerst einen kleinen Wichtel von seinen Schultern zu Boden. Die Gestalt erinnerte Iril äußerst stark an Wrort, den Temm von der Rietburg.

„Du steinalter Gneis!“, fluchte Haamun die Temm an, „Warum bist du stehen geblieben?!“

„Ich wollte den Echsenführer nicht zurücklassen! Will ich immer noch nicht! Du bist einfach davongerannt“, krächzte die Temm zurück. Erst jetzt bemerkte Iril die Fremdheit dieser Sprache, die sie dank ihrer in letzter Zeit dauerhaft aktivierten Übersetzungsrune verstand.

Haamun schüttelte bloß seinen Kopf. Dann erstarrte er, als sein Blick auf die grau gewandte Reka fiel, welche langsam den Berghang hochkraxelte.

Ijsdur tauchte wie aus dem Nichts neben Iril auf.

„Die auf des Hexers Schulter ist eine Temm“, flüsterte er ihr hilfreich zu, „Und sie spricht Tulgorisch“.

„Deine tulgorische Reisegruppe, nehme ich an?“, fragte Iril. Ijsdur nickte.

„Dann befindet sich wohl auch dein ...“

Ein Stöhnen hinter ihnen ließ Iril und Ijsdur herumschnellen. Der Verletzte, welcher durch seinen magischen Spiegel hier gelandet war, war erwacht. Er ließ gleich eine ganze Salve kreativer Flüche ab, während er seine Rippen betastete. Manche davon konnte Iril gar mit ihrer Übersetzungsrune nicht verstehen.

Dann fokussierte der verletzte Fremde sich auf Iril und stieß zwischen zusammengebissenen Zähnen hervor: „Da sind noch welche unten in der Höhle. Ich wurde verletzt und musste ...“ Der Fremde betrachtete Ijsdur genauer. Und erbleichte. „Ijs?! Bruderherz? Bist du das?“, rief er aus.

„Ich bin Ijsdur. Es freut mich, dich zu sehen, Eforas“, korrigierte Ijsdur tonlos.

Ungeachtet seiner Verletzungen stürzte Eforas auf den Eis-Dämon zu. Zaghaft streckte er seine Hand aus, hielt jedoch inne, ehe er Ijsdurs schneeweißes Gesicht berühren konnte.

„Was ... was ist mit dir ...“, stammelte er.

„Was mit mir geschehen ist? Mir wurde eine Eiskristallkette verliehen“, sprach Ijsdur, „Ein neues Leben. Ungeahnte Kräfte. Und ein neuer Körper.“

„Was ist mit den anderen?“

„Keiner sonst wurde zu einem Eis-Dämon, soweit ich weiß. Sie sind allesamt nur umgekommen. Ihr habt doch ihre Leichen gefunden. Es war Ijs‘ Schuld. Und ihre eigene. Ich würde sagen, dass es mir leidtut, doch sind diese Schuldgefühle größtenteils mit Ijs gestorben.“

Eforas‘ Gesicht verzerrte sich. Er schien nach Worten zu japsen.

Dies bemerkend fuhr Ijsdur fort: „Ich sehe, dass dies nicht die optimale Antwort war. Ich wünschte mir, es gäbe die perfekten Worte, um deinen Schmerz zu lindern.“

„Ijs, lass diese Maske des Eises fallen. Zeige, dass du mich erkennst. Ich bin’s doch, dein Bruder!“

„Natürlich. Das ist keine neue Information für mich. Ich trage Jahre von Erinnerungen an dich.“

„Den Ijs, den ich kannte, hätte sich gefreut, mich zu sehen.“

„Ich bin nicht den Ijs, den du kanntest. Doch liegt mir immer noch an dir und deinem Wohlbefinden.“

„Warum klingt das so unglaubwürdig? Ijs, ich erkenne dich nicht wieder.“

„Das sagte Saro auch.“

„Du warst bei Papa?!“

„Kurz nachdem ihr ins Gebirge aufbracht. Er hat mich verstoßen. Er konnte meinen Anblick nicht ertragen. Es ist mir ein Rätsel, wie ich vor euch hierher gelangen konnte.“

„Oh ein ... Saro mal wieder ... oh ... Ijs, das tut mir so leid.“

„Muss es nicht. Mir liegt nichts mehr an ihm.“

„ ... “

„Ich sehe, dass dies erneut nicht die optimale Antwort war. Bitte, empfinde meinetwegen keine negativen Gefühle.“

„Das funktioniert so nicht!“

„Das weiß ich auch. Es tut mir leid.“\bigskip







Iril hatte sich auf den Weg zum Hexer Haamun gemacht, der am ehesten nach einer führenden Persönlichkeit wirkte.

Inzwischen hatte auch die Hexe Reka die Tulgori erreicht. Sie verteilte irgendwelche hellbläuliche schimmernde Tränke, welche sie im Innern ihres Mantels umhergetragen hatte.

Und auf einmal war Haamun nirgendwo mehr zu sehen gewesen.

Iril blickte sich um und versuchte, Haamun aufzuspüren. Sie hörte ein Rascheln in einem nahe liegenden Unterholz einer kleinen Baumgruppe. Ein leiser Aufschrei ertönte, gefolgt von einem Fluchen. Dann flackerte eine grüne Stichflamme im Gehölz auf. Versuchte der Hexer etwa, sich durchs Unterholz zum südlichen Walde zu schleichen? Nicht optimal. Iril hatte genug Gerüchte von launischen Waldgeistern gehört, als dass sie dies für ein gutes Verhalten halten würde. Doch der Hexer ließ sich nicht beirren und verschwand noch tiefer im Wald. Warum floh er wohl so rasch?

Dann sah Iril, wie Reka betont in die andere Richtung sah, während einige andere Tulgori Haamun interessiert nachguckten.

Hatten die beiden etwa gemeinsame Geschichte?

Das war nun nicht wirklich relevant.

Ohne sich an jemand bestimmtes zu richten, rief Iril: „Was geht da unten noch ab? Eforas meinte, da wären noch Leute von euch im Berg?“

Natürlich verstand sie keiner. Immerhin zeigte einer der Reisenden mit dem Finger auf Eforas, der ein bisschen weiter oben am Hang stand und von Ijsdur zurückwich. Seinen Namen hatten sie verstanden.

Da trat ein weiterer buckliger Temm aus der Mitte der Gruppe hervor und stakte ungelenk auf Iril zu.

„Ich bin der Hüter der Zeit“, stellte er sich krächzend vor, „Wir sind uns noch nie begegnet, und doch kenne ich dich bereits, o Iril, Runenmeisterin aus Silberhall.“

Iril wusste besser, als in einem dringlichen Moment weiter nachzufragen, auch wenn ihre Neugierde durchaus geweckt wurde.

Da stolperte auch schon Eforas ihnen entgegen. Ijsdur hielt sorgfältig Abstand.

„Du verstehst mich, oder?“, stellte Eforas klar, ehe er fortfuhr: „Zwei von uns stecken immer noch in den Höhlengängen! Wir wurden verfolgt von seltsamen leuchtenden Riesenwürmen. Der Nomade wollte seine faule Echse nicht im Stich lassen. Die anderen sind geflohen. Die Zurückgebliebenen brauchen Hilfe.“

„Lumiwürmer“, warf die kleine Temm ein, „Das waren Lumiwürmer. Und sie wurden von mir bislang unbekannten Vertretern der Creatura geritten. Solche Humanoiden habe ich noch nie gesehen. Riesig wie ausgewachsener Mensch. Ausgemergelt, doch mit kräftiger Muskulatur. Kreidebleich. Mit mächtigen Hauern am Unterkiefer, die sie wohl kaum nur zum Brezelstapeln nutzen.“

„Ich danke euch dafür, dass ihr unsere Zurückgebliebenen verteidigen geht“, rief der Hüter der Zeit entschieden, „Nehmt den Knochenhelm mit. Und die Takuri-Spiegelscherben. Ack ist eine Takuri-Hüterin, sie wird wissen, was damit zu tun ist.“

Iril sah ihn verwirrt. Dann nahm sie den Helm und die zwei Scherben an, die Eforas ihr in die Hand drückte. Wohl Überreste des zerbrochenen magischen Spiegels, durch den sich Eforas ins Freie teleportiert hatte.

„Und von euch will niemand zurückgehen und die Zurückgebliebenen verteidigen?“, fragte die kalte Stimme Ijsdurs, der immer noch gehörig Abstand von Eforas hielt.

Die Mitglieder des tulgorischen Trupps lagen mehr da, als dass sie standen. Wer nicht verletzt oder erschöpft auf die Seite gekippt war, kümmerte sich um die Verletzten und Erschöpften. Keiner wagte es, Ijsdur zu antworten, auch wenn viele ihm rasche, unsichere Blicke zuwarfen. Iril sah gar jemanden ein Schutzzeichen in den Händen zu formen.

Von ihnen war keine Hilfe zu erwarten.

Iril fasste sich ein Herz: „Was meinst du, Ijsdur? Kommen wir mit ein paar seltsamen leuchtenden Wurmdingern klar?“

„Keine Ahnung“, sprach Ijsdur wahrheitsgemäß, „Aber sofern wir schneller sind als sie, spricht nichts dagegen, in ihre Nähe zu gehen und herauszufinden, ob wir helfen können.“

„Dann sind wir ja einer Meinung“, grinste Iril.

Sie setzte den gehörnten Knochenhelm mit den magischen Edelsteinen auf. Eigentlich sagten Helme ihr nicht wirklich zu, die beengten ihren Kopf zu sehr. Aber dieser magische hier interessierte sie durchaus. Außerdem hätte Ijsdur mit seinem breiten Geweih ihn ohnehin nicht tragen können.

Dann verschwanden die beiden im Innern des Höhlengangs.\bigskip







Reka zeigte den Tulgori einen hellbläulich schimmernden Trank. Nachdem der Hüter der Zeit diesen kurz adressiert hatte, nahm Eforas bereitwillig einen Schluck vom Gebräu. Es schmeckte stark süßlich, doch mit einer leicht bitteren Komponente darunter.

„Na, verstehst du mich nun?“, fragte Reka.

Eforas zuckte zurück. Er nahm die fremden Worte seines Gegenübers immer noch als fremd war, doch verstand er nun auch ihre Bedeutung. Es schmerzte, und Bilder und Eindrücke schossen in seinen Kopf, die er nicht näher einordnen konnte. Erst jetzt fiel Eforas auf, wie sich eine silbrig geschuppte Schlange um den Arm der Hexe vor ihm gewunden hatte und ihn anzischte. Er wankte zurück.

„Keine Angst vor Maro, die treue Seele beißt nicht“, grinste Reka.

Seine Stimme wollte ihm nicht gehorchen. Seine Zunge versuchte, seltsame Bewegungen auszuführen, die sie noch nie zuvor vollzogen hatte.

„Das ist Hexerei!“, stotterte er.

Freundlich erzählte Reka: „Kein große Hexerei. Nur eine dieser kleinen Mixturen, die alten Frauen in den Sinn kommen, wenn sie zu viel Zeit allein in der Wildnis verbringen. Wobei, wartet, nein! Diese Rezeptur stammt nicht von mir, ich habe nur die Essenz von Zucker für den Geschmack hinzugefügt. Das Rezept hat mir vor Jahrzehnten einmal ein gehörnter Geselle anvertraut. Leider weiß ich nicht, wohin er sich verzogen hat. Ich nicht einmal sagen, ob er noch lebt. Er mag vorher umgekommen sein, oder er hält sich an einem Ort auf, von dem ich keine Kunde habe. Sowohl sein älteres als auch sein jüngeres ... ah, egal. Jedenfalls erlaubt dieser Sprachtrank es seinem Trinkenden, Strukturen in gesprochenen Worten zu erkennen und sie besser sortieren zu können, ja, die Sprache von fremden Menschen zu erkennen und sich mit ihnen zu verständigen ...“

Ihre Stimme verlor sich. Vor sich sahen die Reisenden einander unsicher an.

„Brecht auf in den Nordosten, am Waldrand entlang“, schlug Reka Eforas vor, „Ihr werdet hoffentlich einen Schlafplatz finden. Morgen solltet ihr dann an einer Taverne vorbeikommen, wo ihr euch mit Speis und Trank stärken könnt. Und dann gelangt ihr zum Freien Markt. Der Freie Markt liegt in Trümmern, die Händler sind vor einem – inzwischen toten – Drachen geflohen. Dort, wo einst die Marktstände standen, könntet ihr ein Lager aufschlagen, eure Waren verkaufen und den Andori Hilfe beim Wiederaufbau anbieten. Vielleicht wird euch gar unser Prinz Thorald großzügig dafür entlohnen. Und sei es nur, weil er nicht in schlechtem Licht erscheinen will, wenn Fremde den Andori bei ihrer Arbeit helfen.“

„Diese Reise wird noch länger dauern“ rief ein Tulgori, „Wir tragen viel Ausrüstung mit uns. Nicht zu vergessen die schweren Mera-Steine. Kein Lasttier mehr, kein Fahrzeug, dafür sind voller Blasen an den Füßen und voller Erschöpfung.“ Er blinzelte Reka an, als könne die Hexe ihm die Last mit einem Wisch ihres Fingers wegzaubern.

Reka sah ihn mit einem mitleidigen Lächeln an.

„Wir finden schon eine Lösung, Seram“, wandte sich Eforas an die Tulgori. „Wenn wir die Steine nicht mehr selbst zu schleppen vermögen, dann wird dies der Fluss für uns tun. Ein Floß kriegen wir ja hoffentlich noch gebaut.“ Er wandte sich an Reka: „Besten Dank, werte Hexenmeisterin, für all Eure Hilfe und Euren Trunk. Was sind wir Euch schuldig?“

„Nichts“, winkte Reka ab, „Der erste Trank geht immer aufs Haus. Ich hätte allerdings noch allseits beliebte Mücken- und Zeckenschutztränke zu verkaufen. Diese Plagegeister gibt es im Sommer hier in Hülle und Fülle.“

Eforas versuchte, zu antworten, lallte jedoch etwas Unverständliches.

„Zeit für den nächsten Sprachtrank-Schluck“, lächelte Reka, „Wenn ihr nur einen Übersetzer hier hättet, würde es durchaus einfacher.“

„Ich bin doch ein Übersetzer“, rief der Hüter der Zeit fröhlich ein.

„Hallo, alter Freund“, begrüßte Reka den Temm, „Aber du wirst dich gerade nicht als Übersetzer um sie kümmern können. Weil ich dich jetzt klauen werde. Wir haben viel zu besprechen. Die Tipps, die du mir damals in den Trollkriegen verrietst, sind beinahe alle durch.“

„Weiß ich doch schon.“ Der Hüter der Zeit grinste und kletterte Reka bereitwillig auf den Rücken. Gemeinsam schlenderten sie davon und waren schon bald in den ersten Nebelschwaden an der Narne verschwunden.

Rekas heisere Stimme verklang: „Nun denn, erinnerst du dich noch daran, wie du mir von diesem Schattenskral berichtetest? Wie es sich herausstellt ...“\bigskip







Iril und Ijsdur stapften durch den dunklen Höhlengang.

Der Runenhammer in Irils Hand warf einen schwachen grünlichen Schein auf erdige Wände. Hin und wieder standen kleine Holzkonstruktionen in Nischen, aus denen eine gelblich schimmernde Flüssigkeit ein Quäntchen Licht spendete.

Im Lichte dieser Laternen waren in von Boden bis Decke noch immer die Spuren von Schaufeln und Spitzhacken zu erkennen. In regelmäßigen Abständen waren Runenfolgen in den Boden eingelassen.

Iril fand gut Platz, doch Ijsdur schlug immer wieder mit seinem Geweih gegen die erdige Decke. Diese Gänge waren unregelmäßig hoch und nicht auf großwüchsige Menschen angepasst.

„Haben die Tulgori diese Gänge gegraben?“, fragte Iril.

„Nein“, meinte Ijsdur, „Das waren die Temm. Lange, lange Zeit ist es her. Ihre verwinkelten Gänge durchziehen einen Großteil des Fahlen Gebirges.“

„Wie die Schildzwerge das Graue Gebirge“, überlegte Iril.

„Nein“, widersprach Ijsdur, „Ich habe in der letzten Woche ein wenig von den Schildzwergen mitbekommen. Sie sehen die Erde als ihre Heimat an und erringen ihr sorgsam ihre Schätze, während sie die Natur achten. Die Temm hingegen stammen aus Tulgor. Diese Gänge sind für sie nichts weiter als Passagen von A nach B. Und A nach C. Und Y nach Z. Es gibt viel mehr Verzweigungen hier, als eigentlich nötig wäre. Ich hatte auf der Hinreise das Glück, als Dämon des ewigen Eises meine eigenen Abkürzungen schaffen zu können. Aber Menschen aufspüren kann ich nicht. Hoffentlich finden wir die zurückgebliebenen Tulgori, ehe eine Abzwei... oh.“

Iril und Ijsdur hatten eine Abzweigung erreicht. Einen kreisrunden Hohlraum, von dem in regelmäßigen Abständen sechs Wege in andere Richtungen abführten. Durch den einen waren Iril und Ijsdur soeben getreten – Iril ritzte rasch ein Symbol in die Tür, auf dass sie sich daran erinnern konnten, woher sie kamen – doch die anderen fünf Gänge schienen allesamt gleich unscheinbar zu sein. In der Mitte des Raums stand eine Art Podest oder Altar, auf dem Iril im schummrigen Licht ihres Runenhammers Kohlereste erkannte.

„Nun, Iril, in welchen Gang wollen wir ziehen?“

„Es gibt nur eine natürliche Wahl: Die gegenüberliegende.“

„Lass dich nicht zu sicher werden. Temm-Gänge sind bekannt dafür, plötzlich in unerwartete Richtungen abzubiegen oder übereinander zu laufen. Vielleicht könntest du wieder deine Runenscheibe zücken und die nächstmögliche Magiequelle anpeilen?“

„Wenn wir wüssten, über was für eine Art von Magie die Zurückgebliebenen verfügen“, murmelte Iril, „Und selbst wenn, wenn diese Gänge so unerwartet krumm verlaufen können, hilft uns die Richtung allein auch nicht. Und es kostet immer Zeit, die Runenscheibe richtig zu gestalten. Ich wäre eher dafür, einem zufälligen Gang zu folgen, und danach ...“

Iril kam nicht dazu, fertigzusprechen. Denn in diesem Augenblick erschütterte ein Poltern die Höhlengänge, und ein roter Schein leuchtete aus dem zweiten Höhleneingang von links.

Iril packte ihren Runenhammer fester. Ijsdur flatterte mit seinen Fingern. Wie aus dem Nichts sammelte sich Wasser davor und formte sich zunächst länglich, dann zu einem Schwert, welches prompt vereiste.

„Hübscher Trick“, merkte Iril an, während sie losrannten, der Quelle des roten Feuers entgegen.

Mit seinen langen Beinen hängte Ijsdur Iril nur beinahe ab.\bigskip







Der Gang mündete schon nach wenigen hundert Schritten in eine gewaltige Höhle. Auch diese war kreisrund, mit mehreren möglichen Ausgängen in regelmäßigen Abständen auf allen Seiten. Allerdings war sie viel größer. Die Grundfläche hätte der Akademie von Werftheim Konkurrenz gemacht, und in der Höhe konnte sie bestimmt den Alten Wehrturm in seiner höchsten Phase übertrumpfen. Wie war es möglich, dass eine derart riesige Luftblase in einem sich wandelnden Gebirge stabil bleiben konnte?

Iril konnte nicht ihren Finger darauflegen, aber irgendetwas war eigenartig an diesem Raum. Gerillte Linien waren in die Wände, Decke und Boden eingelassen, welche irgendwie gleichzeitig gerade und schief wirkten. Die Linien liefen auf der anderen Seite des Raumes nicht so zusammen, wie sie sollten, sondern irgendwie ... enger? Gekrümmter? Irgendetwas war falsch hier, und wenn es sich nur um eine optische Täuschung handelte. Das sollte sie jedoch nicht aufhalten.

Wie in der letzten Höhle stand auch in dieser ein großer Altar in der Mitte. Anders als in der letzten Höhle brannte etwas darauf lichterloh und erhellte den gewaltigen Leerraum. Doch war dies keine Kohle, sondern ... ein Vogel?!

Tatsächlich, da brannte ein Vogel mit goldenem und feuerrotem Gefieder auf dem Altar. Die Flammen schienen ihm nichts auszuhaben, im Gegenteil, er krächzte fröhlich vor sich hin.

Ijsdur stockte und blickte das Tier ängstlich an. Als er weiterrannte, hielt er sich so, dass Iril sich stets zwischen ihm und dem Vogel befand, auch wenn er dadurch an Geschwindigkeit einbüßte.

Da purzelten auch schon die beiden vermissten Tulgori durch eine der vielen Abzweigungen in die riesige Höhle. Ein Mann in einem langen braunen Mantel, der einen Bogen an seinen Rücken geschnallt und ein silbern glitzerndes Seil um seine eine Hand geschlungen hatte, sowie eine Frau mit einem eleganten roten Umhang und einem golden glänzenden Schwert, deren lange Dreadlocks von flackernden Lichtscheinen überzogen wurde, als stünden sie in Flammen.

Zu guter Letzt erschien der mutmaßliche Grund für das Zurückfallen dieser letzten beiden Mitglieder von Haamuns Reisetruppe: Eine gewaltige graue Echse trampelte in die Höhle, verbunden mit dem Mann durch ebenjenes silbern glitzernde Seil an seinem Handgelenk. Der Mann zog und zerrte gelegentlich an dem Seil, doch die Echse – sein Lasttier? – wirkte alles andere als vor Tatendrang sprühend. Unbeirrbar träge setzte sie einen Fuß vor den anderen, während der Mann sie zur Eile antrieb und fluchte.

Die beiden Neuankömmlinge bestaunten ebenso wie Iril und Ijsdur die riesige Höhle und das magische Feuer in deren Mitte. Iril fiel auf, dass Ijsdur aber auch insbesondere die beiden Fliehenden beäugte. Irrte sie sich, oder zeigte sein üblicherweise regloses Gesicht tatsächlich Überraschung?

Der Feuervogel drehte eine Runde um die Höhle, stürzte von der Höhlendecke hinab und landete auf einem Arm der Frau. Erst jetzt, wo er nicht mehr in einer hohen Feuerschale lag, konnte Iril ihn richtig erkennen.

Er war so groß, dass sein Haupt ihren Kopf überragte, und wirkte fast wie ein Adler. Doch seine Augen glänzten wie gelbe Edelsteine und sein Gefieder schillerte golden und orange, so als lodere ein Feuer darin. Das Tier saß stolz und selbstbewusst auf dem Arm seiner Hüterin, als habe es sich bewusst dafür entschieden, sich genau dort niederzulassen. Iril starrte mit offenem Mund auf den Vogel und konnte ihren Blick kaum von ihm abwenden. Es war das schönste Wesen, das sie je gesehen hatte.

Inzwischen waren Iril und Ijsdur nahe genug an die beiden Neuankömmlinge getreten, um Wortfetzen ihres Gesprächs mitzukriegen.

„Bei den verfaulten Spornwalen der Tiefe, jetzt gib dir doch wenigsten ein bisschen Mühe, Sabri!“, fluchte der Mann. Interessanterweise nicht in der Tulgorischen Sprache. Aber auch nicht in der andorischen.

Seine Begleiterin drängte ebenfalls zur Eile, sie allerdings schon in der Sprache der Tulgori: „Jetzt komm schon, Barz! Ich bin mir nicht sicher, wie kräftig Turr noch ist. Wir können ihn nicht auf ewig ausnutzen, um die Würmer zurückzudrängen.“

„Ich bin nicht das Problem! Es ist Sabri, die ...“

„Meinst du nicht ...“

„Ich lasse sie nicht zurück, Ack! Nicht sie!“

Beim Namen ‚Ack‘ horchte Iril auf. Von ihr hatte der Hüter der Zeit erzählt. Die Takuri-Hüterin, die mit den Spiegelscherben etwas anfangen können sollte! Noch hatten die beiden Fremde ihre Anwesenheit nicht wahrgenommen. Wie nahe sollten sie sein, um sie anzusprechen? Wann passte es? Iril fasst sich ein Herz, und rief noch im Näherrennen laut:

„He, ihr da! Ack und Barz! Ich habe hier ... ach verflixt, die Sprachbarriere! Ijsdur, kannst du ihnen sagen, dass ...“

Weiter kam sie nicht. Ack und Barz wirbelten überrascht zu ihnen herum, erstarrten bei Ijsdurs Anblick, machten ihre Waffen kampfbereit ... und in diesem Augenblick brachen ihre leuchtenden Verfolger durch gleich mehrere Seitengänge in die Höhle hinein.

Drei riesige Würmer mit fahler faltiger Haut, länger als fünf ausgewachsene Menschen und im höher als Iril, glitten überraschend geschmeidig über den Boden, eine Spur aus grünlich leuchtendem Schleim hinter sich herziehend. Giftig glitzernder Dampf dampfte von den Schleimspuren auf.

Lumiwürmer! Das war NICHT GUT!

Die Lumiwürmer besaßen keine Augen oder ähnlich sichtbare Sinnesorgane. Doch die Spitze ihrer grausigen Körper öffneten sich synchron und entblößten mächtige Schlunde, gefüllt mit mehreren unordentlichen Reihen nadelspitzer Zähne. Es wirkte so, als hätte jemand zufällig Nadeln im Innern dieser Rachen verteilt. Und als könnten sie eine ausgewachsene Zwergin mit einem einzigen Happs verschlingen.

Kaum waren die Lumiwürmer aus den Seitengängen – in welche sie teils kaum gepasst hatten – in die riesige Höhle eingebrochen, erhoben sich von ihren Rücken humanoide Kreaturen, die sich dort hineingepresst hatten. Kreidebleich waren die Reiter der Lumiwürmer, spindeldürr und mit langen Hauern, aber auch rudimentären metallenen Klingen in den knochigen Händen. Iril hatte bereits das Vergnügen gehabt. Kreideskrale! Wilde Kreaturen, ungeschützter als handelsübliche Skrale und umso wilder. Mit Tageslicht kamen sie nicht gut klar, doch solch unterirdische Gewölbe waren ganz ihr Element.

Drei Lumiwürmer und vier Kreideskrale. Und dies mochte nur die Vorhut der Verfolger sein. Verwirrender war, dass sich auch ein einzelner Gor zu den Angreifern gesellt hatte – bis Iril sich daran erinnerte, dass dies wohl derjenige Gor sein musste, mit dem Eforas mit seinem magischen Spiegel seinen Platz vertauscht hatte. Nun brüllte auch er unverständliche Laute des Zorns und des Blutrauschs.

Noch hielten sich Iril und Ijsdur, Ack und Barz sowie diese Echse – Sabri? – und der brennende Feuervogel – Turr? – rund um den Altar in der Mitte der Höhle auf. Doch so geschwind, wie die Lumiwürmer und ihre Reiter sich fortbewegten, hatten sie kaum mehr als einige Augenblicke, um sich auf die kommende Konfrontation vorzubereiten.

„Ijsdur, frag sie, ob sie die letzten Reisenden sind, oder ob da noch mehr Hilfebedürftige zu finden sind“, rief Iril. Elende Sprachbarriere.

Doch bevor Ijsdur etwas antworten konnte, meldete sich Barz zu Wort und sprach: „Ja, wir sind die letzten! Kann ich dann davon ausgehen, dass die anderen in Sicherheit sind, wir uns ganz nah an Andor befinden und du eine tapfere Andori bist, die uns mithilfe eines Tricks von diesen Verfolgern rettet?“

„Ja. Ja. Jein.“, sagte Iril rasch, „Wir sind hier, um euch nach unseren Möglichkeiten zu unterstützen.“

„Der Eis-Dämon auch? Wie konnte er überhaupt ...“

„Was haben alle immer nur gegen Eis-Dämonen? Ja, Ijsdur ist auf unserer Seite.“

„Das können wir ja später klären.“

„Genau“, brummte Ijsdur, „Wir müssen nur wenige Minuten in diese Richtung laufen, dann sind wir alle schon wieder im Freien.“

„Den Göttern sei Dank, ein Ende in Sicht!“, rief Barz mit Blick auf die Höhlendecke, „Dann wird es an der Zeit, meine letzten Pfeile zu verschießen und die mächtigen Pulver zu verbrauchen.“

Ehe Iril aus seinen Worten schlau werden konnte, gab Barz es auf, an den Zügeln seiner Riesenechse zu zerren. Stattdessen zückte er einen Bogen, griff in seinen Köcher und lief um Sabri herum, um ein freies Schussfeld auf die heranwälzenden Riesenwürmer zu haben.

Prompt drehte sich Sabri um und setzte an, Barz wieder tiefer in die Höhle zu folgen.

„Bei allen Kreaturen der Tiefe!“, fluchte Barz, und rannte zurück auf die andere Seite der Echse. „Nie folgt das sture Vieh meinem Willen!“

Die Echse grunzte tief auf. Barz‘ Stimme wurde weich. „Ich lasse dich doch nicht im Stich, liebste Sabri, und wenn du mir noch ein Jahrzehnt lang hinterherwatschelst. Aber das ändert nichts daran, dass ich es wirklich begrüßen würde, wenn du jetzt einen Zahn zulegen könntest!“

„Wenn du hinten bleiben musst, um die Echse weiter zu locken, dann sind es wir anderen drei, die sie beschützen werden“, sprach Ijsdur ernsthaft und richtete sein Eisschwert auf die anrollenden drei Lumiwürmer. Iril schwang ihren Hammer in der Luft herum. Mit einem tiefen Brummen erwachten grünliche Schwaden reiner Magie um ihn herum. Ack hielt ein goldenes Schwert bereit. Ihre Füße nahmen eine bestimmte Position ein, mit der Selbstverständlichkeit, mit der ein geübter Tänzer seine Position einnahm.

Dann rannten die drei los, den anstürmenden Lumiwürmern entgegen.

Iril bedachte, dass Ijsdur sein Eisschwert in der linken Hand führte. War er ein Linkshänder? Auf jeden Fall war damit klar, dass er den linken Wurm übernehmen sollte. Iril rannte auf den rechten Wurm zu. Leider hatten sie sich nicht abgesprochen, und so tat Ijsdur es ihr gleich, und wie der Zufall es wollte, Ack ebenso. Alle drei Helden stürzten auf denselben Wurm zu.

Ein gutturales Grollen drang aus dem leuchtenden Schlund des Monsters. Es wurde langsamer und richtete seinen Vorderteil auf, bereit, mit gewaltigen Happsen die drei törichten Winzlinge zu verschlingen, die es wagten, sich ihm entgegenzustellen.

Die restlichen beiden Lumiwürmer schlugen einen leichten Bogen um die drei desorganisierten Kämpfer und schlidderten ungestört weiter auf die für sie so viel schmackhafter scheinende Steppenechse zu.

Überraschend behände grub Ijsdur seine Fersen in den harten Höhlenboden, drehte sich um die eigene Achse und sprang den größeren dieser beiden Lumiwürmer an, auf dessen Rücken gar zwei Kreideskrale saßen. Iril vernahm das Klirren von Klingen und das Fauchen eines Kreideskrals, doch konnte sie sich nicht groß darauf konzentrieren, denn in diesem Augenblick ließ der bedrohlich vor ihr und Ack aufragende Lumiwurm sein unförmiges Maul auf sie niederstürzen.

Iril sprang im letzten Augenblick zur Seite. Die Erde erzitterte, als nadelspitze Zähne nur Fingerbreiten von ihrem Bein entfernt in den kalten Felsen bissen und brachen. Ein Heulen entfloh dem Lumiwurm. Iril flehte den Himmel an, Ack möge ihm ebenfalls ausgewichen sein.

Die kurzen, aber soliden Eisenstäbchen, die Iril in ihrer Reisetasche mit sich führte, waren eigentlich fürs Einritzen magischer Runenfolgen in Stein und Metall gedacht. Doch in solchen Situationen stellten sie sich auch als anderweitig nützlich heraus. Iril packte gleich mehrere der Stäbchen in ihre Faust und trieb sie dem Lumiwurm in die Seite. Eitrige, leuchtende Flüssigkeit triefte hervor und tropfte heiß Irils Arm herunter. Die ätzende Flüssigkeit brannte, doch Iril ließ nicht los. Als der Lumiwurm erneut seine Vorderhälfte in die Höhe hievte, wurde Iril mit ihm hochgeschleudert. Ächzend kletterte sie sich und kniete, nein, thronte majestätisch auf dem Kopf des Lumiwurms.

Von weiter hinten am Wurm blickte ihr der verärgerte Reiter des Lumiwurms entgegen.

Erneut stieß der Lumiwurm blind seinen Kopf auf den Erdboden. Offenbar war Ack ihm das erste Mal entgangen. Iril wurde beim Aufprall heftig durchschüttelt, schaffte es jedoch, auf dem gewaltigen Wurmkopf zu bleiben. Wankend richtete sie sich auf. Das war ja schlimmer, als bei Sturm auf hoher See zu sein!

Iril fixierte den einzelnen Kreideskral, der auf dem Rücken des Lumiwurms saß. Er trug keinen Sattel oder ähnliche Reithilfen, führte jedoch zwei Seile, welche links und rechts am Kopf des Lumiwurms befestigt worden waren. Mit diesen steuerte er das Biest wohl. Aktuell hielt er beide Zügel in derselben Hand und löste mit der anderen eine schartige Klinge von seinem Gürtel.

„Ergib dich! Lass mit deinen Würmern von der Jagd ab und du wirst uns nie wieder sehen müssen!“

Der Kreideskral verstand sie und lachte nur. „Ich werde dich auch nie wieder sehen müssen, wenn du Futter wirst. Die heiligen Würmer dürsten nach Blut.“

Irils Runenhammer dürstete danach, die in seinem Innern gespeicherte Magie loszulassen. Iril stolperte ein, zwei, drei wackelige Schritte vom Kopf des Lumiwurms seinen Hals herunter – falls der Begriff Kopf für ein so unförmiges Wesen überhaupt Sinn ergab – und holte mit dem Hammer aus.

„Angriff, Turr! Erledige den Reiter!“, erklang von irgendwoher Acks helle Stimme. Der glühende Feuervogel erschien wie aus dem Nichts, flatterte vor dem Kreideskral herum und hackte mit brennenden Klauen nach dem Kreideskral. Iril bremste den Schwung ihres Hammers aus und verfehlte den Vogel nur knapp. Glühende Funken rieselten auf sie herunter.

„Den hätte ich gehabt!“, rief Iril Ack empört entgegen, bevor ihr wieder einfiel, dass diese sie kaum verstehen konnte. Der Kreideskral hackte mit seinem verrosteten Schwert nach Turr und traf ihn am Flügel. Turr klatschte kreischend auf den Rücken des Lumiwurms und rollte von dort ungelenk auf den Boden. Ehe der Kreideskral sich neu orientieren konnte, schlug Iril die Kreatur mit dem Runenhammer grob vom Rücken des Lumiwurms. Ein Krachen und Knacken verriet ihr, dass er mindestens mit einigen gebrochenen Knochen zu kämpfen hatte. Damit war er wohl außer Gefecht.

Der Knochenhelm auf ihrem Kopf erwachte zum Leben. Mit einem warmen Summen brummte er auf ihrem Schädel. Eine angenehme Ruhe überkam Iril. Sie fühlte sich erfrischt und war bereit, den nächsten Gegnern entgegenzustehen! Welch faszinierender Kopfschmuck. Sie nahm sich vor, die Tulgori bei der nächsten Gelegenheit nach dem Geheimnis seiner Herstellung zu fragen.

Wackelig richtete sie sich auf dem Rücken des Lumiwurms auf und sondierte die Lage. Ack hatte es offenbar bislang tapfer geschafft, dem stacheligem Wurmmund zu entkommen, denn nun befand sie sich in einem Gefecht mit dem einzelnen Gor.

Ack schwang ihre goldene Klinge kunstvoll herum, stocherte aus der Entfernung nach den ungeschützten Hautstellen der Kreatur und fügte elegant kleine, doch sicherlich schmerzvolle Schnitte zu. Der Gor fauchte frustriert auf und schnappte immer wieder nach ihr, bekam die flinke Tulgori allerdings nicht zu fassen.

Nicht weit davon entfernt hatte Ijsdur es wie Iril auch auf den Rücken eines Lumiwurms geschafft. Gleich mit zwei Kreideskralen legte er sich an, welche er mit seinem kurzen Schwert von sich weghielt. Seine Bewegungen wirkten ungeübt und etwas ungelenk – der schwankende Wurm, auf dem er sich befand, half wohl kaum in der Stabilität – doch schaffte er es soeben, seine Klinge im ungeschützten Nacken der einen Kreatur zu versenken. Noch während der Kreideskral umkippte, heulte der andere auf und verbiss sich mit seinen mächtigen Hauern in Ijsdurs Schulter. Die beiden stürzten vom Lumiwurm und verschwanden aus Irils Blickfeld.

Doch was war mit dem dritten Lumiwurm? Was mit Barz und seiner sturen Echse?

Iril erkannte den grauen Körper Sabris am anderen Ende der Höhle, schon beinahe bei einem der Ausgänge. Das war gut. Doch der dritte Lumiwurm wurde von seinem Reiter stetig näher auf die Echse zugetrieben. Und Barz hatte immer noch nicht seinen Bogen gezückt. Das war weniger gut.

Was tat der Dussel denn nun?! Statt eine Waffe zu ziehen, löste Barz ein türkises Pulver von seinem Gürtel und streute sich eine Prise davon auf die ausgestreckte Zunge. Er schmatzte ein, zweimal und schrie kurz auf, als seine Augen blendend hell aufleuchteten. Dann trat er gewandt einen einzelnen Schritt zur Seite. Der Lumiwurm, der auf ihn zugehalten hatte, verfehlte ihn und Sabri um Haaresbreite, konnte nicht mehr ausweichen und knallte in die Höhlenwand hinein. Der Kreideskral auf seinem Rücken holte mit einem gezackten Stück Metall nach Barz aus, doch dieser drehte sich nicht einmal zu seinem Gegner um, sondern wich mit dem Rücken zu ihm mühelos aus, als hätte er den Schlag schon längst kommen sehen. Als Barz sich endlich umdrehte, sprang er gleich hoch und stach er seinerseits mit einem Pfeil aus seinem Köcher in die ungeschützte Brust des Kreideskrals. Die Kreatur sackte zusammen und hing locker vom Rücken des Lumiwurms, während dieser sich heulend und ziellos die Wand entlangwälzte und selbige mit leuchtender giftgrüner Spucke verätzte.

Barz blickte dem zweiten Lumiwurm entgegen, welcher neuerdings führungslos war und sich nun ebenfalls auf Sabri zuschlich. Leuchtendes, dampfendes Blut tropfte aus tiefen Wunden in seiner Seite, und er bewegte sich langsamer, vorsichtiger, doch nicht weniger unaufhaltsam.

Barz löste ein grünes Säcklein vom Rücken der Steppenechse und setzte sich daran, den anstürmenden Lumiwurm mit seinem Inhalt zu beschenken.

Da tauchte Ijsdur wie auf dem Nichts auf, seine Schulter von schwarzem Kreaturenblut getränkt.

„Das ist doch dein Bannpulver, oder?“, fragte er Barz, „Setze es noch nicht jetzt sein. Es kommt noch eine weitere Verzweigung. Banne einen Wurm lieber dort, dann versperrt er den Verfolgern den Weg nach draußen.“

Barz blickte ihn überrascht an, nickte dann jedoch.

Ein lautes Rumms riss Iril aus den Beobachtungen. Der Lumiwurm, den sie immer noch mehr oder minder erfolgreich ritt, hatte erneut nach Ack gebissen und seinen geifernden Kopf in den Boden gerammt. Ack stürzte zu Boden und der gemeine Gor nutzte die Gelegenheit, um mit seinen Hornklauen ihr Schwert wegzuschleudern. Waffenlos lag Ack da. Sie wäre noch nicht einmal schnell genug, wieder aufzustehen, ehe der Gor ihr nicht ein unschönes Ende bereitete.

Iril ließ sich in dieselbe Stelle plumpsen, wo zuvor der Kreideskral auf dem Lumiwurm gesessen hatte. Verzweifelt zog und zerrte sie an den Zügeln des gewaltigen Wurms. Dessen Kopf wirbelte umher, fegte knapp über Ack hinweg, schleuderte den Gor beiseite und verschluckte die Kreatur mit Haut und Schuppen. Iril zerrte erneut an den Zügeln, doch der Lumiwurm brummte nur wohlig und drehte sich mit einem gefüllten Magen zur Seite.

Gerade noch rechtzeitig sprang Iril von seinem Rücken, ehe sie zu Mus gequetscht worden wäre.

Iril stürzte vom Lumiwurm herunter und schlug mit dem Kopf hart auf dem Boden auf. Das wohlige Brummen des Knochenhelms verschwand, als der Helm sich von ihrem Kopf löste und lautstark über den Höhlenboden kullerte.

Schmerz durchzuckte sie und ihre Sicht verschwamm. Wie aus weiter Ferne nahm sie die Gestalt Acks wahr, die am verdauenden Lumiwurm vorbeihumpelte, nur ihre Silhouette vor dem leuchtend pulsierenden Leib des Wurms sichtbar

Die Tulgori hatte sich wieder aufgerichtet. Nun entdeckte sie den verkrüppelten Turr am Boden und eilte zu ihm.

„Bald wird es wieder gut sein. Bald. Halte nur noch kurz durch“, sprach Ack Turr gut zu. Tränen rannen ihr Gesicht herunter, tropften auf den Vogel und verdampften.

Sie goss eine ölige Flüssigkeit aus einem Lederschlauch über ihre Hände, verrieb sie rasch und hob dann sanft den wimmernden Feuervogel in die Höhe, der vergeblich mit seinem gebrochenen Flügel strampelte. Leuchtende Flammen flackerten über die feurigen Federn und leckten an Acks Armen, doch die Hitze schien ihr nichts auszumachen.

„O Turr, du Armer. Ich spüre, dass du am Ende deiner Kräfte bist. Du wirst sterben, bald, sodass du den Zyklus von vorne beginnen kannst. Doch noch ist es nicht so weit, wir brauchen dich noch beim Kampf gegen die Lumiwürmer, die immer näher rücken. Mit all meiner Willenskraft unterstütze ich dich, auf dass du noch ein bisschen länger durchhalten kannst.“

Ein feuriger Schein umspielte Acks Körper. Die Tulgori sank auf ein Knie und begann zu zittern, während das Leuchten über Turrs Federn abklang.

„Sei beruhigt. Nicht mehr lange, Turr. Nicht mehr lange.“

In diesem Augenblick richtete sich der verbleibende Kreideskral auf. Er hatte den verlorenen Knochenhelm aufgesetzt und Acks goldenes Schwert vom Boden aufgenommen. Die in den Knochenhelm eingelassenen Edelsteine glänzten rötlich und die Spitze des goldenen Schwerts war direkt auf Ack gerichtet.

Die waffenlose Ack streckte den verletzten Turr dem Kreideskral entgegen, als wolle sie ihn ihm übergeben.

„Turr! Feuerball!“, rief sie, und schloss schützend ihre Augen.

Der Feuertakuri gab einen weinerlichen Laut vor sich, doch gehorchte er. Ein gebrochener Flügel klatschte auf einen ganzen, und ein gewaltiger Feuerball hüllte das angreifende Echsenwesen ein. Dann ließ der Feuervogel schwach seinen Kopf hängen und sank in Acks Arme.

Der Feuerball schleuderte den Kreideskral mehrere Meter nach hinten. Der Knochenhelm löste sich von seinem kahlen Schädel. Aschespuren übersäten des Kreideskrals nackte Haut. Doch noch war er nicht geschlagen. Zitternd richtete der Skral sich ein letztes Mal auf und hielt Ack erneut das goldene Schwert entgegen.

Ack streckte den Feuervogel in die Höhe und rief erneut: „Feuerball, Turr!“ Doch diesmal folgte Turr nicht, sondern strampelte nur schwach quietschend in ihren Armen.

Der Kreideskral kicherte humorlos auf. Wut und Schmerz verwandelten sein von schwarzem Blut übersätes Gesicht in eine wütende Fratze.

„So sterbt doch endlich, elende Eindringlinge!“, brüllte er und sprang die schutzlose Ack an.\bigskip







Krgur, Häuptling der Kreideskrale, richtete sich keuchend vom kalten Höhlenbogen auf und bleckte seine Hauer. Die Eindringlinge in sein Heiligtum rannten um ihr Leben. Sein neuerdings unberittener Lumiwurm war der Steppenechse dicht auf den Fersen.

Krgur schmatzte am Stück Schnee, das er diesem Schneegeist aus der Schulter gebissen hatte. Er spuckte aus. Kein Geschmack, kein Nährwert. Und der Rest des Schneegeists war auf einmal nirgendwo mehr zu sehen.

Plötzlich spürte er ein Rieseln auf seiner Haut, als ob es regnen würde. Doch nein: Der seltsame Führer der Steppenechse, mit einem Bogen in der einen und einem seltsamen Beutel in der anderen Hand, bestreute ihn mit bläulichem Staub? Ein Schauer überfiel ihn und sein Körper zuckte unkontrolliert. Wie entwürdigend! Diesen Menschen würde er als erstes fressen! Während er auf ihn zu rannte, traf ihn plötzlich etwas Kaltes im Rücken und warf ihn wieder zu Boden. Hinter ihm stand die eisige Kreatur, weiß wie der hadrische Schnee, die Wunde auf der Schulter schon wieder zugewachsen. Aus eiskalten Augen starrte sie Krgur an. Mit einem Schwert aus Eis drang sie auf ihn zu! Schnell sprang Krgur hinter den uralten Altar der Höhle, um etwas Deckung zu haben. Den Uralten sei Dank: Die beiden Gegner wandten sich der dringenderen Bedrohung durch die heiligen Lumiwürmer zu und verfolgten ihn nicht.

Da, am Boden! Ein gehörnter Helm, und ein goldenes Schwert mit einem Griff in der Form der Mondsichel! Es war keine Schande, die Waffen seiner Gegner gegen sie zu verwenden. Krgur griff nach dem Helm und setzte ihn auf. Sanftes Brummen erfüllte seinen Kopf und gab ihm wieder Klarheit.

Krgur hob das goldene Schwert und richtete es auf die nur wenige Schritte davon entfernt sitzende Frau. Doch da schleuderte der brennende Vogel in ihren Händen einen Feuerball! Hart schlug Krgur auf dem Boden auf. Die Klarheit des Knochenhelms verschwand so rasch, wie sie über ihn gekommen war. Zitternd richtete Krgur sich wieder auf. Verbrannt war er, und entsetzt, doch an Flucht war nicht zu denken. Diese Fremden mussten gestraft werden dafür, in sein Reich eingedrungen zu sein, die Stille der Berge gestört zu haben, seine Sippenmitglieder ermordet zu haben. Sie würden gutes Futter abgeben.

Da stand die Frau, geschmückt mit Federn des Feuervogels, Furcht in den Augen! Allein war sie nicht stark. Wenigstens sie würde er kleinkriegen.

„So sterbt doch endlich, elende Eindringlinge!“, brüllte Krgur, und sprang die schutzlose Frau an.

Doch seine Füße landeten nicht sicher auf dem festen Boden.

Er rutschte aus. Die Welt kippte zur Seite. Die Luft wurde aus Krgurs Lungen gepresst, als er erneut hart auf dem Höhlenboden aufschlug. Wie viel Pech konnte man auch haben. Seine Hand öffnete sich und das goldene Schwert flog daraus hervor.

Als letztes sah Krgur die Klinge der Frau auf sich niederfallen.

Schmerz durchzuckte ihn.

Dann sah er nichts mehr. Und nahm auch nichts mehr war.\bigskip







Der Kreideskral sprang die schutzlose Ack an, rutschte auf einem kleinen Steinchen aus, stürzte zu Boden und versenkte Achs Schwert im eigenen Schädel. Ein Leben wurde ausgehaucht, doch die Gefahr war gebannt.

Ack langte nach vorne und hob das Steinchen vom Boden auf. Grünlich schimmerte es.

„Kein Kieselstein. Ein grüner Edelstein“, murmelte sie zu sich selbst. Ack tastete den Boden ab. „Eine exakte Kreuzung dreier Rillen. Eine markante Kuhle. Woher ... ?“

Es war nicht wichtig.

Die Kreideskrale und der Gor waren erledigt, doch die Lumiwürmer waren noch immer eine große Gefahr. Sabri war noch nicht aus der Höhle draußen.

Es war Zeit zu handeln, rasch, jetzt!

Iril stieß sich an der Höhlenwand in die Höhe. Die ganze Welt schwamm wie auf einem Schiff, aber sonst schien alles in Ordnung. Mehr oder weniger.

Und Iril erinnerte sich endlich an die Takuri-Spiegelscherben, die die Tulgori ihr für Ack überlassen hatten.

Sie zog die Scherben aus ihrer Tasche – schnitt sich an einem Finger und fluchte auf – stolperte nach vorne und hielt sie Ack entgegen.

„Hier, nimm diese Spiegelscherben! Jemand meinte, du könntest etwas damit anfangen.“

Mit geweiteten Augen antwortete Ack: „Woher hast du die? Takuri-Spiegel sind von großem Wert. Selbst ihre Scherben bergen noch besondere Kräfte.“

„Darum gebe ich sie dir ja! Du sollst wissen, wie man sie nutzt.“

„Klar, du kannst natürlich mich ebenso wenig verstehen, wie ich dich“, murmelte Ack niedergeschlagen, „Dennoch danke.“

Dem war dank Irils Übersetzungsrunen zwar nicht so, doch nun war auch nicht der Moment, um diesen Fehlschluss aufzuklären.

Ack richtete sich auf und nestelte an ihrer Halskette herum. Iril konnte nicht genau sehen, was sie damit tat. Auf jeden Fall streute sie irgendetwas aus die Spiegelscherben, woraufhin diese zu glühen begannen.

Ack schloss ihre Faust. Die Scherben zerbröselten und leuchteten noch heller auf, bis die dunklen Schatten von Acks Knochen unter ihrer Haut sichtbar wurden, als ihr Fleisch vom Glühen erfüllt wurde.

Ack rannte nach vorne, wo zwei mächtige Lumiwürmer kurz davor waren, die massige Transportechse zu verschlingen.

„Zurück!“, rief Ack und pustete einen Teil des Scherbenpulvers auf die Würmer. Sanft glitzernd breitete sich der Staub über die Lumiwürmer auf und nestete sich auf ihrer leuchtenden Haut ein. Die Würmer wurden langsam durchscheinend. Iril schüttelte sich, als die Innereien dieser Wesenheiten durch ihre Haut hindurch sichtbar wurden.

Ack schüttelte indes ihre Faust mit dem restlichen Spiegelpulver darin und pustete dieses in die gegenüberliegende Seite der Höhle. Wie von unsichtbaren Fäden dirigiert wirbelten die Ströme des glühenden Pulvers durch die Luft und kamen am anderen Ende des Raums zu stehen.

Ein Lichtblitz erhellte die Höhle und ein lauter Knall war zu hören. Und auf einmal befanden sich die Lumiwürmer nicht mehr direkt vor dem Ausgang und der Steppenechse, sondern wieder weit von den Helden entfernt, am anderen Ende des Altarraums.

Der Weg nach draußen war frei!

Iril und Ack – letztere noch immer mit dem verletzten Turr an sich gepresst – eilten nach vorne, wo Ijsdur und Barz etwas ratlos vor zwei sehr nahe beieinander liegenden Stolleneingängen standen. Sabri röhrte hinter ihnen.

„Durch welchen kamt ihr in diese Höhle hinein?“, fragte Barz Ijsdur. Dieser antwortete nicht. Und auch Iril wusste die Antwort nicht. Sie hatte vergessen, beim Betreten dieses Raums ein Zeichen an der Wand zu hinterlassen.

„Ich weiß es nicht mehr“, gab Iril zu.

Ijsdur überlegte kurz und sprach dann: „Es hat keinen Zweck, länger zu überlegen. Lasst uns einfach einen wählen.“

„Welchen?“, fragte Barz.

„Entschiede doch du.“

Das ist eine hohe Verantwortung.“

„Soll der Zufall lieber entscheiden lassen?“

„Können wir einfach möglichst schnell eine Entscheidung treffen?!“

„Fühle dich frei.“

„Links!“, rief Iril.\bigskip







Nach einigen Minuten der Flucht bog der linke Stollen urplötzlich um eine Ecke und über eine Leiter steil nach unten.

„Da kriege ich Sabri nicht runter“, fluchte Barz.

„Ist egal, es ist ohnehin nicht der richtige Gang“, sprach Ijsdur ernüchternd.

„Können wir noch umdrehen?“

Das Knarren und Knirschen, sowie ein immer heller werdender leuchtender Lumiwurm-Schein aus dem Stollen hinter ihnen, verriet ihnen die Antwort.

Ack sprach dem Feuervogel in ihren Armen gut zu, aber Iril glaubte nicht, dass dieser sie noch retten konnte.

„Ich glaube, das ist das Ende für Sabri“, murmelte Iril betrübt.

„Nein!“, rief Barz, „Nicht nach alledem! Es muss noch irgendeine Möglichkeit geben.“

„Gibt es tatsächlich“, sprach Ijsdur leise. „Ich war nicht nur dank meines Bergsinns so geschwind bei der Unterquerung des Kuolema-Gebirges. Barz, halte bitte Ausschau, wie nah die Lumiwürmer schon sind.“

Barz lugte um die Ecke und meinte ängstlich: „Schwer einzuschätzen, aber wir haben keine volle Minute mehr.“

„Wir brauchen keine volle Minute.“

Ijsdur legte seine Hand auf die rechte Höhlenwand und versteifte sich. Die Schneeflocken und Eiskristalle, die ihn stets umgaben, wirbelten noch schneller umher und wurden ... mehr? Iril zitterte, als die Temperatur merklich sank. Von der Stelle, wo Ijsdur die Stollenwand berührt hatte, breitete sich eine bläulich-weiße Schicht aus Schnee und Eis aus. Innert Augenblicken wuchs ein mannsgroßes, ja, gar steppenechsengroßes Tor aus purem Eis am Rand des Stollens. Dann schleuderte Ijsdur einen Eisblitz auf das Tor. Es knirschte und bröckelte beiseite, bis nur noch der Rand der Eisfläche bestand und in seinem Innern einen langen, breiten Gang aus purem Eis enthüllte. Ein Weg durch den Berg.

„Wie ... was?“, stammelte Iril.

„Jetzt ist doch wahrlich nicht die Zeit dafür“, merkte Ijsdur an, und betrat den frisch generierten magischen Eis-Tunnel. „Folgt mir! Mit etwas Glück stoßen wir am anderen Ende des Tunnels auf den richtigen Höhlengang, von dem aus wir hierherkamen.“

Iril wischte sich ein wenig lockeren Schnee von den Schultern. Sie hatte nicht einmal gemerkt, dass Schnee gefallen war.

Dann betrat auch sie den magischen Gang. Die Kälte ließ sie rasch erzittern.

Beim Zurückblicken sah sie, wie der Gang sich vor den leuchtenden Lumiwürmern wieder verschloss. Doch dunkel war es nicht. Irils Runenhammer und Acks Feuervogel ließen einen flackernden Schein über den kalten Stollen schimmern.

Sie waren in Sicherheit.






















\newpage
\section{Die Spur der Drachenknochen}


Schnaufend brach die Kämpfergruppe aus dem Höhlenstollen hervor. Barz hörte auf, an seiner Echse Sabri zu ziehen. Ack blickte erleichtert in den Sternenhimmel und atmete tief durch.

Es war Nacht geworden. Die Tulgori und Reka waren nirgendwo mehr zu sehen. Andor lag still da.

Die vier ließen sich unzeremoniell zu Boden fallen und begutachteten ihre Verletzungen. Barz rutschte zu Ack, zog irgendein glänzendes Mittel aus einer seiner vielen Manteltaschen hervor und machte sich daran, ihr angeschlagenes Gesicht zu beträufeln.

Ijsdurs betrachtete einen Schnitt in seiner linken Hand, der sich unter seinem Blick wie von selbst wieder schloss.

„Danke für den Tipp“, sprach er zu Iril, „Diese Kreaturen sind wirklich im Nacken verwundbar.“

Iril vermutete, dass die Kreideskrale so ziemlich überall an ihrem Körper verwundbar waren, beließ es aber dabei.

„Bäh!“, spuckte sie aus, während sie giftgrün leuchtende Spucke von ihren Armen rieb. Sie sog zischend Luft ein, als die Berührung viele kleine Wunden an ihrem Unterarm aufbrennen ließ. Sie fluchte. „Was ich in meiner Zeit in Silberland definitiv nicht vermisste, waren die riesigen Insekten und Spinnentiere der südlichen Gebirge. Im hohen Norden sind die wenigsten Viecher größer als eine Fingerbreit. Eine zwergische Fingerbreit. Aber nein, hier wimmelt es nur so von gewaltigen Arpachen und Spornen und Lumiwürmern und dergleichen.“

„Heißt das etwa, dass jemand alle nördlichen Insekten geschrumpft hat?“, fragte Ijsdur interessiert. Während alle anderen sich weiter verarzteten, saß er still am Boden und beobachtete aufmerksam das Verhalten der anderen.

„Nee, hier gibt es ja auch kleine Insekten“, meinte Iril, „Bienen und so. Eher hat jemand im Grauen Gebirge einige bestimmte Insekten vergrößert. Falls es überhaupt ein jemand war.“

Sie blickte zurück in den dunklen Stollen. Im schwachen Schein einer Laterne war in der Wand noch der Ausgang von Ijsdurs Eistunnel erkennbar.

„Wie hast du das gemacht? Mir wird schwindlig beim Gedanken an die nötige Kraft, so viel Felsmasse rasch zur Seite zu quetschen, um einen derart breiten Tunnel zu schaffen.“

„Es ist eher eine magische Passage“, meinte Ijsdur, „In einigen Tagen, wenn Ein- und Ausgang geschmolzen sind, wird einfach wieder blanker Fels dort stehen, wo jetzt der Tunnel liegt.“

„Bewegten wird uns überhaupt durch den Berg hindurch? Was wäre geschehen, wenn wir die Eiswand aufgebrochen hätten?“

„Ich nehme an, wir wären auf Felsen gestoßen. Vielleicht aber auch auf die Feenwelt. Oder das blanke Nichts. Wer kann das schon wissen?“

„Du, hätte ich vermutet?“

„Ich bin auch neu im Leben als Eis-Dämon. Der Durchgang wird auf jeden Fall nicht ewig halten und hat nicht so viel Felsen verschoben, wie du meintest.“

Iril nickte. Sie und Ijsdur schauten nun beide interessiert hinüber zu den beiden Neulingen.

Ack hatte sich hingekniet und streichelte Turr. Der Feuervogel protestierte fiepend, als Ack seinen verletzten Flügel untersuchte. Ack hob ihr goldenes Schwert, sprach den Kleinen gut zu ... und rammte es ihm ohne Umschweife in die Brust.

Iril machte große Augen: „Bist du verrückt! Diese Verletzung hätte noch heilen können!“

Barz hob die Hand und murmelte in der Sprache der Bewahrer: „Nicht verrückt. Er wird heilen. Warte.“

Iril erholte sich wieder vom Schock. Turr der Takuri sank in sich zusammen und begann, zu verglühen. Dann zerfiel er zu Asche. Und aus der Asche erhob sich ein klitzekleines Vogelküken, welches auf Ack zuwatschelte und sie mit großen Kugelaugen anhimmelte.

Barz entspannte sich sichtlich und erklärte: „Ack hier ist nicht nur eine Hüterin der Takuri – sie wird auch von einem dieser mächtigen Wesen begleitet. Turr heißt er. Im Kampf steht er uns zur Seite, und er wird stärker, je älter er wird. Doch auch Takuri sind nicht unsterblich ... also Vorsicht! Brennt die Flamme des Takuri zu heiß, endet sein Lebenszyklus. Ein Glück, dass die tapfere Ack ihn immer wieder beruhigen kann.“ Mit einem Blick auf Ack, die ihren Feuervogel streichelte, fügte er an: „Aber warum sollte man ihn beruhigen, wenn man damit nur sein Leiden verlängert? Ihn umzubringen und regenerieren zu lassen, ist manchmal humaner.“

Iril hatte gerade einiges zu verarbeiten. Sie murmelte: „Ach so. Wiedergebärende Vögel. Das ist ja toll. Takuri nennt ihr sie? Und dieser hier heißt Turr? Und die Dame heißt Ack?“

Barz schwankte seinen Kopf herum in einer Mischung aus einem Nicken und Kopfschütteln: „Ja, es wird ‚Ack‘ ausgesprochen, aber es wird anders geschrieben, als du jetzt wohl denkst: Erst ein A, dann ein C mit einem Akzent, und dann ein H. Also: Aćh.“

„Toll“, wiederholte Iril immer noch etwas baff. „Ich bin ... „

Ijsdur unterbrach sie, indem er sich vorbeugte und auf Tulgorisch meinte: „Ich will auch jemanden vorstellen. Das hier ist die Silberzwergin Iril. Sie hat die Geheimnisse der Runen gemeistert. Mithilfe ihrer Runenscheibe ist sie in der Lage, die Geschicke der Helden im Kampf und auch außerhalb zu beeinflussen. Eine der ungewöhnlichsten Personen, die ich je getroffen habe. Ihre besondere Art der Magie macht sie nicht nur zu einer starken Kämpferin. Sie versteht es auch, mächtige Sonderaktionen einzusetzen, die wiederum allen nützen können. Das macht Iril zu einem starken Teamplayer! Das habt ihr ja schon gerade am eigenen Leibe erlebt.“

„Ich kam mir jetzt weniger wie ein Teamplayer vor, wie ich den Großteil des Kampfes benommen am Boden lag“, murmelte Iril.

„Keine Sorge, das kommt noch“, meinte Ijsdur tröstend.

„Wenn du es meinst. Na dann: Aćh und Barz, das hier ist Ijsdur. Er ist quasi ein lebendiger Schneemann“, stellte Iril Ijsdur kurz angebunden vor. „Und dann bleibt ja nur noch unser Echsenfreund übrig.“

„Ich bin Barz. Und auch wenn mir ‚Silberzwerg‘ nichts Genaues sagt, so erkenne ich doch eine Runenmeisterin, wenn ich eine sehe. Auf der anderen Seite des Fahlen Gebirges kannte kaum jemand die Macht der Runen.“

„Soll das heißen, du stammst ursprünglich von dieser Seite des Fahlen Gebirges?“

„Weit aus dem Osten, ja. Ich komme vom großen See Ava in der Barbarensteppe.“

Iril horchte auf, als ihr müder Geist endlich zwei und zwei zusammenzählte: „Ja, davon habe ich zumindest schon mal im Ansatz gehö ... warte, du bist kein Tulgori?“

„Wie sonst könnte ich deine Sprache verstehen?“

„Ich dachte an irgendein Übersetzungspulver. Magische Pulver scheinen voll dein Ding zu sein.“

Barz kicherte auf. „Über solch ein magisches Mittel verfüge ich leider nicht. Zumindest noch nicht. Vielleicht werden wir es finden. Ich war oft in der Steppe unterwegs auf der Suche nach derartigen Kräutern. Ich und mein Freund Nabib. Wir sind Steppennomaden aus den östlichen Landen, obwohl wir vom eigentlich sesshaften Stamm der Iquar stammen.“

„Nabib“, murmelte Iril. Diesen Namen hatte sie auch schon gehört. Doch wo? An einen anderen Steppennomaden konnte sie sich nicht erinnern. Viele mit solchen Steppenechsen siedelten seit über zwei Jahren draußen im östlichen Rietland, auf Erde, die der König von Andor den anstürmenden Barbarenhorden „großzügig“ überlassen hatte – obwohl sie eigentlich den Schildzwergen gehört hatte. Pah!

Sie verzog ihr Gesicht beim Gedanken daran und begrüßte Barz freundlich: „Freut mich, dich kennenzulernen, Barz. Schöner Name übrigens. Ich kenne einen Handelszwerg, der so heißt. Oder zumindest ganz ähnlich.“

Barz grinste. „Garz, nicht wahr? Wie ich schon sagte, Nabib und ich sind wieder zu weiten Reisen durch die Barbarensteppe aufgebrochen, haben Sippen der Yetohe begleitet, und auch immer wieder fahrende Händler angetroffen. Einer davon, ein gewisser Nader, hat uns von Garz und seinem Riesenrucksack erzählt. Ich kenne den Handelszwerg von nah und fern zumindest vom Hörensagen.“

Irils Kopf schwamm, womöglich immer noch vom harten Aufprall auf den Höhlenboden.

„Und ... und wie gelangtest du nach Tulgor? Wussten die Barbaren schon lange, dass da noch eine andere Welt ist? Oder ... nein, sag nicht, dass ihr weit genug in den Osten wandertet, um wieder im Westen aufzutauchen.“

„Wie klein stellst du dir die Weltenkugel vor?!“, lachte Barz, „Nein, nein, ich habe es nur mit einem Experimentierpulver übertrieben. Zu fahrlässig war ich. Zu rasch wollte ich Nabib wiedersehen. Doch das Schicksal hatte andere Pläne. Auf der Suche nach Nabib führte es mich für eine lange Zeit nach Tulgor, und erst jetzt nach Andor und zu euch. Mein Teleport verfehlte sein Ziel. Oder mein Ziel befand sich nicht dort, wo ich wollte. Und danach verfehlte mein Ziel sein Ziel. Dank der Zukunftssicht des Hüters der Zeit weiß ich schon, dass ich Nabib wiedersehen wurde. Dass er sich irgendwo hier befindet. Doch hieß das nicht, dass dieses Treffen leicht werden wird.“

Barz‘ Blick verklärte sich und er wippte nervös mit seinem Knie herum. Irils einhunderttausend Fragen wurden artig heruntergeschluckt.

Aćh rutschte näher zu Barz, vorsichtig das kleine Takuri-Küken an sich pressend. Sie flüsterte ihm etwas ins Ohr. Auch wenn Iril Wortfetzen hören konnte, reichte dies offenbar ihren Übersetzungsrunen nicht aus, um ihr die Bedeutung zu übermitteln.

Barz‘ Blick zuckte herüber zu Ijsdur, seine Augen verengten sich und Iril dachte zurück an das Misstrauen, das die beiden Tulgori ... nein, die Tulgori und der Barbar dem Eis-Dämon bei seinem Auftauchen entgegengebracht hatten.

„Tu nichts Unüberlegtes, was du bald bereuen kön ...“, brachte Ijsdur noch heraus. Dann hatte Barz bereits hastig an seinen Gürtel gelangt und eine Faustvoll eines grünlichen Pulvers auf Ijsdur geschleudert.

Magisches Knattern ertönte und ein hellgrünes Glühen baute sich um Ijsdur auf. Glitzernder Dampf stieg auf. Barz zog seine Hände in seinen Mantel zurück und wedelte die Dampfreste davon, ehe sie noch mehr befallen konnten.

Ijsdur war bereits zur Gänze betroffen. Er wirkte wie eingefroren, mitten in der Bewegung des Aufstehens erstarrt, sein Mund halb zu einer Erwiderung geöffnet. Sein ganzer Körper war von diesem grünlichen Glühen umgeben. Und nicht nur er, sondern auch die ihn stetig umgebenden Schneeflocken hingen starr in der Luft, als stünde die Zeit still.

„Halt, halt, halt, was wird das denn?!“, rief Iril protestierend, „Ijsdur rettete gerade euer aller Leben!“

„Vielleicht nur, damit wir die nächsten Diener Siantaris werden können!“, rief Barz nun. Emotion schwang in seiner Stimme mit. Ob Wut oder Furcht, konnte Iril nicht sagen. Er brummte weiter: „Die Eis-Dämonen dienen der unergründlichen Herrin des ewigen Eises. Bedingungslos! Wir hatten schon mit einer zu kämpfen. Du weißt nicht, wie es sich anfühlt, ein blinder Diener eines Eis-Dämons zu sein. Das ist ein schauderhaftes Gefühl. Man ist in Kontrolle seiner selbst, doch sein Wille ist nicht der, der er einst war, und es ist einem egal. Ich gehe kein Risiko mit Eis-Dämonen ein. Nie wieder.“

„Ihr handeltet etwas zu hastig. Ijsdur steht nicht mehr auf Siantaris Seite. Ich habe ihn von ihrem Joch befreit. Meine Runen beschützen ihn vor ihrem Willen! Und Ijsdur hätte mir doch nicht von Siantari erzählt, wenn er noch unter ihrem Einfluss stünde.“

„Und woher wollen wir wissen, dass du nicht auch auf ihrer Seite stehst?“

„Abgesehen davon, dass ich kein Schneemann bin und keine Hörner trage? Soll ich nochmal unterstreichen, dass wir gerade euer aller Leben gerettet haben? Unter Dankbarkeit stelle ich mir eigentlich etwas anderes vor!“

Die beiden vor ihr blieben still. Barz war sprachlos und Aćh verstand sie wohl nicht einmal. Iril betrachtete argwöhnisch den magisch gebannten Ijsdur, hielt sich aber betont davon zurück, zu ihrem Hammer zu greifen. Eine Eskalation war das letzte, das sie nun gebrauchen könnten. „Ist das umkehrbar, Barz? Was hast du überhaupt mit ihm gemacht?“

„Das ist ein Bannpulver. Gefertigt aus getrockneten Mondbeeren. Gespickt mit Mitteln des Hexers Haamun. Es schadet Ijsdur nicht. Für ihn wird es sein, als wäre gar keine Zeit vergangen, sobald das Mittel gelöst wird. Dafür bürge ich. Ich war auch schon selbst davon betroffen.“

„Wie beruhigend“, gab Iril zurück. „Könnte dieses Mittel denn nun rückgängig gemacht werden?“

Barz blickte fragend zu Aćh, welche ebenso fragend zurückblickte. Barz erklärte ihr in knappen Worten, was Iril ihm erzählt hatte.

„Selbst wenn er ein freier Eis-Dämon außerhalb Siantaris Einfluss ist, macht ihn das noch nicht zu einem freundlichen Gesellen“, gab Aćh zu bedenken.

„Jedem anderen, der euch aus dieser Höhle gerettet hätte, hättet ihr wohl bedingungslos vertraut!“

Barz kraulte seinen kurzen buschigen Bart: „Und das Bannpulver?! Ijsdur riet mir in der Höhle, dass ich das Bannpulver für später aufbewahren sollte. Wie konnte er vom Pulver wissen? Beobachtete er uns schon seit langem? Las er meine Gedanken? Und wie kam er überhaupt aus dem ewigen Eis hinaus?!“

„Das könnte er dir wohl alles sagen, wenn du ihn nicht gebannt hättest“, gab Iril zu bedenken.

Barz knirschte mit den Zähnen und beriet sich mit Aćh. Dann zog einen violetten Stein aus einer Manteltasche hervor. Er schob ihn in seinen Mund und biss darauf. Es knackte, und als Barz seinen Mund wieder öffnete und den Stein in seine Hand spuckte, war dieser in mehrere spitze Splitter zerfallen. Von ihnen allen ging ein immer heller werdendes Licht aus, welches Iril in den Augen schmerzte. Und wo diese Lichtstrahlen auf das hellgrüne Leuchten des Bannpulvers trafen, verschwand das grünliche Glimmen in der Luft. Das Bannpulver löste sich. Schon war Ijsdurs Kopf wieder frei, während sein restlicher Körper weiterhin mitten im Aufstehen eingefroren schien. Da schloss Barz seine Hand wieder um den Stein und verhinderte, dass Ijsdurs restlicher Körper freikam.

„... ntest. Oh! Das ist ein äußerst unangenehmes Gefühl“, beendete Ijsdurs Kopf seinen Satz.

„Das tut mir leid, Ijsdur“, murmelte Barz.

„Das musst du nicht sagen, wenn es dir nicht wirklich leidtut“, sprach Ijsdur kühl. Er reckte probehalber seinen Hals und gab auf, seinen restlichen Körper bewegen zu wollen.

Barz nickte. „Lassen wir die Höflichkeiten. Falls du vertrauenswürdig bist, hast du soeben unser Leben gerettet und wir verhalten uns äußerst undankbar. Das tut mir durchaus leid. Aber falls du geheime Pläne mit unschönem Ausgang für die wärmeren Teile der Welt hegst, würde ich diesen liebend gerne ein Ende setzen. Wie können wir dir vertrauen?“

Ijsdur blieb einen Moment lang still. Dann sprach er: „Bedenkt meine vergangenen Aktionen. Und hört auf Iril. Ihre Runen ermöglichen es ihr, gemeine Lügen von ehrlichen Aussagen zu unterscheiden. Auch wenn ich selbst noch nicht herausgefunden habe, wie genau es funktioniert, könnt ihr sie als Lügendetektor nutzen. Ich vermute, dass ihr ihr ja vertraut.“

Eine leise Bitterkeit schwang in der sonst so tonlosen klirrenden Stimme mit.

Ijsdur blickte herüber zu Iril, welche betroffen zu Boden blickte. Sie seufzte tief und murmelte: „Ich wünschte, ich könnte da aushelfen. Doch im Namen der Offenheit muss ich zugeben, dass dies eine Täuschung war. Als ich dich frisch getroffen hatte, hoffte ich darauf, dich so auszuhorchen. Doch Lügen sollen nicht weitergesponnen werden, bis sie unniederreißbar groß sind. Alles, was ich sagen kann, ist, dass Ijsdur im Glauben, mir nur die Wahrheit sagen zu können, dasselbe sagte, was ich euch sagte. Er steht nicht mehr in Siantaris Diensten.“

Ijsdurs Miene blieb unergründlich. Nicht einmal eine Antwort kam von ihm. Dafür von Barz: „Hättest du das nicht erst in einigen Minuten auflösen können?! Nun könnte er sich alles Menschenmögliche ausdenken, wenn wir ihn nach dem Bannpulver fragen.“

„Vielleicht hätte ich sollen. Aber vielleicht ist es auch besser ...“

„Was wolltest du mich zum Bannpulver fragen? Warum ich darüber Bescheid wusste? Sagt, erkennt ihr beide mich nicht?“

„Sollten wir?“, fragte Barz argwöhnisch.

Ijsdur verzerrte sein Gesicht zu einer betont fröhlichen Miene und rief in einer melodischeren Stimme: „Willkommen auf dem Hängeschiff ARCTOR, meine sehr verehrten Herrschaften, Damenschaften und allerlei anderen Mitglieder der Gesellschaft. Mein Name ist ....“

„... Ijs?!“, rief Aćh ungläubig, „Der Kapitän des Hängeschiffs, das vom Sturmgeist angegriffen wurde?! Was ist mit dir geschehen?“

„Ich vermute, dass ihr das bereits richtig vermutet. Ijs verirrte sich im tödlichen Kuolema-Gebirge.. Ich bin Ijsdur, sein eisiges Ebenbild, geschaffen aus seinem Leichnam durch die Eiskristallkette meines ‚Vaters‘, des Eis-Dämons Beriandur.“

Barz runzelte seine Stirn und blickte Ijsdurs muskulösen Körper entlang: „Hast du an Muskelmasse zugelegt? Wie läuft das mit diesen Schneekörpern, kannst du bis zu einem gewissen Grad bestimmen, wie er aussieht?!“

„Ich sehe, dass mein Körper nicht identisch zu Ijs altem ist. Und doch glaube ich nicht, direkte Kontrolle über seine Form zu haben. Ein bislang unerforschtes Phänomen.“

„Und Siantari hat dich wirklich nicht in ihrem Griff?“

„Einst tat sie es, doch ich bin kein blinder Diener Siantaris mehr“, sagte Ijsdur, „Doch bin ich auch nicht Ijs. Zu wenig Menschliches ist in mir übrig, um diesen Namen weiterzuführen. Ijs‘ Vater Saro sagte, er erkenne seinen Sohn nicht mehr in mir, und ich stimme ihm teils zu. Tulgor bedeutet mir nichts mehr. Meine Heimat ist das ewige Eis. Die Schluchten des Kuolema-Gebirges. Vielleicht wäre es besser für alle, wenn ich einfach dorthin zurückkehrte. Wenn ihr das wollt ...“

„Machst du Witze?“, meldete sich nun Iril zu Wort, „Bei allen Kreaturen der Tiefe, deine Anwesenheit hier ist ein Geschenk! Wenn ihr beide nicht bald aufhört damit, Ijsdur schlecht fühlen zu lassen, mögt ihr beide euch von uns trennen!“

„Wir waren damals nur vom Wunsch erfüllt, Siantari zu dienen“, erinnerte sich Aćh erschaudernd, „Und wir beiden hatten unsere Ketten nur kurz berührt. Wie kann das bei dir anders sein, erst recht nach all dieser Zeit?“

„Magie“, meinte Ijsdur schlicht. „Runenmagie. Scheint genug, um Siantaris Einfluss fernzuhalten.“

„Für den Moment“, murmelte Barz.

„Für den Moment“, bestätigte Ijsdur, „Siantari hat die Eiskristallketten geschaffen und in den Umlauf gebracht. Niemand außer ihr kann wissen, wie sie aus ihrer Hand befreit werden können. Und das heißt ...“

„... dass Siantari dich immer noch in der Hand haben könnte, wenn du ihr zu nahe kommst“, murmelte Barz bedrückt, „Vielleicht reicht es auch nur, wenn sie genügend stark an dich denkt.“

„Nein“, sprach Iril nun bestimmt, „Diese Runenscheiben funktionieren. Dutzende Geister haben sie bereits ausgetrieben. Wenn sie verfluchte Seelen vom Silberberg fernhalten können, dann auch den Willen einer Eis-Dämonin.“

„Wir können hoffen.“

Stille.

„Kann Siantari wahrnehmen, was du wahrnimmst?“, fragte Barz nun besorgt.

„Vielleicht?“, murmelte Ijsdur bedrückt, „Wenn, dann höchstens auf Wunsch. Es gibt so einige halb verrückte Durs und Doras im Eisgebirge. Sie kann sich kaum all deren Geister gleichzeitig bewusst sein. Vielleicht weiß sie ja nicht einmal, dass ich hier bin.“

„Weiß sie nicht“, bekräftigte Iril noch einmal. „Die Gefahr durch sie ist noch nicht gebannt. Aber Ijsdur ist nicht Teil davon.“

„Großartig, dann kennen wir uns ja alle und sind neue beste Freunde“, meinte Barz. Er öffnete seine Faust wieder. Das letzte Licht der violetten Steinsplitter befreite Ijsdur vollends vom Bann des Bannpulvers. Er stürzte zu Boden und richtete sich betont langsam auf.

Unsicher blickten die vier einander an.

„Was nun?“, fragte Barz.

„Zuallererst erkläre ich wohl meine Übersetzungsrune“, meinte Iril. „Ihr habt alle schon Tulgorisch gesprochen, also nehme ich an, dass ihr das schon kapiert habt, aber noch mal explizit: Dank dieser Rune an meinem Kopf können ihr drei auf Tulgorisch schwafeln und wir verstehen einander alle. Außer Aćh mich. Aber damit sollten wir klarkommen.“

Barz übersetzte und Aćh lächelte erleichtert, beim Gedanken, nicht in Zukunft von den meisten Gesprächen ausgeschlossen zu sein.

„Ansonsten könnten wir demnächst noch die Hexe Reka für einige Übersetzungstränke anhauen“, bemerkte Ijsdur, „Ich bekam mit, wie sie Eforas welche anbot.“

„Au ja, solche Tränke kenne ich schon von meiner Schamanin Asbark!“, rief Barz, „Asbark hat das Rezept dazu vom Fleisch gewordenen großen Büffel höchstpersönlich erlernt. Also, eher von Mitgliedern der Yetohe-Sippe, die es vom Yjotege grate hatten, aber das sollte kein Problem sein, orale Tradierung ist durchaus verlässlich. Auf jeden Fall ein wahres Wunderwerk der Kommunikation.“

Er drehte sich zu Aćh herum und stupste grinsend sie in die Seite: „Sieht so aus, als wären unsere Rollen nun vertauscht. Bist du bereit, eine neue Sprache zu lernen?“

Aćh nickte tapfer.\bigskip







Die vier beschlossen, fürs Erste zusammenzubleiben, ein Lager einzurichten und die Nacht hier zu verbringen. In einer sicheren Senke ein bisschen oberhalb des Stollenausgans.

Barz sammelte Holz. Aćh brach eine feurige Feder – welche sie aus einer Tasche zog und nicht etwa dem kleinen Küken abrupfte – und entzündete das Lagerfeuer. Iril kritzelte auf ihrer Runenscheibe herum. Ijsdur saß gehörig vom Feuer entfernt und schien zutiefst in Gedanken versunken. Hin und wieder ertappte Iril ihn dabei, wie er die beiden Neuzugänge verstohlen anguckte. Und sie ihn ebenfalls.

Aćh traute sich als Erste zu Wort: „Wie hast du es eigentlich aus dem Tal des ewigen Eises geschafft? Wir selbst bekamen mit, wie Eis-Dämonen vom Felsentor aufgehalten wurden. Hat Siantari es endlich geschafft, jemanden außerhalb des Felsentors zu wandeln?“

„Nein, auch Ijs starb mitten im ewigen Eis. Eines Tages sprach Santari einfach, dass wir frei wären. Und wir waren es.“

„Das magische Felsentor war offen? Ihr konntet einfach durchmarschieren?“

„Offenbar. Keine Ahnung, warum.“

Die Helden bereiteten sich auf Schlaf vor. Die beiden Mitglieder der tulgorischen Reisetruppe hatten Decken dabei. Auch Iril trug stets eine dünne Plache in ihrem Rucksack. Das hatte sie schon auf Silberland getan, für die Fälle, wo die Mondscheinspaziergänge mit Burmrit an der Bronze- oder Kupferküste zu lange andauerten, um in die Mine zurückzukehren.

Kaum waren drei der vier in Schlafgewänder gewechselt und hatten sich rund ums Feuer verteilt, blickten sie erwartungsvoll Ijsdur an.

Nun war es Barz, der das Wort ergriff: „Sag, Ijsdur, musst du auch schlafen?“

„Es wäre überaus praktisch, wenn ich es nicht müsste. Aber doch, mein Körper muss ruhen. Selbst Nahrung zu sich nehmen muss er, solange er sich nicht auf dem ewigen Eis befindet und von dort aus Energie aufnehmen kann.“

„Wie kann das überhaupt sein, dass du Nahrung zu dir nehmen musst, obwohl du gar kein Blut hast? Du ... du hast doch gar kein Blut, oder?“

„Woher willst du das wissen? Nur weil du schon mal einen anderen Eis-Dämon abgestochen hast?“, gab Ijsdur ein bisschen schnippisch zurück, „Was soll Essen denn mit Blut zu tun haben?“

Barz erklärte: „Wohin soll die Kraft des Proviants ohne Blut schon hin? Der rote Saft transportiert sie durch den Körper dorthin, wo sie hinsoll. Meine Schamanin Asbark erzählte immer, dass man im Falle einer Überessung ...“

„Quatsch, Essen geht doch nicht ins Blut! Das geht durch den Magen und dann hinten wieder heraus.“

Nun war auch Iril neugierig geworden: „Ijsdur, musst du dich manchmal auch erleichtern? Sieht es auch aus wie Schnee?“

Da fragte Aćh auch schon: „Können Eis-Dämonen immer noch schwanger werden?“

Ijsdur blickte sie alle ausdruckslos an. Zu guter Letzt murmelte er nur: „Das kann ich noch nicht wissen. Und das werde ich vielleicht nie wissen.“

Aćh fragte: „Vermitteln diese Eiskristallketten nicht irgendwie intuitives Wissen zum Dämonen-Dasein?“

„Ja, aber solche Dinge weiß doch nicht mal Siantari. Im ewigen Eis gelten für uns andere Regeln als hier in der warmen Welt.“

„Du hast du dich also noch nie erleichtert?“

„Noch nicht, aber ich bin ja erst kurz vom ewigen Eis weg und habe nur wenig gespeist.“

„Da war ich dabei!“, rief Iril, „Er hat einen ganzen Teller Bohnensuppe in einem Happs in sich hineingeleert und seither kein Quäntchen mehr verzehrt.“

„Ich verstehe nicht, weshalb dich das so mit Freude erfüllt.“

„Ist nur interessant, zu lernen, wie dein Schneekörper funktioniert.“

„Ah.“

Stille.

„Was ist der Plan für morgen?“, fragte Barz.

„Wir sollten die anderen Tulgori aufspüren“, meinte Aćh.

Ehe sie weitersprechen konnte, berichtete Iril: „Nun, Ijsdur und ich waren eigentlich auf den Spuren von Drachenknochendieben, ehe ihr aufgekreuzt seid. Wollt ihr uns dabei unterstützen, sie zu finden? Oder vertraut ihr Ijsdur zu wenig dafür?“

Barz‘ Stimme wurde müder. „Drachenknochen? Faszinierend. Meine Lehrmeisterin meinte, dass Drachenknochen vor Jahrhunderten sehr beliebte Mittel für stärkende Pulver waren“, murmelte Barz müde, „Doch diese Zeiten sind vorüber. Es gibt so gut wie keine Drachen mehr.“

„Genau“, sprach Iril, „Tarok, der hoffentlich letzte Drache, ist gerade erst gestorben. Es geht hier um seine Knochen.“

„Tarok?!“, rief Barz plötzlich wieder hellwach. Seine Stimme hatte einen zitternden Unterton angenommen.

„Ja, Tarok, der Gewaltige. Der Mächtige. Der Rächer. Er soll viele Titel haben.“

Barz murmelte leise, wie zu sich selbst: „Haamun sagte noch, dass das Öffnen des Stollens irgendetwas mit einem Drachen zu tun hätte, aber dass es ausgerechnet Tarok sein musste!“

„Kennst du Tarok?!“, fragte Iril.

„Er hat Sabris Mutter ermordet“, kniff Barz zwischen zusammengebissenen Zähnen hervor.

„Was?!“

Barz atmete schwer und zückte eine kleine Drachenfigur aus seiner Manteltasche. Mit betont kontrollierter Stimme sprach er weiter: „Dies ist eine Figur von Tarok. Unser Kind schnitzte diese Figur, nachdem Tarok unsere Pfahlbausiedlung überfallen und unsere Steppe in Brand gesetzt hatte. Karyz hoffte ... nun, eigentlich ging Tarok einfach allen nicht aus dem Kopf.“

„Das wusste ich überhaupt nicht“, sprach Aćh baff. „Du hast nie erzählt, dass du so nahe mit einem Drachen zu tun hattest.“

„Du hast nie gefragt“, sagte Barz.

Aćh murmelte etwas davon, dass sie ja nichts wusste von diesem Land, in der sie sich hier befanden. „Ich kenne seine Geschichten kaum. Weder seine Orte noch seine Leute. Und nicht einmal von dir, Barz, weiß ich so viel, wie ich dachte. Welcher Geist hat mich schon wieder geritten, hierher zu kommen?“

Sie kuschelte sich an Barz. Dieser legte einen Arm um sie und sprach ihr beruhigend zu. „Die Vergangenheit ist vergangen. Da muss man nicht alles voneinander wissen, um auf gute zukünftige Pfade zu kommen. Ich verstehe, wie man sich in einem fremden Land fühlen kann. Mir ging es nicht anders, als es mich nach Tulgor verschlug. Gar schlechter sogar. Doch du und die deinen waren auf jedem Schritt an meiner Seite. Und im Gegensatz zu mir kannst du jederzeit nach Hause aufbrechen, wenn dich das Heimweh übermannt.“

„Was, an den aufgebrachten Lumiwürmern vorbei?!“

„Zugegeben, so einfach wird das nicht werden. Aber auch nicht unmöglich. Erst recht nicht jetzt, wo der Weg über die Berge wieder offensteht. Mit den richtigen Leuten und Mitteln an unserer Seite ist alles möglich.“

Iril ertappte sich beim Starren und wandte ihren Blick Ijsdur zu. Dieser sah ihr ausdruckslos entgegen. Auf seiner Brust glitzerten wie üblich die Stacheln seiner magischen Kette.

Iril dachte zurück daran, was die anderen erzählt hatten. „Sieht so aus, als wäre ich die Einzige hier, die noch nie einer solchen Eiskristallkette unterlag.“

„Es ist gar nicht so seltsam, sie zu tragen. Willst du ausprobieren, wie es sich anfühlt? So eigenartig ist es gar nicht. Außer ihr Zwerge reagiertet anders darauf mit eurem vor Dunklen Magie gefeiten Blut und so.“

Ijsdur streckte seinen Hals vor. Iril betrachtete die glitzernden Kettenglieder, die mit seinem Hals verschmolzen waren. Iril fasste sich ein Herz, streckte ihre Finger hoch und streifte einen eiskaltes Eiskristall. Er war so kalt, dass die Berührung in ihre Haut stach.

Und auf einmal war alles dumpfer. Ihr Ärger gegenüber den undankbaren Neuankömmlingen. Die Sorge um die diebischen Drachenkultisten. Die sich stetig vordrängelnden Gedanken um ihre tote Runenmeisterin Burmrit und um ihre Familie. Sorge um Iolith. Selbst die vor Kälte bibbernden Finger auf Ijsdurs Eiskristallkette. All diese Wahrnehmungen waren noch hier, doch abgeschwächt, weniger mitreißend.

Irils und Ijsdurs Geister streiften einander. Iril spürte, wie eine fremde Präsenz durch ihre Erinnerungen huschte. Bilder und Emotionen ebbten ebenso schnell auf, wie sie aufgewallt waren. Dann war es schon wieder vorbei. Iril ließ die Eiskristallkette los und trat einige Schritte zurück.

„Beeindruckend“, sprach Iril, „Das muss ich mir merken, für wenn mein Gemüt mir mal übel mitspielen sollte.“

„Du bist gerne eingeladen, die Kühle der Kette zu genießen, wenn du sie brauchen könntest.“

Iril blickte sich um und machte Blickkontakt mit Aćh und Barz, welche offenbar ihr Gespräch unterbrochen hatten und ängstlich Irils Reaktion beobachten.

„Keine Dämonin hier“, grinste Iril. „Wie schon gesagt: Ijsdur ist geschützt vor den fremden Einflüssen aus dem ewigen Eis.“

Die beiden entspannten sich.

Im Norden schrie eine Eule.

Iril blickte in die Ferne. Von ihrem Lagerplatz am Hang des südlichen Gebirges waren in weiter Ferne gerade noch so die Turmspitzen die Rietburg zu erkennen. Feuerschein aus vereinzelten Fenstern erhellte das Gemäuer.

Aćh wog die tulgorische Steinflöte in ihren Händen und betrachtete nachdenklich die Silhouette der Rietburg in weiter Ferne, bevor sie die Flöte an ihre Lippen setzte und behutsam eine Melodie zu spielen begann. Ein ruhiges, doch fröhliches Lied, das die Helden aufmunterte und ihnen half, sich auf die Planungen ihres Vorgehens gegen die bedrohlichen Drachenkultisten zu konzentrieren. Während die Töne der Flöte in die dunkle Nacht schallten, tanzte Turr der Takuri fröhlich über ihren Köpfen zu den kunstvollen Klängen.

Die letzten Klänge der fremdartigen Melodie verklangen. Barz klatschte Applaus. Die anderen stimmten mit ein.

Sabri schnarchte auf.

Es war Zeit, sich schlafen zu legen.\bigskip







Die ganze Küche war voller Schneewirbel. Kristalle schmolzen auf dem heißen Metall des Ofens in der Ecke. Saros Kleidung war eiskalt und feucht, sein langer Bart übersät von Reif, als er zitternd sprach:

„Was willst du noch hier? Du wirst mich bald verlassen, wie Eforas uns verließ. Wie Ijs‘ Mutter uns alle verließ. Es scheint mein Schicksal zu sein, allein zu bleiben. Ziehen wir es nicht unnötig lange heraus. Alles ist gesagt. Du bist hier nicht mehr willkommen, Ijsdur.“

„Ijs ... dur? Bin ich das? Bin ich nicht mehr Ijs? Bin ich kein Mensch? Bin ich ein Eis-Dämon?“

„Du bist kein Mensch“, brummelte Saro, „Du bist nicht als eine Erinnerung, die sich im ewigen Eis festsaß. Und schmerzhafte Erinnerungen gibt es schon so zu viele in diesem Haus.“

„Vater, bitte, deine Schmerzen werden sich durch diese Worte nicht lindern. Und die meinen auch nicht.“

„Vater?!“, lachte Saro bitter, „Du bist nicht mein Sohn. Mein Sohn starb auf dem Gebirge. Ich erkenne dich nicht wieder.“

Er griff nach einer Mistgabel und hielt sie in die Flammen des Ofens, bis ihre Zinken rötlich glühten.

„Du weißt nicht, was du tust“, sprach Ijsdur kalt, „Lass die Waffe sein.“

Saro ließ die Mistgabel los und stolperte einige Schritte zurück. Mit Tränen in den Augen blinzelte er zu Ijsdur hoch. Er öffnete seine Arme, als wolle er ihn umarmen. Dann ließ er sie abrupt sinken und rannte aus dem Raum.



Der Traum wandelte sich, änderte sich, spulte vor.



Die ganze Küche war voller Schneewirbel. Kristalle schmolzen auf dem heißen Metall des Ofens in der Ecke. Saros Kleidung war eiskalt und feucht, sein langer Bart übersät von Reif, als er zitternd sprach: „Nein. Ich lasse dich nicht gehen, Ijsdur.“

„Ich habe dich nicht gefragt, Vater. Ich habe dir mitgeteilt, was ich tun werde. Ich werde nach Andor aufbrechen und Eforas suchen, ganz egal, was du willst.“

„Nein“, wiederholte Saro, „Ich sehe, was du vorhast. Warum du dich plötzlich so sehr für die Länder im Osten interessierst. Der Herr des ewigen Eises hat von Tulgor abgelassen und ein anderes Ziel gerochen. Ein Land, in dem niemand weiß, wie man Eis-Dämonen einsperrt. Du willst Eforas abfangen, ehe er die Andori vor euch warnen kann. Deine Herrin hofft, dass du ihn als seinen Bruder täuschen könntest.“

„Saro, wie kannst du so etwas denken?!“, entrüstete Ijsdur sich, „Die Herrin Siantari entließ uns aus ihren Diensten. Wir Eis-Dämonen sind frei. Wir können handeln, wie wir wollen.“

Das war eine Lüge. Ijsdur war frei gewesen, das ewige Eis zu verlassen und nach Tulgor zurückzukehren. Siantari hatte ihr Interesse daran fürs Erste verloren. Doch sobald Ijsdur von Eforas‘ Aufbruch erfahren hatte und sobald Siantari von Ijsdurs Verbindung zu Eforas erfahren hatte, hatte sie sofort seinen Geist wieder in ihre Klauen gepackt und ihm gedanklich befohlen, die Reisegruppe zu verfolgen. Befohlen, nach Andor zu gehen und mögliche Verteidigungen auszuspionieren. Tari würde nicht ein zweites Mal in eine Falle gehen. Die Andori durften nicht vorgewarnt werden.

Und so hatte Ijsdur sich von Saro verabschiedet. Vielleicht hätte er lügen sollen. Denn nur die Erwähnung Eforas‘ hatte Saros Alarmglocken schellen lassen. Und nun schien Saro ihm nichts mehr zu glauben.

Saro griff nach einer Mistgabel und hielt sie in die Flammen des Ofens, bis ihre Zinken rötlich glühten. Wenn er sich gegen ihn stellte, müsste Ijsdur ihn umbringen. Das wollte er nicht.

„Deine Meinung ist bedeutungslos, Saro“, sprach Ijsdur kalt, „Dein Leben ist bedeutungslos. Darum verschone ich es. Verschwende es nicht, indem du mich aufzuhalten versuchst.“

Saro fuhr herum und hielt die feurige Mistgabel in Ijsdurs Richtung.

Ijsdur lachte kalt auf. „Guck mir in die Augen. Du traust dich nicht. Ich bin immer noch dein Sohn.“

Saro zögerte. Dann schnellte er vor und zog die feurige Waffe über Ijsdurs Brust. Tiefe Furchen hinterließ sie. Die Eiskristall-Kette zerbarst in kleinste Teile, welche wie Rauch verdampften. Ijsdurs Bewusstsein wurde ebenso in kleinste Teile zerteilt, welche alle klagend aufschrien. Ein Zittern durchlief seinen ganzen Körper.

Er schmolz dahin.



„Nein“, dachte er, „So war das nicht. Saro hätte nicht so gehandelt. Er hätte es nur sollen.“\bigskip







Ijsdur schreckte hoch.

Offenbar vermochte er nicht nur zu träumen, sondern auch albzuträumen.

Er schwitzte nicht mehr, doch keuchte er angestrengt. Seine Emotionen mochten gedämpft sein im Vergleich zu denen eines Menschen, doch ausgelöscht waren sie deswegen nicht.

Er musste sich beruhigen.

Es war tiefste Nacht.

Die Sterne funkelten, der Boden brummte und sein Magen grummelte. Ihm wurde bewusst, dass er außerhalb des ewigen Eises wohl tatsächlich hin und wieder eine Notdurft verrichten musste. Ijsdur hatte in seiner kurzen Zeit in Andor schon von den andorischen Scheißhäusern erfahren. Was Komfort und Geruch anging, waren diese leider den öffentlichen Toiletten Tulgors unterlegen, doch waren sie dem Gang in die Natur durchaus vorzuziehen. Wobei Ijsdur hier, am Rande des südlichen Walds wohl kaum einfach so ein Scheißhaus antreffen würde. Leise schlich Ijsdur sich ins nächstbeste Gebüsch genügend abseits des gelöschten Lagerfeuers und tat, wie es ihm die Natur gebot.

Dann hielt er inne.

Ein Windhauch wehte durchs Gebüsch und um Ijsdur herum. Er trug einen Geruch nach Metall und Verwesung mit sich.

Etwas war falsch hier.

Jemand war hier. Jemand beobachtete ihn.

In weiter Ferne glitzerten zwei schneeweiße Augen hinter einem Gebüsch hervor. Pupillenlos, wie die von Kreaturen, Drachen oder Riesen – oder Eis-Dämonen. Ijsdur kniff seine Fenster zur Seele zusammen und spürte, wie sich etwas in seinem Kopf verschob. Auf einmal konnte er die fern liegende Gestalt scharf erkennen.

Es war zweifelsohne eine Person. Sie schwebte über dem Boden, der wallende Umhang ihre tatsächliche Form verhüllend. In der einen Hand führte sie ein Schwert. Ihre zwei glühenden Augen stachen hinter einer gezackten eisernen Maske hervor.

„Der Schwarze Herold, nehme ich an?“, brach Ijsdur die Stille, „Solltest du nicht über deine Kultisten wachen? Was hast du hier bei uns vor?“

Laut klirrte seine kalte Stimme durch die stille Nacht. Die Gestalt rührte sich nicht. Was überlegte sie?

„Wir wollen den Kultisten nichts Böses, das kannst du ihnen ausrichten“, fuhr Ijsdur fort, „Iril will nur die Drachenknochen zurück. Wir finden euch mit ihrer Hilfe bald, da könnten wir uns doch genauso gut gleich auf ein Treffen einigen.“

Der Herold schwebte wortlos in der Dunkelheit, als wäre er nichts mehr als ein Schatten, den ein seltsam geformter Baum im Mondlicht warf. Doch Ijsdur wusste es besser.

„Wenn du dachtest, dass ich auf deiner Seite wäre, muss ich dich enttäuschen“, murmelte Ijsdur, „Ich unterstütze das Böse nicht. Da können dir andere Eis-Dämonen besser weiterhelfen. Bitte, so sage doch etwas.“

Er blinzelte. Auf einmal war der Schwarze Herold verschwunden. Hatte Ijsdur dies nur geträumt? Bis vor wenigen Tagen hatte er nicht einmal schlafen müssen. Nein, noch konnte Ijsdur zwischen Traum und Realität unterschieden. Der Herold war wirklich hier gewesen. Das konnte kein gutes Zeichen sein.\bigskip







Als Ijsdur ins Lager zurückkehrte, drehte sich Aćh rasch auf die andere Seite. Aber nicht so rasch, dass es ihm nicht aufgefallen wäre.

„Belauschtest du meinen nächtlichen Toilettengang?“, fragte Ijsdur leise, während er es sich wieder auf seinem behelfsmäßigen Nachtlager gemütlich machte. „Keine Anschuldigung, nur eine ernsthafte Frage.“

Eine Zeit lang kam keine Antwort. Er seufzte und stellte sich darauf ein, ins Reich der Träume überzugleiten. Doch dann ...

„Ich hörte dich sprechen. Mit wem hast du gesprochen?“

Aćh klang misstrauisch. Das war auch nur zu verständlich, wenn auch ärgerlich. Ijsdur schien es am geschicktesten, offen mit ihr und Barz zu sein, bis sie ihr Misstrauen gegenüber den Seinen revidiert hätten.

„Das war der Schwarze Herold. Ich habe mich mit ihm unterhalten. Einseitig.“

„Soll das ein Witz sein?“, meldete sich eine weitere verschlafene Stimme. Iril war nun auch wach.

„Kein Witz“, bestätigte Ijsdur. „Der Herold hat wirklich einfach hier geschwebt. Und uns beobachtet.“

„Wer ist der Schwarze Herold?“, verlangte Aćh zu wissen.

„Der Vorbote allen Unbills. Der Heerführer der Kreaturen. Der höchste Diener des Drachen. Eine Ausgeburt des Bösen“, gab Iril theatralisch zurück.

„Eine finstere Person, die das Böse unterstützt“, übersetzte Ijsdur knapp für Aćh. Iril schien sich damit zufriedenzugeben, denn sie schnarchte wieder lautstark los.

„Was wollte dieser Herold von dir?“

„Von mir? Keine Ahnung. Vielleicht hoffte er, dass ich ihn unterstütze? Doch warum ausgerechnet mich?“

„Könnt ihr euch alle mal ruhig verhalten?“, erklang Barz‘ Stimme, gefolgt von einem lauten Gähnen. „Manche hier würden gerne schlafen.“

Aćh schwieg. Nachdem keine Antwort oder weitere Frage von ihr folgte, ließ sich Ijsdur ebenfalls endgültig wieder ins Reich von Gevatter Schlaf mitziehen.\bigskip







Barz wurde beim Sonnenaufgang als Erster wach und blinzelte in den sich rötenden Himmel. Er vermisste die muffige Stollenluft wirklich nicht. Er reckte und streckte sich und zog aus in die Umgebung, auf der Suche nach Mondbeeren, welche man besonders gut im Lichte der ersten Sonnenstrahlen finden konnte. Leider wurde er hier nicht fündig. Nur mit einigen gesammelten Apfelnüssen als Ausbeute kehrte er zum Lager zurück, streichelte die im Schlaf grunzende Sabri hinter den Ohren und knuddelte Turr, welcher bereits aktiv umherhüpfte.

Iril schälte sich aus ihrem Nachtlager und beobachtete den Takuri aufmerksam. Turr sah schon wieder älter als ein Küken aus. Seine flaumigen Federn strahlten und seine Schwanzspitze sprühte gar Funken.

„Er wächst viel zu schnell, selbst für einen Takuri“, murmelte die aufgewachte Aćh sorgenvoll, „Schon seit seinem Verschwinden damals mit Barz. Ich frage mich, ob er die Welt in Zeitlupe wahrnimmt. Aber er scheint nicht zu leiden. Also ist es wohl in Ordnung.“

Emsig machten die vier sich ans Aufräumen ihres Lagers und an die Vorbereitungen für die Abreise. Aćh erzählte Barz nebenher von ihrem nächtlichen Traum, in welchem sie einen störrischen Takuri davon überzeugen musste, einen leuchtenden Mera-Stein vom Himmel zu holen, damit der Nistbaum der Takuri mit ihm tanzen konnte.

Barz lachte und präsentierte der Gruppe seine gefundenen Apfelnüsse sowie einige Streifen Dörrfleisch aus einem Sack von Sabris Rücken.

Iril fand in ihrer Reisetasche einige Tarenkugeln – gefüllte Teigtaschen, die der kultivierte und feingeistige Bragor gerne zubereitete und anschließend anderen verschenkte, sofern er sie nicht sofort ohne Essforken und Löffeln verzehrte – sowie einen letzten Brocken würzigen Silberbrots. Barz wurde auf die goldbraunen Teigbälle und das glitzernde Gebäck aus Silberhall aufmerksam und bat um Kostproben – welche er sofort mit einem rötlichen Pulver aus seiner Gewürztasche schmackhaft würzte.

Proviantrationen wurden verteilt, verzehrt und beurteilt.

„Köstlich! Diese Teigkugeln munden auch kalt. Sie würden sich bestimmt großer Beliebtheit erfreuen, wenn sich ihr Rezept weiterverbreitete.“

„Warum ist euer Brot so fade? Da gehört eine gesalzene Ladung Drachenfrüchte darauf. Die Drachenfrüchte am Ava wachsen nach oben, wie es der große Seeadler will. Vielleicht wachsen eure hier nach unten und verlieren deswegen ihren Geschmack.“

„Was, ihr tut gar keine Drachenfrüchte da rein? Ihr verpasst etwas. Drachenbohnen sind eine meiner Leibspeisen.“

„Sollen Apfelnüsse so bitter schmecken?“

„Nicht alle. Aber es ist gut, wenn sie es tun. Dann haben sie meisten Nährstoffe.“

Nur Ijsdur meinte, er sei nicht hungrig, und schaute stumm zu, wie die anderen Helden speisten.

Nach dem Morgenmahl flocht Iril sich einen neuen Haarkranz. Barz stülpte seinen schweren Mantel über und pflegte seinen Bart. Aćh legte ihre zeremonielle Rüstung an und salbte ihre Filzlocken mit feuerfestem Sufar ein. Ijsdur behielt seinen Rock an und blickte gedankenversunken in die Ferne. Vielleicht war seine Kleidung Teil seines Schneekörpers und musste darum nicht gewechselt werden.

Anspannung lag in der Luft. Iril rief sich ihre Hauptaufgabe wieder in Erinnerung: Das Aufspüren der gestohlenen Drachenknochen.

Ehe Iril erneut zu ihrer Runenscheibe greifen musste, um die aktuelle Position der Knochen anzupeilen, stürzte ein kleiner Krark aus dem Himmel herab, kaum größer als Irils Unterarm. Der Raubvogel landete ungelenk auf dem Boden und erhob sich flatternd wieder in die Lüfte. Barz griff hastig nach seinem Bogen, doch Ijsdur bat ihn, ruhig zu sein. „Ich sprach doch davon, dass uns letzte Nacht der schwarze Herold beobachtete. Die Drachenkultisten wissen wohl, dass wir hier sind und sie aufspüren können. Es ist auch in ihrem Interesse, diese Angelegenheit zu klären, ehe das Ganze zu einem Konflikt mit der gesamten Rietgarde führt. Folgen wir doch einfach dem Krark und sehen wir, wohin er uns führt.“

„Ja natürlich, folgen wir einfach dem feindlichen Flatterviech mit scharfen Federn und noch schärferen Klauen“ grummelte Iril, „Es wäre im Interesse der Kultisten, uns in eine Falle zu führen.“

„Dann geben wir halt Acht, dass wir in keine Falle tappen“, lächelte Ijsdur.

Aćh blickte ihn misstrauisch an, nickte dann aber. „Ein wenig Vertrauen kann einen weit bringen. Ich würde lieber nicht schon wieder kämpfen. Folgen wir dem Krark.“

Damit schien das letzte Wort gesprochen zu sein. Nachdem die vier ihr Lager zusammengepackt hatten, brachen sie auf. Barz‘ Steppenechse Sabri ließen sie zurück. Sie konnte in Konflikten kaum aushelfen und hätte sie mit ihrer Trägheit nur aufgehalten.

„Keine Sorge“, sagte Barz, „Sabri kann für sich selbst sorgen, wenn nicht gerade eine Horde Lumiwürmer auf sie aus ist. Und bislang hat sie immer wieder zu mir zurückgefunden, als hätte sie einen siebten Sinn dafür.“

„Es gibt doch mehr als sieben Sinne“, warf Iril reflexartig ein.

„Behauptet ja keiner, dass es der siebte der letzte Sinn sei“, grinste Barz zurück.

Der kleine Krark flatterte immer wieder weiter weg, sobald die vier Reisenden einige hundert Meter an ihn herankamen. Doch unmissverständlich wartete er immer wieder auf die Truppe, ehe er weiterflog. Es bestand kein Zweifel, dass er sie irgendwohin führen wollte.

Iril warf hin und wieder einen Blick auf ihre Runenscheiben, hatte aber nicht die Zeit, einen magischen Kompass in Gang zu setzen.

Das war auch nicht nötig.

Geführt vom jungen Krark, drangen die Helden tief in den südlichen Wald ein, marschierten an Bäumen vorbei und Trampelpfaden entlang, bis sie schlussendlich eine bewohnte Lichtung im südlichen Walde am Hange des Fahlen Gebirges erreichten.

Das Lager der Drachenkultisten.\bigskip







Verdiente diese Lichtung überhaupt die Bezeichnung „Lager“? Es handelte es sich vielmehr um eine Gruppe verwahrloster Zelte, die ein glimmendes Feuer umringten. Ein großer Kochtopf hing über den Flammen. Der leckere, einzigartige Duft nach Krallenflechten ging davon aus.

Die Zelte waren geschmückt mit verschiedenen Zeichen und Zeichnungen in dunkler Farbe auf hellem Stoff. Sie alle schienen Feuer, Drachen und ähnlich angenehme Sujets darzustellen, wobei in manchen der Illustrationen größeres Talent als in anderen zu erkennen war. Immer wieder erkannte Iril unter dem Sammelsurium an Symbolen ein bestimmtes, eine Kralle in einem Kreis.

Ein Logo der Drachen-Sippe?

Um Frühaufsteher handelte es sich bei den Kultisten allerdings nicht. Der kleine Krark, der unsere Helden hierhin geführt hatte, schlüpfte durch ein Loch in eines der größeren Zelte am anderen Ende der Lichtung und ward nicht mehr gesehen. Hier und da zeugten Bewegungen in den Zeltwänden und leises Gemurmel von der Anwesenheit von Lebewesen, doch hier draußen war kaum jemand zu sehen. Ein kleines Kind in einem schmuddeligen Kleid versuchte, einen Schluck der Suppe im großen Topf zu erlangen, während eine großgewachsene Frau es lachend zurücktrieb.

„Nein, Reanna, die ist noch nicht für dich! Zunächst dürfen die Beseelten speisen.“

„Immer die Extrawürste für die Beseelten“, knurrte die kleine Reanna mit knurrendem Magen. Dann zuckte sie die Schultern, breitete ihre Arme aus und rannte fröhlich fauchend im Kreis um das Lagerfeuer herum, während die alte Frau im Topf herumrührte.

Nach einem Hinterhalt sah dies kaum aus, dachte Iril erleichtert. Dennoch hielt sie ihren Hammer bereit, während sie nähertrat.

In diesem Augenblick öffnete sich eine Zeltplache nahe des herumrennenden Mädchens. Eine Gestalt trat heraus und räkelte sich gemütlich. Iril brauchte einige Augenblicke, um zu realisieren, dass die Gestalt ein Skral war, so ungewohnt war der Anblick eines müden Skrals ohne Plattenrüstung, auf dessen Kinn eine schaumige weiße Masse blubberte. Nichtsdestotrotz gellten bei diesem Anblick Warnsignale in ihrem Hinterkopf auf. Insbesondere, als der Skral ein langes Messer von seinem Lendenschurz löste und in die Höhe hob. Sein stachelbewehrter Schwanz peitschte über den Waldboden, während er das kleine Mädchen beäugte.

„Weg von ihm!“, schrie Iril dem Mädchen zu und rannte beschützend nach vorne. Ijsdur und Aćh, welche möglicherweise noch nie zuvor einen echten Skral gesehen hatten, guckten einander verdattert an. Verdattert guckte auch der Skral, als Iril zwischen ihn und das kleine Kind trat und drohend ihren Runenhammer hob. Verächtlich wischte er den weißen Schaum von seinem Kinn und ließ darunter einen Dreitagebart zum Vorschein kommen.

Iril erkannte zu spät am fleckigen Spiegel neben dem Zelt, dass der Skral wohl eher seine Gesichtsbehaarung hatte stutzen wollen, statt das kleine Mädchen abzustechen. Beschämt schlich sie zurück zu den anderen Helden, während der Skral sein Messer wegsteckte und argwöhnisch die Helden betrachtete.

Einen wütenden Blick in ihre Richtung werfend, beugte der bärtige Skral sich zum Kind herunter und sprach verblüffend sanft: „Geh, Rrreanna, rrrenn zu deinem Vaterrr. Hierrr wirrrds vielleicht gleichchch gefährrrlichchch.“

Irils Übersetzungsrunen leuchteten nicht auf. Das war die andorische Sprache. Ungewöhnlich für Skrale, sie zu sprechen. Andererseits hatte sich noch nicht mit sonderlich vielen Skralen konversiert.

Die kleine Reanna tat wie gebeten und verschwand in einer Jurte zu ihrer Linken. Kurz öffnete sich die Plache erneut. Ein Mann warf den vier Ankömmlingen einen finsteren Blick zu. Er trug einen langen, schwarzen Mantel und hatte eine Schnittwunde an der Wange, die leicht blutete. Iril war er nicht geheuer. Irgendwie umgab ihn eine dunkle Aura. Dann zog er sich auch schon wieder in die Jurte zurück.

Iril ließ ihre Hand betont ruhig auf dem Runenhammer von Golja ruhen, während der bärtige Skral auf sie zuschritt. Aćh und Barz hatten ihre jeweiligen Waffen zum Schnellziehen bereit. Ijsdur hielt seine eigenen Arme locker auf der Seite.

Der Skral betrachtete die einschüchternd dreinzusehen versuchenden Vier und legte seinen Kopf schief: „Sprrrechchcht schon, was haben wirrr fürrr ein Prrroblem hierrr?“

Iril berief sich auf die diplomatischen Künste, die ihr in Silberland eingetrichtert worden waren.

„Seid gegrüßt. Ich bin Iril von Silberland und werde begleitet von Ijsdur und Aćh aus dem fremden Tulgor sowie von Barz vom Stamme der Iquar. Seid Ihr die Drachenkultisten, die im vergangenen Mond mit dem Prinzen um Taroks Leichnam stritten?“

Der Skral ließ sich einige Zeit, ehe er gemütlich antwortete: „Grrra. Die sind wirrr. Die wenigen, die nochchch übrrrig sind. Viele zogen wiederrr nachchchhause. Habt ihrrr unserrren Zustand gesehen? Ihrrr müsst euch keine Sorrrgen machen, dass wirrr euchchch angrrreifen würrrden. Koh.“

„Wir fürchten uns nicht vor einem Angriff. Wir sind schon hier auf Geheiß des neuen Regenten Thorald. Weil dessen Drachenknochenfragmente gestohlen wurden.“

Der Skral schnaubte und langte nach einem Messer an seinem Gürtel. Hastig redete Iril weiter: „Wir glauben aber fest daran, dass es eine diplomatische Lösung geben kann. Könnten wir mit eurem Anführer sprechen?“

Der Skral ließ das Schwert in seiner schartigen Scheide stecken und murmelte: „Grrra. Ich hole Sagrrramak. Rrrührrrt euchchch nichchcht.“

Ohne weitere Worte schlurfte er davon.

Iril tausche unsichere Blicke mit Ijsdur aus. Aćh redete dem aufgeregt flackernden Turr beruhigende Worte zu. Nur Barz hatte sich aus irgendeinem Grund entspannt und sogar eine erleichterte Miene aufgesetzt.

Im Schatten zwischen zwei Zelten erkannte Iril zwei weiße Augen, die sie wachsam beobachteten. Sie kniff ihre eigenen Augen zusammen und versuchte, die Gestalt besser auszumachen. Sie glaubte, das flackernde Lagerfeuer auf einer metallischen Fläche rund um die Augen reflektiert zu sehen. Eine eiserne, gezackte Maske. Der Schwarze Herold. Stumm lugte er hinter einem Zelt hervor und beobachtete das Geschehen, ohne sich nur im Geringsten zu rühren.

„Das da ist der Schwarze Herold“, flüsterte Ijsdur Aćh hilfreich zu. Aćh nickte.

Weitere Zelte öffneten sich und interessierte Augen starrten die vier morgendlichen Störenfriede an. Auf die Lichtung traute sich jedoch so gut wie niemand. Sogar die alte Frau hatte sich vom Kochtopf zurückgezogen.

„Ich glaube, ich habe unseren Dieb von der Rietburg gefunden“, flüsterte Ijsdur Iril zu. Seine Augen hatten keine Pupillen, aber anhand seiner Kopfbewegung konnte Iril dennoch erkennen, wen er meinte, ohne dass er direkt auf den Verdächtigen zeigen musste.

Aus einer Jurte, an deren Spitze eine beige Fahne mit dem üblichen Symbol der Kultisten flatterte – der umkreisten Kralle – war ein Wesen getreten, welches Iril so noch nie von nahe gesehen hatte. Es schien, als hätte jemand einen vierfach gehörnten Ziegenkopf mit Fangzähnen auf einen kleinen Menschenkörper gesetzt. Weißes Fell bedeckte den Körper des Ziegenwesens. Milchige blaue Augen sondierten die Umgebung. In der Hand führte es eine kunstvolle Hellebarde.

„Einen Dieb?!“, schnarrte das Ziegenwesen, „Einen Dieb mit guten Ohren hat er gefunden! Und dieses ‚Diebes‘ spitze Hörner wird er bald kennenlernen, wenn er weiterhin so freche Anschuldigungen von sich gibt.“

„Verzeiht, o edler Tarus“, entschuldigte sich Ijsdur, „Ich dachte ...“

„Und dann schimpft er dich auch noch Tarus“, lachte das Ziegenwesen. „Ein blinder Träumer ist er! Oh, Fir, er erkennt dich nicht. Du bist doch ein Zwerg. Nichts weiter als ein vermaledeiter Zwerg aus Cavern.“

„Entschuldigt. Ich meinte bloß, Zwerge aus Cavern sähen ...“

„Und jetzt hat er noch die Unverschämtheit, zu meinen, er wüsste, was einen Zwerg ausmacht und was nicht!“, verwarf das Ziegenwesen, Fir, seine Hände, „Verirrst dich einmal zu tief in der Feenwelt, verlierst deinen alten Namen und deine Axt, und schon meint er, du wärst Teil dieser grasfressenden Stinker vom Norden. Was wird nur aus dieser Welt?“

Ijsdur schien beinahe amüsiert, als er sich ein drittes Mal entschuldigte. Fir schnaubte bloß und fuhr sich durch die Frisur.

Iril schaltete sich ein: „Es gibt ja durchaus dieses eine Märchen, laut welchem der erste Tarus von einer Fee ...“

„Und sie, naive Schildzwergin, die sie wohl ist, erzählt dir von Märchen!“, gluckste Fir. „Obwohl es Märchen gibt, die sagen, dass die Sonne jeden Tag im Meer versinkt, um sich abzukühlen. Wenn das die Quellen der Wahrheit der heutigen Jugend sind, so mögen uns die Drachen gnädig sein.“

Iril setzte eine freundliche Miene auf und überließ das diebische Ziegenwesen sich selbst. Es war nicht die Führungsperson dieses Kults.

Stampfende Schritte kündigten die Rückkehr des bärtigen Skrals an. Wobei nicht der Skral selbst laut stampfte – dessen nackten Füße lösten kaum ein Rascheln auf dem belaubten Boden aus – sondern seine Begleitung.

Sagramak die Schamanin. Iril hatte ihre geschwungene Rüstung schon aus der Ferne glitzern gesehen, als sie mit Thorald um Taroks Leichnam gestritten hatte, und ihr Gesicht von nahe, als sie durch ihren Krark einen Mord begangen hatte.

Von nahe sah man nun die vielen Scharten und Flecken, die die Rüstung aufwies. Der aus der Ferne so imposant aussehende Drachenhelm waren bloß zwei unförmige, entfernt an Flügel erinnerte Metallteile, die jemand auf einen Helm geschmiedet hatte.

Auffällig waren zwei Halsketten aus jeweils einigen wenigen unförmigen Knochenfragmenten, die an Sagramaks Hals baumelten. Die einen Knochensplitter waren grau und rissig, wohl Jahrhunderte alt, weitergegeben von Kultist zu Kultist. Die anderen strahlten hingegen weiß und waren zweifelsohne frisch. Taroks letzte Knochen.

Sagramak ließ ihren Blick über die vier schweifen und blieb insbesondere an Turr und am wie üblich von einem Wirbel aus Schneeflocken umgebenen Ijsdur hängen.

„Ich bin die Schamanin und Führerin dieser Diener der Drachen. Ich bin beseelt vom Drachen Sagrak. Es steht mir zu, seine letzten Überreste zu tragen“, sprach Sagramak stolz und hob die uralte Knochenkette hoch. „Ebenso wie die Knochenfragmente des Gewaltigen.“

„Ihr leugnet es also nicht?“, rief Iril, „Dies sind Taroks letzte Knochen! Ihr habt Unschuldige angegriffen, ja, gar umgebracht, um sie zu stehlen!“

„Das hätte nie so kommen sollen“, knurrte Sagramak, „Doch die Schuld dafür liegt nicht bei mir.“

„Bist du für diese Sippe verantwortlich oder nicht? Unabhängig davon, wie ihr sie erlangt habt, fordern wir die Knochen zurück.“

Sagramak blickte finster auf einen Dolch an ihrem Gürtel herunter und wie von Zauberhand geführt erhob sich der Dolch in die Luft. Drohend schwebte er auf Iril zu.

„Wollt ihr nicht lieber fliehen, solange ihr noch könnt?“, zischte sie finster. „Ihr habt keine Ahnung, mit wem ihr euch eingelassen habt und welche Mächte uns zur Verfügung stehen.“ Das Flackern des Lagerfeuers ließ ihre Züge drächisch und düster wirken. Ihre Zunge züngelte kurz zwischen trockenen Lippen hervor.

„Wollt ihr nicht lieber eine friedliche Lösung finden, ehe wir mit der gesamten Rietgarde angerückt kommen?“, warf Iril zurück.

„Hat die Rietgarde nichts Besseres zu tun? Dass sie die beiden Brücken besetzen, ist der einzige Grund, dass wir überhaupt noch auf dieser Seite der Narne kampieren. Ein einzelner Krark könnte die Knochen außerhalb unser beider Reichweite bringen, wenn ich es nur so wünsche. Wir halten die Astze in der Hand.“

Iril, die noch nie zuvor ein Astz-Kartenspiel gespielt hatte, verstand die Anspielung nicht. Barz hingegen blickte wissend.

Der telekinetisch geführte Dolch schwebte weiterhin langsam auf Iril zu. Iril, Ijsdur und Aćh starrten den fliegenden Dolch alle sorgenvoll an und traten einige Schritte zurück. Nicht so Barz. Dieser trat stolz nach vorne und proklamierte: „Dieser Trick ist durchaus einschüchternd. Bis man bemerkt, dass selbst die besten Telekinesen kaum mehr Kraft auf ein Objekt ausüben können als ein Kleinkind. Jemanden ernsthaft verletzen kann man damit kaum. Lass die Spielereien sein, Sagramak. Ich fürchtete schon, dass es hier zu einem unlösbaren Zwist kommen könnte. Doch nun, wo ich dich hier sehe, vergeht diese Furcht wie Schnee und Eis im Drachenfeuer.“

Sagramak stutzte und suchte nach Worten. Genauer: Nach einem Namen. „Öhm ... öhm ... Ja, dich habe ich schon mal gesehen. Dieser reisende Nomade der Iquar? Bart? Baz?“

„Barz! Mein Name ist Barz.“, rief Barz etwas weniger fröhlich.

Ungeachtet dessen fiel Sagramaks zuvor feindliche Haltung völlig von ihr ab. „Verzeih mir, als Schamanin trifft man so viele Leute. Ja, ja, jetzt erinnere ich mich an dich. Wo ist denn dein fescher Begleiter abgeblieben? Und deine große Echse? Was tust du denn hier?“

„Dasselbe könnte ich dich fragen.“

„Dann ist es wohl eine lange Geschichte“, lachte Sagramak, „Sag, haben wir dir nicht einige Krarks geschickt in den letzten Jahren? Warum kam keine Antwort? Wo verstecktest du dich?“

„Ich steckte in einem fremden Land fest. Das ist auch der Grund, warum es sich gerade so seltsam anfühlt, in unserer Sprache zu sprechen. Ich bin völlig außer Übung. Die Geschichte meiner weiten Reisen erzähle ich bei Gelegenheit gerne. Ein andermal. Ich befürchte, es gibt eine dringlichere Sache. Iril hier will mit dir über die Drachenknochen sprechen, die ihr gestohlen haben sollt?“

Barz wies auf Iril, welche sich angespannt räusperte. Wenn Barz die Schamanin kannte, und Aćh Barz kannte, und Ijsdur sich ohnehin nicht um solche Angelegenheiten kümmerte ... war sie auf einmal die Einzige hier, die sich noch darum kümmerte, dass die Drachenkultisten aufgehalten wurden?

Was wusste sie eigentlich über Barz? Konnte er ein Drachenkultist sein? Aus Osten kam er schon mal. Ehe ihre Befürchtungen sich festigen konnten, verdrehte Sagramak ihre Augen.

„Ich mag solche Verhandlungen überhaupt nicht. Sie sind dröge und führen oft nirgendwo hin. Aber dieser Feuervogel sieht aus wie so ein lieber Bursche, da bin ich doch fast gewillt, Euch noch länger zuzuhören ... sofern ich ihn streicheln darf.“

Barz richtete einige Worte an Aćh. Diese bedachte den kleinen Vogel, den eine kaum spürbare Hitzeaura umgab. Er war ein treuer Begleiter. Aber er war noch sehr schwach und brauchte Feuer, um wieder ins Leben zurückzufinden.

Dann fasste sich sie ein Herz und ließ den kleinen Turr zu Sagramak flattern. Er passierte das Lagerfeuer auf seinem Weg und schien einen Teil der Flammen in sich aufzunehmen. Als er auf Sagramaks Schultern landete, bröckelten Aschereste auf ihre gepanzerte Schulter. Sagramak streichelte Turrs Kinn. Die Flammen schienen ihrer nackten Haut nichts auszumachen.

Turr gurrte fröhlich.

Aćhs Hand bewegte sich langsam in Richtung ihres Schwerts, während sie Sagramak scharf im Auge behielt.

Doch in Sagramaks Augen blitzte statt Wut eine nur schlecht maskierte Trauer auf. Iril erinnerte sich zurück an den gewaltigen Krark, durch dessen Augen die Schamanin geguckt hatte, und den Chada vom Himmel geschossen hatte. Sie hatte noch nie Haustiere besessen und konnte sich nicht ganz in Sagramaks Gemüt einfühlen. Aber sie hatte schon einige Geliebte verloren und verspürte Mitleid mit der Kultistin.

„Wollen wir uns vielleicht an einem etwas privateren Ort beraten?“, meinte Iril. „Nicht, dass das ganze Lager involviert wird.“ Sie nickte zu den verschiedensten Augenpaaren, die aus den Jurten zu ihnen spienzelten.

„Na schön“, antwortete Sagramak, „Kommt in mein Zelt. Diskutieren wir.“











\newpage
\section{Kurze Verhandlungen}



Sagramak wies die Helden ins Zeltinnere. Von innen wurde deutlicher, dass der Stoff schon bessere Tage gesehen hatte. Zahlreiche Flicken übersäten die mit Symbolen bemalten Wände.

Der Blickfänger war ein uraltes, aus einem Baumstamm geschnitztes Drachentotem, das in der Zeltmitte thronte. Figürliche Darstellungen hatten offenbar zu den Talenten des längst vergessenen Schnitzers gehört. Stolz blickte die Drachenstatue auf die hereinkommenden Helden herunter. Sagramak neigte andächtig ihren Kopf.

Ehe sie die Zeltplache wieder schließen konnte, brüllte sie nach draußen: „Nehamal, Fir! Kommt her! Ihr seid mein Begleitschutz. Ich will nicht allein mit diesen Störenfrieden sein.“

Die beiden Angesprochenen folgten Sagramak ins Zelt und nahmen links und rechts der Drachenstatue Platz, ihre jeweiligen Waffen bereithaltend.

Nehamal war der Mann mit dem langen Mantel und der düsteren Aura, der mutmaßliche Vater der kleinen Reanna. Fir war das auffällige Ziegenwesen. Während es sich neben der Drachenstatue niederkniete, konnte es nicht unterlassen, einen wehleidigen Kommentar abzugeben: „O weh dir, Fir. Sie ruft an ihre Seite, wie man ein Hündchen bei Schoß ruft. Hat sie völlig vergessen, dass sie ohne dich noch immer auf Steinen kauen und dem Tode entgegenbibbern würde? Es gibt einfach keine Dankbarkeit mehr in ...“

„Klappe, Fir!“, fauchte Sagramak. Fir verstummte. Iril hätte schwören können, dass für einen Augenblick ein breites Grinsen auf dem Ziegenmund aufgeblitzt war.

„Wollt ihr etwas trinken?“, fragte Sagramak betont ungerührt. Sie gestikulierte in Richtung eines großen Fasses, welches am anderen Ende des Raums lag. Barz leckte sich die Lippen. Iril schüttelte ihren Kopf.

„Na dann“, sprach Sagramak langsam, „Sag, Iril, was hast du gegen uns? Dass du dir die Mühe machst, uns hinterherzutraben statt irgendwelchen dringlicheren Bedrohungen? In meiner Erfahrung lassen sich die meisten Handlungen auf zurückliegende Erfahrungen der Handelnden schließen, und die deinen lassen dich nicht positiv erscheinen.“

Entrüstet empörte sich Iril: „Soll das heißen, dass ich meine schlechten Eindrücke zu Drachen ungerechtfertigt an euch ausleben würde? Was sagen deine Handlungen über dich aus? Der Schwarze Herold ist auf eurer Seite, und er setzt sich bekanntlich nur für das Böse ein! Gibt euch das nicht zu denken?“

„Auf der ‚Seite des Bösen‘. Was für ein lächerlicher Gedanke, den die Andori euch da eingeredet haben“, lachte Sagramak, „Der Herold ist nur auf der Seite des Drachen. Wärt ihr alle zuerst auf unsere Gemeinschaft statt auf die Andori getroffen, hätten sie euren Geist nicht korrumpieren können und ihr würdet euch für uns einsetzen.“

„Die anderen drei vielleicht. Ich nicht“, sprach Iril nun, „Ich habe die Geschichten gehört und die Mahnmale gesehen. Die Drachen waren Bestien. Sie sind es nicht wert, abgebetet zu werden.“

„Ah, natürlich, eine Zwergin, welch unabhängige Beurteilerin der Drachenkriege“, verwarf Sagramak ihre Hände, „Ob die Drachen angebetet werden sollen, hängt nicht von ihren Taten ab. Sondern von ihrer Kraft, über uns zu urteilen.“

„Ach ja, stimmt, ihr glaubt, die Drachen würden nach dem Tode auf euch warten und eurer Nachleben bestimmen, oder? Sagt, wie viele Beweise besitzt ihr dafür? Hat auch nur ein verstorbener Geist sich je wieder bei euch gemeldet?“

„Wie viele Beweise besitzt du dafür, dass Mutter Natur oder welche Urmacht auch immer ihr anbetet, nach dem Tode auf euch wartet?“

„Wir lesen die Schriften der Anhänger ...“

„Und wir hören die Stimmen der Drachen! Meinst du, wir bilden uns das alles ein?!“, rief Sagramak. Sie schritt zu einem Tisch und zog ein Objekt daraus hervor. Einen kleinen Stein, glattgeschliffen, um den sich eine elegante metallene Drachenfigur wandte. Das Artefakt glühte rötlich und summte leise.

Eines von Schmiedemeister Hildorfs Drachenrelikten.

Sagramak hielt es Iril demonstrativ ans Ohr: „So höre doch! Lausche und vernehme die Stimme Sagraks, wie ich es tagtäglich tue.“

Iril hielt sich das Relikt widerstrebend ans Ohr. Sie vernahm nur das leise Rauschen des Bluts in ihren Ohren.

„Ich höre nichts.“

„Die Drachen bleiben doch meistens stumm in der Präsenz Ungläubiger“, mischte sich Barz nun beschwichtigend ein, „Die Stämme der Barbaren sind sich oft uneinig, was die Geschehnisse nach dem Tode angeht. Was die Götter für uns geplant haben. Ich erinnere daran, dass die Jpaxo die einzigen sind, die glauben, dass die Drachen ihr Nachleben bestimmen werden. Doch die Drachenseelen scheinen auch als einzige zu den Jpaxo zu sprechen. Lassen wir doch alle Diskussionen zum Nachleben sein und wenden uns der relevanten Frage zu: Was mit Taroks letzten Knochen geschehen soll.“

„Wir haben unseren Standpunkt klar gemacht“, sprach Sagramak bestimmt. „Die Kette der Knochenfragmente bleibt hier. Wenn ihr für uns ein gutes Wort beim Prinzen einlegt, damit er die Brücken in den Osten wieder freigibt, versprechen wir, nicht nachtragend zu sein und die Sache so beruhen zu lassen. Das ist mehr als gerecht.“

„Es wäre mehr als gerecht, wenn diese Knochen nicht so gefährlich wären“, bedachte Iril. „Vor nicht einmal einem Jahr verschaffte sich ein bösartiger Nekromant Zugriff zu den Drachenknochen im Grunde des Grauen Gebirges. Er bündelte ihre magische Macht, verwandelte die alten Wachtürme der Schildzwerge in leuchtende Fackeln der Magie und labte sich daran. Dies waren jahrhunderte alte, zerfallene Drachenknochen, und doch konnte der Nekromant sich derart stärken, dass er den Urtroll mit einem einzigen magischen Blitz zu verscheuchen vermochte. Den Urtroll! Der, der der Legende nach Mutter Natur mit einem einzigen Schlag in den Totenschlaf schlug.“

Sagramak verdrehte ihre Augen. „Drei Knochensplitter aus Taroks Fuß sind kaum vergleichbar mit der schieren Masse an Skeletten der im Unterirdischen Krieg gefallenen Drachen, die diesem Nekromanten zur Verfügung standen.“

„Taroks Knochen sind frischer und potenter. Drachenknochen sehr seltene und auserlesene Zutaten für mächtige Rituale. Und Tarok war der letzte, der mächtigste, der stärkste aller Drachen, als sei die Kraft aller Drachen der Vorzeit in ihn übergegangen. Sie dürfen nicht in die falschen Hände geraten.“

„Und was, wenn wir dir versicherten, dass es unter uns überhaupt keine Nekromanten gibt?“

Iril schüttelte ihren Kopf: „Da kann ich mich nicht auf dein Wort verlassen, insbesondere, weil du es nicht in die Zukunft geben kannst. Dies ist keine fantastische Märchenwelt, in der alle stets friedlich und fröhlich zusammenleben. Dies ist echt. Manche Leute wollen anderen schaden. Wir wollen verhindern, dass sie die Möglichkeiten haben.“

„Und bei deinem Trunkenbold von einem Prinzen wären die Knochen sicher?“

„Sicherer als hier.“

„Doppelmoralische Heuchlerin! Engstirniger Sturkopf!“

„Und du, Sagramak, bist eine ...“

„Nur keinen Ärger“, sprach Barz beschwörend, seine Hände beruhigend hebend, „Sind wir nicht alle Kinder der drei Brüder? Wir müssen uns nicht streiten. Wir können bestimmt eine Lösung finden. Iril, es soll euer Schaden nicht sein, wenn Sagramak die Knochen behält. Dafür könnte Volk der Schildzwerge, das unter Tarok leiden musste, eine ... gerechtfertigte Belohnung in Form von Gold und anderem Schmuck erhalten?“

Barz setzte sein breitestes Grinsen auf, absolut selbstsicher in seinem Glauben, die Lösung für dieses Problem gefunden zu haben. Sowohl die völlig verarmte Sagramak als auch Iril, welche weder eine Schildzwergin noch goldgierig war, schüttelten den Kopf. Iril flüsterte frustriert: „Ich werde einfach so tun, als hätte ich das nicht gehört. Ich bin vielleicht neugierig, oder wissbegierig, Barz. Nicht goldgierig! Nicht alle Zwerge sind gleich.“

Ijsdur nickte besserwisserisch hinter ihr.

„Genau. Neuerdings weiß ich, dass Menschen und Zwerge praktisch gleich sind.“

„Das meinte ich auch nicht.“

Barz lächelte entschuldigend und murmelte: „Naja, es lässt sich zumindest nicht leugnen, dass wir bislang keinerlei Probleme mit den Drachenkultisten hatten, trotz der alten Knochenketten in ihrem Besitz. Wie macht Tarok da einen Unterschied?“

„Mit den uralten versteinerten Knochen von Sagrak oder Nehal oder wie auch immer die alle heißen, könnte ein Magier kaum etwas anfangen. Falls die Knochen überhaupt echt sind. Doch Taroks Knochen sind frische, potente Mittel und gefährlich. Was, wenn sich plötzlich ein Skelettdrache aus dem Hort dieser Schamanen erhöbe? Was würdest du sagen, wenn plötzlich ein Dunkler Magier Taroks Knochen wiederbelebte, kaum hätten sich die Bewohner Andors vom Wüten des letzten Drachen erholt?“

Barz blickte plötzlich wieder besorgter drein. Er hatte noch nicht vergessen, was für eine Verwüstung Tarok im Land der drei Brüder angerichtet hatte.

Iril wandte sich wieder Sagramak zu: „Ich bitte Euch, lasst die Sache sein. Überlasst uns Taroks Knochen und wir besorgen euch Begleitschutz in die nun freie Knochengrube im Grauen Gebirge, wo ihr euch mit einer Vielfalt an Knochenresten für eure religiösen Gebräuche eindecken könnt.“

„Du weißt genau so gut wie ich, dass die Knochen aus der Knochengrube alt und größtenteils versteinert sind, und dass keiner davon von Tarok stammt. Die Knochengrube haben wir bereits zu Taroks Lebzeiten ausgeschöpft. Wir bedürfen seiner Überreste. Seiner Knochensplitter.“

„Und was tut ihr damit?“

„Nicht mein Spezialgebiet. Nehamal kennt sich besser damit aus. Ist schließlich auch vom erfindungsreichsten Drache beseelt. Nehamal?“

Der angesprochene Nehamal erhob sich von seinem knienden Platz neben der Drachenstatue und zog etwas unter seinem langen Umhang hervor. Eine Kette mit einigen uralten Knochensplittern, die um seinen Hals hing, an deren Spitze ein edles Drachenrelikt rötlich schimmerte.

Er schnarrte: „Ich bin Nehamal. Ich trage diesen Namen seit meiner Erleuchtung, und seit dann führe ich die Seele des Drachen Nehal in einem Blutstein-Relikt mit mir und seine Knochen um meinen Hals.“

„Nehal?“, lachte Iril auf, „Der legendäre Drache, der den Zwergen die Drachenfrucht schenkte? Jung, temperamentvoll und stark, und vor allen Dingen gewillt, etwas Neues zu erschaffen? Der Drache mit dessen Assistenz Kreatok seine vier mächtigen Schilde schmiedete?“

Nehamal rollte mit den Augen. „Nehal war ein bisschen mehr als der Assistent Kreatoks. Manch böse Zunge mag behaupten, Kreatok sei vielmehr der Assistent des Drachen gewesen, weil dieser nur genügend kleine Hände benötigte, um seine filigranen Ideen in Tat umzusetzen.“

„Zeige mir einen Drachen, der auch nur ein Schmiedestück schuf, ganz unabhängig der Größe, und ich zeige dir einen Lügner, der sich mit fremden Federn schmückt.“

„Jetzt kriegen wir uns alle mal wieder ein, in Ordnung?“, rief Barz, „Warum kamst du auf diesen Nehal zu sprechen, Iril?“

„Weil ich die alte Geschichte des letzten der vier Schilde kenne, auch man nur hinter vorgehaltener Hand davon munkelt. Nehal wurde in schwarz-silbrigen Flammen des Dunkelschilds verzehrt. Von ihm sind keine Knochen übrig. Was auch immer Ihr hier um den Hals tragt, von Nehal stammt es nicht.“

Nehamal lachte nur kopfschüttelnd. „Wie ihr die Geschehnisse verdrehen müsst, um unseren Glauben leugnen zu können, um uns als böse auszumachen, ist unter aller Würde. Bedenkt unsere Gaben. Übermenschliche Stärke, Schutz vor Feuer, Sagramak ist gar Telekinesin. Die kommen nicht von nirgendwo.“

„Ich leugne nicht eure Gaben, oder deinen Glauben, und ich halte dich nicht für böse. Doch für getäuscht. Ich bezweifle, dass die toten Drachen noch Gedanken oder Gaben von sich geben. Bedenke, dass Tarok Euren eigenen Worten zufolge der letzte Drache war. Wie kann es sein, dass diese Drachenseelen noch mit euch in Kontakt sind, wenn es gar keine lebenden Drachen mehr gibt, um den Drachenseelenhort Krahal am Leben zu erhalten?“

„Die Seelen unserer Beseelten ruhen nicht mehr in Krahal, sondern hier, in unseren Relikten. Und Krahal erhält sich selbst! Es braucht keine lebenden Drachen, um stabil zu bleiben. Im Gegenteil, die Drachen nutzen die Kraft Krahals, um sich selbst zu stärken! Wie Sagramak mit den Naturgeistern spricht, tu ich es mit den Geistern der Drachen, und sie sind seit Taroks Fall nicht leiser geworden.“

Nehamal schloss seine Augen und lehnte seinen Kopf zur Seite, als lausche er andächtig.

Iril nutzte die Gelegenheit, um zu fragen: „Was wollt ihr denn nun mit Taroks Knochen tun?“

„Das geht euch nichts an.“

„Tut es wohl, wenn wir darüber urteilen sollen, wer diese Knochen behalten soll.“

„Wir wollen nicht, dass ihr über uns urteilt.“

„Das macht unser Urteil nicht besser.“

Nehamal stockte einen Augenblick. Dann sagte er: „Wir werden die Knochen ehren und wahren. Ihnen ein würdigeres Begräbnis verschaffen, als sie die Narne herunterzuspülen. In Krahd mag man Verstorbene in einen Lavafluss werfen, um sie von einem Dasein als Skelettkrieger zu bewahren, doch in zivilisierteren Kreisen gehört sich das nicht. In den Bergen bestatten wir unsere Toten in der Erde, auf dass aus ihnen die Erde genährt werden kann. Und dies wollen wir auch Tarok tun. So wenig von Tarok, wie wir kriegen können.“

Iril überhörte die Spitze in seiner Aussage geflissentlich.

Überraschenderweise meldete sich Ijsdurs kalte Stimme zu Wort: „Die Übersetzungsrunen in meiner Brust können nicht nur die Bedeutung gesprochener Worte übertragen. Sie geben mir manchmal auch ein Gefühl der Intention des Sprechers dahinter. Oft nur klein, fein, beinahe vergesslich. Aber Lügen kann ich mit diesen Runen von Meilen her riechen. Und ich rieche Lügen in deinen Worten. Was wollt ihr wirklich mit den Knochenfragmenten?“

Iril blickte Ijsdur überrascht an.

Nehamal starrte ihn ebenso überrascht an. Dann knurrte er: „Na gut. Seien wir offen, auch wenn es euch nicht gefallen wird. Wir werden die Knochensplitter nutzen, um Taroks Seele an ein Relikt zu binden. Wir werden einen Träger ernennen, jemanden, der vielleicht den Namen Taromak annimmt. Was danach geschieht, hängt von Taroks Willen ab. Er wird uns mit seinen Gaben leiten und wir werden seinem Willen folgen, wie wir alle dem Willen der Götter folgen sollten.“

„Und was, wenn Tarok euch den Angriff auf unschuldige Andori befiehlt? Was, wenn er euch einen Nekromanten aufsuchen lässt, um mithilfe dessen Macht den Tod auf Erden zu entfesseln?“

„Ängstigt dich das? Glaubt ihr nicht daran, dass Taroks Geist nun im ewigen Glück weilt? Jemand Glückliches wird keine bösartigen Gedanken mehr hegen.“

„Das glauben die Andori, nicht ich. Wird dieser Taromak mit außergewöhnlichen Kräften gesegnet sein?“

Nehamal hauchte andächtig aus: „Der größte und stärkste von uns allen.“

Iril nickte: „Damit steht meine Entscheidung fest. Rückt die Drachenknochen heraus! Im Namen des Königs von Andor!“

Sagramak mischte sich wieder ein. „Andor hat aktuell keinen König. Und selbst wenn du für ihn sprächest, wissen wir beide, dass dies nicht gerecht ist. Ein feiner Schnösel in seiner hohen Festung soll nicht bestimmen können, was wir, nicht einmal seine Untergebenen, mit Artefakten anfangen dürfen.“

„Ich bin nicht seinetwegen hier. Rückt die Drachenknochen heraus. Meinetwegen im Namen von Iril von Silberhall.“

„Und im Namen von Barz vom Stamme der Iquar“, ergänzte Barz mit entschuldigender Miene..

Sagramak zog die Kette mit Taroks Knochenfragmenten ab und hielt sie grimmig vor sich. Dann hielt sie inne. Ein Grinsen schlich sich auf Sagramaks Gesicht, als wäre ihr eben erst etwas eingefallen.

„Barz, was tust du so, als stündest du hinter dieser Zwergin? Als wärst du nicht auch ein Anhänger der Drachen? Warst es nicht du, der unsere Sippe über Krarks – erbarmungslos vom Himmel geschleuderte Krarks, übrigens – um Sternkraut anbettelte? Ich weiß, was für Rituale man mit Sternkraut vollziehen kann. Wir sahen alle, wie Tarok der Gewaltige sich nur Tage später von seinem Schlafplatz erhob und die Steppe in Brand setzte. Du bist einer von uns, Barz. Stehe zu den deinen.“

Barz blickte auf einmal ganz unwohl in seiner Haut drein. Sagramak grinste.

„Ist das wahr?“, fragte Iril anschuldigend, „Habt ihr Tarok geweckt? Du sagtest uns, Tarok habe Sabris Mutter ermordet!“

„Tat er auch“, versicherte Barz, „Wir wollten ihn nicht rufen. Es war ein Versehen!“

„Leugne es nicht, Barz“, knurrte Sagramak.

„Was bringt es dir, mich von etwas überzeugen zu wollen, an das ich nicht glaube?!“

„Zwiespalt“, sprach Ijsdurs klirrend klare Stimme. „Sie sät Zwiespalt, weil ihr nichts anderes übrigbleibt.“

Barz ächzte: „Komm, Sagramak, so eklig musst du nicht tun.“

„Deine Echse sollte sich glücklich schätzen, im Feuerstrahl des Gewaltigen umgekommen zu sein“, spuckte Sagramak aus, „Ihre Seele ruht nun sicher in der Sphäre der Drachen. Ob du dorthin kommst, steht noch in Frage. Schließe dich uns an und verteidige unser Recht, Barz, oder die Drachen werden dich bis in alle Ewigen im Drachenfeuer schmoren lassen – dich, und deine Familie, und deine Echse, und ...“

„Advaria meza“, hauchte Barz. Hätten seine Augen vor Wut Funken sprühen können, hätten sie es in diesem Augenblick getan.

Aćh, die sich bislang mehr auf die Wandzeichnungen als auf die ihr unverständlichen Gespräche konzentriert hatte, fuhr herum, als sie diese Worte vernahm.

Der auf Sagramaks Schultern sitzende junge Turr legte seinen Kopf schief, als wäre er nachdenklich. Dann klatschte das kleine Vögelein seine Flügel zusammen und schleuderte einen faustgroßen, glühenden Ball aus Feuer in Sagramaks Wange. Diese brüllte auf und klappte zur Seite.

Ijsdur wich vor der urplötzlich aufgeflammten Hitze zurück und stolperte zu Boden. Barz half ihm auf.

Iril stürzte nach vorne und schnappte sich die Kette mit Taroks Knochenfragmenten aus Sagramaks Hand.

Sagramak schoss wieder auf. Ihre Wange wirkte angeschlagen, doch nicht verbrannt.

„Verräterischer Vogel“, zischte sie, „Vielleicht behalte ich dich, nachdem wir deine Besitzerin erledigt haben.“

Turr gurrte bösartig. Fast schien es, als sei er ein wenig größer und kräftiger als vor einer Minute.

Aćh zog ihr Mondschwert und richtete es auf Sagramak, welche ihre waffenlosen Hände hob und einige Schritte zurückwich.

Nehamal zückte einen unter seinem Mantel verborgenen Degen und kreuzte die Klingen mit Aćh.

Fir stand ächzend auf und lamentierte Schmerzen in seinen Knien. Überraschend geschmeidig schlug das Ziegenwesen mit seiner Hellebarde nach Iril, doch diese duckte sich darunter hinweg.

Es klapperte, das Aćh Nehamals Degen zu Boden dirigierte. Aćh kickte die Waffe weg und nahm die Beine in die Hand. Er rief ihr irgendetwas hinterher darüber, wie ihre Zeit abgelaufen sei und er ihrem lächerlichen Dasein noch ein Ende bereiten würde.

Ijsdur riss ein Loch in die Zeltwand. Die vier stürzten ins Freie, nur weg von hier, solange die restlichen Kultisten noch nicht mitgekriegt hatten, was hier abging.

Iril hielt Taroks Knochenkette fest umklammert.

Der Wald lag ruhig da.

Zu ruhig.

Augen blitzten im Gehölz auf.

Der Schwarze Herold stellte sich ihnen in den Weg. Drohend hob er sein Schwert. Sein Umhang flatterte wild. Iril wurde nicht langsamer, sondern schleuderte ihren Hammer. Der Herold wurde mit einem eisernen Klang zu Boden geschlagen. Im Vorbeigehen tippte Iril auf ein Runentattoo auf ihrer Schulter und streckte ihre Hand aus. Der Hammer glühte auf ebenso wie die Runen auf Irils Arm, machte eine Kurve in der Luft und flog in ihre ausgestreckte Hand zurück.

„Schöner Trick!“, komplimentierte Ijsdur.

Die vier rannten einen Waldhang hinauf, auf in Richtung Norden, tiefer in den südlichen Wald hinein.

Im Rennen stampfte Ijsdur immer stärker auf. Die Eisspur, die er hinter sich herzog, wurde breiter und dicker. Dann zuckte ein Blitz vom Himmel herab und traf den Boden vor ihnen. Eine weite Eisschicht breitete sich in Windeseile über den Hang aus.

Faszinierenderweise war das Eis rutschig, doch rutschten Iril und ihre Begleiter nicht den Berghang hinab, sondern hinauf. Iril musste Acht geben, nicht zu stolpern.

Im Zurückblicken erkannte sie, wie die Drachenkultisten unter ihr die Eisfläche zu betreten versuchten. Für sie schien das glatte Eis sich normal zu verhalten. Sie kamen nicht weit und rutschten danach aus.

Aus der Ferne krächzte Fir: „O weh, welch finstere Tricks der Natur sich nun gegen dich wenden. Nicht einmal der Grund vermag uns zu halten. Sieh, da rutscht die edle Sagramak auch schon an dir vorbei. Wäre sie vorhin netter zu dir gewesen, hättest du sie vielleicht vor dem Fallen bewahrt.“

Seine Stimme wurde leiser und verklang, als Ijsdur und seine drei Begleiter weiter den Hang hinaufschlidderten.

Barz lächelte: „Schön, dass wir offenbar schon zu deinen Freunden zählen.“

„Ich mache das nicht“, antwortete Ijsdur, „Nicht absichtlich. Faszinierend.“

„Wir sind sicher“, schnaufte Iril, „Die Drachenkultisten sind keine großen Kämpfer. Sie wagten nicht den offenen Kampf gegen Thoralds Rietgarde, sondern wählten den heimlichen Weg. Und wir haben die Knochen. Alles ist gut.“\bigskip







Am Horizont ragten die Türme der Rietburg stolz in den Himmel hinauf.

„Nicht mehr lange, dann könnt auch ihr von der Gastfreundschaft der Andori profitieren“, versprach Iril den beiden neu Hinzugestoßenen.

Aćh gähnte. „Ich vermisse ein gutes Bett.“

„Der Prinz wird dir bestimmt mit Freuden eines überlassen, wenn wir ihm die Knochensplitter zurückbringen.“

„Sollten wir sie nicht vernichten?“, warf Barz nun ein. Er zückte ein kleines gelbes Pulversäcklein. „Nichts einfacher als das. Ich kann die Knochen im Nu verschwinden lassen. Oder zumindest verwandeln“

„Lieber nicht“, sprach Iril, „Denkt an alles Gute, was mit diesen Drachenknochen angestellt werden kann.“

Da mischte sich nun auch Ijsdur ein: „Während sie in der Schatzkammer eines Prinzen verrotten? Wenn es dem um das Gute gegangen wäre, das man damit anstellen kann, hätte er kaum den gesamten Rest des Drachen die Narne hinunterspülten lassen.“

„Ich kann mir nicht sicher sein, dass Thoralds Ziele hehrer sind als die der Drachenkultisten“, sprach Iril, „Nur, dass er bestimmt nichts mit solchen Knochen anzufangen weiß. Und seine Berater sind hoffentlich feine Gesellen, die sich entweder gar nicht mit finsterer Knochenmagie auseinandersetzten oder aber die Weisheit besitzen, davon abzulassen. Und dann, eines Tages, wenn die Knochen benötigt werden, werden wir sie aus der Schatzkammer holen und verwenden.“

„Wofür könnten wir solche Knochen je nutzen wollen?“

„Ich habe in den Tagen seit Taroks Tod von der Hohen Gelehrten der Rietburg so einiges zu ihnen erfahren. Auch ich trauere nun der die Narne heruntergespülten Magie hinterher. Solche Knochen können als Komponenten gewisser uralter Rituale verwendet werden, welche selbst gewaltige Flüche zu brechen vermögen.“

„Denkst du etwa an Narkon?“, meinte Ijsdur mit hochgezogener Augenbraue.

„Nein“, sprach Iril bestimmt, „Wir mögen uns nicht ganz sicher sein, welches Übel Seekönig Varatan einst auf Narkon bannte, doch halte ich es definitiv für unklug, Varatans Fluch leichtfertig zu brechen. Doch wer weiß, welche finsteren Gestalten in der Zukunft Flüche auf Unschuldige schleudern wollen könnten? Und dann können wir die Drachenknochen aus der Schatzkammer des Königs zurückholen und diese Flüche brechen.“\bigskip







Am Tor der Rietburg – neben einem fröhlich orange vor sich hin flackernden Ewigen Feuer – wurden die vier vom jungen Peta empfangen, der sie über den neusten Klatsch und Tratsch informierte. Auf dem Burghof herrschte geschäftiges Treiben. Mägde liefen mit großen Körben unter dem Arm an ihm vorbei und einige Kinder rannten hinter einem zotteligen Hund her.

Prinz Thorald eilte hastig durch das Rietdorf. Obwohl die Sonne schon hoch am Himmel stand, war er noch in ein Nachtgewand gekleidet. Er trug die Rietgraskrone nicht, hatte sich aber zumindest einen königlichen Umhang übergeworfen.

„Ihr habt die Drachenkultisten überwunden? Ihnen die ungerechterweise angeeigneten ...“

Thorald verstummte, als er Turr erblickte, der stolz auf Aćhs Schulter saß und die Drachenknochenkette in seinen Krallen hielt.

„Welch außergewöhnlich wunderschönes Wesen“, flüsterte Thorald. Er streckte eine Hand nach dem Takuri aus, doch das Vogelwesen sträubte seine orangegoldenen Federn und zischte den Prinzen bedrohlich an. Thorald wich erschrocken einen Schritt zurück.

„Was für eine Bestie!“, wetterte er. „Hochgefährlich! Sag ihm, dass er das lassen soll!“

Aćh schüttelte den Kopf. „Ein Feuertakuri ist es komplexes Wesen mit einem schwer zu bändigenden Gemüt. Turr lässt sich nicht so leicht vorschreiben, wen er mögen soll und wen nicht.“

Ehe der Prinz rot anlaufen konnte, überreichte Iril dem Prinzen mit einer angedeuteten Verbeugung die errungene Knochenfragment-Kette.

Thorald kniete sich zu ihr nieder und nahm sie an sich.

Mit Blick auf Ijsdur fügte er an: „Ich habe mich getäuscht in euch, Eis-Dämon. Ihr alle habt eine große Gefahr abgewendet. Morgen sollen wir die Feiern nachholen, die von dem dreisten Diebstahl unterbrochen wurden. Morgen wollen wir die neu ernannten Helden von Andor feiern!“

Er kratzte sich am Bart. Dann, als wäre ihm auf einmal eine Eingebung gekommen, sprach Thorald: „Und ihr tapferen Vier sollt zu ihnen gehören!“

Er nickte einem Fanfarenspieler zu, der in sein Instrument blies und in Windeseile eine Menge Rietburgbewohner zusammentrommelte.

Für die Zeremonie ließ Thorald sich dann doch noch einen eleganteren Umhang bringen und legte einen Gurt mit einem zeremoniellen Schwert aus der Schmiede des alten Wulfron bringen.

„Kniet nieder“, sprach der Regent.

Die Helden taten wie gebeten.

„Ohne Furcht und ohne Zögern habt ihr die letzten Knochensplitter Taroks zurückgebracht und damit einen Schlussstrich unter das dunkelste Kapitel unserer Geschichte gesetzt. Von heute an und für immer seid ihr Helden von Andor, im Kampf für ein Leben in Freiheit und gleiche Rechte aller, die in diesem Land leben.“

Er schwenkte sein Schwert über den Köpfen der Knienden. Das hatte er irgendwie auch schon motivierter gekonnt. Und bei Ijsdur hielt er tatsächlich größtmöglichen Abstand, während er die zeremonielle Schwertbewegung durchführte. Nichtsdestotrotz war es danach vorbei.

„Das wär’s. Ihr seid nun Helden von Andor. Willkommen im Team. Meldet euch bei ... hmmm ... ich glaube, Eara kümmert sich um die Organisation der Aufgaben? Ah, ihr werdet es schon herausfinden. Viel Glück. Das Königreich braucht mehr wie euch in den Wilden Jahren, die da kommen.“

Schon stolzierte der Prinz wieder davon, etwas davon murmelnd, wo zum Himmel er nun ein sichereres Versteck für die Drachenknochen finden konnte.

Iril schluckte schwer. Das war eine äußerst überrumpelnde Erfahrung gewesen. Keine Helden von Andor waren je so rasch ernannt worden. Vermutlich war der Prinz noch immer etwas neben der Spur.

Grundsätzlich hatte der Titel keine genau definierte Bedeutung. Und dennoch ... Irils Magen verkrampfte sich, als sie sich vorstellte, was nun auf einmal für Verantwortungen auf ihr lasten konnten. Dem Prinzen schien es selbstverständlich zu sein, dass die vier neuernannten Helden sich nun bei Eara melden würden und ihre Hilfe für das Königreich zur Verfügung stellten. Und Iril war auch gerne bereit dazu, hatte sie das ja gerade schon die letzten Tage getan. Doch für wie lange?

Nun, sie konnte sich in nächster Zeit Gedanken darüber machen. Als Helden von Andor würden sie von der Krone mit Nahrung und Logis unterstützt. Sie müsste sich keine Gedanken mehr und ihre schwindenden Goldreserven machen.

Still standen die vier neuernannten Helden im Kreis und starrten dem abziehenden Thorald hinterher. Inzwischen hatte sich ein kleiner Trubel aus Burgbewohnern zusammengefunden, welche sie neugierig begutachteten und flüsterten. Neue Helden, schon wieder?





Iril flüsterte zu den drei anderen: „Und, wollen wir das tun? Helden von Andor sein, zumindest für die nächste Zeit?“

„Aćh und ich sind ein gutes Team“, meinte Barz.

„Ijsdur und ich stellen uns nicht schlecht an“, meinte Iril.

„Wir alle konnten den Lumiwürmern ziemlich einheizen, ohne irgendeine Koordination zu haben“, bemerkte Ijsdur, „Die Frage ist weniger, ob wir ein gutes Team wären. Sondern ob wir eines sein wollen.“

„Ich wäre gerne dabei, es auszuprobieren“, meinte Aćh. „Ich hatte mir unter meinem Besuch hier ohnehin vorgestellt, auszuhelfen und diplomatische Beziehungen aufzubauen. Welch bessere Gelegenheit?“

„Das klingt durchaus gut“, murmelte Barz, „Da ist nur die Sache mit ...“

„BARZ!“, ertönte ein lauter Ruf aus der Menge, „Barz! Habe ich dich endlich gefunden! Wie in aller Welt kommst du hierher?!“

Barz‘ Gesicht hellte sich auf. „Nabib? Bist du es?“, entfuhr es ihm. „Wenn man an die Götter denkt ...“ Ohne weitere Erklärung drehte er sich um und raste in die Menge, zur Quelle der Stimme.

„Diesen Namen kenne ich. Das ist ein Freund von ihm“, erklärte Aćh, „Auf der Suche nach ihm wollte Barz damals nach Andor reisen. Wie es scheint, haben sie sich nun endlich wiedergefunden.“

Die drei folgten Barz in die Menge. Sie fanden den Steppennomaden eng umschlungen in einer Umarmung mit einem hochgewachsenen Krieger, in der die beiden einander leise Dinge zuflüsterten. Einen Augenblick trennten sie sich voneinander und blickten einander ins Gesicht, nur um gleich wieder aufeinanderzustürzen und in einen langen Kuss zu versinken.

„Das sieht doch nach ein bisschen mehr als Freundschaft aus“, grinste Iril.

Aćh blickte überrascht drein und murmelte dann: „Sprache ist kompliziert. Vielleicht kennen die Barbaren nur ein Wort für ... ich meine, der Unterschied zwischen einer tiefen Freundschaft und einer Romanze ist ohnehin minim ...“

„Es wäre angebracht, sie nicht mehr länger anzustarren“, sagte Ijsdur, so laut, dass es auch Barz und Nabib hören mussten.

Die beiden Steppennomaden trennten sich aus ihrer Umarmung und blickten zurück zu den drei Neuankömmlingen. Tränen glitzerten in ihrer beider Augen. Barz schluckte und murmelte dann:

„Natürlich, natürlich. Vorstellungen. Nabib, das hier ist Aćh, die beste Astzspielerin, die ich je getroffen habe. Ich verdanke ihr mein Leben. Und die beiden anderen Helden sind Iril und Ijsdur, die wir gerade erst getroffen haben. Noch könnten sie theoretisch bessere Astzspieler sein.“

„Ijs hat dich schon vor drei Jahren einmal angetroffen“, meinte Ijsdur, „Als Kapitän eines Hängeschiffs.“

Barz machte große Augen und nickte. „Schon wieder vergessen. Es ist herausfordernd, dich gedanklich mit deinem früheren Selbst zu verbinden.“

Aćh nutzte die Gelegenheit, sich an Nabib zu wenden und zu sagen: „Nabib, das hier ist Barz, der beste Astzspieler, den ich in meinem Leben je getroffen habe. Ich verdanke ihm mein Leben. Ich habe tatsächlich schon von dir gehört.“

Barz übersetzte wichtigtuerisch.

Iril nickte Nabib zu. Sie erkannte gleichzeitig seinen Namen und sein Antlitz. Sie hatte ihn bereits im Lazarett an der Rietburg getroffen, wo er aktuell unter Heiler Readem arbeitete.

Barz murmelte: „Das Schicksal war es, das mich auf der Suche nach dir hat von Weg abkommen lassen, Nabib. Und nun erreiche ich nicht einmal mein Ziel, da du es bist, der mich gefunden hat.“

„Da würde ich dem Schicksal nicht böse sein“, lächelte Nabib.

Auf einmal schien Barz fahrig. Er nestelte an seinem Gürtel herum, während er Nabibs Blick auswich und leise flüsterte: „Ich habe unsere Ringkette hoch oben im Gebirge verloren.“

Ebenso betroffen murmelte Nabib: „Und ich habe das reich verzierte Amulett deiner Großmutter verkauft.“

„Weißt du, wie egal mir das gerade ist?“

Während sie einander erneut in die Arme fielen, flüsterte Ijsdur: „Ist es möglich?“

Ijsdur hob seine schneeige Hand ... und steckte sie einfach in seinen schneeigen Bauch hinein! Als er sie wieder hervorzog, war sie zu einer Faust geformt. Er präsentierte Barz ihren Inhalt: Eine elegante Kette. Zwei Ringe hingen daran: Ein geschnitzter aus dunklem Holz und ein gehauener aus hellem Stein.

„Diese Kette habe ich am Felsentor zum ewigen Eis gefunden, als ich es verließ. Gehört sie zufälligerweise ...“

„Ja!“, hauchte Barz. „Bei den Göttern, wie kann dies sein?“ Misstrauisch blickte er die Kette an. Das letzte Mal, als er eine Kette von einem Eis-Dämon angenommen hatte, war es nicht gut gekommen. Dann jedoch fasste er sich ein Herz, schnappte sich die Kette und dankte Ijsdur von ganzem Herzen.

Während er sich wieder Nabib zuwandte, fragte Iril Ijsdur: „Wie viele andere Dinge bewahrst du so in deinem Körper auf?“,

„Nicht viele. So viel findet man da oben im Gebirge nicht.“

„Schon an einen Gürtel gedacht?“

„Ist doch viel unauffälliger, es im Körper mitzuschleppen.“

Iril musste zustimmen.

Da schob sich eine große Echse durch die Menschenmenge. Schnaufend blieb Sabri vor Barz stehen.

Nabib blickte sie verwirrt an.

„Sie ist ... kleiner geworden?“

„Sie ist gestorben“, meinte Barz trocken, „Das ist Sabri, ihre einzige Tochter.“

„Oh, Barz, das tut mir leid.“

„Muss es nicht, es ist schon länger her.“

„Wer ist das?“, fragte Aćh.

Sie blickte zur Schmiede, hinter welcher eine grau gekleidete Gestalt hervorgetreten war und mit energischen Schritten näherkam. Sie trug einen langen Umhang, der bis beinahe an den Boden reichte, und einen dicken braunen Rucksack auf ihren Rücken.

Iril hatte sie schon früher einmal gesehen, wie sie nach Taroks Tod und bei Brandurs Totenfest umhergewuselt war und mit wichtig aussehenden Leuten gesprochen hatte. Nun trat sie näher und räusperte sich wichtigtuerisch:

„Iril von Silberhall?“

„Die bin ich! Und wer seid Ihr?“

„Mein Name ist Sanja. Ich bin meines Zeichens Bewahrerin der Lieder, Sagen und Schriften in den heiligen Archiven des Baums der Lieder.“

Iril lächelte: „Ah, ihr befragt die Anwesenden nach den wichtigen Geschehnissen um Taroks Tode und notiert sie für die Nachwelt? Und nun würdet ihr gerne uns nach unseren Erlebnissen ausfragen?“

Sanja die Bewahrerin nickte wichtigtuerisch: „Eine gewaltige Aufgabe. Wir fühlen uns sehr geehrt, dass der Oberste Priester Melkart ausgerechnet uns beide wählte, um für diesen Zweck den Wachsamen Wald zu verlassen. Bekanntlich tun das nur sehr wichtige Bewahrer zu wichtigen Zeiten.“

„‚Wir‘?“, fragte Iril. Sie brauchte nicht lange, um den Grund zu erkennen: Hinter der Ecke der Schmiede lugte eine weitere in einen langen grauen Umhang gewandter Gestalt nervös hervor.

„Wird dein Begleiter ... oder Begleiterin? ... will deine Begleitung noch zu uns stoßen?“

„Ich glaube, heute fühlt sie sich eher nach einer Sie an. Und ja, sie wird noch zu uns stoßen. Sie muss schließlich alle wichtigen Berichte notieren. Ich stelle nur die Fragen. Meine Handschrift die reinste Krakelei und noch dazu bin ich langsamer. Sie ist nur etwas schüchtern vor Fremden. Jorna, getrau dich doch mal her!“

Jorna, die Bewahrerin, glitt nervös auf die Gruppe zu. Sanja deutete im Nebenbei auf einen grünen Wimpel, der auf Brusthöhe an Jornas grauem Bewahrergewand befestigt war. Durch ihre Übersetzungsrunen konnte Iril den Zeichen darauf ihre Bedeutung zuordnen. „Sieh da. Eine Sie.“

Sanja fuhr fort: „Nun denn, dann könnten wir uns der Befragung widmen, sofern das für Sie in Ordnung ist. Was hattet Ihr soeben mit Prinz Thorald zu tun?“

Während Jorna hastig eine Schreibfeder und eine mit einem Pergament bespannte Steintafel aus ihrer Reisetasche zog, fragte Iril verwirrt: „Wart Ihr nicht anwesend? Habt Ihr das nicht mitgekriegt?“

„Wir sind die Bewahrer vom Baum der Lieder. Wir bewahren die Berichte anderer. Wir versuchen, so wenig wie möglich einzugreifen, und wir stellen möglichst offene Fragen“, gab Sanja zurück.

„Oh, na dann“, murmelte Iril, „Nun, wenn ich es richtig verstanden habe, wurden wir vier soeben zu Helden von Andor benannt für das Zurückbringen der von Drachenkultisten entwendeten Flügelknochenfragmenten des Drachen Tarok.“

Jornas Schreibfeder glitt in Windeseile über die Schriftrolle. Iril vermutete, dass sie irgendein Zeichenkürzelsystem nutzte. Niemand, den sie kannte, konnte beim Niederschreiben mit einer rasch sprechenden Person mithalten.

„Fußknochen“, widersprach Sanja, „Es waren Fußknochen. Flügel hätten eine gelblichere Färbung.“

Iril wusste nicht, was sie darauf antworten sollte. Da stellte Sanja schon die nächste Frage: „Was bedeutet es, ein Held von Andor zu sein?“

Ein bisschen hilflos guckte sich Iril zu den übrigen Helden um. Barz unterhielten sich weiter mit Nabib, Ijsdur starrte planlos in den Himmel und Aćh hätte die Frage nicht einmal verstanden.

„Das weiß ich nicht so wirklich. Ein guter Ruf im Königreich, nehme ich an. Vielleicht auch andere Vorteile? Eine gewisse Verpflichtung darüber, diesem Ruf und Titel treu zu bleiben. Doch bin ich mir nicht sicher, ob wir ihn überhaupt verdient haben. Das geschah alles etwas kurzfristig.“

„Und ihr habt etwa eine eigene Echse?!“, rief Sanja. Iril folgte ihrem ausgestreckten Finger bis zur Steppenechse Sabri, welche soeben von Barz und Nabib hinter den Ohren gekrault wurde und ein wohliges Gurgeln von sich gab.

„Das ist Sabri“, erklärte Iril. „Sie gehört zu Barz. Dem Steppennomaden im langen Mantel mit den vielen Säckchen daran. Er verfügt über uraltes Wissen zu magischen Pulvern. Dieses versetzt ihn in die Lage den Kampf mit anderen Augen zu sehen. Darüber hinaus hält er so manche Überraschung für uns alle bereit. Der mysteriöse Barz versteht es ...“

„Schreibfehler, Jorna. Er heißt Barz, nicht Braz“, zischte Sanja zu Jorna, woraufhin letztere hastig ihre Notizen korrigierte.

Iril fuhr indes fort. „Barz versteht es, verschiedenste Pulver herzustellen, was ihn zu einem der vielseitigsten Helden Andors macht. Dabei gilt es immer abzuwägen, wie viel Pulver er einsetzt, denn das kostet ihn immer auch Kraft. Nur gut, dass das Echsenwesen Sabri weitere Pulver für ihn trägt und ihm stets folgt.“

„Beeindruckend“, rief Jorna, „Hat Eure Heldengruppe denn schon einen Namen gefunden?“

Wieder blickte Iril etwas hilflos zu den restlichen Helden zurück.

„Brauchen wir denn einen? So, ohne Gruppenbesprechung werde ich mir wohl kaum einen solchen aus den Fingern saugen können. Können wir vielleicht ohnehin etwas später mit dieser Befragung fortfahren? Aufregung liegt hinter uns

„Natürlich, natürlich“, sagte Sanja, „Wir wollen Sie nicht weiter aufhalten. Wir werden bei der nächstbesten Gelegenheit wieder auf Sie zurückkommen. Es gibt noch so einiges Faszinierendes über Sie zu erfahren. Das verrät mir mein Bauchgefühl.“

Sanja tippte Jorna auf der Schulter. Die beiden Bewahrerinnen verabschiedeten sich mit einer Verbeugung.\bigskip







Iril zog sich in einen der provisorischen Massenverschläge zurück und versenkte sich in ihre Runen. Ijsdur hatte sich inzwischen in den Augen des Prinzen so sehr verdient gemacht, dass er ebenfalls dort ruhen durfte. Aćh schloss sich ihnen an, auch wenn ihr hin und wieder brennender Vogel allerlei Blicke auf sich zog, sowohl bewundernde als auch besorgte.

Später am Tag bekam Iril mit, wie die Takuri-Hüterin auf ihrer Steinflöte eine wilde Melodie spielte und ihr Takuri in einem Flammenwirbel verschwand.

„Kann er etwa auch unsichtbar werden?“, fragte Iril. Ijsdur übersetzte.

„Ne, nur nach Tulgor zurückspringen. Meiner Familie und Bekannten am Nistbaum signalisieren, dass es uns gut geht. Briefe kann er schlecht mit sich führen, doch sein Auftauchen sollte erste Sorgen beheben. Es gibt so einige, die sich Sorgen um mich machen könnten, sobald Haamuns erste Nachrichten vom Lumiwurmüberfall zu ihnen gelangen. Nelímar. Òkôkó. Nugal. Efroćhin. Yrbstschly.“

Barz trennte sich von der restlichen Gruppe. Er wollte raus ins Rietland. Dorthin, wo laut Nabib einige andere Barbaren ihre Zelte aufgeschlagen hatten.

Auch wenn es sich dabei hauptsächlich um Mitglieder verschiedener Yetohe-Stämme handelte, erkannte Barz den einen oder anderen wieder. Nabib stellte ihm die besten Köche, Schnitzer, und gar die legendäre Jägerin Naldia vor, die der Legende nach schon mal einen Felltroll mit einem einzigen Schuss erledigt hatte.

Vor dem Lagerfeuer erzählten die Barbaren einander Geschichten von vergangenen Zeiten. Und als die Sonne untergangen war und das Sternenmeer am Himmel funkelte, zogen sich Barz und Nabib in eine leerstehende Jurte zurück.

Zum ersten Mal seit langem waren sie wieder allein miteinander.

Und so redeten sie über all jenes, was ihnen widerfahren war. Und was sie in Zukunft sein wollten.\bigskip







Barz‘ Hand schoss an seine wiedererlangte Ringkette. Seine Stimme zitterte: „Und Yafka geht es wirklich gut? Karyz und Zan auch? Und Asbark? Und ...“

„Du darfst noch zehnmal danach fragen und ich werde dich jedes Mal beruhigen. Einiges hat sich geändert in Thakkum in den letzten Jahren. Doch deinen Lieben geht es allen prima, den Umständen entsprechend.“

Ein Kloß der Furcht machte sich in Barz breit „Was soll das heißen?“

Nabib lehnte sich zurück. „Hast du schon von der Ewigen Kälte gehört? Die hat den großen See Ava ganz besonders hart getroffen. Die Nahrungsbeschaffung war hart. Zan wurde vom Eisschlaf erwischt.“

„Meinst du diesen unnatürlich langen Winter? Ja, natürlich habe ich von dem gehört. Tulgor war von der unnatürlichen Kälte ebenfalls betroffen. Wir dachten schon, die Eis-Dämonen des ewigen Eises hätten einen Weg durchs Felsentor gefunden. Erst der Hüter der Zeit konnte uns darüber aufklären, dass der Ursprung dieser Kälte jenseits des Fahlen Gebirges lag und dass die Angelegenheit um den Winterstein hier gelöst werden würde. Die Feuertakuri mochten das Ganze dennoch überhaupt nicht. Einige sind in den Eisschlaf gefallen.“

Barz hatte einige Worte verwendet, die Nabib nicht kannte. Statt nachzufragen, berichtete er leise: „Ich reiste einmal zurück an den Ava. Begleitete einen zwischen dem Osten und dem Rietland umherpendelnden Händler. Es waren beruhigende Monate. Aber ohne dich irgendwie nicht dasselbe. Und auch nicht so aufregend. Hier in Andor ist hingegen der Fluch der Götter los. Gute Güte, stets greifen irgendwelche finsteren Gesellen und Kreaturenarmeen an.“

„Es sind wilde Zeiten.“

Nabib wurde ernst: „Yafka erzählte mir, was mit dir geschah. Wie du dich hierher teleportieren wolltest. Nach Andor. Zu mir. Und wie du in einem Flammenbausch verschwunden seist, und sie seither kein Wort mehr von dir vernommen hätten.“

Er blieb einen Augenblick lang stumm, dann rief er: „Du Dummkopf! Was haben wir immer gesagt? Vorsichtig mit neuen Mitteln umgehen! Testen, dann handeln! Du hättest wer weiß wohin versetzt werden können! Ins eisige Meer des Nordens oder tief in die Lavameere des Südens!“

„Ich wusste, was ich tat! Ich landete in der Heimat des Phoenixes. Nur war diese halt nicht Andor.“

„Leichtsinnig! So leichtsinnig! Warum nur?“

„Ich wollte bei dir sein. Rasch. Du hattest dich so lange nicht gemeldet ...“

„Weil ich im Fiebertraum im Lazarett lag!“

„Da hätte ich erst recht an deiner Seite sein sollen! Ich wünschte mir so oft, du wärst bei mir. Du hättest einen Blick auf diesen fremden Sternenhimmel geworfen und gewusst, wo wir uns befinden. Ich hingegen konnte kaum mit dem Sonnenstand etwas anfangen!“

„Du überschätzt mich, Barz“, grinste Nabib. Seine Mundwinkel zuckten, doch seine Augen blieben wehmütig. Darin erkannte Barz den Blick, den er nun seit über zwei Jahren vermisst hatte.

Barz stand ruckhaft auf. „Nachrichten! Jetzt, wo ich wieder hier bin, jetzt, wo wir wieder die Möglichkeit haben, muss ich eine Nachricht nach Thakkum schreiben. Allen sagen, dass es mir gut geht. Sie fragen, wie es ihnen geht.“

„Klar, Barz. Wir werden das tun, erste Priorität morgen. Ich kenne den königlichen Falkner gut, er wird uns eines seiner besten Tiere überlassen.“

„Auf dass die Krarks es nicht erwischen mögen.“

„Jetzt tu mal nicht so düster, die Lage ist erheblich besser geworden. Apropos Flugviecher: Du wirst ja wohl kaum schon gehört haben, dass Tarok gefallen ist? Dieser elende ...“

„Oh, das habe ich sehr wohl gehört“, sagte Barz stolz, „Seine letzten Knochenfragmente haben wir soeben dem Prinzen ausgehändigt.“

Barz‘ Miene fiel, als er anhängte: „Wir haben diesbezüglich Schamanin Sagramak von den Jpaxo wiedergetroffen. Und uns gleich mit ihr zerstritten. Ich hoffe sehr, dass das kein übles Nachspiel haben wird.“

„Was wollen sie schon tun, sich mit den hiesigen Schamanen um ihren Glauben streiten? Wusstest du, dass die dich hier nur eine Person heiraten lassen? Die Bewahrer meinen, ihre heilige Mutter Natur wolle es so.“

„Mutter Natur?“

„Der Kult am Baum der Lieder hat sich ihr verschrieben. Ich glaube aber, dass dies nur ein anderer Name für den Großen Specht ist. Die Bewahrer meinen, sie segne jeden solchen Bund hier in Andor. Und dass es zu jeder Zeit nur einen solchen Bund pro Person geben könne, auf dass er Bund nicht an Bedeutung verliere.“

„Na, das ist ja limitierend.“

„Lass ihnen Zeit, Traditionen wandeln sich langsam. Vor einigen Generationen sagten die Schamanen der Yetohe auch noch, dass die Götter ausschließlich mit Männern sprächen.“

Barz lächelte. „Der Barbarenkönig und seine Sippen mussten sich bestimmt daran gewöhnen, dass die Leute hier unabhängig ihres Geschlechts gleich viel gelten. Iquar verlangte dies schon vor Urzeiten von uns.“

„Na, die Andori lassen auch erst seit kurzem Frauen für ihr Land kämpfen. Rein männliche Berufe waren eigentlich ein Ding der Sklaventreiber aus dem Süden. Es braucht Zeit, solche Joche der Krahder zu überwinden.“

„Es braucht Zeit und Einsatz. Und es wird uns Zeit und Einsatz kosten, wieder zueinander zu finden. Oh, Barz, ich spürte, dass du am Leben warst, und deine restliche Familie auch, und wir hofften, dass du einfach an einem netten Ort feststecktest. Aber es war schon gruselig, wie kein Laut von dir zu hören war.“

„Ich versuchte, euch mithilfe der Verbindung durchs Meditationspulver Nachrichten zu schicken, aber das ging fehl.“

„Asbark, deine Schamanin, hat ebenfalls versucht, Kontakt aufzunehmen. Irgendwann haben wir aufgegeben. Ich habe hin und wieder deine Stimme gehört, Barz. Es hat mir Kraft gegeben, auch wenn ich deine Worte nicht verstehen konnte.“

„Ja, gezielt Nachrichten zu übermitteln, scheint noch nicht Teil der Pulverkräfte zu sein. Auch wenn es irgendwie möglich sein muss. Ich werde mich dem intensiven Studium des Meditationspulvers widmen. Jetzt, wo ich wieder diesseits des Kuolema-Gebirges ruhe, kann ich mir wieder frische Krarkfedern liefern lassen und ...“

Mit Blick auf Nabibs Blick unterbrach sich Barz und meinte: „... doch nun ist vor allem wichtig, dass unsere Zeit der Trennung hinter uns liegt.“

„Natürlich. Auch wenn ich mir nicht sicher sein konnte, dich je wiederzusehen, betete ich die Götter dafür an. Immerhin wusste ich, dass du noch lebtest und an mich dachtest. Aber die Eindrücke kamen immer seltener mit der Zeit.“

Die leise Anschuldigung blieb nicht unbemerkt.

„Verzeih mir, Nabib. Ich hätte mich häufiger melden sollen. Ich wusste, dass wir uns wiedersehen würden. Es wurde mir prophezeit. Und doch wusste ich nicht, wie es dir ging. Ob ich dir auf die Nerven ging mit meinen Meditationen. Ich gebe zu, ich fürchtete mich gar etwas vor unserem Wiedersehen.“

Nabib gluckste auf. „Alles ist gut. Wir haben einander wieder. Ich hoffe, diese Befürchtungen haben sich nicht bewahrheitet. Hast du nun etwa Geheimnisse vor mir?“

„Niemals. Ich werde dir alles erzählen, was du hören willst über meine Zeit in den fremden Landen. Natürlich ist es anders. Wir sind nicht mehr dieselben, die wir einst waren. Und doch erkenne ich in dir dieselbe einzigartige, liebenswürdige Person, die ich damals ... die mich damals ... du ...“

Angespannte Stille. Das letzte Mal, als sie einander gesehen hatten, waren sie sich uneins gewesen. Barz hatte Nabib ziehen lassen. Nabib hatte Barz zurückgelassen. Solche Spannungen konnten Beziehungen zerstören, das wussten sie beide. Keiner von ihnen wagte, diese letzte Nacht anzusprechen, als der Barbarenkönig mit seinen Sippen in der Pfahlbausiedlung vorstellig geworden war. Keiner wollte das Vergangene Revue passieren lassen.

Nabib kuschelte sich an Barz Schulter. Seine kräftigen Hände massierten, nein, streichelten Barz‘ Rücken. Barz tastete nach der Wange seines Freundes. Ein Glimmen leuchtete in Nabibs Blick auf.

Barz lächelte schief. „Ich habe mich seit Tagen nicht gereinigt. Der Gestank ...“

„Könnte mich kaum weniger kümmern. Irgendetwas Neues, was ich wissen sollte?“

„Ich gab gut Acht, mir nichts einzufangen.“

„Ebenso.“

Barz legte fragend seinen Kopf schief. Nabib ebenso.

Barz nickte enthusiastisch. Nabib ebenso.

Barz schlüpfte aus seinem Mantel. Nabib ebenso.\bigskip







Urplötzlich wich Nabib von Barz zurück und wandte sich ab.

„Was ist? Nabib, was ist?“

Nabib wedelte wegwerfend mit seiner Hand und sprach keuchend: „Nein, es ist nichts. Lass uns ...“

Ein Schluchzer schüttelte ihn, ehe er weitersprechen konnte. Barz hielt angebracht Abstand, während er fragte: „Wie kann ich helfen?“

Nabib schüttelte seinen Kopf und murmelte etwas unverständliches vor sich. „Nein, das ist ... es ist lange her. Ich musste nur daran denken ...“

Barz setzte sich auf den Boden und gestikulierte zu Nabib, es gleich zu tun. Nabib hielt weitere Schluchzer zurück, doch sein starrer Blick zeugte davon, dass nicht alles in Ordnung war.

„Meinst du, es könnte helfen, darüber zu sprechen?“

„Ist bei den meisten Dingen so“, druckste Nabib herum.

„Dann willst du davon erzählen?“

Nabib grummelte etwas. „Ich schäme mich. Ich will nicht ...“

„Du brauchst dich nicht zu schämen. Nicht vor mir.“

Nabib mahlte mit seinen Zähnen und zog sich etwas zurück. Leise grummelte er: „Oh, wie verdammt stolz ich damals war, an der Seite des Yetohe-Stammes nach Andor zu ziehen. Unsere Familien zu verteidigen, indem wir anderen ihr Land raubten. Ich nehme nie wieder mein Schwert in die Hand. Ich habe Unrecht getan hier. Ich habe Bauern und Kinder aus ihren Häusern vertrieben. Ich ... Barz ...“

Nabib schluckte schwer.

„Ich kenne noch nicht einmal ihre Namen. Sie standen plötzlich vor mir, Mistgabeln in den Händen. Sie erwischte mich in der Brust. Ich führte meine Axt und ... ich war bereit, mein Leben für unseren Stamm zu riskieren, und für die Stämme der Brüder, und ich habe Unschuldige umgebracht. Ich werde dieses Unrecht niemals bereinigen können. Und das verlangt niemand von mir. Sie alle behandeln mich wie einen der ihren, diese Andori. Nun, die meisten wissen es auch nicht. Aber ich weiß es. Und die Götter wissen es.“

Barz blieb stumm. Er wusste nicht, was darauf zu sagen war. Sorgsam hob er seine Hand.

„Soll ich dich drücken?“

Nabib hustete trocken auf. Dann nickte er stumm und Barz umschlang ihn.

Leise schluchzte Nabib in seine Schulter hinein: „Lieber gebe ich mein Leben hin, als dass ich wieder in solche Zwiste verwickelt werde. Kämpfen werde ich nie mehr. Ich widme mich der Heilkunde. Ich lag eine Zeit lang flach nach dem Angriff. Monate, gar. Meine Rippen sahen schon bessere Tage. Eine Wunde hatte sich entzündet und warf mich in fiebrige Albträume. Doch Heiler Readem rettete mich. Mich, einen Angreifer in sein Reich! Noch während meiner Genesung begann ich, bei ihm auszuhelfen. Stellt sich heraus, dass die Zeichenkünste, die ich mir als Kartograph angeeignet habe, durchaus sehr geeignet sind für das Skizzieren von Karten des Körpers. Faszinierende Angelegenheit, das.“

Nabib blinzelte seine Tränen beiseite.

„Hast du schon mal gesehen, wie ein Körper von innen aussieht? Unsere Schamanen vollzogen solche Riten ja immer im Geheimen. Hier kann jeder Interessierte die Körperkunde lernen. Was Heiler Readem mir alles gezeigt hat von Innereien ...“

Barz verzog sein Gesicht und drückte ein „Schön, dass dir das gefällt“ heraus.

Nabib grinste schwach, als er Barz‘ angeekelte Miene sah.

„Und, Nabib? Wir das deine Zukunft sein? Du lebst nun hier, in diesem Land?“

„Ich mag es hier.“

„Und Yafka lebt weiter im Lande der drei Brüder?“

„So ist es.“

„Oh, da wird es Entscheidungen zu treffen geben. Wohin ziehen mich die Götter? Wohin wollt ihr mich ziehen sehen? Und wohin will ich ziehen? Es fühlt sich alles noch so unwirklich an. Bist du wirklich hier? Geschieht das hier wirklich? Es geht alles so schnell. Dieser Prinz ernannte mich einfach zu einem Helden von Andor. “

„Gratuliere, mein Held. Diese Ehre erfahren nicht viele. Was hast du getan?“

„Ich? Nur einige Knochensplitter zurückgebracht.“

„Dann stellt sich vielmehr die Frage, was du noch tun wirst.“

„Ist das Zurückkehren zu unserem vorherigen Leben als Steppennomaden noch eine Option?“

„Ach, Barz ... ich bin mir nicht sicher, ob ich das überhaupt will. Ich habe mich hier lange niedergelassen. Vielleicht gefällt mir dieses Leben mehr.“

Barz nickte. „Ich weiß nicht, was ich mir vorgestellt habe. Dass ich dich einfach hier aufgable und wir wieder ins Land der drei Brüder zurückkehren, unsere großen Reisen durch die Steppe weiterführen? Was sind wir? Sind wir immer noch eine Familie?“

„Schwer zu sagen, Barz. Wollen wir noch eine Familie sein? Kannst du mir verzeihen?“

„Oh, Nabib. Ich war nie wütend auf dich, dass du gegangen bist.“

„Du solltest es sein.“

„Bist du wütend auf dich selbst?“

Nabib antwortete nicht. Mehr zu sich selbst als zu Barz murmelte er: „Es wird Aufwand kosten, die lange Zeit voneinander zu überbrücken. Wir werden aneinanderkrachen und zerbrechen und, so die Götter es wollen, uns wieder zusammensetzen und unsere gemeinsame Zeit aufs Neue genießen. Ich bin zuversichtlich, dass wir das hinkriegen. Selbst wenn es wieder so eine Nacht wie im Goldtal geben sollte.“

„Daran denke ich nicht gerne zurück. Ich gebe mein Bestes, dass ich meine Gefühle besser unter Kontrolle behalte. Schlimmstenfalls schnappe ich mir Ijsdurs Kette ...“

„Wäre das wirklich gesund? Gefühle müssen auch gelebt werden. Barz, lass gut sein. Verzeih, ich hätte das Goldtal nicht aufbringen sollen.“

„Ich nannte dich damals einen gehörnten ...“

„Lass es gut sein, wirklich. Ich hatte mich damals auch wirklich gehörnt verhalten. Und wenn wir diese Nacht überstehen konnten, werden wir auch das hier überstehen können.“

„Das gefällt mir.“

„Mir auch.“\bigskip







Einige Zeit später, einige Distanz entfernt.\bigskip



Najuk sah die weiße Gestalt, die sich aus den nebelverhangenen Bergen des Fahlen Gebirges löste. Eine Wolke aus Schnee und Eis wirbelte um sie herum, und es schien, als würde sie darauf schweben. Die Erscheinung trug das Antlitz einer Frau. Sie hatte etwas Faszinierendes an sich.

„Wie viele wollen von euch kommen denn noch?!“, lachte er und machte sich auf, Ijsdurs mutmaßliche Kumpanin freundlich zu begrüßen.

Dann hielt er inne. Seine Beine wollten ihm nicht mehr gehorchen. Sein Blick war auf die fremde Frau fixiert. Najuk konnte sich ihrem Bann kaum entziehen. Sie strahlte eine solche Eiseskälte aus, dass er spürte, wie seine Glieder langsam erstarrten. Bewegungslos sah er die durchscheinend anmutende Gestalt auf sich zukommen. Kalt sprach sie: „Ich bin Siantari. In den tiefen Schluchten des Kuolema scheint niemals die Sonne auf das ewige Eis, und dort ist mein Reich.“

Najuk verstand ihre fremden Worte nicht. Doch sie waren das letzte, das er hörte, bevor seine Sinne ihn verließen.\bigskip







„Wie weit ist es noch?“, fragte Sagramak protestierend. Der Schwarze Herold antwortete nicht, sondern schwebte weiterhin schweigend den Hügel hinauf. Oben an der Hügelkuppe hielt er inne. Die glühenden Augen hinter der gezackten Maske verrieten keine Regung. Mit seinem langen Schwert zeigte er über die verschneite Ebene. Auf die einsame weiße Gestalt, die mit ausgebreiteten Armen an Ort und Stelle schwebte, während ein Wirbelsturm aus Eis und Schnee um sie herumtobte. Vor ihr lag der zusammengesunkene Körper eines Andori.

Die Gestalt zog aus ihrem schneeweißen Gewand eine Kette aus spitzen Eiskristallen hervor. Sie beugte sich herunter, als wollte sie dem starren Andori die Kette anlegen.

„Siantari!“, rief Sagramak laut. „O holde Dämonin des ewigen Eises! Möchtest du deine Macht nicht lieber mit jemandem teilen, der tatsächlich davon profitieren kann? Der vielwissende Herold der Drachen berichtet mir, dass du diese Lande unter einer dicken Schicht ewigen Eises verschwinden lassen willst. Doch sind diese Gefilde gefährlich für Feinde der Krone. Wir könnten uns zusammentun, zumindest eine Zeit lang. Das wäre für uns beide von Vorteil.“

Siantaris eisblaue Augen fixierten die Schamanin. Sagramak spürte, wie sich eine Kälte in ihr ausbreitete. Dann warf Siantari – weiterhin stumm – Sagramak ihre Eiskristallkette zu. Die Schamanin unterdrückte den Instinkt, die Kette mit bloßen Händen zu fangen. Der Herold hatte sie davor gewarnt. Stattdessen tappte Sagramak in sich hinein und nutzte das magische Geschenk, das die Drachen ihr vermacht hatten.

Mit ausgestreckter Hand gebot Sagramak der Eiskristallkette telekinetisch Halt. Siantaris Miene blieb ausdruckslos.

Rasch schluckte Sagramak den Kloß in ihrer Kehle herunter und sprach: „Versteh mich nicht falsch. Ich möchte keine Eis-Dämonin werden. Ich bin vom Feuerdrachen Sagrak beseelt. Die Drachenseele in mir hätte gar keine Freude daran, wenn ich diese eiskalte Kette anfasste. Doch vielleicht kann ich auch so etwas Schönes mit ihr schaffen. Ich spüre die ungezähmte Kraft, die darin eingesperrt ist. Sagraks Seele fühlt es auch. Sie musste sich schon so lange mit meinem schwachen Körper begnügen. Sie drängt darauf, wieder in einen größeren Leib zu schlüpfen.“

Siantari blickte ausdruckslos drein. Sie verstand kein einziges Wort, das die fremde Frau vor ihr sprach. Doch das musste sie auch nicht. Die Eiskristallkette war nun nahe genug an der Haut der Fremden, damit Siantari flüchtige Eindrücke ihres Geistes erhaschen konnte. Und was sie da sah, gefiel ihr.

Siantari nickte und breitete ihre Arme aus. Eine Wolke aus Schnee und Eis erhob sich um sie und trug die Dämonin hoch in den Himmel. Dann wirbelte sie davon.

Zurück blieben der Schwarze Herold und Sagramak, vor deren Hand eine kleine, unscheinbare Kette schwebte.

„Wow!“, rief Sagramak, „Den Drachen sei Dank, du hattest recht! Wie knapp bin ich soeben mit meinem Leben davongekommen?“ Nachdenklich blickte sie der davonschwebenden Siantari hinterher.

„Tu es!“, drang eine zischende Stimme unter der eisernen Maske des Schwarzen Herolds hervor.

„Hetz mich nicht“, protestierte Sagramak. Allerdings schloss sie fügsam ihre Augen und konzentrierte sich. Blauweiße Schleier der Magie umschwirren sie.

Als Sagramak ihre Augen wieder öffnete, flackerte ein roter Schein daraus hervor. Laut rief sie: „Sagrak, o edler Drache, dessen Seele schon zu lange in einem zu kleinen Relikt um meinen Hals steckte. Ich beschwöre dich. Möge der Geist, der in mir ruhte, nun ein neues Gefäß erhalten. Ein besseres. Ein magisches.“

Aus Sagramaks Rüstung erhob sich, wie von unsichtbaren Händen geführt, eine lange eiserne Kette, an der ein rotes Drachenrelikt schimmerte. Die metallene Drachenfigur darauf zeigte Sagrak, wie er zu Lebzeiten ausgesehen haben mochte.

Sagraks Knochenfragmente, die ebenfalls an einer Kette um Sagramaks Hals gehangen hatten, glommen bläulich auf. Sie brachen die eisernen Kettenglieder zwischeneinander und schwebten nach vorne. Knapp über dem Boden blieben die Drachenfußknochen zitternd schwebend stehen.

Die Eiskristallkette summte und brummte. Wasser tropfte aus dem Boden in den Himmel, kondensierte aus der Luft, erstarrte zu Eis und Schneeflocken. Wie aus dem Nichts formten sich weitere Drachenknochen aus purem Eis und ordneten sich neben den echten an. Dann wurden sie von einer Schicht aus Eis und Schnee überdeckt, die zunächst Muskeln, dann Haut und schließlich messerscharfen Schuppen glich.

Die Eiskristallkette schwebte aus Sagramaks ausgestreckter Hand nach vorne und fügte sich nahtlos in die Brust des riesigen Eis-Drachen ein, der vor Sagramak und Siantari schwebte und zitternd ein- und ausatmete.

Schneeflocken wirbelten wild umher, als der schneeweiße Drache seinen vereisten Kopf schüttelte und aus Versehen mit einem ungewohnten vielzackigen Dämonengeweih gegen den Boden stieß. Gewaltige eisblaue Schwingen wurden gespreizt, während eisige Blitze über den Himmel zuckten. Klirrende Dampfwolken bildeten sich vor der langen, durchscheinenden Zunge, die aus einem vielzahnigen Mund herabzüngelte. Dann wurde der Mund aufgerissen und die dahinter liegenden Stimmbänder entließen ein ohrenbetäubendes Gebrüll.

Sagrakdur war erwacht.




















\newpage
\section{Die Spur der Eis-Dämonin}




Die tapferen Helden Chada, Thorn, Eara, Kram, Fenn, Hogo, Bragor, Kheela, Iril, Ijsdur, Aćh und Barz fanden sich beim ersten Sonnenlicht im Rietland zusammen. Leise Worte wurden gewechselt.

„..., zehn, elf, zwölf“, zählte Chada, „Wo ist Tenaya?“

Eara kannte die Antwort: „Tenaya wollte unbedingt mit ihrem ehemaligen Meister Lifornus und diesen beiden Danwaren an der Stätte der heiligen Flammen in der Barbarensteppe diesen Lavastein in des Feuerkriegers Brust untersuchen. Sie hoffen wohl, mit ihren Feuerzaubern etwas darüber herauszufinden, was den beiden Orden Danwars entgangen wäre. Und ausgerechnet so weit im Osten! Mit ihnen – und mit Flaps – können wir in der nächsten Zeit nicht rechnen. Meister Lifornus meinte gar, er wolle bis im nächsten April an der Stätte der heiligen Flammen verbleiben. Irgendeine mächtige Sternenkonstellation soll dort am 1. Tag des 4. Mondes eine besondere Macht enthüllen.“

„Lifornus?“, horchte Barz auf. „Etwa derselbe Lifornus ...“

„Derselbe Lifornus, der so leichtsinnig einen Drachen und einen Takuri beschwor?“, fragte Chada mit einem schiefen Grinsen, „Ich kenne die Geschichten der großen Geschehnisse in Thakkum. Nein, Tenayas Lifornus ist erheblich älter und etwas weiser als der eure.“

„Der Name ist in Hadria weit verbreitet“, steuerte Eara bei.

Barz wollte schon zur Verteidigung der Weisheit seines Lifornus antreten, überlegte es sich dann jedoch wieder und schloss seinen Mund.

„Folglich sind Tenaya, Jarid und Trieest heute allesamt nicht zu erwarten“, rekapitulierte Chada.

Kram fügte an: „Und Orfen ist mal wieder mit Merrik im Grauen Gebirge unterwegs. Auf ihn müssen wir auch nicht warten.“

„Orfen und Merrik?“, meinte Barz, „Dieser brummige Wolfsfreund und dieser Kartenzeichner, der keine rohen Fische essen will, ja, der sogar Stahlfischöl ausschlägt? Ich kenne sie beide, einst verschlug es sie mitten in unsere schöne Steppe. Wie geht es ihnen?“

„Du kennst sie?“, meldete sich Iril, „Orfen haben wir erst kürzlich getroffen. Er hatte vor Taroks Tod einen beeindruckenden Sieg über den Schwarzen Herold errungen.“

Thorn sagte: „Zu schade, dass Merrik weg ist. Er hätte den Tulgori bestimmt bei ihrer Kartographie aushelfen können.“

Kram meinte: „Ich bin mir da nicht so sicher. Kartographen sind so ein Völkchen, die wollen manchmal lieber ihre eigenen Karten zeichnen, als die anderer zu kopieren.“

„Zumindest abgleichen hätte man sie anschließend können.“

„Können sie ja später immer noch.“

Chadas Stimme unterbrach die Gespräche: „Verzeiht, werte Anwesende, doch wollen wir den Tratsch nicht in die Taverne verschieben und uns zunächst auf die dringlicheren Probleme berufen?“

Es wurde still in der Versammlung der Helden. Barz‘ Fokus richtete auf Chadas Bogen, den die Bewahrerin an einen Stein gelehnt hatte und der nun direkt vor ihm lag. Barz griff danach, staunte über seine Länge und zupfte probehalber an der Sehne.

Er nickte Chada anerkennend zu. Diese bat ihn freundlich lächelnd, Audax doch bitte in Ruhe zu lassen. Barz‘ Mund formte lautlos die Silben „Audax“. Dass die Bewahrer ihren Waffen eigene Namen gaben, half ihrem mythischen Status.

Chada eröffnete die wöchentliche Versammlung der Helden von Andor. Es gab Berichte über die Heldentaten der letzten Woche auszutauschen, verschiedene Hilfsbedürfnisse abzuwägen und Pläne zu ihrem Handeln zu treffen. Auch wenn die Helden von Andor keine strikte Struktur wie der Bewahrerorden besaßen und ihre Mitglieder eigentlich frei handeln konnten, hatten sich in der wilden Zeit nach Taroks regelmäßige Absprachen als äußerst hilfreich erwiesen.

Insbesondere jetzt, wo sie die neugierigen Tulgori bei ihrer Kartographie des Landes zu beschützen hatten.

Die Helden waren zum Aufbruch bereit, als sie in der Ferne einen Mann sahen, der schnell in ihre Richtung lief.

„Wir brauchen eure Hilfe! Andor ist in großer Gefahr! Dreifacher Gefahr!“, rief der Andori atemlos, als er endlich die Heldengruppe erreicht hatte. Vierundzwanzig Heldenaugen und einige Tulgori blicken ihn erwartungsvoll an.

In Irils Magen machte sich ein Unwohlsein breit, als sie die Verzweiflung im Gesicht des Mannes deutete. In letzter Zeit hatte sie wie die meisten restlichen Helden auch den Andori geholfen, Taroks Spur der Zerstörung ungeschehen zu machen. Bauernhäuser waren wieder aufgebaut worden. In die Rietburg geflüchtete Bauern konnten endlich wieder zu ihren Katen und Feldern zurückkehren.

Doch die Lage sah nicht gut aus. Kaum eines der Felder hatte den Zorn des Drachen überstanden und so versuchten die Andori durch Fischfang und Jagd zu überleben. Doch mussten sie stets fürchten, selbst zur Beute zu werden. Und wie sollten die tapferen Andori das wenige übrige Korn ernten, wenn sie jederzeit einen Angriff fürchten mussten? Der kommende Winter würde bald seine eiskalte Hand ausstrecken. Die Zeit drängte.

Und so hatten sich die Helden von Andor, darunter auch die neu dazu gestoßenen Iril, Ijsdur, Aćh und Barz, nach getaner Arbeit nicht etwa ausruhen können, sondern mit doppeltem Elan an die Verteidigung der tapferen Andori gemacht.

Tapfer waren die Andori wirklich. Schon von Kind auf bekamen sie von ihren Eltern eingebläut, die Kreaturen zu fürchten, ja, es gab sogar Kinderreime dazu. Abgehärtet, wie sie waren, lösten einzelne Gors und Skrale in ihnen kein Schlottern aus, sondern nur einen zügigen temporären Rückzug in sichere Verstecke. Manchmal gar einen Griff zur Mistgabel. Was konnte also vorgefallen sein, dass dieser eine Andori vor ihnen derart ängstlich angerannt kam? Von welch dreifacher Gefahr sprach er wohl?

„Es ist Shron, der Sohn des Hark, der uns bedroht“, sagte der Mann, nachdem er sich etwas beruhigt hatte. „Er zieht durch das Land und rüstet sich für einen Angriff auf die Rietburg. Mehr und mehr Kreaturen schließen sich ihm an.“

„Den übernehmen wir!“, rief Thorn aus, „Mit ein bisschen Hilfe von Reka konnte ich vor einigen Jahren schon seinen Vater ausschalten. Ich fühle, Harthalts Tod ist noch nicht genug gerächt! Ich dürste nach Skralblut!“

Kram wich etwas zurück, als er die Mordlust in Thorns Blick erkannte. Chada trat hinter Thorn und legte ihm besänftigend eine Hand auf die Schulter. Er verstummte. Chada erhob selbst ihre Stimme und sprach etwas gefasster: „Wenn wir Shron einen Schrecken einjagen können, der für den Rest seiner Tage auf seinem Gesicht geschrieben steht, genügt das schon.“

„Viel kann man von seinem Gesicht nicht sehen“, meinte Eara, „Wir haben Shron gerade erst bei der Befreiung der Rietburg getroffen. Er war der Skral mit der eisernen Maske.“

„Umso besser!“, lachte Thorn, „Da kann man tatsächlich einen Schrecken draufschreiben.“

Chada, Thorn, Eara und Kram wichen etwas zurück und berieten sich darüber, wie man am besten die Bildung eines Kreaturenheers verhindern könnte. Der Andori, der die Botschaft überbracht hatte, atmete rasch ein und aus, als er nach weiteren Worten rang.

„Du sagtest, da sei noch mehr?“, hakte Kheela nach, die Hüterin der Flusslande, die oft mit einem danwarischen Stab einen Wassergeist umherlenkte. Gerade tanzte Vara über ein nahegelegenes Feld und löste die Schneespuren auf, die ein gewisser unachtsamer Eis-Dämon beim Darüberspazieren hinterlassen hatte.

„Ja, da ist noch mehr. Es ist Skuvar“, nickte der Mann, nachdem er sich etwas beruhigt hatte, „und mit ihm sind die Maasavi erwacht. Skuvar ist ein uralter Erdgeist. Die Maasavi sind seine Schatten, die Seele der andorischen Erde. Ich weiß nicht, was sie hervorgelockt hat, doch Skuvar wurde unweit der Rietburg gesehen.“

Der Mann schnappte kurz nach Luft und sprach dann: „Ihr könnte ihn nicht verfehlen. Ein massiger Körper mit einem dehnbaren Mund, der blau leuchtet. Rücken- und Schwanzstacheln, länger als mein Unterarm. Und gewaltige Vorderbeine, auf denen er sich manchmal im Handstand fortbewegt. Seid wachsam. Die Maasavi folgen seinem Ruf und werden angetrieben von Eurem Willen, Andor zu verteidigen.“

„Ich kenne mich mit Erdgeistern aus!“, rief Kheela, „Vara wird uns helfen, sie zu besänftigen. Wir hatten in den Flusslanden schon mal mit einigen zu ringen.“

„Gehen wir vorhin zum Thronsaal und holen uns den Bruderschild, auf dass Bragor uns an der Kraft seiner prallen Muskeln Anteil haben lassen kann?“, schlug Fenn vor.

„Ich verstehe immer noch nicht, warum wir den Schild diesem König geben mussten“, flüsterte Bragor laut genug, um von allen Anwesenden gehört zu werden. Taren hatten bekanntlich unglaublich gute Ohren, flüsterten aber offenbar dennoch ziemlich laut. „Jetzt vergammelt der Bruderschild die größte Zeit in der Burg, während Brandurs Sohn, der Schürzenjäger, oft nicht mal in dort ist, und den Schild noch seltener nutzt.“

Hogo stupste Bragor zwar in die Rippen, um ihn vom unbedachten Sprechen abzuhalten, doch handelte er sowohl zu spät als auch zu sanft, damit Bragor es überhaupt registrierte.

Barz blickte Bragor fasziniert an, wie jedes Mal in den letzten Tagen, wenn Bragor sprach. Barz hatte Iril zwar erklärt, dass er im Tarus nicht den fleischgewordenen großen Büffel sah, den die Yetohe in ihm gesehen hatten. Dennoch schien ihn etwas an diesem Fremden aus Sturmtal zu faszinieren.

Hogo, Bragor, Fenn und Kheela koppelten sich ebenfalls von der restlichen Heldengruppe ab.

Damit blieben noch die vier neusten Helden von Andor übrig.

Die Augen des Mannes wurden leicht glasig, als er zu schlottern begann. Als er sich wieder etwas beruhigt hatte, berichtete er den Helden von einer weißen Gestalt, umgeben von Wirbeln aus Schnee und einem solch kalten Blick, dass man meinte, sofort zu Eis zu erstarren. Nur Augenblicke später sei die Gestalt wieder verschwunden.

„Siantari ...“, flüsterte ein Tulgori blass. „... eine Dämonin des Kuolema. Ihr müsst sie finden und aufhalten, sonst wird sich das ganze Land in eine Eiswüste verwandeln. Es hat sicher schon begonnen ...“

Ijsdur, der die Tulgori überhaupt erst vor Siantari gewarnt hatte, nickte stumm. Der Mann, dem Ijsdurs Anwesenheit überhaupt erst jetzt aktiv aufzufallen schien, blickte ihn erwartungsvoll an. Dieser erwiderte den Blick still, offensichtlich nicht kapierend, was man von ihm erwartete. Da meldeten sich Aćh und Barz beinahe gleichzeitig zu Wort:

„Das klingt nach einem Fall für uns! Wir hatten schon einmal mit einer Eis-Dämonin zu tun und waren siegreich, und da waren wir nur zu zweit! Eigentlich war es sogar Aćh allein.“

„Und du hast meinen Arm gebrochen“, sagte Aćh in Tulgorisch.

„Technisch gesehen hast du ihn dir selbst gebrochen“, gab Barz zurück.

„Da schließe ich mich doch euch an“, hob Iril ihre Hand. Alle blickten gespannt zu Ijsdur. Dieser nickte ebenfalls.

Die vier waren in den Tagen nach dem Erringen der Drachenknochen nur enger zusammengewachsen. Sie hatten einander zu schätzen gelernt. Ijsdur hatte sein strategisches Geschick im Astz-Kartenspiel enthüllt und damit die Herzen von Barz und Aćh errungen. Iril und Barz hatten sich in tiefgründigen Gesprächen über ihre Spezialgebiete der Magie mitreißen lassen.

Und nun machten sie sich auf den Weg, Ijsdurs Dämonin zu überwinden.\bigskip







Barz verteilte die zerriebenen Rietgrasblüten auf dem gewaltigen Eisblock, der aus der Narne bis hin ins Fischerdorf ragte. Langsam schmolz das Gebilde und gab die Boote der ängstlich danebenstehenden Andori frei. Jubel brandete auf. Unsere Helden konnten allerdings nicht einstimmen.

„Das bringt so nichts“, murmelte Barz, „Wir sind schon im ganzen Land Siantaris Spuren gefolgt und haben unter großem Aufwand die Überreste ihrer niederregnenden Eisblitze beseitigt. Doch noch immer sind wir ihr kein bisschen näher als zuvor!“

Als wollte er seine Worte unterstützen, zuckte ein weiterer Eisblitz über den Himmel. Ein erst einige Zeit später ertönendes dumpfes Donnern verriet, dass er in weiter Ferne niedergegangen war. Ein weiteres Bauernhaus getroffen? Ein weiterer einsamer Wanderer im Eis erstarrt? Was es auch war, es würde sie wieder Zeit kosten, ohne mehr über die Position der Urheberin des Übels zu verraten.

„Danke für die exzellente Zusammenfassung, Barz“, sprach Iril, „.Wir müssen unsere Taktik ändern.“ Sie setzte sich entschlossen auf einen Stein am Wegesrand, packte ihre steinerne Runenscheibe aus und begann, daran zu werkeln.

„Ein Fernrohr zum Spähen wäre nicht schlecht“, meinte Aćh, „Nach dem, was Ijsdur erzählt hat, ist Siantari alles andere als unauffällig.“.

„Die Tulgori haben den zerstörten freien Markt zu ihrem Lager erklärt“, warf Ijsdur ein, „So geschickte Spiegel- und Brillenbauer sie auch sind, tragbare Fernrohre scheinen nicht Teil ihres Repertoires zu sein. Ich wüsste nicht, wo wir sonst in der Nähe welche kaufen könnten.“

Barz schlug sich an den Kopf: „Wie konnte ich das vergessen! Wir müssen nur mein ...“ Er griff an seinen Pulvergürtel und seufzte. „Natürlich. Ich habe meinen Vorrat an Umwandlungspulver bei Sabri gelassen.“

Er blickte den Weg durch das Rietgras zurück, der die Heldengruppe zum vereisten Fischerdorf geführt hatte.

„Vermutlich stapft sie brav unseren Fußstapfen nach. Oder stampft gerade einem andorischen Bauern quer durchs Feld und ich werde wieder dafür geradestehen müssen. Wie dem auch sei, ich mache mich besser daran, Sabri aufzuspüren. Ich werde uns mit dem Umwandlungspulver ein Fernrohr herbeizaubern. Und dann kehren wir auf den höchsten Turm der Rietburg zurück und sondieren die Umgebung. Ist jemand mit mir?“

Iril schüttelte ihren Kopf: „Ich bleibe lieber hier sitzen und kümmere mich um meine Runenscheibe. Wenn ich es schaffe, sie auf Eismagie anspringen zu lassen ...“

„Dann sollte ich lieber nicht zu nahe sein und das Ergebnis verfälschen“, sprach Ijsdur. „Ich werde dich begleiten, Freund Barz.“

„Danke dir, Freund Ijsdur!“, grinste Barz. Welch Animosität er einst gegenüber dem Eis-Dämon gehegt haben mochte, sie war geschmolzen wie Schnee in der Wüste, nachdem Ijsdur Barz‘ verlorene Ringkette zurückgegeben hatte.

„Turr und ich bleiben sonst bei Iril und sorgen dafür, dass sie vor lauter Konzentration nicht plötzlich einen anstürmenden Gor übersieht“, meinte Aćh.

Damit schienen alle einverstanden. Während die Fischer in ihr vom Schnee befreites Dorf zurückkehrten und die Helden allesamt zum Essen einluden, brachen Barz und Ijsdur Seite an Seite auf, zurück in Richtung Sabri.

„Passt gut aufeinander auf!“, riefen Iril und Aćh ihnen hinterher.

„Nicht wir sind es, die an einem Ort rasten, der bis soeben noch von einem Eisblitz bedeckt war.“

„Man sagt doch, Blitze schlügen nie zweimal am selben Ort ein“, gab Iril zurück.

„Eisblitze vielleicht schon. Und das Sprichwort stimmt ohnehin nicht. Genau dafür gibt es auf Thakkum ja Blitzfänger-Stangen“, protestierte Barz.

„Ihr fangt Blitze?“, fragte Ijsdur überrascht.

„Keine große Kunst. Du musst nur genug hoch in den Himmel vordringen und die Blitze springen dir von selbst entgegen.“

Still wanderten Steppennomade und Eis-Dämon Seite an Seite am Narnenufer entlang durch das weite Rietland. Dunkle Wolken warfen ihren Schatten auf die wenigen Bäume, die hier und da umherstanden.

„Was wäre denn größere Kunst? Dein Umwandlungspulver?“, fragte Ijsdur nach einer Weile.

„Ja, magische Pulver sind durchaus einiges komplexer als Blitzfänger-Stangen. Die richtige Mischung von Materialien zu finden ist aufwendig und fehleranfällig. Fehler, die überaus gefährlich sein können. Aber die Pulverschamanen der Iquar fanden schon einige stabilere Verbindungen, deren magischen Effekte man mit der entsprechenden Expertise ziemlich konsistent hervorrufen kann. Einige solche Pulver führe ich nun mit mir.“

„Könntest du mir eigentlich beibringen, sie zu nutzen?“

Barz schluckte schwer. „Könnte vielleicht schon, will aber nicht zwingend. Jedenfalls nicht so husch-husch. Es ist besser, wenn nur ich meine Pulver handhabe.“

Ijsdur blickte ihn schief an: „Wie kommst du denn darauf?“

„Ach, weißt du“, verwarf Barz seine Hände, „Ich schenkte Aćh einst eines meiner Pulver, einen Nixenstaub. Leider wusste ...“

„Nixenstaub? So was sah ich dich noch nie einsetzen.“

„Mein Vorrat ist ja auch alle. Scheinbar gibt es weder in Tulgor noch in Andor nennenswerte Nixendörfer.“

„Dafür aber in Silberland. Glaube ich. Wenn du willst, könntest du Iril mal danach ansprechen, sie erzählte mir jedenfalls einmal von der Nixe, die einst ihren Runenhammer schuf.“

„Naja, vielleicht ist es auch besser, wenn dieses Pulver nicht allzu verbreitet ist. Denn wie schon gesagt: Leider wusste Aćh natürlich nichts über die Feinheiten des Nixenstaub-Einsatzes. Insbesondere nicht, dass man es lieber nicht auf verletzte Gegner werfen sollte. Beinahe wären wir deswegen beide zu Eis-Dämonen geworden.“

„So schlimm ist das gar nicht“, meinte Ijsdur mit einem gezwungenen Lächeln.

„Für dich vielleicht nicht mehr. Siantari hätte uns genutzt, um die ganze Welt zu vereisen versuchen. Es war fahrlässig von mir. Ich werde meine Pulver nicht mehr so unvorsichtig mit anderen teilen. Wenn ich damit Fehler mache, liegt die Schuld wenigstens vollständig bei mir und ich ziehe nicht noch andere Leute rein.“

„Das ist die Lektion, die du daraus ziehst? Was wäre damit, Leuten, denen du Pulver schenkst, angemessen zu erklären, wie sie zu benutzen sind?“

„Ich konnte zu diesem Zeitpunkt vielleicht zehn, zwanzig tulgorische Wörter! Und das Pulver war auch mehr eine symbolische Geste. Weißt du, es war mein Fehler, der Turr im ewigen Eis festsitzen ließ, und die arme Aćh schluckte ihre Wut darüber und ihre Sorgen um Turr herunter und half mir, kümmerte sich um mich! Da musste ich doch irgendwie ausdrücken, dass ... ich meine, Worte allein haben nicht dieselbe Wirkung wie eine solche Geste, das verstehst du, oder?“

Ijsdur hob beschwichtigend seine Hände. „Durchaus. Dennoch finde ich, dass die Beschränkung deiner Pulver auf dich selbst nicht der optimalste Schluss ist, den du aus diesem Vorfall ziehen könntest. Du hast ja auch nicht aufgehört, mit neuen Pulvermischungen zu experimentieren, nur weil das Experiment mit Turr schieflief, oder?“

„Nein. Aber es lässt mich all meine künftigen experimentellen Thesen zweimal überdenken, insbesondere wenn andere Wesen involviert sind. Wir alle machen Fehler, aber wenn wir daraus lernen, machen wir sie in Zukunft seltener. Einfach keine Experimente mehr durchzuführen, um keine Fehler zu riskieren, ist auch nicht die glücklichste Lösung.“

„Genau! Siehst du, was ich meine?“

Barz blickte Ijsdur überrascht an, nickte dann aber.

„Also dann, magst du mir demnächst mal den Umgang mit einem deiner Pulver beibringen?“

Barz‘ Grinsen erstarb und er druckste weiter herum. „Es ist ein langwieriger Prozess, und kompliziert, und es darf auch wirklich nichts schiefgehen, also ...“

„Und es ist ein langwieriger Prozess, einen Menschen von etwas zu überzeugen, gegen das sich seine Schuldgefühle sträuben“, beendete Ijsdur die Konversation, „Belassen wir es fürs erste dabei? Ich freue mich schon darauf, in Zukunft weiter darüber zu diskutieren.“

Barz nickte und blieb eine Zeit lang mit seinen Gedanken allein.

„Schau, da vorne, ist das sie?“, fragte Ijsdur.

„Beim großen Seeadler, was hast du denn für Augen? Ich sehe nur Nebel!“

„Da vorne bei der Marktbrücke ruht sie, deine Echse, ich sehe sie immer deutlicher. Schläft tief und fest.“

Barz‘ Schritte wurden beschwingter.

„Die Marktbrücke ist einer meiner liebsten Orte in ganz Andor!“, sprach er.

Als er ansetzte, auf ihre wundervollen Besonderheiten einzugehen, protestierte Ijsdur: „In Tulgor gibt es so viel schönere und stabilere Brücken. Die Marktbrücke wirkt, als habe ein eifriger Bauherr so rasch wie möglich die Narne überwinden wollen und danach einfach zwei Zwergenstatuen auf jede Seite geklatscht. Du willst mir doch nicht etwas sagen, dass diese Brücke stabil sei?! Jeder halbgroße Krake sollte diese Brücke im Nu auseinandernehmen können.“

„Ich kenne mich nicht mit Stabilität aus, vertraue den Werken der Schildzwerge aber diesbezüglich durchaus. Und elegant ist die Brücke doch auf jeden Fall.“

Die beiden erreichten die Markbrücke. Unter ihnen rauschte die Narne. Zwei Zwergenstatuen bewachten die Marktbrücke. Daneben schnarchte Sabri gut vernehmlich.\bigskip







„Arrogscheiße!“, fluchte Iril auf, als der Eisenstift von der Runenscheibe abrutschte und ihre Hand aufschürfte.

„Was ist ein Arrog?“, fragte Aćh grinsend, „So langsam gewöhne ich mich an den Sprachtrank dieser Hexe, doch ein Arrog sagt mir noch nichts.“

„Stell dir eine gewaltige Klippe vor, aber lebendig. Mit muskulösen Armen, gepanzertem Rücken, einem riesigen Maul voller spitzer Zähne und ... ich bin mir nicht mal sicher, wie sie unter der Wasseroberfläche aussehen. Sag, gibt es in deiner Heimat Tulgor auch solch bösartige Kreaturen?“

„Gibt es faszinierende Kreaturen nicht überall? Oh, von welch fantastischen Wesen ich dir berichten könnte. Doch bösartig sind die wenigsten. Viele folgen nur ihren Instinkten. Auch der netteste Takuri könnte mit einem einzelnen Funken der Begeisterung die Rote Steppe niederbrennen.“

Iril schnaubte. „Takuri werden ja auch nicht von einer bösartigen Macht angetrieben.“

„Redest du vom Drachen, der die Kreaturen antrieb?“

„Ja. Oder Dunkle Magier und finstere Nekromanten, die anderen gerne ihren Willen aufzwingen. Oder Mächte des Meeres, die es nicht mögen, wenn man zu tief in den Ozean vordringt. Mir scheint, nur wenige Kreaturen sind wirklich frei.“

„Nun, in Tulgor gibt es weder Mächte des Meeres noch Nekromanten. Und die meisten unserer Magier leben zurückgezogen in hohen Türmen und kümmern sich wenig um die Wesen der Wildnis.“

Aćh streichelte Turrs Kopf sanft und blickte in die Ferne, wo die wolkenverhangenen Hänge des Fahlen Gebirges hoch aufragten. Wehmut lag in ihrem Blick.

Iril hörte auf, das Blut von ihrer verletzten Hand zu putzen, und fragte: „Vermisst du Tulgor sehr? Wie gut lebst du dich hier ein? Wann geht es für dich wieder zurück?“

Ein Moment der Stille herrschte, nur unterbrochen von Turrs fröhlichem Gurren. Dann antwortete Aćh leise: „Ich weiß nicht so recht. Ich bin eine Fremde in einem fremden Land hier. Ich wusste, worauf ich mich einließ, als ich die Reise hierhin antrat. Aber ich war nicht darauf vorbereitet, dass auch Barz seine eigene Geschichte hier hatte. Er und Nabib verbringen viel Zeit miteinander. Und Barz ist nicht nur meine einzige Verbindung zu den Andori, er war auch meine direkte Verbindung zu den tulgorischen Minenarbeitern und anderen Reisenden. Er arbeitete lange Zeit in den Mera-Stollen, nicht ich. Ich ... ich bin ziemlich allein hier.“

„Da sprichst du mir aus der Seele“, meinte Iril, „Auch ich fühlte mich als Fremde hier. Meine Verwandtschaft ... meine Schwester, von der ich nicht einmal wusste, dass es sie gab ... die Schildzwerge wollen mich nicht und die Andori natürlich auch nicht. Und auch in Silberhall würde ich nur stetig daran erinnert, dass Burmrit, meine Lehrmeisterin, nicht mehr unter uns weilt. Doch es wurde besser.“

„Wir können zusammenhalten. Wir beide“, meinte Aćh nun.

„Können wir. Wenn du hier in Andor bleiben willst. Du hast, wenn ich es richtig verstanden habe, im Gegensatz zu mir noch eine Familie auf der anderen Seite der Berge. Du kannst noch dorthin zurückkehren, wenn dir das Rietgras zu den Ohren raushängt.“

„Das ist ja mal ein eigenartiges Sprachbild“, schmunzelte Aćh, „Ja, ich habe in Tulgor noch eine Familie. Verwandte, Bekannte, Freunde. Vielleicht kehre ich eines Tages dorthin zurück. Falls du dich bis dahin nirgendwo eingelebt hast, könntest du da vielleicht auch mitkommen. Wenn du willst. Ich könnte dir den Nistbaum und die weite Steppe zeigen, statt nur darüber zu reden.“

„Vielleicht. Wo immer man unsere Hilfe brauchen kann“, nickte Iril träumerisch. „Und sofern wir uns bis dahin noch ausstehen können.“

„So, wie ich dich in den letzten Wochen kennengelernt habe, mache ich mir da keine Sorgen. Aber für den Moment bleibe ich ohnehin gerne hier. Andor ist ein konfliktreiches Land. Die Leute hier brauchen nach dem Wüten des Drachen mehr Hilfe als die Tulgori. Die Sprache ist ein Hindernis, aber abgesehen davon gefällt es mir ungemein hier. Und Leuten in Not zu helfen, wirklich zu helfen, fühlt sich einfach gut an.“

„Es wäre sogar gut, Leuten zu helfen, auch wenn es sich nicht gut anfühlte. Das gute Gefühl ist nur ein Bonus.“

„Genau. Na los, Iril, lass uns Eis-Dämonen jagen.“ Mit Blick in den Süden, wo Ijsdur davonstapfte, fügte Aćh an: „Nur gefährliche natürlich.“

Iril runzelte ihre Stirn und kehrte zurück zur Arbeit auf ihrer Runenscheibe.

Eine Viertelstunde später war sie soweit. Iril hob ihren Runenhammer in die Höhe und murmelte etwas vor sich. Die in die uralte Waffe eingeritzten Runen glommen grünlich auf.

Interessiert blickte Aćh zu. Noch immer hatte sie keine Ahnung, woher dieses Artefakt genau seine Kraft bezog.

Iril ließ den Hammer auf die Scheibe niederfahren. Ein blechernes Klingen ertönte. Doch das magische Glühen sprang nicht vom Hammer auf die Scheibe über. Vielmehr verstärkte sich das Leuchten des Hammers abrupt. Die Scheibe glitt zu Boden und rollte nutzlos ins Rietgras.

Grünliche Schwaden steigen von Irils leuchtenden Tattoos auf. Aćh fiel es auf einmal schwer zu glauben, dass die Runentattoos nur aus unter der Haut bugsierten Farbteilchen bestanden. Wie kleine Lebewesen, die sich verselbstständigten, wanden die leuchtenden Runen sich. Iril schrie auf und klappte zusammen. Ihr ganzer Körper zitterte.

Aćh fiel neben Iril aufs Knie und streckte ihre Hand aus. Zwischen klappernden Zähnen stieß Iril hervor: „Nein! Fass .... fass mich nicht ... du bist ... Mensch, du bist ... Dunkle Magie ... nicht gewachsen ... Stimme ... verlockend ...“

„Was soll ich tun?“

Irils Körper lag flach wie ein Brett am Boden und wurde von Krämpfen geschüttelt. Eine erschreckend lange Zeit antwortete sie nicht und atmete flach, dann würgte sie hervor: „Geht ... vorbei. Keine Sorge, das ... nicht tragisch.“

Aćh war nicht überzeugt. Besorgte Flammenbäusche überzogen auch Turrs Gefieder. Dann flüsterte Iril: „Bleibe ... bei mir ... bitte.“

Das Schütteln ließ langsam nach. Irils verkrampfte Hand öffnete sich und ließ den schweren Runenhammer zu Boden plumpsen. Das grüne Glühen verlosch. Iril blieb noch eine Weile liegen und blinzelte schwach.

„Guck bitte ... nicht ... einschlafe“, hauchte sie erschöpft. Aćh war sofort wieder auf ihren Beinen und achtete darauf, dass Iril ihre Augen offenhielt. Nach einer Weile hatte sich der Atem der Zwergin wieder normalisiert. Iril streckte ihre Hand aus und die sitzende Aćh half ihr auf die Beine.

„Alles wieder klar, Runenmeisterin?“

„Alles klar, Takuri-Hüterin“, sprach Iril schwach.

„Bei den sieben Feuern des Himmels, was war das?“

„Kompliziert“, murmelte Iril, „Die Runen können Kraft leiten, in Dinge hinein oder hinaus. So was wie vorhin sollte nicht passieren, wenn man genügend vorsichtig ist beim Runenzeichnen. Dieser Hammer ... Die meisten Runen werden vom Licht des Mondes oder der Sonne gespeist. Dieser Runenhammer ist sehr nützlich, um Runen unabhängig von der Tageszeit oder dem Mondzyklus zu aktivieren. Aber er ist immer noch ein von Dunkler Magie erfülltes Artefakt. Und diese fordert meistens ihren Tribut. Die Runen auf dem Hammer erlauben uns, den größten Teil dieser Last nicht selbst tragen zu müssen. Ja, ohne diese Runen könnte auch ich die Magie in seinem Innern nicht nutzen. Aber so mächtig die Runen auch sind, sind sie selbst für die größten Meister schwer zu bändigen. Manchmal scheint es mir, als verselbstständigten sie sich langsam. Und dann zehren sie an meiner Willenskraft. Und selten, ganz selten, bricht die Dunkle Magie völlig aus dem Hammer aus und überzieht die Runen auf meinem Körper. Die saugen mir dann wie kleine Egel die Energie aus. Diese Vorfälle sind unregelmäßig, unberechenbar, meistens ganz unscheinbar, aber hin und wieder einfach unausstehlich. Ich wünschte, ich hätte einige Runensteine bei mir. Darin lässt sich die bei solchen Anfällen durchströmende unvorstellbare Energie speichern und produktiv nutzen. Du hättest nicht zufälligerweise welche bei dir gehabt?“

„Runensteine?“, fragte Aćh, „Ich bin mir nicht sicher, ob ich das Wort verstehe. Ich kenne Edelsteine und Mera-Steine, doch ein Runenstein ist mir unbekannt. Ist das einfach ein Fels, in den Runen gemeißelt worden waren? Die Temm waren Meister darin, solche Gänge zu bauen.“

„Na, ein bisschen mehr als bekritzelte Steine sind Runensteine schon“, meinte Iril, „Aber im Großen und Ganzen hast du recht. Bevor das Geheimnis ihrer Erschaffung verloren ging, schufen die Runenmeister der Schildzwerge vor Urzeiten Unmassen von Runensteinen. Auch heute noch kann man in den Minen und selbst hier draußen im Land welche finden. Leider trifft man sie kaum auf den Inseln des Nordens an, sonst hätte ich schon längst eine ganze Sammlung davon angelegt. Die Kraft der Runensteine kann sehr vielseitig nützlich sein und auch ohne magische Kräfte genutzt werden. Auch wenn sie leider primär zu Kriegszwecken eingesetzt wurden. Gegen die Trolle, gegen die Drachen, sogar in Kämpfen von Zwerg gegen Zwerg. Wie viel potenzielles Wissen wohl verloren gegangen ist, weil wir uns gegenseitig abschlachteten, statt zusammenzuarbeiten? Ohnehin waren die Runenmeister der Urzeiten geschickter als wir es heute sind. Sie schufen ein weit verbreitetes Netzwerk aus Gängen, welche durch magische Zwergentüren verbunden waren, die die Gegensätze von Feuerrunen und Wasser vereinten. Und dann erst die legendären unterirdischen Runengänge in unbekannte, fremdartige Reiche und Gefilde, die angeblich nur dann, wenn der Rote Mond hoch oben am Himmel steht, sichtbar werden und etwaige Durchquerer eine Zeit lang mit einer raffinierteren Übersetzungsrunen versehen sollen – so wie die, die ich Ijsdur anhängte – ach Aćh, so viele Kenntnisse der vergangenen Runenmeister sind verloren gegangen. Geheimnisse aus der Vergangenheit, die wir in harter Arbeit aufs Neue ergründen. Ich komme glatt ins Schwärmen.“

„Es freut mich vor allem, dass es dir wieder besser geht“, schmunzelte Aćh, „Eklige Egel, diese Runen. Doch ich werde daran denken, falls ich einen Runenstein sehen sollte. Wobei, solche massiven Felsen sind bestimmt schwer zu transportieren.“

„Massive Felsen? Wo denkst du hin, Runensteine sind teils kaum größer als Kiesel! Die passen bequem in deine Reisetasche. Glaube mir, von Größe auf magische Macht zu schließen, kann trügerisch sein.“

„Dem ist wohl so!“, erklang Barz‘ fröhliche Stimme hinter ihnen, „Sabri hat etwa noch keinen Funken magisches Talent gezeigt.“

„Ihr seid zurück!“, rief Aćh auf, denn Ijsdur und Barz waren zurückgekehrt.

„Unser Vorhaben war von Misserfolg gekrönt“, berichtete Iril bedrückt.

„Na, dann ist doch umso besser, dass wir etwas herausgefunden haben“, sagte Barz. Stolz präsentierte er ein Fernrohr, welches golden glitzerte. Wohl frisch umgewandelt.

„Ich blinder Büffel hätte sie glatt übersehen, doch Ijsdur hat eine Spur von Santari gefunden. Die wird euch jedoch nicht gefallen.“

Barz übergab das Fernrohr an Aćh und sprach: „Sieh, dort drüben. Am Alten Wehrturm. Dort, wo Tarok fiel.“

„Was in aller Welt ist das?!“, rief Aćh aus.

„Was? Was siehst du?“, fragte Iril ungeduldig.

„Ein Eis-Drache“, sprach Ijsdur tonlos. „Bei der Ruine des alten Wehrturms wütet ein Wirbelsturm der Kälte, wie an so vielen Orten im Reich. Und darin ruht ein gewaltiges Wesen aus Eis und Schnee. Ich hoffte, es möge nur eine Formation sein. Doch sah ich, wie es sich bewegte. Vier Beine, zwei Flügel und ein stacheliger Kopf auf einem Schlangenhals.“

„Ist es Siantari gelungen, einen Eis-Drachen-Dämon zu beschwören?“, fragte Iril.

„Das sollte sie nicht können!“, sagte Ijsdur, „Siantari ist mächtig, aber sie kann nicht aus dem Nichts Leben formen.“

„Dann bleibt uns nur etwas übrig. Auf, zum alten Wehrturm!“\bigskip







„Ich bin mir nicht sicher, ob ich das kann“, murmelte Aćh. „Gegen diese Kultisten vorgehen. Ich habe bislang nur ein Leben genommen. Diese Eis-Dämonin oben im Ewigen Eis, die mich und Barz zu einer der ihren machen wollte.“

„Das kommt schon“, sprach Barz, „Ich habe schon in der Steppe Banditen bekämpft. Der Tod ist Teil des Lebens.“

„Macht es nicht leichter“, sprach Ijsdur, „Ijs ist für den Tod seiner Freunde verantwortlich.“

Niemand wusste, was darauf zu sagen war.

„Ich habe noch nie getötet“, meinte Iril, „Nur Kreaturen erledigt. Die zählen ja kaum als wertvolle Lebewesen.“

Barz widersprach: „Denken und fühlen können sie doch auch.“

„Aber bitte, ein Gor ist noch instinktgesteuerter als ein hungriger Wolf. Und es besteht doch ein himmelweiter Unterschied darin, ob du einen anderen Barbaren oder eine Echse röstetest.“

„Beim Großen Affen, warum würde ich Sabri je rösten?!“

„Verzeih, ich nahm an, dass sie am Ende ihres Lebens ...“

„Ist das, was ihr in Silberhall tut? Am Ende eures Lebens von Gefährten verspeist werden?“

„In Silberhall geben wir uns größtenteils mit den Erträgen des Meeres als Speise zufrieden. Algen und Seetang lassen sich überraschend gut züchten. Wir hatten keine Ahnung, als wir von Cavern dorthin reisten. Doch die Nixen lehrten uns.“

„Die Nixen lehrten auch uns so einiges, als unser Stamm sich im großen See Ava niederließ.“

Das Gespräch zog sich ähnlich weiter, während die Helden weiter durchs Rietland schritten.

Sie hatten kaum einen Drittel der Strecke zum Alten Wehrturm überwunden, da hieß Ijsdur sie an, innezuhalten und das Fernrohr wieder hervorzuholen. Etwas bewege sich.

Tatsächlich! Der gewaltige Eis-Drache am Wehrturm hatte seine Schwingen ausgebreitet und war darauf und daran, sich in die Höhe zu schwingen!

„Ich erkenne eine dunkle Gestalt, die neben ihm schwebt. Der Schwarze Herold?“, berichtete Iril, durch die verschwommenen Linsen blinzelnd.

„Ich vermute es stark. Doch wird er kaum allein sein.“, gab Ijsdur zurück. „Darf ich kurz das Fernrohr ... danke sehr! Ja, da reiten gleich eine ganze Gruppe grimmiger Gestalten auf dem Eis-Drachen. Allesamt eingewickelt in schwere Decken und Tücher. Über ein kompliziertes Geflecht aus Gurten an den Drachen geschnallt. Das wird bestimmt einige Zeit gekostet haben, diese herzustellen.“

Iril staunte wieder einmal über den scharfen Blick des Eis-Dämons und fragte sich, wie seine eisigen Augen funktionierten.

„Erkennst du die Personen?“, fragte Barz, der sich nicht von solcherlei Überlegungen ablenken ließ.

Ijsdur bejahte. „Die eine Rüstung würde ich im Schlaf wiedererkennen.“

„Lass mich raten: Sagramak, die Schamanin der Drachenkultisten?“

„Genau, da ist Sagramak. Aber da sind noch mehr. Dieser Nehamal, und das Ziegenwesen Fir, das die Knochen einst stahl. Und einige anderer ihrer Sippe, allesamt bewaffnet. Sie alle haben sich auf den Rücken des Eis-Drachen geschnallt. Sie wollen irgendwo hin.“

„Das ist kein großes Rätsel. Die wollen zur Rietburg, die Knochenkette zurückfordern“, sprach Aćh, „Warum haben wir Taroks Knochen schon wieder dem Prinzen überlassen?“

Iril brummelte: „Weil wir dachten, dass die Rietburg irgendwie sicher wäre. Hätte ja keiner geglaubt, dass die einen Eis-Drachen herbeirufen können.“

„Das sollten sie auch nicht können, sonst hätten sie das auch viel früher getan.“

„Siantari war das aber auch nicht“, sagte Ijsdur, „Wenn sie von selbst ein derart mächtiges Leben schaffen könnte, wäre das Felsentor zum Tal des ewigen Eises schon viel früher gebrochen worden.“

„Aber sie kann Eiskristallketten schaffen!“, erinnerte sich Barz, „Sagramak trug doch Knochensplitter um den Hals! Knochen von Sagrak dem Drachen. Was würde geschehen, wenn man diese mit einer Eiskristallkette verbinden würde?“

Ijsdur blieb stocksteif stehen: „Ich weiß, was geschehen würde, wenn man einen gut erhaltenen Drachenleichnam mit einer Eiskristallkette verbände. Aber wenige Knochen, uralt und verrottet?“

„Es sind nun mal magisch potente Mittel“, knurrte Iril, „Seht ihr nun, warum ich ihr nicht Taroks Knochen überlassen wollte? Irgendwie haben sich die Drachenkultisten mit Siantari verbündet. Und irgendwie haben sie einen Eis-Sagrak herbeigerufen.“

„Sagrakdur“, flüsterte Ijsdur.

Lautes Donnern übertönte seine Stimme und erfüllte die kalte Luft. Unvermittelt begann die Erde zu beben. Die Helden wandten sich um und sahen in der Ferne, am alten Wehrturm, riesige Wolken aus Schnee und Staub aufwirbeln. Dann erhob sich ein dunkler Schatten am Horizont.

Sagrakdur war gewaltig. Er breitete seine Flügel aus, stieß in den grauen Himmel empor, und in einem weiten Bogen überflog er langsam das schneebedeckte Land.

Unzweifelhaft näherte sich das Biest der Rietburg. Doch der Wirbelsturm der Kälte über dem alten Wehrturm blieb bestehen.

„Siantari ist nicht mit den Drachenkultisten mitgezogen. Sie lauert noch am alten Wehrturm“, zischte Ijsdur.

„Falls wir sie erledigten, würde das auch den Eis-Drachen vom Himmel holen?“, fragte Aćh.

„Keine Ahnung. Aber wenn die Ermordung Siantaris den Eis-Drachen vernichtete, dann auch mich“, antwortete Ijsdur. „Ich präferierte eine längere Existenz.“

„Wir könnten zur Rietburg gehen und uns den Eis-Drachen selbst vorknöpfen“, meinte Iril.

„Oder wir reden mit Sagramak“, sprach Barz, „Sie ist eine vernünftige Person. Wenn sie will.“

Ijsdur meldete sich wieder zu Wort: „Es könnte von Vorteil sein, Siantari nur zu überwältigen und die Drachenkultisten vor ein Ultimatum zu stellen. Ihr Eis-Drache sollte sich wie alle Eis-Dämonen Siantaris Willen beugen müssen. Schließlich trägt er anders als ich keine beschützende Runenscheibe im Hals. Und wenn wir erst einmal Siantari in der Hand hätten ...“

Da sprach sich Iril dagegen aus: „Der alte Wehrturm und die Rietburg sind etwa gleich weit entfernt. Was, wenn wir uns entschließen, Siantari aufzuspüren, sie überwältigten und dann klar würde, dass sie gar keine Macht mehr über den Eis-Drachen hat? Dann hätten wir wertvolle Zeit vergeudet, in der dieser Eis-Drache was weiß ich für bösartige Bosheiten vollbringen könnte.“

Aćh konterte: „Und was, wenn wir zur Rietburg gehen, den Eis-Drachen schmelzen und uns erst danach um Siantari kümmern können? Jetzt wissen wir, wo sie sich befindet. Bis dahin könnte sie schon wieder in alle Welt geflohen sein.“

„Und einen finsteren Plan in Tag umgesetzt haben“, warf Ijsdur ein, „Ich befürchte, ihr seid zu optimistisch, was unsere Gewinnchancen angeht. Siantari ist eine formidable Gegnerin. Unterschätzt sie nicht. Und die Stärke eines Eis-Dämons im Körper eines Drachen jagt selbst mir einen Schauer über den Rücken.“

In der Ferne erklang weiteres raues Gebrüll. Sagrakdur hatte seinen Mund geöffnet und einen bläulich-weißen Strahl der Kälte und des Schnees aufs tief unter ihm liegende Rietland gespuckt.

„Gütige Mutter!“, rief Ijsdur aus.

„Seit wann glaubst du denn an die Mutter?“, fragte Iril.

„Tu ich nicht, diese Redewendung habe ich mir von dir abgeguckt.“

„So schlimm ist die Lage gar nicht“, meinte Iril, „Im Vergleich mit Tarok ist dieser Eis-Drache hier geradezu winzig.“

„Aber die Helden, die Tarok besiegten, sind im gesamten Lande verstreut und haben ihre eigenen potenziell königreich-endenden Gefahren zu bekämpfen.“

„Dann bleibt diese ganze Sache wohl an uns hängen. Wir müssen uns aufteilen. Ich verstehe nichts von Eis-Dämonen, aber so einiges von Drachen. Ich biete mich an, zur Rietburg zu ziehen und mich dem Eis-Drachen zu stellen. Siantari würde ohnehin gar nicht verstehen, was ich ihr für Beleidigungen an den Kopf werfen würde.“

„Ich komme mit dir!“, rief Barz, „Ich bin froh, wenn ich meiner Lebtag keiner Eiskristallkette mehr nahekommen muss – Anwesende ausgenommen. Und ich will eine friedliche Lösung mit den Drachenkultisten nicht aufgeben. Auf mich wird Sagramak am ehesten hören. Ohne dir nahe treten zu wollen, Iril, könnte es dafür sogar besser sein, wenn ich allein dorthin ziehen würde.“

„Willst du von Sagrakdur in einem einzigen Angriff zu einem Eisblock reduziert werden?“, fragte Iril schnippisch. „Meine Runenmagie könnte das einzige effiziente Mittel gegen ihn sein.“

Aćh räusperte sich und gestikulierte zum brennenden Feuervogel auf ihrer Schulter. „Das einzige Mittel?! Mein feuriger Turr könnte aus Sagrakdur bestimmt eine große Pfütze machen.“

„Wenn er nicht zuerst in einen Eiswürfel verwandelt wird“, murmelte Ijsdur, „Ich sähe ihn lieber auf unserer Seite.“

Aćh blickte ihn fragend an. „Unsere Seite?“

„Nun, es scheint passend, dass ich mich Siantari stellte“, erklärte Ijsdur, „Und ich wäre lieber nicht allein dabei. Irils Runen können bestimmt etwas ausrichten gegen einen Dur, aber vielleicht nicht gegen die Tari selbst. Und Hitze schmilzt kleinere Mengen Schnee erheblich schneller als große. Dass Siantari signifikant kleiner ist selbst der kleinste Drache, muss ich dir wohl nicht sagen. Alles in allem ...“

„Klingt nach einem Plan!“ Barz klatschte in seine Hände. „Ist das gut so? Aćh, kümmerst du dich mit Ijsdur und Turr um Siantari?“

Aćh nickte stumm, sah aber alles andere als glücklich aus.

Iril setzte sich prompt wieder an ihre Runenschreibe und kritzelte darauf herum. Ijsdur beobachtete sie stumm. Barz hingegen starrte Aćh an, welche an ihrer Unterlippe kaute. Rasch flüsterte er ihr etwas zu. Iril vernahm Floskeln der Entschuldigung und der Nachfrage. Wohl fragte er, ob sie wirklich einverstanden damit war, sich an Ijsdurs Seite einer Eis-Dämonin entgegenzustellen. Er hatte das Erlebnis mit Nesdora nicht vergessen.

„Fećht!“, fluchte Aćh. Sie stieß einige Sätze auf Tulgorisch aus, ganz leise und schnell. Iril hätte sie nicht einmal gut verstehen können, wenn sie ihre Übersetzungsrune aktiviert hätte. Barz hingegen verstand und machte große Augen. Seine Mundwinkel zuckten, als müsse er sich ein Grinsen verkneifen. Dann nickte er aber ernst und hörte weiter zu. Schließlich erklärte er etwas davon, dass sein Bannpulver ohnehin nicht gegen Siantari wirken könnte. Aćh unterbrach ihn erneut. Doch schließlich nickten die beiden, umarmten einander und gingen dann zu ihren jeweiligen Kameraden für die kommenden Aufgaben.

Aćh hatte nichts mehr einzupacken und brach in Richtung des alten Wehrturms auf. An ihrer Seite folgte Ijsdur, der ohnehin nie etwas einzupacken hatte. Über den beiden drehte ein fröhlicher Turr seine Runden.

Iril studierte in ihren Notizen die genaue Runenfolge, welche in Silberhall so effektiv Geister vertreiben konnte. Eine Variation davon hatte Ijsdur von Siantari befreit. Eine andere Variation hätte ihn in der Eiskristallkette einschließen können. Das hätte sie für den Eis-Drachen gerne.

Iril verschloss ihre Reisetasche und erblickte, wie Barz seinen Pulvergürtel – an dem soeben ein türkises, ein braunes und ein blaues Säcklein hingen – lange betrachtete und zu überlegen schien, ob er lieber noch einen weiteren Abstecher zu Sabri machen und seine Pulver umverteilen sollte. Dann aber zuckte er mit den Schultern und wandte sich Iril zu.

„So, dann machen wir zwei Hübschen uns auf zur Rietburg, oder?“, rief Barz. „Was ist unser Plan?“

„Was meinst du, für was unsere Zeit reicht?! Rechtzeitig ankommen und draufhauen, das ist der Plan.“, rief Iril.

„Reden!“, murmelte Barz, während er seine Schuhe enger band. „Sagramak hat meines Wissens noch niemanden umgebracht, der mit ihr verhandeln wollte. Außer ... außer diesem einen Händler, der ihr die ganze Zeit ... na, das ist eine Geschichte für ein anderes Mal.“

Das konnte ja heiter werden.\bigskip







Vor der Rietburg schwebte der Schwarze Herold bedrohlich auf einer kleinen rauchigen Sturmwolke. Das Ewige Feuer hatte sich violett verfärbt. Der lila Feuerschein flackerte um seine dunkle Silhouette und spiegelte sich in der gezackten Maske, die er auf dem Gesicht trug. Die Zeit war nah. Die meisten Helden von Andor waren in Scharmützel mit allen möglichen Gegnern verstrickt. Jetzt war der Zeitpunkt, seinen Meister zurückzubringen. Er, der Schwarze Herold, war der Antreiber der Kreaturen und der Vorbote des Feuers, und nun auch der Vorbote des Eises. Wo er auftauchte, verloren die guten Menschen von Andor den letzten Mut.

Der Herold begann zu sprechen. Er flüsterte, und doch hörte jeder in Andor seine Stimme:

„Ich verkündige die Ankunft des mächtigen, des gewaltigen Sagrakdur. Vernichtet wurde sein Körper vor Jahrhunderten. Nun kehrt er zurück und holt sich, was der Letzte seiner Art nicht konnte. Die Archive vom Baum der Lieder sollen zerbröseln, wenn mein neuer Meister seinen eisigen Atem über den Wachsamen Wald streifen lässt. Die Knochen der Zwergenbrut sollen zersplittern unter der Kälte meines Gebieters. Und auch Brandurs Sippe soll nicht verschont werden.“

Alarmglocken wurden geläutet. Rufe erklangen. Menschen eilten von Turm zu Turm, von Tür zu Tür. Die Nachricht verbreitete sich in Windeseile. Ein Drache schoss auf die Rietburg zu, schon zum zweiten Mal in diesem Jahr. Und diesmal standen keine Helden von Andor bereit, um sich ihm entgegenzustellen.

Am Ausguck wurde berichtet, dass es sich bei diesem anfliegenden Drachen augenscheinlich um ein magisches Wesen aus purem Eis und Schnee handelte. Einige Bewohner der Burg verkrochen sich in ihren Kellern. Viele flohen ins offene Rietland. Die Rietgarde rüstete sich für den kommenden Kampf.

Und über all dem schwebte der Schwarze Herold. Er genoss den Aufruhr.

Hinter ihm kündigte das Rauschen großer Schwingen die Ankunft des Eis-Drachen an. Der Herold blickte ihm stolz entgegen. Ein Meisterstück der Magie. Sein Meisterstück. Er hatte Siantari und Sagramak auf einen gemeinsamen Pfad geführt. Nun war es an seiner Zeit, zu glänzen.

Sagrakdurs Kopf schwenkte auf seinem langen Schlangenhals nach unten. Seine durchscheinende Zunge züngelte. Sah er einige fliehende Andori, welche sich in Ermangelung angreifender Kreaturen in alle Himmelsrichtungen zerstreuten? Kurz schien der Eis-Drache zu bedenken, die Fliehenden angreifen zu wollen, dann aber schüttelte er seinen Kopf und hielt direkt auf die Burg zu. Pfeile der verteidigenden Rietgarde gruben sich in seinen Bauch, doch kümmerten diese ihn nicht einmal. Bolzen von der großen Balliste der Rietburg hätten ihm wohl geschadet, doch diesen wich Sagrakdur gekonnt aus.

Ein bläulich-weißer Strahl der Kälte und des Schnees ging aus Sagrakdurs Mund auf die Balliste nieder. Sie und die Kriegerin, die sie betätigt hatte, wurden unter einer dicken Schicht aus Eis und Schnee bedeckt. Jubel ertönte von den zahlreichen Gestalten auf dem Rücken des Eis-Drachen.

„Jeder, der sich ergibt, wird verschont!“, verkündete Sagramaks schallende Stimme. „Verlasst das Gemäuer und gebt uns den Weg zum Prinzen frei!“

Ein, zwei Menschen desertierten und rannten ins Rietland hinweg. Und die Drachenkultisten schienen sehr stolz darauf, was für gute Menschen sie doch waren.

Die Kultisten. Der Schwarze Herold lachte innerlich beim Gedanken an diese selbstgerechten Knochendiebe. Sie waren nützliche Spielsteine für seine nächsten Züge. Und manche von ihnen teilten seine Überzeugungen und Ziele gar. Aber nicht alle. Viele ersuchten die Macht der Drachen, um ihre Sippen zu stärken und sich ein gutes Leben zu verschaffen. Sie waren keine wahren Diener der Drachen. Die wahren Diener, das waren der Herold und die von ihm angetriebenen Kreaturen. Und mit seiner Hilfe würden sie bald wieder einen Meister haben.

Sagramak spie weitere Eisstrahlen, bis alle Krieger des Königs sich in ihre steinernen Türme zurückgezogen oder ihr Leben mit einem eiskalten Atemzug ausgehaucht hatten.

Schwer landete der Eis-Drache inmitten der Rietburg, auf einem mit Rietgras besetzten Dach, das unter ihm nachgab. Stab und Dreck stob auf. Stolz stolziert er zu einem freien Fleck, drehte sich zur Seite und entließ die Drachenkultisten von seinem Rücken. Als allererste landete die selbstbewusste Sagramak auf dem festen Boden der Rietburg.

Sie breitete ihre Arme aus – welche einen Schild und einen zeremoniellen Speer führten – und zog tief Luft ein. „So riecht es also in der Rietburg“, murmelte sie, „Stinkt mehr, als ich erwartet hätte. Danke sehr für die feurige Ansprache, werter Herold. Nun denn, wenden wir uns der Sippe Brandurs zu. Wo ist euer Prinz? Wo hält er die Drachenknochen versteckt? Bringt uns Thorald!“\bigskip







Aćh und Ijsdur zogen gemeinsam durchs Rietland. In der Ferne war bereits der alte Wehrturm erkennbar. Weiterhin wirbelte ein Sturm von Schneeflocken und Eisschwaden um ihn herum. Er war stärker geworden. Inzwischen war es beinahe unmöglich, die Ruine des Turms zu erkennen.

Schnell stapfte Aćh voran. Ijsdur, der wie üblich eine Linie aus Schnee und Eis hinter sich herzog, blieb immer mehr zurück. Als sie sich auf einer Hügelkuppe umdrehte und erblickte, wie Ijsdur ein halbes Feld hinter ihr zurückgeblieben war, blieb sie stehen. Davonlaufen konnte und sollte sie ihm nicht.

Stumm holte Ijsdur zu Aćh auf und lief einige Schritte neben ihr her. Sie hoffte, dass er einfach stumm bliebe. Aber natürlich tat er das nicht.

„Das wäre wir uns damals im Hängeschiff nicht im Traum eingefallen, dass wir ein paar Jahre später in einem fremden Land jenseits des Kuolema-Gebirges Seite an Seite kämpfen würden, oder?“

Aćh antwortete nicht.

„Tut mir leid, dass du dich mit mir abfinden musst, Aćh. Ich sehe, dass dir meine Gegenwart nicht behagt. Ich befürchte, dass es für das Wohl des Landes sein muss.“

Stille.

„Ich bin nicht wie diese andere Eis-Dämonin, die du getroffen hast. Ich bin meine eigene Person. Du hast keinen rationalen Grund, mich derart nicht zu mögen.“

War das nun eine Anschuldigung? Schuldgefühle machten sich in Aćh breit. Ihre Familie hatte sie erzogen, möglichst höflich und doch offen zu sein.

„Es ist nicht so, dass ich dich nicht mag. Es ist nur so, dass ich für Feuer stehe und du für Eis. Das harmoniert einfach nicht.“

„Feuer und Wasser werden oft als Gegensätze beschrieben. Mit dieser Hüterin der Flusslande und ihrem Wassergeist schienst du dich dennoch bestens zu verstehen.“

„Tja, bei ihr fühle ich mich sicherer. Ich habe schon schlechte Erfahrungen mit Eiskristallketten gemacht. Und wenn ich allein mit dir unterwegs bin, müsstest du nicht einmal eine passende Gelegenheit abwarten, um mich zu überwältigen und dem Gefolge Siantaris anzuschließen.“

„Meine Eiskristallkette funktioniert nicht mehr so. Und du hast ja Turr. Wenn ich dir gefährlich werden sollte, kann er mich in Grund und Boden schmelzen“, grinste Ijsdur.

Ein gezwungenes Lächeln spannte sich über Aćhs Gesicht.

„Und wenn ich etwas Übles tun sollte, nähmen Iril und Barz bestimmt meine Verfolgung auf und würde mir mit ihrem mächtigen Hammer den Garaus machen.“

„Da fühle ich mich doch schon ganz sicher.“

„Bin mir nicht sicher, ob du das ironisch meinst, aber das sollte es wirklich. Vielleicht vertraust du eher auf meinen Selbsterhaltungstrieb als auf meine Freundlichkeit. Was auch immer dich des Nachts ruhiger schlafen lässt.“

„Ich werde ruhig schlafen können, wenn Siantari dieses Reich verlassen hat.“

Stumm bewegten die beiden Helden sich weiter in Richtung des alten Wehrturms.

Diesmal war es Aćh, die die Stille brach.

„Sag mal, Ijsdur: Du hast deinen Körper schon quasi aus dem Nichts wieder zusammengesetzt. Kannst du dich nicht auch einfach in einen Eis-Drachen verwandeln? Auch ein sehr kleiner Drache würde schon immens helfen.“

„Verzeih mir, Aćh“, ließ Ijsdur seinen Kopf hängen, „Aber das ist nicht so, wie es funktioniert. Ich habe schon versucht, Anpassungen an meinem Körper vorzunehmen. Mental und physisch. Einmal habe ich sogar buchstäblich Wasser aus einem Brunnen genommen und an mich gepatscht, auf dass es in der gewünschten Form dort gefrieren könne. Nicht einmal Stacheln konnte ich mir geben.“

„Zu schade.“

„Vermutlich hat es mit meinem Selbstverständnis zu tun. Von irgendwoher muss der Schnee und der Eis in mir ja wissen, welche Form er zu haben hat – und nicht von meinem Leibe, denn dieser war erheblich weniger muskulös, hatte keine spitzen Ohren und kein Geweih. Das scheint sich auch auf Waffen zu beziehen. Ich kann mir ein magisches Eisschwert rufen, aber keine ausgefeilte Arcuballiste. Am Ende fühle ich mich wohl einfach wie ein Schwertkämpfer und kein Fernkämpfer. Leider scheint es, dass ich kein Eis-Drache sein werde, solange ich mich nicht als Eis-Drache sehe.“

„Zu schade“, wiederholte Aćh.

Warum hatte sie nicht mit Barz und Iril zur Rietburg mitreisen können? Warum hatte man sie hier mit dem elenden Eis-Dämon auf der Jagd nach Siantari zurückgelassen? Eigentlich wusste sie die Antworten auf diese Fragen. Aber dies bedeutete nicht, dass es ihr gefallen musste.

Ijsdurs Anwesenheit ließ sie weiterhin frösteln, und nicht nur wegen der Kälte, die stets von ihm ausging.

„Siantari ist eine mächtige Dämonin. Wir werden sie nicht mit schierer Stärke überwältigen können“, gab Ijsdur zu bedenken.

„Was ist mit Feuerbällen eines Feuertakuri?“

„Ich weiß es nicht. Ich würde lieber nicht unsere Zukunft darauf wetten.“

Aćh kaute an ihrer Unterlippe. „Was wäre mit einem feurigen Seil? Ich habe noch einige Phiolen Takuri-Asche bei mir. Eine Dose Sufarsaft, die einen vor Hitze und Kälte gleichermaßen beschützt. Und ein reißfestes Seil der Meraminenarbeiter. Wir könnten das Seil mit Saft und Asche behandeln. Es würde sich bei Kontakt entzünden, selbst aber unversehrt bleiben. Wenn man damit keine Eis-Dämonin fesseln kann, esse ich einen Besen.“

„Was bin ich froh, dich an meiner Seite zu haben!“, rief Ijsdur. Seine Stimme blieb dabei wie meistens monoton. „Doch werden wir nicht einfach mit einem Seil zu Siantari aufmarschieren und sie gefangen nehmen können. Und pure Gewalt dürfte gegen ihren erstarrenden Blick kaum helfen. Es scheint mir klüger, eine List anzuwenden.“

„Eine List?“

„Noch denkt Siantari, ich wäre auf ihrer Seite. Lass mich dich gefangen nehmen. Nur zum Schein. Ich bringe dich zu ihr und erzähle ihr etwas davon, wie du der Schlüssel dazu bist, das ewige Eis über diese Lande zu verbreiten.“

„Siantari dürfte sich wundern, dass sie dich nicht mehr kontrollieren kann.“

„Auch das können wir vielleicht dir und deiner Feuermagie in die Schuhe schieben. Davon abgesehen wissen wir nicht einmal, wie es sich anfühlt, Herrin aller Eis-Dämonen zu sein. Vielleicht merkt sie nicht einmal aktiv, dass ich ihrer Kontrolle entschlüpft bin.“

„Und vielleicht muss sie sich nur stark genug auf dich konzentrieren, um Irils Runen zu überwinden und dich wieder unter ihren Willen zu kriegen. Dann hätten wir ihr prompt uns beide ausgeliefert. Das kann doch kaum das Ziel sein.“

„Natürlich nicht. Aber es hilft unserem Ziel auch kaum, blind mit zwei Schwertern auf sie zuzustürmen und sie zu erledigen hoffen.“

Stille.

„Wenn du meinst“, murmelte Aćh. Sie kramte ihre Dose Sufarsaft hervor, träufelte eine gehörige Prise Takuri-Asche hinein und rieb ihr Seil in der Flüssigkeit ein. Anschließend schwang sie es an ihre Hüfte, wo es hoffentlich möglichst natürlich zu ihrer zeremonielle Rüstung passte.

Ijsdur nahm Aćhs Schwert an sich und hängte es an seine Hüfte, an der sich auf magische Art ein passender Gurt mit Scheide formte. Dann hielt er Aćhs Arme fest, als wäre sie seine entwaffnete Gefangene, und schob sie vor sich hin.

So bewegten die beiden sich weiter.

„Da vorne!“, rief Ijsdur. Aćh konnte noch nicht gut erkennen, was genau er meinte. Sie hatte die riesige Säule aus umherwirbelndem Schnee und Nebel aber ebenfalls schon erspäht. Der Alte Wehrturm war kaum sichtbar. Falls das Gemäuer nicht schon beim Kampf gegen die Kälte eingestürzt wäre, hatte Siantari ihm nun definitiv den Rest gegeben. Worauf wartete sie nur?

„Da vorne ist Siantari“, wiederholte Ijsdur. „Ich sehe ihre Silhouette. Sie thront oben auf der Ruine des alten Wehrturms. Nun, inzwischen ist es eher ein Schneehügel. Ausgestreckte Arme. Sie befiehlt den Sturm. Und sie ist nicht allein.“

„Drachenkultisten? Etwa gar ein weiterer Drache?“

„Ich glaube nicht. Das sind Gefangene! Mindestens zehn Personen. Klein. Kinder? Sie sitzen in einer Reihe weiter unten am Hügel. Da!“

Ijsdur zeigte in den Schneesturm hinein, doch Aćh konnte bloß dunkle Schemen erkennen.

„Jetzt kusch, Turr! Warte auf deinen Moment!“, rief Ijsdur, und wedelte mit den Armen. Turr gurrte protestierend, flatterte dann aber davon, hoch in den Himmel, in die dunklen Wolken, in denen er hoffentlich nicht einmal von den scharfen Augen eines Dämons erspäht werden konnte.

Ein mulmiges Gefühl machte sich in Aćh breit. Ohne Turr kam sie sich so alleingelassen vor.

Hier lief sie, waffenlos, ungeschützt. Kurz davor, einer bösartigen Eis-Dämonin entgegenzustehen, die sie mit einer raschen Berührung zu einem ihrer Diener machen könnte. Schaudernd dachte Aćh daran, wie sich Aćhdora angefühlt hatte.

Der neben ihr stehende Eis-Dämon hatte versprochen, nicht auf Siantaris Seite zu stehen. Konnte sie ihm wirklich vertrauen? Turr hatte Ijsdur noch nie gemocht. Was, wenn dies die ganze Zeit sein Plan gewesen war? Aćh und Turr an Siantari auszuliefern und die Gefahr des Feuers auszuschalten? Tenaya die Feuerwächterin und Lifornus der Feuerzauberer, ja selbst Trieest der Feuerkrieger befanden sich weit weg von hier, das hatte Eara bei der letzten Heldenversammlung mitgeteilt.

Kalter Wind heulte und klatschte Schneeflocken gegen Aćhs Gesicht. Ihre Schritte wurden schwerer. Ihre Stiefel sackten immer tiefer in den lockeren Schnee ein. Ijsdur hingegen wurde beschwingter. Er lief voran, ohne in den Schnee einzusinken.

Sie hatten den Schneesturm erreicht. Inzwischen konnte Aćh die Silhouette des alten Wehrturms wieder besser erkennen. Neben dem beständigen Rauschen des Sturmwinds in dieser klirrenden Kälte glaubte sie auch, Schmerzensschreie zu hören.

Tappte Aćh soeben in eine Falle? Ihr Atem beschleunigte sich. Nein, redete sie sich ein, das waren nur ihre Sorgen, die sich verselbstständigen. Wenn schon, hätte Ijsdur doch lieber Iril ausgeschaltet, ihre Runenmagie schein die größte Gefahr für die Eis-Dämonen. Naja, außer die Runen funktionierten gar nicht und Ijsdur täuschte ... stopp, sie durfte diesen Gedanken nicht länger nachhängen. Ijsdur hatte ihr Leben gerettet, schon mehrmals. Und er hatte nie ein Anzeichen bösartigen Willens gezeigt, während Nesdora damals schon innert Kürze ihr wahres Gesicht gezeigt hatte. Doch was war mit diesen gefangenen Gestalten vor Siantari? Und was war mit diesem Wirbelsturm?

„Ijsdur, können wir kurz anhalten?“, fragte Aćh, „Ich muss meine Gedanken sortieren.“

„Jetzt hat uns Siantari vielleicht schon erspäht. Wenn wir anhalten, gefährdet dies unsere Tarnung.“ Ijsdur zog an Aćhs Arm. Sie stolperte weiter.

„Ijsdur, halt an. Ich habe kein gutes Gefühl bei der Sache. Wirklich nicht.“

„Umso besser, das wirkt viel authentischer.“

„Ijsdur, lass mich los!“

Aćh riss sich los von Ijsdur. In einer geschmeidigen Bewegung riss sie ihr Schwert von seinem Gurt und taumelte einige Schritte zurück. Blut pochte in ihren Ohren. Das Feuer der Furcht raste durch ihre Adern. Ijsdur blickte sie aus kalten, schneeweißen Augen an. Genau wie Nesdora es getan hatte. Ijsdurs Eiskristallkette glitzerte und glomm schwach umgeben von all diesem Schnee. Furcht überkam Aćh. Sie wirbelte herum und nahm ihre Beine in die Hand.

Im Heulen des Windes hörte sie nicht, ob Ijsdur sie verfolgte. Eine Schneewehe gab unter Aćhs Stiefeln nach. Sie kippte zur Seite, verwickelte sich in ihrem langen Umhang und kullerte unfreiwillig in eine Kuhle. Es war nass und kalt und nass. Bäh! Sie verschnaufte und verfluchte zu gleichen Teilen sich selbst, Ijsdur und Siantari.

Dann schob sie ihren Kopf leicht über die Anhöhe und blickte dorthin, woher sie gekommen war. Der Sturmwind machte es schwer, irgendetwas zu erkennen. Immerhin würde er auch ihre Spuren rasch verwischen.

Da, eine Gestalt! Gehörnt und zu Fuß unterwegs. Ijsdur trat näher, bis er nur noch einige Mannslängen von Aćh entfernt war. Er sah sich ratlos um, blickte allerdings weit über Aćhs aus der Senke hervorlugenden Kopf hinweg.

Doch er war nicht allein. Der Sturm nahm an Stärke zu, als zweite Gestalt in den Fokus kam. Auch sie war gehörnt, doch lief sie nicht, sondern schwebte mit ausgestreckten Armen auf einem Kissen aus Luft. Ihr langes Kleid flatterte im Wind.

Eisblaue Augen richteten sich auf Ijsdur und eine Stimme erklang, so klar und so kalt wie Eis, und noch viel klirrender als Ijsdurs.

„Sieh an. Hat es doch noch ein anderer Eis-Dämon bei Verstand durchs Felsentor geschafft. Der verlorene Sohn kehrt zurück.“

Ijsdur drehte sich um und fiel vor Siantari auf das Knie. „Herrin! Endlich habe ich Euch wiedergefunden!“














\newpage
\section{Die eisigen Drei}


Sagramak klopfte gegen das schwere Holztor, das den Zugang zum Thronsaal verwehrte. Sie war königlicher Laune.

„Hallo? Jemand zuhause?“, rief sie munter. „Werter Prinz? Ein Vögelein zwitscherte mir, dass Ihr allein wüsstet, wo Taroks letzte Knochen versteckt sind. Die Teilung dieses Wissens könnte Eurem Volk allerlei Leid ersparen.“

Als keine Antwort erklang, schlug sie mit ihrem Speer tiefe Furchen ins Holz. Die massive Tür gab nicht nach.

„Ein wenig Hilfe hier?“

Sie bekam keine sofortige Antwort, nur Waffengeklirr. Ein Blick verriet, dass hinter ihr erneut Gefechte ausgebrochen waren. Einige Krieger des Königs hatten sich aus ihren Türmen getraut und den wenigen anwesenden Drachenkultisten entgegengestellt. Warum ergaben sich diese Idioten nicht einfach? Sie mussten doch sehen, dass sie gegen einen Eis-Drachen nichts ausrichten konnten.

Wobei der Drache nicht mal kämpfte, sondern nur zuschaute, was seine Anhänger taten. Nehamal hatte seinen Degen ausgepackt und demonstrierte sein Geschick im Umgang mit der edlen Klingenwaffe. Fir sprang hektisch umher und verteilte hirnerschütternde Huftritte, gleichzeitig laut über die ihm entgegengebrachte Animosität protestierend. Der Schwarze Herold schwebte über den Kämpfenden und stach nach unbehelmten Kriegerköpfen. Doch waren sie zahlenmäßig weit unterlegen. Schon sah Sagramak einen Skral niederfallen und nicht mehr aufstehen.

Was zum Himmel dachte sich Sagrakdur? Nutzlos stand er über den Scharmützeln und blickte umher, als könnte er sich nicht entscheiden, wohin er seinen Eisstrahl richten sollte. Der unerfahrene Eis-Drache hielt nicht, was der Herold versprochen hatte.

„Links von dir!“, brüllte Sagramak. Sagrakdur wirbelte herum. Eine Kriegerin mit einer eisernen Lanze preschte auf einem edlen Schimmel auf Sagrakdur zu. Die Waffe grub sich tief in seinen Brustkorb. Sagrakdur kippte zur Seite und riss die Kriegerin von ihrem Pferd. Er sammelte seinen Atem und sortierte seinen Schneekörper. Dann erhob er sich auch schon wieder und richtete seinen stacheligen Schädel auf die Angreifer.

Der Mensch blieb wie erstarrt stehen. Das Pferd suchte sein Heil in der Flucht. Er würde sie beide erwischen. Der glorreiche eisblaue Drache richtete seinen Schlangenhals auf und spürte verzückt das vertraute kalte Brodeln, wie es von seinem Magen langsam seinen Hals hinaufkletterte und seinen Schlund erreichte. Sagrakdur blähte seine Nüstern, öffnete seinen Rachen und grinste in Antizipation dessen, was er diesem mickrigen Spornfraß vor sich gleich antun würde. Dann spie er einen Strahl puren magischen Eises, bläulich glänzende Wirbel, denen nichts und niemand standhalten konnte.

Nichts außer einem Schild.

Ein weiterer Krieger – ein ganz junger diesmal – warf sich zwischen den Eisstrahl und die vor Furcht erstarrte Lanzenträgerin. Er wuchtete einen Buckelschild zwischen sich und Sagrakdur. Der Eisstrahl traf den Schild und überzog dessen Oberfläche mit Reif, ließ den sich dahinter kauernden Jungen jedoch unversehrt. Der blaue Schneewirbel verpuffte. Verdutzt klappte Sagrakdur seinen Mund wieder zusammen.

„Für Prinz Thorald!“, schrie der Junge mit einer viel zu hohen Stimme. Der Schwarze Herold wirbelte zu ihm und stach ihn nieder, mit solch entsetzlicher Wucht, dass er mehrere Meter weit fortgeschleudert wurde.

„Hierher, Sagrakdur!“, schrie Sagramak und gestikulierte zum Tor des Thronsaals. Sagrakdur schlängelte seinen eisigen Körper zu ihr. Er drückte das Tor mit einer Pranke ein, aus bestünde es aus dünnem Pergament.

Schreie ertönten. Andori, die sich im Palas versteckt hatten, drückten sich noch tiefer in die Ecken des Raumes. Sagramak starrte ins Dunkel. Da, an der Wand, zwischen zwei Säulen hing ein Gemälde, das drei Personen zeigte. König und Königin, mit einem jungen schwarzhaarigen Burschen, welcher zwischen seinen beiden Eltern stand und stolz in die Welt hinausschaute. Dies war Thorald.

Doch der erwachsene Prinz war nirgendwo zu sehen. Und auch keine Drachenknochenkette. Sagramak fluchte.\bigskip







Schon aus der Ferne war der flackernde lila Lichtschein des ewigen Feuers erkennbar.

Einige flüchtende Andori kreuzten den Weg von Iril und Barz. In alle Richtungen rannten sie von der Rietburg weg, ihr wichtigstes Hab und Gut – oder schlicht Proviantrationen – in Bündeln mit sich schleppend.

Sie achteten kaum auf die beiden Helden, die die großartige Idee hatten, der besetzten Burg näher zu kommen.

Barz löste sein türkises Vorhersehungspulver aus Silberblumenblütenpulver von seinem Gürtel und streute sich eine Prise davon auf die ausgestreckte Zunge. Er schmatzte ein, zweimal und schrie kurz auf, als seine Augen blendend hell aufleuchteten. Dann ächzte er auf und legte seine offene Hand auf seine Brust, Handfläche nach außen. Ein Stoßgebet an die Götter.

„Sie sind echt stark. Dieser Eis-Drache ist ein gewaltiger Gegner. Und auch die Kultisten selbst verfügen über verblüffende Kräfte. Sie zu überwinden wird beinahe unmöglich.“

„Keine andere Wahl“, knurrte Iril. Die beiden rasten den Hügel hinauf und unter dem Torbogen in die Rietburg hinein. Kampfeslärm erwartete sie. Schreie und Waffengeklirr. Tote und Verwundete, Krieger unter purpurroter Fahne und Kämpfer mit auf ihren Gesichtern gemalten Zeichen der Drachen.

Ein Krachen ertönte. Der blauweiße Eis-Drache mit dem gehörnten Kopf trat vom Palas zurück. Er hatte ein tiefes Loch im Tor hinterlassen. Immerhin war es kein Feuerdrache, unter dessen Atem die Rietburg mit ihren Rietgrasdächern sofort in Flammen aufgegangen wäre.

Die Drachenkultisten waren den Kriegern der Rietburg in Anzahl ziemlich unterlegen. Ihre Vertreter wurden soeben zusammengetrieben. Doch wie lange konnten die tapferen Andori diesen Vorteil in Anwesenheit eines Eis-Drachen halten? Manche hatten aus ihren Speeren Fackeln gebastelt und fuchtelten damit in Sagrakdurs Richtung, als ob er sie nicht mit einem einzigen Eisstrahl löschen könnte.

Iril erkannte das Ziegenwesen Fir, das soeben vor dem jungen Krieger Malin zurückwich.

Sie sah Nehamal, der eine in einen langen Mantel gekleidete Frau zu Boden warf, ehe sie von einem antrampelnden Pferd überrannt werden konnte.

Doch wo war ... da war sie! Sagramak die Schamanin, Anführerin der Angreifer, stand oben am Palas, stolz über das Geschehen hinwegblickend, und rief: „Dort drüben, Sagrakdur! Vernichte die Törichten!“

Sagrakdur richtete seinen langen Schlangenhals auf und ließ einen Schwall aus Eis und Schnee auf die Krieger von Andor niederregnen. Hatten sie vorhin gerade noch das Dutzend anwesender Drachenkultisten zurückgetrieben, waren sie nun selbst gezwungen, aus dem Weg zu springen und zurückzuweichen. Nur diejenigen, die direkt in einen Zweikampf mit einem Kultisten verwickelt waren, blieben vor dem tödlichen Odem gefeit.

Ein Pfeil schoss aus einem Hausfenster heraus und traf Fir in der Brust. „Das ist das Ende für dich, o Fir!“, heulte das Ziegenwesen, „Ein so kurzes Leben, und so unerfüllt! Was für eine wildere Welt du verlässt als die, in die du hineingeboren wurdest.“ Dann sackte Fir zusammen.

Sagrakdur fuhr herum und spie einen weiteren gewaltigen Eisstrahl. Eisklötze, groß wie Zwerge, trafen das Tor über Irils Kopf, zerbarsten und regneten als grobkörniges Pulver auf sie herunter.

„SAGRAMAK!“, brüllte sie, „Gebiete deinen Drachen, einzuhalten. Lass uns reden!“

„Mit denen lässt sich nicht reden!“, rief eine grimmige Stimme von der Mauer herunter, wo Iril den mürrischen Zwerg Lafgar erkannte.

Iril starrte ihn böse an und gebot ihm, zu schweigen. Doch Sagramak reagierte tatsächlich nicht auf ihren Ruf. Still und stur schaute sie zu, wie Sagrakdur ein weiteres Haus dem Erdboden gleichmachte. Staub wallte hervor.

Da rannten Menschen aus dem zusammenbrechenden Haus hervor. Da war Nabib! Er zog den alten Heiler Readem hinter sich her.

Barz erstarrte, als er Nabib erkannte.

Nabib erstarrte, als er Barz erkannte.

„Was ... nein ... oh, du Idi ... bring dich in Sicherheit!“, stammelte Nabib.

„Niemals!“, rief Barz, seinen Bogen ziehend. Der Eis-Drache drehte sich zu ihm um. Bläulich schimmerte die Kehle des Drachen auf, als er Kraft für seinen Eisatem sammelte.

„NEIN!“, rief Nabib, „Kommt nicht in Frage! Guck hierher, du dummes Schneeding! Renn schon weg, Barz!“

Der Drache schwenkte seinen geöffneten Kiefer wieder in Richtung Nabib. Hinter ihm rannte Heiler Readem hastig davon und verkroch sich hinter einer umgekippten Steintafel.

„NEIN!“, rief Barz nun und rannte nach vorne, um sich zwischen den gewaltigen Kiefer der eisigen Echse und Nabib zu werfen. Doch anstatt den Drachen anzublicken, drehte er sich in Richtung Sagramak, die das Geschehen amüsiert beobachtete.

Schwer atmend rief er: „Sagramak, lass gut sein, bitte. Ihr habt eure Macht demonstriert. Ihr müsst keinen weiteren Schaden anrichten.“, rief Barz. Über ihm knurrte der Eis-Drache weiterhin bedrohlich. Doch als Sagramak stolz ihre Hand hob, ließ er fügsam von den beiden ab.

„Oh, das wird amüsant“, lachte die Schamanin schallend. „Ich dachte nicht, dass du auch hier wärst. Und jetzt willst du reden? Wolltet ihr beim letzten Mal nicht auch nur mit Gewalt bestimmen, wer welches Schicksal erhält?“

Während Barz und Nabib darum rangelten, wer sich schützend vor den anderen stellen durfte, stapfte Iril den Weg zum Palas hoch. Auch wenn man kaum mit Sagramak verhandeln konnte, war es geschickt, den Weg zu ihr zurückzulegen, ehe der gewaltige Eis-Drache ihr da einen Strich durch die Rechnung machen konnte.

Blutgetränkter Schnee knirschte unter ihren Stiefeln. Um sie herum kamen die Zweikämpfe zwischen Kultisten und Kriegern langsam zu einem Halt. Manche, weil ein Kontrahent einen anderen so weit verletzte, dass er nicht weiterkämpfen konnte. Andere, weil die Kämpfer eine Gelegenheit sahen, nicht weiter nacheinander zu stechen, und ihre Aufmerksamkeit auf das Geschehen vor ihnen zu lenken.

Iril hielt Sagramaks Aufmerksamkeit und wählte ihre Worte mit Bedacht. „Ich wollte dir damals nicht schaden, und will es immer noch nicht. Doch muss man kleine Übel annehmen, um größere zu verhindern.“

„Nun, Iril, um ganz offen zu sprechen“, meinte Barz von weiter hinten versöhnerisch, „Vielleicht wäre gar nichts geschehen, wenn wir ihnen die Knochen einfach überlassen hätten. Es war doch nur ihr neu entfachter Hass, der sie dazu trieb ...“

„Der Schwarze Herold hätte sie bestimmt dazu anstacheln können! Bitte, stell dich nicht auf ihre Seite.“

„Es geht nicht um Seiten!“, rief Barz, „Barbaren oder Andori. Drachenkultisten oder Bauern. Wir alle sitzen im selben Boot. Wir alle wollen Frieden.“

„Wenn ich mich einmischen darf: Wir sitzen nicht ganz im selben Boot“, grinste Sagramak vom Hügel herab. „Einige von uns sitzen auf einem mächtigen Eis-Drachen.“

„Es geht euch um Taroks Knochensplitter, oder?“, versuchte Barz mit erhobenen Händen, die Situation weiter zu bereinigen, „Ich bin mir relativ sicher, dass Thorald sie euch nun mit Freuden aushändigen würde, wenn ihr ihn nicht weiter belästigt.“

„Ja, darum geht es. Die Knochen, die du uns bei unserem letzten Treffen so unhöflich entwendet hast. Wo sind sie?“

„Ich weiß es nicht“, gab Barz zu.

Ein Pfeil schoss aus einem Turm der Rietburg hervor und schrammte an Sagramaks Rüstung ab.

Das brach den Bann. Die Zweikämpfe im Burghof brachen wieder aus. Nehamal schrie etwas Unverständliches und stach mit seinem Degen einen Zwerg vor ihm ab. Der Eis-Drache brüllte auf und sprang den Turm an, aus dem der Pfeil geschossen worden war. Sein Schwanz zuckte haarscharf an Barz vorbei und schleuderte Nabib gegen einen Felsen. Barz hechtete ihm nach.

„Siehst du, wie viel das Reden gebracht hat?“, knurrte Iril mehr zu sich selbst. Sie schwang ihren Hammer und legte mit raschen Schritten die letzten Meter zwischen sich und Sagramak zurück. Diese lachte nur schallend.

Der Runenhammer zischte pfeifend durch die Luft. Statt auf harte Rüstung zu treffen, streifte er nur das metallene Blatt am Ende von Sagramaks Speerwaffe. Blitzschnell wirbelte Sagramak umher und trieb Iril zurück. Bei der Mutter des Steins, die Schamanin war stark!

Furcht machte sich in Iril breit. Furcht, sich überschätzt zu haben. Furcht, ihre Chance soeben verspielt zu haben. Lähmende Furcht, die in einem Kampf nichts zu suchen hatte. Iril war klein und hatte nur einen Hammer. Sagramak war überragend groß und führte eine lange Stangenwaffe, mit der sie Iril locker auf Distanz halten konnte. Wann immer Iril ihr näher zu kommen versuchte, war der Speer bereits dort. Es ermüdete sie, den Stichen und Schnitten auszuweichen. Lange würde sie nicht mehr durchhalten können. Dieser Speer musste gehen, und zwar jetzt!

Iril gönnte sich eine Verschnaufpause und trat einige Schritte zurück. Sie hatte schon einen Schnitt in der Stirn, dessen Bluttropfen ihr die Sicht erschwerte. Die weiterhin grinsende Sagramak war hingegen unversehrt. Sie umkreisten einander. Iril atmete tief durch. Fasste ihren Runenhammer fester. Spürte, wie er unglaubliche Energie aus der Umgebung aufsaugte. Ein riesiger magischer Drache war anwesend, das musste doch für etwas gut sein.

Es war Iril, die als erste wieder nach vorne sprang und dem Kampf seinen Rhythmus gab. In rascher Abfolge kamen ihre Schläge, und nicht mehr auf Sagramak selbst gerichtet waren sie, sondern auf ihren Speer. Erschütterung nach Erschütterung traf den Speer und rüttelte am Griff der ihn Führenden. Das Grinsen der Schamanin erstarb. Link, rechts, links, links, schlug Iril darauf los, stets darauf achtend, sich von ihren Schwüngen nicht zu weit treiben zu lassen. Und noch immer hielt sie die Magie im Runenhammer fest.

Sagramak stolperte. Die Speerspitze streifte den Boden. Iril hob ihren Hammer hoch und ließ ihn mit aller Kraft auf den Speer niederfahren. Die Magie der Runen entlud sich. Grünes Feuer blitzte den Speer entlang. Es krachte und splitterte.

Sagramak stürzte zu Boden. Iril fegte ihren glänzenden Helm von ihrem Kopf.

„Ergib dich! Pfeif deinen Eis-Dra ...“, setzte sie an. Sagramak brüllte bloß auf, sprang nach vorne und trieb Iril die abgebrochene Stange ihres Speers in den Magen. Nun war es an Iril, zurückzustolpern. Sagramak stieß die zweite Hälfte ihres Speers mit einem Stiefel in die Höhe und fasste sie elegant. Mit zwei halben Stangenwaffen bewaffnet, jagte sie Iril zornerfüllt nach und ließ ihr keine Ruhe. Hatte sie sich zuvor zurückgehalten?

Der Runenhammer kam nicht nach. Stockschläge trafen Irils Schulter und Arm. Sie verlor ihren Hammer und kam nicht einmal dazu, ihre Rückrufrune zu aktivieren, da gab auch schon der Boden unter ihr nach.

Ein letzter Stoß Sagramaks ließ Iril rückwärts den Abhang hinunterkullern, von der Höhe des Palas‘ bis zu den tief darunter liegenden Steinen des Pferdepfads. Es knirschte unschön in ihrer Reisetasche. Iril hoffte, dass ihre Runenscheibe noch ganz war. Hoch über ihr am Hügel stand Sagramak und guckte selbstverliebt auf sie herunter.

Iril versuchte sich aufzurichten, kitzelte sie eine Klinge am Rücken. „Ergib dich“, klang eine schnarrende Stimme. „Die Drachen wollen deinen Tod nicht. Du hast Platz in ihren Plänen.“

Iril hob ihre Hände und drehte sich langsam herum. Hinter ihr stand Nehamal, der Degenträger, der vor ihr in die Hocke gegangen war und ihr nun seinen Degen an die Kehle hielt.

Ein Blick hinter sich zeigte, dass es um ihre Gefährten nicht besser stand: Von den wenigen im Hofplatz verbliebenen Kriegern des Königs bewegte sich keiner mehr. Der Rest hatte sich wohl verschanzt oder das Weite gesucht. Die Scharten, aus denen die Bogenschützen der Andori hatten Pfeile herabregnen lassen, waren mit einer dicken Schicht aus Eis besetzt. Der Eis-Drache Sagrakdur hatte sich auf den Rücken gedreht und wälzte sich wohlig auf der Ruine eines Hofhauses. Bei allen Kreaturen der Tiefe, wo war Barz?!

Abrupt sprang Iril zurück. Nehamal reagierte zu spät. Sein Degenstoß ging ins Leere. Iril atmete tief durch, sammelte sich und aktivierte ein Runentattoo, das den Runenhammer von Golja zu ihr zurückrufen sollte. Nichts geschah. Sie fluchte, als sie erkannte, dass der gewaltige Schatten Sagrakdurs auf sie fiel und das magische Sonnenlicht von ihr abschirmte. So hatte sie keine Möglichkeit, ihre Tattoos ohne Runenhammer zu aktivieren. Doch noch war sie nicht waffenlos.

Iril zog eine bestimmte dünne metallene Runenscheibe aus ihrer Reisetasche hervor, die für genau solche Zwecke geeignet war. Sie brauchte weder direktes Mond- oder Sonnenlicht noch einen magischen Hammer, sondern konnte sich durch Rotationen mit Energie speisen. Iril drehte mit schnellen Bewegungen ihre Runenscheibe, ihre Augen begannen vielfarbig zu glühen und bunt leuchtende Runen stiegen langsam aus dem Boden. Bunte Schlieren überdeckten ihr Sichtfeld, doch nahm sie noch wahr, wie Nehamal vorsichtig zurückwich. Er hatte wohl noch nie jemanden Runen in die Luft zeichnen sehen und konnte deren Gefahrengrad nicht einschätzen.

Vorsichtig tippte Nehamal mit dem Degen eine Luftrune an und sah interessiert zu, wie sie sich unter dem Stich wellte und verpuffte.

Zu schade, dass dies ihm den Eindruck geben musste, die Runen wären ungefährlich. Wenn er nur lange genug stillhalten würde, könnte Iril ihn problemlos mit solchen Runen fesseln. Aber nun würde er kaum still stehen bleiben. In Ermangelung einer anderen Waffe schleuderte Iril weitere Runenscheiben aus ihrer Reisetasche wie Frisbees Nehamal entgegen.

Erfolglos. Effekte hatten diese natürlich nicht und außer einem Schnitt an seiner Wange kümmerte Nehamal sich nicht um die heranregnenden Scheiben.

Mit drei Schritten war er bei Iril angekommen und griff nach ihr.

Iril sollte eigentlich stärker sein als ein dahergelaufener Mensch. Dennoch überrumpelte er sie mühelos. Ein leises Grollen drang aus Nehamals Kehle und sein Griff um Irils Handgelenke versteifte sich noch mehr. Wie ein Schraubstock hielt er sie fest.

„Du dachtest, du kommst einfach so davon? Du hast nicht die leiseste Ahnung, mit wem du dich eingelassen hast und welche Kräfte mir zur Verfügung stehen.“\bigskip







Barz öffnete die Holztür mit einem Tritt und schleppte den bewusstlosen Nabib über die Schwelle. Dieses Haus stand am Rand der Rietburg, direkt mit der Wehrmauer verbunden. Natürlich war aktuell kein Ort hier im Riethof komplett sicher, aber dieser hier bot zumindest relative Sicherheit.

Barz ließ Nabib sanft auf eine Decke gleiten und atmete durch. Nabib war in Sicherheit. Zeit, zum Kampfgeschehen zurückzukehren

Da bemerkte Barz das Augenpaar, das ihn aus einer Ecke beobachtete. Er traute seinem eigenen Augenpaar kaum.

Ein großer Krieger mit stahlblauen Augen blinzelte Barz wortlos entgegen. Und hinter ihm saß Thorald, Prinz und künftiger König von Andor, zusammengesunken in einer Ecke der Hütte und zitterte. Nicht nur das, er hatte eine Flasche in der Hand, deren Geruch ihren Inhalt verriet. Ein schöner Anführer war das.

„Achtet auf ihn, ja?“, fragte Barz laut und deutete auf den bewusstlosen Nabib. Thorald reagierte nicht, doch Armond nickte. Barz schüttelte seinen Kopf und verließ das Haus wieder. Es war wohl geschickter, Thorald nicht noch weiter in das Geschehen zu verwickeln.

Draußen schlug ihm auch schon wieder die Kälte des allgegenwärtigen Schnees entgegen. Barz versuchte, sich einen Überblick über die aktuelle Lage zu verschaffen.

Der Kampfeslärm war größtenteils verstummt. Von den andorischen Kriegern, die im offenen Riethof herumlagen, rührte sich keiner mehr. Die Scharten, aus denen Verstärkung die sich neu sammelnden Drachenkultisten hätten beschießen können, waren vereist worden.

Der Eis-Drache Sagrakdur balancierte auf einem hohen Steinturm und warf seinen langen Schatten auf ... da vorne! Iril kämpfte gegen Nehamal. Drei andere, teils weniger, teils schwerer verletzte Kultisten näherten sich den beiden aus verschiedenen Richtungen. Unschöne Lage.

Nicht weit von Barz entfernt stand Sagramak auf dem Hügel vor dem Palas und grinste. Ihre Waffe war zerbrochen. Neben ihr lag Irils Runenhammer nutzlos am Boden.

Sie hatte die Macht, diesem Kampf ein Ende zu bereiten. Konnte Barz sie vielleicht überrumpeln und von den anderen Kultisten eine Ergebung verlangen? Das klang durchaus gut.

Barz nahm seinen Bogen und atmete tief durch. Schätzte die Distanz zu Sagramak ab.

Zog.

Zielte.

Schoss.

Der Pfeil vollführte einen wunderschönen Bogen und prallte nutzlos an Sagramaks Beinpanzer ab. Ihr Blick richtete sich auf Barz. Sie lächelte schief und pfiff. Ein Dolch schoss aus ihrem Gürtel auf Barz zu. Die telekinetische Waffe sauste an ihm vorbei und durchtrennte seine Bogensehne, ehe er einen weiteren Pfeil abschießen konnte.

„Immer noch ein nutzloser Trick?“, fragte die Schamanin Barz schnippisch. Barz schnaubte bloß. So schnell wie möglich stapfte er über Schneehaufen und herumliegende Körper auf den Palas zu.

Er warf eine Prise dunkelblauen Schwächungspulvers in die Luft, welches sich magisch glitzernd im umherwirbelnden Wind verteilte und um Sagramak herumwirbelte. Diese fauchte auf, als die ätzende Cantharis-Komponente des Pulvers sich in ihre Haut grub. Barz fühlte frische Energie durch seinen Körper schießen.

Jetzt zählte jeder Augenblick. Barz war kein geschickter Nahkämpfer, doch wenn es ihm gelänge, Sagramak genügend rasch auszuschalten, um den anderen Kultisten zu drohen ... und bevor der Eis-Drache bemerkte, was sich unter ihm abspielte ...

Irils Runenhammer lag nur einige Schritte von ihm entfernt, seine Besitzerin nirgendwo zu sehen. Barz hechtete zum Hammer, nahm ihn auf und haute damit nach Sagramak, welche dem Schwung mühelos auswich.

Interessanterweise leuchtete der Hammer nicht magisch auf, wie er es bei Iril getan hatte. Und das Ding war verdammt schwer. Barz konnte sich drei weitere Schwinger in Sagramaks grobe Richtung erlauben, ehe er diesen Steinklotz von einer Waffe wieder zu Boden sinken ließ.

Wie schaltete Iril die im Hammer gespeicherte Macht frei? Musste er dafür irgendeinen Muskel anspannen? Einen bestimmten Griff? Eine bestimmte Rune? Barz betatschte den Hammer hastig, ohne ein Leuchten auszulösen.

Eine Dolchklinge kitzelte ihn an der Kehle.

„Ergib dich“, befahl Sagramak.

Barz war weise genug, Folge zu leisten. Er hob seine Hände.

„Auf, auf zum Verlies mit dir!“, rief Sagramak. „Du kannst Iril Gesellschaft leisten.“\bigskip







Siantari blickte kalt auf Ijsdur herab. Aćh zitterte in ihrem kalten Versteck, nur wenige Schritte davon entfernt, doch durch den mächtigen Sturm gut verdeckt.

„Herrin!“, wiederholte Ijsdur, „Endlich habe ich euch wiedergefunden.“

„Mein Kind“, sprach Siantari kalt. Sie landete sanft auf dem schneeigen Untergrund und legte eine Hand an Ijsdurs Wange. Ihr Gesicht zeigte keinerlei Regung. „Ich hatte dich schon verloren geglaubt. Was ist mit dir geschehen? Warum sind deine Gedanken vor mir verborgen?“

„Ich wurde überrumpelt. Ich wollte durch eine List aus Freund der Menschen auftreten und mehr über ihre Verteidigungskapazitäten erfahren. Ich versagte. Mächtige Runenmagie hielt unsere Geister lange getrennt, zu lange. Doch bin ich Euch noch immer untertan.“

Siantari schloss ihre Augen. „Ich fühlte deine Präsenz in diesem Land, mein Kind. Ich nahm wahr, wie du untergingst. Ich hätte nicht gedacht, dass du es schaffst, allein weiter zu bestehen. Du warst tapfer. Lass mich dich wieder zu einem Teil meiner selbst machen.“

„Gewiss Herrin, wenn ich nur wüsste, wie.“

Siantari legte ihren Kopf schief. Mit weiterhin geschlossenen Augen bewegte sie ihre Hände Ijsdurs Körper entlang, vielleicht irgendwelche Ströme der Magie fühlend.

Aćh packte ihr Schwert fester. War dies der Moment, die abgelenkte Eis-Dämonin zu fangen, vielleicht gar zu erledigen? Wenn sie den Hang hochgestapft wäre und Siantari erreicht hätte, wäre der Überraschungsmoment schon lange verflogen.

„Wie hast du mich wiedergefunden? Hast auch du die Macht des Wintersteins gespürt?“, ertönte Siantaris klirrende Stimme.

„Der gewaltige Wirbelsturm war schwer zu übersehen.“

Siantari kicherte tatsächlich kurz auf, ehe sie wieder tonlos fortfuhr: „Der Sturm war mir nicht einmal bewusst. Dann werden meine Kräfte tatsächlich schon amplifiziert.“

„Von diesem ... Winterstein, den ihr erwähntet? Befindet er sich hier?“

„Nein, der Winterstein ist eben nicht mehr hier. Ich spüre deutlich die Spuren eines mächtigen Artefakts der Ewigen Kälte. Es befand sich hier, eine ganze Zeit lang, und noch nicht so lange her. Nur ein Bruchteil seiner Kraft sickerte in den Boden, und schon der ist gewaltig. Mit einem solchen Relikt könnte das ewige Eis schon in Jahren die gesamte bekannte Welt überdeckt haben. Und ich kenne nichts außer seinem Namen.“

„Winterstein.“

Siantari nickte. „Die Gefangenen nennen es so. Doch will mir niemand verraten, wo es hingebracht wurde. Oder sie wissen es nicht. Oder sie verstehen mich nicht richtig.“

„Gefangene?“

„Gute Güte, ich vermisste es nicht, mit anderen zu sprechen“, seufzte Siantari. „Ja, Gefangene! Dort hinten, am alten Wehrturm. Und sie werden mir verraten, wo der Winterstein sich nun aufhält. Die einzige Alternative dazu ist ein baldiges Ende ihrer elenden Leben. Sprich, wer war dieser Mensch, den du eben noch bei dir hattest? Weiß sie vielleicht etwas über den Winterstein?“

Aćh erstarrte in ihrem Versteck.

Ijsdur gab langsam und bedacht Antwort: „Eine Takuri-Hüterin aus Tulgor. Eine, die schon einmal fast in unsere Fänge geraten wäre. Sie wird eine würdige Eis-Dämonin abgeben.“

„Nicht sie“, flüsterte Siantari, „Sondern derjenige, den sie hütet. Stell dir das vor. Ein Eis-Takuri, der ewigen Schnee über diese Lande brächte und selbst im Tode nicht von seinem Auftrag abkäme. Das wäre durchaus ein Traum. Schwach von dir, sie einfach so entkommen zu lassen.“

Ijsdur reagierte nicht darauf. Siantari schien auch keine Entschuldigung zu erwarten.

„Ihr Takuri heißt Terebros“, hängte Ijsdur an. Auf einmal sprach er wieder laut, als wolle er den Sturm übertönen. „Ein mächtiges Wesen. Aktuell ist es nichts als ein kleines Küken. Sein Zyklus dauert selbst im Vergleich mit anderen Takuri sehr lange. Doch eines Tages wird es zweifelsohne wieder gewaltig groß sein.“

Aćh hörte nicht mehr zu. In ihrem Kopf hatte es soeben geklickt. Ijsdur hatte Siantari angelogen, mehrfach, demonstrativ. Laut und deutlich, als wolle er gehört werden. Er trug kein Takuri-Seil. Er führte keinen Feuertakuri. Allein konnte er nichts gegen Siantari ausrichten. Er brauchte sie, um dem Plan zu vollziehen. Versuchte er, Aćh von seinem freien Willen zu überzeugen, in der Hoffnung, dass sie ihn hörte? Konnte sie ihm vertrauen?

„Da!“, rief Siantari, und öffnete ihre Augen wieder, „Ja da fühle ich etwas Falsches in dir. Eine eigenartige Scheibe, aus der magische Ströme fließen, die sich an deinen Körper, ja, gar deine Seele klammern. Sie abschirmen und zugleich fesseln. Meine Magie rutscht von ihr ab wie Eisenkufen von einer Eisfläche. Darum bleibt mir dein Geist verborgen. Und sie anzufassen, wage ich nicht. Doch andere könnten sie bestimmt berühren.“

Zielstrebig wirbelte Siantari herum und bewegte sich zurück in Richtung des Alten Wehrturms. Wie üblich breitete sie ihre Arme aus und ließ sich von einer Schneewolke tragen, statt zu laufen.

Ijsdur trottete ihr hinterher. Aćh biss die Zähne zusammen, hielt ihr Schwert und ihr Takuri-Seil bereit und stapfte in angemessenem Abstand hinter, sodass sie nicht entdeckt wurde.

Vor dem alten Wehrturm waren vielleicht ein knappes Dutzend Andori in einer Reihe angeordnet. Beinahe alle schienen Kinder zu sein. Ein Schulausflug, unglücklich geendet? Sie duckten sich hinter die groben Mauerreste des Wehrturms und kuschelten sich aneinander, in der Hoffnung, Wärme zu finden.

Dies war das Auge des Sturms. Der Wind hatte abgeflaut. Weit über sich konnte sie gar einen Flecken Himmel erkennen. Es wurde Nacht. Sterne funkelten. Und ... ein weit entfernter rötlicher Funke verriet ihr, dass Turr sie nicht verlassen hatte. Er drehte weit, weit über ihr seine Kreise.

Aćh duckte sich hinter einen etwas weiter weg liegenden großen Felsen, um nicht gesehen zu werden. Doch die Eis-Dämonen kümmerten sich nicht um sie.

„Du!“, schnauzte Siantari die erstbeste Gefangene an. Die Eis-Dämonin zerteilte ihre Fesseln. „Ich will, dass du etwas aus der Brust des Eis-Dämons vor dir ziehst.“

„Was?!“

„Ich werde dir genau zeigen, wo. Lange hinein und ziehe die metallene Scheibe heraus, die in ihm steckt.“

„Was ... wie?“, stammelte die kleine Andori hilflos weiter. Siantari schlug ihr auf den Hinterkopf, zeigte verlangend auf eine bestimmte Stelle in Ijsdurs Körper und machte eine Geste des Herausziehens. Die Gefangene streckte ihre Hand aus und griff nach Ijsdur. Seine schneeige Haut brach unter der Berührung.

Es war Irils Scheibe, die Ijsdur schützte. Selbst wenn er zuvor Siantaris Ziele nicht geteilt hatte, stünde er gleich wieder unter Siantaris Bann. Jetzt war ihre letzte Chance.

Aćh brüllte aus vollem Leibe: „ADVARIA, TURR! ADVARIA MEZA!“ Sie stürzte aus ihrem Versteck hervor und rannte auf die beiden Eis-Dämonen zu, zum Himmel hoffend, dass Turr sie erhört oder wenigstens erspäht hatte.

Er hatte.

Turr stürzte aus den Wolken hinaus. Siantari wirbelte herum und hob ihre Hand. Ein Eisklotz, größer als ein ausgewachsener Zwerg, fegte den Feuervogel vom Himmel. Es knirschte unangenehm, als der Klotz mit Turr gegen den alten Wehrturm krachte und dabei einige lockere Steine aus der Ruine löste. Turr war nicht mehr zu sehen.

Hoffentlich hatte er als Ablenkung gereicht.

Die kleine Gefangene duckte sich verängstigt zu Boden.

Aćh erreichte Siantari, das goldene Schwert zu einem tödlichen Schlag ausgeholt. Sie stach zu. Siantari fegte die Waffe verächtlich aus Aćhs Hand und blickte sie regungslos an. Eine Kälte überfiel Aćhs Glieder. Kraftlos stürzte sie zu Boden.

Da kam Ijsdur ins Spiel. Er bewegte sich blitzschnell, manifestierte ein Eisschwert und stach damit in Siantari hinein. Oder versuchte es zumindest. Schon beim Aufprall zerstieb das Schwert in einen dünnen Nebelschleier, der keine Schneckenmaus verletzen könnte.

Siantari drehte sich schwebend zu Ijsdur um, die Andeutung eines Lächelns auf ihrem eisigen Gesicht.

„Du wagst es?! Ich bin die Herrin des ewigen Eises. Was meinst du, woraus du gewachsen bist? Du bist MEIN!“

Sie streckte ihre Hand gebieterisch aus. Ijsdurs Körper explodierte in eine Wolke aus Schneeflocken. Als seine Stimmbänder sich in Wassertröpfchen auflösten, brach sein Schrei ebenso schnell ab, wie er begonnen hatte.

Siantari schüttelte ihren Kopf und wandte sich ab, zurück zu Aćh, welche mit klammen Fingern ihr Schwert aufklaubte, als sie schon wieder die eisige Kälte übermannte.

„Heda!“, rief eine schwache Stimme. Siantaris Augenbrauen hoben sich kaum merklich, als sie sich überrascht zu Ijsdur zurückdrehte.

Während der größte Teil von Ijsdurs Schneekörper als lockerer Haufen auf dem Boden verteilt lag, wirbelte ein Nebel aus Eiskristalle um ein Konstrukt, das man mit etwas Fantasie als ein schneeiges Skelett bezeichnen konnte. Stetig rissen Siantaris Sturmwinde Fetzen heraus, doch stetig fügten sich aufs Neue Schneeflocken und Wasserwogen wieder ein, um die Lücken zu schließen. Und in der Mitte dieses unförmigen Schneeskeletts schwebte eine metallene Runenscheibe, aus der blendend weißes Licht schien.

Ein durchscheinender Mund öffnete sich und Ijsdurs klirrende Stimme sprach: „Das ewige Eis ist nicht nur dein. Wir sind freie Eis-Dämonen und du ka ...“

Siantari versetzte dem durchscheinenden Schneewirbel, der Ijsdur war, einen verächtlichen Tritt. Erneut pulverisierte sich sein gesamter Körper und fiel zu Boden. Und erneut flossen Schnee, Wasser und Eis wie von Zauberhand um die Runenscheibe zusammen, um Ijsdurs Körper zu rekonstruieren.

„Genug“, zischte Siantaris eiskalte Stimme. Sie hob ihre Arme. Der gesamte Schnee der Umgebung stieg in die Höhe. Der sich zusammensetzende Ijsdur wurde nicht mehr auseinandergerissen, sondern von Siantaris Magie geleitet zu einem Körper zusammengesetzt. Ein Körper, der strampelnd in der Luft hing.

Da hatte Aćh endlich wieder ihre Kraft gefunden. Sie trieb ihr goldenes Schwert glatt durch die Dämonin hindurch. Kurzzeitig ließ Siantaris Kontrolle über Ijsdur nach, als ihr Torso von ihrer Hüfte glitt. Dann fing sie allerdings sowohl sich selbst als auch Ijsdur. Ihr Körper verschloss sich wieder. Ijsdur wurde erneut von magischen Winden in die Höhe gehoben. Siantari drehte sich erneut zu Aćh herum und strafte sie mit einem zornigen Blick.

Mit bloßen Händen griff die Dämonin nach dem Schwert der Takuri-Hüterin. Raureif überzog es. Dann zerbrach das schockgefrostete Metall wie mürbes Holz unter ihren Fingern. Stechender Schmerz durchzuckte Irils Arm.

Aćh presste die unterkühlte, nutzlose Hand an sich. Sie sank aussichts- und hoffnungslos zu Boden. Ijsdur hing gefangen in der Luft. Was konnten sie beide schon gegen eine solche Gefahr tun?

Ein Blitz zuckte über den Himmel, hinein in Ijsdurs ausgestreckte Hand und sprang von dort auf Siantari über. Der Donner war betäubend laut. Die Eis-Dämonin wurde in den Rücken getroffen und einige Schritte über Aćh hinweggeschleudert, ehe sie sich fing. Mehrere kleinere Blitze zuckten aus ihrem eigenen Körper. Dann leitete sie den Eisblitz auf Aćh weiter, welche noch weiter zurückgeschleudert wurde, gegen den alten Wehrturm knallte und zusammensackte.

Ijsdur wehrte sich tapfer, doch nun, wo die gesamte Aufmerksamkeit Siantaris auf ihm lag, hatte er keine Chance mehr. Seine Gliedmaßen verdrehte sich unnatürlich in alle möglichen Richtungen, während sein Körper mehrfach zu Boden geschleudert wurde.

Da traf Siantari ein Schuh in die Seite, der beinahe ein Loch in ihren Schneekörper riss. Mehr verwirrt als verletzt blickte die Dämonin zu ihren Geiseln um. Welche von ihnen hatte es gewagt, sich gegen sie zur Wehr zu setzen?!

Eine der Gefangenen, eine ältere Dame, wohl eine Aufseherin der vielen andorischen Schulkinder, strampelte gerade damit, ihren zweiten Stiefel auszuziehen und damit nach Siantari auszuholen. Einen Augenblick lang war Siantari verwirrt. Wie konnte eine derart alte Menschenfrau die Stärke aufbringen, sich gegen sie zu stellen?! Beinahe amüsiert schleuderte die Eis-Dämonin einen Eiszapfen auf die Törichte und befreite sie von ihrem jämmerlichen Leben. Das farbige Leuchten in den Augen der Alten erlosch.

Sie ließ eine Schneewelle auf die restlichen Gefangenen niederregnen.

Die Kälte traf auch Aćh und weckte sie aus unruhigen Gedanken. Aćh schüttelte sich und versuchte, das Ringen in ihren Ohren zu unterdrücken. Ihr rechter Arm pochte. Jedes Mal, wenn sie mit einer Eis-Dämonin kämpfte! Beim letzten Mal hatte gelernt, ihr Schwert mit einer beliebigen Hand zu führen. Doch nun hatte sie gar kein Schwert mehr.

Sie orientierte sich: Schnee, Stein, zersplittertes Schwert, Geiseln, ein regloser Takuri. Sie befand sich am Alten Wehrturm, sie musste kämpfen, sie musste Ijsdur helfen, jetzt, jetzt, JETZT!

Aćh kraxelte zum sterbenden Turr. Er durfte sich noch nicht jetzt in ein Küken zurückwandeln! Sie ließ von ihrer Kraft hineinfließen, um ihn zu beruhigen, um ihn zu stärken. Was sie jetzt brauchten, war ein Takuri in der Blüte seines Zyklus.

Kälte durchzuckte Aćhs Glieder. Sie kippte zur Seite, doch Turr stieg in die Höhe. Ein goldener, feuriger Glanz umgab ihn. Stolz breitete er seine Flammenflügel aus. Ein roter Schein stieß daraus hervor, trieb Eis und Schnee zurück und zwang Siantari, ihren Blick abzuwenden.

Ijsdur nutzte die Gelegenheit, um Siantari anzuspringen und zu Boden zu zwingen. Zwei dämonische Geweihe verhakten sich. Die beiden Eis-Dämonen rollten strampelnd über den Boden. Dann schaffte Ijsdur es, Siantari in einen Schwitzkasten zu kriegen.

Aćh stolperte hinzu, zückte ihr Takuri-Seil, rannte hoch und umschlang Siantaris zappelnden Körper damit. Das Takuri-Seil glühte auf und brannte sich leicht in Siantari hinein.

Der Sturm um den alten Wehrturm legte sich schlagartig.

Die Herrin des Kuolema-Gebirges war gefesselt.

Ijsdur stolperte zurück. Denn Turr hatte nicht abgelassen, feurige Wärme auf die Dämonen zu werfen. Während Ijsdur sich die schmelznasse Stirn abwischte und kühlen Abstand hielt, erlitt Siantari die volle Hitze eines Takuri in der Blüte seines Zyklus.

Aćh und Ijsdur guckten einander erleichtert an. Dann konnten sie nicht anders, als aufzulachen. Siantari war gebannt. Der Wirbelsturm abgeflaut. Die Kinder befreit.

„Verzeih mir, Ijsdur, ich hätte dir vertrauen sollen“, flüsterte Aćh.

Ijsdur winkte ab. Ihn sollten solche Sachen schon lange nicht mehr schmerzen. „Lass die Vergangenheit vergangen sein. Ich hätte dich nicht drängen sollen. Belassen wir es dabei und lernen wir daraus.“

Ein Schrei holte sie zurück in den Moment. Jetzt, wo der Schneesturm Siantaris sich gelegt hatte, konnten sich ihre Gefangenen befreien, aus dem Schnee freischaufeln und aufwärmen. Dabei war auch die Leiche derjenigen ausgegraben worden, welche der Eis-Dämonin todesmutig einen Stiefel angeworfen und den Helden so die nötige Zeit verschafft hatte. Es war eine alte Dame, elegant gekleidet, vermutlich eine große Gelehrte. Zumindest einen anderen Lehrer hatten die jungen Andori bei sich gehabt, und dieser hatte nun seine größte Mühe damit, die Kleinen aus dem Schnee zu bringen und von der Leiche ihrer Lehrerin fernzuhalten.

Während die drei Erwachsenen sich um die Schüler kümmerten, bemerkte Ijsdur ruhig: „Wir konnten nicht alle retten.“

„Aber wenn wir nur ein klein wenig schneller gewesen wären ... Wenn ich nicht davon gerannt wäre ...“

„Natürlich können wir uns bessere Verläufe der Vergangenheit vorstellen. Aber auch schlechtere. Mehr als daraus lernen können wir nicht. Und zum Heldendasein, so wenig ich davon verstehe, gehört auch, Verluste einzustecken. Aufzugeben, weil man nicht perfekt handeln kann, scheint mir nicht sinnvoll.“

„Ich redete nicht davon, aufzugeben“, sagte Aćh traurig, „Aber es ist durchaus wichtig, dieses Gefühl der Schuld zuzulassen. Sonst geben wir uns mit allzu leicht mit unseren Handlungen zufrieden, ohne darüber nachzudenken, was wir hätten besser machen können.“

„Wir wissen doch, was wir hätten besser machen können. Und wir können uns gerne Zeit zum Trauern nehmen. Keine Trauer kann schließlich ebenso schädlich sein wie eine ewige Trauer. Doch nun können wir unsere Zeit besser nutzen. Nun gilt es, einen Eis-Drachen von der Rietburg fernzuhalten.“

Beider Blicke wanderten in Richtung Norden, zur Rietburg, auf deren Zinnen sich immer noch ein Eis-Drache räkelte. Die Kultisten konnten nicht ernsthaft vorhaben, das Gemäuer für sich selbst zu besetzen. Oder?

„Iril und Barz sollten schon nun bei der Burg angekommen sein, oder? Wie es ihnen wohl geht?“

„Iril? Die Runenmeisterin?“, fragte eine leise Stimme. Ein von Kälte bläulich angelaufener Junge krabbelte herbei und stellte sich als Jandro vor. „Ich kenne Iril. Sie kam in der Schule vorbei und zeigte uns ihre Runenscheibe. Was ist mit ihr? Geht es ihr etwa auch wie meinem Vater? Wie unserer Hohen Gelehrten? Wird sie auch nie wieder zurückkommen?“

Tränen wallten in den Augen des Jungen auf. Aćh kniete sich neben ihm nieder und sprach ihm die wenigen andorischen Worte des Trostes zu, die sie bislang gelernt hatte.

Da fiel etwas auf. Neben der Leiche der alten Gelehrten lag eine kleine Schatulle, die der Dame aus der Tasche gefallen war. Sie hatte sich geöffnet. Ein kleiner, unscheinbarer, gelb glühender Stein war daraus gerollt.

Aćh hob den Stein in die Höhe. Eine Rune war darin eingeritzt, die an einen Torbogen erinnerte. Und in der Schatulle fand sie zwei weitere solcher Steinchen. Einen grünen und einen blauen. Runensteine, hatte Iril gesagt.

„Die hier sollten bei Iril sein“, murmelte sie zu Ijsdur.

„Prima, Turr soll sie doch bringen!“

„Turr ist schlecht darin, etwas Entzündliches zu transportieren. Und wir brauchen ihn hier, um Siantari zu schwächen.“ Aćh deutete zur Herrin des Kuolema herüber, welche schmelzend vor Turrs ausgebreiteten Flügeln stand und sie hasserfüllt anblickte.

Mit erhobener Stimme wendete sich Aćh an die Befreiten.

„Hat jemand hier einen Falken?“

Ein Bauer mit eleganter Pferdeschwanzfrisur meldete sich zu Wort. „Jawohl! Ich bin zwar kein Falkner, aber mein Bruder ist einer. Ein Lehrer an der Rietburg. Ich sprang heute ein für ihn.“

Ohne weitere Ausführungen zog der Bauer eine kleine eiserne Pfeife hervor – diese hatte an einer Kette um seinen Hals gehangen – und blies hinein. Es dauerte nur eine Minute, da stieß ein mächtiger Falke vom Himmel herab, landete auf seinem nackten Arm und grub seine Krallen hinein. Der Bauer hauchte kurz vor Schmerzen auf.

„Mein treuer Fleo bringt eure Post zielsicher zum Ziel. Egal wie weit, egal zu wem. Sprecht, was wollt ihr wohin senden?“\bigskip







Das Verlies der Rietburg war dunkel, still und verlassen. Nur selten wurden Andori hier in der Kaverne eingesperrt. Öfter wurde als Strafe für Vergehungen eine Verbannung von der Burg verhängt, was in jeder Zeit einem Todesurteil gleichkam.

Doch hin und wieder musste es auch Personen gegeben haben, die eine solche Gefahr für die Öffentlichkeit dargestellt hatten, dass es gerechtfertigt gewesen war, sie hier einzusperren. Düstere Diebe, Schattige Hexen und Dunkle Magier – wobei zumindest letztere ein außergewöhnliches Talent dafür hatten, sich nicht lange von einer Gefängniszelle halten zu lassen.

Oder gerade andersherum – es musste einige Personen gegeben hatten, deren Vergehen keine Verbannung von der Rietburg gerechtfertigt hatte, die aber immer noch kurzfristig einen Sicherheitsabstand zwischen sich und den restlichen Andori brauchen konnten, bis beispielsweise ihr Rausch abgeklungen war.

Sagramak und ihre Kultisten hatten von irgendwoher den Schlüssel zu den Kerkergewölben ergattert und Iril und Barz kurzerhand dort eingesperrt. Sie waren die einzigen. Hatten alle anderen Krieger bis zum Tode gekämpft? Kämpften sie immer noch da draußen, während die Kultisten die Burg nach Thorald und Taroks letzten Drachenknochen durchsuchten?

„Keine Sorge, ich bin dir nicht böse“, hatte Sagramak beim Einsperren fröhlich zu Barz gesagt. „Nicht allzu sehr. Du wirst überleben. Vielleicht verpasse ich dir noch einen Denkzettel, sobald wir die Drachenknochen haben. Aber nicht Gewaltigeres. Wir wollen dich nicht leiden lassen. Das kommt noch früh genug nach deinem Tod, wenn die hohen Drachen dich richten. Die Runenmeisterin hingegen ... sie werden wir mitnehmen und schauen, dass diese mächtigen Runen nicht mehr gegen uns eingesetzt werden, sondern für uns.“

Dann war sie abgezogen, um die Rietburg nach Thorald und Taroks letzten Knochensplittern zu durchsuchen. Das war vor einer halben Stunde gewesen.

Barz brummelte vor sich hin: „Wir hätten Ijsdur mitnehmen sollen. Er hätte uns einen schönen Eisblitz-Tunnel hier raus zaubern können.“

„Nicht das Ausbrechen ist das Problem, sondern die Stärke unserer Gegner“, stellte Iril klar.

„Ich Idiot!“, nervte sich Barz, ohne auf Irils Antwort zu achten. Nicht zum ersten Mal in der letzten halben Stunde. „Ich hätte das Vorhersehungspulver bei Sabri lassen und das Umwandlungspulver mitnehmen sollen. Ich hätte uns starke Schilde schenken können. Oder noch besser, mir einen neuen Bogen. Wir hätten aus deinem Hammer ...“

„Mein Hammer wird doch nicht einfach so umgewandelt“, entrüstete sich Iril, „Und überhaupt, Was-wäre-wenn-Szenarios sind sehr interessant durchzudenken, aber wir beide können unsere Zeit hier sicher sinnvoller nutzen.“

„Was, während wir darauf warten, dass uns jemand retten kommt?“

„Nein, während wir uns befreien! Wir sind zwar in einem uralten Verliesgewölbe eingesperrt, aber wir sind allein. Unbewacht. Wer weiß, was wir alles anstellen können.“

„Jedenfalls nicht kämpfen. Mein Bogen wurde zerschnitten. Und liegt ohnehin noch draußen.“

„Ich schenke dir einen neuen, wenn das alles vorbei ist. Aber nur, wenn du aufhörst zu jammern und mir hier aushilfst. Wo mein Runenhammer? Ich sah ihn dich führen, als Nehamal mich überwältigte.“

„Spornwaldreck! Den habe ich auch draußen verloren.“

„Kein Problem.“

Iril suchte sich einen Spalt im vergitterten Fenster, durch das das Licht der untergehenden Sonne reinstrahlte. Unter den Strahlen glühten einige Runen auf ihrem Arm schwach auf. Iril streckte ihre Hand aus und murmelte etwas zwischen zusammengekniffenen Zähnen hervor.

So verharrte sie eine Weile. Dann drehte sie sich herum. Aus der Ferne vernahm man ein Poltern und Knarren, ein Knirschen und Rumoren. Und dann, unregelmäßig stockend, rutschte der Runenhammer zwischen den Eisenstäben der Gittertür hindurch und in den Raum hinein. Ein Glück, dass niemand der Kultisten auf ihn geachtet hatte.

Iril packte den Hammer, welcher prompt zu glühen begann. Im schwachen Schein des Hammers wirkte Barz‘ in Schatten getauchtes Gesicht ganz gruselig.

„Wie machst du, dass er so leuchtet?“, fragte er neugierig, „Als ich ihn hielt, war er einfach ein ganz gewöhnlicher Hammer.“

„Das ist auch gut so. Dein Menschenblut könnte der Dunklen Magie in seinem Innern erliegen, wenn du seine Macht nutztest.“

„Ich glaube, zu verstehen. Bei meinen magischen Pulvermischungen spüre auch ich die Last der Magie und bin stets darauf bedacht, das Gleichgewicht zu wahren. Doch bedeutet das für mich oft, mich den finsteren Künsten ganz zu entsagen, um gar nicht erst in Versuchung zu kommen. Du meinst, Zwerge wären gegen ihre flüsternden Stimmen immun?“

„Zumindest immuner als die meisten Menschen“, meinte Iril. „Oder hast du schon einmal einen zwergischen Dunklen Magier gesehen?“

„Die gäbe es sehr wohl, wenn es mehr Zwerge in Hadria gäbe.“

„Bist du sicher?“

„Nicht unsicherer, als du es sein kannst.“

Dabei beließen es die beiden.

Neben ein wenig Stroh und einem behelfsmäßigen Scheißeimer hatte ihre Zelle kaum etwas zu bieten. Iril machte sich auf zur vergitterten Tür und spähte in den dunklen Gang dahinter. Es war Nacht geworden und die Beleuchtung hier war selbst bei Tage äußerst dürftig. Wenn sogar ihre Zwergenaugen kaum erkennen konnten, wie es weiter hinten im Gang aussah, musste es für Barz wahrlich stockfinster sein. Wie so oft wünschte sie sich einen Casamatuc, um solche Türen gewaltlos zu öffnen. Doch diese seltenen Zwergenwerkzeuge wurden nicht jedem anvertraut.

Das Schloss der Gittertür zerbarst unter einigen Schlägen von Irils Hammer. Die beiden Insassen horchten, ob ein Drachenkultist den Lärm gehört hatte und nachsehen kam. Nach fünf Minuten der Stille trauten sie sich in den Gang hinaus.

Zu ihrer Linken und Rechten lagen einige weitere vergitterte Zellen. Dahinter führte ein natürlicher Höhlenausgang in den dunklen Burghof der Rietburg, natürlich ebenfalls von einem Gitter verschlossen. Sterne funkelten vom Himmel herab. Ein regelmäßiges dumpfes Stampfen zeugte vom dahinter umherlaufenden Eis-Drachen. Der war wohl auch der Grund, warum Sagramak es nicht für nötig befunden hatte, Wachen aufzustellen. In ihrem jetzigen Zustand waren Iril und Barz ihm alles andere als gewachsen.

Zu ihrer Rechten führte ein verschlungener Gang tiefer ins unterirdische Kellergewölbe der Rietburg. Iril wusste aus ihrer Zeit in den letzten Wochen, dass der Gang in ein Labyrinth von Sackgassen mündete. Hier befanden sich vor allem Vorratskammern und natürlich andere Kerkerzellen. Man konnte der Burg so nicht entkommen. Und eigentlich wollte sie das auch nicht, solange es noch eine Gefahr zu bändigen gab.

„Was starrst du so, Iril? Gucken wir uns lieber noch die anderen Zellen an oder nicht?“

Iril nickte stumm.

Es war überraschend interessant, das Innere der verschiedenen Kerkerzellen zu inspizieren. Zahlreiche Kritzeleien zierten die Wände. Letzte Botschaften von Dieben und Halunken, die nach einem Aufenthalt hier aus der Burg verbannt würden. Man sah einfache Spiele zwischen Eingesperrten, deren Verläufe mit Löffelstielen in die Wand geritzt worden waren. Hier ein Gedicht auf verführerischen Met, da ein Schmähspruch auf die Wachen. Eine Zelle war anders, tiefschwarz gefärbt, die Runen mit Wänden verstehen, die Iril nur vom Hörensagen nach kannte. Hier hatte jemand die Dunkelheit selbst einzusperren versucht. Oder eine Schattenhexe.

Weiter hinten fanden Iril und Barz auch einen großen unterirdischen Raum, in dem zahlreiche Weinfässer der Burg gelagert wurden. Offenbar diente diese in die tief in den Hügel geschlagene Kaverne auch als Lager für den Keltermeister der Rietburg. Doch das schmiedeeiserne Tor war fest verschlossen. Was für edle Tropfen dahinter liegen mussten, nur für den Prinzen und seine engsten Vertrauten bestimmt.

„Kommt nicht in Frage, dass wir uns jetzt betrinken!“, rief Barz kopfschüttelnd, „Ihr Zwerge und euer Durst.“

„Nicht alle Zwerge fahren derart auf betäubenden Trunk ab, wie du meinen magst“, protestierte Iril. „Sieh doch. Da hinten an der Mauer glänzt etwas.“

Barz blickte ins Dunkel und grinste plötzlich auf. „Kannst du die Tür hier aufschlagen? Es wird es definitiv wert sein.“

Irils Hammer brach das Schloss, während Barz Ausschau hielt dass kein Drachenkultist auf die Idee kam, nach dem Ursprung dieser Geräusche zu suchen.

Kaum war die Tür offen, raste Barz an den gestapelten Weinfässern vorbei an die Rückwand der Höhle. Durch eine klitzekleine Öffnung am oberen Rand der Wand warf der Mond sein silbriges Licht in den großen unterirdischen Raum. Und ließ eine unscheinbares, scheinbar aus der Wand sprießendes Gewächs glitzern.

Anerkennend nickte Barz die milchig weißen Beeren an der fremdartigen Pflanze an. „Mondbeeren. Besonders leicht in den ersten Sonnenstrahlen eines neuen Tages zu finden, wenn sie leuchten. Manchem, der die Beeren isst, verleihen sie neue Kraft und manch anderem schien es, als bliebe sie Zeit stehen.“

„Super“, freute sich Iril, und schwang ohne bestimmtes Ziel ihren Hammer herum, „Zeit ist das, was wir am besten gebrauchen können, um uns für den bevorstehenden letzten Kampf zu stärken.“

„Oh, wenn es nur um Zeit geht, kann ich auch aushelfen“, rief Barz. Er löste ein kleines braunes Pulversäcklein von seinem Mantel und schüttelte es.

„Meditationspulver. Kann uns mehr Zeit verschaffen. Ich kann es nicht nutzen, wenn zu viele Personen um mich herum sind – dann verwaschen sich alle ihre Präsenzen irgendwie zu einem neutralen Hintergrundrauschen – aber, wenn nur wir zwei in einem abgeschiedenen Kerker herumlungern, kann ich mich prima auf deine Präsenz konzentrieren und dir mehr Zeit schenken. Nur, dass wir mit mehr Zeit keine größere Chance gegen einen verdammten Riesendrachen haben.“

„Nicht aufgeben, lieber Barz. Es ist ein Anfang.“

Iril blickte in die nächste Zelle und untersuchte im düsteren Schein ihres Runenhammers deren Ecken. Es fühlte sich an, als locke sie ein unsichtbares Seil dorthin. Sie folgte dem Gefühl und gelangte an eine kleine Fensteröffnung, aus der einige Köpfe über ihr ein heller Lichtschein ins Dunkel hineindrang.

Etwas flatterte am Fenster vorbei. Iril zuckte zurück. Ein Schatten ließ sich davor nieder. Vogelaugen starrten auf sie herunter. Dann plumpste es, als etwas durch das Fenster auf den Boden geworfen wurde.

Iril kniete sich nieder und tastete danach. Dann lachte sie auf und krümelte etwas aus dem Dreck hervor. Eine hölzerne Schatulle, kaum eine Zwergenhand breit, deren Oberfläche unverkennbar das Zeichen der Hohen Gelehrten der Rietburg trug.

Iril öffnete sie und stieß einen erleichterten Lacher hervor. Eine fröhliche Nachricht von Aćh! Und als Iril den weiteren Inhalt der Schatulle untersuchte, fand sie drei unscheinbar wirkende Steine ...

Die drei Steinchen lagen auf einem schwarzen Tuch. Ein gelbes, ein grünes und ein blaues. Die uralte, in ihnen gespeicherte Magie ließ sie leise vor sich hin brummen.

Runensteine.\bigskip







Es dauerte bis tief in die Nacht hinein, bis Sagramak und ihre Drachenkultisten endlich Thorald fanden. Er hatte sich in Begleitung Hauptmann Armonds in einem Gebäude an der Burgmauer verschanzt und seine Position erst aufgegeben, nachdem der Eis-Drache Sagrakdur das mit Rietgras bedeckte Dach eingerissen hatte.

Hoch erhobenen Hauptes schritt Thorald aus dem Gebäude hervor und warf sein Schwert vor Sagramaks Füße. Armond tat es ihm nach und erinnerte den Regenten daran, auszuhandeln, dass er die Drachenknochen nur übergeben würde, wenn die restlichen Andori verschont würden. Sagramak stimmte lachend zu.

Die Nacht war bereits lange angebrochen und ihre finsteren Schatten schmiegten sich an die Mauern der Rietburg, als Thorald den Thronsaal erreichte. Der rote Mond schien durch die Fenster hinein, als er an einer Klappe hinter dem hölzernen Thron herumfummelte. Das Versteck war größtenteils leer, denn der Bruderschild war zum Glück gerade bei anderen Helden im Einsatz. Thorald langte hinein und produzierte einen unscheinbaren samtenen Sack, dessen Inhalt leise klapperte.

Hoch erhobenen Hauptes schritt Thorald wieder aus dem Thronsaal hervor und übergab die Drachenknochen Sagramak.

Hinter ihr standen triumphierend die übrigen Drachenkultisten, allen voran Nehamal und Fir. Das Ziegenwesen trug einen provisorischen Verband um seine Brust und schimpfte übel über Blutflecken auf seinem edlen Gewand. „Noch wird diese Welt dich nicht los, Fir, du tapferer Bursche mit deinen exzellenten Heilkräften! Doch welch Schande, dass dieser Pfeil ein Loch in dein edles Gewand reißen musste. Wollen die Drachen etwa, dass du dies selbst flickst?“

Über allen dreien schwebte der Schwarze Herold, sein langes Schwert still in einem eisernen Griff haltend. Die Spitze streifte beinahe Sagramaks Haarschopf. Ihren Helm hatte sie im Zweikampf mit Iril verloren.

Die Schamanin packte aus dem soeben erhaltenen samtenen Säcklein die Kette mit Taroks Knochensplittern, die sie vor einigen Wochen noch selbst getragen hatte.

„Brav, brav, werter Prinz“, lächelte sie und tätschelte Thoralds Wange herablassend. Thorald lief unter seinem Bart rot vor Wut an, beherrschte sich allerdings.

„Und nun tut, was ihr verspracht“, sprach er stolz, „Zieht ab mit eurem Preis. Lasst uns in Frieden unsere Toten bestatten und die Schäden beheben.“

Das mitleidige Lächeln auf Sagramaks Gesicht wurde zu einem finsteren Grinsen. Noch immer starrte sie gierig Taroks letzte Knochen an. „Ihr hättet uns nicht anders behandelt, oder? Warum sollte ich Euch verschonen? Ihr, die Ihr euch einen Dreck um unsere Anliegen schertet! Ihr, die Ihr am liebsten hättet, wenn es uns gar nicht gäbe!“

Nehamal zog sie zur Seite: „Tu das nicht, Sagramak. Gibt ihnen nicht die Genugtuung ...“

„Der Prinz soll fallen!“, zischte Fir, „Durch seine Missachtung unserer Bedürfnisse hat er es verdient. Eine Mahnung an all unsere Feinde soll er sein!“

Sagramak blickte zischen beiden hin und her und wandte sich an Thorald. „Wenn ich euch verschone, werdet ihr uns jagen, sobald sich eine Gelegenheit ergibt, oder uns in Ruhe lassen?“

„Irrelevant. Ihr gabt uns unser Wort, uns in Frieden zu lassen, wenn wir euch die Knochen gäben!“, rief Thorald.

„Eines jeden Wort kann mindestens einmal gebrochen werden. Wenn es sich am besten lohnt“, sprach Sagramak. Ein bösartiges Grinsen überfiel ihr Gesicht. Rotes Feuer glomm in ihren Augen auf. „Und wer seid Ihr, um über gebrochene Worte zu diskutieren? Gab Euer verräterischer Vater nicht sein Wort, dass er die mächtigen Schilde an die Zwerge zurückgäbe?“

„Das zählt nicht! Das war, bevor Rudnar ... nein, darüber will ich gar nicht reden. Erst recht nicht mit Euch. Ich bin nicht mein Vater. Ich erwarte von Leuten, die mit mir verhandeln, dass sie ihr Wort nicht brechen, da ich meine Versprechen auch stets halte.“

„Wir werden nie wieder miteinander handeln“, sprach Sagramak.

„Vielleicht sollten wir Gutwille walten lassen“, meldete sich nun die tiefe Stimme Sagrakdurs. Der Eis-Drache hatte seinen Kopf nachdenklich in seine Tatzen gelegt und brummelte: „Es ist bereits viel Leid über diese Menschen gebracht worden im Namen Taroks. Wir haben, was wir wollen. Wir können gute Gewinner sein.“

Sagramak schüttelte enttäuscht ihren Kopf.

„Dafür, dass ich beinahe mein halbes Leben lang deine Seele in mir getragen habe, hätte ich gedacht, dich besser zu kennen. Sagrak, was ist aus deinem Blutdurst geworden? Aus deiner Rachelust?“

„O Schamanin, du magst von der Seele Sagraks erfüllt gewesen sein, dessen Blut brannte voller Rache auf die Welt, die er viel zu früh verlassen musste. Doch ich bin Sagrakdur. Meine Glut brennt nicht so heiß. Ich dürste nur noch nach dem ewigen Eis.“ Tief schnaufte der Eis-Drache durch. „Diese Welt ist so verwirrend. Nichts ist mehr wie beim Unterirdischen Krieg. Die Drachen sind alle tot. Die Trolle sind an unserer Seite statt unsere Gegner. Die elenden Agren, Zwerge und Krahder leben zurückgezogen in ihre Reiche. Doch überall haben sich diese Menschen ausgebreitet. Sie alle werden ohnehin vergehen, wenn das ewige Eis erst einmal diese Lande überzieht. Bis dahin sollen sie doch ruhig noch weiter bestehen dürfen.“

Der Schwarze Herold schwebte zu Sagramak herunter und flüsterte ihr etwas ins Ohr. Sagramak nickte und ließ ihre Stimme erschallen, während der Herold wieder etwas über ihr schwebte.

„Ich sehe, wie es ist. Mein lieber Sagrak, du warst mir lange Zeit ein treuer Begleiter und ich dir eine treue Dienerin. Doch nun ist es an der Zeit, Abschied zu nehmen.“

„Was ist das für ein Unterton in deiner Stimme?“, fragte der Eis-Drache argwöhnisch.

„Am Ende bist du doch nur ein Abklatsch des Gewaltigen. Dein Vater wird deine Bedenken nicht teilen.“

„Ja!“, zischte der Schwarze Herold.

Der Himmel rötete sich in der Ferne, Anzeichen des kommenden Sonnenaufgangs.

Sagramak drehte Taroks Fußknochen triumphierend in ihren Fingern. Dann streckte sie ihre Hand aus. Telekinetisch geführt schwebten die Fußknochen nach vorne. Beherrschend hielt Sagramak ihre Hand ausgestreckt und gestikulierte zum riesigen Eis-Drachen neben ihr. Ihre telekinetischen Kräfte schlugen an.

„Nein! Tu nichts, was du bald bereuen ...“, jaulte Sagrakdur auf. Die Eiskristallkette in seiner Brust war winzig im Vergleich zu seinem Körper. Und doch schrie er auf, als die Eiskristalle telekinetisch aus seiner Brust gerissen wurden. Knackende Spalte breiteten sich entlang seines Körpers auf. Während Eis-Sagrak zerfiel, gebot Sagramak der schwebenden Eiskristallkette, sich vor Taroks Fußknochen zu versammeln.

Die Eiskristallkette summte und brummte. Wasser tropfte aus dem Boden in den Himmel, kondensierte aus der Luft, erstarrte zu Eis und Schneeflocken. Wie aus dem Nichts formten sich weitere Drachenknochen aus purem Eis und ordneten sich neben den echten an. Dann wurden sie von einer Schicht aus Eis und Schnee überdeckt, die zunächst Muskeln, dann Haut und schließlich messerscharfen Schuppen glich. Eis-Sagraks Überreste zerliefen zu Wasser, während der große Eis-Tarok immer größer und stärker wurde. So, als ginge die Kraft des kleinen Drachen in ihn über.

Die Eiskristallkette schwebte von Sagramaks ausgestreckter Hand geführt nach vorne und fügte sich nahtlos in den Nacken des riesigen Eis-Drachen ein, der vor Sagramak schwebte und zitternd ein- und ausatmete.

Schneeflocken wirbelten wild umher, als der Drache seinen vereisten Kopf schüttelte und aus Versehen mit seinem ungewohnten vielzackigen Dämonengeweih gegen ein Haus stieß. Gewaltige Schwingen wurden gespreizt, während eisige Blitze über den Himmel zuckten. Klirrende Dampfwolken bildeten sich vor der langen Zunge, die aus einem vielzahnigen Mund herabzüngelte. Dann wurde der Mund aufgerissen und die dahinter liegenden Stimmbänder entließen ein ohrenbetäubendes Gebrüll.

Tarokdur war erwacht.


















\newpage
\section{Der Zorn des Eis-Drachen}




Tarokdur war erwacht.

„Ich ... ich ...“, stammelte der gigantische Eis-Drache nach Worten, „Was ... was ist mit mir? Was ist mit ... Krahal? Die Stimmen der Altvorderen? Mein Zorn? Meine Wut? Mein Schmerz?“

„Alles Teil der Vergangenheit“, rief der Schwarze Herold triumphierend, „Willkommen zurück unter den Lebenden, mein Gebieter.“

Der gewaltige Schädel des Eis-Drachen bog sich auf seinem langen Hals, um den Herold ins Auge zu fassen. Tief und klirrend dröhnte die Stimme des Drachen in den Ohren der Anwesenden, ohne dass sein Maul sich geöffnet hätte.

„Ah, mein Herold. Mein treuster Diener. An dich ich erinnere mich gut. Doch irrst du dich. Ich bin nicht derjenige Tarok, den du anbetetest. Ich bin Tarokdur.“

„Pipifax! Trägst du nicht sein Aussehen? Teilst du nicht seine Erinnerungen? Du bist mir Tarok genug. Mit dir an unserer Seite können wir uns endlich an Andor rächen! Alles soll in Schutt und Asche liegen!“.

Tarokdur blinzelte zweimal verwirrt, ehe er seinen Kopf mit den überlangen Dämonengeweihen schüttelte und nickte.

„So sei es. Statt im Feuer zu brennen, sollen die Eindringlinge in mein Drachenland im Eis umkommen. Danach suchen wir die Reiche der verräterischen Zwerge auf und frieren ihre Tiefminen zu. Und dann reisen wir nach Krahd! Es gibt eine alte Rechnung mit den Riesen zu begleichen. Nomions Brut soll ein für alle Mal im Lavasee versinken. Der Enran soll von einer dicken Schicht aus Eis und Schnee bedeckt werden! Die ganze Welt soll leiden für das Aussterben der größten Spezies aus uralter Zeit!“

„Moment einmal“, warf Sagramak ein, „Wir Kultisten sind natürlich gerne dazu bereit, Taroks Wünschen nach unseren Möglichkeiten zu dienen. Doch wollen wir nicht die ganze Welt unter einer Eisschicht untergehen sehen. Das wäre ... exzessiv. So wollen es die anderen Drachenseelen nämlich auch nicht. Sagrak mag dieses Land.“

Demonstrativ hielt die Schamanin das rote Drachenrelikt aus Roteisenstein in die Höhe, in dem bis vor Kurzem Sagraks Seele geruht haben sollte – und es nun vielleicht wieder tat. Da Sagramak soeben selbst noch bedacht hatte, Thorald und die Rietburg dem Erdboden gleichzumachen, ging es der Schamanin wohl vielmehr darum, Macht gegenüber Tarokdur zu demonstrieren.

Nichtsdestotrotz sorgte sie dafür, dass Tarokdur seine Augen zusammenkniff und seinen Kiefer dem kleinen Menschen zuwandte. Welcher Wurm wagte es, seine Autorität zu hinterfragen?! Wer war so töricht, für seinen viel zu früh gestorbenen Sagrak sprechen zu wollen?

Sagramak fuhr fort und verlieh ihrer Stimme einen besänftigenden Klang: „Anstelle einen Völkermord zu begehen, wollen wir doch lieber den Bruderschild und ähnliche uns zustehende Artefakte von den Andori verlangen und uns dann wieder in unsere Bergwohnungen zurückziehen, fortan gut schlafend im Wissen, dass die Andori keine ernst zu nehmende Gefahr mehr sind. Und danach magst du dir immer noch Krahd vornehmen. Bei Nehal, wenn es nicht anders geht, magst du auch Prinz Thorald reißen. Und die Helden von Andor. Aber nicht sein ganzes Volk. Sie können nichts für die Taten ihrer Helden.“

Tarokdur starrte sie eisig und ausdruckslos aus blau glühenden Augen an. Hatte sie gerade ernsthaft ihm die Erlaubnis zum Töten zu verwehren oder zu erteilen versucht? Ihm, dem letzten und mächtigsten aller Drachen?!

„Es geht hier nicht nur um deinen Willen, Tarokdur“, rief Sagramak bestimmend, „Ich gab dir Leben, ich kann es dir auch wieder nehmen.“

Bedrohlich spreizte Sagramak die Hände, bereit, Tarokdur telekinetisch die Eiskristallkette aus der Brust zu reißen.

Ein tiefes, humorloses Lachen schüttelte den gewaltigen Körper des Eis-Drachen. Er blinzelte dem Schwarzen Herold zu. Der Herold, welcher bislang wortlos über Sagramak geschwebt hatte, ließ sich fallen und versenkte sein rostiges Schwert in Sagramaks Schädel. Die Schamanin blinzelte überrascht. Der Herold riss sein Schwert wieder heraus. Tarokdur fackelte nicht lange. Sein langer Schlangenhals beugte sich in die Tiefe und verschluckte die überraschte Schamanin samt Haut, Haar und Rüstung mit einem einzigen Schluck.

„Tu, was du nicht lassen kannst“, gluckste Tarokdur, „Doch tu es nicht mehr in diesem Leben. Man verhandelt nicht mit Drachen. Man folgt ihnen.“

Er warf den restlichen Drachenkultisten einen vielsagenden Blick zu. Diese sahen mit Furcht zum riesigen Eis-Drachen auf, der soeben ihre Anführerin in einem Happs verspeist hatte. Nehamal fasste sich als erster ein Herz und kniete schlotternd nieder.

„Gebietet über mich, o gewaltiger Gebieter.“

Fir tat es ihm gleich. Ausnahmsweise blieb das Ziegenwesen stumm.

Der Schwarze Herold kicherte, schwang sich auf Tarokdurs Rücken und machte es sich zwischen zwei Rückenstacheln bequem.

Sein Plan war geglückt, alle Gefahr beseitigt. Nun konnte er zurücklehnen und das Gemetzel der Andori genießen.\bigskip







Aćh zerrte am feurigen Takuri-Seil. Die gefesselte Siantari stolperte weiter. Turr drehte wie üblich seine Runden über der geschwächten Herrin des Kuolema-Gebirges und hinderte sie durch seine schiere Wärme, Kräfte zu sammeln. Turrs Hitze schmolz auch Teile der Schnee-Spur, den Ijsdur auf dem Weg durch das Rietgras hinterließ.

Turr und das Takuri-Seil waren die größte Beleuchtung in der finsteren Nacht.

Die unterkühlte Schulklasse hatten sie beim nächstgelegenen Gehöft ins Warme bringen können, wo ihnen müde Bauern heißen Cocoa servierten und ihnen Strohbettchen bastelten, in der Hoffnung, die traumatisierten Kinder eines baldigen Tages wieder in eine befreite Rietburg zurückbringen zu können.

Aćh und Ijsdur waren hingegen mit ihrer gefesselten Gefangenen weiter in den Norden aufgebrochen, auf dass die Rietburg auch möglichst bald befreit würde. Siantari hatte seit ihrem Aufbruch kein Wort mehr gesprochen und sich stolz stumm fortbewegt, für einmal nicht schwebend.

Ijsdur stapfte mit ein wenig Vorsprung zu den drei anderen in sicherer Kühle in Richtung Rietburg. Er starrte besorgt auf das nicht mehr allzu weit entfernte Gemäuer, dessen Silhouette kaum erkennbar war vor dem dunklen Sternenhimmel.

„Etwas tut sich!“, rief er zurück zu Aćh. Eine Zeit lang war der Eis-Drache der Drachenkultisten hinter den hohen Türmen der Festung nicht mehr zu sehen gewesen. Nun ragte wieder ein mit einem stattlichen Dämonengeweih besetzter schneeiger Drachenschädel neben dem Palas der Rietburg empor. Doch handelte es sich kaum um denselben Drachen. Dieser hier war um einiges größer, mit ausgeprägteren Rückenstacheln und einem unförmigeren Maul. Und als er seine löcherigen, gewaltigen Flügel ausbreitete und diese wie von einer inneren Magie zu leuchten begannen, da wusste Ijsdur, dass sie einer größeren Gefahr als zuvor bevorstanden.

Eine Hitzewelle kündigte an, dass Turr – und damit auch Siantari und Aćh – Ijsdur erreicht hatten.

Aćh zitterte: „Der ... der ist um einiges riesiger, als er zuvor aussah. Hat er an Stärke gewonnen?“

„Das ist nicht derselbe Drache. Ich befürchte, die Kultisten haben Taroks Knochen gefunden.“

„Vielleicht konnten sie eine friedliche Einigung erzielen und fliegen gleich davon?“

Ijsdur legte seinen Kopf schief. „Ich halte das für unrealistisch. Wir sollten uns lieber auf den schlimmsten Fall vorbereiten. Wir sollten davon ausgehen, dass Iril und Barz höchstwahrscheinlich versagt haben und Tarokdur kurz davor ist, seinen Zorn zu entfesseln.“

„Na dann, schnell, schnell“, spornte Aćh ihn an, rannte los und zerrte Siantari hinter sich her „Wenn wir die Rietburg rechtzeitig erreichen, können wir vielleicht noch etwas ausrichten. Wenn nicht, wird die Burg bald nur noch ein Schneehaufen sein. Und mit ihr das gesamte Rietland. Und danach ...“

Tarokdur öffnete sein Maul und spie einen gewaltigen Eisstrahl auf die Spitze des Königsturms. Steinblöcke wurden aus dem Turm geschlagen. Zwei Gestalten, die dort oben gestanden und Bögen gezogen hatten, wurden von einer Schneewelle ergriffen und vom Turm geschleudert. Aus der Entfernung wirken sie wie kleine Strohpuppen, die von einem Tisch zu Boden plumpsten.

Ijsdur blieb stehen und mahlte mit seinem Unterkiefer.

„Die Zeit reicht nicht. Wir kommen nicht mehr rechtzeitig zur Burg. Und selbst wenn – was wollen wir gegen einen Eis-Tarok anrichten?“

„Wir könnten Siantari dazu zwingen, ihn aufzulösen. War das nicht der Plan, warum wir sie mit uns schleppten?“

„Nicht genau. Siantari hat keinen Grund, unseren Anweisungen zu folgen. Für sie zählt nur das ewige Eis. Wir haben ihr nichts anzubieten außer ihre Freiheit, und die werden wir ihr nicht geben, da sie ist das größere Übel ist. Wir hätten vielleicht über die Drohung, Siantari zu vernichten, die Drachenkultisten zu etwas zwingen können. Aber nicht, wenn wir sie nicht mehr erreichen, ehe das Königreich dem Erdboden gleichgemacht wird. Nicht, wenn die Kultisten nun Taroks Willen folgen.“

Nun hielt auch Aćh an und drehte sich zu ihm um. Angst stand in ihr Gesicht geschrieben. Turr drehte weiterhin große Kreise über ihr. „Aber dann ... was können wir tun in der wenigen Zeit, die uns gegeben ist?“

Es gab nur noch etwas zu tun.

Es stand zu vermuten, dass Siantari die Kontrolle über diesen Eis-Drachen hatte, so wie sie alle Eis-Dämonen des Kuolema-Gebirges kontrollierte. Und wenn sie ihnen Kraft gab, wenn sie die Kraft ihres Willens und ihrer Magie war ...

Ijsdur stieß Siantari unsanft in den Rücken. Die Herrin des ewigen Eises ging vor ihm in die Knie.

„Ijsdur? Ijsdur, was machst du da?“, fragte Aćh argwöhnisch. Ihre Stimme wurde schriller. Ijsdur ignorierte sie und ließ seine rechte Hand von Eis überziehen.

Zum ersten Mal seit Stunden meldete sich Siantari wieder zu Wort. Kalt und emotionslos flüsterte sie: „Guck mir in die Augen. Du traust dich nicht. Wenn du mich endest, dann auch dich.“

„Darauf zähle ich!“, stieß er zwischen zusammengebissenen Zähnen hervor.

Vor seinem inneren Auge wägte er zwei Bilder ab.

Das eine Bild zeigte das Land Andor, von Schnee und Eis überdeckt. Einige wenige Überlebende, die um ein kleines Lagerfeuer gequetscht dasaßen und sich schlotternd um Proviant stritten, immer wieder furchtsam gen Himmel starrend. Hinter ihnen heulten der Wind und die Schneekreaturen. Ein überlebender Ijsdur starrte aus der Ferne auf sie. Sie starrten finster zurück.

Das zweite Bild zeigte hingegen ein unversehrtes, blühendes Land, mit glücklich herumspringenden Kindern. Kein Eis-Drache. Keine Eisdecke. Keine Eis-Dämonen.

„Ijsdur, halte ein. Lass uns noch einmal alle Optionen durchgehen“, meldete sich Aćh zu Wort.

„Keine Zeit“, presste Ijsdur hervor. Er wusste selbst, dass es diese paar Minuten keinen relevanten Unterschied mehr machen sollten. Er wollte sich wohl nur nicht länger als nötig die Gelegenheit geben, sich von Furcht übermannen zu lassen.

Ijsdurs Hand schnellte nach vorne und brach in Siantaris Rücken hinein. Tausend Nadelstiche trafen ihn, doch ließ er sich davon nicht abkriegen. Er spürte etwas Festes, Scharfes, griff zu und zog seine Hand zurück.

Siantari gab keinen Laut von sich, als Ijsdur den faustgroßen blauen Kristall aus ihrem Körper riss. Der Edelstein war ungeschliffen, roh und von einem finsteren Glimmen umgeben.

Siantaris eisiges Herz. Die Quelle ihrer Kraft. Die Quelle von Tari.

Siantari sprach, und doch war es nicht nur sie. Eine tiefe Stimme summte aus der Tiefe des Eiskristalls hervor und drang tief in Ijsdurs Geist hinein. Synchron zu Siantaris Lippen sprach die Stimme des ewigen Dämons in Ijsdurs Schädel:

„Guuut. Es ist an der Zeit, Sian loszulassen. Lass mich dich stärken. Lass mich dich zu Ijstari machen. Dann kannst du den Eis-Drachen befehlen. Dann kannst du das ewige Eis selbst befehlen. Es kann eine Koexistenz geben zwischen dem Eis und den Hitzköpfen. Tu nichts, was du bereuen könntest.“

Glücklicherweise würde Ijsdur, wenn alles gut lief, diese Entscheidung nie bereuen können. Einen Augenblick lang haderte er. Doch er kannte die Geschichten über Tari und die gemeinen Dämonen. Wenn er jetzt einen Pakt schlösse, wäre er in Kürze wieder vom ewigen Eis überzeugt.

Er holte tief Luft und ließ einen gutturalen Schrei seiner Kehle entfliehen. Entschlossen hielt er Taris Kristall gen Himmel. Merkwürdig, dachte er, dass ihm jetzt gerade die Schneeflocken auffielen, die wie gefrorene Tränen um hin herumwirbelten.

Dann wünschte er sich ganz fest, der Kristall möge von einem Blitz getroffen werden. Es knitterte und zischte. Aus Ijsdurs Fingerspitzen sprangen glühend weiße Blitze, sprangen auf den Eiskristall über und ließen dessen raue Oberfläche splittern. Ijsdur drückte zu und der Kristall zersprang in tausende kleinste Splitter, welche sich wie Rauch auflösten.

Siantari hauchte ein letztes Mal aus und kippte zur Seite. Ihr Schneekörper war bereits zerfallen, ehe er auf dem Boden aufschlug.

Ein fernes Brüllen und Fauchen berichtete von den Schmerzen des Eis-Drachen über der Rietburg.

Ijsdur sah Aćh ein letztes Mal an. Er spürte eine Schwäche in sich aufsteigen und ein Kribbeln in seinen Fingerspitzen. Er hätte sich passende letzte Worte überlegen sollen. Aber es war auch in Ordnung so.

Sein Mund verzog sich zu einem letzten Lächeln. Ijsdur schloss seine Augen und ließ sich in die Dunkelheit sinken.

Er hatte so einigen Schmerz in seinem Leben ausgelöst, doch dieses Opfer sollte das wieder gut machen. Ohne Siantari keine Eis-Dämonen. Kein Ijsdur. Kein Eis-Drache.

Frieden.

Er konnte im guten Gewissen sterben, Frieden über diese Welt gebracht zu haben.

Die Schwäche übermannte ihn.

Die Dunkelheit überkam ihn.

Dann schwand sein Bewusstsein.\bigskip







„Ijsdur? Ijsdur, wach auf!“

Ijsdur schlug seine Augen auf und blinzelte verwirrt gegen den dunklen Sternenhimmel, vor dem sich ein brennender Feuervogel abzeichnete, und wiederum davor ein dunkler, besorgter Menschenkopf. Ein wunderschönes Bild, das sich wie in Zeitlupe bewegte. Und eines, das ein toter Eis-Dämon nicht mehr erlebt hätte.

„Ijsdur? Lebst du noch?“, fragte Aćh erneut.

Das Kribbeln in Ijsdurs Fingerspitzen legte sich.

„Ach, verdammt“, fluchte er. Während er sich wieder aufrichtete, erkannte er wie befürchtet immer noch einen über der Rietburg herumflatternden eisigen Tarokdur. Er zog sich an Aćhs Arm in die Höhe. Siantaris Körper lag verschrumpelt und zerfallen vor ihnen. Die Dämonin war tot. Doch als Ijsdur an seinen Hals griff, spürte er noch immer die Kraft, die aus seiner eigenen Eiskristallkette durch ihn floss und ihn mit Entschlossenheit erfüllte.

Siantaris Tod hatte nichts gebracht. Die Eiskristall-Ketten konnten ohne sie existieren. Ijsdur und der Eis-Drache konnten ohne sie existieren.

„Das war sehr tapfer von dir“, flüsterte Aćh.

„Und ineffizient“, gab Ijsdur bissig zurück.

„Verzage nicht. Noch besteht Hoffnung. Auf, weiter zur Rietburg mit uns!“

Ijsdur verdrückte eine einzelne eiskalte Träne, die an seiner Wange festfror. Keine Zeit, sich jetzt mit seinen Gefühlen zu beschäftigen. Es gab weiterhin einen Eis-Drachen zu stoppen.

„Immerhin brauchen wir Turr nicht mehr hier, um Siantari zu schwächen“, meinte Aćh. Sie zog ihre tulgorische Steinflöte aus ihrer Manteltasche und spielte eine wilde Melodie, die abrupt endete.

Turr verschwand in einem Flammenwirbel und sauste in Richtung Rietburg davon.

Aćh und Ijsdur eilten ihm hinterher.

Sie hinterließen einen zerfallenen Schneehaufen, der einst Siantari gewesen war. Langsam schmolz er in den ersten Sonnenstrahlen des neuen Tages dahin.\bigskip







Tarokdur schwebte über der Rietburg. Mit Schlägen seiner gewaltigen Schwingen fegte er Gras, Stein und selbst das eine oder andere Lebewesen im Burghof umher. Was war dies für ein eigenartiges Gefühl in seiner Brust gewesen? Schwäche hatte ihn überkommen, und zugleich war ein gewaltiges Gewicht von seiner Seele genommen werden. Verwirrt blickte er um sich.

„Fokus, mein Gebieter“, zischte die durchdringende Stimme des Schwarzen Herolds.

Tarokdur drehte seinen Kopf auf dem langen Schlangenhals und blickte aus blau glühenden Augen auf Thorald herunter, der von Nehamal an der Schulter festgehalten wurde. Im Gegensatz zu herumrennenden Hunden, Pferden und vereinzelten Burgbewohnern machten die Winde von Taroks Schwingen Nehamal und den restlichen Kultisten kaum etwas aus.

Der Schwarze Herold fläzte sich triumphierend auf dem Rücken des Eis-Drachen und zischte mit seiner blechernen Stimme: „Nun, mein Meister, was wollt Ihr zuerst tun? Die Rietburg in Schutt und Asche legen? Oder den Sohn desjenigen ermorden, der Euch einst bezwang?“

Tarokdur ließ seinen gewaltigen Schädel zwischen dem Königsturm und Thorald hin- und herschwenken, als würde er Ene-Mene-Mu spielen. Er kam vor Prinz Thorald zu stehen, welcher von Nehamal zu Boden gepresst wurde und vergeblich umherwackelte.

Thorald schrie jämmerlich auf, als der Blick des Eis-Drachen auf ihn fiel. Dann jedoch straffte er seinen Körper und blickte Tarokdur stolz in die Augen. „Mein Vater fürchtete dich nicht. Und ich fürchte dich nicht.“

Sein zitternder Körper strafte seine tapferen Worte Lügen. Dennoch nickte der Herold anerkennend. Er reckte seine Faust in den Himmel, als wolle er Tarokdur den Befehl zum Töten erteilen – obwohl inzwischen klar war, dass niemand Tarokdur Befehle erteilen würde. Tarokdur richtete seinen langen Schlangenhals auf. Ein immer lauter werdendes Knistern drang daraus hervor, während er seinen tödlichen Eisatem sammelte.

Da streckte die aufgehende Sonne ihre ersten Strahlen über die Mauer der Rietburg und erhellte des Herolds erhobene Faust ebenso wie des Drachen gewaltigen Kopf. Die Kälte der Nacht wich langsam der Wärme des Morgens.

In weiter Ferne erhob sich der glühende Himmelskörper über die Ausläufer des Grauen Gebirges im Osten und sandte die ersten Sonnenstrahlen des neuen Tages über die Zinnen der Burg, an blinzelnden Kultisten vorbei und bis in den staubigen Kerker der Rietburg.

Die wenigen noch nicht verspeisten Mondbeeren an den Wänden der Kaverne glühten auf. Doch Barz pflückte keine mehr. Der Steppennomade lag in sich zusammengesunken auf einem Strohhaufen an einer Wand des Kelterraums und schnarchte vor sich hin.

Nicht so Iril. Sie war hellwach und starrte aus der Gittertür in den Rietburghof hinaus. Sie sondierte die Lage. Sagrakdur und Sagramak waren beide nicht mehr. Von den Kriegern der Rietburg war nichts zu sehen. Tarokdur war kurz davor, den ungekrönten Regenten Andors die Narne hinunterzuschicken. Jetzt war die Zeit, zu handeln.

Die Tür zum Kellergewölbe zerbarst unter einem einzigen Schlag ihres leuchtenden Runenhammers. Die verbogene Gittertür schlidderte über den vereisten Burghof bis vor Nehamals Füße.

Tarokdur blickte ebenso überrascht wie die restlichen Anwesenden – und wie Thorald, in dem ein Funken Hoffnung entfachte.

„Ist schon ein Wunder, was man mit ein paar Stunden Zeit eingesperrt in einem Kerker alles anstellen kann“, lächelte Iril zu sich selbst, „Sofern man die richtigen Mittelchen hat.“ Sie drehte ihren letzten Runenstein, den grünen, zwischen Zeigefinger und Daumen umher. Dann drückte sie zu und pulverisierte auch dieses Steinchen so mühelos, als wäre es einer dieser staubtrockenen Kekse, die man dem Santa Gor zur Winterzeit vor die Tür stellte.

Magischer Staub stieg von den Pulverresten des Runensteins auf und wehte um Irils Hand herum. Ihre Muskeln, welche schon zuvor unnatürlich angeschwollen gewesen waren, spannten sich noch weiter.

Irils Runentattoos hatten bereits gelblich und bläulich geleuchtet. Nun schossen Schlieren in allen möglichen Farben der Magie aus Irils Tattoos hervor und überzogen ihren Körper. Das Weiß in ihren Augen wurde von einem sich immer veränderndem Wirbel von leuchtenden Farben überdeckt.

„Ergebt euch und händigt die Knochensplitter aus“, verlangte Iril. Ein magisches Echo ihrer Stimme hallte zwischen den Mauern der Rietburg umher.

Stolz trat sie vor die versammelten Kultisten. Täuschte sie sich, oder erzitterte der Erdboden tatsächlich, als sie auf ihn trat?

Ein Drachenkultist in einem langen Kapuzenmantel zog einen Dolch hervor und sprang Iril an. Diese schlug mit dem Hammer tadelnd den Dolch weg, packte den überrumpelten Angreifer mit ihrer freien Hand und schleuderte ihn mehrere Meter weit. Die volle Kraft der Runen pulsierte durch ihren Körper. Wäre sie ein klein wenig größer gewesen, hätte sie den Gegner zuvor theatralisch in die Luft gehoben.

Ein Raunen ging durch die Reihen der restlichen Kultisten. Tarokdur starrte Iril amüsiert an. Nehamal ließ Thorald los. Und als wäre dies ein Signal gewesen, öffnete sich die verbarrikadierte Tür zu einem Turm. Eine letzte Handvoll verzweifelter andorischer Krieger jagte hervor und trieb die überraschten Kultisten zurück.

Tarokdur wirbelte zu den neuen Angreifern herum und sammelte seinen Atem. Iril zweifelte nicht daran, dass er auch seine eigenen Anhänger vereisen würde, um die Andori zu erwischen. Diese durften nicht seine Priorität werden.

„He, Schnee-Tarok!“, rief Iril, und ihre laute Stimme hallte zwischen den Mauern der Rietburg umher, „Flatternder Feigling! Komm auf den Boden der Tatsachen zurück und stell dich doch einer wirklichen Herausforderung!“

Schnee-Tarok wandte seinen massigen Kopf vom zitternden Thorald und dessen tapferer Rietgarde ab und wandte sich Iril zu. Seine eisblauen Augen funkelten böse. Der Schwarze Herold kicherte leise.

„DU WICHT DENKST, DU KÖNNTEST ES MIT TAROKS MACHT AUFNEHMEN?“, knirschte die tiefe Stimme des Drachen in Irils Ohren.

„Mit Taroks Macht ziemlich sicher nicht“, gab Iril zu, „Aber Taroks Macht trägst du auch nicht zur Gänze in dir, oder? Du, Tarokdur, bist nichts als eine Schneefigur, ein schwächlicher Abklatsch des wahren Taroks, gestärkt durch nichts als ein paar mickriger Fußknochen. Dich mache ich im Schlaf fertig.“

Selbstbewusstsein konnte Berge versetzten, das hatte Runenmeisterin Burmrit immer wieder gesagt. Und es konnte unbesonnenere Gegner in Rage versetzen. Auch wenn es nur gespielt war.

Mit einem lauten Krachen plumpste Tarokdur auf die freie Fläche vor Iril. Der Boden erzitterte. Seine Schwingen klappten ein. Immerhin erwischte er kein weiteres Haus bei dieser Landung. Doch sah er nicht mehr amüsiert aus.

Blitzschnell schoss Tarokdurs Kopf herunter, um Iril mit einem einzigen Happs zu verschlucken.

Iril blinzelte.\bigskip







Entfernt vom Geschehen lag Barz weiterhin auf einem Strohhaufen an der Kellerwand und schnarchte tief. Eine bräunliche Pulverpaste verkrustete seine Nase. Meditationspulver. Unter Barz‘ Lidern rollten seine Augen umher, als träume er stark.

Verwaschene Bilder schossen durch seinen halbwachen Geist. Eindrücke der Gesichtsfelder von Menschen und Zwergen, die Klingen ineinander vergruben. Schreiende Kinder in kaputten Häusern. Nabib, der in einem behelfsmäßigen Bett umherrollte. Und Iril, die einem gewaltigen Drachenkopf entgegensah.

Das Feuer der Furcht jagte durch Barz‘ Körper. Diesen Drachenkopf kannte er nur zu gut, auch wenn er nun schneeweiß war. Zuletzt hatte er ihn gesehen, als Tarok die Steppe seiner Heimat in Brand gesetzt hatte. Barz fasste sich, so gut sein müder Geist dazu überhaupt in der Lage war. Tarok war tot. Und was auch immer hier vorging, dies war Barz‘ Gelegenheit, Jirisa zu rächen.

Barz‘ Augen rollten noch schneller unter ihren Lidern umher. Die Pulverpaste um seine Nase begann, magisch zu glitzern.\bigskip







Iril blinzelte.

Sie fühlte sich leicht. Noch immer jagte das Feuer der Runen über ihre Haut, stärkte jeden ihrer Schritte und Schläge. Noch immer stiegen um sie herum vielfarbige magische Ströme aus dem Boden und pumpte sie voller Kampfkraft. Doch war dies anders. Tarokdurs gewaltiger Kiefer, der auf sie zujagte, wirkte auf einmal träger. Seine Bewegung verlangsamte sich.

Oder waren es Irils Bewegungen, die sich beschleunigten?

Farbiges Glitzern überzog ihre Haut, als Iril locker zur Seite trat und Tarokdurs lange Zähne neben ihr ins Leere schnappten.

Tarokdur hob eine Pranke, holte aus und wie im Traum bewegte sich der tödliche Schlag auf Iril zu. Iril tat einen Sprung, der über zwergische Fähigkeiten hinausging, und entkam dem mächtigen Schlag.

Iril federte den übernatürlich hohen Sprung mit einer übernatürlich starken Landung ab. Dieses magische Werk war nicht allein auf ihre Runen zurückzuführen. Ihre zauberhafte Geschwindigkeit kam von Barz. Barz und seinem Meditationspulver. Und gute Güte, fühlte es sich gut als.

Der Drache stellte sich auf die Hinterbeine und richtete, hoch wie ein Berg vor Iril aufragend, seine Flügel auf.

Mit doppelter Geschwindigkeit huschte Iril zur Seite, ehe Tarokdurs Tatzen sich an derjenigen Stelle in den Rietboden grub, wo sie soeben noch gestanden war.

Das magische Glitzern auf Irils Haut ließ noch nicht nach, ebenso wenig ihre übernatürliche Geschwindigkeit. Erneut schnappte Taroks stinkender Schädel an Iril vorbei.

Die Runenmeisterin sprang auf den Kopf des Eis-Drachen und rannte seinen Hals entlang, während Tarokdur sich unter ihr schüttelte und sie abzuwimmeln versuchte. Dass er sich halb so schnell bewegte, vereinfachte den Balanceakt für Iril, machte ihn allerdings nicht zu einem Kinderspiel.

So sehr war Iril damit beschäftigt, nicht von den langen Rückenstacheln Tarokdurs erwischt zu werden, dass sie beinahe den Schwarzen Herold übersehen hätte.

Erst im letzten Moment duckte sich Iril unter der langen Klinge des Herolds hinweg. Der Herold zischte langsame Beleidigungen hinter seiner gezackten Maske hervor und stach erneut nach Iril.

Diese schwang ihren Hammer in seine Richtung, doch natürlich hielt der Herold einen angemessenen Sicherheitsabstand. Und gleichzeitig auf ihn und auf die Bewegungen von Tarokdurs Kopf achten konnte Iril kaum. Sie machte einen Misstritt, verlor ihren Halt und stürzte in Richtung Erde.

Im Fall sammelte sie ihre Gedanken. Herold und Eis-Drache, beide konnte sie potenziell mit ein bisschen Runenmagie aus dieser Sphäre der Realität bannen. Doch der Drache war das aktuere Problem. Er musste ihre Priorität bleiben.

Von magischem Glitzer und leuchtend vielfarbigen Schlieren verfolgt, landete Iril auf dem schneebedeckten Boden, rollte sich gekonnt ab und stieß sich gleich wieder zu einem übernatürlich hohen Sprung in die Höhe.

Eine Drachentatze verfehlte sie nur knapp. Der schwebende Herold hatte nicht mit Irils hohem Sprung gerechnet. Iril klammerte sich an seinen flatternden Umhang und riss den Herold in Richtung Boden. Jetzt, wo sie ihm so nahe war, drang ein unangenehmer Geruch nach Verwesung in ihre Nase. Bei allen Kreaturen der Tiefe, was war dies für ein Wesen?

Herold und Iril knallten wieder auf die schneebedeckte Erde. Iril presste die schwertführende Hand des Herolds mit einem Knie nach unten und versetzte dem Wesen einen hoffentlich betäubenden Hammerschlag auf die Maske. Ihre leuchtenden Runentattoos spiegelten sich dumpf darin. Die eiserne Maske brummte tief wie eine große Glocke. Darunter knackten Knochen. Der Herold ächzte. Man konnte ihm also doch schaden. Doch Iril wusste von Orfen, dass der Herold schon viel Schlimmeres überlebt hatte.

Da erwischte auch schon ein stacheliger eiserner Handschuh Iril und fegte sie zur Seite. Der Herold erhob sich, weiterhin vor Schmerzen ächzend, und flog wie von Zauberhand wieder in die Luft. Und nun, wo sie nicht mehr auf seinem Diener lag, entlud Tarokdur wieder seinen Zorn über sie.

Immerhin waren seine Eisstrahlen viel fokussierter, als Taroks feuriger Drachenatem es gewesen wäre. Geschwind wich Iril dem knisternden Eisstrahl aus und huschte unter den Drachen.

Mit einem gewaltigen Satz schleuderte Iril ihren Hammer auf den ungeschützten Bauch des Drachen. Mit einem blitzschnellen Hieb des stachelbewehrten Schwanzes parierte Tarokdur die Attacke. „Ich hoffe, das kannst du besser“, höhnte er ihr entgegen.

Er richtete sich zu seiner vollen Größe auf und spie einen mächtigen Eisstrahl. Iril versuchte, zur Seite zu springen – zu langsam! Von dutzenden Eiszapfen getroffen, wurde sie zur Seite geschleudert und prallte gegen eine steinerne Hausmauer. Tarokdur lachte höhnisch. Schnee und Eis regneten auf Iril herunter und drückten sie zu Boden, hinderten sie am Aufstehen.

Es wurde kalt um Iril, während sie verzweifelt versuchte, sich freizuschaufeln. Dabei rührten sich ihre Gliedmaßen kaum. Zu schwer war das Gewicht des Schneehaufens auf ihr. Zu stark drang die klirrende Kälte in ihre übermüdeten Glieder. Iril öffnete ihre Augen und erblickte nichts außer einer Schicht blauweißen Schnees vor sich.

War dies das Ende von Irils Geschichte?

Da erklang ein freudiges Kreischen. Wärme durchfuhr Irils Glieder. Der weiße Schnee vor ihrem Gesicht schmolz beiseite und gab den Blick frei auf ein wunderschönes goldenes Wesen, von leuchtenden Flammen übersät, das auf dem Schneehaufen saß, Iril befreite und dabei allerlei gurrende Laute von sich gab.

Turr war gekommen, um sie zu retten, und er war größer und strahlender denn je. Hoffentlich hieß das, dass Aćh und Ijsdur erfolgreich gewesen waren.

Während die letzten Reste von Irils Schneehaufen zu einer unscheinbaren Wasserpfütze zusammenschmolzen, breitete Turr seine leuchtenden Flügel aus und erhob sich in die Lüfte. Er drehte seine Runden über dem gewaltigen Eis-Drachen. Dessen gewaltiger gehörnter Kopf folgte dem pfeilschnellen Feuervogel wie gebannt.

Turr schoss in die vom hellen Morgenlicht erstrahlten Wolken hinauf und wieder daraus herab. Für einen Augenblick verschwand er hinter den Zinnen der Rietburg. Sowohl Iril als auch Tarokdur blickten ihm verdutzt hinterher. Dann schwirrte der Takuri aber auch schon wieder durch das Tor der Rietburg und auf den Eis-Drachen zu.

Sein Gefieder brannte nicht mehr feuerrot, sondern violett. Turr hatte sich am ewigen Feuer in der Schale vor der Rietburg gelabt. Seine golden glänzenden Federn waren von lilafarbenen Flammen übersät. Er sah wunderschön aus.

Das ewige Feuer, Symbol für die Stärke der Andori, konnte auch ein Symbol für die Stärke eines gewissen Feuervogels werden. Violett leuchtend drehte Turr eine Runde um die Rietburg, ehe er seine Flügel zusammenklatschte und... nein, es war keine Explosion und kein Feuerball, sondern ein kontrollierter violetter Flammenwirbel, der auf Tarokdur zujagte, sich sauber durch den linken Flügel des Eis-Drachen hindurchbrannte und ein gewaltiges Loch in seiner Seite hinterließ.

Tarokdur brüllte auf und kippte zur Seite. Turr drehte eine weitere Runde.

Doch ehe Iril sich versah, schlossen sich Tarokdurs Wunden auch schon wieder. Iril wusste zwar von Ijsdur, dass dies mehr eine kosmetische denn eine tatsächliche Heilung war. Nichtsdestotrotz wurde ihr mulmig im Magen. Der Eis-Drache richtete sich zu seiner vollen Größe auf, breitete seine Flügel aus und schwang sich in die Höhe, den Schlangenhals weiterhin auf Turr gerichtet. Dieser flatterte inzwischen eher panisch als graziös. Dutzende Zwergenhöhen über der Rietburg drehte sich Turr um, breitete seine Flügel aus und sammelte Kraft für einen weiteren Feuerwirbel ... da schnappte der Kiefer Tarokdurs zu und verschluckte den Takuri zur Gänze. Kurzzeitig schwappte lilafarbenes Feuer aus Tarokdurs Maul, dann war es auch schon wieder verschwunden. Der Eis-Drache leckte sich die Lippen.

Wenn Sagramak noch am Leben gewesen wäre, hätte sie sich sicher jetzt schmerzerfüllt aufgeschrien.

Doch nicht nur Sagramak hatte Turr zum Sterben süß gefunden. Da gab es auch zwei rasch näher zur Rietburg rennenden Gestalten, eine davon mehr außer Atem als die andere.\bigskip







„Turr!“, schrie Aćh auf, und nahm nochmal an Tempo zu. Ihr Atem ging inzwischen unregelmäßig.

Ijsdur, der neben ihr halb rannte, halb über eine Eisspur am Boden glitt, blickte ebenso entsetzt in den Himmel.

Die beiden waren nur noch einige Minuten von der Rietburg erreicht. Weit über ihnen und weit über der Rietburg schmatzte Tarokdur vergnügt an etwas, das unzweifelhaft ein violett brennender Turr gewesen war. Dann sank er flatternd wieder etwas weiter in die Tiefe. Diesmal landete der Eis-Drache jedoch nicht auf dem Boden, sondern hielt sich mit windigen Schwüngen seiner Schwingen in der Luft. Seinen Schlangenhals beugte er nach unten. Eisstrahlen spie er, einen nach dem anderen, doch keiner davon schien sein von der Burgmauer verdecktes Ziel zu treffen. Er brüllte frustriert auf.

„Was tut er?“, fragte Aćh Ijsdur schnaufend. Dieser zuckte wortlos mit den Schultern.

Da! Eine kleine Gestalt war für kurze Zeit über den Zinnen der Rietburg sichtbar und verschwand gleich wieder dahinter. Dann tauchte die Person erneut auf. Und dann noch einmal. Es sah aus, als hüpfe jemand Tarok entgegen und versuchte, ihn zu erreichen. Aber das war natürlich Unsinn, kein Mensch konnte ein Vielfaches seiner Körpergröße in die Höhe hüpfen. Und die Sprungfähigkeit von Zwergen war noch schlechter. Ganz abgesehen davon, dass die Gestalt in allen Regenbogenfarben strahlte und glitzerte. Ein magisches Wesen war dies, unzweifelhaft.

Und doch kam Aćh der aus der Entfernung nur knapp erkennbare Hammer in der Hand der Gestalt äußerst bekannt vor.

„Ist das etwa Iril?“, fragte Aćh verblüfft.

„Was für andere hammerschwingende Zwerge kennst du noch?“

„Aber seit wann kann sie so hoch springen?“

„Vermutlich stärkt sie sich mit Runenmagie“, zuckte Ijsdur mit den Schultern, „Du hast ihr doch entsprechenden Steinchen geschickt. Dumm ist nur, dass sie immer noch nicht zum Drachen kommt. Fliegen ist schon ein unnatürlicher Vorteil im Kampf.“

Er lachte auf. „Iril lebt noch! Der Kampf ist noch nicht vorbei. Wir haben eine Chance. Komm, rasch, weiter zur Rietburg, vergeuden wir sie nicht.“

„Moment mal“, murmelte Aćh, „Iril könnte deine Magie eher benötigen als dein Schwert. Damals, als Barz und ich beim Felsentor auf Nesdora stießen, konnte diese mit einem Eisblitz eine riesige Eisbrücke wachsen lassen, damit wir über eine tiefe Schlucht steigen konnten. Siantari schaffte es, aus weiter Entfernung Eisblitze im Rietland niedergehen zu lassen. Sagramak ist nicht einmal eine Eis-Dämonin, und sie konnte dennoch mithilfe der Eiskristallketten einen riesigen Schnee-Drachen erschaffen. Und Ijs kannte sich eigenen Angaben nach aus mit den Brücken Tulgors wie kein anderer. Siehst du, worauf ich hinauswill?“

„Ich glaube, schon. Diese Eiskristallketten sind durchaus sehr mächtig. Doch habe ich noch nie versucht, durch Eisblitze ein Konstrukt herbeizurufen. Erst recht kein so riesiges wie eine Brücke. Oder über eine solche Distanz“, gab Ijsdur zu.

„So weit weg sind wir gar nicht“, sprach Aćh ihm gut zu. „Wann probieren, wenn nicht jetzt? Ich glaube an dich!“

Ijsdur blieb entschieden stehen und konzentrierte sich. Er schloss seine Augen, legte seinen Kopf in den Nacken, breitete seine Arme aus und schrie etwas Unverständliches in den heulenden Wind. Schnee, Eis und Kälte breiteten sich von ihm aus wie noch nie zuvor, sodass Aćh zurückweichen musste. Ijsdurs Körper begann, bläulich zu schimmern, gar taghell zu leuchten. Dann zeigte Ijsdur in den Himmel. Funken zuckten zwischen seinen Fingern und glitzernde Eiskristalle stieben in die Höhe. Über der Rietburg donnerte es, als dunkle Wolken sich versammelten und zu drehen begannen.

Erschöpft plumpste Ijsdur zu Boden. Über ihm zuckte ein riesiger Blitz über den Himmel, knapp an Tarokdur vorbei, und schlug mit Karacho in den Boden. Die Erde zitterte. Vom Einschlagort wuchs unglaublich schnell eine Eisbrücke in die Höhe, spiralförmig, vielstufig, wunderschön. Sie jagte an Tarokdur vorbei, der ihr verblüfft hinterherguckte und beinahe vergaß, seine Flügel zu schlagen.\bigskip







Iril war gerade ächzend auf dem Boden gelandet, als Ijsdurs Blitz eingeschlagen hatte. Kurz hatte sie ihn für einen weiteren Eisstrahl Tarokdurs gehalten, doch war dem nicht so gewesen. Unter lautem Getöse wie von berstendem Eis wuchs unter Irils staunendem Blick ein eisiges Konstrukt in die Höhe. Staunend bestarrte sie ein architektonisches Wunderwerk aus purem Eis, eine Dutzende Meter hochragende, dünne, spiralförmige, durchscheinende, magisch glitzernde Treppe.

Ein wie aus dem Nichts kristallisiertes Wunder.

Iril war niemand, der ein Wunder vergeudete.

Das magische Glitzern von Barz‘ Pulver überzog Iril aufs Neue. Unnatürlich rasch sauste Iril die Eisbrücke (oder vielmehr Eiswendeltreppe?) hoch und an Tarokdur vorbei. Hoch oben endete das Gebilde in einer Art offnen Plattform. Iril raste über die Brüstung hinaus und sprang von oben auf den völlig verdutzten Tarokdur herunter.

Sie lächelte.

Dann prallte sie auf den Rücken des Eis-Tarok und wurde gleich wieder meterhoch in die Luft geschleudert, knallte an einen seiner mächtigen Flügel, der sich gerade am Heben war, und rutschte daran herunter, bis sie am unteren Flügelansatz saß, beinahe nicht mehr bei Bewusstsein.

Iril schnappte nach Luft. Sie hatte Mühe, die Situation zu begreifen. Sie lag auf dem Rücken von Tarok. Sie lag auf dem Rücken eines vom abhebenden Drachen. Man könnte argumentieren, dass sie einen Drachen ritt – sie, eine einfache Silberzwergin!

Monströse, messerscharfe Schuppen bedeckten den gigantischen Körper. Iril achtete darauf, nicht auf ihre Kanten zu treten.

Ihre Augen erfassten eine Bewegung. Leise flatterte ihr der schwarze Herold entgegen. Iril hielt ihren Hammer bereit. Sie schlug zwei, dreimal auf den Herold ein. Sein langes Schwert verbeulte. Iril zeigte ihm, dass er sie diesmal nicht aus dem Gleichgewicht bringen konnte. Bevor sie eine passende Runenscheibe ziehen und das garstiges Gespenst aus dieser Sphäre zu vertreiben versuchen konnte, sprang der Herold von Tarokdurs Rücken und schwebte davon.

„Feigling!“, rief Iril ihm hinterher.

Unter ihr brüllte Tarokdur etwas und verdrehte seinen Hals, um Iril von seinem Rücken zu beißen. Sie huschte von seinen langen Zähnen zurück und klammerte sich an seinen Körper, an eine Stelle kurz vor seinem Halsansatz, die Tarok selbst unter größten Verrenkungen nicht erreichen konnte.

Und da fand sie sie. Die Eiskristallkette, die Tarokdur mit ihrer Magie einhüllte. Sie war der Ursprung seiner Kraft, seines starken Willens und seiner Gedanken.

Ein so kleines Ding, in diesem riesigen Körper eingelassen. Die Quelle so großen Unheils.

Iril ließ sich nieder, kuschelte sich in den nasskalten Schnee des Eis-Drachen und ließ ihren Runenhammer mit Magie aufheizen.

Die Welt stellte sich Kopf, als Tarokdur eine Rolle drehte, um Iril von seinem Rücken zu schütteln. Gerade noch rechtzeitig versenkte Iril ihren Hammer tief in Tarokdurs Fleisch, ließ ihn sich verankern, sich von ihm mitführen. Fliehkräfte zerrten Iril in die eine und andere Richtung, die Rietburg tief unter ihr wurde kleiner, Möwen stürzten ihr entgegen. Tarokdur flüchtete panisch in die Wolken.

„INSEKT! WURM! SPORNFRASS!“, schrie er.

Doch Iril, magisch gestärkt durch die Kraft der Runen, ließ ihren Hammer nicht los. Und ihr Hammer war magisch verankert im Eis-Drachen. So leicht konnte er sie nicht loswerden.

Mit ihrer freien Hand öffnete Iril ihre Reisetasche und zog einen kleinen Eisenstab hervor. Dann beugte sie sich vor und ritzte eine bestimmte Runenfolge in Tarokdurs eisige Schuppen, rund um die Eiskristallkette herum. Runen im Schnee. Ein Glück, dass sie diese bestimmte Runenfolge zuvor noch gebüffelt hatte. Bei diesem Geflatter konnte sie ihre Notizen kaum studieren.

Sie wartete, bis Tarokdur sich wieder richtig orientiert hatte. Dann riss sie ihren Runenhammer frei und ließ ihn gleich wieder auf die Eiskristallkette niederfahren.

Magische Ströme in allen Regenbogenfarben sprangen vom Hammer auf die Runenfolge über und ließen sie gleißend hell aufleuchten. Tarokdur brüllte. Ein letztes Ächzen drang aus dem riesigen Maul. Damit schied Tarokdur, der letzte Drache, für immer aus dem Leben.

Iril zerschlug Tarokdur. Oder vielmehr zerschlug sie ihn weniger, als dass er einfach wässerig wurde, seine Form verlor, ja, schmolz, und als riesiger Haufen Schnee-Matsch vom Himmel fiel. Ihre Runen taten wieder einmal, was sie sollten.

Iril wurde ganz leicht ums Herz, als ihre Haare und die Runenscheiben aus ihrer Reisetasche zu schweben begannen. Eiskristalle wirbelten um sie herum. Sie befand sich im freien Fall. Ihr Magen kehrte sich um.

„WEG DA!“, brüllte Iril unten, in der Hoffnung, die wenigen im Rietburghof Anwesenden sähen die Gefahr rechtzeitig.

Ijsdurs wunderschöne Eisbrücke knackste unschön, als Tarokdurs Schneekörper daran abprallte und herunterrollte.

Der Aufprall wurde vom vielen Schnee gedämpft und war dennoch heftig. Iril hatte sich mit unnatürlicher Geschwindigkeit und Stärke darauf vorbereitet und rollte den Fall gut ab, als sie gemeinsam mit Tonnen an Schnee, Eis und Wasser auf neben dem Palas der Rietburg einschlug.

Etwas benommen brach Iril aus dem Schneehaufen hervor, der soeben noch Tarokdur gewesen war. Sie blickte sich um.

Ein glimmendes Turr-Küken kugelte aus dem eisigen Schneehaufen hervor und schüttelte seine Flügelchen. Daneben erkannte Iril eisige Überreste einer Leiche, die einst wohl Sagramak gewesen war.

Doch weiter vorne war das relevante Artefakt.

Da lag sie, die elende Eiskristallkette, die Tarokdur erweckt hatte, und in welcher sein Geist nun womöglich eingesperrt war. Iril schaufelte sich frei und richtete ihren Hammer drohend auf Nehamal. Denn der Drachenkultist hatte sich ungläubig dem Schneehaufen genähert und war kurz davor, die Eiskristallkette hervorzuklauben.

Iril ließ von Sagramaks Leiche und dem Turr-Küken ab, stolperte den Schneehaufen herunter und reckte drohend ihren Hammer in die Höhe.

„Lass die Kette sein“, befahl sie. Nehamals starrte die ausnahmsweise auf Augenhöhe stehende Zwergin finster an und langte nach seinem Degen. In seinen Augen spiegelte sich die Kraft der Runen, welche weiterhin farbenfroh auf Irils Haut herumwirbelte.

„Sei kein Idiot“, murmelte er mehr zu sich selbst als zu Iril. Sein dunkler Umhang bauschte sich auf, als Nehamal herumwirbelte und davonrannte. Einige der restlichen Drachenkultisten, die nicht von andorischen Kriegern überrumpelt worden waren, taten es ihm gleich. Durch das offene Tor der Rietburg sprinteten sie in eine ungewisse Zukunft.

Iril ließ sie ziehen. Für heute hatte sie genug vom Tod und Verderben.

Nun galt es nur noch, die finstere Kette zu vernichten. Iril beugte sich nieder und hob Tarokdurs Eiskristallkette, um sie in ihrer bloßen Faust du zerquetschen. Oder besser gesagt, wollte sie das. Denn in demjenigen Augenblick, in dem die Kette ihre Hand berührte ...\bigskip







... hielt Irildora inne. Die Strapazen der letzten Minuten fühlte sie kaum mehr. Das Glühen der Runen und Glitzern des magischen Pulvers ließen nach.

Warum sollte sie die Eiskristallkette vernichten? Das war nicht zweckdienlich. Siantari mochte tot sein, doch das ewige Eis war es nicht. Noch steckte es unberührt in einem Tal hoch oben im Kuolema-Gebirge. Doch eines Tages sollte es die ganze Welt beherrschen. Und von allein würde das nicht geschehen. Irildora musste einen Plan schmieden. Nein, Irildora musste das Felsentor aufsuchen und das ewige Eis mit ihrer Runenmagie stärken, sie sollte sofort aufbrechen. Aber nein, Irildora sollte ihre Intentionen geheim halten, die Kette verstecken und die nächste Zeit als unscheinbare Heldin auftreten. Nein, Irildora sollte kuzkan rećhiar bolinur.

Unkontrollierte Überreste der Geister uralter Eis-Dämonen schwirren in ihrem Kopf herum und schrien ihr widersprüchliche Anweisungen an, die sie nur zur Hälfte verstand. Irildora musste ihre Gedanken sortieren. Doch die Dämonen ließen sie nicht. Erstarrt blieb Irildora stehen.\bigskip







Nehamal und andere Drachenkultisten kreuzten Aćhs Pfad, als sie durchs offene Tor in die Rietburg rannte. Der Riethof war chaotisch. Zwei Häuser waren eingestürzt, dafür ragte ein gewaltiges Konstrukt aus gefrorenem Wasser neben den Palas in die Höhe. Krieger rannten umher. Blut sickerte aus toten Köpern in unregelmäßig verteilte Schneehaufen. Kein Eis-Drache war mehr zu sehen. Nur noch ein gewaltiger Haufen vor dem Palas. Und davor ... das war Iril!

Es war ein Aćh bereits bekanntes Bild: Irils Runentattoos leuchteten grünlich wie auch ihr Hammer, und grünliche Schwaden stiegen von ihr auf. Doch diesmal zitterte Iril nicht verkrampft am Boden herum, sondern stand nur stocksteif da und starrte auf ihre Hand herunter. Ihr Runenhammer hing locker an ihrer Seite. Ihre Tattoos erloschen nach und nach, doch in ihrer Hand funkelte etwas durchgehend magisch.

Eiskristalle, deren Anblick Aćh nur allzu gut kannte. Sie zwang sich zu einem letzten Effort und rannte zum Palas hoch. Im Rennen löste sie die beiden Takuri-Federn von ihrem Umhang, der im Winde davonflatterte, und knackte die Federstiele entzwei. Bei Iril angekommen, legte Aćh die brennenden Federn um Irils Faust mit der Kristallkette darin. Dann drückte sie zu, bis es knirschte. Und Iril aufschrie.

Irils Faust öffnete sich wieder. Die einstige Eiskristallkette rieselte als harmloses Pulver auf den Schnee nieder.

Iril blinzelte überrascht und blickte ins Aćhs Gesicht hoch, welches wiederum freundlich auf sie heruntergrinste.

„Eiskristallkette angefasst. Anfängerfehler“, zwinkerte Aćh, „Alles wieder klar, Runenmeisterin?“

„Alles klar, Takuri-Hüterin“, sprach Iril schwach.

Sie spürte, wie die Kraft der Runensteine sie verließ. Das vielfarbige Brummen ihrer Tattoos verlöschte. Auf einmal wirkte Irils Hammer tonnenschwer. Ihre Beine gaben nach. Schwach plumpste sie auf den Boden.

„Darf ich dich jetzt berühren?“, fragte Aćh sorgsam.

„Die Runen sind erloschen. Es besteht keine Gefahr. Und stören tut’s mich auch nicht“, sprach Iril leise. Aćh kniete sich nieder, verschnaufte und stützte die zitternde Iril. Zumindest so lange, bis ein zitterndes Turr-Küken auf sie zuzuwatscheln kam und ebenfalls Aufmerksamkeit verlangte.

Dann kam auch Ijsdur durchs Tor hereinspaziert, erschöpft und zitternd, doch mit einem nicht einmal so sehr gezwungen aussehenden Lächeln auf den Lippen.

Er zeigte auf die Eistreppe und sprach tonlos, doch rasch: „Hat es funktioniert? Hast du meine Eisbrücke gesehen? Ich kann Eisbrücken bauen. Ich kann mit meinen Eisblitzen wie aus dem Nichts Konstrukte wachsen lassen, selbst aus der Entfernung. Welch wundervolle Neuigkeit.“

Aćh winkte ihn zu sich, immer noch das kleine Turr-Küken streichelnd. „Das ist noch nicht alles. Ijsdur hätte sich beinahe geopfert, um den Eis-Drachen aufzuhalten. Ohne ihn wäre Siantari noch nicht annähernd gefallen.“

„Das ist doch keine Heldentat im Vergleich zum Erschlagen des Drachen“, protestierte Ijsdur.

„Das wäre ohne euch nicht gelungen. Ohne uns alle.“

Iril guckte ihnen tief in die Augen und bekräftigte: „Danke, ihr beide. Einfach nur Danke.“

„Kriege ich auch ein Danke?“, ertönte eine leise Stimme. Barz war müde aus der offenen Verliestür gewankt und betrachtete staunend Tarokdurs Überreste. „Hübsche Arbeit! Ist es vorbei?“\bigskip







Natürlich war es noch nicht vorbei. Kurz danach grub sich ein gewaltiger Erdgeist aus dem Boden unter der Rietburg nach oben und versuchte, einige vertriebene Maasavi-Erdgeister zu rächen. Und er wurde von den glorreichen Helden Andors zurückgetrieben – wenn auch nicht diesen vieren hier. Doch das ist eine Legende für ein andermal.

Vorbei war es erst später.

Als die geflohenen Burgbewohner sich zurück trauten.

Als die Verletzten versorgt und die Toten aufgebahrt waren.

Als die Zeit fürs Feiern gekommen war.

Die vier neusten Helden, Iril, Ijsdur, Aćh und Barz, wankten durch den Trubel. Hier und da rief man ihnen gute Wünsche zu.

Von irgendwoher rannte Nabib herbei. Er hatte eine üble Beule an der Stirn, doch kümmerte er sich nicht darum, sondern stützte Barz und redete ihm gut zu.

„Wie viel Pulver hast du eingenommen?!“, fragte Nabib besorgt. Barz lallte irgendetwas halb Verständliches.

Kurz darauf kehrten auch noch die letzten Helden in die Rietburg zurück.

Eara berichtete mit voller Stimme, Skral-Häuptling Shron habe durch einen herzhaften Sprung in die reißende Narne dem Urteil der Helden entkommen können, doch seine Horde sei zerschlagen.

Chada streichelte eine silberne Schlange, die Iril seltsam bekannt vorkam. Thorn hielt vorsichtig Abstand und beäugte die Schlange aus dem Augenwinkel, während er sich um die Pferde der Rietburg kümmerte.

Kheela trug einen Beutel mit Mera-Steinen, den sie einigen Tulgori übergab. Mit einem Augenzwinkern fügte sie hinzu, dass sie beim nächsten Mal besser auf ihr Floß aufpassen sollten.

Etwas überrascht stellte Iril fest, dass Hogo und Fenn – die eine große Menge Erde und Dreck von ihren Waffen wuschen – sich nebenbei mit dem Hexer aus Tulgor unterhielten, der sich bei der Ankunft der Tulgori in Andor so rasch von der Reisegruppe getrennt hatte. Offenbar hatte er sich nun doch bei einigen Helden Anschluss gefunden. Des Hexers Augen blieben aber nicht bei Hogo und Fenn, sondern wanderten immer wieder hinüber zu Eara, welche mit blau schimmerndem Feuer von der Spitze ihres Zauberstabs mühelos ein Lagerfeuer entzündete.

Aćh gesellte sich zu dem dreien – Haamun, Hogo und Fenn – und unterhielt sich mit ihnen. Iril erinnerte sich daran, dass Haamun an Aćhs und Barz‘ Seite unter dem Kuolema durchmarschiert war. Vermutlich hatten sie sich einiges zu erzählen.

Doch ehe Iril sich ihnen anschließen konnte, zog sich der Hexer auch schon wieder an Hogos Seite vom Trubel zurück, in den Schatten eines Häuserdachs. Fenn und Aćh blieben allein zurück. Die beiden großen Vogelliebhaber unter den neu ernannten Helden. Wenn Barz Aćh in seiner Zeit in Tulgor die Barbaren-Sprache etwas nähergebracht hatte, konnte sich Aćh vermutlich mit dem ehemaligen Barbaren Fenn besser unterhalten als mit allen anderen Anwesenden – Ijsdur, Barz und Übersetzungs-Iril einmal ausgenommen.

Kram stellte sich stolz zu Iril und gratulierte ihr von ganzem Herzen zum Erschlagen ihres ersten Drachen. Er richtete ihr die besten Glückwünsche seiner Familie aus und informierte sie darüber, dass Schmiedemeister Hildorf sich nach einer unangenehmen Interrogation bei Fürst Hallgard aus Cavern zurückgezogen hatte. Keiner wusste so genau, wo er sich aufhielt. Iril hatte die Vermutung, dass er nicht allzu weit entfernt unter Gleichgesinnten weilte und seiner Schmiede nachtrauerte.

Als letzter der zurückkehrenden Helden tauchte im Burghof ein über und über mit kleinen Stichwunden übersäter gehörnter Geselle auf. In seinem langen Haar steckte eine rosa Rietgrasblüte. Sein Gesicht grinste trotz seiner arbakschen Verletzungen breit, als er Iril erblickte. Bragor rannte mit großen Schritten auf Iril zu und hob sie hoch in die Luft, während er sie im Kreis herumwirbelte. Iril fühlte sich wie eines der Gewichte, die Bragor täglich im Riethof stemmte.

Mit tiefer Stimme rief der Tarus: „Oh, Iril, wie sehr es mich freut, dich unversehrt zu wissen! Wir machten uns solche Sorgen, als wir den Eis-Drachen aus der Ferne wüten sahen. Wenn wir nur früher hier eingetroffen wären ... und diese Bestie hast du eigenhändig niedergerungen? Noch in einem Dutzend Tagen werden wir diese Heldentaten nicht vergessen haben!“

Iril protestierte gegen das übermäßige Lob. Sie hatte Bragor und seinen Zahlenverständnisse gut genug kennen gelernt, dass ein Dutzend für ihn schon fast mit Unendlich gleichzusetzen war.

Ijsdur trat hinzu und tauschte einen komplexen Handgruß mit Bragor aus. Zwei großgewachsene Gehörnte mit einer Abneigung gegenüber sperrigen Schilden und einer Vorliebe für spärliche Kleidung. Sie könnten beide kaum unterschiedlicher sein, und doch hatten sie sich schon bei ihrer ersten Begegnung prächtig verstanden.

Die Hüterin Kheela gesellte sich zu ihnen und zog Bragor nach kurzer Zeit auch schon wieder zur Seite. Offenbar würde der königliche Barde Grenolin bald mit musikalischer Untermalung beginnen und Kheela wollte den Tarus zum Tanzpartner haben.

Vara der Wassergeist waberte am Rande der Festigkeiten herum. Ihr Gesicht war eine Maske der Trauer. Ijsdur bewegte sich zu ihr und legte ihr eine Hand auf die Schulter. Die Hand fuhr durch den Wassergeist hindurch und wurde durchscheinend, während Schneekristalle sich entlang Varas Schulter ausbreiteten. Weder Ijsdur noch Vara sagten etwas.

Einige Schritte entfernt war Barz auf einem Holzhocker zusammengesunken, während Nabib ihn in seine Arme geschlossen hatte und ihm süße Worte zumurmelte. Fenn schien nicht viel um ihre Privatsphäre zu kümmern, denn er trat laut grölend dazu. Den wenigen Wortfetzen nach, die Iril aufschnappte, versprach er den beiden „echt umhauendes Barbaren-Bier“, wenn sie ihm das Geheimnis der Anwendung von Silberblumen verrieten.

Iril blickte sich nach Aćh um und fand sie ein wenig verloren wirkend abseits der Gesellschaft, wie sie Turr und Morar Apfelnüsse fütterte. Iril winkte sie zu sich. In diesem Augenblick setzte Grenolin der Barde eine andorische Flöte an seine Lippen und begann mit dem Spielen einer wilden Melodie.

Sie feierten gemeinsam bis tief in die Nacht hinein. Selbst der sonst eher zurückhaltente Hogo ließ sich von Fenn zu einem wilden Tanzduell locken.

Die Gefahr war gebannt.

Iril war nicht mehr allein.

Alles war gut.\bigskip


Am nächsten Morgen – oder eher schon gegen Mittag – saßen Iril, Ijsdur, Aćh und Barz mit einigen anderen Andori an einem Lagerfeuer und genossen gehörig gesalzene Fladenbrote von Meisterbäcker Karmat, als zwei grau gewandte Bewahrer zu ihnen traten.

Solche waren inzwischen kein seltener Anblick mehr auf der Rietburg. So viel dazu, dass die Bewahrer vom Baum der Lieder angeblich so selten wie möglich den Wachsamen Wald verließen.

Wie es sich herausstellte, hatte der Oberste Priester Melkart seit der Ankunft der Tulgori in Andor ein weiteres Paar Chronisten ins Rietland gesandt, die sich ausführlich mit den Erlebnissen dieser fremden Reisetruppe auseinanderzusetzen hatte. Doro und ... irgendein schwierig auszusprechender Name, der Iril entfallen war. Sie hatte auch nicht viel mit diesen beiden Bewahrern Kontakt gehabt, nein, als Tulgoribeauftragte hatten jene sich viel eher für Ijsdur und Aćh interessiert.

Erleichtert erkannte Iril, dass sie auch gar nicht an den Namen zu erinnern versuchen musste. Denn die beiden sich anschleichenden Mitglieder des Bewahrerordens waren nicht die Tulgoribeauftragten, sondern die schon zu Taroks Tod ausgesandten Sanja und Jorna. Wie üblich war Sanja als erste bei den Helden angekommen, während Jorna schüchtern hinterherhuschte.

„Endlich erreiche ich Euch alle zusammen einmal“, rief Sanja triumphierend, „Stets scheint mir mindestens einer aus dem Weg zu gehen. Hättet Ihr denn gerade etwas Zeit, dem Bewahrerorden einige dringende Fragen für die Nachwelt zu beantworten?“

Hastig packte Jorna wieder ihre Schreibtafel hervor, zückte eine Feder und machte einige hastige Notizen am oberen Rand eines neuen Pergaments.

„Wir haben doch noch nicht mal etwas gesagt, was schreibt sie denn nun schon auf?“, fragte Ijsdur neugierig.

„‚Er‘“, korrigierte Sanja beiläufig, „Aktuell zieht er ‚Er‘ vor.“

Dabei zeigte sie auf einen gelben Wimpel, der auf Brusthöhe an Jornas grauem Bewahrergewand befestigt war. Jornas Wangen röteten sich, doch er lächelte.

Temporeich berichtete Sanja schon weiter: „Und er schreibt nur mal Datum und Titel für die spätere Transkription seiner Notizen in einen vollwertigen Bericht nieder. Nichts Besonderes, irgendetwas im Sinne von ‚Befragung der siegreichen neuen Helden nach dem Vertreiben des großen Eis-Drachen, vernommen von Sanja und Jorna am Xten Tag des Xten Monats des Jahre 65 nach andorischer Zeitrechnung‘. Wobei ‚neue Helden‘ noch ein Platzhalter sein könnte. Hat eure Heldengruppe inzwischen schon einen Namen gefunden?“

Die vier blickten einander an.

„Wollen wir uns denn einen Namen zulegen?“

„Wenn das von uns verlangt wird? Welch bessere Gelegenheit als jetzt?“

„Was vereint uns denn?“

„Nicht viel. Wir stammen ja aus allen möglichen Ecken der Welt. Ferne Helden? Fremde Helden?“

„Ich komme technisch gesehen ursprünglich schon aus Andor“, warf Iril ein, „Oder zumindest aus Cavern.“

„Was ist mit Reisenden Helden?“

„Haben wir nicht vor, uns zumindest eine Zeit lang hier niederzulassen?“

„Das ist wohl wahr.“

„Neuere Helden? Neuste Helden?“

„So ein Titel würde rasch alt.“

„Wir haben schon Turr und Sabri; besorgen wir Ijsdur und Iril noch Tiere und nennen uns Haustier-Helden?“

„Tiere, die mit Schnee und Eis wenig anfangen können, mögen mich üblicherweise nicht.“

„Da hast du vielleicht einfach noch nicht die richtigen Tiere getroffen.“

„Was ist mit Magie? Wir alle haben etwas mit Magie zu tun, oder etwas nicht?“

„Nun, dieser Runenhammer ist definitiv von Magie erfüllt.“

„Meine Pulvermischungen sind auch magisch.“

„Takuri sind ziemlich magische Wesen. Und die Steinflöte kann mithilfe von Magie auch in weiter Entfernung vernommen werden.“

„Und alles an mir, von meiner Eiskristallkette über meine Eisblitze bis hin zu meinem schneeigen Körper, ist magisch.“

„Na, dann ist es klar. Wir sind die Magischen Helden!“

Jorna kritzelte etwas auf seiner Schreibtafel. Dazu, das Interview durchzuführen, kamen die beiden Bewahrer allerdings nicht, denn in diesem Augenblick rannte Prinz Thorald auf die frischgetauften magischen Helden zu. Der Prinz war von den gestrigen Feiern absent geblieben, sondern hatte seinen frisch vor einem Eis-Drachen geretteten Hintern rasch in ein sicheres Bett bugsiert.

„Was ist denn das für ein Klamauk! Und was macht dieser elende Eis-Dämon immer noch hier?! Wir sahen doch soeben alle, dass wir denen nicht trauen können. Oh, wie töricht ich war, dich zu einem Helden zu ernennen. Ein Glück, dass wir solche Titel im Zweifel auch wieder entziehen können. Raus mit der Sprache, was hattest du mit diesem riesigen Eis-Drachen zu tun?!“

„Wir können ihm trauen!“, sprach Aćh. Sie stellte sich zwischen Ijsdur und den zornigen Prinzen. „Ijsdur, der Eis-Dämon, steht auf der Seite der Helden. Mit seinen magiegeladenen Eisblitzen hilft er nicht nur im Kampf. Seine Eismagie macht ihn zu einem der stärksten Helden Andors. Mit dem Erschaffen von Eisbrücken erzeugt er außerdem neue Wege und Abkürzungen, die allen Helden zugutekommen.“ Dann warf sie einen Blick hinter sich. Die gloriose Eisbrücke in den Himmel, dank der Iril den Eis-Drachen hatte erreichen können, war inzwischen kaum mehr als solche zu erkennen, sondern kaum mehr als ein eisiger Zapfen, der aus einer immer größeren Wasserlache hervorstach.

„Leider ist all dies nicht von Dauer, denn nach einer Weile schmelzen seine glitzernden Werke dahin“, hängte Aćh melancholisch an. Ijsdur sollte dieses bestimmte Werk lieber manuell schmelzen, ehe es etwa noch auf dem Palas stürzte.

Thorald suchte nach Worten, fand keine, verwarf seine Hände und stopfte weiterhin wütend murmelnd davon.

„Ich würde das so interpretieren, dass du noch Teil des Teams bist“, meinte Barz grinsend zu Ijsdur.

Aćh fügte an: „Und wenn nicht, werden wir für dich einstehen. So viel schulden wir dir, und noch so einiges mehr.“

Ijsdur blieb stumm.

Iril betrachtete den prinzlichen Abgang nachdenklich. Dann langte sie in ihre Reisetasche und zog zwei Ketten hervor, die nach dem Gefecht gefunden worden waren.

Zwei Ketten, an denen einige unförmige Knochenfragmente befestigt worden waren.

Taroks und Sagraks letzte Überreste.

„Es dauert vermutlich nicht lange, bis Thorald sich an die Knochen erinnert und herumfragt, ob sie schon gefunden wurden. Ich bin mir nicht sicher, ob wir sie ihm wieder überlassen sollten. Diese Angelegenheit ist noch nicht vorbei. Zweimal haben die Drachenkultisten die Knochen nun schon gestohlen. Sie werden es wieder versuchen. Wie können wir verhindern, dass es ein drittes Mal geschieht? Wollen wir sie vielleicht doch vernichten?“

„Ich hätte da eine bessere Option“, meinte Barz. Er nestelte an seinem Pulvergürtel herum und präsentierte einen rosa Beutel. „Ich liebe es, mit neuen Pulvern zu experimentieren. Was dabei herauskommt, weiß keiner so genau. Mit nur einer Prise dieser bestimmten Mischung hier schaffte ich es schon, eine Pfeilspitze, eine Blume und eine Schneckenmaus zu versteinern. Wir könnten vermutlich die Drachenknochen damit in Gestein verwandeln. Sicherlich würde dies sie als magische Quelle untauglich machen, woraufhin wir sie gefahrlos den Drachenkultisten überlassen könnten.“

„Bist du sicher?“, fragte Aćh argwöhnisch, „Das letzte Mal, als du eine solche experimentelle Pulvermischung ausprobiert hast ...“

„Schlimmer als die Knochen zu vernichten kann es ohnehin nicht sein“, meinte Iril bestimmt, „Und wir überlassen sie den Kultisten nur, wenn wir danach keine Magie mehr in ihnen spüren. Ich kann das mit meiner Runenscheibe überprüfen.“

„Es kann immer schlimmer kommen. Prinzipiell“, gab Ijsdur zu bedenken, „Aber wahrscheinlich ist es wohl nicht. Und Thorald können wir immer noch erzählen, dass die Knochen vernichtet worden seien.“

„Ich halte es für gut, ihn nicht anzulügen“, meinte Barz, „Das scheint mir ein würdiges Prinzip zu sein.“

„Ihr mit euren Prinzipien ... dann sagen wir ihm halt, dass wir die Knochen zumindest unschädlich machen konnten. Das sollte auch funktionieren.“

Draufhin ergänzte Aćh: „Davon abgesehen nützen den Kultisten die Knochen allein auch wenig. Tarokdur konnte nur dank einer von Siantaris Eiskristallketten erweckt werden. Und davon sind keine mehr übrig. Oder?“

Ijsdur zuckte mit den Schultern.

„Dein Wort ins Ohr des großen Flederfuchses“, murmelte Barz, „Wir können hoffen.“

Da überlegte Iril: „Aber was, wenn eines Tages ein gewaltiger Fluch über das Land hereinbricht und wir die Drachenknochen bräuchten, um ein magisches Ritual ...“

Barz konterte: „Was, wenn ein zwielichtiger Berater Thorald einflüsterte, dass er mit den Knochen in seinem Thronsaal etwas Böses anstellen sollte? Wer kann schon die Zukunft kennen. Vielleicht ist es besser, einen Schlussstrich unter diese Sache zu ziehen.“

Iril seufzte und stimmte zu.\bigskip







Als Iril und Barz zurückkehrten, stand das Lager der Drachenkultisten am Hang des Kuolema-Gebirges noch. Gerade noch.

Die meisten Zelte waren im Begriff, abgebaut zu werden. Menschen, Kreaturen und ein Ziegenwesen packten ihr Lagermaterial auf große Karren. Das Ziegenwesen murrte über ein Ziehen in seiner verletzten Schulter. Niemand beachtete es.

Nehamal stand vor einem brodelnden Kessel und stocherte mit einem Ast im Feuer herum.

Neben ihm saß die kleine Reanna am Boden und stapelte flache Steine aufeinander.

Nehamal machte sich nicht einmal die Mühe, wütend aufzusehen, als Iril und Barz nähertraten. Knurrend sprach er: „Ihr! Habt ihr nicht schon genug angerichtet?“

Nicht einmal seinen Degen zog er. Seine blutunterlaufenen Augen starrten ins Leere. Die letzten Tage hatte ihm nicht gut zugetan.

Iril sprach: „Es tut mir leid. Es hätte nie so weit kommen können. Ich wollte das alles nicht. Ich wollte nur nicht, dass ihr mit den Knochensplittern ... ihr seht, was geschehen ist.“

„Nur, weil ihr Sagramak angereizt habt. Sie hätte niemandem geschadet, wenn ihr uns einfach zu Beginn an Tarok rangelassen hättet!“, schnüffelte Nehamal.

Iril murmelte: „Es tut mir leid. Thorald und seine Krieger sollten nicht über euch bestimmen. Ich auch nicht. Doch die Gefahr, die von diesen Knochen ausgeht ... ich würde weiterhin behaupten, richtig gehandelt zu haben.“

Nehamal schnaubte.

„Wir hätten die Knochen ungefährlich machen sollen, statt sie zu konfiszieren“, meinte Barz, „Lassen wir die Vergangenheit vorerst ruhen. Nehamal, wir ersuchen, den Konflikt zwischen den Andori und den Drachenkultisten hinter uns zu lassen. Wir bringen euch den Leichnam Sagramaks zurück.“

Barz deutete zu Sabri, welche über den Hügel anzutrotten kam. Über ihren Rücken war ein blütenweißes Totentuch gelegt. Eine Ausbuchtung deutete an, dass sich darunter ein menschengroßes Etwas befand. Und die verbeulte Rüstung an Barz‘ Rucksack verriet, um wen es sich dabei handelte.

Nehamal unterdrückte einen Schluchzer.

Barz hängte an: „Und wir bringen euch die Knochen von Sagrak. Beerdigt Sagramak nach euren Riten, wie die Drachen es von euch verlangen. Und ernennt eine neue Sagramak, die dem Andenken der Schamanin würdig ist.“

„Unter einer Bedingung“, sprach Iril. „Wir haben die Ermordeten unter euren Angriffen nicht vergessen. Wir wollen verhindern, dass so etwas wieder geschieht. Dass ihr und die deinen keine unschuldigen Seelen mehr schaden.“

Nehamal wedelte mit seiner Hand und scheuchte die kleine Reanna zu einer in der Nähe herumstehenden älteren Kultistin. Dann erst blickte er auf und zischte Iril entgegen.

„Ich habe kein Leben genommen. Nicht dieses Mal. Und Eure Krieger nennt ihn unschuldig?!“

„Sagramaks Leichnam steht euch zu, außer Frage. Doch damit wir euch die Drachenknochen Sagraks geben, müsst ihr schwören, dass ihr diese Sache hinter euch lässt. Schwört auf die Drachen der Urzeit, diese Wut auf die Helden und das Königshaus sein zu lassen. Diesen Konflikt zu beenden. Lasst ihr aller Leben in Ruhe und zieht euch zurück ins Gebirge, aus dem ihr stammt.“

„Ich soll schwören? Auf die Drachen, in die ihr nicht einmal glaubt?“

„Das muss ich auch nicht, damit dieser Schwur Bedeutung hat.“

Nehamal betrachtete die ihm entgegengestreckte Knochenkette, schluckte und sprach dann heißer: „Dann schwöre ich. Im Namen der hohen Drachen, die uns alle nach unserem Tod richten werden.“

„Was schwörst du?“

„Ich schwöre, den Konflikt um die Drachenknochen sein zu lassen. Den Andori nicht nachtragend zu sein. Zurück in unser Gebirge zu ziehen.“

Barz blickte Iril hoffnungsvoll an. Iril nickte: „Gut genug für mich.“

Nehamal nahm die Knochenfragmentkette reflexiv an sich. Dann blickte er die näher schlurfende Sabri mit einem unergründlichen Ausdruck an. Sein Kinn zitterte. Leise murmelte er: „Sie ... sie war so rasch weg. Ich dachte, dass wir noch Jahrzehnte miteinander hätten. Es ist ... es ist nicht fair.“

„Der Tod ist nicht fair“, murmelte Barz nickend, „Wie es auch das Leben nicht ist. Wir geben unser Bestes, diese Unfairness zu richten ...“

Nehamal unterbrach Barz: „Nicht der Tod ist unfair. Nicht nur. Seine Personifikation. Der geflügelte Tod. Tarok.“ Nehamal spuckte aus. „Tarok war schuld! Er und der vermaledeite Herold, die keinen Dreck um den Willen der restlichen Drachen geben!“

Iril murmelte: „Das ist jetzt unangenehm. Wir wollten euch auch die versteinerten Knochen von Tarok überlassen. Ihr magisches Potential ist versiegt. Sie von euch fernzuhalten, wäre inzwischen bloß ein Akt des Trotzes..“

„Taroks Knochen? Versteinert?“, horchte Nehamal auf.

Barz zeigte die steinerne Kette und meinte: „Daraus wird nie wieder ein Schnee-Drache erwachen können. Aber für eure Rituale sollte es hoffentlich genügen.“

„Und dem hat Prinz Thorald zugesagt?“

Iril grinste schief. „Sagen wir einfach mal, für ihn sind die Knochen verschollen gegangen.“

Barz legte nach: „Nach dem, was Nabib und Fenn mir von Thoralds Wesen erzählt haben, werden seine Gedanken schon bald wieder woanders sein.“

Nehamal blickte ihnen argwöhnisch entgegen. Dann schnappte er sich die Knochenfragment-Kette und murmelte: „Danke. Taroks Übeltaten hin oder her, wir können die Überreste seiner Seele durchaus nutzen, um uns zu stärken. Wir werden einen Tamarok ernennen. So hätte es Sagramak gewollt.“

„Können wir irgendwie sonst helfen?“

„Ja! Fischt uns Taroks Überreste aus dem Hadrischen Meer!“

„Ich dachte eher etwa an Nahrung für Eure Armen und Schwachen. Diese können schließlich nichts ...“

„Wir können uns noch gut um uns selbst kümmern, ganz herzlichen Dank“, schnaubte Nehamal. „Ich bin schon froh, wenn wir Andor wieder hinter uns gelassen haben. Danke für Sagramaks Körper und für die Knochen, aber auch nicht für mehr. Heilige Hüter im Himmel, was für ein Desaster diese ganze Aktion war. Wenn ich diesen Herold in meine Finger kriege ...“

Iril und Barz verabschiedeten sich wieder und überließen die Kultisten sich selbst. Im Abgehen sahen sie, wie Nehamal in weiter Ferne mit sich selbst zu reden schien.

Dann gab er sich einen Ruck und half den restlichen Kultisten beim Abbau des Lagers. Er packte einen ganzen Karren und atmete tief durch. Seine Augen glühten rot auf. Dann zog Nehamal den Karren in Richtung Osten. Ochsenstärke. Drachenstärke.

Die Drachenkultisten zogen sich wieder in ihr Gebirge im Osten zurück.

Das war wohl für alle das Beste.\bigskip







Aćh ließ den kleinen Turr fliegen und verfolgte seinen feurigen Flug über den andorischen Himmel. Sie mochte es hier in Andor. Nicht, dass sie etwas dagegen hätte, demnächst einmal Barz‘ Heimatland oder die Inseln des Nordens zu besuchen. Doch das andorische Heldendasein war eine wirkliche Freude.

Nur die Sprache war ein Ärger. Warum mussten Wörter auch so viele unterschiedliche Endungen haben, je nachdem, wo sie in einem Satz standen? Warum kam das Verb oft erst mitten in den Satz statt an den Anfang? Warum musste jeder Begriff einem von drei Genera zugeordnet werden, die man sich einfach merken musste? Muttersprachlern mochte dies beim Verständnis von Sätzen helfen, doch weh jedem, der diese Sprache erlernen wollte, um von Übersetzungstränken unabhängig zu werden. Damit hatte Barz damals nicht zu kämpfen gehabt. In Tulgor hatte sich die Sprache schon seit langem aus dem Ur-Tulgorischen mit seinen gefühlt einhunderttausend Geschlechtern, Fällen und Tempi über einige verwirrende Sprachkonzepte hinaus entwickelt.

Aćh verbrachte viele Abende in der Taverne von Andor und lernte, sich langsam an die Sprache zu gewöhnen.

Der gute Geist der Taverne zum Trunkenen Troll hatte sie besonders gerne nach Informationen zu Tulgor gefragt. Und diese Informationen konnte sie inzwischen immer besser kommunizieren.

In der Ferne sah Aćh zwei wohlbekannte Gestalten näherkommen. Eine gedrungene Hammerträgerin neben einem hochgewachsenen Mantelträger mit Bogen. Offenbar hatten sie die Drachenknochen gut abgegeben. Sabri war nirgends zu sehen.

Ja, Iril und Barz waren dabei, das Lager der Tulgori zu besuchen. Unter den gespannten Zelten herrschte emsiges Treiben.

Barz erkannte Aćh und eilte zu ihr. Sie hatte in den letzten Tagen von den unreißbaren goldenen Bogensehnen aus Tulgor geschwärmt, aus einem Material, welches angeblich ganze Schiffe auf Seile hängen können sollten. Nun wollte Aćh am Lager der Tulgori eine solche Sehne für seinen zerschnittenen Bogen finden. Ein neues Mondschwert für sich selbst hatte sie bereits erstanden und in einer schönen Zeremonie mit Goldfarbe überzogen.

Iril blickte den beiden hinterher und ließ das Stimmengewirr auf sich einwirken. Sie mochte die Stimmung hier. Stets war etwas los.

Eine Tulgori mit Augengläsern auf der Nase polierte einige Mera-Steine. Iril blinzelte ihr neugierig entgegen, störte sie allerdings nicht. Wie sehr waren die vom roten Mondlicht gestärkten Mera-Steine wohl mit den Runensteinen der Schildzwerge verwandt?

Einige andere Tulgori beugten sich über detaillierte Karten des Landes, die ihre Truppe in den letzten Tagen hergestellt hatten. Bald wollten die Tulgori weiterziehen. Die Geheimnisse der Zwergenmine Cavern interessierten sie. Bisher hatten sie nur die imposanten Portale der Mine gesehen, und die Aussicht, ins Innere des Berges und in das Reich der Schildzwerge zu gelangen, beflügelte sie.

Überrascht stellte Iril fest, dass sich inzwischen auch Wrort, der reisende Temm, im Lager der Tulgori niedergelassen hatte und sich auf Tulgorisch mit einigen Reisenden unterhielt. Gerade diskutierte er mit Ijsdurs Bruder Eforas darüber, ob irgendeine gewisse uralte dunkle Prophezeiung auf den Königsturm im Grauen Gebirge zutreffen könnte.

Eforas‘ Anwesenheit erinnerte Iril an seinen Bruder. Ijsdur war nicht hier. Vermutlich zog er gerade ziellos durchs Land, eine Spur aus Eis und Schnee hinter sich herziehend. Zwischen Ijsdur und Eforas stand es nicht besser. Ijsdur hatte ihr versichert, dass ihm dies nicht viel ausmachte, und dass ihm dieser Fakt beinahe mehr schmerzte als sein ihm gegenüber kalter Bruder.

Iril hatte ihn dennoch zu trösten versucht.

In ein bisschen Abstand vom Lager der Tulgori setzte Iril sich auf einen Stein, zog ihre Runenscheibe hervor und begann, an einer bestimmten Runenfolge zu werkeln. Es konnte doch nicht so schwer sein, Edelsteine als ähnliche Speicher magischer Kraft wie Runensteine zu nutzen. Eines Tages würde sie dieses Geheimnis knacken, da war sie sich sicher.

Die Zeit rannte vorbei, während Iril unter brütender Sonne arbeitete.

Barz gesellte sich zu Iril und brachte ihr einen Trinkschlauch voller erfrischenden Wassers, auf dass Iril vor lauter Runen nicht das Trinken vergaß. Dann setzte er sich neben sie in den Schatten eines Baumes. Er zückte zwei Phiolen mit glitzerndem Staub darin und begann, sorgfältig Portionen davon zu mischen und auf einem kleinen Holzteller zu platzieren. Es knisterte leise. Zwischendurch sprangen Funken über. Barz fluchte auf und Iril kicherte leise.

Versunken in ihre jeweiligen Interessen, arbeiteten Steppennomade und Runenmeisterin Seite an Seite. In Ruhe und Frieden. Die ultimative Lebensqualität.

Es wurde nur noch besser, als Iril Barz danach fragte, was er eigentlich hier tat, und Barz begeistert von magischen Reagenten mit Ochsenstaub als Katalyst zu erzählen begann, und davon, wie man unter Umständen dank Ijsdurs übernatürlicher Kälte diese bislang zu explosive Reaktion zweier Pulver vielleicht kontrollierter auslösen könne, wenn man zuvor gewisse Vorsichtsmaßnahmen vornahm.

Iril war keine Pulvermeisterin, doch hatte sie an der Akademie von Werftheim genug Intuition aufgeschnappt, um erst nach einigen Minuten von Barz‘ Erläuterungen abgehängt zu werden. Leider hatte ihr Pulvermeister sie damals kaum für diese Lehre begeistern können. So, wie Barz von seiner Schamanin Asbark in Thakkum berichtete, war deren Feuer für diese Kunst kaum einzudämmen.

Schneeflocken landeten auf Barz‘ Holzteller und ein kalter Wind strich um Irils Haar. Ijsdur war zurückgekehrt. Der Eis-Dämon ließ sich neben seinen beiden Mithelden ins Gras sinken und blickte in den Himmel. Sah den Wolken beim Vorbeiziehen zu. Streckte seine Hand in die Höhe und beobachtete, wie die Schneewirbel um ihn herum seinem Willen folgten.

Da schoss ein armlanger Turr vom Himmel herab und ließ sich neben Ijsdur nieder, ja, erlaubte dem Eis-Dämon sogar, seinen Schnabel zu streicheln. Und setzte dabei nur beinahe das trockene Rietgras in Brand.

Zu guter Letzt ließ auch Aćh nicht mehr lange auf sich warten und setzte sich zur Gruppe. Sie hatte sich einen neuen eleganten Umhang gekauft und präsentierte diesen stolz den restlichen Magischen Helden.

„Was nun? Bleiben wir fürs erste hier im Rietland und verteidigen die Hilfsbedürftigen? Ziehen wir woanders hin?“

Barz blickte gedankenversunken in den Osten, wo Familie auf ihn wartete. Dann zur Rietburg, wo Nabib ruhte. Dann zum Kreis der Magischen Helden, in dem er sich nun befand.

„Ich will demnächst noch das Grab meiner Eltern aufsuchen“, meinte Iril.

„Da könntest du dich den Tulgori anschließen“, schlug Aćh vor, „Nach dem, was ich gehört habe, wollen sie demnächst nach Cavern aufbrechen.“

„Vielleicht. Vielleicht bleibe ich auch noch einige Tage hier. Die Gesellschaft ist einfach zu gut.“

„Du willst deine neue Familie nicht schon wieder verlassen?“, lächelte Barz.

„Eine neue Familie? Sind wir das denn?“, fragte Ijsdur.

„Noch nicht“, lachte Aćh. „Aber wir könnten es werden.“

„Das klingt schön“, murmelte Iril.

In der Ferne kündigte ein tiefes Röhren die Ankunft von Sabri an, welche Streicheleinheiten verlangte. Lachend winkte Barz seiner Steppenechse entgegen. „Na komm schon, Sabri! Die schnelle Echse fängt den Fisch!“

Die Sonne hatte sich zwischen den Wolken hervorgeschoben. Hell und beinahe magisch erleuchtete sie das Königreich Andor. Ihr goldener Schein legte sich wie eine wärmende Decke über unsere ruhenden Helden.

Es war ganz so, als wolle sie sagen:

„Ruht wohl und erholt euch. Ihr habt noch eine Menge Abenteuer vor euch.“








\newpage
\section{Epiloge}


„... und so ernannte Prinz Thorald in einer seiner ersten Amtshandlungen als Regent – obwohl er das Königsamt offiziell selbst jetzt noch nicht angenommen hat – ganze zehn neue Helden von Andor. Manche waren etwas gesprächsfreudiger als andere, aber wir sollten zumindest von allen ihre Namen und Professionen aufgelistet haben.“

Jorna händigte die Liste der Heldennamen und Professionen an den Obersten Priester Melkart aus, welcher seine Augen zusammenkniff und die verschiedenen Namen musterte. Dann hob er überrascht eine Augenbraue.

Sanja setzte an, mit ihrer Rekapitulation der Ereignisse der letzten Wochen fortzufahren, doch Melkart hob seine Hand und gebot ihr, innezuhalten. Er fuhr sich durch die perfekt gekämmten Haare und murmelte leise vor sich hin.

Ohne eine Erklärung abzugeben, stand er auf, verließ den kleinen Archivraum und machte sich auf in eine andere Abteilung des Baums der Lieder. Sanja und Jorna blickten einander fragend an und folgten dem Obersten Priester.

Die Wendeltreppe mit den viel zu großen Stufen hoch, durch den Eingang auf die Balustrade hinaus, um die halbe Balustrade herum, die Raumflucht betreten, erste Tür rechts. Melkart suchte ein kleines Pergament hervor, auf dem eine Skizze einer Steintafel abgebildet war. Die Steintafel trug mehrere Zeilen geschwungener Runen. Alte Zwergenrunen, die selbst ein Oberster Priester wie Melkart kaum lesen konnte. Glücklicherweise war unterhalb der Skizze eine Übersetzung angebracht, vermutlich von einem waschechten Schildzwerg.

„Die Tafel berichtet über die mutigen Taten von vier neuen Helden: Fenn, Kheela, Arbon und Bragor“, las Melkart leise vor. „Ich dachte schon, dass mir einige dieser Namen bekannt vorkamen, als damals Hallgards Nachricht mit der Übersetzung eintraf.“

Er verglich die Tafel mit Jornas Notizen.

„Jetzt wissen wir es mit Sicherheit: Sowohl Fenn als auch Bragor schlossen sich den Helden von Andor an, nachdem wir sie ins Rietland schickten. Und nun verlieh Thorald ihnen offiziell den Titel. Nicht länger sollen sie in der Geschichtsschreibung Andors untergehen. Wir werden ihre Abenteuer in unsere Archive aufnehmen. Nur ... Hallgards Tafel erzählt auch von einem gewissen Arbon. Ich dachte, dass unser Arbon diese Lande längst verlassen hätte. Der unschöne Vorfall in den schwarzen Archiven ist schon fast ein Halbdutzend Jahr her. Fenn und Bragor sollten ihn doch eigentlich zu uns bringen ... kann es sein, dass sie ihn aufgespürt, aber unseren Auftrag abgelehnt haben? Die versprochene Belohnung war doch gewaltig! Und Prinz Thoralds Heldenernennungsliste enthält keinen ...“

„Dazu wollten wir uns auch melden“, berichtete Sanja. „Einer dieser neu ernannten Helden erscheint mir verdächtig. Dieser ‚Hogo, Knecht aus dem Rietland‘. Er kleidete sich beim öffentlichen Anlass in einen schlichten Umhang, aber wir haben ihn kämpfen sehen. Er führt eine einzigartige Arcuballiste verdeckt mit sich. Und er kämpft in einem dunklen Gewand, das der Kleidung einer schwarzen Wache ungemein ähnelt. Etwas ist faul an ihm.“

„Arbon“, zischte Melkart, „Wir müssen den Rat der Bewahrer informieren. Schwarze Wachen aussenden. Ihn zurückbringen zum Baum der Lieder. Ihn verurteilen lassen für das Lesen unserer verbotensten Geheimnisse.“ Dann unterbrach Melkart sich und sank in einen Sessel zurück. „Oder aber wir lassen ihn einfach in Ruhe?“

Jorna schluckte schwer und sprach dann: „Mit Verlaub: Bislang kam das ja nie wirklich gut, Arbon einsperren lassen zu wollen. Wenn er die Geheimnisse des Schwarzen Archivs in den letzten sechs Jahren nicht für finstere Zwecke einsetzte, wird er das wohl auch in den nächsten nicht tun.“

Melkart nickte leise: „Ja, ich habe in den letzten Wochen genug Leid und Tod im Rietland gesehen. Wenn Arbon sich als Held von Andor dafür einsetzt, dieses zu vermeiden ... und dennoch sollten seine Vergehen gegen unseren Kodex nicht ungesühnt blieben.“ Er kratzte sich an der Stirn. „Der Rat der Bewahrer wird darüber tagen. Ich danke euch für die Information. Bitte, fahrt fort mit eurem Bericht um Taroks Tod.“\bigskip







Im Norden, einige Zeit später.\bigskip



Ein riesiger Drachenschädel ruhte auf dem Grund des Hadrischen Meeres. Kleine rote Fischchen schwammen zwischen den riesigen Zähnen umher.

Da bebte der Boden. Die kleinen Fischchen suchten hastig das Weite. Ein mächtiger Tentakel platschte auf den Meeresgrund. Staub wirbelte auf. Der Tentakel wand sich um Taroks Schädel, brach brüchige Knochen, quetschte verfaulte Muskeln und begann zu saugen.

Mit einem hässlichen Plopp löste sich eines der Augen des Drachen. Eine kopfgroße, rötlich schimmernde ovale Kugel ohne erkennbare Pupillen. Nervenstränge, die dahinter hervorragten, zeugten jedoch, dass dieses Auge ähnlich wie das eines Menschen funktionierte.

Der Tentakel hielt das rote Augen fest und transportierte es zu einem gewaltigen Mund voller spitzer Zähne, die daran herumzuschmatzen begannen.

Neue Kraft pulsierte durch den Körper des Kraken. Drachenmagie, wie er sie noch nie verspürt hatte. Ein wohliges Grunzen drang aus seinem glubschigen Innern.

Mehrere weitere Tentakel näherten sich dem Drachenschädel, ploppten sich daran fest und verzehrten noch mehr der seltenen Speise.

Und mit jedem Bissen wuchs Oktohans Macht ein Stückchen mehr.\bigskip



\az{Jahr 66}



Im Osten, einige Zeit später.\bigskip



Es war der 1. Tag des 4. Mondes.

Meister Lifornus umklammerte seinen elegant geschwungenen Zauberstab und blickte ohne zu Blinzeln in den Nachthimmel.

Er schlang seinen Mantel enger, um sich vor der Kälte zu schützen.

Dann war es so weit.

Das Licht zweier hintereinanderstehender Planeten, deren schwacher Schein nur durch ein besonderes Fernrohr – ein Geschenk des Zeitzauberers Kirr – erkennbar war, verschwand.

Der rote Mond hatte sich soeben vor sie geschoben.

Mond, Kurip und Xoriol überdeckten einander am Nachthimmel, standen in einer Linie. Auf dieser Linie lag auch die Stätte der heiligen Flammen im Land der drei Brüder. Und in der Mitte der Stätte stand Meister Lifornus, breitbeinig.

Lifornus hob seinen Zauberstab in die Höhe. Ohne sein eigenes Zutun entzündete sich das magisch gestärkte Holz. Rote, Grüne und violette Flammen tanzten über den Stab und kitzelten Lifornus‘ Hände, ohne sie zu verletzen. Dann sprangen die Funken auf die Stätte der heiligen Flammen über und ließen den Steinkreis in allen Regenbogenfarben aufleuchten.

„Beim Horror der Unterwelt! Hombudts Gefasel hatte tatsächlich eine Bedeutung! Hombudts Worte waren nicht wirr!“, hauchte Lifornus.

Eine uralte Zauberformel in der althadrischen Sprache vor sich murmelnd, ließ er den Fuß seines Zauberstabs auf den Ritualkreis donnern. Leuchtendes Glimmen lief über die Runen, löste sich von ihnen schoss in die Höhe. Der Schein formte sich zu einer rotierenden, durchscheinenden Scheibe. Zischend öffnete sich das magische Portal. Rauch, Dampf und eine gehörige Menge Lava schossen daraus hervor. Ein leises hämisches Kichern erklang und verschwand ebenso schnell wieder.

Mühelos wischte Lifornus die gefährlichen Substanzen mit einem Schwung seines Zauberstabs zur Seite und trat näher ans Portal.

Vorsichtig spienzelte er hinein. Dunkelheit waberte ihm entgegen. Er spürte die Nähe unglaublich starker Ströme reiner Magie. Was hatte Hombudt damals entdeckt? Worin bestand das Geheimnis hinter dieser Konstellation?

Schon verlangsamte sich die Rotation des magischen Portals. Seine Ränder schrumpften. Lifornus letzte Chance, in diesem Jahr etwas über dieses Phänomen herauszufinden! Alle Vorsicht fahren lassend, sprach Lifornus einen kurzen Schutzzauber über seine Hand und griff blind in die Dunkelheit hinein. Es war schleimig und glitschig, aber auch ... da! Da war etwas Festes, Greifbares! Lifornus ergriff es.

Jemand oder etwas tippte Lifornus auf die Schulter. Er wirbelte herum, doch da war niemand. Ein leises hämisches Kichern erklang und verschwand ebenso schnell wieder.

Gerade noch rechtzeitig zog Lifornus seine Faust zurück, ehe das Portal sich darum geschlossen hätte. Mit einem letzten Plopp verschwand das wirbelnde Portal.

Lifornus öffnete seine Faust und erblickte eine kleine hölzerne Schatulle, finster wie die Nacht und kaum eine halbe Handbreit lang. Wohin hatte das Portal gereicht? Woher hatte Lifornus diese finstere Schatulle gezogen? In den Deckel eingelassen, blinzelte ihm ein glühendes Auge entgegen. War dies ein geschickter Mechanismus? Das Auge wirkte erschreckend lebendig.

Mit zitternden Fingern machte sich der Meister des Feuers daran, die Kiste zu öffnen.

Doch noch ehe er den Deckel berühren konnte, sprang dieser von selbst auf. Eine stinkende Flüssigkeit spritzte daraus hervor, direkt in Lifornus‘ Augen. Es brannte. Lifornus erschrak, quietschte auf und ließ die Schatulle fallen. Noch im Fall löste sie sich in Rauch auf.

Ein leises hämisches Kichern erklang und verschwand ebenso schnell wieder.

Die vielfarbigen Flammen erloschen.

Verwirrt und allein stand Meister Lifornus in der dunklen Nacht und rieb sich blinzelnd die Augen.\bigskip





\az{Jahr 67}

Im Süden, einige Zeit später.\bigskip



„Griun! Warte auf mich!“

Iolith raste den Hügel hinauf. Oben verschnaufte sie und schnappte nach Atem. Der Ausblick auf das winterliche Graue Gebirge war ohnehin atemberaubend und der rasche Aufstieg hatte nicht weitergeholfen.

Nun, wo hatte sich Griun wieder versteckt?

Ein Rascheln, ein Schnauben, und schon wurde Iolith von hinten angefallen. Sie stürzte auf weiches Gras, rollte einige Schritte weit und blieb in einem Blumenfeld liegen. Es roch gut. Sie erhaschte einen Blick auf graue Wolken weit über ihr, ehe sich ein grüner Kopf in ihr Gesichtsfeld schob. Lange, verfilzte Haare fielen auf sie und kitzelten sie in der knubbeligen Nase.

„Beruhige dich, Griun“, protestierte Iolith, „Meine edlen Kleider. Es wird Stunden dauern, die ganzen Grasflecken ...“

Griun beruhigte sich nicht, sondern übersäte Ioliths Gesicht mit Küssen. Dazwischen erzählte sie ausführlich, wie viel besser Ioliths edle Kleider mit Grasflecken darauf aussehen würden.

Der wilde Blick in Griuns Augen hatte sich noch nicht gelegt, als sie Iolith auf die Beine zog. Iolith befürchtete – oder hoffte – jeden Moment wieder wie von einem wilden Warbock angesprungen zu werden.

Doch Griun führte ihre Geliebte ohne weitere Abschweifungen zum Ziel des heutigen Ausflugs. Griun hatte Iolith von Nehals Stein erzählt, einst ein beliebter Schlafplatz des legendären Nehal, des vielleicht beliebtesten aller Drachen. Eine riesige Steinstaue ragte auf der Seite des Bergs hervor. Griun, welche keine lobenden Worte an die zahlreichen Bauwerke („Verschandlungen der Natur“) der Schildzwerge auf und unter den Bergen verlor, sah in der Statue gar keine Statue. Manch ein Agren munkelte, bei der Statue an Nehals Stein handelte es sich um den versteinerten Nehal selbst, weswegen Vertreter des Agrenvolkes periodisch diesen Ort aufsuchten, um die Statue zu polieren.

Iolith hingegen war sich sicher, dass es sich bei Nehals Stein um ein künstliches Mahnmal der Schildzwerge an die Macht der Drachen und die Fragilität von Bündnissen hielt. Nichtsdestotrotz hatte sie ihn noch nie persönlich gesehen und freute sich darauf.

Schon aus der Ferne merkte sie, dass etwas nicht so war, wie es sein sollte. Melodisches Stimmengewirr erklang. Griun wies Iolith an, leise zu sein. Sorgsam schlichen die beiden den Berg hoch.

Auf dem Gipfel hatte sich eine Ansammlung in Kutten gekleideter Personen in einem Kreis. versammelt. Daher kam also der wirre Gesang. Doch nicht nur Leute in Kutten waren anwesend.

In der Mitte des Kreises knieten zwei Menschen, nackt bis auf Ketten mit ... Knochenfragmenten? – um ihre Hälse. Hinter beiden stand ein gehörntes Ziegenwesen, ebenfalls kleidungslos, auf dessen faszinierende Anatomie wir hier nicht näher eingehen wollen. Es präsentierte zwei echsenartige Statuetten in seinen erhobenen Händen und brummelte leise etwas vor sich hin.

Ein langhaariger Mann hob soeben ein rötliches Relikt in die Höhe und sprach salbungsvoll: „Nun kommen wir zur Hauptattraktion dieser dunklen Messe. Ich, Nehamal, der ich vom Drachen Nehal beseelt bin, ernenne euch beide, die ihr hier vor mir und vor Nehals Stein ruht, zu erleuchteten Drachenkultisten. Zu Beseelten. Möget ihr die Stimmen der Drachen deuten und uns leiten in den Tagen, die da kommen. Von nun an hört ihr nie mehr auf die Namen Niamos und Kataka, denn euch werden würdigere zuteil. Legt diese Ketten an und nehmt eure neuen Namen an: Samagrak und Tamarok!“

Jubel ertönte.

Iolith stolperte zurück, fiel auf ihren Rücken und ließ lautstark Geröll den Hang herunterrollen. Während sie selbst hinter eine Erhebung rollte und verborgen blieb, presste die vor Schreck erstarrte Griun sich für alle Augen sichtbar an den Hang.

Die Kultisten blickten auf. Mache lugten ängstlich unter ihren Kutten hervor. Die beiden nackten Menschen schauten vielmehr verwirrt einander an, als ob sie sich nicht ganz sicher wären, ob dies zur Zeremonie gehöre. Nehamal zog gar einen spitzen Opferdolch hervor.

Als er Griun erblickte, lächelte er allerdings erleichtert. „Eine Agren. Nur eine Agren.“ Die umstehenden Personen in hohen Kutten entspannten sich. Samagrak und Tamarok blickten weiterhin unsicher drein.

Dann erkannte Nehamal die zweite Anwesende. Iolith, die sich erhob, den Hang hocheilte, sich schützend vor Griun stellte und eine Sichel kampfbereit hielt.

Er lachte. „Keine Sorge, wir beißen nicht. Ich bin mir bewusst, dass unsere Riten für Außenstehende ein wenig ... ungewohnt wirken mögen.“

Der nackte Ziegenwesen hinter ihm legte seinen Kopf schief. „Eine Agren und eine Schildzwergin? Hier, bei der Statue eines Feuerdrachen? Was haben die Drachen denn nun geplant für uns?“

„Ich bin keine Schildzwergin“, trotzte Iolith.

Nehamal fragte nach: „Bist du sicher? Dein Gesicht erinnert mich wirklich stark an eine. Nur die Tattoos fehlen. Und die Haare sind ganz ...“

„Ich weiß nicht, wovon du sprichst“, sprach Iolith, „Ich mag vielleicht einst eine gewesen sein. Doch glaube ich kaum, dass mich die Schildzwerge wirklich noch als eine der ihren ansehen.“

„Verzeih mir, da scheine ich einen Nerv getroffen zu haben“, sprach Nehamal mit erhobenen Händen. Betont ruhig steckte er seinen Dolch weg, „Manchmal schickt das Schicksal einem die spannendsten Gestalten über den Weg. Bitte, gesellt euch zu uns, setzt euch, erzählt von euch.“

„Wir wollen nicht stören“, meinte Griun nun, „Wir sind gleich wieder ...“

„Unsinn, ich bestehe darauf! Ich mag es ungemein, Geschichten von ehemaligen Schildzwergen zu hören. Ihr wisst nicht zufälligerweise, wo Kreatoks versiegelte Kultstätte liegt? Eine gewisse alte Seele vermisst ein dort eingeschlossenes Artefakt zutiefst.“

Iolith hob ihre Augenbraue: „Das verborgene Heiligtum der Schildzwerge?“

„Genau dieses“, grinste Nehamal.

„Woher sollte ausgerechnet ich seine Lage kennen?“

Ein rotes Glimmen glitzerte in Nehamals Augen auf. „Früher oder später müssen wir doch auf jemanden treffen, der darüber Bescheid weiß.“\bigskip







\az{Jahr 563}

Und zu guter Letzt im Westen, beinahe fünfhundert Jahre später.\bigskip



Die Bewahrer vom Baum der Lieder schrieben das Jahr 563 nach andorischer Zeitrechnung.

Eine einsame Gestalt lief durch über das ewige Eis. Ein langer Bart fiel auf eine junge Tulgori, die in eine Decke eingewickelt von muskulösen Armen getragen wurde. Arme, so blauweiß wie das Haar und die Kleidung der Gestalt.

„Und in diesem Augenblick richtete sich Tarokdur zu seiner vollen Größe auf und verschluckte Sagramak mit einem einzigen Bissen!“, berichtete Ijsdur.

„Nein, das ist gemein!“, protestierte Nalle schwach. Erneut schüttelte sie ein Hustenanfall.

Ijsdur nickte und passte seine Geschichte an: „Was ich sagen wollte: Tarokdur wollte Sagramak mit einem einzigen Bissen verschlingen. Doch in diesem Augenblick brachen die ersten Sonnenstrahlen des neuen Tages über die Rietburg. Und prompt zum ersten Hahnenschrei zerbarst die Verliestür. Iril trat stolz heraus, ihr Hammer so leuchtend wie ihre ganze Haut, und stellte sich tapfer vor den gewaltigen Drachen. Sie schwang ihren Hammer auf ihre Runenscheibe, ein grüner Blitz zackte vom Himmel herab, und der Drache zerfiel in hunderte kleine Schneeflocken. Das Land war gerettet. Und seither trafen wir uns immer wieder mit den restlichen Helden von Andor, um zu helfen, wo wir konnten. Für mich ist das Ganze schon beinahe 500 Jahre her.“

„Woah“, gab Nalle schläfrig von sich. Ijsdur hatte ihr noch gar nicht erzählt, wie er und Aćh such um Siantari gekümmert hatten, doch Nalle schien das nicht zu stören. Vermutlich hatte sie sich mit seiner Geschichte primär von den düsteren Gedanken zu ihrem baldigen Dasein als Eis-Dämonin ablenken wollen. Und nun war sie so schläfrig, dass die meisten solcher Gedanken von selbst fernblieben.

„Warum trägst du nun Irils Hammer?“, fragte Nalle auf einmal, „Kannst du inzwischen auch mit Runen umgehen?“

„Sie hat mich den einen oder anderen Runentrick gelehrt. Aber so gut wie sie war, werde ich niemals sein“, erzählte Ijsdur, „Ein Glück, dass sie mich fand. Wer weiß, an wen sonst die gewaltige Macht dieses Hammers hätte gehen können. Stell dir mal vor, wie überrascht sie war, als sie herausfand, dass ich als Eis-Dämon ohne jegliches Blut in meinem Körper auch relativ gut gegen die Einflüsse der Dunklen Magie geschützt war.“

„Blut schützt vor Dunkler Magie?“, flüsterte Nalle verwirrt.

„Nun, Irils Zwergenblut schien jedenfalls die verlockende Stimme der Dunklen Magie ...“ Ijsdur hielt inne. Vermutlich hatte er Nalle noch nicht einmal davon erzählt, welche Gefahren in diesem Hammer lagen. Es war wirklich schwer, den Überblick zu behalten zwischen seinen Erinnerungen an die vergangene Zeit und davon, wie viel er Nalle soeben wirklich erzählt hatte. Gedanken waren so viel schneller als ihre Ausformulierung und Kommunikation. Erst recht, wenn man außer Übung war, mit anderen außer sich selbst zu sprechen.

„Ich trage Irils Sprachrune immer noch auf mir“, erzählte Ijsdur als Themenwechsel. Er hob seinen langen Bart und zeigte eine überraschend simple Rune aus wenigen Strichen, die unten an seinem Hals zu sehen war.

„Bei unserer ersten Begegnung zeichnete Iril sie viel größer als nötig auf meine Brust. Eine Zeit lang trug ich sie hinten am Hals, damit man sie nicht sehen konnte. Aber inzwischen trage ich sie wieder vorne. Da kann man sie mit meinem langen Bart auch nicht sehen und es wird einfacher, sie nachzuzeichnen. Ich brauche sie eigentlich nicht mehr, um mit anderen zu sprechen. Ich habe viele Sprachen gemeistert in meinem langen Leben. Ich könnte die Rune verblassen lassen. Aber ich ziehe sie gerne nach. Wie ein Ritual. Eine Erinnerung.“

Ijsdur wurde wieder nachdenklich und stumm. Nalle, die primär still in seinen Armen zitterte, sagte auch nicht.

Stumm bewegten sie sich einige Minuten lang über das ewige Eis. Dann war es so weit.

„Dies ist das Zentrum des ewigen Eises. Wir sind hier“, sagte Ijsdur. Sanft setzte er die kleine Nalle aufs ewige Eis.

„Ich werde mir nun die Eiskristallkette abziehen und dir aufsetzen“, sagte er, „Danach werde ich im Eis versinken und diese Welt verlassen. Doch du wirst dich wieder erheben als Nalledora, meine ‚Tochter‘. Es wird ein ungewohntes Gefühl für dich sein, doch kein schmerzhaftes. Im Moment des Übergangs werden sich unsere Seelen berühren und wir werden Eindrücke vom Leben des anderen sehen. Das braucht dich nicht zu fürchten. Alles wird gut werden. Vielleicht siehst du sogar einige Ausschnitte von Irils Geschichte, die ich dir vorhin erzählt habe. Bist du soweit?“

„Ich ... ich weiß nicht so recht“, flüsterte Nalle schlotternd, „Wirst du ... wirst du meine Erinnerungen wirklich sehen? Da ist nicht alles gut. Als ... als ich noch klein war, habe ich mal ... ich habe eine ganze Keksdose geklaut und es auf mein Geschwister geschoben.“

Ijsdur stockte einen Augenblick. Was für viel schlimmere Dinge die Tulgori aus seinen Erinnerungen sehen mochte, wollte er sich gar nicht ausmalen.

„Vielleicht werde ich das, wenn es für dich eine wichtige Erinnerung ist. Aber das braucht dich nicht zu kümmern. Mich wird es ohnehin sehr bald nicht mehr geben.“

Er haderte mit seiner Entscheidung. Wäre es nicht besser, jemand Älteres, Weiseres als nächsten Träger der Eiskristallkette zu suchen? Konnte er jemand so Unerfahrenem derart viel Macht verleihen? Doch letzten Endes konnte er nur hoffen, dass sein Nachfolger das Geschenk würdig weitertragen würde. Niemand konnte abschätzen, wie der neue Eis-Dämon handeln würde. Die Erinnerungen der Vorgänger würden sie anleiten. Und ein viel zu kurzes Leben wie Nalles zu retten, wäre bestimmt edler, als das einer überalten Fürstin zu verlängern. Oder?

Ijsdur versuchte ein Augenzwinkern. Nalle wirkte nicht wirklich beruhigt. Dann jedoch schluckte sie tief und sprach: „Ich bin soweit.“

Ijsdur nickte und schluckte den allzu menschlichen Kloß in seiner eigenen Kehle runter.

„Weißt du, Nalle, ich habe ein längeres und volleres Leben geführt als die meisten anderen Menschen, ja, selbst als die langlebigeren Zwerge und Taren, Temm und Trolle. Ich bin glücklich mit dem, was ich mit meiner Zeit gemacht habe. Ich kann auf viele schöne Erinnerungen zurückblicken und darauf, was ich alles gelernt habe. Ich konnte mich mit meinem kommenden Tod abfinden. Ich kann friedlich und rasch gehen. So viel, das sich so viele andere wünschten, wurde mir geschenkt. Und doch, jetzt, wo der Moment hier ist, will ich noch nicht gehen. Ich nehme an, nur die wenigsten wollen das je wirklich. Wie dem auch sei ...“

Ijsdur kniete sich neben Nalle hin und hielt seine Hand vor seinen Hals. Warum zitterte seine Faust so sehr? Er griff nach der Eiskristallkette, die tief in seiner Haut verankert war. Seine Eisfinger glitten in seinen Hals, als bestünde dieser aus weicher Butter. Die seltsame Empfindung schüttelte ihn.

Ijsdur setzte an, seine Eiskristallkette abzureißen und sein Dasein zu beenden.

Da hielt er inne.

Ein weiterer Windhauch wehte über die riesige Eisfläche und wirbelte um die beiden liegenden Gestalten herum. Er trug einen Geruch nach Metall und Verwesung mit sich.

Etwas war falsch hier.

Nalle bemerkte, wie er verharrte.

„Was ist?“, fragte sie sorgenvoll.

„Etwas ist falsch hier. Spiel mit“, zischte Ijsdur zurück. Er gebot der obersten Schicht der Schneefläche, sich in den beißenden Wind um ihn herum zu erheben und die Sicht auf ihn und Nalle zu verschlechtern. Dann tat Ijsdur so, als würde er etwas von seinem Hals zu Nalles führen. Er wankte ein wenig theatralisch vor und zurück und ließ sich aufs kalte Eis sinken. Nalle guckte ihn weiterhin verwirrt an. Dann aber folgte sie ihm und ließ sich neben Ijsdur aufs ewige Eis sinken.

Eine Minute lang geschah nichts. Ruhig lagen die beiden auf dem ewigen Eis. Ijsdur totenstill, Nalle zitternd, während sie langsam blau anlief.

„Warte noch. Warte. Nicht mehr lange“, wisperte er ihr beruhigend zu.

Und auf einmal, wie aus dem Nichts, huschte eine völlig schwarz gewandte Gestalt aus dem Schneetreiben hervor. Wie ein flatternder Umhang schien sie zunächst, doch die gezackte Maske aus Metall und die Eisernen Handschuhe verrieten, dass sich unter dem Umhang mehr als nur Stoff befand. Die Gestalt schwebte einige Ellen über dem Schneeboden und glitt rasch auf die beiden liegenden Gestalten zu. Das lange Schwert in ihrer linken Hand zitterte leicht. Es war auf Nalle gerichtet und darauf und daran, sie zu durchstechen.

Doch Ijsdur war vorbereitet. Der Eis-Dämon warf sich in die Höhe, in seiner rechten Hand einen mächtigen Eisblitz. Er schleuderte ihn auf die schwarz gewandte Gestalt, die überrascht wurde und nur haarscharf ausweichen konnte.

Donner schüttelte das ewige Eis.

„Bist du mutiger geworden? rief Ijsdur spöttisch, „Bist du überhaupt noch derselbe Herold wie zu Taroks Zeiten? Früher hättest du dich noch nicht einmal getraut, ein bewusstloses Kind ohne Unterstützung anzugreifen!“

Der Schwarze Herold sagte nichts. Seine gezackte Maske stand starr in der Luft, während sein langer Umhang darunter herumwehte.

Ijsdur setzte nach: „Ich habe die Geschichten deiner Herkunft gehört. Wenn ich dich deiner dreckigen Stiefel entledige, wirst du uns dann endlich in Ruhe lassen?!“

Eine Stimme, flüsternd und doch so durchdringend, klang hinter der eisernen Maske des Herolds hervor: „Viele Sagen und Legenden ranken sich um mich, und nur die wenigsten enthalten ein Körnchen Wahrheit. Die Geschichte mit dem Kutscher ist erstunken und erlogen. Genauso wie die Geschichten deines angeblichen Triumphs über den Urgeist des Avas.“

Ijsdur überhörte das. „Ah, du kannst ja doch sprechen! Hast du irgendetwas zu sagen, was mich dein Leben verschonen lassen würde?“

Die Antwort des Herolds klang abfällig, gelangweilt: „Lass die Einzeiler sein, Eis-Dämon. So was ist unter deiner Würde.“

„Auch gut, dann bringen wir es rasch hinter uns. Ich habe lange auf den Moment gewartet, an dem dich aus dieser Sphäre vertreibe.“

„Und ich habe lange auf den Moment gewartet, an dem ich mir deine Eiskristalle einverleibe und mächtiger werde denn je. Hätte sie lieber der Kleinen abgenommen, aber mit dir allein nehme ich es auch noch auf.“ Der Herold legte seinen Kopf schief und fügte gehässig an: „Wenn du artig bist und dich ergibst, verrate ich dir Irils letzte Worte, ehe ich dich vernichte.“

Eine offensichtliche Provokation mit dem Ziel, Ijsdur unvorsichtig zu machen. Es funktionierte. Ijsdur öffnete seinen Mund zu einem wortlosen Schrei des Hasses, zückte einen weiteren Eisblitz und sprang mit einem gewaltigen Satz auf den Schwarzen Herold zu, den Eisblitz wie einen riesigen Eiszapfen auf den Herold niederstechend.

Der Eiszapfen kollidierte mit dem rostigen Schwert des Herolds und vereiste dessen Spitze, ehe der Herold zur Seite glitt, sein Schwert quer durch Ijsdurs Brust zog und den Eis-Dämonen in zwei Hälften geteilt auf die Eisfläche prallen ließ. Nutzlos rollte der kristallisierte Eisblitz vor die zitternde Nalle.

Die gezackte eiserne Maske des Herolds drehte sich hin, dann wieder her. Ausdruckslos. Tonlos. Der Herold schien zu überlegen, ob er sich lieber ganz Ijsdur widmen oder zuerst Nalle abzustechen versuchen sollte.

Ijsdurs Körper setzte sich zusammen und richtete sich auf. Ijsdur brauchte einige Augenblicke, um sich neu zu orientieren. Diese Zeit nutzte der Herold, um sein Schwert in Ijsdurs Kopf zu versenken. Blind griff Ijsdur nach vorne und bekam Stoff zu fassen. Mit einem Ratschen löste sich ein Teil des Umhangs des Herolds, als der Finsterling erneut zur Seite glitt.

Als Ijsdurs Augen sich endlich wieder geheilt hatten, sammelte sich der Schwarze Herold noch. Sein einer Handschuh betastete seinen anderen Arm, wo Ijsdur einen Ärmel seiner schwarzen Kleidung abgerissen hatte. Darunter war gräuliche, verfaulte Haut sichtbar geworden.

Aus dem Schnee und dem Eis formte sich ein Schwert in der Hand Ijsdurs, dessen Haut nun blau zu schimmern begann. Er holte aus und schlug mit diesem Eisschwert nach dem Schwarzen Herold. Dieser holte mit seinem eigenen Schwert aus und zerschlug Ijsdurs Eisschwert, als bestünde es ... naja ... aus Eis.

Der Herold machte verächtliche Geste. „Gib auf. Ohne Eisblitze bist du so gut wie wehrlos.“

„Zur Not nehme ich dich mit Fäusten auseinander! Aber zum Glück habe ich das nicht nötig.“

Ijsdur löste Irils Runenhammer von seinem Gürtel und hob ihn in die Höhe. Die darauf eingravierten Runen begannen, flackernd zu glühen.

„Runenmagie, du? Wirklich?!“, lachte der Herold, „Du kannst ja noch nicht einmal mit Runensteinen umgehen!“

„Die Meister der Magie wissen, was sie tun, wenn sie die Balance wahren. Mit Runensteinen hätte ich dich schon längst vernichtet!“, knurrte Ijsdur. Nun wusste er mit Sicherheit, dass der Herold sich seiner Sache nicht mehr sicher war. Üblicherweise hielt er es nicht nötig, zu Spott zu greifen, um seine Gegner zu verunsichern.

Erneut sprang Ijsdur den Herold an. Diesmal schwang er den Runenhammer, dem ein magischer Schleier grünlichen Schimmers folgte. Der Schwarze Herold wich zur Seite aus und ließ sein Schwert auf Ijsdurs Hand niederfahren. Selbige löste sich samt Runenhammer und schlidderte nutzlos übers ewige Eis. Da schlug Ijsdur auf eine bestimmte Rune auf seiner Schulter, welche rötlich aufleuchtete. Der Runenhammer hob sich wie von selbst in die Luft und flog zurück in den Stumpf von Ijsdurs Arm, an dem sich in Windeseile eine neue Faust formte. Andernorts wäre es vielleicht effektiv gewesen, seinen Körper zu zerteilen, bis er nicht mehr genug Kraft zum Regenerieren hatte. Doch dies war das ewige Eis, das heilige Reich der Eis-Dämonen. Hier musste man schon seine Eiskristallkette vernichten, um ihm ernsthaft zu schaden. Und wenn der Herold diese zu klauen gehofft hatte, würde er dies nur äußerst ungern tun.

Nun durfte der Herold nur nicht verschwinden. Nicht jetzt, wo endlich eine reale Chance bestand, dieses uralte Übel aus der Welt zu befördern.

Als hätte der Schwarze Herold seine Gedanken gelesen, begann er wortlos, immer höher zu schweben und sich von Ijsdur zu entfernen. Schon flog er viele Mannlängen über ihm. Unerreichbar. Oder doch nicht?

Siantari hatte zu fliegen vermocht. Und nun war Ijsdur der Herr über das ewige Eis. Er breitete seine Arme aus. Ein Wirbelsturm aus Schnee und Eis erhob sich um ihn und riss ihn in die Höhe. Doch noch war er nicht schnell genug, den Herold einzuholen.

Ijsdur griff an seinen Gürtel und zückte ein weiteres Artefakt, das ihm einiges bedeutete. Ein schwarzes Pulversäcklein. Ein gewisser Steppennomade hatte ihm einst auf dem Sterbebett ein geheimes Rezept verraten. Und Ijsdur hatte vor dem Besuch bei Nalle nicht ohne Grund den Nestbaum der Takuri besucht und einen von Aćhs Ur[ur\^{}n]enkeln nach Turrs Asche gefragt.

Ijsdur griff tief in den schwarzen Pulversack hinein und drückte seine Faust mit übermenschlicher Stärke zusammen. Er unterdrückte einen Schmerzenslaut, als die heiße, pulverisierte Takuri-Asche sich in seiner Faust zu einem Klumpen verklebte, sich entzündete und sich in seinen Schneekörper fraß. Dann holte Ijsdur aus und schleuderte den Pulverklumpen hoch in die Luft. Die Asche entflammte beim Aufprall, ein Feuerball umhüllte den Schwarzen Herold, und auf einmal saß der Herold wieder auf dem Boden, auf dem ewigen Eis. Seine gezackte Maske blickte sich kurzzeitig verwirrt um. Dann schoss er erneut in die Höhe und versuchte, davonzufliegen, nur um erneut von einer Prise Takuri-Pulver getroffen und prompt wieder auf die kalte Eisfläche teleportiert zu werden.

Ijsdur landete einige Schritte neben ihm und grinste. So leicht würde der Herold nicht entkommen.

Der Schwarze Herold fauchte frustriert auf und zeigte drohend mit seinem langen Schwert auf Ijsdur. Dieser verschloss das Pulversäcklein wieder und hob drohend den Runenhammer, während der Herold langsam auf ihn zuschwebte.

Doch urplötzlich erstarrte der Herold und fuhr herum.

Die kleine Nalle hatte den zur Seite gerollten letzten Eisblitz gepackt und sich von hinten dem Herold genähert. Nun verharrte sie ängstlich, als der Herold zu ihr herumfuhr. Dann setzte sie eine tapfere Miene auf und stach dennoch nach ihm. Genervt schlug der Herold die kleine Tulgori mit einer eisernen Ohrfeige zur Seite. Doch ihr knisternder Eisblitz machte Kontakt mit seinem Unterarm.

Und die Wut Ijsdurs entfaltete sich.

Dies war das ewige Eis, und er war sein Eis-Dämon. Dies war sein Reich. Sein Eis. Hier herrschte er!

Von der Stelle, wo der Eisblitz den Unterarm des Herolds berührt hatte, breiteten sich in Windeseile wachsende Eiskristalle aus, welche im Nu seinen gesamten Arm mit einer dicken Eisschicht überzogen hatten. Seine Maske blieb ausdruckslos, doch daran, dass der Herold in die Höhe flog und mit seine anderen Faust nach dem sich ausbreitenden Eis auf seinem Arm schlug, erkannte Ijsdur seine Verzweiflung. Das gefiel ihm. Insbesondere auch, weil das Eis so auch auf die andere Faust des Herolds überspringen und sich von dort aus weiter ausbreiten konnte.

„Du kannst mich nicht töten. Keiner kann das!“, rief der Herold, als wollte er sich selbst davon überzeugen, dass er nicht vernichtet werden konnte. Und üblicherweise hätte Ijsdur ihm recht gegeben. Doch er gebot über Mächte, welche das Unmögliche möglich machten. Er umfasste Irils Runenhammer fester.

Bald schon war der gesamte Körper des Herolds mit Eis überdeckt. Das Gewicht zog die dunkle Gestalt in die Tiefe. Mit einem lauten Krachen stürzte er auf die ewige Eisfläche. Seine gefrorene Schale platzte, doch Ijsdur war da, um ihn gleich wieder aufs Neue im ewigen Eis einzufrieren.

Eine behandschuhte Hand des Herolds blieb frei und wackelte nach Ijsdur, doch der Rest seines Körpers war vollends mit einer dicken Schicht aus Eis und Schnee übersehen.

Gut. Sehr gut.

Ijsdur trat näher und hob Irils Runenhammer. Dann begann er, in den Schnee zu zeichnen. Runen im Schnee über dem Schwarzen Herold, im Schnee neben dem Schwarzen Herold, ja, er zeichnete gar einen Runenkreis im Schnee rund um den Herold herum.

Dabei musste er immer wieder innehalten und stark überlegen. Er wünschte sich wie so oft, Iril wäre noch hier. Ihr waren diese Kritzeleien immer so viel leichter gefallen.

Dann war er fertig. Jetzt hieß es, nur, zu hoffen, dass er keinen Fehler gemacht hatte.

Ijsdur langte in seine Brust, versenkte seine Finger in seinem Körper, als wäre er durchlässig, und zog etwas aus seinem Inneren hervor, was seit Jahrhunderten dort drinnen gesteckt hatte.

Eine völlig verrostete Metallscheibe, in die einige Runen eingeritzt worden waren. Die Runenscheibe, die Iril vor all dieser Zeit genutzt hatte, um Ijsdur von Siantaris Willen zu befreien. Voller Runen, die dem Vertreiben von fremden Geistern dienten. Ohne sie fühlte er sich ganz nackt. Aber es gab keinen Dämon des ewigen Eises mehr, der von ihm Besitz ergreifen wollte. Er musste sich nicht mehr mit dieser Scheibe schützen.

Ijsdur verankerte die verrostete Runenscheibe fest im Eisblock, in welchem der Schwarze Herold zitterte. Dann trat er einen Schritt zurück, hob den Runenhammer, ließ ihn Magie ansaugen und ließ ihn auf die Runenscheibe niederdonnern.

Das uralte Metall zersplitterte in hunderte Brösel, doch ein grünlich schimmerndes Abbild der Runen blieb bestehen.

Das ewige Eis rumorte. Es knackte und knirschte unter Ijsdur. Das grünliche Glühen des Hammers sprang auf die ins Eis geritzten Runen und überzog den Eisblock des Schwarzen Herold.

Dann erlosch das Glühen, so schnell, wie es gekommen war.

Der freie eiserne Handschuh des Herolds hörte auf zu zucken.

Ijsdur langte auf das ewige Eis, welches sich seinem Willen folgend verformte und zerbröckelte. Der Umhang des Schwarzen Herolds wehte davon. Seine Maske und sein Schwert, seine Eisernen Handschuhe und seine blanken Stiefel blieben übrig und kullerten sich über die Eisfläche. Leer. Der Schwarze Herold war nicht mehr.

Ächzend eilte Ijsdur zu Nalle zurück. Diese lag weiterhin frierend auf dem ewigen Eis, doch zitterte sie nicht mehr und reagierte auch nicht mehr auf Ijsdurs Rufe. Ihr Puls war unregelmäßig.

War es schon zu spät? Sollte er lieber einen anderen Kandidaten für den nächsten Eis-Dämon auswählen, statt zu riskieren, dass sie schon nicht mehr zu retten war?

Nein, redete sich Ijsdur zu, das nicht sein rationaler Geist, der hier sprach. Noch immer hing er ein klein wenig zu sehr an seinem Leben. Doch war es nun an der Zeit, dieses aufzugeben.

Ijsdur ergriff Eiskristallkette, die um seinen eigenen Hals hing, und riss sie gewaltsam von seinem schneeigen Körper. Er krümmte sich und ächzte, während sich von den zurückgelassenen Einbuchtungen der Eiskristalle in seinem Hals Splitter und Spalte in alle Richtungen seinen Körper entlang ausbreiteten.

Mit rissigen Händen hängte Ijsdur die Kette um Nalles Hals. Kaum hatten die kalten Kristalle ihre Haut berührt, beruhigte sich ihr Atem. Ihre Furcht und ihre Anspannung legten sich. Eine Taubheit breitete sich von ihrem Hals aus über ihren ganzen Körper aus.

„Willkommen, Nalledora, zu Deinem zukünftigen Dasein als Hüterin des ewigen Eises. Achte die Vergangenheit. Nutze die Erfahrungen deiner Vorgänger. Handle weise. Handle mit Herz. Hilf, wo du kannst. Für eine bessere Zukunft.“

Das waren die letzten Worte Ijsdurs. Er trat einige Schritte zurück, stolperte dann und verschmolz mit der Eisfläche, versank buchstäblich darin.

Nalle ersuchte vergeblich, die fremden Eindrücke und Erinnerungen in ihrem Geist zu sortieren und zu vorstehen. Ihr Körper gehorchte ihr nicht mehr. War dies ihr Ende?

Mit diesen Gedanken starb Nalle, und mit ausgebreiteten Armen und eisigem Blick erhob sich Nalledora. Ihr Körper gehorchte ihr wieder. Sie hob ihre nicht mehr zitternden Hände. Diese waren schneeweiß geworden.

Langsam erhob sie sich. Es schien, als ob sie dem ewigen Eis des Kuolema-Gebirges entwuchs. Sie schaute auf die Gestalt, die vor ihr lag. Die Gestalt ihres "Vaters", des Eis-Dämons Ijsdur, die mehr und mehr vor seinen Augen verschwamm und mit der endlosen Eisfläche eins wurde, bis nur noch einige wenige Eiskristalle übrigblieben. Sie rieb sich die nun eisblauen Augen.

Vier Artefakte lagen über Ijsdurs aufgelöstem Körper auf der Eisoberfläche. Artefakte, die vorhin noch an Ijsdurs Gürtel gehangen hatten. Artefakte, die sie dank Ijsdurs Erinnerungen näher einordnen konnte. Eine gespaltene Heldenbrosche. Ein schwarzer Pulversack. Eine feuerrote Schreibfeder. Ein Runenhammer.

Warum hatte Ijsdur sie ihr überlassen, statt sie würdigeren Trägern zu übergeben?

Er musste eine große Hoffnung in sie setzen.

Nalledora spürte gedämpfte Freude darüber, wieder auf kräftigen eigenen Beinen stehen zu können. Sie hopste auf und ab. Dann hielt sie inne und betrachtete das eiskalte Gewand, das das ewige Eis ihr verliehen hatte. Sie strich mit den Fingern darüber und spürte, wie es sich ihren Wünschen gemäß verformte. Tiefe Taschen erschienen an der Seite des Kleids. Nalledora kniete sich hin und nahm die vier magischen Artefakte an sich.

Dann drehte sie sich um, und schlenderte davon. Und bald schon schritt sie durch das Felsentor und verließ das ewige Eis und das Fahle Gebirge.

Zeit, nach Tulgor zurückzukehren. Ihre Väter mussten schon umkommen vor Sorge.








