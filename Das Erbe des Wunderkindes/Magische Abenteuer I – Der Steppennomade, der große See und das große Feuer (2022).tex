\begin{chapterbox}
    \chapter{Der Steppennomade, der große See und das große Feuer (2022)}
    \label{Der Steppennomade, der große See und das große Feuer (2022)}
    \az{62 bis 63}

    \begin{center}
        Teil I der Magischen Abenteuer
    \end{center}
    
    Die Steppennomaden Barz und Nabib, zwei Mitglieder des Iquar-Stammes, kehren nach einer langen Reise in ihre Heimat zurück. Doch auf ihrer Pfahlbausiedlung Thakkum im großen See Ava wird gerade der Barbarenkönig vorstellig, um Unterstützung für die Invasion eines nahe gelegenen Königsreichs einzufordern. Denn der Krieg der Barbaren gegen die finstere Skelettarmee aus dem Süden wirkt immer aussichtsloser.
\end{chapterbox}






\section{Der Angriff der Barbaren}

\az{Jahr 62}



Sonnenaufgang, 42 Tage vor dem großen Unheil.\bigskip



Barz rupfte das Büschel Mondbeeren sorgfältig aus dem Boden und verstaute es in einem kleinen Beutel an seiner Kleidung, in dem sich bereits einige weitere getrocknete Mondbeeren neben einigen Heilkräutern aufhielten. In der Ferne konnte er den großen See immer deutlicher erkennen. Ava. Seine Heimat. Ruhig und still lag der See vor hohen Berggipfeln und glitzerte – wie die Mondbeeren – verheißungsvoll im Licht der aufgehenden Sonne.

„Guten Morgen und nichts wie raus aus den Federn! Die frühe Echse fängt den Fisch! Wenn wir uns beeilen, sind wir noch vor Sonnenhoch zuhause“, rief Barz und rüttelte Nabib beinahe sanft wach. Nabib rieb sein Gesicht und nuschelte ungehalten etwas von unterbrochenen Abenteuern im Reich der Träume. Doch auch Nabib konnte sich ein Grinsen nicht verkneifen, als er müde blinzelnd die Silhouette das Avas am Horizont erspähte. Lautstark gähnend reckte und streckte er sich und wurde dafür von Barz mit einem Kuss belohnt.

Jirisa war erheblich schwerer in die Gänge zu kriegen. Barz‘ treue, aber auch träge Steppenechse hatte sich ein wenig unterhalb der Anhöhe zum Schlafen niedergelassen und weigerte sich vehement, sich auch nur um eine Handbreit zu verschieben.

„Komm schon, Jirisa, die Zeit für deine Winterstarre ist noch nicht gekommen!“, protestierte Barz und kraulte Jirisas ledrige Haut an einer ganz bestimmten Stelle hinter einem Ohr. Die Beine der Echse zuckten wohlig und ein dumpfes Brummen drang aus ihrem Hals. Als Barz mit Kraulen aufhörte, öffnete Jirisa eines ihrer Augen und schielte empört nach ihm. Ihre lange blaue Zunge schlängelte sich an einem schief aus ihrem Unterkiefer ragenden Fangzahn vorbei und leckte nach Barz. Als dieser stoisch nicht darauf reagierte, hievte die Steppenechse widerwillig ihren massigen Körper auf ihre stämmigen Beine, woraufhin Barz sie brav weiterkraulte.

„Na also, geht doch. Wenn wir vor der Mittagshitze den Ava erreichen, besorge ich dir eine Extraladung Fische zum Abendessen. Das ganze Gras muss dir doch schon lange zum Halse raushängen.“

„Wenn du Jirisa fertig geliebkost hast, kannst du mir dann helfen, das Lager zusammenzuräumen?“, ertönte Nabibs Stimme vom Lagerplatz.

„Spüre ich da Eifersucht?“, erwiderte Barz lachend und kehrte zu seinem Freund zurück. Nach allerlei Liebkosungen und einem kurzen Frühstück aus faden, doch nahrhaften Sträuchern (mit einem rötlichen Pulver aus Barz‘ Gewürztasche schmackhaft gewürzt) kamen Barz und Nabib tatsächlich dazu, ihr Gepäck auf Jirisas Rücken zu verstauen und mit Jirisa im Schlepptau in Richtung des großen Sees Ava aufzubrechen.

Sie alle waren erschöpft von der langen Reise, die hinter ihnen lag, doch zumindest die beiden Menschen blickten zufrieden zurück auf einige Monate voller aufregender Abenteuer und fantastischer Entdeckungen. Barz hatte Proben von allerlei ihm unbekannten Substanzen gesammelt und konnte es kaum erwarten, diese gemeinsam mit seiner Schamanin zu untersuchen. Nabib hatte einen ganzen Rucksack mit Skizzen der Landschaft und ihrer Bewohner gefüllt und hatte vor, dieses Wissen in den kommenden Winterwochen an die nächste Generation weiterzugeben versuchen, ehe Barz und Nabib im Frühjahr zu ihrer nächsten Reise aufbrechen würden. Vielleicht wäre dann sogar Yafka wieder einmal mit von der Partie.

Dieses Jahr hatten Barz und Nabib es auf ihrer Reise bis an einen salzigen See im Norden geschafft, der so groß war, dass sie dessen anderes Ufer mit ihrem besten Fernrohr nicht hatten erspähen können. Wollten sie beim nächsten Mal vielleicht das südliche Gebirge zu erklimmen versuchen? Nun, sofern die Lage dies dann überhaupt noch erlauben würde. Die Gerüchte von verlustreichen Schlachten des Yetohe-Stammes gegen grausame Skelettarmeen aus dem Süden gelangten bereits seit einigen Jahren bis an den friedlichen Ava... und die Gerüchte wurden tendenziell immer düsterer.

Barz hatte keine Ahnung, wie die Lage im südlichen Gebirge im Moment aktuell aussah. Schuld daran waren wie so oft die elenden Krarks. Barz ließ seinen Blick über die weite Steppe schweifen. Es kreisten gleich zwei dieser riesigen Raubvögel majestätisch im Aufwind über der Ebene. Nicht sonderlich verwunderlich, aber doch ärgerlich. Krarks waren schon seit jeher eine Plage für die Bewohner dieses Landes gewesen, doch in diesem Jahr war ihre Population wie aus dem Nichts in die Höhe geschossen. Und mit diesen Bestien war nicht zu spaßen, falls sie der Hunger überkam.

Einen einzelnen Falken hatten Barz und Nabib zum Antritt ihrer Reise aus Thakkum mitgenommen. Dieser war, mit einer schönen Nachricht an ihre Heimat im Gepäck, kaum außer Sichtweite geflogen, da hatte das triumphierende Jagdgeheul eines Krarks ihnen schon verraten, dass ihr Brief nie den Ava erreichen würde. Und kein einziger Falke hatte Barz und Nabib während ihrer ganzen Reise erreicht, obwohl sie sonst förmlich mit solchen überschüttet wurden. Was das bedeutete, war ihnen klar: Der ganze Himmel war zu einem Jagdrevier für Krarks geworden und kein Falke war mehr sicher. Immerhin konnte Barz die nun fast allgegenwärtig zu findenden schwertschneidescharfen Krarkfedern verbrennen und das daraus entstehende Pulver kauen, um seinen Geist in einen meditativen Zustand bringen, in welchen es ihn nicht störte, dass keiner am Ava mit ihnen in Kontakt treten konnte. Manchmal schien es ihm gar, als würde er unter dem Einfluss dieses Meditationspulvers wie durch einen Nebel hindurch Schemen seiner Bekannten und seiner Familie erhaschen, und bis er seine Augen wieder öffnete, schien ihm jegliches Zeitgefühl abhandengekommen zu sein. Faszinierende Effekte, die er zurück in Thakkum gerne weiter untersuchen würde.

Doch früher oder später musste Barz seine Augen immer wieder öffnen. Und dann störten ihn die Krarks wieder, und Jirisas Sturheit, und Nabibs Schnarchen. Dann störte es ihn wieder, den Kontakt zum Ava verloren zu haben, und dass sie erst mit zwei Wochen Verspätung zurückkehren würden. Doch nun würde es kaum mehr einige Stunden dauern, bis Barz seinen lange vermissten Familienmitgliedern wieder in die Arme fallen konnte. Seine Schritte beschleunigten sich.

Nabib grinste und tat es ihm nach.

Jirisa trotte gemächlich hinter den beiden her und wurde keinen Deut schneller.\bigskip







Sonnenhoch. 42 Tage vor dem großen Unheil.\bigskip



Die Holzdielen knarrten wie eh und je unter Barz Stiefeln, als er einen ersten Fuß vom kleinen Böötchen auf die Pfahlbausiedlung Thakkum inmitten des großen Sees Ava setzte. Die Wasseroberfläche plätscherte ruhig und verheißungsvoll. Der große See! Seine Heimat! Nach seinen langen Reisen kehrte er immer wieder gerne an diesen Ort zurück. Er gab ihm neue Kraft. Und nicht nur ihm. Hier, auf der Pfahlbausiedlung Thakkum siedelte der Iquar-Stamm schon seit Jahrhunderten. Genauer gesagt, seitdem die drei legendären Barbaren-Brüder Yetohe, Iquar und Jpaxo sich miteinander verstritten und ihre Stämme allesamt andere Gebiete des Landes für sich beansprucht hatten: Yetohe die weite Steppe, Iquar den großen See und Jpaxo das hohe Plateau im südwestlichen Gebirge. Der Iquar-Stamm war längst nicht so zahlreich, wie der Jpaxo-Stamm es einst gewesen und der Yetohe-Stamm es immer noch waren. Doch im Gegensatz zu den anderen beiden Stämmen lebten die Iquar in einer Pfahlbausiedlung auf dem großen See, und so mussten sie weder Hunger noch Furcht vor Raubtieren oder feindlichen Armeen erleiden.

Das einzige Manko war der begrenzte Platz auf den Pfahlbauten. Barz und Nabib hatten die Steppenechse Jirisa bei den Echsenstallungen neben einer Weide mit ihren Artgenossen gleich außerhalb des Sees zurücklassen müssen. Glücklicherweise hatten sie rasch einige unbeschäftigte Burschen gefunden, die ihnen bereitwillig beim Transport all ihres Gepäcks von Jirisas Rücken ins Innere der Siedlung ausgeholfen hatten.

Tatsächlich hatten sich außergewöhnlich viele unbeschäftigte Menschen am Seeufer getummelt. Viel mehr Steppenechsen als sonst waren in und um die Stallungen gestanden, die meisten von ihnen mit schön verzierten Gurten und Sätteln bestückt. Und gleich dutzende schlichte Jurten waren daneben aufgestellt worden, zwischen denen es von Menschen nur so wimmelte. Durch die Öffnung einer Jurte hatte Nabib in deren Innerem eine hölzerne Statue erkannt und Barz darauf aufmerksam gemacht. Die Statue stellte wohl eine Art Rind dar, einen Büffel – auch wenn es dem Schnitzer an Talent zu mangeln schien. Barz und Nabib sahen sie nicht zum ersten Mal und wussten um ihre Bedeutung. Offenbar war sogar der Häuptling der Büffel-Sippe höchstpersönlich hier anwesend. Wenn sie es nicht besser wüssten, hätten Barz und Nabib anhand der schieren Anzahl Anwesender angenommen, dass der halbe Yetohe-Stamm auf Besuch in Thakkum war.

Nun trennte sich der Weg der beiden Freunde. Nabib konnte es kaum erwarten, seine Familie wiederzusehen. Die zahlreichen Zeichnungen, die er auf seiner Reise angefertigt hatte, würde er erst nach ausgiebigen Plaudereien mit seinen Eltern bei den Hütern des Wissens abliefern. Barz wollte hingegen lieber sofort seine Substanzproben ins Haus seiner Schamanin bringen und sich danach voll und ganz aufs Wiedersehen mit seiner Freundin Yafka konzentrieren können, mit der er auf dieser Reise wegen der Krarks nicht einmal Briefkontakt hatte halten können.

Beim Gedanken an Yafka tastete Barz unbewusst nach seiner Halskette, die er stets unter seiner wärmenden Kleidung trug. Zwei Ringe hingen daran: Ein geschnitzter Ring aus Holz, zu dem ein identischer auf Yafkas Halskette aufgefädelt war, und ein gehauener Ring aus Stein, zu dem Nabib einen identischen um seinen Hals trug.

„Heute Abend sehe ich dich wieder, zur sechsten Glocke vor dem großen Versammlungssaal“, versprach Nabib nach einer letzten Umarmung. Ganz gemäß Tradition würden heute Abend viele Iquar draußen rund um den Versammlungssaal speisen, und Nabib und Barz würden der Stammesleiterin Naquila und den versammelten Stammesmitgliedern vom abenteuerlichen Verlauf ihrer langen Reise berichten. Barz freute sich bereits ungemein auf den Anlass.\bigskip







Die Pfahlbausiedlung Thakkum war auf Pfählen inmitten des großen Sees Ava gebaut worden, an einer Stelle, wo der Seegrund vergleichsweise nahe unter der Wasseroberfläche lag. Barz hatte natürlich schon von mehrstöckigen Häusern gehört und auf einer Reise in die nördlichen Ausläufer des Grauen Gebirges tatsächlich bereits einige mehrstöckige Steintürme der Zwerge zu sehen gekriegt, doch verfügte Thakkum selbst bloß über ein einziges hohes Gebilde: Der dünne Zeitturm hinter dem Versammlungssaal, an dem die treuen Zeitmeister der Iquar zu jeder Stunde eine andersfarbige Fahne hissten. So konnte man von überallher in Thakkum in die Höhe sehen und die Zeit erkennen, auch wenn die Sonne gerade nicht mitspielte.

An diesem Zeitturm konnte sich Barz auch im üblichen Gewusel auf den Planken Thakkums gut orientieren und sein Haus rasch finden. Nabib wohnte gleich gegenüber, vermutlich war er schon lange eingetroffen. Barz atmete tief durch, zügelte seine Vorfreude und schritt näher.

Barz‘ und Yafkas Heim hatte aus der Ferne wie immer ausgesehen. Beim Nähertreten bemerkte Barz nun wie üblich einige Veränderungen, die in seiner Abwesenheit stattgefunden hatten. Hier war ein Fensterrahmen angeknackst, dort drüben wuchsen frische farbige Blumen aus einem Topf und die Haustür war neu bemalt worden, mit einer hübschen Rune, die vermutlich für Frieden oder Gesundheit stand. Runen waren noch nie Barz‘ Spezialität gewesen.

Die Tür schwang unter seinem sanften Druck auf und gab überraschenderweise den Blick frei auf ein kleines Kind, welches im Eingangsbereich saß und mit einem umherhüpfenden Ball spielte. Es warf Barz einen trotzigen Blick zu und fragte: „Wer bist du denn?“

„Dasselbe könnte ich dich fragen“, gluckste Barz und betrat sein Zuhause. Er verstaute seinen Bogen in einem Fach neben der Tür und hängte seinen langen Mantel mit den vielen Pulvertaschen daneben an die Wand.

„Mama! Mamaaaaa! Mama Yafka! Da ist ein Mann einfach so reingekommen!“, schrie das kleine Kind und rannte davon.

„Barz! Na endlich! Ich wusste doch, dass ich aus dem Fenster einen heimkehrenden Nabib erspäht habe“, rief eine Barz wohlbekannte Stimme von weiter hinten, und Barz wurde warm ums Herz. Yafka trat ins Licht, wischte sich ihre Hände an ihrer Schürze ab und beugte sich zum kleinen Kind herunter.

„Karyz, das ist Barz. Mein Freund, der auf Entdeckungsreise im ganzen Land war. Der mit der mächtigen Steppenechse Jirisa, ich habe dir doch Bilder von ihnen gezeigt.“

Während das Kind nickte und umgehend seinem Springball in ein anderes Zimmer nachhechtete, wandte Yafka sich Barz zu und erdrückte ihn fast in ihrer herzlichen Umarmung: „Endlich bist du zurück! Wir hatten uns schon solche Sorgen gemacht, als ihr nicht wie geplant vor zwei Wochen den Ava erreicht hattet. Mein tapferer Steppennomade. Komm her, lass dich herzen!“

„Papperlapapp, die richtigen Steppennomaden sind die Yetohe, die auch den Winter in der Kälte da draußen verbringen. Nabib und ich konnten viel von ihnen lernen, doch ist der Ava noch immer unsere feste Heimat, und per Definition muss ein No...“

Barz wurde von einem herzlichen Willkommenskuss unterbrochen.

„Ach, rede dich nicht runter. Willkommen zuhause! Schön, dass du es doch noch geschafft hast! Wegen dieser Krarks konnten wir nicht einmal Falken zu euch aussenden, um zu hören, ob es euch gut geht. Die alte Resi hat großspurig geweissagt, dass du und Nabib endlich in ein fremdes Land durchgebrannt wärt.“

„Ich würde dich nie freiwillig hier sitzen lassen. Ich hoffe, das weißt du.“

„Natürlich weiß ich das. Wie ist es euch ergangen?“

„Wir haben den großen Salzsee im Norden berührt!“, berichtete Barz mit leuchtenden Augen, „Selbst der Ava ist nichts als ein kleiner Tropfen dagegen.“

Yafka machte selbst große Augen und ratterte dann in beachtlichem Tempo herunter: „Ach, wirklich? Unglaublich! Kannst du mir alles erzählen? Bist du erschöpft? Magst du mir noch in der Küche helfen? Ich muss bis Sonnenhoch zwei Dutzend Rüben rüsten.“

Barz bejahte alle ihre Fragen und folgte Yafka in die Küche, wo ein beträchtlicher Stapel Goldrüben auf ihre Zubereitung warteten.

„Ich habe ein Geschenk für dich“, flüstere Barz verschwörerisch, „Sieh her!“

Er griff in eine seiner unzähligen Manteltaschen und warf daraus eine Prise glitzernden Staubs in die Höhe, welcher ohne Effekt zu Boden rieselte.

„Hübsch. Was war das?“, fragte Yafka, immer noch dem glitzernden Pulver nachsehend.

„Ein Ablenkungspulver!“, grinste Barz. Er hatte urplötzlich eine silbern schimmernde Blume in der Hand.

„Und das hier ist eine äußerst seltene Silberblume. Sie wächst am Ufer des großen Salzsees und ist ungenießbar, aber spür nur mal ihre Struktur. Überraschend robust für ein Gewächs, aber auch sehr selten. Ah, das Funkeln des roten Mondlichts auf ihren Blättern... ich musste sofort an deine Augen denken. Ich habe noch einige mehr gepflückt. Wenn irgendjemand es schafft, daraus Ableger zu ziehen, dann...“

„...dann wäre das meine Mutter selig gewesen.“

„Willst du zumindest versuchen, in ihre Fußstapfen zu treten? Die Wassermänner an der Küste schworen bei der Sauberkeit ihrer Schuppen, dass man die Zukunft schmecken kann, wenn man sich ein Blatt dieser Blume unter die Zunge legt.“

„Na, selbstverständlich. Ich kümmere mich gern darum. Sofern deine andere Schwiegermutter sie mir nicht abknöpft, sobald sie davon erfährt. Vielen Dank, Barz. Es ist schön, dass du wieder hier bist.“

Die beiden fielen sich erneut in die Arme.

„Apropos Mütter: ‚Mama‘?“, ahmte Barz die Stimme des kleinen Kindes belustigt nach, „Na, so lange bin ich aber nicht fortgeblieben, oder?“

Yafka gluckste und zog als Antwort ihre Halskette unter ihrem Kleid hervor. Neben Barz‘ Holzring baumelte nun ein zweiter, dünner Ring daran, der nach hellem Metall ansah.

„Oh Yafka, hast du wieder jemanden gefunden? Mein Glückwunsch! Wer kann sich denn so glücklich schätzen?“

„Ihr Name ist Zanyitaz und sie ist eine Fischerin vom Süddock. Du hast sie sicher auch schon gesehen. Karyz ist ihr Kind. Die beiden haben frisch das Gästezimmer bezogen. Zanyitaz ist gerade am Markt, doch sie wollte zum Mittagsmahl wieder hier sein. Ich kann es kaum erwarten, euch einander vorzustellen. Es würde mich sehr wundern, wenn ihr euch nicht verstündet.“

Yafka beäugte Barz ein bisschen besorgt, vermutlich auf seine Reaktion wartend. Sie ergänzte:

„Verzeih, wir wollten mit Austausch der Ringe und ihrem Einzug hier mindestens warten, bis du wieder hier bist. Oder zumindest bis du deinen Segen geben kannst. Aber dann konnten wir uns wegen der Krarks nicht mit euch austauschen und dann hattet du und Nabib solche Verspätung und dann musste Zanyitaz ihr Haus aufgeben – das ist eine spannende Geschichte für ein anderes Mal – und da dachten wir, wenn die Götter ihren Willen wollen, so sollen sie den halt kriegen. Du meintest doch immer wieder, dass dieses Haus zu groß sei für uns beide und dass ich, wenn du fort bist...“

Yafkas Stimme verlor sich. Barz, in dem plötzlich wieder ein Gefühl unglaublicher Zuneigung anschwoll, lächelte ihr zu, drückte sie fest an sich und küsste sie auf die Stirn.

„Keine Sorge, du kennst mich doch. Wir werden suchen, und wir werden Wege finden, mit denen wir allesamt zufrieden sind. Ich freue mich sehr für dich. Und bin natürlich gespannt darauf, diese Zanyitaz kennenzulernen. Und Karyz. Bei den Göttern, ein Kind in diesem Haushalt, wer hätte es für möglich gehalten!“

Yafka entspannte sich etwas und widmete sich wieder ihren Rüben. Barz folgte ihrem Vorbild und setzte nach: „Und da wirst du von Karyz bereits Mama genannt?“

Yafka grinste.

„Das wäre nicht meine erste Entscheidung gewesen, aber es hat sich so ergeben und.... ich mag’s irgendwie doch. Magst du mir jetzt vom riesigen Salzsee im Norden erzählen?“

„Klar, könnte ich, aber noch nicht zu viel. Sonst birgt der Reisebericht bei der Versammlung heute Abend ja gar keine Überraschungen mehr für dich.“

Yafka versteifte sich abrupt: „Oh, Barz, hast du es noch nicht vernommen? Die Versammlung heute Abend wird sich leider nicht um eure Rückkehr drehen können.“

„Ach? Ist etwas vorgefallen?“

„Hah! So könnte man es ausdrücken. Der König kommt zu Besuch.“\bigskip







Sonnentief. 42 Tage vor dem großen Unheil.\bigskip



Wie es bei Feiern üblich war, räumten an diesem Abend dutzende von Familien ihre Tische und Bänke nach draußen und verteilten Unmengen an schmackhaften Esswaren darauf. Als wollten sie sich alle gegenseitig dabei überbieten, die kreativsten Kreationen zum Konsumieren zu kredenzen, die man aus den kargen Pflanzen der Steppe und dem, was der See ihnen schenkte, kreieren konnte.

Ein einzelner Feuerzauberer aus einer fernen Eisinsel (Hadda? Harda? Der Name wollte Barz gerade nicht mehr in den Sinn kommen, es war jedenfalls diejenige Insel, von der er sich das seltene magische Cantharis-Pulver zusenden ließ) war während Barz‘ und Nabibs Abwesenheit an den Ava gelangt. Offenbar gehörte es zur Ausbildung der Zauberer, eine lange Reise in ein weit entferntes Reich zu machen, mit ihrer Magie zu helfen und Erfahrungen zu sammeln. Eine Einstellung, die die reiselustigen Barz und Nabib nur allzu gut nachvollziehen konnten. Dieser Feuerzauberer war möglicherweise ein wenig eitel (er nannte sich selbst stolz „den großen Lifornus“, ein Name, der nicht zuletzt mit seinem großen spitzen Zaubererhut zusammenhing, ohne den er nur selten gesehen wurde) verfügte aber auch über außergewöhnliche besondere Fähigkeiten, deren Demonstrationen ihn besonders beim Nachwuchs der Iquar eine begeisterte Beliebtheit verschaffte. Sobald Barz mit seiner Familie zur Versammlung trat, rannte Karyz zu Lifornus und bat ihn, Feuer zu spucken. Lifornus winkte lächelnd ab, doch während er sich abwandte, ging plötzlich sein Bart in Flammen auf, bloß für einen Augenblick. Karyz machte große Augen und Lifornus schlenderte davon, als wäre nichts geschehen.

Barz hatte einen wundervollen Nachmittag im Kreise seiner neu erweiterten Familie verbracht. Zanyitaz hatte sich als etwas schüchterne, aber gewitzte Person herausgestellt. Die Frauen hatten Barz in ein altes Strategiespiel eingeweiht, das im Süddock offenbar gerade der letzte Schrei war. Dann war Karyz dazugekommen und die Spielrunde war durch einen Spaziergang durch die Pfahlbausiedlung ersetzt worden. Barz hatte endlich wieder das Gefühl überkommen, zuhause zu sein. Ganz melancholisch war ihm geworden. Karyz hingegen war aufgedreht gewesen und war fröhlich den unter den Holzplanken hindurchschwimmenden Nixenkindern nachgerannt.

Nun, gegen Abend, floss Barz‘ Hochstimmung in einen düsteren Mischmasch aus Enttäuschung und Sorge über. Üblicherweise durften von Reisen zurückkehrende Iquar am Langtisch der Stammesleiterin Naquila neben dem Versammlungssaal speisen und von ihren Erlebnissen berichten, während die Nahestehenden aufmerksam lauschten und die Erzählungen nur leicht verfälscht an die weiter weg stehenden Tische weitergaben. Nun allerdings mussten Barz, Nabib und ihre jeweiligen Familien weiter weg Platz nehmen, der Versammlungssaal gerade noch knapp in Sichtweite. An ihrer statt saß neben der Stammesleiterin ein groß gewachsener Mann mit dichtem schwarzem Bart und einer imposanten goldenen Krone auf dem Kopf, deren breite Zacken selbst aus der Ferne noch stark glitzerten. Der Barbarenkönig. Begleitet wurde der König von gleich vier Häuptlingen aus verschiedenen Yetohe-Sippen. Je zwei Häuptlinge hatten links und zwei rechts des Königs Platz genommen hatten. Wenn Barz und Nabib näher säßen, könnten sie bestimmt die geschnitzten Büffel, Einhörner und dergleichen ausmachen, die die Kleidung der Häuptlinge zierten und ihre Clan-Mitgliedschaft deklarierten.

„Ich habe noch nie so viele Häuptlinge auf einmal gesehen“, raunte Yafka.

„Die Lage muss ernst sein“, nickte Zanyitaz.

„Wir waren einmal bei einem Treffen des Rats der Häuptlinge anwesend. Selbst da waren nur vier von denen dort, oder?“, flüsterte Nabib zu Barz. An die anderen gewandt führte er aus: „‚Anwesend‘ soll heißen, dass Barz und ich planlos in der Nähe der Zelte herumstanden, in denen die wichtigen Gespräche stattfanden. Die Yetohe sind bei politischen Diskussionen erheblich weniger offen als wir.“

Barz beendete seinen Gedankengang nachdenklich: „Mit der Anwesenheit seiner Häuptlinge will der König mächtig wirken, aber dass er überhaupt zugestimmt hat, sein Anliegen so öffentlich vorzutragen, zeugt von seiner Verzweiflung. Er ist auf die Iquar angewiesen. Ich werde reißen wie eine Bogensehne vor Spannung, wenn er nicht bald spricht. Will jemand noch Goldrüben?“

„Jaaa, ich will!“, rief Karyz und langte zu.\bigskip







Sonnenuntergang. 42 Tage for dem großen Unheil.\bigskip



„Hochverehrte Stammesleiterin, ich kann Euch nicht genug danken für die Gastfreundschaft, die die Iquar ihrem König entgegengebracht haben“, erscholl die tiefe, theatralische Stimme des Barbarenkönigs, woraufhin die in der Nähe sitzenden allesamt hastig von ihrem Geschirr aufsahen. Der große Lifornus verschluckte sich theatralisch und hustete zur großen Freude der umstehenden Jugendlichen einige Flämmchen hervor, ehe er ihnen zuzwinkerte und sich ebenfalls aufmerksam dem Barbarenkönig zuwandte.

„Es ist uns Ehre und Freude zugleich“, antwortete Stammesleiterin Naquila mit einem betont gelangweilten Unterton, der verriet, dass sie endlich zum Geschäftlichen kommen wollte.

„Euer König kommt mit schlechten Neuigkeiten aus dem Reich“, kam der König brav auf den Punkt, „Die Gefechte des Yetohe-Stammes gegen die dunkle Armee aus dem Gebirge im Süden verliefen lange Zeit triumphierend, und weiterhin kann blanker Knochen wenig gegen guten Stahl ausrichten. Doch scheinen die Reserven des Feinds wahrlich unerschöpflich, und mit jedem unserer gefallenen Krieger schenken wir seiner Sache einen weiteren tapferen Kämpfer.“

„Na, wenn der weitere tapfere Kämpfer für diesen Krieg anheuern will, dann wählt er die falschen Worte“, brummelte Nabib und füllte seinen Metkrug wieder bis zum Rand auf. Bereits zum zu vielten Mal, wie Barz an Nabibs leicht lallender Aussprache zu erkennen glaubte.

Der König ließ sich vom Getuschel nicht unterbrechen und verkündete mit von Bitterkeit triefender Stimme: „Es schmerzt mich, euch mitteilen zu müssen, dass wir das südliche Gebirge nicht bis in alle Ewigkeit werden halten können. Wir werden diese Lande verlassen und uns ein neues Zuhause suchen müssen. Häuptling Absorak von der Büffel-Sippe hat seine besten Krieger dem Wind Frais in den Westen folgen und das Königreich Andor hinter den Bergen ausspionieren lassen. Mit den zahlreichen Informationen, die Absorak nun zur Verfügung stehen, werden wir ihre kleinen Bauerndörfer im Nu überrennen, ihre baufällige Burg erstürmen und unseren Anspruch auf ihr fruchtbares Land geltend machen.“

Der Barbarenkönig schlug dem rechts neben ihm sitzenden Häuptling, vermutlich war das Absorak, wohlwollend auf den Rücken. Dieser verschluckte sich an seinem Eintopf, hustete heftig und blickte betreten zu Boden. Barz fragte sich, ob hinter seinem Blick mehr steckte als nur Scham über die verschüttete Suppe. Der Barbarenkönig ließ sich davon jedenfalls nicht den Wind aus den Segeln nehmen und fuhr schallend fort:

„Höre, o Stamm des Iquar! Dein König ruft dich an, sich den Yetohe beim Aufbruch in ein fremdes Land anzuschließen! Lange habt ihr Stämme zweier Brüder einander ignoriert, doch möget ihr nun wieder gemeinsame Sache machen. Mögen deine Krieger an vorderster Front bei der Eroberung unseres zukünftigen Reichs mitkämpfen! Denn die Gefahr aus dem Süden betrifft uns alle.“

Stille herrschte, nur unterbrochen vom Plätschern des großen Sees. Die unausgesprochene Warnung des Barbarenkönigs war unüberhörbar. Hätten sich die Yetohe erst einmal von der Front im südlichen Gebirge zurückgezogen, würde es nicht mehr lange dauern, bis die Skelette und Riesen aus dem Süden das Barbarenland gestürmt hätten. Der König glaubte offenbar nicht, dass die Pfahlbausiedlung Thakkum auf sich alleine gestellt dem Ansturm standhalten könnte. Oder er wollte sie das zumindest glauben lassen, damit die Krieger der Iquar ihn bei seiner Eroberung dieses Königreichs Andor unterstützen würden.

Der Appell des Königs war gleichzeitig als heroischer Aufruf und als Befehl formuliert gewesen. Doch schon die reine Anwesenheit des Königs und seiner Häuptlinge hier strafte seine selbstsicheren Worte Lügen. Wäre der Angriff auf Andor eine längst so sichere Sache, wie er das gerne den Anschein lassen wollte, so hätte er bloß einen Boten geschickt und sich nicht groß um die Antwort der Iquar geschert. Dass er hier war, zeigte doch, dass er verzweifelt auf Unterstützung hoffte. War der König auch bereits beim dritten Stamm, den Drachenkultisten der Jpaxo im westlichen Gebirge, vorstellig geworden? Hatten sie ihn abgewiesen? Technisch gesehen hatte der König Befehlsgewalt über alle drei Barbaren-Stämme, doch in der Realität folgten ihm nur die Sippen der Yetohe beim Wort, während die anderen beiden sich in allem außer dem Namen von seiner Regentschaft losgelöst hatten.

Dies wurde umso deutlicher, als Stammesleiterin Naquila das Wort ergriff und mit unverhohlener Abneigung sprach: „Dieses fremde Königreich, von dem ihr sprecht, dieses Andor... erinnere ich mich falsch, oder haben unsere Vorfahren nicht mit den dortigen Bewahrern einen Friedenspakt eingegangen?“

Der Barbarenkönig brummelte etwas in seinen Bart und antwortete gereizt: „Pakte sind ehrenhaft, ja, doch auch alte Pakte müssen manchmal gebrochen werden, wenn es ums reine Überleben geht. Das war den Bewahrern ebenso klar wie unseren Vorfahren, als sie damals den Frieden schlossen.“

„Und wie sieht es um die Landschaft in Andor aus? Selbst wenn Euer kühner Eroberungsplan aufgeht: Kann das Land nicht nur ein, sondern gleich zwei zusätzliche Völker ernähren? Sollen wir Iquar unsere sichere Heimat von Jahrhunderten einfach aufgeben, für eine unsichere Zukunft voller Kriege und Gefahren?“

„Kriege und Gefahren stehen den Iquar bevor, ganz gleich, wie der Stamm handelt. Die Yetohe ziehen Bauerngesindel und friedliche Bewahrer als Gegner einer Armee von Skeletten und Riesen vor. Wenn sie erst einmal vor euren Stegen stehen, würdet ihr euch nicht wünschen, andere Prioritäten gesetzt zu haben?“

„Mit diesem Krieg, den die Yetohe durch ihren unvorsichtigen Feldzug ins Land der Riesen begannen, haben wir Iquar nichts zu tun. Wir spüren keinen Druck durch diese Armee. Der große See ernährt uns, und er wird uns beschützen, wie er schon seit Anbeginn tat.“

„Wollt ihr euch etwa eurem König widersetzen?!“, brüllte Häuptling Absorak ungehalten auf, ehe er unter dem strengen Blick seines Königs verstummte.

Wildes Getuschel waberte durch die Reihen der Zuhörer. Selten wurde die Autorität des Barbarenkönigs und des Rats der Häuptlinge offen angezweifelt, und auch der König selbst schien diese Möglichkeit lieber nicht offen angesprochen zu haben.

Stammesleiterin Naquila richtete sich auf und alle Augen auf sie. Sie blickte die restlichen hochrangigen Tiere der Iquar der Reihe nach an. Barz konnte deren Gesichter nicht genau erkennen, vermutete aber kalte Entschlossenheit auf ihnen. Naquila nickte und sprach langsam, deutlich und bedacht, aber nicht ohne einen gewissen genüsslichen Unterton in ihrer Stimme: „Die Iquar kennen keinen König. Wir folgen einzig dem Gebot der Götter. Und die Götter haben nicht gesprochen. Wir sind mitfühlend gegenüber der hoffnungsarmen Lage der Yetohe und wünschen ihnen den Segen der Götter bei der Eroberung Andors. Doch wir werden keinen Teil daran haben.“

Die Herausforderung in ihren Worten konnte Barz nur schwer übersehen. Der König hatte keine Befehlsgewalt über die Iquar, solange sie die seine nicht anerkannten, und er konnte es sich nicht leisten, in einem offenen Gefecht gegen die Iquar seine Macht zu demonstrieren. Nicht, solange er alle seine Krieger fit für eine Invasion brauchte.

Häuptling Absorak schlug mit seiner Faust auf den Tisch und verteilte gute Suppe über seine Nachbarn. Der Barbarenkönig hingegen nickte ergeben. Als er wieder aufstand, schien alle Hochmütigkeit von ihm abzufallen. Leise sagte er:

„Dann bleibt mir bloß noch, eine Bitte an den Stamm der Iquar zu richten. So rasch wie möglich werden wir mit der Invasion Andors beginnen, doch nicht alle Mitglieder der Sippen können oder sollen in der ersten Welle mitreisen. Es gibt Alte und Schwache in unseren Reihen. Bitte, gestattet ihnen, ihre Jurten neben dem Ava aufzuschlagen, bis die Lage in Andor nicht mehr so gefährlich für sie ist. Unterstützt sie mit Nahrung und Heilmitteln. Und falls die Armee der Skelette zuvor bereits den Ava erreicht, mögen uns die Götter davor bewahren, so weist sie bitte nicht ab, sondern bietet ihnen Zuflucht auf den sicheren Pfählen von Thakkum.“

Stammesleiterin Naquila trat einige Schritte zurück und beriet sich mit den hochrangigen Tieren des Stammes. Dann verkündete sie kalt: „Ihre Jurten mögen die Zurückbleibenden in der Steppe aufschlagen, wo sie wollen. Falls dies neben dem Ava sein soll, so sei es so. Doch unsere Vorräte und unser Platz sind begrenzt. Falls es zu einer Belagerung durch diese Armee der Untoten kommen sollte, so werden die Iquar zwei Dutzend Mitglieder Yetohe in die Pfahlbausiedlung lassen, ehe wir die Stege kappen.“

„Zwei Dutzend?!“, schrie Absorak ungehalten, „Das reicht nicht im Geringsten! Wollt ihr uns etwa ausrotten?“

Der Barbarenkönig zischte ihm etwas zu und bedankte sich beinahe unterwürfig bei den Iquar für ihre Unterstützung in den düsteren Zeiten, die da kamen. Weiteres Getuschel ertönte in den Reihen der Iquar. Hier und da sah man Kopfschütteln. Auch Häuptling Absorak schüttelte entschieden seinen Kopf. Er richtete sich auf und richtete sein Wort an die gesamte Versammlung:

„O Iquar! Lasst ihr etwa zu, dass eure Stammesleiterin eure Geschwister so behandelt? Mir scheint, es wird hier über euren Kopf hinweg entschieden. Es ist nicht möglich, dass ihr alle wie sie gleichsam feige und egoistisch seid. Es geht hier ums Überleben eines ganzen Volkes!“

Zwei weitere Häuptlinge johlten Beistimmung. Der Barbarenkönig, obschon er seinen Kopf noch gesenkt hielt, beobachtete die Lage aus wachsamen Augen. Stammesleiterin Naquila versuchte, Absorak das Wort abzuschneiden, doch Absorak schrie nur noch lauter:

„Macht euch nichts vor in eurer ‚sicheren‘ Siedlung! Nicht nur unser Überleben steht auf der Kippe, sondern auch das eure. Ich habe die Riesen mit eigenen Augen gesehen. Das sind keine moralischen Wesen, es sind Monster. Sie werden diesen See überrennen, und alle, die sich bis dann noch hier aufhalten, werden Teil ihrer untoten Armee werden. Denkt an eure Geliebten und eure Kinder. Seht dieses Mädchen da vorne, wie es unbedacht mit seinem Essen spielt. Wenn ihr uns nicht auf der Suche nach einer neuen Heimat unterstützt, wird sein Skelett in wenigen Monden mit dem Feind marschieren, einen rostigen Säbel in der kleinen Hand!“

Rufe von Seiten der Zuhörer wurden laut, Absorak solle gefälligst seine Klappe halten. Stammesleiterin Naquila blickte sich hilflos um. Auch Barz blickte besorgt um sich. Er zweifelte nicht daran, dass die Iquar auf ihren Pfahlbauten sicher waren vor einer Armee aus dem Süden, die nicht einmal Boote besaß. Doch konnte es keine guten Folgen haben, wenn Absorak die Angelegenheit persönlich machte und Ängste weiter schürte. Die Yetohe waren verzweifelt und im Gegensatz zu den Iquar mussten sie jetzt handeln, also war ihr Verhalten durchaus verständlich. Doch der Plan des Barbarenkönigs war hirnrissig. Barz war noch nie selbst in Andor gewesen, doch hatte er genug über dieses Königreich gehört, um die Unwahrscheinlichkeit dieses Vorhabens abzuschätzen. Die Andori hatten fast die gesamte Trollpopulation des Landes ausgemetzelt. Trolle! Selbst eine gepanzerte Steppenechse konnte einem Troll nicht das Wasser reichen. Sich den Yetohe in diesem Kriegszug anzuschließen, glich Selbstmord.

„Gibt es denn niemanden unter euch, der sich noch um seine Familie kümmert?! Gibt es keinen, in dessen Adern noch das ehrbare Blut der ersten Brüder fließt? Wir Yetohe brechen morgen auf nach Andor. Zeigt, dass ihr besser seid als eure feige Stammesleiterin. Jeder, der sich uns bei der Invasion anschließt, wird fürstlich belohnt werden!“, beendete Absorak seine Tirade.

Hier und da ertönte zustimmendes Gejohle aus dem Publikum. Einige Iquar standen gar auf und hoben ihre Metkrüge. Am nächsten Morgen würden diese sich wohl den Yetohe anschließen und ihre Leben im Kampf gegen einen überlegenen Gegner riskieren. Barz schüttelte bloß seinen Kopf.

Dann drehte er sich um und sein Herz hüpfte in seine Hose. Nabib war auch aufgestanden, hatte seine Hand zur Faust geballt und klopfte sich damit auf die Brust.

„Setz dich wieder hin!“, zischte Nabibs Schwester ihm zu und zupfte an seinem Umhang, doch Nabib blieb standhaft stehen und johlte gemeinsam mit den übrigen aufgerichteten Iquar. Immer mehr erhoben sich, doch Barz kümmerte sich nicht mehr darum.

„Nabib! Du bist angetrunken, jetzt solltest du lieber keine wichtigen Entscheidungen treffen!“

In diesem Moment verkündete die tiefe Stimme des Barbarenkönigs: „Yetohe! Iquar! Im Morgengrauen ziehen wir los. Wer so tapfer ist, sich den Invasoren Andors anzuschließen, möge sich dann vor dem Steg versammeln. Ich danke jedem von euch bereits jetzt. Eure Kinder und Kindeskinder werden das in der Zukunft dann auch tun.“

Nabib nickte Barz mit einem traurigen Grinsen zu: „Du siehst: Es heißt jetzt oder nie.“\bigskip







Sternenhoch. 42 Tage vor dem großen Unheil.\bigskip



Erst nachdem die Nacht schon lange eingebrochen war und das Sternenband sich über den Himmel gelegt hatte, löste sich die Versammlung auf. Tische und Stühle wurden zurück ins Innere der Häuser verfrachtet und die Yetohe zogen sich wieder in ihre Jurten draußen auf dem festen Steppenboden zurück.

Barz hatte Nabib den Großteil des Abends sich selbst überlassen. Nun bewegte Nabib sich von seiner Schwester geleitet in Richtung seines Zuhauses und Barz fühlte, dass dies seine letzte Chance werden würde, ihn umzustimmen. Er langte hinter seinem Rücken nach einem Pulvergürtel.

„Nabib! So warte doch!“, rief Barz. Er erreichte ihn gerade noch, ehe er im Inneren seines Hauses verschwunden wäre.

„Ich hab’s ihm schon den ganzen Heimweg auszureden versucht“, sprach Nabibs Schwester, „Tu, was du nicht lassen kannst, aber lass ihm lieber seinen Willen, als dass ihr euch im Zwiste voneinander verabschiedet.“

Mit diesen Worten wurde Nabib stehen gelassen. Sofort legte Barz auf:

„Nabib, bitte, halte ein und sinniere nach über das, was du tun willst. Du hast Familie hier. Willst du sie wirklich im Stich lassen?“

„Oh, Barz, wenn ich hier dringend gebraucht würde, wäre ich doch gar nicht erst auf so lange Reisen aufgebrochen. Das gilt auch für dich, Barz. Du könntest mitkommen nach Andor.“

Barz hielt für einen Augenblick inne.

„Wir sind doch gerade erst heimgekommen. Ich will Yafka nicht gleich wieder verlassen.“

„Ich doch auch niemanden hier zurücklassen. Aber im Gegensatz zu dir sehe ich über meine aktuellen Wünsche hinweg und sehe, dass die einzige Art, wie ich meiner Familie und meinen Freunden tatsächlich helfen kann, darin besteht, dafür zu sorgen, dass für sie eine Zukunft... garantiert... ooh, dieser Satz ist zu lange für meinen müden Schädel. Jedenfalls ist es das, was ein wahrer Held tun würde.“

„Dann bin ich offensichtlich kein wahrer Held. Doch ist euer Vorgehen doch nichts als eine Verzweiflungstat.“

„Es ist zweifelsohne heldenhaft, für eine sichere Zukunft seiner Geliebten zu sorgen.“

„In ein fremdes Land einzudringen und einen aussichtslosen Kampf anzufangen, das nennst du eine sichere Zukunft?“

„Sicherer, als wenn wir hier bleiben und nichts tun, während eine finstere Armee aus dem Süden anrückt.“

„Das kannst du nicht mit Sicherheit wissen! Der Ava hat uns schon seit Jahrhunderten vor allen möglichen Gegnern bewahrt und kann es wieder tun.“

„Das kannst du ebensowenig mit Sicherheit wissen. Diese Skelettarmee ist nicht wie die Gegner, mit denen die Barbaren zuvor bereits zu kämpfen hatten. “

Stille machte sich zwischen den beiden breit. Dann...

„Nabib, ich werde dich vermissen. Ich werde mir solche Sorgen um dich machen. Wir waren noch nie so lange getrennt...“

Barz zog mit seiner freien Hand aus einem der dutzend kleinen Täschchen in seinem Mantel ein reich verziertes Amulett hervor. Seine Großmutter hatte es ihm zu seiner Volljährigkeit geschenkt. Nun führte er es zu Nabibs warmer Hand und legte es sanft hinein.

„Nimm dieses Amulett, auf dass es dich beschützen möge. Und dass du mich nicht vergisst.“

„Oh, Barz, du alter Romantiker. Wenn du mir einen Antrag machen willst... nicht jetzt. Glaub mir, es ist schon so schwer genug. Ich habe mich entschieden, und ich werde diese Entscheidung durchziehen. Und glaube ja nicht, dass du mir nicht fehlen wirst.“

Nabibs Stimme zitterte leicht, als seine Finger sich von Barz‘ lösten und er das reich verzierte Amulett an sich nahm. Im Dunkeln war sein Gesicht nur schwer zu erkennen, aber Barz glaubte, Tränenspuren glitzern zu sehen.

„Na, komm schon her, du sturer Hornbär“, brachte Barz hervor. Dann fielen die beiden Steppennomaden einander in die Arme. Barz drückte Nabib fest an sich und ließ für Minuten nicht mehr los. Nabib erwiderte die Umarmung. Barz atmete Nabibs Geruch ein letztes Mal ein. Dann ließ er ihn los. Nabib winkte ihm ein letztes Mal zu, betrat sein Haus und machte sich vermutlich umgehend ans Packen.

Barz trat einige Schritte zurück und betrachtete das Sternenband über dem klaren Himmel, die glitzernden Sterne vor dem schwarzen Hintergrund.

Dann öffnete er seine Hand und die Prise Bannpulver, die er darin bereitgehalten hatte, verstreute sich grünlich glitzernd im Wind, rieselte durch die Holzplanken und versank im schwarzen Ava. Barz‘ Hand kribbelte magisch, als er den letzten Rest des schimmernden Pulvers davon abrieb.

Es war eine gute Entscheidung gewesen, das Pulver nicht zu benutzen, sprach er zu sich. Es war gut, Nabib seinen eigenen Weg gehen zu lassen, statt ihn entgegen seines unbrechbaren Willens hier zu behalten. Doch glaubte Barz das noch nicht wirklich.

























\newpage
\section{Der Angriff der Riesen}


Sonnenaufgang. 41 Tage vor dem großen Unheil.\bigskip



Am Morgen flossen am Ufer des Avas noch einmal die Tränen, als sich Nabib endgültig von seiner Familie und von Barz verabschiedete. Er strubbelte seiner Nichte durch die Haare, umarmte Yafka, küsste Barz und hastete dann so schnell es sein Stolz ihm erlaubte davon, ehe er noch länger mit seiner Entscheidung hadern konnte.

Mit ihm brach der Großteil der Yetohe und eine Vielzahl an freiwilligen Iquar auf. Auf in Richtung Andor. Ein schier endloser Zug von Barbaren, einige von ihnen auf riesigen Steppenechsen reitend, und noch mehr Steppenechsen mit Jurten, Nahrung und Waffen beladen. Ganz vorne ritt der Barbarenkönig, umringt von seinen treusten Häuptlingen und einigen tapferen Kriegern. Barz kniff die Augen zusammen und versuchte zu erkennen, ob Nabib sich gar zu diesen Elitekämpfern gesellt hatte. Er hoffte es nicht. Nabib war ein Zeichner, ein Kartograph. Dass er außergewöhnlich gut mit einer Axt umgehen konnte, machte ihn noch lange nicht zu einem Meisterkämpfer.

Nachdem Barz zum letzten Mal sichergestellt hatte, dass kein Yetohe seine geliebte Steppenechse Jirisa mit den anderen mitgeführt hatte (und nachdem Barz, an Jirisas Seite gekuschelt, ein letztes imaginäres Streitgespräch mit Nabib zu dessen Abziehen durchgekaut hatte), schweiften seine Gedanken zu denjenigen Yetohe, die nicht fortgezogen waren.

So einige Jurten waren zurückgeblieben. Die Alten und Schwachen, die Jungen und Friedliebenden, sie alle warteten nun neben dem Ava auf eine gute Nachricht aus Andor. Ebenso wie diejenigen Iquar, deren Familie und Freunde nun auf einem Kriegszug waren. Waren vielleicht gar mehr Iquar mit dem Barbarenkönig aufgebrochen, als sie Yetohe hier abgeladen hatten? Schwer zu sagen.

Barz verbrachte den Rest des Sonnenlichts damit, Pfeile in Zielscheiben zu versenken und seine Rückkehr an den Ava zu verfluchen. Er und Nabib hatten bereits solche Verspätung gehabt. Wären sie nur einen Tag später heimgekommen, so hätten sie den Abzug der Yetohe verpasst. Und Nabib wäre noch hier.

Barz jagte einen letzten Pfeil ins Schwarze einer Zielscheibe. Dann kehrte er nach Hause zurück. Immerhin hatte Yafka die Vernunft besessen, in Thakkum zu verblieben. Und Zanyitaz und Karyz. Es war an der Zeit, sich an seine neue Familiendynamik zu gewöhnen. Und an der Zeit, mit seiner Schamanin die gefundenen Substanzen der letzten Reise zu analysieren. Wer weiß, vielleicht würde er ja gar etwas herausfinden, was gegen die unweigerlich in dieser Gegend aufkreuzende Armee der Riesen aus dem Süden helfen werden könnte.\bigskip







Sonnenaufgang. 23 Tage vor dem großen Unheil.\bigskip



Die Tage waren ins Land geflossen und zu Wochen geworden. Barz hatte viel Zeit mit seiner Familie verbracht, und viel Zeit bei seiner Schamanin für die Analyse der Stoffproben aufgewendet, welche er auf seiner langen Reise gesammelt hatte. Das konnte ihn zwar nicht immer von seinen Sorgen um Nabib ablenken, doch oft genug.

Nun kam der Tag, an dem die Armee des Riesen am Horizont erspäht werden konnte. Barz hatte sie als einer der ersten erblickt, als er mit Jirisa im Schlepptau und Karyz auf deren Rücken einen Spaziergang in der Nähe des großen Sees Ava gewagt hatte. Sie hatten gesehen, wie sich aus der Ferne eine ungeheure Streitmacht dem Ava näherte. Wie sie sich wie ein großer grauer Wurm über die weite Ebene der Barbarensteppe wälzte. Und man musste kein Held, kein Krieger und kein Feldherr sein, um zu wissen, was das hieß: Der letzte standhafte Trupp der Barbaren, der sich nicht der Invasion Andor angeschlossen, sondern sich weiterhin dem Heer der Riesen entgegengestellt hatte, war schlussendlich überrannt worden. Niemand stand mehr zwischen den Skeletten und dem großen See Ava. Und niemand stand mehr zwischen den Skeletten und dem Königreich Andor, wenn die Riesen beschlossen sollten, den Spuren der Yetohe in den Norden zu folgen.

Die übriggebliebenen Yetohe brachen ihre Jurten ab und flohen nach Thakkum, auf die schützenden Pfähle. Die Stammesleiterin Naquila versuchte, nach drei Dutzend Hilfesuchenden einen Schlussstrich zu ziehen, doch schien kein Iquar wirklich erpicht darauf, diese Entscheidung durchzusetzen, und so wurde sie in stiller Übereinkunft missachtet.

Die letzten Jurten und Totems, die nicht mit auf den See genommen werden konnten, wurden feierlich in Brand gesetzt, auf dass sie nicht dem Feind in die Hände fallen mochten. Die letzten Steppenechsen, die nicht gen Andor gezogen waren, wurden freigelassen. Barz blickte Jirisa lange in die Augen, ehe er sie mit einem Klaps zum Forteilen anhielt. Natürlich hielt Jirisa sich nicht daran und blieb störrisch neben den Stallungen stehen, egal, wie sehr Barz weit weg von hier zeigte. Am Ende musste er sie gewahren lassen.

Alle Boote, die noch am Ufer befestigt gewesen waren, wurden auf den See geschifft und mit der Siedlung verbunden. Die Pfahlbausiedlung lag einsam im großen See.

Nachdem sich so viele Iquar dem Angriff auf Andor angeschlossen hatten, gab es theoretisch ausreichend Platz auf der Pfahlbausiedlung für die übriggebliebenen Yetohe, auch wenn die Verteilung auf die verschiedenen Haushalte ein organisatorischer Albtraum werden würde. Barz, Yafka und Zanyitaz nahmen ein uraltes Ehepaar bei sich auf, welches sich oft zankte, insgeheim aber wohl beide füreinander durchs Feuer springen würden. Die beiden konnten gute Geschichten ohne Ende erzählen. Karyz mochte sie besonders und Zanyitaz fand in ihnen würdige Gegner in so manchen Brettspielen.

Doch war niemandem in Thakkum wirklich wohl zu Mute.

Denn rund um den See herum befanden sich nun sie.

Die Untoten.

Das grusligste an ihnen war die Stille. Jede andere marschierende Armee wäre schon von fern an Stimmen, Gebrüll, vielleicht gar Marschmusik erkannt worden. Doch die Skelettkrieger der Riesen liefen pausenlos im Gleichschritt, während nichts das leise Klappern ihrer Rüstungen und Waffen sowie das rhythmische Trommeln ihrer knöchernen Füße auf die weiche Erde zu vernehmen waren.

Zunächst hatten sie sich dem Ava kaum genähert, als wollten sie einfach nur am See vorbeiziehen und weiter in Richtung Andor marschieren, auf den Spuren der Armee des Barbarenkönigs. Dann aber schien sich etwas zu ändern. Vielleicht hatte eine der Kommandanten die Siedlung Thakkum inmitten des großen Sees erspäht. Egal, was es gewesen war, die Armee der Skelette hatte jedenfalls Halt gemacht und Stellung rund um den Ava bezogen. Doch standen sie nicht still. Stattdessen umrundeten die Skelette das Seeufer. Stetig marschierten sie, Tag und Nacht, ohne zu ermüden, ohne vom Pfad abzukommen. Sonst taten sie nichts. Selbst die herumstehenden Steppenechsen am Seeufer ließen sie in Ruhe, wie Barz überrascht (und erfreut) feststellte, als er seine eigene Steppenechse Jirisa ungestört nur einen Steinwurf entfernt von der vorbeimarschierenden Armee weiden sah. Was wollten diese Skelette nur?\bigskip







Sonnenhoch. 20 Tage vor dem großen Unheil.\bigskip



Am ersten Tag nach der Ankunft der Armee am großen See ertönten einige dumpfe Trommeln gerade dann, als die Sonne am höchsten Punkt am Himmelszelt stand. Eine riesengroße Gestalt trat nach ans Ufer des Avas. Still stand der Riese dort, während seine Armee hinter ihm vorbeimarschierte. Dann hob er seinen Arm und zeigte auf die Siedlung. Er brüllte etwas in einer tiefen, kehligen Sprache, das keiner der Anwesenden verstand. Doch blieb er weiter stehen, als wartete er auf eine Antwort. Hin und wieder wiederholte er seinen Ruf:

„Ungadahr Ferntahr! Parahnu Krahder! Darack Ambacus Berand Impalahr Tugabahr! Konthra Impalahr Rah!“

Glücklicherweise verfügte Barz‘ Schamanin über einen Trank, der ihr erlaubte, Strukturen in gesprochenen Worten zu erkennen und sie besser sortieren zu können, sodass es ihr gelang, die Sprache der Riesen anhand der wenigen Worte, die sie aus der Ferne gehört hatte, zu erkennen. Das Rezept für diesen Trank hatte sie einst von einem Schamanen der Yetohe erlernt, der ihn wiederum vom Großen Büffel höchstpersönlich gelernt bekommen hatte. Zumindest hatte dieser Schamane der Yetohe bei seinem Bart darauf geschworen, und den Yetohe waren ihre Bärte durchaus wichtig.

Nun konnte Barz‘ Schamanin ein bisschen von diesem außergewöhnlichen Trank schlucken, um danach für die restliche Barbarengesellschaft zu übersetzen:

„Der Riese schreit: ,Name Ferntahr, Prinz Unsterbliche.‘ Dieses eine Wort, ‘Krahder‘, ‚Unsterbliche‘, bedeutet noch mehr. Mir scheint, das könnte auch der Name dieser Riesen für ihr eigenes Volk sein.“

„Er hat sich uns also vorgestellt. Wie höflich“, spottete Stammesleiterin Naquila, „Wie geht es weiter?“

„Gebt mir einen Augenblick, diese Interpretation ist kein Kinderspiel... ‚Menschen... Ambacus... Fügsamkeit... Folge... Belohnung! Widerstand... Folge... Tod!‘ Mir scheint, er fordert uns auf, sich ihm zu ergeben. Das Wort ‚Ambacus‘ ist interessant. Ich spüre eine Assoziation mit ‚Folgsamen‘, aber mit einem unschönen Unterton dahinter. Vielleicht wäre ‚Unfreie‘ passender? Oder gar ‚Sklaven‘?“

„Ja, dass die Riesen aus dem Süden nicht nur Skelette, sondern auch Sklaven befehligen, ist schon seit längerem bekannt. Meint dieser Prinz wirklich, wir würden uns so leichtfertig ergeben? Wer würde einem solchen Feind einfach die Tore öffnen, nur weil er darum bittet?“

„Niemand, der doch Hoffnung hat. Und über die verfügen wir zum Glück ja.“\bigskip







Sonnenaufgang. 19 Tage vor dem großen Unheil.\bigskip



Am nächsten Morgen wurde die ganze Siedlung von den kehligen Rufen eines Krahders geweckt. Dieses Mal war es nicht Prinz Ferntahr, der sich ihnen stellte, sondern eine vergleichsweise kleine Riesin mit krummen Rücken, die sich auf einen beinernen Stab stützte, an dem mehrere gehörnte Schädel einer unbekannten Spezies hingen. Die Riesin hielt irgendetwas grün dampfendes an ihre Kehle und ihre Stimme drang viel lauter über den See, als Prinz Ferntahrs es geschafft hatte.

„Ungadahr Nahrack! Kosdivahr Krahder! Cohuiron Impalahr Vahnur! Zeruhor Gorb-Palacs Ihra!“

Wieder versammelten sich einige Schaulustige und Stammesleiter um die Schamanin, als sie die Worte zu übersetzen versuchte:

„Name... Nahrack! Hexe... Krahder! Kampf... Folge... Verlust! Waffen... Niederlage... Befehl!“

„Die hoffen immer noch darauf, dass sie uns zermürben können und uns ergeben, ehe sie zu aufwendigeren Mitteln greifen müssen. Wollen wir ihnen langsam eine Antwort geben, ehe sie zum Schluss kommen könnten, dass wir sie nicht verstünden?“, meinte der große Lifornus nachdenklich. Dieser Feuerzauberer, der eigentlich nur zum Abschluss seiner Ausbildung an den Ava gekommen war, war lieber in Thakkum geblieben, statt sich den Invasoren des Barbarenkönigs nach Andor anzuschließen oder alleine in die Steppe zu wandern. Es war möglich, dass er nun daran zweifelte, ob das die richtige Entscheidung gewesen war. Sein einst so schneidiger Bart war ungepflegt und man sah Lifornus immer häufiger in seinem Schlafmantel umherlaufen.

„Was wissen wir über diese Krahder?“, fragte Stammesleiterin Naquila in die Runde.

„Es gibt tatsächlich bloß zwei, die die Skelettarmee befehligen“, antwortete Barz. Seitdem er mithilfe seiner Schamanin aus einigen seltenen Materialen, die er in seiner letzten Reise mit Nabib gesammelt hatte, ein einzigartiges Umwandlungspulver geschaffen hatte, war er quasi über Nacht zu einer Koryphäe in Sachen Fernrohrproduktion geworden und hatte so irgendwie die Verantwortung über die Beobachtung der feindlichen Armee erhalten.

„Es mag einen anderen Anschein haben, weil wir verteilt über die ganze Armee immer wieder größere Riesengestalten sehen, aber unsere Fernrohre enthüllten uns, dass die meisten von ihnen Untote sind, nichts als puppetierte Knochenhaufen. Tatsächlich sahen wir bislang bloß diese zwei lebenden Krahder, diesen Prinz Ferntahr und diese Hexe Nahrack. Nur zwei Personen in einem riesigen Heer aus leblosen Puppen. Es wäre lächerlich, wenn unsere Lage nicht so prekär wäre.“

Naquila schmunzelte tatsächlich leicht, ehe sie nachhakte: „Und was könnt Ihr uns über diese beiden Krahder sagen?“

„Prinz Ferntahr bewegte sich gestern immer wieder rastlos umher und musterte die Umgebung. Er hat sich sogar schon einmal vorsichtig einer Steppenechse genähert und sie gestreichelt. Er scheint beinahe... aufgeregt? Wir können sein Gesicht selten sehen, er trägt oft einen Schädel darüber. Wie eine Art Krone vielleicht? Ich bin mir nicht ganz sicher über die Spezies, zu der dieser Schädel einst gehörte. Es könnte gut ein gefallener Krahder sein. Oder ein kleiner Drache? Und dann ist da diese Nahrack, diese Hexe. Sie verfügt über irgendwelche düsteren magischen Kräfte. Sie verbringt die meiste Zeit zwar im Zelt, das die Krahder am Ufer aufgebaut haben, doch einmal wurde sie gesehen, wie sie nach vorne trat und einen grünen Blitz auf einen zu tief fliegenden Krark schleuderte.“

„Sie ist eine Dunkle Hexe“, murmelte der große Lifornus, „In Hadria haben wir davon gehört. Einige unserer älteren Adepten hatten sich in ihren Reisen bis ins Graue Gebirge und darüber hinaus gewagt. Und unangenehme Begegnungen gemacht. Es ist zweifelsohne Dunkle Hexerei, die die toten Knochen dieser Skelette umherwanken lässt. Die Frage ist, ob die Dunkle Hexe sie aktiv steuert oder ob sie weiterhin agieren könnten, nachdem man die Hexe stürzte.“

„Wie stellt Ihr es Euch vor, die Dunkle Hexe von hier aus zu stürzen?“, warf Naquila ein.

„Jedenfalls einfacher als eine riesige Skelettarmee.“

Lifornus und Naquila funkelten sich gegenseitig an, während sie beide verstummten.\bigskip







Sonnenhoch. 18 Tage vor dem großen Unheil.\bigskip



Am nächsten Tag war die Lage noch nicht besser geworden. Die Bewohner Thakkums waren relativ ratlos, und den Krahdern schien es nicht besser zu entgegen. An diesem Tag stand pünktlich zum Sonnenhoch wieder die Krahder-Hexe Nahrack am Ufer, doch dieses Mal hatte sie eine Begleiterin neben sich stehen. Eine großgewachsene, doch neben Nahrack immer noch mickrig erscheinende Menschenfrau, die allem Anschein nach noch lebte. Eine Unfreie, eine Ambacu. Sie war mit seltsamen dunklen Tuchstreifen bekleidet und ihr Gesicht war noch weißer als das eines Skelettkriegers. Vermutlich Gesichtsfarbe. Kriegsbemalung? War dies nur Tradition oder steckte eine magische Bedeutung dahinter?

Gebückt und unterwürfig stand die Menschenfrau da, doch als Nahrack ihr ein grünlich dampfendes Etwas vors Gesicht drückte, klangen zunächst ihr stockender Atem und dann auch ihre stockenden Worte gellend laut über den Ava. Die Ambacu sprach eine andere Sprache als die Krahder, doch leider ebenfalls keine, die die Iquar ohne Schamanentrank verstanden hätten. Barz‘ Schamanin übersetzte, und als die Botschaft wieder als eine Mischung aus Appellen zum Ergeben und Versprechen von Unversehrtheit bei raschem Aufgeben ihrer Position bestand, wurde auch dieser Versuch der Kontaktaufnahme der Krahder verworfen.

Stammesleiterin Naquila sprach feierlich: „Der große See schützt uns und schenkt uns Nahrung. Wir können theoretisch beliebig lange im Ava aushalten. Eigentlich ist es sogar besser, wenn die Krahder uns belagern, statt die Invasoren nach Andor zu verfolgen.“

Die Anwesenden nickten allesamt zufrieden. Dass die Krahder und ihre Skelette nicht einfach so vom Ufer verschwinden würden, war allen klar, doch solange sie keine akute Gefahr darstellten, wirkte plötzlich niemand mehr so erpicht darauf, nach einer Lösung zu suchen.

Und so schien die Sache fürs Erste gegessen. Es traten auch keine weiteren Ambacus oder Krahder ans Seeufer, nur die stetig umhermarschierenden Skelettkrieger verblieben. Ob die Riesen glaubten, dass die Bewohner Thakkums sie nicht verstünden, oder ob sie verstanden, dass sie vermutlich keine Antwort erhalten würden, war nicht klar.

Klar war somit nur, dass der nächste Zug in diesem Spiel von den Krahdern gemacht werden würde.\bigskip







Kurz vor Sonnenaufgang. 14 Tage vor dem großen Unheil.\bigskip



Barz wurde von einem leisen Rumpeln geweckt. Yafka regte ich leise neben ihm.

„Hast du das auch gehört?“, zischte Barz ihr zu. Yafka nuschelte etwas und vergrub sich wieder in ihrem Kissen.

Nun ertönte gar ein Klirren von der Küche. Barz richtete sich auf und schlich vorsichtig aus dem Raum. Eigentlich gab es keinen Grund, besorgt zu sein. Der Ava schützte die Pfahlbausiedlung vor feindlich gesinnten Personen und Einbrüche waren eine ausgesprochene Seltenheit. Doch seit der Ankunft der Krahder lagen seine Nerven wie die von vielen Einwohnern Thakkums blank. Auch wenn sich vermutlich nur Karyz in die Küche geschlichen hatte, wollte er lieber doppelt nachprüfen, statt etwas Übles zu übersehen.

Barz lugte um die Ecke und erkannte: Es hatte sich tatsächlich jemand in die Küche geschlichen! Doch war es kein Kind. Zanyitaz stand mit einem Topf bewaffnet vor dem Esstisch und rührte sich nicht.

„Zan? Ist alles in Ord...“

Weiter kam Barz nicht, denn Zanyitaz zuckte bei seinen Worten zusammen. Ihr Blick glitt hinüber zu Barz und ihre Ablenkung wurde gnadenlos ausgenutzt. Aus einer aus Barz‘ Blickfeld nicht sichtbaren Ecke der Küche jagte ein Skelettkrieger mit einer rostigen Wurfaxt in der Hand hervor und ließ selbige auf Zanyitaz niedergehen. Diese wuchtete ihren Topf gerade noch rechtzeitig nach oben, damit die Axt sie nur an der Schulter streifte. Einen üblen Schnitt hinterließ sie dennoch.

Nahkampf war noch nie Barz‘ Stärke gewesen, hatte es auch nicht sein müssen, da man in der Steppe jeden Gegner in Pfeilreichweite schon lange sah, ehe eine Situation derart brenzlig werden konnte. Wenn Barz auf seinen Reisen in ein unschönes Gefecht geraten war, war es also mit Pfeil und Bogen ausgetragen worden. Mit dem Bogen, welcher zwei Räume weiter drüben unnütz an einem Haken hing, während Zanyitaz hier in akuter Lebensgefahr war. Also tat Barz das einzige Vernünftige: Er raste auf den Skelettkrieger zu und packte dessen Faust, welche die Wurfaxt umschlossen hatte.

Da sah Barz ein, wie unvernünftig das gewesen war. Der Skelettschädel drehte sich auf seiner losen Wirbelsäule, bis er Barz aus leeren Augenhöhlen einen feurigen Blick entgegenwerfen konnte. Dann vergrößerte sich irgendwie der Abstand zwischen den Knochen bei den Gelenken des Skelettkriegers. Dieser tat einen mächtigen Schritt und wanderte buchstäblich durch Barz hindurch. Barz spürte das Kratzen von Knochen auf allen Seiten seines Körpers, während das Skelett sich über ihn stülpte. Und plötzlich war es nicht mehr Barz, der von hinten den Arm des Skeletts fixierte, sondern das Skelett, welches von hinten Barz‘ Arm auf seinen Rücken verdrehte. Immerhin hatte Barz weiterhin die Wurfaxt im Griff.

Das war wirklich faszinierend. Es schien, als wären die Fäden Dunkler Hexerei, die den Skelettkrieger zusammenhielten und seine Bewegungen ermöglichten, völlig durchlässig für gewöhnliche Materie. Eine solche Verbindung wäre für zahlreiche andere Anwendungen überaus nützlich. Wobei, zuerst müsste man vermutlich testen, wie stabil die Verbindung war. Barz hatte schon davon gehört, wie starke Kämpfer der Yetohe ein Skelett in seine Einzelknochen zerrissen hatten, woraufhin es sich auch nicht mehr zusammensetzen hatte können. Und selbst die außergewöhnlichste Verbindung nützte kaum etwas, wenn sie schnell riss.

Leider war im Moment gerade keine Zeit für solcherlei Experimente. Stattdessen rangelte Barz noch kurz mit dem Skelettkrieger, bis dieser ihn in sein Ohr biss und Barz die Wurfaxt fallen ließ. Innerlich schloss er bereits mit seinem Leben ab.

Da ertönte ein lautes Klonk direkt neben Barz‘ Ohr. Zanyitaz hatte mit ihrem schweren Kochtopf den Schädel des Skelettkriegers glatt von dessen Wirbelsäule gefegt und das Skelett wankte zurück. Barz dankte den Göttern, ließ die Axt fallen, rannte zu einer seiner Substanzschublade, griff nach dem erstbesten Pulver, welches dort gelagert wurde, und warf eine gehörige Portion davon in Richtung des Skeletts.

Leider war das das Meditationspulver gewesen, und folglich wenig hilfreich. Braunroter Rauch stieg auf und das Skelett wankte ungestört weiter herum. Sein Fuß stieß auf die fallen gelassene Axt und hob sie hoch.

Barz griff erneut in die Schublade und nahm sich diesmal die Zeit, nach dem grünlich schimmernden Bannpulver Ausschau zu halten. Er griff sich eine Handvoll davon, ballte seine Faust fest und schüttelte sie. Als er sie wieder öffnete, glitzerte das Pulver rasch immer heller und Barz‘ Hand begann, magisch zu kribbeln. Rasch drehte er sich um und schleuderte die Faustvoll dem nun kopflosen Skelettkrieger entgegen, welcher gerade Zanyitaz mit erhobener Axt entgegensprang.

Das Bannpulver traf das schädellose Skelett mitten in die Rippen, und auch wenn einiges davon wirkungslos zwischen selbigen durchflog, hatte Barz genug der seltenen Substanz geschleudert, damit es seine Wirkung entfalten konnte. Magisches Knattern ertönte und ein hellgrünes Glühen baute sich um das Skelett auf, dessen Sprungangriff sich verlangsamte, bis es mitten in seinem Schlag festgefroren in der Luft hing.

Der Schädel des Skeletts klapperte ein bisschen entfernt vom restlichen Geschehen auf dem Küchenboden herum, doch konnte dieser auf sich alleine gestellt bloß seinen Unterkiefer auf- und zuklappen und sich dadurch nicht fortbewegen. Der war also keine Gefahr.

Barz schnappte sich ein Tuch und wedelte den Rest des Bannpulver-Dampfs aus dem zerstörten Fenster hinaus, ehe sich dessen magische temporäre Wirkung weiter ausbreiten konnte. Ah, da war der Skelettkrieger also eingedrungen. Dann wandte Barz sich Zanyitaz zu und half ihr auf. Die Wunde an ihrer Schulter sah unschön aus.

„Danke für die Hilfe“, sagte er. Zanyitaz verzog ihr Gesicht und nickte schwach.

„Yafka! Schnapp dir ein Messer, greif dir Karyz und verbarrikadier dich im Schlafzimmer!“, rief Barz laut, während er seine Pulverschubladen nach heilendem Cirathans-Pulver durchsuchte.

„Was ist los?“, erklang Yafkas verschlafene, doch alarmierte Stimme dumpf durch zwei Wände hindurch.

„Skelettkrieger in der Küche!“

„Skelettkrieger in der Küche?! Was ist der über den See gekommen?“

„Woher sollen wir das wissen?“

„Geht es euch gut?“

„Uns geht’s soweit gut. Die akute Gefahr ist gebannt!“

„Jetzt hört auf zu schwafeln, schnapp dir endlich Karyz und versperre Tür und Fenster, wir kommen dann nach!“, brüllte Zanyitaz überraschend laut auf und kippte zur Seite. Barz zuckte zusammen und wandte sich wieder seiner Suche nach dem stärkenden Cirathans-Pulver zu, während polternde Schritte aus dem Nebenzimmer von Yafkas raschem Handeln zeugten. Dann endlich hatte Barz das Säcklein mit Cirathans gefunden. Nun, ob es wirklich ein Pulver war, war debattierbar, für manch peniblen Beobachter (darunter immer wieder gerne Barz‘ Schamanin) mochte die grobkörnige Substanz aus zu großen Körnern bestehen, um noch als echtes Pulver gelten zu dürfen. Aber das war jetzt nicht relevant. Barz pickte drei mittelgroße blaue Cirathans-Körner heraus, rannte zu seinem Regal der eingesperrten Winde und griff sich eine Phiole mit luftdicht verschlossenem grünem Cantharis-Pulver daraus.

Keine Minute zu früh. Kaum hatte Barz die Phiole ergriffen, erglühte die kleine Staude Mondbeeren auf dem Schrank strahlend weiß. Barz wusste genau, was das bedeutete: Die ersten Sonnenstrahlen der aufgehenden Sonne fielen durch die Fenster ins Haus und ließen die Mondbeeren magisch aufleuchten. Dies war ein sehr nützlicher Effekt, um die sonst sehr unscheinbaren Beeren aufzuspüren. Dies war allerdings alles andere als nützlich, wenn man die temporären Anomalien, die Mondbeeren auslösen konnten, ausnutzen wollte, um eine Gefahr zu bannen. Und Bannpulver bestand bekanntlich zu einem Großteil aus getrockneten Mondbeeren.

Barz wandte sich zurück zum gebannten Skelett und sah gerade noch, wie auch der einst grüne Schimmer des Bannpulvers um es leuchtend weiß wurde, ehe er sich vollständig verflüchtigte. Der Skelettkrieger beendete seinen langen Sprung durch die Küche. Seine Axt grub sich in die Wand, wo zuvor noch Zanyitaz gestanden war.

Der Schädel des Skeletts klapperte weiterhin nutzlos am Boden, während der Krieger ziellos seine Axt umherschwang und im Zimmer umhertorkelte. Brauchten Skelette etwa ihre leeren Augenhöhlen, um zu sehen? Faszinierend.

Doch diesmal war Barz vorbereitet. Er schleuderte die Phiole mit dem giftigen Cantharis dem Skelett entgegen. Während er zwar nie der allerbeste Schütze des Iquar-Stammes gewesen war, so hatte er auch noch nie sein Ziel vollkommen verfehlt. So kam es auch hier: Die Phiole traf das Skelett an der Schulter, zerbrach und setzte das Cantharis-Pulver frei, welches sich auf der Schulter des Skelettkriegers verteilte. Im Nu erschienen Verätzungserscheinungen auf seinem Schulterblatt und grünlich schimmernder Giftdampf rauchte hervor.

Zeitgleich mit dem Freisetzen des Cantharis-Dampfs begannen die drei großen blauen Cirathans-Körner in Barz‘ Hand zu magisch zu glänzen. Rasch kniete sich Barz neben Zanyitaz und presste die Pulverkörner in deren verletzte Schulter. Augenblicklich breiteten sich schwach fluoreszierende bläuliche Ranken von jedem Korn aus, mehrere Zentimeter lang über Zanyitaz‘ Schulterwunde und ins verwundete Fleisch hinein.

Im selben Masse, wie das Skelett durch das Gift geschwächt wurde, heilte Zanyitaz‘ Schulterwunde. Diese Verbindung von Giftpulver aus Cantharis und Stärkungspulver aus Cirathans war ein wunderbares Beispiel für das Gleichgewicht der Magie, welches Barz mit seinen Werken stets zu wahren versuchte.

Als Zanyitaz‘ Wunde größtenteils verkrustet war und sie erleichtert aufatmete, verdunkelten sich die farbigen Ranken der blauen Körner und fielen zu Boden. Barz packte den großen Kochtopf und schleuderte ihn ein letztes Mal auf das umhertorkelnde kopflose Skelett, welches zusammenbrach. Barz packte den Topf erneut und ließ ihn immer wieder auf das Skelett niedergehen, bis kein einziger seiner langen Knochen sich mehr rührte. Gruslige Viecher waren das, bäh!

Erst da wurde Barz sich bewusst, dass er soeben die sterblichen Überreste eines Menschen zerschlagen hatte. Das war eine Person mit Gefühlen und Wünschen gewesen, ein tapferer Krieger der Yetohe oder ein unter unglaublichen Widrigkeiten durchhaltender Sklave der Riesen. Barz legte seine offene Hand auf seine Brust, Handfläche nach außen, und sandte ein Stoßgebet an die Götter. Neben ihm tat Zanyitaz es ihm gleich.

Dann rannte Barz zwei Räume weiter, packte seinen Köcher und seinen Bogen und rannte nach draußen, um nach weiteren Skelettkriegern Ausschau zu halten.\bigskip







Kurz vor Sonnenaufgang. 14 Tage vor dem großen Unheil.\bigskip



Die Pfahlbausiedlung der Iquar war mitten im Ava gebaut worden, an einer Stelle, an der der Seegrund beinahe die Wasseroberfläche erreichte, sodass die Holzpfähle nicht allzu lange sein mussten. Es war in der Steppe schon schwer genug, stabile Holzstämme zu finden. Manche behaupteten gar, dass damals, als die ersten Pfähle der Siedlung gesetzt worden waren, die Siedlung noch am Ufer einer Insel inmitten eines damals über einen erheblich tieferen Wasserstand verfügt habenden Avas gelegen hatte. Damals, ehe der unterirdische Fluss, der den Ava speiste, plötzlich viel mehr Wasser geführt hatte.

Der springende Punkt sollte sein, dass nicht der ganze Ava so flach war wie unter der Siedlung der Iquar. Tatsächlich fiel der Seegrund rund um die Siedlung relativ rasch und tief ab. In diesen Tiefen lebten zahlreiche Lebewesen. Fische und Seetang, die von den Iquar gezüchtet wurden. Doch auch Nixen und Seerösser. Ja, manche erzählen gar von Riesenkraken und Spornwalen, auch wenn seit Jahrzehnten keiner von diesen mehr gesehen wurde. Doch uns geht es nicht um Riesenkraken oder Spornwale. Wir wollen uns lieber einer Unterwasserstadt der Nixen zuwenden.

Denn zeitgleich mit dem Angriff eines einzelnen Skelettkriegers auf Barz‘ Haus gingen, nur einige hundert Meter von Thakkum entfernt, ähnliche finstere Dinge vor sich.

Die Nixen des Avas standen in engem Kontakt mit den Bewohnern der Pfahlbausiedlung, wie es sich zwischen einander gut gesinnten Nachbarn gehörte. Und auch sie fanden die Anwesenheit einer fremden Skelettarmee vor ihren Ufern äußerst beunruhigend und hatten einige Wachen aufgestellt. Diese Wachen der Nixen patrouillierten oft die Ufer und sollten sofort Alarm schlagen, wenn plötzlich etwas Unvorhergesehenes geschehen würde.

Heute wurde Alarm geschlagen. Denn eine Nixe namens Jarna hatte ein totes Seeross erspäht. Die seltenen Seerösser galten den Nixen als sehr wertvoll, und einen Leichnam zu finden, sorgte immer für Unruhe. Doch besonders schlimm war, dass das Seeross unzweifelhaft an Stichwunden verstorben war. Welche Nixe würde es wagen? Die Antwort war klar: Keine Nixe. Knöcherne Fußspuren im Seeschlamm zeugten vom Übertäter. Ein Skelettkrieger der Krahder hatte sich unter den See gewagt! Natürlich, fiel es ihnen plötzlich wie Schuppen vom Fischschwanz: Skelettkrieger mussten nicht einmal immer wieder an der Wasseroberfläche Luft holen gehen, wie es die Nixen taten. Sie konnten theoretisch beliebig lange unter halb der Wasseroberfläche umherspazieren und Schaden anrichten.

Jarna und zwei weitere Wächterinnen jagten den Fußspuren nach. Sie zerlegten den einzelnen Skelettkrieger, welcher in den Ava vorgestoßen war, ohne große Mühe, und brachten seinen immer noch tapfer klappernden Schädel zu Ilfara, der Königin der Nixen.

Ilfara war natürlich außer sich.\bigskip







Sonnenhoch. 14 Tage vor dem großen Unheil.\bigskip



Das nächste Treffen der oberen Stammesmitglieder der Iquar war um einiges angespannter als das letzte. Glücklicherweise waren nur einige wenige, vereinzelte Skelettkrieger an verschiedenen Orten in Thakkum eingedrungen, kaum ein halbes Dutzend an der Zahl, und die meisten waren gar unschädlich gemacht worden, ehe sie ein Leben hatten nehmen können. Dies war kein Angriff auf die Siedlung gewesen, sondern nur eine Machtdemonstration. Ein Zeichen der Krahder, dass Thakkum nicht so sicher war, wie die Iquar es gerne hätten. Ein Zeichen, dass der so für Sicherheit gesorgt habende große See für ihre Skelettkrieger kaum mehr als ein leicht durchquerbares Sumpfland war. Ein Zeichen, dass sie sich besser bald ergeben sollten.

Nicht nur die menschlichen Bewohner des Sees litten plötzlich wieder mehr Kummer und Sorgen. Bei dieser Versammlung der oberen Stammesmitglieder waren nun auch Vertreter der Nixen anwesend. Eine Linguistin der Iquar kniete neben einigen Löchern am Holzboden. Hin und wieder tauchte ein Nixenkopf mit grimmiger Miene aus einem Loch herauf, um der Linguistin eine gluckernde Nachricht in der blubbernden Nixensprache zuzugurgeln, woraufhin die Linguistin heftig nickte und den weiter vorne liegenden Stammesmitgliedern etwas zuzischte. Barz zweifelte daran, dass auch die beste Linguistin je Nixisch aussprechen könnte, doch schien das auch nicht nötig, denn allem Anschein nach verstanden die Nixen die Barbaren bestens. Und der Lautstärke ihrer Beschwerden über die angreifenden Horden der Krahder nach würden viele Stammesmitgliedern wohl liebend gerne so tun, als verstünden sie weniger von den Nixen, als sie tatsächlich taten.

Das Nixenvolk sah die Schuld für diesen Angriff klar auf Thakkums Schultern. Barz teilte diese Meinung nicht zwingend. Klar, ohne die Pfahlbausiedlung hätten die Krahder wohl kaum Stellung um den Ava herum bezogen. Doch jetzt, wo die Krahder von ihrer aller wussten, könnten sich die Nixen doch nicht einmal eines zukünftigen Friedens sicher sein, wenn die Barbaren sich augenblicklich den Horden ergeben und in den Süden verschleppt werden würden. Nein, sie saßen alle im selben Boot, und sollten sich besser alle gemeinsam um diese Lage kümmern.

Durch den Angriff hatte sich die Stimmung der Obersten gegenüber der Gefahr gewandelt. So sprach Stammesleiterin Naquila gerade: „Wir haben uns geirrt. Der See schützt uns nicht auf ewig vor den Skeletten. Wir werden natürlich Vorsichtsmaßnahmen treffen. Wachen aufstellen und Patrouillen umhermarschieren lassen. Die Pfähle, das Fundament unserer Siedlung, mit Stacheln ausstatten oder aber sie so glitschig machen, dass kein Skelett je daran hochsteigen könnte.“

Unabhängig davon, ob Naquila tatsächlich an die Möglichkeit ihrer Worte glaubte oder nur Hoffnung verbreiten wollte, sprach sie rasch weiter, ehe eine der anwesenden Nixen über diese nur die Menschen schützenden Maßnahmen protestieren konnten.

„Doch wenn selbst die Seerösser der Nixen an den Skeletten zugrunde gehen, so können wir nicht darauf hoffen, auf ewig hier in Thakkum ausharren zu können. So sehr es mich wurmt, das zugeben zu müssen: Wir müssen evakuieren.“

Eine der Nixen hob ihre Hand, vermutlich um zu fragen, wie es den Nixen helfen sollte, wenn Thakkum evakuiert würde und sie in einem von Skeletten belagerten See übrig blieben. Dann senkte sie sie wieder. Vielleicht war ihr aufgefallen, dass es ohnehin so gut wie keine Möglichkeit zur Evakuation gab.

Dies schien auch Naquila einzusehen, als sie niedergeschlagen fortfuhr: „Doch wenn nicht einmal die Armee des Barbarenkönigs etwas gegen die Krahder und ihre Armee ausrichten kann, hege ich keine großen Hoffnungen für uns.“

Bei diesen Worten erhob sich der große Lifornus von seinem Sitz, breitete theatralisch seine Arme aus und sprach, als wäre sein Moment gekommen: „Nun, soweit ich weiß, hatte die Armee des Barbarenkönigs keinen Magier bei sich.“

Naquilas Miene verdüsterte sich, doch schien sie zu erschöpft, um eine bissige Bemerkung zurückzugeben. Auch Barz fragte sich, ob es Lifornus ins Hirn geschneit hatte. Wollte er etwa andeuten, dass seine Macht der dutzender gefallener Schamanen an den südlichen Fronten überlegen war?

„Was soll das? Habt ihr irgendwelche versteckte Kräfte, mit denen ihr die ganze feindliche Armee verjagen könnt?“, fragte die Dolmetscherin der Nixen spontan.

„Ich wünschte, es wäre so“, fuhr Lifornus melancholisch fort, „Doch verfüge ich über etwas fast gleichsam Hilfreiches: Die Macht, Kraft meiner Magie Hilfe zu rufen.“

„Um Hilfe wollt ihr rufen?“, meldete sich Stammesleiterin Naquila nun wieder, „Wer würde uns denn noch helfen wollen? Der Stamm der Yetohe, die womöglich bereits in ihrer Invasion Andors gescheitert sind oder durch sie zumindest unglaublich erschöpft wurden? Der Stamm der Jpaxo, der sich tatenlos im Gebirge verkrochen hält? Deine hochwohlgeborenen Zaubererfreunde aus dem hohen Norden, die sich einen feuchten...“

„Beruhigt Euch bitte, ich bin auf Eurer Seite. Lasst mich ausholen.“

Lifornus räusperte sich und erzählte mit einer Baritonstimme, die den besten reisenden Barden ansehend nicken lassen würde:

„Schon seit jeher brechen junge Zauberer in meiner Heimat Hadria zum Abschluss ihrer Ausbildung in die weite Welt auf. Viele kehren mit allerlei neugewonnenem Wissen zurück, manche gar mit seltenen Substanzen oder Erfindungen. Und einst gab es eine Zauberin, die bei ihrer Rückkehr aus dem Königreich Andor – welches damals noch nicht Andor hieß und auch noch kein Königreich war – von einem brennenden Vogel berichtete, einem sogenannten Phoenix.“

Lifornus legte eine Kunstpause ein.

„Dieser Phoenix war angeblich vor Generationen von einer Einsiedlerin gefunden worden, welche versucht hatte, ein unüberwindbares hohes Gebirge im Westen zu überwinden, dessen Gipfel stets in Wolken gehüllt waren. Manche munkelten gar, das Gebirge sei unendlich hoch, obwohl diese These anhand der endlichen Länge des Schattens des Gebirges natürlich unzweifelhaft widerlegt werden kann.

Doch zurück zu unserer bergsteigenden Einsiedlerin. Sie kehrte um, als sie dieses kurz vor dem Tode stehende magische Feuerwesen entdeckte, und nährte es, bis es zu einem stattlichen feurigen Vogel anwuchs, größer als jeder Hornfalke. Sie dachte, dass sie durch ihre Hilfe den Vogel vor dem sicheren Tod bewahrt hätte. Doch wie es sich herausstellte, ist der Tod für diesen Phoenix nicht dasselbe Ärgernis wie für uns. Wann immer sich der Lebenszyklus des Vogels seinem Ende näherte, verging er in Flammen und hinterließ ein kleines Feuerküken, welches von neuem heranwuchs.“

Eine Nixe blubberte etwas, und die Dolmetscherin übersetzte leise: „Magst du mal zum Punkt kommen?“

Etwas verärgert, doch nicht weniger pathetisch, fuhr Lifornus fort: „Nun, dieses Feuertier konnte nicht nur sein Leben immer wieder aufs Neue beginnen. Nein, er schlug nicht nur der Zeit, sondern auch dem Ort ein Schnippchen: Der Phoenix konnte sich auch von Ort zu Ort teleportieren, wenn er nur so wollte: Er würde an einem Ort in einem Flammenbausch verschwinden und an einem anderen in einem ebensolchen wieder erscheinen. Zurück blieb jeweils nur ein kleines Häuflein Asche. Unsere Zauberin, welche auf ihrer Reisen in Andor in Kontakt mit dem Phoenix gekommen war, brachte ein Säcklein dieser Phoenix-Asche nach Hadria zurück. Dort studierte sie es und schaffte es, mithilfe der Asche kleine Münzen zwischen Orten hin und her zu teleportieren. Daraufhin ließ sie immer mehr Phoenix-Asche von Andor nach Hadria transportieren. Es gelang ihr, mithilfe dieser Phoenix-Asche immer größere Objekte und später auch kleine Lebewesen zu teleportieren. Ihr Durchbruch kam, als sie aus der Asche einen Zauberspiegel der Teleportation zu formen vermochte.

Die Zauberin ist inzwischen schon lange verstorben, doch steht zu vermuten, dass der Phoenix noch immer irgendwo in Andor gehalten wird, weitergereicht von Generation zu Generation. Wenn es uns gelingt, den Phoenix hierher zu rufen und aus seiner Asche einen Zauberspiegel zu formen, so könnten wir alle auf dieser Siedlung festsitzenden Menschen in Sicherheit teleportieren. Und ich kenne ein Ritual, welches den Phoenix rufen könnte. Doch ich muss euch warnen! Es handelt sich um ein gefährliches Ritual, und niemand kann das volle Ausmaß der Konsequenzen einschätzen, sollte es schiefgehen.“

Es kam zu einer Abstimmung. Die knappe Mehrheit der Stammesleiter sprach sich für Lifornus‘ Ritual aus. Stammesleiterin Naquila war prominent dagegen, ebenso alle anwesenden Nixen, denen Naquila daraufhin plötzlich Stimmrechte zuschrieben wollte. Auch wenn nur wenige Nixen auf einmal durch die Löcher im Boden erscheinen und der Linguistin ihre Wünsche vortragen konnte, zeugte zahlreiches wütendes Plätschern, Blubbern und Gurgeln von unterhalb des Holzbodens von der Anwesenheit zahlreicher anderer Wasserbewohner unter der Versammlung. Barz schlich sich zur Linguistin und bat sie, den Nixen zu vermitteln, dass auch sie von diesem Teleportationsspiegel Gebrauch machen könnten und dass es auch andere, sicherere Gewässer für sie gab. Vielleicht irrte er sich, doch glaubte er, dass daraufhin das wütende Gluckern unterhalb des Bodens etwas nachließ.

Lifornus fuhr indes ungeachtet dessen fort: „Sehr schön, die Mehrheit hat wieder einmal die beste Entscheidung getroffen. Dann können wir uns ja an die Arbeit machen und diesen Phoenix herbeirufen! Doch lassen wir uns lieber genügend Zeit, dieses Ritual vorbereiten. Wisst Ihr, es ist besser, wenn ich mir einige Tage Zeit lasse mit dem genauen Ausformulieren der Zauberformel. Die Welt beugt sich dem Willen der Zauberer, und oftmals reicht ein starker Glaube in die Magie und ein Zauberstab, um unsere Kräfte wie gewünscht zu leiten. Doch für große, weitreichende Wirkungen kann es geschickter sein, die Kräfte der Zauberei direkt anzurufen. Nun, wir stellen es uns so vor, als ob wir mit ihnen reden würden, doch ob die Essenz der Magie tatsächlich ein Bewusstsein hat, ist eine ganz andere Frage...“

Naquila verdrehte bereits wieder ihre Augen und Lifornus unterbrach sich:

„Nun, um unseren Willen den Kräften der Zauberei zu kommunizieren, sprechen wir die uralte althadrische Sprache. Doch es ist manchmal vertrackt, die passenden Worte zu finden. Das Althadrische kennt zum Beispiel kein eigenes Wort für ‚Phoenix‘ und wenn ich die Mächte der Zauberei dazu aufrufen würde, mir einen ‚Vartum‘, also einen ‚Feuervogel‘, zu bringen, dann könnte es übel enden. Denn die Vokabel ‚Tum‘ steht sowohl für ‚Vogel‘ als auch für ‚Sturm‘. Sprachen sind nun mal kompliziert. Und wir wollen definitiv keinen ‚Feuersturm‘ in unserer Mitte beschwören. Lasst mich darum... oh, oh, wartet!“

Lifornus schlug sich an den Kopf – sein spitzer Zauberhut wackelte bedrohlich – und murmelte dann beschämt: „Mir fällt gerade auf, dass die Formulierung der Zauberformel unser geringstes Problem ist. Das Ritual bedarf verschiedenster Zutaten, die wichtigste darunter ein gewisses Kraut, welches gar wundersame Wirkungen hat. Es vermehrt sich in der Blütezeit rasch, eine einzelne Staude nährt einen gestandenen Zwerg für einen ganzen Tag, gar seltene Fieber vermag es zu heilen. Doch heimisch ist es in diesen Landen nicht, und manch ein Zauberer brach auf der Suche danach in den Süden auf, nur um mit leeren Händen zurückzukehren. Eine Zucht davon in Nordgard, was gäbe ich nicht dafür. Ich befürchte, dass ich mir selbst und auch euch ohne dieses Kraut nicht werde helfen können.“

Da erhob sich Barz. Das Auffinden seltener Pulver und Tränke, Steine und Pflanzen war schließlich sein Spezialgebiet. Sein Moment war gekommen. Und der seiner Schamanin, die wohl über die außergewöhnlichste Kräutersammlung diesseits des Grauen Gebirges verfügte.

„Noch ist nicht aller Tage Abend, lieber Lifornus. Meine Schamanin verfügt über die wohl außergewöhnliche Kräutersammlung diesseits des Grauen Gebirges und ich bin auf meinen Reisen in den Kontakt mit Kräuterkundler der verschiedensten Völker gekommen. Nennt uns den Namen.“

„Sternkraut. Die Pflanze heißt Sternkraut. Sie soll laut den Chroniken aus dem Oktron im Grauen Gebirge wachsen.“

Barz suchte unter den Anwesenden nach seiner Schamanin, doch diese antwortete auf seinen fragenden Blick nur mit betrübtem Kopfschütteln. Doch so rasch würde Barz nicht aufgeben.

„Dann werden wir weitersuchen. Boten können wir in dieser Zeit der Belagerung wohl kaum welche aussenden. Doch ich werde umgehend einen Brief an den Stamm der Jpaxo im südwestlichen Gebirge verfassen. Ich kenne eine ihrer erleuchteten Drachenkultisten ziemlich gut. Wenn das Kraut dort in der Nähe wächst, wird sie uns das mitteilen können.“

Lifornus nickte und lehnte sich zurück.

Die Stammesleiterin Naquila hingegen warf ein: „Mit Verlaub, Pulvermeister Barz, ich will uns diese Hoffnung nicht streitig machen, doch wie im Namen der Götter gedenkst du, einen Brief an unseren Gegnern vorbeizuschleusen?“

„Die Krahder und Skelette? Keiner von ihnen scheint einen funktionsträchtigen Bogen bei sich zu haben. Von ihnen droht einem Nachrichtenvogel keine Gefahr.“

„Halte mich nicht zum Narren, Barz, ich meine natürlich die Krarks. Diese Raubvögel machen schon seit langem unseren Himmel unsicher. Kein Falke wird durch die Steppe bis ins Gebirge reisen können, ohne von einem Krark angefallen zu werden.“

Barz grinste schelmisch: „Keiner hat gesagt, dass wir die Nachricht mit einem Falken übermitteln würden.“\bigskip







Sonnenhoch. 13 Tage vor dem großen Unheil.\bigskip



„Das ist Wahnsinn“, murmelte Zanyitaz.

„Das ist Hoffnung“, widersprach Barz. Er schüttete lieber noch ein bisschen mehr orangefarbenes Schlafpulver in den Ködersack und band diesen zu, so fest er konnte.

„Krarks können fast unbegrenzt lange im Himmel kreisen, wenn sie so wünschen, und so überqueren sie auf der Suche nach Nahrung riesige Gebiete. Doch ihre Nester liegen oftmals im Gebirge.“

„Da vorne fliegt einer!“, rief Yafka Barz zu. Barz unterbrach seine Erklärung und blickte durch das Fernrohr, das ihm seine Freundin entgegenhielt. Tatsächlich! Ein braunroter Fleck schwebte hoch oben über ihnen am hellblauen Himmel. Jetzt musste er nur noch näher kommen. Er bereitete seinen Bogen vor.

„Die Drachenkultisten der Jpaxo leben an vereinzelten Stellen im Gebirge. Sie versuchen hin und wieder, Krarks abzurichten, und manchmal gelingt es ihnen sogar“, meinte Barz an Zanyitaz gewandt, „Wenn wir einem Krark eine Nachricht ans Bein binden, so klein und so leicht, dass sie ihm selbst kaum auffallen wird, dann wird früher oder später ein Mitglied der Jpaxo es bemerken und den Brief an die gewünschte Kultistin weiterleiten. Und sie kann uns über einen brav abgerichteten Krark Sternkraut zurücksenden.“

Zanyitaz hatte diese Erklärung schon mindestens zweimal gehört, doch nickte sie brav, ehe sie bedrückt anfügte: „Sofern diese Kultistin überhaupt Sternkraut besitzt. Und sofern der Brief noch rechtzeitig bei ihr landet. Sofern unser Krark überhaupt in die Nähe der Jpaxo kommt.“

Yafka deutete auf den Stapel von kleinen, fein zusammengerollten Pergamentrollen, die neben ihnen lag: „Und wegen all dieser Eventualitäten senden wir ja nicht bloß eine Nachricht fort, sondern so viele wir können. Los, Barz, er ist tief genug!“

Barz spürte, wie sein Herz laut gegen seine Brust pochte, als er den Pfeil aus dem Köcher zog. Er spannte seinen Bogen und richtete den Pfeil auf eine Stelle oberhalb des Avas, doch weit unterhalb des Krarks. Nicht, dass der Vogel sich bei seinem Sturz noch verletzte.

Er atmete tief durch und ließ los. Sirrend zischte der Pfeil von der Bogensehne und zog den Ködersack hinter sich her. Dieser Sack war in Stahlfischöl getränkt worden, bis er so richtig nach Fisch stank, und noch dazu mit Barz‘ glitzerndem Ablenkungspulver bestreut. Kein Krark konnte einem hell funkelnden Edelstein widerstehen, irgendetwas daran schien sie wie magisch anzuziehen. Sie sammelten solche Klunker gerne in ihren Nestern. Vielleicht, um Brutpartner zu beeindrucken? Und da die Iquar keine Edelsteinsammlung gehabt hatten, hatte Barz sich mit seinem glitzernden Ablenkungspulver aushelfen müssen. Ob Geruch oder Geblinke, irgendetwas würde den Krark hoffentlich auf den Ködersack aufmerksam machen.

Der geschossene Ködersack vollführte eine schöne Parabel von der Pfahlbausiedlung Thakkum weit über den großen See Ava. Barz, Yafka und Zanyitaz blickten dem schwächer werdenden Glitzern wie gebannt hinterher, während es seinen Flug vollführte. Bestimmt ging es einigen anderen Bewohnern auf der Pfahlbausiedlung ähnlich.

Da! Der rotbraune Fleck am Himmel hatte urplötzlich seine Flügel zusammengezogen und ließ sich fallen. Beinahe pfeilschnell stürzte der Krark in Richtung des Sees, in Richtung des immer langsamer fliegenden Köders. Dann, als der Köder den Höhepunkt seiner Flugbahn erreicht hatte, erreichten ihn die Klauen des Krarks, scharf wie eine Schwertschneide. Das ferne Glitzern wurde kaum merklich größer und färbte sich leuchtend orange, als der Krark den Köder streifte und der Köder mit einem Knall zerbarst. Sein Inhalt verteilte sich in seiner Umgebung. Dieses Schlafpulver, gewonnen aus dem beruhigenden Gift der Steppensporne, löste seine Wirkung zwar nicht so schnell aus wie das grüne Bannpulver aus Mondbeeren, doch dafür würde der Krark nicht mitten in der Luft stehen bleiben.

Tatsächlich konnte der Krark dem Schlafpulver nicht lange widerstehen. Er flatterte noch einige Male ungelenk mit seinen Flügeln und stürzte dann der Wasseroberfläche entgegen.

Platsch.

„Jetzt!“, rief Barz, doch das wäre nicht nötig gewesen. Zwei, drei, nein, gar vier Nixen hatten bereits den sachte strampelnden Krark ergriffen und schleppten ihn in Richtung Thakkum. Auf halbem Wege nahmen ihn zwei Mitglieder der Iquar auf einem Boot entgegen. Sie würden eine von Barz‘ Nachrichten an seinem Bein befestigen und darauf hoffen, dass der Krark sie nicht bemerkte und entfernte, ehe er in die Nähe der Jpaxo erreichte.

Dies war nur der erste Krark gewesen, doch dass es so fabelhaft funktioniert hatte, erfüllte Barz mit Hoffnung.

Seine Vorräte an Schlafpulver waren bereits ziemlich limitiert, doch für ein halbes Dutzend weiterer Krarks sollten die Reste noch reichen. Hoffentlich würden diese genau so leicht zu ködern sein.\bigskip







Sonnenaufgang. Zwei Tage vor dem großen Unheil.\bigskip



Im Laufe der letzten beiden Wochen hatten die Krahder hin und wieder weitere Skelette in den Ava gesandt, doch inzwischen waren die Nixen vorgewarnt und hatten anrückende Übeltäter aufhalten können, ehe sie eine der Nixen-Städte oder Thakkum erreicht hätten. Irgendetwas schien die Riesen davon abzuhalten, einen vollen Angriff auf den See und die Pfahlbausiedlung zu starten, der einen signifikanten Teil ihrer Truppen aufs Los setzen würde. Fürchteten sie sich vor etwas? Oder war einfach kein Kriegsherr oder geborener Taktiker unter ihnen anwesend? Wie Barz Tag um Tag Prinz Ferntahr in der Gegend umherspazieren und die Landschaft und Bewohner der Steppe untersuchen sah, oder wie Barz immer wieder die Hexe Nahrack vor ihrem Zelt beim Rühren in finster rauchenden Kesseln beobachtete, da hatte er das Gefühl, dass die beiden Krahder diesem Konflikt nicht annähernd dieselbe Signifikanz zuordneten wie die Bewohner des großen Sees.

Wenn Barz an die wenigen Augenzeugenberichte aus dem finsteren Land der Krahder und an ihren von Lavaflüssen und Feuerseen übersäten Boden dachte, war ihm klar, dass diese Situation auch für die Krahder Neuland war, hatten sie sich doch zuvor wohl kaum um durchquerbare Flüsse und Seen gekümmert, ganz zu schweigen von Belagerungen. Aber selbst ein unerfahrener, doch zielstrebiger Kriegsherr hätte inzwischen wohl bereits gehandelt und Thakkum von Untoten überrennen lassen. Es folgte der Schluss, dass Ferntahr und Nahrack keinen Wert darauf legten, diese Belagerung rasch hinter sich zu bringen. Vielleicht war eine Reise in ein schöneres Land wie dieses ja gar eine angenehme Erfahrung für sie. Falls sie überhaupt einen Sinn für Schönheit besaßen.

Es machte auch Sinn, dass die Krahder, welche sich ursprünglich durch einen Einfall der Yetohe in ihr Reich zu einem mächtigen Gegenangriff provoziert gesehen hatten, nach der Flucht der Yetohe keinen Druck zum raschen Handeln spürten. Dass sie nur zwei Vertreter ihrer Spezies so tief in den Norden gesandt hatten, zeugte ja gerade von ihrer Unbekümmertheit... oder ihrer Furcht, es könnte sich hinter Thakkum eine Falle verstecken?

Was es auch war, dass die Krahder von einem raschen Angriff abhielt, Barz schlief etwas ruhiger im Wissen, dass die Krahder, mochte ihre Armee auch viel gefährlicher sein für Thakkum als ursprünglich angenommen, nicht handeln können würden, ohne dass im ganzen See von den Nixen Alarm geschlagen würde.

An diesem Tag wurde Barz‘ ruhiger Schlaf von einem Kratzen an seinem Fenster gestört.

Barz wachte im Glauben auf, sein Zimmer wäre leer, denn dies war eine der Nächte gewesen, die Yafka in Zanyitaz‘ Raum verbrachte. Umso überraschter war Barz also, als er bemerkte, dass er nicht alleine war.

Statt der Sonne glotzte ein riesiger Vogelkopf durch sein Fenster herein. Ein Krark! Ein ganz friedlicher, seinem Halsband nach zu urteilen. Bei den Göttern, wie hatte dieser nur sein Haus aufspüren können? Er hatte vom Intellekt dieser Raubvögel gehört, doch diese Zielfindung war ihm erst recht alles andere als geheuer.

Rasch, noch im Nachthemd, hatte Barz sein Fenster geöffnet. Der Krark streckte ihm brav sein Bein entgegen und erlaubte ihm, ein daran befestigtes türkises Säcklein abzunehmen. Es schien überraschend stabil, trotz seiner Leichtigkeit, und enthielt ein kompliziert zusammengefaltetes Stück Papier. Barz steckte das Säcklein grinsend ein. Er hatte schon so eine Ahnung, dass er eines Tages eine Verwendung dafür finden würde. Ein Horder zu sein, war für einen Pulvermeister keine zwingend schlechte Eigenschaft.

Barz hatte natürlich schon einen Krark so nahe gesehen, aber noch nie einen wachen. Schaudernd musterte Barz dessen Schnabel und dessen unterarmlangen, gebogenen Krallen, die fast so scharf waren wie die Schneide eines Schwerts. Der Vogel beobachtete Barz aus viel zu wachen, aufmerksamen Augen. Dies war ein intelligentes Wesen, wurde Barz bewusst.

Wartete er auf eine Belohnung? Vorsichtig strich Barz dem Krark über den Kopf, immer von oben nach unten, um sich nicht an den scharfen Federn zu verletzen – das hatte er beim Umgang mit den bewusstlosen Krarks rasch und schmerzhaft gelernt.

Als der Krark sich auch nach der Streicheleinheit nicht gerührt hatte, rannte Barz in die Küche (und beinahe Karyz über den Haufen), griff sich eine Handvoll Seetang und kehrte damit in sein Zimmer zurück. Der Krark schnappte sich den Tang aus Barz‘ Handfläche – sein Schnabel kratzte sie nur leicht, doch Barz zuckte umgehend zurück – und würgte ein bisschen am glitschigen Mahl, ehe er sich abwandte und wieder davonflatterte. Barz blickte ihm immer noch etwas verdattert hinterher.

Erst jetzt wandte sich Barz der Botschaft zu, die der Krark hinterlassen hatte. Mit vor Aufregung zitternden Fingern entfaltete er das Papier. Einige flach gepresste Kräuterbüschel fielen heraus. Ein kurzer Spruch aus nur drei Wörtern stand auf dem Papier:

„Mögen die Götter mit euch sein.

– Sagramak“\bigskip







Sonnenhoch. Der Tag des großen Unheils.\bigskip



Barz trat an den großen Ritualkreis. Zwei Runenzeichner der Iquar hatten ihn in den letzten Tagen sorgfältig auf ein Tuch gemalt und dieses über eine Lücke zwischen den Häusern Thakkums gespannt. Nur so konnte die benötigte Fläche erzielt werden, derart große freie Gebiete gab es in der Pfahlbausiedlung sonst nicht, und das von Skeletten patrouillierte Seeufer bot sich kaum als Ritualstandort an.

Der große Lifornus streckte gebieterisch seine Hand aus und Barz überreichte ihm das Büschel gepressten Sternkrauts, das er von Sagramak, einer Schamanin und Drachenkultistin der Jpaxo, erhalten hatte. Vermutlich nicht ansatzweise so theatralisch, wie Lifornus es gewollt hatte. Aber das sollte schon in Ordnung sein.

Lifornus räusperte sich und breitete bedacht seine Arme aus. Er steckte einen kleinen Teil des Sternkrauts in seine Tasche und hob das restliche Bündel so hoch er konnte in die Höhe. Mit geschlossenen Augen zerrieb er es energisch. Leise murmelte er dazu in einer fremden Sprache. Leichter Rauch stieg von seinen Handflächen auf. Dann entflammten die Kräuter. Lifornus warf sie elegant in die Mitte des Ritualkreises. Das Tuch war von Barz‘ Schamanin mit einem speziellen Mittelchen präpariert worden, auf dass es nicht selbst Feuer fing. Barz bemerkte sie, wie sie angespannt auf das Tuch starrte und hoffte, dass es lange genug durchhielt.

Lifornus betrachtete den Ritualkreis mit dem brennenden Sternkraut darin aufmerksam und nickte dann.

Episch rief er: „Narbi gicein, prento varafera! Dirqo on te bini earim!“

Plötzlich war ein fernes Grollen zu hören, wie von einem Gewitter. Lifornus zuckte zusammen. Auch die restlichen Beobachter sahen sich unruhig an.

„Narbi gicein, prento varafera! Dirqo on te bini earim!“

Ein Windstoß fuhr durch das Steppengras und über die Wasseroberfläche. Das Ritualtuch flatterte.

„Narbi gicein, prento varafera! Dirqo on te bini earim!“

Der Himmel wurde immer düsterer, und Lifornus schrie jetzt beinahe.

„Narbi gicein, prento varafera! Dirqo on te bini earim!“

Sekunden nachdem das letzte Wort verklungen war, ertönte aus der Ferne ein tiefes Brüllen, das den Boden erzittern ließ. Im gleichen Moment war im Westen kurz eine helle Flamme am Himmel zu sehen.

Barz blickte durch sein Fernrohr und erspähte eine Rauchfahne vom westlichen Gebirge aus aufsteigen.

„Dort drüben, seht!“

Die Geräusche waren das erste, was die Beobachter beunruhigte. Ein Knarren und Knirschen, ein Knacken und Bersten, als würde das Gebirge selbst in sich zusammenstürzen. Die Rauchfahne nahm an Größe zu und eine weitere helle Flamme stieß in den Himmel. Dann erklang ein tiefes Röhren, das nicht vom Berg stammen konnte.

Ein Lebewesen.

Ein äußerst lautes und äußerst großes Lebewesen.

„Ist das etwa dein Phoenix?“, fragte Naquila erbleicht.

„Nein, nein, das kann nicht sein“, murmelte Lifornus, „Die Zauberin war sich sicher darin, dass auch der Phoenix in seinem nie endenden Wachstum eine maximale Körpergröße hatte. Spätestens bei dieser würde er in Flammen vergehen. Und diese Größe konnte laut ihr von einem Krark bei weitem übertroffen werden. Was auch immer das für ein riesiges Wesen ist, welches im westlichen Gebirge tobt, es ist nicht der Phoenix!“

„Was genau habt ihr für eine Zauberformel verwendet?“, fragte Naquila, „Könnte das hier ein weiterer ‚vartum‘-Versprecher sein?“

„Ich hieß die Kräfte der Zauberei an, mir ein ‚varafera‘, ein Feuertier, hierher zu leiten.“, erklärte Lifornus aufgebracht, „Wie viele Feuertiere außer den Phoenix gibt es wohl schon in dieser Umgebung?“

Obwohl das zuvor unmöglich erschienen war, erbleichte Naquila noch mehr.

„Du Narr! Weißt du etwa nicht, wer in diesem Gebirge lebt?! Weißt du etwa nicht, welche Spezies die Kultisten des dritten Barbaren-Stammes anbeten?!“

Lifornus schüttelte zitternd seinen Kopf.

Erneut erklang ein tiefes Brüllen aus dem Gebirge. Eine gewaltige schwarze Pranke schwang hinter einer Bergspitze hervor und knallte darauf herunter. Die Bergspitze bröckelte und gab den Blick frei auf die schreckliche Echse, welche sich dahinter in die Höhe stemmte. Ein riesiges Geschöpf. Monströse, messerscharfe Schuppen bedeckten den gigantischen Körper. Riesenhafte Flügel ragten aus dem Rupf und der Kopf saß auf einem schlangenartigen Hals. Blutrot und gemein blickten seine Augen. Und diese glutroten Augen dieses Feuerdämons waren direkt auf den Ava gerichtet.

Direkt auf Thakkum.

Barz‘ Herz hüpfte in seine Hose.

Ein Drache.






















\newpage
\section{Der Angriff des Drachen}


Sonnenhoch. Der Tag des großen Unheils.\bigskip



Die Drachenkultisten der Jpaxo mussten gerade außer sich sein vor Freude. Seit Jahrzehnten hatte man am Ava nichts mehr von Drachensichtungen gehört. Barz fragte sich unweigerlich, ob wohl Nabib gerade irgendwo weit im Westen gen Himmel blickte und den Ursprung dieser schaurigen Geräusche zu ergründen versuchte.

„Das... das ist nicht möglich!“, stammelte Lifornus, „Die Drachen sind doch schon seit Jahrhunderten ausgestorben! Dieser Krieg mit den Zwergen und Riesen... die Aufzeichnungen im Oktron besagten eindeutig...“

Barz sandte ein Stoßgebet an die Götter. Er hatte schon von Drachen gehört und Zeichnungen gesehen, aber es war noch einmal eine ganz andere Angelegenheit, einen mit eigenen Augen zu erblicken.

Der Drache kraxelte über einen weiteren Berghang auf den Ava zu, den Kopf weiterhin direkt auf die Pfahlbausiedlung Thakkum gerichtet. Barz beobachtete die Lage durch sein Fernrohr und erkannte Bergziegen, welche verzweifelt aus dem Pfad des Drachen hüpften. Als er seinen Blick auf den wolkenverhangenen Himmel warf, konnte er keinen einzigen Krark mehr darin erblicken. Kluge Vögel.

Nun hatte der Drache ein letztes großes Gebirgsplateau erreicht, welches steil in die Steppe abfiel. Langsam entfalteten sich zerknitterte Flügel, welche vielleicht seit Jahrhunderten nicht mehr in Gebrauch gewesen waren. Der Drache reckte und streckte sich, schüttelte seine Schultern, tat einen gewaltigen Satz und warf seinen bestimmt tonnenschweren Körper in die Tiefe.

Majestätisch rauschte die Riesenechse über die trockenen Steppengräser, hin und wieder mit ihren riesigen Flederflügeln schlagend und sich selbst wieder etwas in die Höhe katapultieren. Der Wind heulte um sie herum und Barz sah, wie sich das dürre Steppengras dem Sturmwind hinter den Drachenflügeln ergab, ja, zum Teil gar durch orkangleiche Böen der Erde entrissen wurden.

Dann richtete das Wesen seinen schlangenartigen Hals in die Höhe und riss seinen Mund weit auf. Barz hörte etwas gluckern und sprühen. Dann spie der Drache einen gewaltigen Feuerstrahl. Die trockenen Büschchen und Gräser der Steppe brauchten kaum mehr als das, und schon standen sie lichterloh in Brand.

Da erklang eine Stimme in Barz Kopf, grollend und laut, so laut, dass er verzweifelt seine Hände an seine Ohren presste, ohne dass ihm dies Linderung schenken würde:

„WER WAGTE ES, EINEN URTITANEN AUS SEINEM SCHLUMMER ZU REISSEN?!“

Barz blickte um sich und sah erheblich viele weitere Bewohner der Iquar auf ihren Knien, mit den Händen über ihren Ohren. Yafka kauerte neben ihm auf dem Holzboden, die Zähne zusammengebissen. Wie ging es Zanyitaz und Karyz? Ein Glück, dass die beiden im Haus zurückgeblieben waren. Konnte diese schreckliche geistige Stimme auch dort eindringen?

„WER WAGTE ES, DIE VERKÖRPERUNG DES ZORNS BEFEHLIGEN ZU WOLLEN?!“, brüllte der Drache.

Der große Lifornus setzte zu einer neuen Zauberformel an, doch seine auf einmal krächzende Stimme gehorchte ihm nicht mehr.

Der Drache hatte nun den Ava erreicht, schwebte einige Haushöhen über dem Wasser, während er unweigerlich auf Thakkum zuhielt und das Wasser sich unter seinen Flügelschlägen teilte. Dann rauschte er blitzschnell über die ersten Häuser und schon befand er sich über dem Ritualkreis.

„WER WAGTE ES... TAROK ZU RUFEN?!“, erklang die schaurige Stimme des Drachen ein drittes Mal. Mit mächtigen Schlägen seiner gewaltigen Schwingen bremste Tarok seinen Anflug, während unter ihm Menschen umgeworfen und Holzbretter umgeweht wurden. Der hohe Zeitturm mit der 12-Uhr-Flagge daran rumorte und wackelte bedrohlich. Barz sprang schützend über Yafka und blickte furchtsam in die Höhe. Taroks rote Augen glühten auf ihn herunter. Der lange Schlangenhals richtete sich zu seiner vollen Größe auf und erneut erklang dieses schreckliche Gluckern daraus, welches Feuer und Tod versprach.

Barz riss Yafka mit sich, weg von der Holzsiedlung, nur hinein ins kühle Seewasser. Er sprang. Augenblicklich wurde sein Mantel durchnässt und schwer und zog ihn in die Tiefe. Barz versuchte, nicht an all die wertvollen Pulver zu denken, die er in den vielen Taschen und angesteckten Säcklein aufbewahrt hatte. All die Stunden, die er mit Mörser und Häcksel, Schalen und Gläsern, kühlendem Schnee und wärmenden Heizsand in der Hütte seiner Schamanin verbracht hatte, um die magischen Bestandmittel in genau der richtigen Reihenfolge und Menge zu mischen... Die meisten waren nun sicherlich verdorben. Aber das war aktuell nicht wichtig.

Tarok spie einen weiteren gewaltigen Feuerstrahl, vermutlich heißer als alle Feuer, mit denen Barz bislang experimentiert hatte. Freilich konnte Barz diesen bloß verschwommen von unterhalb der Wasseroberfläche erkennen, denn sein Mantel zog ihn immer noch in die Tiefe. Verzweifelt versuchte er, seine Stiefel abzustreifen und seinen Mantel zu lösen, aber zuerst musste er seinen Köcher lösen und seine Lunge schrie schon jetzt wieder nach Luft. Sein Gesichtsfeld wurde enger.

Da packen ihn aus dem Nichts rettende Arme, die geschwind seinen Köcher lösten und den Mantel abstreiften. Während sein Blickfeld sich weiter verdunkelte, sah Barz verschwommen unter ihm seinen Mantel, wie er auf den Grund des Avas sank. Dann durchbrach Barz‘ Kopf wieder die Wasseroberfläche. Er japste nach Luft und blickte sich um, halb in der Erwartung, eine Nixe habe ihn gerettet. Stattdessen tätschelte ihm Yafka besorgt die Wange.

Chaos herrschte. Taroks Anflug und seine Feuerstrahlen hatten ein mächtiges Loch in die Pfahlbausiedlung gerissen. Der hohe Zeitturm war niedergestürzt und hatte eine Bresche in eine Häuserreihe gerissen. Links und rechts um Barz und Yafka schwammen dutzende von Holzbalken und -planken, zum großen Teil noch brennend. Weiter vorne verbrannte das eigentlich feuerfeste Tuch mit dem Ritualkreis drauf soeben zu grauer Asche.

Zahlreiche Iquar, obschon eigentlich exzellente Schwimmer, strampelten voller Panik im See nach Luft. Und die zahlreichen Yetohe, von denen nur die wenigsten zu schwimmen vermochten, waren keinesfalls besser dran. Hier und dort tauchten die ersten helfenden Nixen auf, doch von vielen anderen sah man nur noch sich rasch vom Chaos entfernende Fischschwänze.

Barz trat Wasser, griff ziellos nach Holzstücken und zog sie drehend durch das Seewasser, um das Drachenfeuer auf ihnen zu löschen, ehe er sie verzweifelt strampelnden Menschen zuschob. Neben ihm tat Yafka es ihm nach.

„Wir müssen hier weg“, erklang die Stimme des großen Lifornus, eine Oktave höher als sonst. Der Zauberer hatte es irgendwie geschafft, auf der Siedlung im Trockenen zu bleiben, auch wenn er seinen spitzen Zaubererhut verloren hatte. Er besänftigte soeben mit großen Gesten ein loderndes Feuer an einer Seitenwand des Versammlungssaals.

„Wohin?! Willst du etwa ans ‚sichere‘ Ufer schwimmen?“, antwortete Stammesleiterin Naquila von irgendwoher, auch ihre Stimme um einiges schriller als gewohnt.

Barz blickte sich verzweifelt um. War die immer noch am Ufer wartende Skelettarmee der Krahder einem die Siedlung angreifenden Drachen vorzuziehen? Sein Blick richtete sich nach oben. Der Drache Tarok schwebte noch immer über der Siedlung und wirbelte mit seinen Flügeln Holz und Wassermassen, ja gar einige Menschen umher. Doch war sein glühend roter Blick nicht mehr auf den Ritualkreis oder die Siedlung gerichtet. Stattdessen zeigte sein Kopf auf seinem Schlangenhals direkt auf das Zelt der Krahder am Ufer.

Gerade kam Prinz Ferntahr aus dem Zelt gestürmt. Um ihn herum versammelten sich einige seiner menschlichen und zwergischen Gehilfen mit der eigenartigen schwarzen Bänderkleidung. Barz konnte ihre weiß bemalten Gesichter nicht genau erkennen, erst recht nicht jetzt, wo sein Fernrohr ihm abhanden gekommen war. Vermutlich dümpelte es friedlich auf dem Grund des Avas.

Dem hektischen Umhereilen Prinz Ferntahrs und seiner Gefolgsleute nach waren die Krahder ebenfalls nicht adäquat auf einen angreifenden Drachen vorbereitet. Sie hatten ja nicht einmal Bögen bei sich. Irrte Barz sich, oder sah er dort drüben drei Unfreie die Gelegenheit beim Schopf packen und zu fliehen versuchen? Ob vor der Knechtschaft der Krahder oder vor Taroks Zorn, war schwer zu sagen. Prinz Ferntahr fackelte jedenfalls nicht lange, sondern zeigte auf die Flüchtigen und brüllte etwas. Eine Ambacu an seiner Seite senkte ihre Arme in den Boden, blaues Licht umspülte sie und eine riesige, unförmige Knochengestalt mit mehreren Schädeln rannte den Flüchtigen nach. Der Rest der Skelettarmee trottete weiterhin ungestört das Ufer entlang.

Ein lautes Rauschen oberhalb von Barz brachte ihn wieder in den Moment zurück. Taroks Flügel schlugen noch mächtiger als zuvor. Wie durch ein Wunder entfernte sich die Echse von der Pfahlbausiedlung und wandte sich stattdessen den Krahdern zu. Mit Lifornus‘ Hilfe kletterte Barz aus dem Ava zurück in die Siedlung und half Yafka, dasselbe zu tun. Stammesleiterin Naquila war nirgendwo mehr zu sehen, dafür umso mehr Schreie von anderen Barbaren zu hören, die unter Feuer, Wind oder Holzsplittern gelitten hatten oder immer noch litten.

Barz, Yafka und ein immer noch ziemlich verdattert den davonfliegenden Tarok anstarrenden Lifornus machten sich daran, erste Hilfe zu leisten, wo sie konnten. Tun konnten sie allerdings nicht viel, ehe die geistige Stimme Taroks erneut erscholl, diesmal ein wenig dumpfer, vermutlich distanzbedingt:

„IHR RIESEN DACHTET WOHL, IHR KÖNNTET EUCH AN MIR VORBEISCHLEICHEN? IHR DACHTET, DER OLLE TAROK VERSCHLIEFE EURE DREISTE INVASION? LASST MICH EUCH ZEIGEN, WIE SEHR IHR EUCH GEIRRT HABT. IHR LAUFT DURCH MEIN GEBIRGE. MEIN REICH. MEIN DRACHENLAND. UND KEIN VERDAMMTER RIESE WIRD ES BETRETEN UND ÜBERLEBEN, SOLANGE MEIN BLUT NOCH FLÜSSIG IST!“

Dann hatte Tarok das Zelt der Krahder erreicht. Er ließ einen mächtigen Feuerschwall darauf niederregnen und setzte es in Brand. Prinz Ferntahr hechtete heldenhaft zur Seite und plantschte in den Ava. Verzweifelt versuchte der Riese, seinen massigen Körper ganz unter Wasser zu befördern. Barz hätte aufgelacht, wenn die Lage nicht auch für ihn und seine Liebsten so prekär gewesen wäre.

Da erlosch das Drachenfeuer um das Krahderzelt in einem Schlag. Das Zelt fiel in sich zusammen und löste sich in farbige Fussel auf. Die gen Boden sinkenden Tücher enthüllten die Krahder-Hexe Nahrack, umgeben von grün leuchtenden Fäden, Stränge reiner Magie. Sie zielte mit zwei Fingern auf Tarok und kreischte irgendeinen Krahder-Fluch. Tarok zuckte zusammen. Unterhalb von ihm kraxelten Skelettkrieger übereinander und aufeinander, bildeten einen riesigen Haufen, versuchten, Taroks in der Luft schwebende Pfoten zu erreichen.

Erneut ließ ein tiefes, rhythmisches Geräusch alle Anwesenden zusammenzucken. Barz‘ Bauch kribbelte unpassend wohlig und seine Mundwinkel zuckten nach oben. Er brauchte einige Momente, um das Geräusch und diese Emotionen zu verstehen.

Tarok lachte.

Tarok lachte von ganzem Herzen.

Dann ließ der Drache sich vom Himmel fallen und pulverisierte bei seinem Aufprall mit schierer Masse den Haufen Skelettkrieger, der ihn zu erreichen versucht hatte. Ein weiteres Mal ließ Tarok Feuer auf die Überreste des Krahderzelts niederregnen. Barz konnte nur hoffen, dass egal, was für Ressourcen Tarok auch für sein Feuer benötigte, diese irgendwann bald aufgebraucht wären.

Die Dunkle Hexe Nahrack kreischte auf, als das Drachenfeuer sie erreichte, und verging in einem Flammenwirbel. Tarok klappte seinen Mund zu. In der grauen Asche, die einst Nahrack gewesen war, regte sich etwas. Ein bläuliches Glühen flammte auf, als sich eine blasse Gestalt aus den schwelenden Überresten erhob.

„OH NEIN, HEUTE NICHT!“, brüllte Tarok und trampelte mit seinen Pranken auf dem Boden herum, bis kein bläuliches Schimmern und keine blasse Gestalt mehr zu sehen waren. Dann wandte sich sein Schlangenhals dem Prinzen zu. Ferntahr war inzwischen weiter in den Ava hineingewatet und senkte gerade in dem Augenblick seinen kleinen Kopf unter die Wasseroberfläche, als Taroks nächster Flammenstrahl ihn erreichte.

Das Wasser um ihn herum begann zu kochen.

Ehe Ferntahr wieder zum Atmen auftauchen musste, hatten die an Tarok hochkletternden Skelettkrieger seine Flügel erreicht und begannen, mit ihren Schwerten und Äxten hineinzustechen. Leider schien sie das Ableben der Dunklen Hexe nicht in ihrer Aktivität einzuschränken.

Tarok brüllte auf, diesmal physisch, nicht geistig, und erhob sich wieder in die Lüfte. Er schien widerspenstig. Dass diese kleinen Wesen ihm tatsächlich schaden konnten, musste an seinem Selbstbild rütteln. Tarok drehte eine Pirouette, die auch die letzten Skelette von ihm abschüttelte und zum Teil bis weit in die Barbarensteppe oder den Ava schleuderte. Er legte seine Flügel an und schoss das Ufer des Avas entlang. Wie in einem wilden Wahn wütete Tarok durch die Reihen der Skelettkrieger, biss und kratzte, zermalmte und schleuderte Überreste weit, weit weg.

Prinz Ferntahr tauchte gerade rechtzeitig wieder aus dem Ava auf, um zu sehen, wie drei weitere seiner Ambacus das Weite suchten. Diesmal war niemand an seiner Seite, denen er befehlen konnte, sie aufzuhalten.

Barz wandte seine Aufmerksamkeit von diesen fesselnden Geschehnissen ab und wieder seiner Umgebung zu. Inzwischen hatten einige Heiler und Schamanen das Loch erreicht, welches Tarok in die Pfahlbausiedlung Thakkum gerissen hatte. Die meisten im Wasser strampelnden Barbaren waren ins Trockene gebracht worden, die meisten Brandherde eingedämmt und die wenigen Skelettkrieger, deren Überreste in die Nähe der Siedlung geschleudert worden waren, wurden rasch beseitig. Doch noch immer befand sich eine zornige Riesenechse gefährlich nahe. Gerade war Tarok damit beschäftigt, die letzten ordentlichen Reihen der Skelettkrieger um den Ava zu Knochenstaub zu zerschlagen, doch wenn er sich danach entscheiden sollte, dass Thakkum wie eine ideale Lagerfeuerstätte aussah, würde dies das Ende von dessen Bewohnern sein. Was tun, was tun?

Im Geiste ging Barz die wenigen Mittelchen durch, von denen er immer noch Vorräte in seinem Haus hatte.

Das grüne Bannpulver aus getrockneten Mondbeeren war sehr nützlich gegen kleinere Gegner, doch gegen jemanden dieser Größe und Stärke konnte es kaum etwas anrichten. Wenn Barz mehr als nur eine Tatze Taroks bannen wollte, bräuchte er Unmengen dieses Materials. Mehr, als es aktuell wohl auf der gesamten Welt gab.

Fürs orangefarbene Schlafpulver aus dem Schlafgift der Steppensporne galt das genau gleiche Problem. Zu großes Ziel, zu kleine Wirkung. Ganz zu schweigen davon, dass Barz seine letzten Vorräte davon vor zwei Wochen beim Anlocken der Krarks verbraucht hatte.

Das braune Meditationspulver aus Krarkfedernasche könnte Barz sicherlich beruhigen, ihm vielleicht gar etwas Zeit zum weiteren Überlegen schenken, doch mehr nicht.

Das graue Nebelpulver aus dem fernen Narkon, welches... nein, Barz besaß doch gar kein... Barz blieb einige Sekunden an einem Gedanken hängen, ehe dieser sich verflüchtigte. Worüber hatte er gerade nachgedacht? Egal, wichtiger als ein angreifender Drache konnte es kaum sein.

Das goldene Umwandlungspulver war Barz neuste Kreation, mit welcher er bereits ziemlich routiniert gewisse Alltagsgegenstände beschwören konnte, so etwa die Fernrohre, die in den letzten Tagen die Krahder ausspioniert hatte. Aber nur gewisse Gegenstände. Und nur manchmal. Noch war Barz sich nicht sicher, ob er diese Gegenstände tatsächlich erschuf, oder ob er sie nur aus z.B. einem nahe gelegenen Ausrüstungslager der Yetohe zu sich teleportierte. Er hatte bereits einige Briefe in Fernrohre umgewandelt, um zu gucken, ob plötzlich Falken mit Antworten aufkreuzen würden, doch noch war dies nicht der Fall gewesen. Aktuell hatte Barz noch nicht einmal herausgefunden, dass er mit dem Umwandlungspulver auch lebendige Falken ins Spiel bringen konnte. Faszinierendes Zeug, dieses Umwandlungspulver, aber dieser Drache brauchte mehr als ein paar frische Bögen, um erlegt zu werden.

Das silbern glitzernde Ablenkungspulver war vor zwei Wochen auch ganz verbraucht worden beim Anlocken der Krarks. Ganz zu schweigen davon, dass sich auch Tarok kaum davon ablenken lassen würde.

Dann war da nur noch das bläuliche Schwächungspulver, welches Barz erst kürzlich aus grünem Cantharis-Giftpulver und tiefblauen Cirathans-Körnern zu einem einzigen Pulver statt einer Zwei-Komponenten-Verbindung hatte optimieren können. Doch auch dieses würde dem Drachen wohl kaum mehr als eine Schuppe wegätzen können, und das kleine bisschen Heilkraft, welches es einem Verwundeten verleihen könnten, würde wohl kaum einen signifikanten Unterschied mehr machen.

Barz schüttelte resigniert seinen Kopf. Diese Überlegungen brachten ihm nichts. Gegen Tarok konnte er nichts anrichten. Dafür brauchte es größere Konstruktionen, so etwas wie riesige Katapulte. Oder mächtigere Magie. Doch der einzige Magier hier, der die Kraft oder Kenntnisse dafür besitzen konnte, machte soeben durch hektisches Herumgefuchtel mit seinem Zauberstab deutlich klar, dass er nichts ausrichten konnte. Barz ließ sich auf die knarzenden Planken sinken und betete zu den Göttern. Doch nur kurz, denn dann riss Yafka ihn wieder in die Höhe und deutete auf zahlreiche weitere Hilfesuchende in ihrer Umgebung. Barz machte sich auf, Heilkräuter aus der Hütte seiner Schamanin zu holen. Um Tarok würden sie sich sorgen, wenn es an der Zeit war.

„IHR MACHT ES MIR SCHON FAST ZU EINFACH!“, erscholl Taroks dumpfe Stimme ein weiteres Mal in seinem Kopf. Diesmal hatte Barz sich schon beinahe daran gewöhnt und stolperte nur kurz, ehe er weiterrannte. Er hätte schwören können, dass die Stimme einen amüsierten Unterton gehabt hatte. Da er aktuell zwischen zwei Hausreihen durchpreschte, war sein Blick darauf versperrt, was mit dem Drachen am Ufer abging. Barz lieferte die drei Büschel Heilkräuter, die er in seinen Hosenbund gestopft hatte, der nächsten Heilerin ab, rannte zwischen zwei Häuser und versuchte, durch die Lücke zu erspähen, was Tarok derart amüsierte.

Da, ganz in der Ferne, glaubte Barz einen weiteren Feuerstrahl zu erkennen. Er kniff seine Augen zusammen und kapierte plötzlich. Die Armee der Skelettkrieger war noch nicht vollständig an den Ava angetreten. Noch immer wälzte sich ein Wurm von Skelettkriegern aus dem Land der Krahder durch die Steppe der Barbaren an den Ava. Wie eine lange dünne Linie zog sich die feindliche Armee durch die Steppe und durch die südlichen Berge... vermutlich bis in ihre finstere Heimat. Ein Anblick, der Barz bereits seit einiger Zeit Albträume bereitet hatte – hatte diese langsam anrückende Armee denn wahrlich kein Ende? Doch nun musste er grinsen. Der Wurm von Skeletten konnte auch eine Lunte sein. Nun flog Tarok rasch und gezielt die Reihen der anrückenden Skelette an und setzte sie eines nach dem anderen in Brand. Und er lachte. Oh, wie er lachte. Das alles musste Erinnerungen zurückbringen an Jahrhunderte lang zurückliegende Kriege, in denen Tarok Seite an Seite mit anderen Drachen gekämpft und gesiegt hatte. Der Drache schwelgte kurz unbefreit in blutiger Nostalgie, ehe ihn seine Trauer wieder einholte.

Barz verdrückte eine Träne.

Dann brach der Strom fremder Emotionen abrupt ab und Barz besann sich wieder auf sich selbst. Was für einen außergewöhnlichen Geist diese Drachen doch hatten, dass sie andere an ihren Gedanken und Gefühlen teilhaben lassen konnten, ob diese es wollten oder nicht. Oder ob sie es selbst wollten oder nicht.

Nun, im Kontext von Taroks Erinnerungsfetzen, sah Barz ein, dass sie großes Glück im Unheil gehabt hatten, welches Lifornus‘ Ritual ausgelöst hatte. Wenn Barz die verwirrenden Eindrücke des Drachen aus der Zeit des unterirdischen Krieges richtig einschätzen konnte, so waren die Drachen und Krahder zutiefst verfeindet. Kaum etwas anderes als eine Krahderarmee hätte Tarok von Thakkum ablenken können. Und kaum etwas anderes als ein Drache hätte die Krahder zur Flucht zwingen können. Das Schicksal konnte einen miesen Sinn für Humor haben.

Weiter vorne sah Barz den Krahderprinzen Ferntahr vor mit Speeren und Dreizacken bewaffneten Nixen aus dem Wasser fliehen und sich fassungslos umsehen. Hier und da krabbelten und hüpften noch einige seiner Skelettkrieger mit fehlenden Gliedmaßen umher, dort drüben standen gar noch zwei zitternde Ambacus und warteten auf Ferntahrs Befehle. Doch was einst rund um den Ava herum eine schier unüberwindbare Armee aus wandelnden Leichnamen und Knochen-Golems gewesen war, war nun größtenteils bloß noch ein desorientierter Haufen aus Knochensplittern und teils verrosteten, teils geschmolzenen Waffen. Und die Linie des Skelett-Nachschubs, die sich vom Ava weit in den Süden zog, wurde soeben von einem mächtigen Drachen abgeflogen, der keinen einzigen Grashalm im Schatten seiner Flügel stehen ließ, und erst recht keinen Skelettkrieger.

Ferntahr brüllte panisch auf und suchte das Weite in der weiten Barbarensteppe.

Barz jubelte innerlich auf.\bigskip







Kurz nach Sonnenhoch. Der Tag des großen Unheils.\bigskip



Barz‘ Boot fuhr hart aufs Seeufer auf, doch Barz kümmerte sich nicht darum. Er schwang sich über die Bordwand, zog einen Pfeil mit einer rosa Spitze aus seinem Köcher und versenkte ihn im Schädel eines Skelettkriegers, welcher nach ihm zu grabschen versuchte. Neben ihm stach Yafka energisch mit ihrem Schwert auf einige sich schwach regende Beinknochen ein.

Eine Hitzewelle rauschte an Barz vorbei. Der große Lifornus trat an Barz‘ Seite, drei weitere Feuerkugeln seinen Zauberstab umkreisend.

„Na, das hat ja wie gewünscht geklappt, nicht?“, rief Barz Lifornus zu. Dieser unterdrückte ein Schluchzen, riss sich dann zusammen und sprach: „Rumuno kerkum!“

Zwei weitere Skelettkrieger ohne Arme und ohne Waffen wurden in ein nahestehendes Drachenfeuer geschubst und fielen in sich zusammen.

Hinter Barz, Yafka und Lifornus kletterten weitere Kämpfer aus Booten ans Seeufer und machten sich ebenfalls daran, die wenigen „überlebenden“ Überreste von Skelettkriegern gekonnt zu zerteilen. Wenn sie weiterhin solche Fortschritte machen könnten, würde der Ava noch vor Sonnenuntergang wieder völlig von Feinden befreit sein. Doch mussten sie schnell handeln, ehe die Überbleibsel der Skelettkrieger sich zu einer Horde zusammenrotten und einen gezielten Gegenangriff starten konnten. Falls sie nach Taroks Angriff und dem Tod ihrer Dunklen Hexe überhaupt noch dazu in der Lage waren.

Der Drache Tarok war inzwischen hinter dem südlichen Gebirge verschwunden und hatte eine Spur aus brennendem Steppengras und eingeäscherten Skeletten hinterlassen. Barz blickte immer wieder sorgenvoll in diese Richtung. Es war immer noch möglich, dass Tarok plötzlich genug davon hatte, die Armee seiner uralten Feinde zu zertrümmern, und sich wieder denen zuwenden wollte, die es gewagt hatten, ihn zu rufen. Dann müsste Thakkum so schnell wie möglich evakuiert werden. Es schauderte ihn beim Gedanken an seine Familie und Freunde. Zumindest waren Zanyitaz und Karyz aktuell in Sicherheit. Ehe Yafka sich Barz beim Landgang angeschlossen hatte, hatte sie ihm mitgeteilt, dass Zanyitaz sich einige andere Fischer, Karyz und einige weitere Kinder geschnappt hatte und mit diesen auf einem Fischerboot in den See gestochen waren. Dort draußen glaubten sie sich aktuell sicherer als in den brennbaren Pfahlbauten Thakkums. Und Barz stimmte ihnen zu.

Zum ersten Mal seit dem Anfang von Taroks Angriff erlaubte sich Barz einen Moment des Verschnaufens. Sein Blick driftete locker über seine Umgebung. Viele Skelettkrieger waren nicht mehr übrig. Doch dann erschrak er, als ihm bewusst wurde, dass die größte Gefahr nicht mehr von der Armee der Krahder ausging.

Die Steppe brannte lichterloh.

Kleine Flicken des trockenen Grases, welches Tarok in Brand gesteckt hatte, waren zu großen Flammenwirbeln geworden, deren Rauchsäulen weit in den Himmel reichten.

Sowohl als Bewohner einer Holzsiedlung als auch als Bereisender einer teils trockenen Steppe waren Barz die Gefahren des Feuers oft genug eingebläut worden, damit er großen Respekt vor ihnen hatte. Es gab einen Grund, warum sich die Schamanen der meisten Yetohe-Sippen aus Prinzip weigerten, mit Feuergeistern in Kontakt zu treten.

Und nun brannte die Steppe lichterloh, und es gab nicht wirklich etwas, das sie tun könnten. Vermutlich hätten zu diesem Zeitpunkt nicht einmal mehr die legendären Löschzwerge aus dem Westen groß etwas ausrichten können.

Im ersten Augenblick dachte Barz nur erleichtert daran, dass die meisten Steppenbewohner zu ihrer verzweifelten Invasion Andors aufgebrochen waren. Dort hatten sie wenigstens die Chance, von ihren Gegnern verschont zu werden. Das Feuer war hingegen gnadenlos.

Im zweiten Augenblick war Barz erleichtert darüber, dass die zurückgebliebenen Yetohe mitten im Ava sicher sein sollten vor der Feuerhölle, welche dort draußen herrschte.

Im dritten Augenblick freute sich Barz gar darüber, dass Taroks Rückkehr umso unwahrscheinlicher war, wenn die halbe Steppe seinetwegen brannte. Was wollte er noch schlimmer machen?

Erst im vierten Augenblick erinnerte sich Barz an die weiteren Bewohner der Steppe. Die ganzen nichtmenschlichen Tiere, welche aktuell vor dem Flammenmeer wegzurennen versuchen mussten. Die Büffel und die Hornbären. Die Einhörner und die Steppensporne. Die Flederkatzen und allgemein alle anderen Totemtiere.

Und dann gab es auch noch diejenigen, die sicherlich nicht schnell genug waren, um dem Feuer davonzurennen.

Die Steppenechsen.

„Jirisa!“

Noch während Yafka ihren Blick auf Barz richtete und zu verstehen versuchte, was er mit seinem Ausruf gemeint haben könnte, rannte Barz los. Ehe die Skelettkrieger der Krahder den Ava erreicht hatten, waren die wenigen Steppenechsen der Yetohe, die nicht als Reit-Echsen beim Angriff auf Andor zum Einsatz gekommen waren, feierlich freigelassen worden. Doch bewegten sich Steppenechsen nur selten, wenn man sie nicht dazu drängte. Und die Skelettkrieger der Krahder hatten die Echsen in Ruhe gelassen, das hatte Barz bei ihrer Ankunft überrascht beobachtet. Somit stand es zu vermuten, dass einige Steppenechsen sich immer noch in der Nähe ihrer einstigen Ställe, Felder und Nester aufhielten, wenige hundert Meter von hier entfernt. Und mit ihnen auch Barz‘ Steppenechse Jirisa.

Ein althadrischer Spruch ertönte hinter Barz. Vor ihm teilte sich das Drachenfeuer kurzzeitig. Er rief Lifornus ein Danke zu, hechtete durch die Feuerlücke und erreichte die ersten Stallungen der Steppenechsen. Diese standen bereits in Flammen. Der Haupteingang war gar in sich zusammengestürzt. Hitze und Rauchen schlug Barz entgegen. Schweiß rann seine Stirn hinunter und die feurige Luft kratzte in seinem Hals. Doch davon ließ er sich nicht aufhalten.

Zwei mächtige Steppenechsen standen in den Stallungen und röhrten panisch. Barz öffnete den Türverschluss der Hintertür und winkte sie ins Freie. Hinter sich vernahm er erneut Lifornus‘ tiefe Stimme einen Zauberspruch sprechen. Ein kleiner Spalt öffnete sich im Drachenfeuer, durch den gerade noch so das Blau des rettenden großen Sees Ava erblickt werden konnte. Barz klatschte den an ihm vorbeieilenden Steppenechsen an die Seiten, in der Hoffnung, sie noch etwas anzutreiben. Keine von ihnen war Jirisa.

Barz eilte weiter ins von Brandherden übersäte Feld, auf denen die Steppenechsen üblicherweise rasteten und grasten. Anstelle des Himmels sah er bloß noch schwarze Rauchschwaden über sich. Dort drüben lag der kokelnde schwarze Kadaver einer Steppenechse, an dessen Flanke immer noch Taroks Flammen leckten. Hier vorne sah er zwei weitere massige Echsen davonrennen, ebenfalls in Richtung des Seeufers. Zumindest hoffte Barz das. Es wurde immer schwerer, sich zu orientieren. Er spielte mit dem Gedanken, ebenfalls ans Ufer zurückzukehren. Hier gab es nicht viel, was er tun konnte, außer sein Leben aufs Spiel zu setzen.

Doch zumindest an einer letzten Stelle wollte Barz noch nach Jirisa oder weiteren Steppenechsen sehen, ehe er guten Herzens umkehren konnte. Die Nester der Steppenechsen lagen in der Nähe der Ställe. Eine kleine Senke, in der die Echsen unter guten Umständen getrocknetes Gras zusammenscharren, ihre Eier legen und diese wärmen konnten, bis ihr Nachwuchs schlüpfte. Barz musste schwer schlucken, als ihm bewusst wurde, wie brennbar diese Nester waren. Und wie die treuen Steppenechsen-Mütter ihren zukünftigen Nachwuchs dennoch kaum aufgeben würden.

Barz hechtete über eine Anhöhe und wischte sich das schweißnasse Gesicht ab. Er hustete und hielt seinen Kopf gesenkt, um dem hitzigen Rauch zu entgehen, doch das half kaum, wenn überall um ihn herum Brandquellen in den Himmel rauchten. Er blinzelte und versuchte, die Nester der Steppenechsen zu erspähen. Glücklicherweise waren die meisten leer. Doch wie befürchtet erkannte Barz am unteren Ende des Abhangs eine schwarz verfärbte Steppenechse, welche auf einem brennenden Flecken Steppengrases zusammengebrochen war. Noch war sie am Leben, und noch streckte sie ihren Kopf verzweifelt zu einem lichterloh brennenden Strohnests, in deren Mitte Barz drei schimmernde große Eier erblickte.

Jirisa hatte schon seit einigen Jahren keine Eier mehr gelegt, versuchte Barz sich einzureden, wie groß war da schon die Wahrscheinlichkeit, dass sie gleich bei ihrer Rückkehr an den See welche gelegt hätte? Doch er glaubte sich selbst nicht. Barz wusste, dass bei viel umherreisenden Steppenechsen das Legen von Eiern vertagt werden konnte, bis die Echsen eine Zeit lang am selben Ort verbrachten. Und dass sie nicht zwingend das beste Gespür dafür hatten, ob die Zeit vor der Winterstarre fürs Schlüpfen ihres Nachwuchses reichen würde.

Barz schlidderte den Abhang herunter und sein Herz sank erst recht in seine Hose, als die Steppenechse ihn erblickte, laut röhrte und ihr Gesicht ihm zuwandte. Diese Stimme kam ihm derart bekannt vor und diesen schiefstehenden Fangzahn in ihrem Unterkiefer hätte er im Schlaf erkannt. Das war Jirisa.

Jirisa röhrte ein weiteres Mal auf und stemmte ihren massigen Körper mit zitternden Beinen in die Höhe. Sie schien Barz nun auch erkannt zu haben. Drei Schritte weit kam sie auf ihn zu, ehe sie wieder zu Boden plumpste und schwach brummte. Die Flammen des brennenden Nests leckten weiterhin an ihrem Schwanz, doch zumindest war der Rest von ihr nun frei. Barz stürzte auf sie zu und streichelte ihre raue, viel zu warme Stirn.

„Du Dussel!“, fluchte er, „Du treuherziger Dussel, hast du etwa versucht, die Eier aus dem Feuer zu retten?“

Ein dumpfes Tröten drang aus Jirisas Kehle. Es klang beruhigend. Barz konnte nicht sagen, ob sie sich selbst oder ihn zu beruhigen versuchte. Bildete er sich ein, oder wirkte die Echse entspannter? Barz hatte ihr schon so oft geholfen, als sie sich einen Dorn in den Fuß getreten hatte, als sie in einem Sumpf stecken geblieben war, als sie sich im Netz einer Steppensporne verfangen hatte. Glaubte sie, dass Barz die Lage noch zum Guten wenden konnte? Verband sie Barz mit Gefühlen der Sicherheit und Geborgenheit? Gefühlen, denen er nun einfach nicht gerecht werden konnte?

Barz blickte an Jirisa herunter. Erschöpft und verbrannt, wie die Echse war, würde sie kaum mehr laufen können, und Barz konnte sie nicht transportieren. Er stand auf und versuchte, das Feuer hinter der Echse auszustampfen. Nicht einmal das gelang ihm vollständig. Er tastete seinen Mantel nach Heilmitteln ab, ehe ihm einfiel, dass dieser weiterhin versunken auf dem Grund des Avas lag. Er griff in seinen Gürtel nach Heilkräutern, doch diese hatte er einer Heilerin abgegeben, ehe er ans Ufer geschifft war.

Barz blickte verzweifelt zum Nest mit Jirisas drei weiterhin brennenden Eiern. Diese mussten inzwischen völlig gebraten sein. Auch dort konnte Barz nicht mehr helfen. Er verfluchte das Feuer, er verfluchte Tarok und er verfluchte den großen Lifornus, der Tarok gerufen hatte. Dann fiel sein Blick auf ein viertes, einzelnes Ei, welches ein wenig abseits des brennenden Nests lag. An einer Steppengras- und damit auch feuerfreien Stelle. Ein einzelnes dunkles Ei, in dessen schimmernden Schale sich das Flammeninferno seiner Umgebung spiegelte. Jirisa musste es rechtzeitig aus dem Nest bugsiert haben. Es war unversehrt.

Der Abstand zum Ei war in Windeseile überwunden. Es war klein für ein Steppenechsen-Ei, und doch größer als Barz Hand. Vorsichtig hob er es in die Höhe und sah davon sein eigenes verzerrtes Gesicht zurückspiegeln, ein dunkler Schatten vor rotem Hintergrund. Er hastete zurück zu Jirisa, kniete neben seiner langjährigen Begleiterin auf den heißen Grund und streichelte sie ein letztes Mal, während er ihr versichere:

„Oh Jirisa, es tut mir so leid, so leid, dass ich zu spät gekommen bin. Ich... ich kann nichts mehr für dich tun, alte Freundin. Oh, Jirisa. Tapfer hast du uns lange Zeit begleitet, und tapfer hast du versucht, deine Nachkommen vor diesem Sturm zu bewahren, der auch meine Schuld ist. Und zumindest eines davon wird dank dir fortbestehen können. Dein letztes Ei werde ich mit meinem Leben hüten, das verspreche ich dir. Aus diesem Ei wird eine stattliche Echse wachsen, und ich werde sie hegen und pflegen, so es in meiner Macht stehe. Ich werde sie... Sabri nennen. Kannst du sie dir vorstellen? Wie sie über die weiten Weiden der Steppe ziehen wird, stoisch, störrisch und stark? Das wird sie von dir haben. Oh, Jirisa, erinnere dich nicht an diese Flammen und die Hitze. Denke zurück an die guten Zeiten, an saftiges Steppengras, fliegende Fische, an das erquickende Wasser des Avas. So stelle ich mir das gute Leben nach dem Tode vor, und so eines hast du definitiv verdient. Bald wirst du dort sein und mit den anderen Steppenechsen frei herumtraben können. Zuvor musst du noch durch dieses Leiden durchstehen, doch danach... “

Ein Hustenanfall unterbrach Barz‘ Monolog. Er wischte sich die von der Hitze schweißnasse Stirn ab. Seine Hand zitterte und sah unschön rot aus. Doch noch ging Jirisas rasselnder Atem, und so kuschelte sich Barz erneut an seine Steppenechse und fuhr fort, redete von der unendlichen Steppe und den farbigen Flüssen im Paradies der Götter, auch wenn er im Hinterkopf ahnte, dass er damit vermutlich mehr sich selbst beruhigte als die Echse. Auch wenn er im Hinterkopf wusste, dass im Moment um ihn herum noch einiges Schlimmeres geschah, mit dem er sich später befassen musste..

Doch nicht jetzt.

Dieser Moment galt ihm und Jirisa.

Ein letztes Mal leckte ihre raue Zunge Barz‘ wunde Hände.

Ein letztes Mal atmete sie rasselnd aus.

Dann war es vorbei.

Tränen stiegen in Barz auf, ob vom beißenden Rauch oder von Trauer, konnte er nicht genau sagen. Er umhüllte Sabris Ei sorgfältig in ein Stück Stoff, kam endlich auf die Idee, ein anderes über seinen Mund zu stülpen, und rannte aus dem Flammeninferno fort, weg von Jirisas Leichnam, weg von allem Unheil. Das Unheil, das sein Sternkraut zu verantworten hatte.

Später würde Barz nicht mehr genau wissen, wie er es zurück ins rettende Wasser des Avas geschafft hatte. Irgendwann war Yafka plötzlich da und zog ihn mit sich, dann sanft die Böschung hinab. Seine wunden Hände wurden im Seewasser gekühlt. Von irgendwoher wurde ein bitteres Heilkraut in seinen Mund gelegt. Dann war da auf einmal ein Boot, welches ihn zurück zur Pfahlbausiedlung brachte. Yafka saß an seiner Seite, streichelte seinen Rücken und seinen von Feuersalven lädierten Bart. Sie schalt ihn nicht, doch Barz spürte ihre Sorge und dass sie alles andere als zufrieden war, dass er blindlings ins Feuer gerannt war. Heldenhaft, aber auch unvorsichtig. Dass er sich für Yafka gleich verhalten würde, musste er nicht sagen, das war ihr klar. Schien auch nicht zwingend beruhigend zu sein für sie. Barz konnte das auch nachvollziehen, war er auch selbst etwas erschrocken über seinen Wagemut. Und doch konnte oder wollte er das Gefühl nicht abschütteln, dass er richtig gehandelt hatte. Ohne ihn wäre Sabris Ei im Steppenfeuer verbrannt. Und ohne Jirisa auch. Das gab ihm Halt. Das gab ihm das Gefühl, dass Jirisas Tod nicht völlig unnötig gewesen war. Auch wenn er das gewesen war.

So hielt Barz während der ganzen Heimfahrt Sabris Ei fest umklammert.\bigskip







Sonnentief. Einen Tag nach dem großen Unheil.\bigskip



Einen kurzen Moment des Schreckens gab es noch, als Tarok bei seinem Rückflug aus Krahd erspäht wurde. Doch offenbar hatte er gerade für seinen Geschmack genügend Chaos angerichtet. Späher der Iquar berichteten, dass Tarok sich in seine Höhle ins Graue Gebirge zurückzog. Vermutlich würde das riesige Wesen nun wieder für Jahre oder Jahrzehnte in einen tiefen Schlaf zurücksinken, bis jemand so töricht war, es zu wecken.

Der große Lifornus lachte verzweifelt über die Nachfrage, ob er als Feuerzauberer denn nicht etwas gegen die wütenden Flammen draußen in der Steppe tun könnte. Selbst die besten Wassermagier aus dem fernen Danwar aus hätten keine Chance, dieses Feuer zu löschen, spottete er.

Die Schamanen beteten zu den Göttern, und zu den Wassergeistern des großen Sees, und zu den Wassergeistern aller kleinen Seen, die über die Steppe verteilt sein mögen, ja, sie beteten selbst zu den Drachen aus uralter Zeit. Und als ob ihre Gebete erhört worden wären, öffneten sich tatsächlich bald darauf die Schleusen des Himmels. Drei Tage lang strömte Wasser aus dunklen Wolken und kämpfte gegen die letzten Reste des Drachenfeuers draußen in der Steppe, doch auch dieses konnte nicht ewig durchhalten.

Der einst staubtrockene Boden wurde zu Schlamm aufgeweicht. Graue Asche wurde in die Flüsse und in den Ava gespült, womit die Nixen, Seerösser und Spornwale für einige Zeit zu kämpfen hatten.

Steppenechsen und weitere Bewohner der Steppe wurden gesucht und gepflegt. Barz stellte beim Nest der Steppenechsen einen Todesstein auf in Jirisas Namen und konzentrierte sich dann mit doppeltem Elan darauf, Sabris Ei zu hegen und zu pflegen.

Lifornus wurde für sein misslungenes Ritual vor ein Gericht gestellt, doch in Anbetracht seiner Reue, seines guten Willens und seines tatenkräftigen Einsatzes bei der Zerstörung der letzten Skelettkrieger ließen die Iquar von einer Strafe ab.

Lifornus hatte hoffentlich aus seinem Missgeschick gelernt, doch vom Abhalten von Ritualen konnte ihn ihn nicht abbringen: Eines Morgens wanderte er geheimnisvoll tuend mit dem letzten Rest des getrockneten Sternkrauts, einem Tuch und einem Kohlestift in die verbrannte Steppe hinaus, nur um am nächsten Abend mit einem brennenden Vogelküken auf seiner Schulter zurückzukommen. Quasi als Beweis, dass sein Ritual doch wie gewünscht funktionieren und einen Phoenix herbeirufen konnte, wenn er nur die richtigen Worte wählte.

Barz sah den kleinen Phoenix gerade lange genug, um sich von dessen Existenz zu überzeugen (und um eine bissige Bemerkung über Lifornus‘ Wagemut runterzuschlucken), als das Vögelchen auch schon wieder mit einem leisen Plopp in einem Feuerbausch verschwand und nichts als ein kleines Häufchen Asche auf Lifornus‘ edlem Gewand hinterließ. Irrte Barz sich, oder hatte er dabei leise Flötentöne vernommen? Er sammelte das Häufchen Phoenix-Asche sorgfältig ein und riet Lifornus, lieber keinem der höheren Stammesmitglieder davon zu berichten, dass er sich erneut an so einem Ritual versucht hatte, erst recht nun, da die Iquar keinen Bedarf mehr daran hatten. Denn solange die Riesen aus dem Süden einen weiteren Vorstoß ins Barbarenland mit einem Angriff Taroks verbanden, würden sie sich so bald nicht mehr in die Steppe trauen.

Langsam, aber sicher wurden die Schrecken der vergangenen Tage zu nichts weiter als unangenehmen Erinnerungen. Eines Tages waren sie gar nur noch Erinnerungen. Trupps der Iquar zogen aus in die wenigen Wälder an den nahe gelegenen Berghängen und kehrten mit einer angemessenen Menge an Holz zurück. Das Loch in der Siedlung wurde wieder zu- und ausgebaut. Der Zeitturm wurde wieder aufgestellt, höher und stabiler denn je. Der große Lifornus konnte wieder Feuertricks vorführen, ohne böse Blicke zu erfahren. Barz‘ Bart wuchs wieder nach. Und die Kinder blickten nicht mehr furchtsam gen Himmel, sondern bastelten fröhlich kleine Drachenfigürchen in der Schule, von welchen Karyz ganz stolz je eine an Zanyitaz, Yafka und Barz überreichte. Barz stellte seine in seinen Arbeitsraum und richtete immer wieder ein paar dankbare Gedanken an die Götter, wenn sein Blick während des Analysierens der Phoenix-Asche an der Figur hängen blieb.\bigskip







Sonnenaufgang. 15 Tage nach dem großen Unheil.\bigskip



Über Nacht war der erste Schnee gefallen und hatte die Steppe in ein weißes Glitzerfeld verwandelt. Aufgeregte Kinder wurden von ihren Eltern ans Ufer gefahren, damit sie dort Schneeballschlachten ausfechten konnten. So manch ein Erwachsener gesellte sich ebenfalls dazu. Und während die Steppenechsen in ihre reglosen Starren verfielen, verschwanden auch immer mehr Krarks aus dem Himmel, wie sie es die meisten Winter taten. So kam es, dass kurz nach dem ersten Schneefall endlich ein Brief mit einem Falken nach Andor ausgesandt werden konnte.\bigskip







Sonnenhoch. 22 Tage nach dem großen Unheil.\bigskip



Ein eleganter Falke erreichte den großen See Ava und präsentierte stolz einen langen Brief an seinem Bein. Die fröhliche Antwort aus Andor hatte nicht lange auf sich warten lassen, und wurde bald darauf vor dem Versammlungssaal von einer nicht unzufrieden dreinblickenden Stammesleiterin Naquila verlesen: Der Angriff der Barbaren auf Andor war gescheitert, und dennoch großartig gewesen!

Eine Gruppe von Andori hatte sich der anstürmenden Vorhut der Barbaren gestellt, kaum dass diese über die nördlichen Ausläufer des Grauen Gebirges ins Königreich Andor eingedrungen waren. Eine Gruppe bestehend aus Kriegern, Bogenschützen, ja, gar Zwergen und Zauberern. Wenn man den Gerüchten trauen durfte, hatten sich auch ein abtrünniger Spion aus der Büffel-Sippe und selbst der Fleisch gewordene große Büffel höchstpersönlich den Barbaren entgegen gestellt. Ab dann hatten die meisten gewusst, dass diese Invasion eine törichte, verlorene Sache gewesen war.

Doch hatten die Andori den Barbarenkönig leben lassen, und mit ihm sein Volk. Der Barbarenkönig war vor den König von Andor getreten und hatte ihm seine Krone dargeboten. Dem König von Andor war das Riesenvolk, die Krahder, wohl bekannt, war er doch selbst als junger Mann aus ihrer Gefangenschaft geflohen. Er hatte Verständnis für die Barbaren gehabt und sie im östlichen Rietland willkommen geheißen.

Der Mehrheit der Barbaren ging es nun offenbar gut im fremden Königreich. Sie hatten ihre Jurten im goldenen Rietgras aufgebaut. Manche waren gar in Kontakt mit den bereits dort ansässigen Bauern getreten und waren daran, Freundschaften aufzubauen. Als geschickte Schmiede, gute Holzwerker und eine willkommene Unterstützung gegen die bösen Kreaturen wurden die Barbaren offenbar gut geschätzt. Das Rietland war groß und Platz hatte es genug. Nur die Zwerge aus Cavern schienen von den neuen Nachbarn nicht viel zu halten und hatten die Wachen an den Minenausgängen verdreifacht. Mit dem Winter hatten die Yetohe in der Steppe schon oft gekämpft, und nun konnten sie gar mit Nachbarn über Vorräte von Korn, Brot und dergleichen verhandeln. Manche munkelten, Häuptling Absorak habe gar die dünne Plörre, die die Andori Met nannten, zu schmecken gelernt.

Weitere Falken wurden ausgetauscht zwischen dem Ava und Andor. Für die Iquar schien die ganze Sache größtenteils vorbei zu sein, doch die verschiedenen Sippen der Yetohe hatten zum Teil ganz unterschiedliche Vorstellungen, wie sie fortan leben sollten. Manche wollten im sicheren Rietland verbleiben, vielleicht gar dort ansässig werden. Manche wollten lieber im nächsten Frühjahr, sobald ihre Reit-Echsen aus der Winterstarre erwacht wären, in die verbrannte Steppe zurückkehren, die bereits so lange ihre Heimat gewesen war. Irgendwie musste es sich doch auch dort wieder überleben lassen. Und natürlich gab es auch einige Krieger der Iquar, die plötzlich mit dem Gedanken spielen mussten, ob sie im Rietland verbleiben und vielleicht ihre Familien dorthin einladen oder an den Ava zurückkehren wollten.

Barz freute sich über alle guten Nachrichten, und doch füllte sich sein Herz mit Sorge. Einige tapfere Krieger der Barbaren im verzweifelten Angriff ihr Leben gelassen oder sich üble Verletzungen zugezogen. Und Barz hatte noch nichts von Nabib gehört.

Immerhin wusste er, dass Nabib noch am Leben war. Schon seit einiger Zeit hatte Barz immer wieder mal die Wirkung seines Meditationspulvers genossen, um für einige Stunden vor sich hin zu sinnieren und sich zu entspannen. Wenn er dabei nur fest genug an Nabib dachte, schien es ihm, als könne er aus der Ferne dessen Stimme vernehmen. Dann sprach Barz zu sich selbst, und auch zu Nabib, immer wieder, wie ein Mantra: „Nabib, wo auch immer du gerade sein magst: Wir werden uns wiedersehen. Ich werde dich finden.“

Und auch wenn Barz nichts Genaueres empfand als die reine Existenz dieser meditativen Verbindung zu Nabib, so wurde ihm dabei immer warm ums Herz und er spürte, dass Nabib noch da war, irgendwo weit weg von ihm in dieser wilden Welt.

\az{Jahr 63}

Dann kam der Jahreswechsel, und dann zog Winter auch schon wieder vorüber und der Schnee wich Blumen, welche rund um den Ava aus Skelettschädeln sprossen. Inzwischen war eine Normalität in Barz’ Leben zurückgekehrt, welche er lange vermisst hatte. Das Leben mit Yafka, Zanyitaz und Karyz, ganz zu schweigen vom zankenden Ehepaar der Yetohe, die einander tatsächlich heroisch aus dem von Tarok ausgelösten Feuer gerettet hatten, war abwechslungsreicher und aufregender, als es Barz‘ Aufenthalte in Thakkum über die Winterwochen üblicherweise gewesen waren. Doch das empfand er nicht als Nachteil.

Als das Tauwetter gekommen war, wurde es für viele Bewohner Thakkums so langsam Zeit, zu Reisen aufzubrechen. Manche der dort aufgenommenen Yetohe und auch einige Iquar wollten über den Wachsamen Wald nach Andor reisen und sich wieder mit dem Rest ihrer Familien vereinen. Lifornus hatte offenbar genug vom Ava gesehen und wollte mit dem großen Zug zum Baum der Lieder reisen, um seine aufgefrischten Erkenntnisse zum angeblichen Aussterben der Drachen mit den dortigen Chroniken abzugleichen. Und auch für Barz war es wieder an der Zeit, auf der Suche nach seltenen Substanzen und außergewöhnlichen Pulvern zu einer Reise aufzubrechen.

Diesmal hatte er eine klare Idee, wohin es ihn verschlagen sollte: Nach Andor! Das Königreich, in dem viele Barbaren inzwischen ein Leben im Luxus führten. Dort endeten Nabibs Spuren, und dort würde Barz mit seiner Suche nach Nabib beginnen können.

Yafka würde auch dieses Jahr zuhause im Ava bleiben, erst recht nun mit Zanyitaz und Karyz an ihrer Seite. Und auch die Steppenechse Jirisa würde bei dieser Reise nicht mehr an Barz‘ Seite sein. Falls es ihm nicht gelingen würde, Nabib aufzuspüren, wäre er ganz alleine unterwegs. Barz unterdrückte die aufwallende Melancholie. Er würde nicht ganz alleine sein, er hatte ja stets noch Sabris Ei an seiner Seite. Bald wäre es für die kleine Sabri an der Zeit, zu schlüpfen, und Barz würde dabei sein.

Doch wollte Barz nicht mit dem restlichen Tross der Barbaren nach Andor aufbrechen. Nein, nicht ohne Grund hatte er in den letzten Wochen ausgiebig das Häufchen Phoenix-Asche studiert, welches er von Lifornus erhalten hatte.

Er war sich nun beinahe vollständig sicher, eine Verbindung einzigartiger Materialien gefunden zu haben, welche ihm erlauben sollte, seine eigene Position und die des Phoenixes zu vertauschen – und da der Phoenix bekanntlich vermutlich wahrscheinlich aus Andor stammte, würde dieser Trick es Barz erlauben, direkt im fremden Königreich aufzutauchen, ohne vorher eine lange Reise im Tross der Barbaren auf sich zu nehmen. Den Phoenix selbst sollte das nicht stören, der hatte ja schon bewiesen, selbst im Nu in seine Heimat zurückteleportieren zu können, falls er von dort versetzt wurde. Und Barz würde mit etwas Glück schon in kürzester Zeit Nabib wieder in die Augen blicken.

Ja. Das war ein guter Plan.\bigskip







Sonnenaufgang. 72 Tage nach dem großen Unheil.\bigskip



Barz legte seinen neuen langen Mantel mit dem vielen Taschen an, den er von Yafka geschenkt bekommen hatte. Diese unzähligen Taschen waren vollgestopft mit verschiedensten kleinen Pülverchen und Mittelchen, die ihm da draußen in der weiten Welt behilflich sein konnten. Neu darunter war das Vorhersehungspulver, welches er in den letzten Wochen mithilfe seiner Schamanin zusammengestellt hatte. Eine Hauptzutat waren natürlich Silberblumenblüten gewesen. Yafka war es tatsächlich gelungen, Ableger dieser seltenen Blumensorte vom großen Salzsee zu ziehen. Eingepackt war die visionäre Mischung im kleinen türkisen Säcklein, welches einst am Bein eines Krarks Sternkraut von den Jpaxo nach Thakkum transportiert hatte. Sternkraut, welches das große Unheil ausgelöst und doch Thakkum von den Krahdern befreit hatte. Barz verdrängte den Gedanken daran.

Eine weitere seiner Manteltaschen enthielt die kleine Drachenfigur, welche er von Karyz geschenkt bekommen hatte. Zahlreiche Taschen waren natürlich auch leer, damit Barz sie bei seinen Reisen im fremden Land erst füllen konnte. Zusätzlich hatte Barz einige Taschen mit Nahrung, Trinkschläuchen, Lagermaterial und weiteren vollen und leeren Probensäcklein umgeschnallt, und ebenso seinen eleganten Köcher. Er vermisste es, eine treue Lastechse mit sich zu führen. Für den Moment begnügte er sich mit Sabris Ei, welches er in einer besonders gut gepolsterten Umhängetasche vor seinem Bauch transportierte und mit einer gehörigen Portion Heizsand wärmte.

Zu guter Letzt schnürte er seine Stiefel, schwang sich seinen Bogen um und fühlte sich bereit für die Abreise.

„Ich werde zurückkommen, mit oder ohne Nabib, und mit allerlei neuen Abenteuergeschichten“, versprach er. Bei seiner Verabschiedung von seiner Familie flossen wieder Tränen. Karyz weigerte sich fast länger als Yafka, ihn loszulassen. Barz blickte ihnen allen ein letztes Mal tief in die Augen, inbesondere Yafka. Stolz und Freude wallten in ihm auf, hier in dieser Runde dazuzugehören.

Kurz schwankte sein Entschluss, quasi alleine zu dieser Reise ins Ungewisse aufzubrechen. Niemand zwang ihn, auf eine Reise zu gehen. Er könnte auch einfach in Thakkum verbleiben und weiterforschen. Dann aber schweiften seine Gedanken wieder zu Nabib. Noch lebte dieser, doch musste ihn irgendetwas vom Schreiben aufgehalten haben. Wo er auch war, vielleicht war er auf Barz angewiesen. Und dann würde ihn Barz auf keinen Fall im Stich lassen wollen.

So stand Barz breitbeinig in der Mitte seines Wohnzimmers und gebot allen Anwesenden, ausreichend Abstand zu halten. Er ergriff die Prise Phoenix-Asche, welche er mit zahlreichen weiteren Pülverchen aus dem Hause seiner Schamanin aufgepeppt hatte, ließ drei Tropfen Wasser aus einem Trinkschlauch darauf fallen, zerrieb die entstehende Masse, pustete darauf und ließ zu guter Letzt einen Funken darauf sprühen. Seine Handfläche glühte rosa glimmernd auf, bis es ihn beinahe blendete. Karyz machte große Augen, in denen sich das rosa Glühen spiegelte.

Barz dachte ganz fest an den Phoenix, den er gesehen hatte, führte das pink glühende Pulver vor sein Gesicht, ballte seine Faust und...\bigskip



...benommen richtete sich Barz auf. Sein Umfeld nahm er nur schemenhaft wahr. Stark pulsierend floss das Blut durch seinen Körper. Was war geschehen? Wie war er hierhergekommen? Er erinnerte sich einzig allein daran, dass er gerade dabei gewesen war, diese neue Pulvermischung auszuprobieren. Dann war alles verschwommen und auf einmal hatte er sich in dieser ihm unbekannten Umgebung wiedergefunden. Hatte er es diesmal übertrieben? Hätte er lieber auf konventionellem Wege nach Andor reisen sollen? Befand er sich überhaupt in Andor?

Vorsichtig richtete Barz sich auf. Es war überraschend dunkel. Nur von einer Seite fiel helles Sonnenlicht auf den Boden und zeichnete dort einen hellen Fleck auf einen glatten Steinboden mit einigen Strohnestern darauf. Aber das war ja gar kein Stroh! Es waren helle, weiße Stränge eines Barz unbekannten biegsamen Materials. Schnell steckte er einen ein, für spätere Analyse.

Allem Anschein nach befand sich Barz in einer kleinen Höhle, deren Decke gerade hoch genug lag, damit ein Mensch aufrecht stehen konnte. Das Sonnenlicht schien durch den kleinen Höhleneingang auf das eigenartige weiße Material, das den Boden bedeckte. Barz huschte zum Eingang, blickte hinaus und staunte. Das war ganz und gar nicht so, wie er sich Andor vorgestellt hatte.

Barz‘ Höhle befand sich nicht auf dem Boden, sondern weit oben in der Höhe! Außerhalb des Eingangs fiel eine steile Steilwand in die Tiefe, und sowohl weiter oben als auch weiter unten erspähte Barz weitere Einbuchtungen, die zu Höhlen führen konnten. Dieser Ort sah aus wie ein riesiger löcheriger Baum, doch aus purem Stein. Eine hohe Konstruktion aus Plattformen und Stangen reichte um den Baum herum. Menschen und kleinere Wesen wuselten auf dem Gerüst herum. Bauten sie irgendetwas?

Am für Barz‘ Geschmack viel zu tief unter ihm liegenden Boden reichten die Wurzeln des Steinbaums weit in die Breite und wurden von sattem Gras bedeckt – in die Richtung, die Barz erspähen konnte, breitete sich eine weite Steppe aus. Doch war das Gras nicht golden, wie das Rietgras Andors beschrieben worden war, sondern feuerrot. Wo befand er sich hier nur? Stammte der von Lifornus gerufene Phoenix etwa gar nicht aus Andor?

Die Steppe schien endlos, doch es gab viele Dörfer, und an verschiedenen Stellen glaubte Barz, Schafe und Ziegen zu sehen, die in kleinen Herden über das flache Land getrieben wurden.

Ein hoher freudiger Schrei erklang, und eine riesige Gestalt huschte am Eingang von Barz Höhle vorbei, feuerrot, Wärme ausstrahlend, ja, gar buchstäblich in Flammen stehend. Barz stolperte zurück, das Bild von Jirisas verbranntem Körper urplötzlich wieder vor seinem inneren Auge aufblitzend. Der Phoenix vor der Höhle rauschte indes in die Tiefe, breitete seine riesigen Flügel aus und schoss von einem Flammenschweif verfolgt wieder in die Höhe, während er einen melodischen Schrei ausstieß. Drei weitere, kleine Phoenixe verfolgten den großen, doch flogen sie noch weitaus weniger elegant und weniger kontrolliert. Während das gesamte Gefieder des großen Phoenixes in Flammen zu stehen schien, konnte er bei den drei kleineren höchstens ein paar Funken am Ende ihrer Schwanzfedern erkennen.

Nur langsam traute sich Barz wieder nach vorne. Gefährlich, wie das Feuer auch war, war es doch auch wunderschön anzusehen. Barz hätte aus dieser sicheren Entfernung noch lange den majestätischen Flug der Feuervögel verfolgen können, doch in diesem Augenblick schwang sich wie aus dem Nichts ein kleines buckliges Männlein von einer erhöhten Plattform in Barz‘ Höhle, erschrak gewaltig, als es Barz sah und hob daraufhin bedrohlich seine Hände, in welchen ein grünliches magisches Licht aufglomm.

„Ich will dir nichts tun. Ich habe mich hier verirrt“, sagte Barz rasch und versuchte, möglichst ungefährlich dreinzuschauen. Er hatte keine Ahnung, ob das Kerlchen ihn tatsächlich verstehen konnte.

Der kleine Wichtel musterte ihn weiterhin argwöhnisch, nickte dann aber grimmig. Offenbar hielt er Barz für ungefährlich genug, dass er sich traute, seinen Kopf wieder aus der Höhle rauszustrecken und laut zu rufen:

„Aćh! Aaaaaćh! Hüterin Aćh! Da sitzt ein Mann einfach so im Nest der Takuri!“

Barz, der der Sprache der Tulgori damals noch nicht mächtig war, verstand den Temm natürlich nicht. Ebensowenig verstand er die Antwort, welche von weiter unten im Steinbaum erschall, und da sie nicht sonderlich freundlich klang, rechnete er lieber mit einer unangenehmen Konfrontation und schob seine Hand in einen Pulversack mit Schwächungspulver.

Die nächste Person, die in Barz‘ Höhle kletterte, war kein Wichtel, sondern Menschenfrau mit einem leuchtend roten langen Umhang, welcher durch zwei gekreuzte Phoenix-Federn eindeutig die Verbindung der Frau zu diesen mythischen Feuervögeln signalisierte. Sie blickte ziemlich verwirrt drein, als sie den über und über mit Taschen bedeckten Barz vor einem leeren Nest stehen sah. Nichtsdestotrotz hatte sie rasch ein goldenes Schwert gezückt und vor Barz‘ Kehle gehalten.

„Wer bist du? Woher kommst du? Was suchst du hier?“

Was für eine faszinierende fremde Sprache das war, mit einer Menge abgehackter Laute und einem ‚s‘, das stark nach Lispeln klang. Barz versuchte, die passenden Worte und Gesten zu finden, um möglichst friedlich zu kommunizieren, dass er die Worte seines Gegenübers nicht verstand. Dann gab er auf und wechselte Kurs.

„Barz“, sprach er, auf sich selbst zeigend.

Die Phoenix-Hüterin nickte vorsichtig, legte sich die eigene Hand auf die Brust und sprach ihm nach: „Barth.“

Barz schüttelte freundlich lächelnd seinen Kopf. Zumindest das Lächeln und einfache Kopfbewegungen schienen hier dieselben zu sein wie die Barz altbekannten.

„Barz“, wiederholte er, auf sich selbst zeigend, „Barz.“

Dann zeigte er auf die Phoenix-Hüterin und machte ein möglichst verwirrtes Gesicht, um sie nach ihrem Namen zu fragen.

„Barbarth“, wiederholte sie, seinen verwirrten Gesichtsausdruck imitierend, „Barbarth Barth.“

Barz musste grinsen, denn bei diesen Worten kam ihm ganz unpassenderweise der alte Kinderreim vom Barbaren Barz in den Sinn. Dieser Namensvetter von Barz, ein legendärer bärtiger Barbier aus dem Barbarenland, war eigentlich ein Bierbrauer gewesen und hatte auch eine eigene Bierbar zum Ausschank seines berühmten Barbarenbiers besessen, doch hatte seine Passion halt eben dem eleganten Stutzen von Barbarenbärten gegolten. Ein Barbarenbierbrauer-Barbarenbierbarbesitzer-Barbarenbartbarbier-Barbar Barz halt. Köstliche Wortkonstellationen konnte man da zusammenstellen, unabhängig davon, ob ein solcher Barbar Barz je wirklich existiert hatte. Als Kind hatten Barz‘ Eltern ihm immer wieder versprochen, dass sie ihn nicht aufgrund des Reims benannt hatten.

Fokus! Barz unterbrach sein Grinsen, blickte der Phoenix-Hüterin in die dunklen Augen und zeigte auf verschiedenste Dinge, die er sehen konnte und anschließend laut benannte – das Nest – „Nest“ –, den Wichtel – „Wichtel“–, einen vorbeirauschenden Phoenix – „Phoenix“–, und zu guter Letzt sich selbst – „Barz“.

Nun schien der Hüterin ein Lichtlein aufzugehen. Auch sie zeigte nun auf einen vorbeirauschenden Phoenix – „Takuri“ –, Barz – „Barth“ – und zu guter Letzt auf sich selbst – „Ack“.

Ack, die Takuri-Hüterin. Sehr schön, damit war der Startstein für ihre Kommunikation gelegt. Barz wiederholte Acks Namen. Nun grinste sie und schüttelte ihren Kopf, vermutlich wegen Barz‘ grässlicher Betonung. Er trat einen Schritt näher. Sofort zückte Ack wieder ihr elegantes goldenes Schwert.

Da vernahm Barz ein Geräusch, das verdächtig nach dem Knacken einer Eierschale klang. Für einen kurzen Augenblick dachte er, dass er soeben irgendein Takuri-Ei zertreten hatte, und nun Acks Zorn erleiden müsste.

Dann blickte er an sich hinunter. Er glaubte, eine Bewegung in Sabris Tragetasche zu erkennen. Und da verstand er.

Er hatte tatsächlich eine knackende Eierschale gehört. Sabri! Seine Steppenechse schlüpfte!

Erschrocken und auch etwas hilflos blickte er wieder zurück zu Ack.

Und in ihren Augen erkannte er Verstehen.

\begin{center}
    Weiter geht es in \hypref{Der verschwundene Feuertakuri (2022)}.
\end{center}












\newpage
\section{Epiloge}

\az{Jahr 62}

„Du hast nicht nur den Tod einer unserer mächtigsten Hexen und der Verlust unserer halben Armee auf deinen Schultern zu tragen, sondern auch, dass du einen waschechten Drachen in unser Reich gelockt hast. Hätten wir nicht diese letzte Zera aus dem unterirdischen Krieg noch hier stehen gehabt, so frage ich dich: Wie stünde es um Borghorn?! Wie stünde es um unsere Familie?!“

Prinz Ferntahr schluckte tief und wagte es nicht, aufzublicken. Er verbarg sein vom Drachenfeuer entstelltes Gesicht hinter dem Schädel seiner Mutter.

König Gonhar, sein Vater, schlug ihm den Schädel aus dem Gesicht.

„Du bist es nicht wert, ihr Antlitz zu tragen! Du bist es nicht wert, unseren Namen zu tragen! Wenn du Undankbarer das nächste Mal diese ‚große Leere‘ oder ‚Lust auf mehr‘ in dir verspürst, wenn du dich das nächste Mal in diesen Hallen nicht heimisch fühlst, so verschwinde doch einfach und erlöse uns von deiner Dummheit! Cuhor hatte dich gewarnt, im Norden Vorsicht walten zu lassen!“

Jetzt wurde es Ferntahr doch zu viel: „Aber Vater! Cuhors Warnungen sind weniger verlässlich als der Tratsch eines altes Waschweibs! Und doch habe ich alle Vorsicht walten lassen, die man von mir verlangen könnte. Wir sind langsam vorgegangen, haben nie viel riskiert, ja, Nahrack hat gar jeden Tag die Geister der Gefallenen gefragt, ob...“

Gonhar beachtete seinen Sohn nicht einmal, sondern redete sich weiter in Rage: „Großzügig erfülle ich dir deinen Wunsch, diese fremde Land zu bereisen und diese törichten Norderländler für ihren Einfall in unser Königreich zu strafen, und das ist dein Dank?!“

Neben dem Thorn lehnte sich Prinzessin Ennevahr gegen einen Stein. Ein hässliches Grinsen verzerrte ihr Gesicht, während sie die Demütigung ihres Bruders beobachtete. Nun reckte sie sich zu Gonhar hoch und fragte betont unschuldig:

„Sag, Vati, darf Ferni nach diesem Debakel weiterhin seinen Teil der Armee der Toten führen? Oder müsste dieses Kommando ihm vielleicht entzogen...“

Ferntahr biss seine Zähne zusammen und knurrte: „Enn, halte dich da raus! Du kämst ja nicht einmal mit der Führung von zehn Skeletten klar, da...“

„Ferntahr, guck mich an! Du redest hier mit mir. Mit deinem König! Nicht mit deiner Schwester! Und die Antwort lautet: Nie wieder! Kein einziger unserer Hexer wird dir je wieder irgendwohin folgen! Kein einziges unserer Skelette wird sich deinem Befehl beugen!“

Ferntahr stöhnte auf: „Vater, habt Erbarmen. Wer konnte schon wissen, dass diese Barbaren einen verflammten Drachen auf ihrer Seite hatten? Und Nahracks Tod unterliegt nicht meiner Verantwortung, sie selbst legte sich mit Tarok an, anstatt zu fliehen. Das kann doch nicht meine...“

„Schweig, Unseliger!“, donnerte Gonhar, „Ich bin deine Ausreden satt! Ein fliehender Feigling bist du also auch noch. Was habe ich nur für einen Sohn erzogen?!“

Ferntahr versuchte, Ennevahrs Selbstgefälligkeit auszublenden und presste sich vor dem Thron noch tiefer auf den Boden, während er flehende Worte an seinen Vater richtete:

„Ich bitte Euch, lasst mir wenigstens das Kommando über das dritte Bataillon der Toten. Ich kann besser sein. Ich werde besser sein.“

„Das Kommando über deine Handvoll Ambacus magst du behalten, doch die mickrigen Überreste deines Heeres wirst du so bald nicht mehr sehen. Dabei bleibt es! Ich habe gesprochen!“

Gonhar atmete tief durch und ließ seinen massigen Körper ächzend zurück auf den grob gehauenen Felsen sinken. Diese Tirade hatte ihn ermüdet.

Ennevahr streckte Ferntahr die Zunge raus.

Ferntahr trat gegen einen Felsen. Er wusste, dass seines Vaters Gemüt rasch aufbrauste und dass er seinen Worten dann nicht allzu viel Glauben schenken sollte. Ferntahr konnte wieder in seinem Ansehen aufsteigen. Doch vergessen würde König Gonhar das Geschehene nicht. Wenn Ferntahr das nächste Mal irgendwo aufbrechen würde, würde er dies alleine tun müssen. Alleine, ohne Hexer, ohne Skelette. Höchstens mit einigen Ambacus.

Er knurrte frustriert und schlug erneut gegen einen Felsen.

„Darh, Rahha, kommt!“, rief er zweien seiner Ambacu-Hexen ungehalten zu, „Ruft mir zwei Golems und lasst sie meine Sänfte in meinen Turm tragen!“

„Mein Prinz“, druckste Darh herum, „Meister Corion verlangte nach mir, um...“

„Scheiß auf Corion, dein Prinz befiehlt es dir!“, brüllte Ferntahr auf. Selbst Ambacus wagten es, ihm zu widersprechen. Seine Augen wurden wässerig vor Schande.

Der Thron Borghorns würde ihm verwehrt blieben, solange Undavahr, dem erstgeborenen Prinzen, nichts geschah. Undavahr, der Feigling, der Ferntahr davon abgeraten hatte, die elenden Barbaren in den Norden zu verfolgen. Undavahr, der perfekte Sohn, der nie in die weite Welt aufbrechen wollte und zufrieden mit seinem Platz in dieser Ödnis schien.

Ferntahr seufzte.

Was wollte er hier in Krahd überhaupt noch?\bigskip



Nabib hustete und prustete, als der andorische Wassergeist ihn endlich aus seiner Gewalt (und aus seinem ertränkenden Körper) entließ. Die schimmernde Wassergestalt trat einen Schritt zurück und betrachtete ihn mit einem traurigen Blick aus blauen Augen hinter durchscheinenden blauen Haaren. Nabib hustete und prustete nach Luft, versuchte verzweifelt, sich aufzurichten, seine Waffe zu finden, schnell, rasch, ehe...

Ein stumpfer Schlag traf Nabibs Seite, woraufhin er endgültig auf den Boden klatschte und dort liegen blieb. Ängstlich starrte er nach oben und ins breite Gesicht eines Zwergs mit kurzem blonden Bart, welcher ihm eine aufwändig verzierte Doppelaxt unters Kinn hielt.

„Hübsch unten bleiben“, brummte der Zwerg. Er sprach die Sprache der Bewahrer, welche jedem reisenden Nomaden zumindest im Ansatz beigebracht wurde, wenn er in Kontakt mit anderen Völkern treten wollte. Immerhin bedeutete das, dass Nabib mit ihm verhandeln konnte.

Nabib betastete vorsichtig seine schmerzenden Rippen und erschrak, als er seine Finger hochhielt und helles Blut darauf sah. Er presste seine Hände auf die Wunde, während sein Geist raste. Bislang hatten Barz und er nur ein einziges Mal mit Zwergen zu tun gehabt, damals, als sie in die nördlichen Ausläufer des Grauen Gebirges aufgebrochen waren, um mit den Jpaxo zu reden. Die Zwerge waren aus dem Nichts aufgetaucht und hatten Barz und Nabib erst ziehen lassen, nachdem diese ihnen einen „angemessenen Zoll“ in Form ihrer goldenen Armbänder und Ketten überlassen hatten. Damals hatten die beiden Steppennomaden gelernt, dass Zwerge äußerst gierig sein konnten.

„Verschont mein Leben“, rief Nabib dem Zwerg zu. Dieser hielt ihm weiterhin seine Axt unters Kinn und legte seinen Kopf schief. Jetzt trat sogar noch ein weiterer Krieger der Andori an seine Seite, ein Mensch mit einem roten Haarschopf und einem Körperbau, den Nabib ungut an einige Krieger der Yetohe erinnert, an deren Seite er in den letzten Tagen gekämpft hatte. War es denn wirklich wahr, dass einige Barbaren gar auf die andere Seite übergelaufen waren? Nun, das war jetzt nicht relevant.

Nabib presste die Hände an seine Wunde. „Verschont mein Leben“, rief er erneut. „Es soll euer Schaden nicht sein“, ergänzte er und hielt dem Zwerg das reich verzierte Amulett hin, welches Barz ihm zu seiner Abreise aus Thakkum geschenkt hatte. Es schmerzte ihn, die Aufgabe dieses Andenkens überhaupt in Betracht zu ziehen, aber letzten Endes war sein Leben um einiges wichtiger als Barz‘ Erbstück. Barz würde das gut verstehen können.

„Das ist nicht nötig“, brummelte der rothaarige Mensch. Ja, Nabib glaubte eindeutig, einen Hauch barbarischen Akzents in seiner Stimme mitschwingen zu hören.

„Kein Held von Andor würde jemanden, der sich ergibt, angreifen“, bekräftigte nun auch der stämmige Zwerg. Dann traten er und der rothaarige Mensch an Nabibs Seite und halfen ihm auf.

Ein schwarzer Rabe ließ sich auf der Schulter des rothaarigen Menschen nieder, plusterte sein Gefieder und wirkte überaus zufrieden mit sich selbst. Er hielt seinen Schnabel an das Ohr des Rothaarigen und krächzte leise vor sich hin. Nabib hätte schwören können, dass die Stimme des Raben menschlich klang. Der rothaarige Mensch blickte den Zwerg und Nabib noch einmal prüfend an, nickte dann und wandte sich von den beiden ab, anderen Verletzten zu.

Während Nabib vom Zwerg in ein Lazarett in der Rietburg begleitet wurde, sah er aus dem Augenwinkel, wie der Barbarenkönig von seiner Reit-Echse stieg und seine Krone zu Füßen eines riesigen Büffelwesens mit großen Hörnern auf einem ansonsten relativ menschlichen Körper legte. War dies etwa der Fleisch gewordene Große Büffel, aufgrund dessen Auftritts vorgestern Absorak und der gesamte Büffel-Clan die Waffen niedergelegt hatte?

Nun, wenn selbst der Barbarenkönig sich den Helden von Andor ergab, war die Sache gelaufen. Was für ein Debakel die Invasion Andors doch gewesen war. All diese Toten und Verletzten, nur für nichts und wieder nichts. Mit Furcht dachte Nabib an die Zurückgebliebenen in Thakkum. Egal, ob die Riesen aus dem Süden mit ihrer Untotenarmee am großen See Ava Halt machten und die Pfahlbausiedlung belagerten, oder ob sie auf den Spuren der Barbarenarmee in Richtung Andor zogen, nahm diese Geschichte kein gutes Ende. Es schüttelte ihn.

Und dennoch... während der Zwerg ihn an einem riesigen orangen Feuer in einer eisernen Schale vorbei und durch ein riesiges Tor in die mächtige Rietburg eskortierte, konnte Nabib das Gefühl nicht abschütteln, dass seine Zukunft nicht allzu düster aussah. Denn urplötzlich trat Barz‘ Gesicht vor sein inneres Auge. Urplötzlich erfüllte Nabib eine innere Ruhe und Gelassenheit, die er nicht einmal für möglich gehalten hatte. Als hätte er bereits seit Stunden meditiert. Und es war ihm, als höre er die leise Stimme Barz‘ in seinem Ohr.

„Nabib, wo auch immer du gerade sein magst: Wir werden uns wiedersehen. Ich werde dich finden.“\bigskip



„Bitte, haltet ein mit Euren Lobpreisungen, ich bin wahrlich keine Gottheit. Ich bin nur Bragor, ein Tarus aus dem fernen Sturmtal. Werter Häuptling Absorak, es ehrt mich, wenn Ihr in mir etwas Besonderes seht. Und natürlich will ich, dass diese Streitigkeiten hier beigelegt werden. Keine Königreiche profitieren von einem solchen Krieg, weder Andor noch die Barbaren. Doch will ich Euch nicht täuschen. Mit Eurem heiligen Büffel...

... verzeiht, mit Eurem \textit{Großen} Büffel habe ich nichts zu tun. Ich habe überhaupt noch nie einen Büffel gesehen in meinem kurzen Leben.

Na ja, keinen lebendigen jedenfalls. Und diese Statue zeigt doch überhaupt erst, wie unterschiedlich wir sind. Guckt mich doch an, ich habe Finger! Und Daumen! Welcher vernünftige Büffel besäße denn Daumen?

Nein, da müsst Ihr euch irren, ich war noch nie...

Nein, ich habe auch noch nie einen Bart getragen. Wisst ihr, bei uns Taren dauert das ein bisschen länger als bei Euch Menschen.

Was?! Mögt ihr das noch einmal wiederholen, werter Häuptling?

Ein Tarus wie ich? Seid Ihr Euch ganz sicher?

Beim Wüten der Sturmgeister! Wie lange ist das her? Wo befindet sich dieser Tarus jetzt?

Das Graue Gebirge? Auf der Suche nach Sternkraut?! Das ist er! Das muss es sein!

Wer? Na, mein Opa. Wegen ihm bin ich überhaupt erst nach Andor gekommen. Und dank Euch habe ich nun endlich einen Hinweis darauf, wo ich weiter nach ihm suchen könnte.

Nein, nicht zwingend gerade jetzt. Nicht, solange hier noch Helden benötigt werden. Seht Ihr den Schild dort drüben? Das ist der mächtige Bruderschild, bei dem haben wir uns der Hilfe der Hilfsbedürftigen hier verschrieben, und dieses Versprechen werde ich nicht leichtfertig brechen. Aber irgendwann mal, wenn die Lage hier ruhiger ist.

Ihr habt mir soeben Hoffnung geschenkt. Ich danke Euch von ganzem Herzen, werter Absorak.

Opa, wo auch immer du gerade sein magst: Wir werden uns wiedersehen. Ich werde dich finden.“


