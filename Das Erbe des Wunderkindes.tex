% Noch nicht genug? In einer der hiesigen Geschichten wird Thermosta erwähnt. Sie bot Feuermeisterin Nidwal einst einen gewissen Betrag für ein Fernohr an. Diese drei Zahlen verknüpft ergeben eine fünfstellige Zahl. Und wir alle wissen, was man in der Taverne mit fünfstelligen Zahlen so anstellen kann. ^^

% !TeX program = lualatex

\documentclass[10pt, a4paper, oneside]{book}
\usepackage[left=3.5cm,right=3.5cm, top=1cm,bottom=1cm,includeheadfoot]{geometry} %definiere Ränder und so
\usepackage[ngerman]{babel} % für deutsche Titelbezeichnungen
\usepackage{graphicx} % für Bildereinfügen
\usepackage{eso-pic} % unter anderem für die Hintergrundbilder
\usepackage{tikz} % unter anderem für die Hintergrundbilder

\usepackage{csquotes} % für Anführungszeichen
    \MakeOuterQuote{"}

\usepackage{multicol}

\usepackage{soul} % für Befehle wie \st (durchgestrichen)

\usepackage{hyperref}
    \hypersetup{
        colorlinks=true,
        linkcolor=blue,
        filecolor=blue,      
        urlcolor=blue,
        pdftitle={Das Erbe des Wunderkindes},
        pdfpagemode=FullScreen,
    }


\newcommand{\hypref}[1]{%
    \hyperref[#1]{#1}%
}

\newcommand{\hintergrund}[1]{%
    \AddToShipoutPictureBG{%
        \tikz[remember picture, overlay] \node[opacity=1, inner sep=0pt] at(current page.center){\includegraphics[width=\paperwidth,height=\paperheight]{Das Erbe des Wunderkindes/Hintergrund/#1}};%
    }
}

\newcommand{\titelseite}[1]{%
    \AddToShipoutPictureBG*{%
        \tikz[remember picture, overlay] \node[opacity=1, inner sep=0pt] at(current page.center){\includegraphics[width=\paperwidth,height=\paperheight]{Das Erbe des Wunderkindes/Hintergrund/#1}};%
    }
}

\usepackage{tocloft}
\setlength{\cftsecindent}{-7pt} % entferne Einrücken im Inhaltsverzeichnigs

\usepackage{fontspec}
\newfontfamily{\andorfont}{Andor-Schriftart}
[Extension = .otf]

\usepackage{sectsty} %section stiling
\partfont{\fontsize{50}{60}\andorfont}
\chapterfont{\centering}
\sectionfont{\centering}
\renewcommand\thesection{} %damit sections im Inhaltsverzeichnis nicht nummeriert werden
\renewcommand\thesubsection{} %damit subsections im Inhaltsverzeichnis nicht nummeriert werden

\usepackage{titlesec}
\titlespacing*{\chapter}{0pt}{30pt}{20pt}
\titleformat{\chapter}{\centering\normalfont\Large\bfseries}{}{0pt}{\huge}
\titleformat{\section}{\centering\normalfont\large\bfseries}{}{0pt}{\huge}

\usepackage{fancyhdr} %für fancy Headers und Footers

\usepackage{extramarks} %für \lastrightmark und xmark's

\pagestyle{fancy}

\newcommand{\az}[1]{%fügt einen "nach andorischer Zeit"-Block ein
    \begin{center}
        \includegraphics[width=180px]{Das Erbe des Wunderkindes/verzierung1.png}\\
        {\Huge #1} \\
        {nach andorischer Zeitrechnung}\\
        \includegraphics[width=180px]{Das Erbe des Wunderkindes/verzierung2.png}
    \end{center}
    \extramarks{}{#1 a.Z.}
}

\newcommand{\jdh}[1]{%fügt einen "im Jahre des Herrn"-Block ein
    \begin{center}
        \includegraphics[width=180px]{Das Erbe des Wunderkindes/verzierung1.png}\\
        {Im Jahre des Herrn}\\
        \vspace{5pt}
        {\Huge #1} \\
        \includegraphics[width=180px]{Das Erbe des Wunderkindes/verzierung2.png}
    \end{center}
    \extramarks{}{A.D. #1}
}

\newcommand{\bildmitts}[2][height=0.5\textwidth,width=0.75\textwidth,keepaspectratio]{%
    \begin{center}
        \includegraphics[#1]{Das Erbe des Wunderkindes/Bilder/#2}
    \end{center}
}

\newcommand{\bildlinks}[2][height=0.32\textwidth,width=0.48\textwidth,keepaspectratio]{%
    \begin{wrapfigure}{L}{0px}
        \includegraphics[#1]{Das Erbe des Wunderkindes/Bilder/#2}
    \end{wrapfigure}
}



\renewcommand{\footnoterule}{%stretche footnote-Linie über die ganze Länge
  \kern -3pt
  \hrule width \textwidth
  \kern 2pt
}


\renewcommand{\chaptermark}[1]{\markboth{#1}{}} %keine Nummerierung für die chapter-header
\renewcommand{\sectionmark}[1]{\markright{#1}} %keine Nummerierung für die section-header

\fancyhf{}%lösche alle Header-Footer-default-Einstellungen
\fancyfoot[C]{\thepage} %Seitenzahl im Footer, zentriert 
\fancyhead[L]{\nouppercase{\lastleftmark}}
\fancyhead[C]{}
\fancyhead[R]{\lastrightxmark}



\usepackage{environ}
\usepackage[most]{tcolorbox} % für farbige Boxen
\NewEnviron{chapterbox} %box für chapter-Anfänge
{   
    \newpage
    \begin{tcolorbox}[arc=0pt, colback=white, boxrule=0pt]
        \BODY
        \vfill
    \end{tcolorbox}
}


\renewcommand\thechapter{\Alph{chapter}} %damit chapters mit Großbuchstaben bezeichnet werden


\usepackage{contour}
\contourlength{1.2px}


\title{Das Erbe des Wunderkindes\\Stand 2025}
\author{Fan-Geschichten von Butterbrotbär,\\basierend auf Werken von Michael Menzel, Stefanie Schmitt,\\Christoph Kling, Dorothea Michels, Andreas Kälber, Matthias Miller,\\Jens Baumeister, Timo Grubing, Peter Gustav Bartschat,\\Gerhard Hecht, Inka und Markus Brand, Stefan Blanck\\und allerlei anderen andorischen Autor*innen sowie\\vielen fantastischen Fan-Kreationen verfasst von\\allen anderen atemberaubenden Andori}
\date{\vfill\textbf{Spoiler-Gefahr} für die offiziellen Andorversen.\\Für eine Übersicht über selbige, siehe „Chronik der Andorversen“: \url{https://legenden-von-andor.de/forum/viewtopic.php?f=11&t=10314}.\bigskip\\Dieses Dokument wurde erstellt mit \LaTeX{}. Für die neuste Version siehe \url{https://legenden-von-andor.de/forum/viewtopic.php?f=11&t=10516}.}

\begin{document}

\hintergrund{Hintergrund.jpg}



\begin{titlepage}
    \titelseite{Andor Fan-Karte Inkarnate v2.0 textlos A4.jpg}
    \centering

    \textbf{ }

    \vspace{2em}
    
    \includegraphics[width=\textwidth]{Das Erbe des Wunderkindes/BBBs Fan-Geschichten von Andor Logo.png}

    \vspace{32em}

    \fontsize{72}{100}
    \textcolor{white}{\andorfont \contour{black}{Das Erbe des}}
    
    \textcolor{white}{\andorfont \contour{black}{Wunderkindes}}

    \fontsize{36}{50}
    \textcolor{white}{\andorfont \contour{black}{Stand 2025}}
\end{titlepage}



\maketitle

\thispagestyle{plain}

\setcounter{page}{3} %lasse bereits das Titelbild als Seite 1 zählen

\vspace*{\fill}

\textit{Jahrhunderte vor den Legenden von Andor schufen der erfinderische Zwerg Kreatok und der tempramentvolle Drache Nehal gemeinsam die vier mächtigen Schilde aus uralter Zeit, nach denen sich wir Schildzwerge uns noch heute benennen. Doch das Wunderkind Kreatok kreierte auch zahlreiche andere Wunderwerke: Fein ausgedachte Fallen und Verteidigungsanlagen, mit magischen Runenfolgen verzierte kunstvolle Reliefs sowie die Sphäre, ein in den Chroniken unseres Volkes fast völlig verschollen gegangenes Artefakt, welches es ermöglicht, Portale zwischen Vergangenheit und Zukunft zu spannen.}\bigskip

\textit{In den Wirren des Unterirdischen Kriegs zwischen Schildzwergen, Drachen, Krahdern und Trollen gingen die meisten Werke Kreatoks und Nehals verloren. Erst in den Siebzigerjahren nach andorischer Zeitrechnung wurden die vier mächtigen Schilde wieder vereint, nur um bei der legendären Zerstörung der Krahder-Feste Borghorn unwiderbringlich im Lavameer zu versinken. Der Zufall – oder das Schicksal – will es, dass wenige Jahre zuvor Kreatoks Sphäre nach Jahrhunderten des verstaubten Herumliegens wieder aus der geheimen Fürstenkammer entnommen und ebenso unwiderbringlich zerschmettert wurde. Ebenso wurde Kreatoks versiegelte Kultstätte kürzlich von Drachenkultisten entweiht. Ein Wunder, dass der dort gelagerte Rubin von Kreatok wieder zurückerobert werden konnte, ehe auch er vernichtet würde.}\bigskip

\textit{Wir leben in einer Ära, in welcher sich das Erbe des Wunderkindes augenscheinlich abrupt seinem Ende nähert. Dies ist an sich weder gut noch schlecht. Früher oder später findet alles ein Ende, so will es der Lauf der Dinge, und Kreatoks Kreationen haben in ihrer langen Existenz vergleichbar viel Leid verursacht wie Freude geschenkt. Doch als jemand, dem die seltene Ehre zuteil wurde, den zeitreisenden Kreatok persönlich anzutreffen, fände ich es zu schade, wenn das Wissen über die Werke des ersten Schildzwergs ebenso rasch dem Vergessen anheimfallen würde. Denn jene spielten in den legendären Geschehnissen der letzten Jahrzehnte eine erheblich größere Rolle, als selbst die sonst so allwissend scheinenden Bewahrer vom Baum der Lieder ihnen zuschreiben würden.}\bigskip

– Aufzeichnungen von Mart, stellvertretender Fürst über Cavern in Absenz von Fürstin Marun und Fürst Kram, verfasst im Frühjahr 76 a.Z.

\vspace*{\fill}

\newpage

 
\tableofcontents







\part{Das bärig gebutterte Andorversum}

\begin{chapterbox}
    \chapter{Tion erzählt von Mutter Natur (2021)}
    \label{Tion erzählt von Mutter Natur (2021)}
    \az{Jahr 67}
    Novize Komu vom Baum der Lieder fragt Bewahrer Tion nach den Urzeiten des Andorversums. Tion kann diesem riesigen Thema natürlich niemals gerecht werden, will es aber nicht lassen, eine Geschichte über seine liebste Göttin vorzutragen.
\end{chapterbox}

\az{Jahr 67}

\textit{"Wie ein Gewitter, das man fürchtet und gleichzeitig bewundert."}

\textit{"Sie sind die Seele des Flusses und erhalten ihn am Leben."}

\textit{"Ein magisches Wesen, das als der Urvater aller Trolle [gilt]."}

\textit{"[...]dem wohl mächtigsten Wesen der ganzen südlichen Welt."}

\textit{"Die Trolle lebten schon immer in Andor."}

\textit{"Die Zwerge [gruben] tief unter der Erde Schächte und riesige Hallen, die den Drachen als Heimstatt dienten. Im Gegenzug übergaben die Drachen den Zwergen die Geheimnisse des Feuers und erlaubten ihnen damit, die ewige Düsternis unter dem Gebirge zu vertreiben."}

\textit{"Krahal [...] wurde vor vielen Hundert Jahren tief erschüttert."}

\textit{"Was ich sah, war ein Geflecht aus Düsternis, Umrisse eines Baumes, in dessen Adern rotes Blut floss."}

\textit{"Die letzten Drachen allerdings starben oder versteinerten mit den Jahren. Nur bei einem von ihnen, dem jungen Drachen, der die Kraft des Feuerschildes mit eigenen Augen gesehen hatte, erlosch die Flamme des Zorns nie."}

\textit{"Sie beteten leise zu Mutter Natur, sie möge dem Kranken Kraft schenken und ihn gesunden lassen."}

\textit{"Dort, glaubten die Andori, würde das ewige Glück auf sie warten."}\bigskip

Novize Komu vom Baum der Lieder leerte derart viele Schriftrollen und Schriftstück-Fetzen auf einmal aus der Holzkiste, dass einige davon fröhlich vom Schreibpult rollten und auf den Stammboden kullerten. Während Komu hastig auf den Boden sank und die Rollen von Weiterrollen abhielt, legte Meister Tion seinen Kopf schief und zog seine buschigen Augenbrauen zu einem fragenden Blick zusammen.

"Warum so hektisch, Komu? Liegt dir etwas auf dem Herzen?"

Komu stockte einige Male, aber sobald der Redeschwall einmal begonnen hatte, war dieser kaum zu bremsen. Enthusiasmus für die Arbeit der Bewahrer hatte das Kind, das musste man ihm lassen.

"Nun, ich habe mich kürzlich gefragt... wir haben ja in der Schule viele... viele Bruchstücke aus der reichen Vergangenheit Andors haben wir schon vernommen, aber... nun, es ist so... die Stücke der Geschichten sind da, aber zwischen ihnen liegen immer noch große Lücken, und ich habe Ega darauf angesprochen. Ega meinte, dass wir noch ganze zwei Jahre warten müssten, bis wir soweit wären, einen ersten geordneten Überblick über die Urzeiten zu erhalten... dass das anspruchsvolle Texte sind, das verstehe ich ja, aber so lange warten mag ich nicht, und da... da fragte ich mich... nun... wie passen diese Stücke alle zusammen? Wie lautet denn die ganze Geschichte der Urzeit? So, nur in der Kurzfassung. Also, falls man das schon verstehen kann mit unseren Kenntnissen."

Tion legte seine Schreibfeder beiseite und kratzte seinen langen Bart. Dann schmunzelte er.

"Ach, Komu, eine ‘ganze’ Geschichte gibt es nicht, das wirst du wahrscheinlich bald verstehen. Wir bewahren bloß Schriften aus den vergangenen Zeiten und fertigen neue Aufzeichnungen an. Je länger deine Ausbildung sich hinziehen wird, desto mehr Schriftstücke wirst du kennen lernen und miteinander in Verbindung setzen können. Dabei wirst du aber auch feststellen, dass es nicht den ‘einen’ Verlauf einer Legende gibt, sondern ganz unterschiedliche Berichte, welche zum Teil gar widersprüchlich sind. Es ist eine Kunst, einen möglichst objektiven Blick auf das Vergangene anzustreben, eine Kunst, die man wie alle Künste nur durch langes Üben meistern kann."

Tion wandte sich schon wieder seiner Schreibfeder zu, da hielt er inne und murmelte: "Was soll‘s, ich könnte ohnehin etwas Ablenkung gebrauchen. Hast du eine spezifische Frage oder Geschichte im Sinn, zu der ich ausführen könnte?"

Komu überlegte kurz und plauderte dann fröhlich los: "Hat Mutter Natur uns nun erschaffen oder nicht? Weil, meine Eltern sagten, dass das so sei, aber Ega behauptete, dass in den alten Texten was ganz anderes steht... und wie kam es eigentlich zum Streit zwischen Mutter Natur und dem Urtroll? Weil, so, wie ich das gehört habe, war der Urtroll am Anfang doch ein ganz gutmütiger Geselle, und jetzt wird er als so gefährlicher Gegner beschrieben, und irgendwie hängt doch alles damit zusammen, dass der Baum der Lieder überhaupt..."

"Irgendwie hängt doch alles mit allem zusammen", unterbrach Tion Komus Redeschwall zwinkernd, dann fuhr er fort, "Mutter Natur behüte deinen Eifer, junger Novize. Ich werde dir natürlich keine eindeutigen Antworten geben können. Wie viel Wirken Mutter Natur in unserer Entstehung hatte, ist debattierbar, auch wenn ihr Einfluss insbesondere auf unseren Orden sicherlich gigantisch war. Und ich könnte vielleicht eines Tages die Geschichte mit dem Urtroll erzählen, so wie ich sie in Erinnerung habe. Aber wie viel Wahrheit noch dahinter steckt, musst du am Ende selbst beurteilen. Es ist schon eine Zeit lang her, dass ich die Urtexte studiert habe. Das Niederschreiben der aktuellen Ereignisse im Lande nimmt einige Aufmerksamkeit in Anspruch."

Komu war fröhlich aufgehüpft, sobald Tion eine mögliche Geschichtenerzählung angesprochen hatte. Tion konnte nicht anders als zu grinsen und brummelte: "Nun gut, jetzt hast du mich mit deinem Fieber angesteckt. Trommle uns ein paar junge Adepten zusammen, ich bin in Erzähllaune."

Komu stieß eine Faust in die Höhe und trabte davon, um geschwind wie der Wind Freunde zu rufen. Tion blieb zurück und rief sich die wichtigsten Punkte zur Geschichte von Mutter Natur und dem Urtroll in Erinnerung. Auch wenn er es nicht direkt zugeben würde, mochte er es ungemein, dem Nachwuchs die alten Geschichten näher zu bringen. Würde er es dieses eine Mal schaffen, nicht allzu stark abzuschweifen? Wahrscheinlich nicht. Aber das war auch nicht die Hauptsache. Hmmm... wo wäre ein guter Einstieg für seine Erzählung?\bigskip

"Vom Helden Fenn haben wir erfahren, was die wilden Barbaren des Ostens ihren Kindern erzählen. Dass diese Welt einst leer und kahl gewesen sei bis auf die drei Götter, welche sie Stück für Stück mit Leben füllten – manche davon kreative Neukreationen wie die Menschen und Büffel, andere bloß Mischungen ihrer vorherigen Kreationen, etwa die Nixen. Die Götter ließen ihre Geschöpfe gegeneinander antreten, doch dann zogen sie sich immer mehr zurück, während ihr Interesse nachließ, bis sie bloß noch mit den wenigen Schamanen der Barbaren-Stämme kommunizierten, und auch das immer seltener.\bigskip

Manche Schildzwerge glauben, dass sie einst von einer Urmacht geschaffen wurden, um in deren Namen der Erde ihre Geheimnisse zu entlocken. Sie erzählen Legenden von einer utopischer Unterwelt tief unter der unseren, voller Gold, glitzernder Edelsteine und magischer Schätze, aber auch voller Feuer, Gefahren und finsteren Kreaturen. Sie graben selbst heute noch im Auftrag dieser Urgewalt tiefer und tiefer in die Tiefminen. Viele hoffen auf einen Fingerzeig dieser Entität. Seit einigen Jahrzehnten wird hier und da die Behauptung aufgerufen, dass der Tod Jari Dorrs und die Entdeckung der Silbermine ein solches Zeichen gewesen sei, das die Zwerge in den Norden riefe, doch die meisten älteren Schildzwerge tun dies als Humbug ab.\bigskip

Eine Menge Zauberer aus der fernen Eiswelt Hadrias meinen, dass höchstwahrscheinlich niemand eine Hand in ihrer Entstehung gehabt habe – höchstens eine Faustvoll wilder Magie, die aus der Hadrischen Unterwelt wie Dampf aufsteigt und alles Leben zumindest ein wenig formt.\bigskip

Nun, wir, die Bewahrer vom Baum der Lieder, glauben an das, was unsere Vorfahren lange vor unserer Zeit erlebten und niederschrieben. Diese Schriftstücke wurden in diesem Baum über die Jahrhunderte hinweg bewahrt, und viele detaillieren die Geschichte von Mutter Natur.

Die ersten Bewahrer beschreiben Mutter Natur als das erste Wesen, das das Licht der Welt erblickte und durch diese Landschaften strich. Manche waren sich sogar sicher, dass sie es gewesen sei, die dieses Licht der Welt überhaupt erst erschaffen habe. Fest steht, dass Mutter Natur sich seit Anbeginn der Flüsse, Berge und Wälder um die Natur sorgte und sich daran erfreute, wie die Sonne das Land beschien, die nassen Felsen in ihrem Licht glitzerten und allerlei Getier durch das Gras wuselte. Und da Freude durchs Teilen nur verstärkt wird, beseelte Mutter Natur das Land mit ihren Kindern. So schlüpften die Wassergeister in die Flüsse, die Erdgeister ins Fahle Gebirge, die Waldgeister in die Bäume und die Geister des Windes und der Wolken, angeführt vom edlen Arkteron, in die Lüfte. Die Naturgeister waren weitaus gescheiter als die Tiere des Landes, da Mutter Natur in jeden von ihnen einen Kern ihrer selbst pflanzte. Und Mutter Natur erfreute sich an den Seelen des Landes und hegte und pflegte sie so wie die restliche Natur.

Doch als Mutter Natur die Naturgeister erschaffen hatte, so konnte – oder wollte – sie sie nicht länger kontrollieren. Und während viele Geister damit glücklich waren, einfach zu existieren, zu interagieren und gemeinsam innerhalb der Ordnung dieser Welt umherzuspielen, so kam es, dass manche Geister sich von Mutter Naturs Führung abwandten und sich der Zerstörung zuschrieben. Diese empfanden nur Freude daran, anderen die Freude zu nehmen. Fürchterliche Feuergeister setzen die Bäume in den Brand, verdampften das Wasser, vertrockneten die Erde und verbrannten die Pflanzen.

Da war Mutter Natur traurig und versuchte, mit ihnen zu reden. Einige Feuergeister baten um Entschuldigung, und die ward ihnen gewährt. So entstanden die Feuertakuri des Westens.

Doch andere abtrünnige Geister, des Feuers, des Schattens und der dunklen Erde, wollten ihr Verhalten nicht ändern. Sie fürchteten, dass Mutter Natur sie richten würde, und zogen davon, verließen Mutters Lande und suchten sich ihr eigenes Heim, weit im Süden, wo Mutters Einfluss immer schwächer wurde. Die Erdgeister unter ihnen falteten das Land und schichteten das Graue Gebirge auf zwischen dem Südland, ihrer neuen Heimat, und dem Norderland von Mutter Natur. Sie hatten die Hoffnung, dass das Graue Gebirge sie vor Mutter Natur schützen würde, falls diese je beschließen sollte, etwas gegen ihre Zerstörung zu unternehmen.

Doch in ihrem eigenen Reich waren die abtrünnigen Geister immer noch von Sehnsucht nach ihrer alten Heimat erfüllt. Darum versuchten sie, diese zu reproduzieren. Sie erschufen Flüsse, doch nichts als Lava floss in ihnen. Sie formten Bäume, doch zerfielen diese zu Staub und Dreck. Sie erschufen ihre eigenen Tiere, doch diese klappten kraftlos und kalt zusammen, denn die Naturgeister kannten das Geheimnis des Lebens nicht.

Auch wenn keine schönen Kreationen dabei zustande kamen, so zogen die Taten der Geister die Aufmerksamkeit eines mächtigeren Wesens auf sich, das durch die Südlande streifte. Es war der Urtroll, und er kannte das Geheimnis des Lebens. Bislang hatte er sich nicht groß um diese Welt geschert. Doch als er sah, was die abtrünnigen Naturgeister zu schaffen versuchten, da fühlte der Urtroll Mitleid mit ihnen, und er hauchte ihren Kreationen Leben ein. Die entstandenen Wichte waren krank und missgestaltet, und alsbald krochen sie in die Höhlen unter dem Lande und versteckten sich vor dem Anblick derjenigen, die direkt Mutters Güte entstammten.

Da wurde der Urtroll dazu inspiriert, seine eigenen Wesen zu schaffen. Er formte aus der Erde des Grauen Gebirges die Trolle, und hauchte ihnen Leben ein, und die Trolle breiteten sich in die umliegenden Lande aus und richteten Verwüstung an. Mutter Natur hatte schon einige Zeit lang mit Sorge die Handlungen der abtrünnigen Geister betrachtet, und nun hielt sie es für nötig, einzugreifen, um ihre Welt vor den Trollen zu beschützen. Doch sah sie auch, dass der Urtroll stolz war auf seine Kinder, und dass man mit ihm reden konnte. So beschloss Mutter Natur, mit dem Urtroll verhandeln. Sie wollte allerdings nicht selbst in die Südlande reisen und Furcht und Schrecken unter den abtrünnigen Naturgeistern verbreiten, denn sie hatte auch ihre entflohenen Kinder immer noch lieb. So suchte sie stattdessen nach Boten, die ihre Nachricht zum Urtroll bringen könnten.

In der Zwischenzeit hatte Mutter Natur weiterhin die Tiere, Pflanzen und Waldpilze in ihrem Reich gehegt und gepflegt, und fröhlich erkannte sie, dass viele davon durch ihre gute Pflege selbst Seelen entwickelt hatten, ja einige gar in rudimentärem Kontakt zu den friedlichen Naturgeistern des Landes standen. Unter diesen vielen Arten fiel Mutter Natur besonders eine ins Auge, eine Sorte von breitwüchsigen Wesen, welche tief unter der Erde nach Geheimnissen suchten. Mutter Natur gab ihnen den Namen Zwerge und schenkte ihnen die Magie der Runen. Sie erschien dem Zwerg Eibert und sang ihm eine sanfte, klare Melodie vor, die ihn mit Kraft und Mut erfüllte. Mutter Natur bat Eibert, diese Töne mit seiner Zwergenschar zu teilen, sich durch das Graue Gebirge zu graben und das Lied dem Urtroll und dessen Brut vorzuspielen. Dies würde ihn hoffentlich besänftigen, sodass er der Zerstörung der Welt durch seine Kinder gewahr werden und eingreifen möge.

Eibert nannte die Melodie Steinsang, und er tat, wie von Mutter Natur gebeten. Doch kaum hatten die Zwerge mit den Grabungen begonnen und erste Konflikte mit den Erdgeistern ausgetragen, da stürzten die Drachen aus dem Himmel herab. Das waren mächtige magische Wesen, die vom roten Mond stammten und die Kunst des Feuers beherrschten. Manche unter ihnen waren wie die abtrünnigen Naturgeister dem Durst der Zerstörung erlegen und jagten bösartig durch das Land, doch andere von ihnen waren gutherzig und suchten die Nähe der restlichen Wesen.

So kam es, dass die Zwerge von Mutter Naturs Auftrag abkamen, um sich stattdessen um die Plage der bösartigen Drachen zu kümmern. Sie schlossen Frieden mit den gutmütigen Drachen und bauten ihnen große Hallen unter der Erde. Im Gegenzug teilten die Drachen mit ihnen das Geheimnis des Feuers. Die Zwerge nutzen das Feuer, um tiefer denn je unter die Erde zu dringen, tiefer als alle anderen Gänge Caverns, und dort schlugen sie eine gigantische Höhle in den Fels, groß genug für hunderte von Drachen. Die gutmütigen Drachen versammelten sich und ließen ihre ganze Magie in diese Höhle fließen, und da wuchs aus ihr ein riesiger, weiß schimmernder Baum, durch dessen kräftige Adern stetig rotes Blut gepumpt wurde. Ein lebendiges Zeichen des Friedens.

Dieser magische Ort war Krahal. Die Geister der Drachen vereinten sich in Krahal und wurden eins. Die Güte der Drachen, die den Ort gegründet hatten, umschloss den Zorn der bösartigen Drachen und verschloss ihn im weißen Baum. Das Böse war zumindest für eine Zeit lang nicht mehr Teil der Drachen. Ein Frieden zwischen Drachen und Zwergen wurde geschlossen. Anstatt sich auf Mutter Naturs friedlichen Auftrag zu besinnen und Steinsang zum Urtroll zu bringen, spielten die Zwerge und Drachen Steinsang nur für sich selbst, um sich mit noch mehr Mut und Kraft zu erfüllen. So drohten sie den Trollen.

Mutter Natur hatte mit Freude beobachtet, wie die Drachen sich auf die Seite ihrer Welt stellten, doch sah sie nun mit Sorge, wie der Urtroll um seine Kinder fürchtete und sich noch mehr von ihr zurückzog. So suchte sie nach einem neuen Botschafter. Ein Schattengeist trat ins Licht und bot an, Mutter Naturs Bitten zum Urtroll zu tragen. Mutter Natur konnte ihm nicht vertrauen, doch wie sich der Konflikt zwischen den Zwergen, Drachen und Trollen stetig zuspitzte, drängte die Zeit. Darum übergab sie dem Schattengeist eine neue Melodie und bat ihn, diese zum Urtroll zu bringen und ihm vorzuspielen. Diese Melodie sollte ihm ihren Blick auf die Welt zeigen und ihn einen Frieden zwischen den Völkern wertschätzen lassen.

Doch Mutter Naturs Botschafter war ein gerissener Geist. Es war ein Schattengeist, der sich innerlich schon längst von Mutter Natur abgewandt hatte, doch nicht mit den restlichen abtrünnigen Naturgeistern in den Süden geflohen war. Es war der Schwarze Herold, und nun erkannte er seine Gelegenheit, großes Leid anzurichten, und freudig tat er es. Er verdrehte die Melodie aus dem Mund von Mutter Natur und als er die dadurch korrumpierten Töne dem Urtroll vorsang, so wiegelte er den Urtroll gegen Mutter Natur auf.

Der Urtroll erhob sich brüllend vor Wut und sog die Kraft aller abtrünnigen Geister in sich ein. Er wuchs und wuchs, bis er so gigantisch war wie das Land selbst, sein Körper weit aus den Wolken ragte und sein linkes Horn gegen den roten Mond splitterte, von dem die Drachen einst gefallen waren. Der Urtroll stampfte mit seinem steinernen Fuß auf und das Graue Gebirge splitterte entdrei, mit tiefen Schluchten, die bis ins Reich der Zwerge vordrangen. Mit einem einzigen Schritt stapfte der Urtroll tief in das Herz von Mutter Naturs Reich, packte sie mit seiner gigantischen Faust und schleuderte sie auf die Erde, so heftig, dass ihr Land sich unter der Schockwelle in tausend Stücke teilte. Diese tausend Stücke drifteten vom Grauen Gebirge davon und wurden zu den tausend Inseln des Hadrischen Meeres.

Mutter Natur lag auf dem letzten Stück Land, welches noch in Verbindung zum Grauen Gebirge stand, und lag im Sterben. Doch selbst in diesem Moment schlug sie nicht gegen den Urtroll zurück, sondern hielt ihm ihre uneingeschränkte Liebe entgegen. Zum Glück erkannte der Urtroll da, dass die misstönige Melodie, die ihn so sehr in Rage versetzt hatte, nie von so einem puren Wesen hätten stammen können, und er verfluchte den hinterhältigen Schwarzen Herold, der ihn in Rage versetzt hatte.

Voller Reue versuchte der Urtroll, Mutter Natur zu retten. Er ließ seine Macht in sie fließen, und er schrumpfte und schrumpfte, bis er kaum grösser war als die größten Trolle, die er damals geformt hatte. So rettete der Urtroll ihr Leben, doch seine Tat des Hasses konnte nicht so einfach ungeschehen gemacht werden. Mutter Natur schlug ihre Augen nicht mehr auf. So sank der Urtroll neben Mutter Natur auf den Boden, bettete sie auf ein Bett aus Kräutern und sang ihr mit seiner urtiefen Stimme eine eigene Melodie, voller Trauer und Scham, aber auch voller Kraft und Hoffnung. Und die Zwerge kamen zusammen, und spielten Steinsang auf Flöten. Die friedlichen Naturgeister traten hinzu und ließen ihre Klagelieder ertönen. Auch unsere Vorfahren kamen daher und trugen ihre eigene Musik vor. Und alle diese Lieder vereinten sich zu einem riesigen Gebilde. Entgegen aller Erwartung überlagerten sie sich zu einem wunderschönen Klang und trieben jedem Zuhörer die Tränen in die Augen. Das war das Schlaflied von Mutter Natur.

Das Land summte und brummte mit der Kraft, die das Schlaflied von Mutter Natur mit sich trug. Dort, wo Mutter Natur auf einem Bett aus Kräutern schlief, wuchs ein prächtiger Baum, grösser und kräftiger als alle anderen des Waldes, und vollkommen von der Magie der Lieder erfüllt. Erst als seine Wurzeln Mutter Natur vollkommen umschlossen hatten, verklang die Musik der vereinigten Völker und sie gedachten ihr.

Unsere Vorfahren blieben zurück und versuchten, das Gehörte zu Pergament zu bringen. Es sollte ihnen nicht gelingen, den perfekten Klang des Schlafliedes auch nur im Ansatz zu reproduzieren, doch hörten sie nie mit dem Niederschreiben von Liedern auf. Als ihnen die Lieder auszugehen begannen, so notierten sie sich die Geschichte des Landes, und als auch diese immer ausgeschöpfter wurde, wandten sie sich schlussendlich den Legenden zu. Sie besiedelten den mächtigen Baum über Mutter Naturs Schlafstätte und füllten ihn mit Liedern und Wissen. Mit der Zeit sollten aus ihnen die Bewahrer des Baumes der Lieder werden. Doch noch war es nicht an der Zeit.

Nun, da Mutter Natur im Schlafen lag, gab es natürlich niemanden mehr, der die Natur unter Kontrolle hielt. Erdbeben tobten über das Land. Stürme entwurzelten die Bäume. Blitze spalteten Stein. Es war, als ob die Erde selbst für viele Jahre rebellierte.

So kam es in der Zeit der Not, dass viele Völker sich voneinander teilten und sich nur noch um sich selbst kümmerten. Die Trolle richteten weiterhin Verwüstung an, und quälten Mensch und Tier. Die Menschen wurden zu vielen kleinen Clans, die sich in alle Himmelsrichtungen verstreuten und ihre Herkunft vergaßen. Die Zwerge zogen sich mit den Drachen ins Gebirge zurück. Manche wandten sich gar von den Drachen ab und wurden zum zurückgezogenen Gebirgsvolk der Agren. Viele Naturgeister vergingen in Trauer und wurden wild, als Mutter Naturs Präsenz sie verließ. Arkteron, der Herr der Winde, hielt es in der Nähe von Mutters Schlafstätte nicht mehr aus und zog, von Trauer und Hass erfüllt, gen Norden. Dort ward er zum Herrn der Stürme und traf auf zwei weitere mächtige Wesen. Gemeinsam ersuchten sie, das Wunder des Lebens dem Urtroll nachzumachen und ihre eigenen Trolle zu erschaffen. Doch das ist eine Geschichte für ein anderes Mal.

Tiefe Schluchten zogen sich seit dem Anfall des Urtrolls durch das Graue Gebirge. Eine von ihnen wurde im Laufe der Zeit zu einem gigantischen Drachenhort, in dem sie ihre Körper ruhen ließen. Krahal-Schlucht wurde sie genannt. Eine andere hingegen wurde zu einem steinernen Mahnmal der Geschichte, denn der Urtroll selbst legte sich in dieser Schlucht nieder, als er von Scham und Zorn erfüllt ins Graue Gebirge zurückkehrte. Er wollte nur noch schlafen und nicht mehr wissen, zu welchem Unheil der Schwarze Herold ihn angestiftet hatte, und so bat er die Urahnin aller Agren, die alte Korn, ihn alles vergessen zu lassen.

Korn fühlte mit dem Urtroll und wirkte gemeinsam mit den Druiden der Agren einen Zauber, der den Urtroll in einen tiefen steinernen Schlaf sinken ließ. Sie ließ aber eine kleine Knochenflöte anfertigen, die den Urtroll wieder ins Reich der Wachen zurückholen würde, sollten die Agren je auf seine Hilfe angewiesen sein. Fortan wachten Korn und ihre Nachfahren über die Knochenflöte und die Schlucht, welche bald als Korn-Schlucht bekannt wurde.

Nur die abtrünnigen Naturgeister des Südens freuten sich ob des Machtvakuums im Lande. Sowohl Mutter Natur als auch der Urtroll schliefen nun, und so trat der Schwarze Herold erneut hervor. Er hatte beobachtet, wie der Urtroll die Trolle erschaffen hatte. Er kannte jetzt das Geheimnis des Lebens. Und er hatte vor, ein Volk zu erschaffen, welches in seinem Namen die Welt knechten solle. Doch alleine hatte er nicht die Kraft dazu. So versammelte der Schwarze Herold weitere Schattengeister, Erdgeister und Feuergeister, und er schöpfte aus ihrer Macht. Gemeinsam hatten sie die Mittel, eine Spezies zu erschaffen, größer und stärker als alle anderen Humanoiden, ohne Skrupel und ohne Gewissen. Das waren die Riesen, die das öde Reich des Südens für sich beanspruchten.

Doch konnte bald jeder sehen, dass die Riesen den Drachen bei weitem unterlegen waren, und dass sie dem Schwarzen Herold nicht gehorchten. Sie wandten sich von ihm ab und ihren eigenen Herrschern zu. Dunkle Hexerei entdeckten manche, und die Krahder nannten sie sich fortan. Nach und nach verloren immer mehr abtrünnige Naturgeister das Vertrauen in die Führung des Herolds, und da verließen sie ihn und zogen alleine durch eine Welt, welche immer geteilter und verrückter wurde. Selbst die Drachen und die Zwerge verwickelten sich in einen Unterirdischen Krieg, der das Graue Gebirge erschütterte und beide Völker an den Rande der Vernichtung bringen sollte.

Der Schwarze Herold blieb alleine zurück und hatte einen letzten Plan, die Anerkennung seiner Kreationen wieder zu erringen. Es zog ihn nach Krahal, den magischen Ort, den die Drachen erschaffen hatten. Krahal war erfüllt von Wut und Verzweiflung ob des wütenden Krieges, und der darin stehende weiße Baum verfaulte langsam unter den Qualen des Drachenvolks. Der Schwarze Herold erhoffte noch immer, das Reich des Südens zu einem Abbild von Mutter Naturs Reich zu machen, und so ergriff er eine Schwarze Wurzel des verfaulten Baumes, brach sie ab und brachte sie zu Nomion, einem gerissenen Krahder-Hexer. Dieser solle sie nähren, aus ihren einen mächtigen Schwarzen Baum wachsen lassen, daraus Kraft ziehen und dafür den Schwarzen Herold anbeten.

Doch wie schon der Schwarze Herold zu Mutter Natur wandte sich auch hier die Kreation von ihrem Kreator ab. Nomion ließ seine Schüler die Schwarze Wurzel aus Krahal in Krahd einpflanzen, und daraus wuchs der Schwarze Baum der Dunklen Hexerei. Nomion nutzte die Kraft des Baumes, um das ewige Leben des Schwarzen Herolds zu rauben. Der Schwarze Herold wurde zu einem Schatten seiner selbst degradiert, nur noch eine leere Hülle, welche wie ein Unwetter durch die Lande strich und das Böse unterstützte, wo immer sie konnte. Der Schwarze Herold verfluchte Mutter Natur dafür, ihn erschaffen zu haben, und richtete seine Wut auf sie.\bigskip

Hier enden die Berichte und Legenden unserer frühsten Vorfahren. Doch können wir natürlich folgern, was mit dem Schwarzen Herold geschah, während die Zeit verging.

Während der Unterirdische Krieg zwischen Zwergen und Drachen tobte, stahl Nomion Korns Knochenflöte und weckte den Urtroll. Dieser hatte in seinem tiefen, traumlosen Schlaf alles vergessen bis auf seine Wut, und so war er kaum mehr eine Urgewalt, die ohne Sinn und Verstand tobte und zerstörte. Eine Urgewalt, die Nomion mit Dunkler Hexerei zu lenken vermochte. Wäre Nomions Körper nicht kurz darauf vernichtet worden, wer wüsste, was er alles hätte anrichten können. Nur durch das ewige Leben des Herolds, das Nomion diesem durch den Schwarzen Baum gestohlen hatte, konnte ein Teil seiner Seele fortbestehen. Aber das ist eine Geschichte für ein anderes Mal.

Auch ohne Nomion und den Urtroll zogen die Krahder aus und raubten Menschen, Zwerge und Trolle aus den umgebenden Ländern, um sie sich selbst untertan zu machen. Der Schwarze Herold setzte seine Hoffnung in die Könige der Krahder, doch all diese wagten nicht, weiter gen Norden zu ziehen und Mutter Naturs Baum der Lieder zu verbrennen, denn auf dem Weg wachten noch immer die letzten Drachen. Nicht mehr viele von ihnen waren seit dem Verklingen des Unterirdischen Krieges noch übrig, und mit der Zeit waren auch von den Überlebenden viele gestorben oder versteinert. Doch noch immer waren sie eine zu große Gefahr, als dass die Krahder es wagen würden, sich ihnen direkt zu stellen.

Unter den letzten Drachen war Tarok, der stärkste der Drachen, der den Ausbruch des Krieges miterlebt hatte und vor Zorn schon beinahe wahnsinnig geworden war. So setzte der Schwarze Herold Hoffnung in Tarok und zeigte ihm, wie der Urtroll einst die Trolle erschaffen hatte. Tarok folgte den Weisungen seines Herolds. So formte Tarok in Krahal die Körper der gefallenen Drachen zu wüsten Geschöpfen, unsagbare Kreaturen mit zerstörerischen Kräften, in Anlehnung an die Zwerge und Menschen der Umgebung. Sie sollten in seinem Auftrag ins Zwergenreich einfallen und den Ausgang des Unterirdischen Krieg doch noch an die Drachen reißen.

Dazu kam es nicht. Taroks unsägliche Taten erschütterten den magischen Ort zutiefst. Rotes Blut spritzte aus dem verfaulten Baum Krahals. Die Geister der restlichen Drachen vergingen und ihre Körper stürzten seelenlos zu Boden. Einzig der wahnsinnige Tarok, für den Angst und Schmerz, Trauer und Verzweiflung nunmehin zum Lebenselixier geworden waren, konnte sich mit schierer Willenskraft an diese Welt klammern und die Erschütterung Krahals überstehen. Fortan musste er damit klarkommen, dass er es gewesen war, der den Untergang der letzten Drachen ausgelöst hatte, und der letzte Funke Freude in seinem Herzen erlag dem Schmerz.

Taroks Kreaturen verließen Krahal und scharten sich in den Bergen und Wäldern von Andor, um für den letzten Drachen zu kämpfen und zu töten. Und wie sie Krahal verließen, so drang Taroks Geist in die Trolle ein und auch sie folgten fortan seinem bösartigen Befehl. Und der schwarze Herold diente Tarok und trieb seine Kreaturen an.

Doch wie wir alle wissen, sollte Tarok sich nicht an die Wünsche des Schwarzen Herolds halten und nicht den Baum der Lieder fällen. Der letzte Drache wurde von seinem Rachedurst gegen König Brandur von Andor abgelenkt und ließ von den Plänen des Herolds ab. Am Ende wurde er von den Helden von Andor überwunden. Die Überreste seines Körper wurden die Narne hinuntergespült, und auch wenn seine Kreaturen noch immer hasserfüllt unser Land belagern, so hoffen wir doch, dass selbst Tarok im Tode das offene Meer erreichen konnte und, nachdem er dort seine bösartigen Taten besehen und berichtig hat, wie lange es dafür auch brauchen möge, im ewigen Glück seinen Frieden finden kann.

Der Wunsch des Schwarzen Herolds ging nicht in Erfüllung. Das Böse in Andor mehrt sich nicht, sondern wird stetig schwächer. Stetig mehrt sich hingegen Mutter Naturs Macht, selbst im Traum vermag sie die Geschicke dieser Welt schon wieder zu beeinflussen und uns Kraft zu verleihen. Und wie Mutter Natur sich in ihrem Schlafe heilt, so heilt auch das Land von dem Übel der abtrünnigen Naturgeister. Eines Tages wird auch der Schwarze Herold seine Verfehlungen einsehen und auf die Seite des Lichts treten. Und eines Tages wird sich so viel Frieden verbreitet haben, dass Mutter Natur vollends erwachen und ihre Rolle als Hüterin des Landes wieder aufnehmen kann.

Wir wissen nicht, ob sie in einem Jahrzehnt erwachen wird, oder in einem Jahrhundert, oder gar in einem Jahrtausend. Aber wir glauben fest daran, dass Mitglieder unseres Ordens sie erwarten werden, wenn sie sich wieder erhebt. Bis dahin verweilen wir Bewahrer vom Baum der Lieder stets hier im Wachsamen Wald und schützen ihre Ruhestatt. Wir wahren alles Wissen, das wir können, ohne groß einzugreifen. Wie sie es uns gelehrt hatte. Und wir können es kaum erwarten, ihr berichten zu dürfen, was alles während ihres Schlafes vorfiel."











\begin{chapterbox}
    \chapter{Die Agren, der Schildzwerg und der Feuerdrache (2020)}
    \label{Die Agren, der Schildzwerg und der Feuerdrache (2020)}
    \az{Jahr –921}

    \begin{center}
        einst versteckt unter den Codes 53074 und 54154
    \end{center}
    
    Die Agren Gouri erlebt mit, wie der Schildzwerg Prip vom Feuerdrachen Sagrak angegriffen wird. Sie trifft eine mutige Entscheidung, die sie direkt in den Unterirdischen Krieg zwischen Zwergen, Drachen und Trollen hineinzieht. Und da lauern eine uralte Zwergenfeste, ein Meisterschmied mit seinen mächtigen Schilden, der magische Ort Krahal und sogar der gefährlichste Drache aller Zeiten.
\end{chapterbox}

\section{Gouri, Prip und Sagrak}

\az{Jahr –921}

„DRACHEE!“

Der verzweifelte Schrei hallte durch die Täler des Grauen Gebirges und schreckte allerlei Kleingetier auf, welches schleunigst das Weite suchte, selbst wenn es die Warnung nicht verstehen konnte. Die wenigen Agren, die sich ausserhalb ihrer sicheren Höhlen aufhielten, schreckten auf. Sie rannten allerdings nicht blindlings davon, sondern richteten ihre Augen furchtsam gen Himmel, blickten nach oben, suchten den kleinen schwarzen Punkt, welcher Feuer und Tod ankündigen würde.

Gouri, eine vergleichsweise junge Agren (sie hatte doch erst kürzlich ihren vierzigsten Jahrestag gefeiert), versuchte ebenfalls, die Gefahr im Himmelszelt zu erkennen. Den Kopf in den Nacken gelegt, starrte sie nach oben, direkt in die blendende Sonne. Sie senkte ihren Blick auf die übrigen grünhäutigen Angehörigen des friedlichen Bergvolkes, die ein ganzes Stück von ihr entfernt (leider) die Pflanzenwachsstätten umgegraben hatten. Normalerweise mochte Gouri die Zeit alleine im Freien, wo man in der Mitte einer Wiese sitzen und über den Lauf der Welt nachsinnieren konnte. Nun stand sie aber alleine auf weitem Felde, und der Drache könnte sie von allen Seiten überraschen. Furcht breitete sich in ihrer Magengegend aus.

Da! Ein weiterer Schrei ertönte, diesmal von einem der Agren auf den Pflanzenwachsstätten! Die übrigen Agren rannten vereint los in Richtung einer der rettenden Höhlen, sorgsam darauf achtend, niemanden zurückzulassen. Niemanden ausser Gouri, von der sie nicht einmal wussten, dass sie sich hier draussen aufhielt. Die verängstigten Agren rannten vor Gouris aktueller Position weg. Diese war sich bewusst, was das zu bedeuten hatte: Der Drache befand sich irgendwo hinter ihr!

Innerlich schon mit ihrem Leben abschliessend, drehte sich Gouri um, um einen atemberaubenden Anblick zu erleben. Ein riesiges Reptil, dessen schwarze Schuppen im hellen Sonnenlicht glänzten und glitzerten, brach im Sturzflug durch eine nahe Wolke hindurch. Was zunächst ein kleiner Fleck am Himmel gewesen war, wurde rasant grösser, bis Gouri sogar die gebogenen Zähne im offenen Maul erkennen konnte. Alle Instinkte in ihr schrien sie an, sie solle losrennen, sich vor dieser wandelnden Naturkatastrophe verstecken, irgendetwas tun.

Gouri wusste es besser. Sie blieb zitternd stehen und versuchte, sich zu fokussieren.

In ihrem Kopf bereitete Gouri sich bereits auf den Mentalen Schrei des Drachen vor, welcher jeden Moment ertönen konnte. Sie konzentrierte sich auf ein Kinderlied, welches ihre Mutter ihr jede Nacht vorgesungen hatte, als die Welt noch heil gewesen war – ehe der Meisterschmied Kreatok ein Massaker unter den Drachen angerichtet hatte und der Krieg ausgebrochen war, welcher seitdem das Graue Gebirge plagte.

So sang Gouri innerlich:

\textit{‚Eia, Heja und Naidoa, der Nordwind, der Ostwind und Westwind sind da...’}

Da sandte der angreifende Drache auch schon seinen Mentalen Schrei gegen alle Wesen in seiner Nähe aus:

„MÖRDER! DIEBE! VERRÄTER! VERROTTEN MÖGT IHR, IHR UND ALLE, DIE EUCH ZU BESCHÜTZEN VERSUCHEN!“

\textit{‚Eja, Heja und Naidoa, die Winde, die sausen und brausen durch Tal...’}

Solange Gouri sich auf den Gutenachtreim konzentrierte und so den geistigen Angriff des Drachen verdrängte, gelang es ihr, wenn auch schwankend, auf den Beinen zu bleiben. Vielen anderen Wesen in der Nähe ging es allerdings nicht so. Insektenhirne dachten zu wenig, um gross vom Mentalen Schrei betroffen zu sein, doch Vögel fielen links und recht wie von Steinen getroffen vom Himmel, unfähig, viel mehr zu tun als das Kreischen des Drachen zu erleiden. Dann waren da die Kaninchen, die Rehe, die Füchse, die Wölfe der Gegend, die sich am Boden wälzten und verzweifelt versuchten, etwas zu verscheuchen, was man nicht verscheuchen konnte. Und dann gab es noch den armen gepanzerten Schildzwerg, welchen Gouri in jenem Augenblick entdeckte.

Gouri wusste nicht, ob es daran lag, dass der Zwerg nie gelernt hatte, eine mentale Attacke abzuwehren, oder ob der Drache sich in seinem Schrei ganz besonders auf ihn fokussierte, aber der Zwerg steckte das Ganze erheblich schlechter weg als Gouri. Er war auf die Knie gefallen und hatte sich dann zusammengerollt, unregelmässig atmend, die Zähne zusammengebissen und die Hände auf die Ohren gepresst – doch was konnte das schon ausrichten, wenn der Schrei nicht durch die Ohren in seine Wahrnehmung trat?

„LUMPENGESINDEL! DAS IST FÜR NEHAL! DAS IST FÜR SERAFIM! DAS IST FÜR NERUPAK! SIE ALLE SIND GEFALLEN, EURETWEGEN! IHR BLUT KLEBT AN EUREN DRECKIGEN HÄNDEN UND DUNKLEN SCHILDEN!“

Einen Mentalen Schrei aufrechtzuhalten kostete Unmengen an Energie, lange konnte der Drache dies bestimmt nicht mehr durchstehen. Das war aber auch nicht sein Ziel. Noch immer im freien Fall breitete er seine ledernen schwarzen Flügel aus – der abrupt darin gefangene Wind ruckte schmerzhaft an seinen Gelenken – fing seinen Fall ab und wehte dabei ein ganzes Dutzend kleinerer Bäume um. Schnell jetzt, ehe diese Ratte von einem Wicht sich wieder fassen konnte! Der Boden erzitterte unter seiner Landung. Der glorreiche schwarze Drache – Sagrak war sein Name – richtete seinen Schlangenhals auf und spürte verzückt das vertraute warme Brodeln, wie es von seinem Magen langsam seinen Hals hinaufkletterte und seinen Schlund erreichte. Sagrak blähte seine Nüstern, öffnete seinen Rachen und grinste in Antizipation dessen, was er diesem mickrigen Spornenfrass vor ihm gleich antun würde. Dann spie er einen Strahl puren Drachenfeuers, golden glänzende Flammen, denen nichts und niemand standhalten konnte.

Gouri konnte deutlich die bösartige Freude fühlen, die in Wellen vom Feuerdrachen ausströmte, und sie war sich sicher, dass der Schildzwerg dies auch spürte. Den Mentalen Schrei hatte der Drache – Sagrak? – offenbar schon gemeistert, aber die Fähigkeit, seine Emotionen zu beherrschen und in seinem eigenen Kopf zu verwahren, war ihm noch nicht gegeben. Er musste ein Jungtier sein. Darauf deutete auch seine Grösse hin – selbst der Zwerg reichte ihm bis an die Schulter. Nichtsdestotrotz war auch ein Jungdrache eine nahezu unüberwindbare Gefahr.

Unter enormer Anstrengung klaubte der immer noch am ganzen Leib zitternde Zwerg jetzt einen angeschlagenen Buckelschild hervor und wuchtete ihn in letzter Sekunde zwischen seinen ungeschützten Körper und den Flammenwall, welchen der Drache auf ihn spie. Drachenfeuer war heisser als jedes andere natürlich vorkommende Feuer dieser Erde, sodass auch der beste Schild in diesem Inferno bald vernichtet gewesen wäre. Und der Schild, hinter welchem der Schildzwerg sich verbarg, war wahrlich nicht der beste Schild. So verbrannte das Holz und das Eisen schmolz buchstäblich dahin, während der arme Schildzwerg gezwungen war, unten am Boden zu bleiben und zu hoffen, dass der Schild ihn noch lange genug schützen würde. Lange genug, bis der Drache wieder Luft holen musste. Lange genug, um am Leben zu bleiben. Gut sah es nicht aus. Die Haare an seinen Armen waren schon längstens verglüht, und sein mächtiger brauner Bart würde als nächstes folgen, das wusste er. Gouri sah, wie er die Augen schloss und ein Stossgebet an die Mutter des Steins sandte. Wartete er auf das Ende dieser elenden Qual, welches auch sein Ende sein würde?

Da klappte Sagrak seinen Schlund zu und der Feuerstrahl versiegte. Schwefelgeruch lag in der Luft. Tief durchatmend stand der Drache da, und sowohl Gouri als auch der Schildzwerg nahmen seine Erschöpfung und Verwirrung wahr – er hatte eindeutig seine Feuerkraft überschätzt. Das sollte aber nicht heissen, dass er des Zwerges Lebenslicht nicht immer noch durch einen mächtigen Hieb mit einer seiner Pranken einfach auslöschen könnte.

Gouri beobachtete gebannt und immer noch zittrig die Szene, die sich vor ihr abspielte. Sie war sich uneinig, was sie nun tun sollte. Jetzt war ihre Gelegenheit, sich umzudrehen und in die sichere Haushöhle zurückzukehren. Zurück zu ihren Eltern, die sich bestimmt schon riesige Sorgen machten. Die ihr wahrscheinlich verbieten würden, innerhalb der nächsten drei Sonnenzirkel das Tageslicht wieder zu betreten. Gouri seufzte.

Dieser Schildzwerg... er schuldete ihr nichts, denn es waren die Zwerge gewesen, die ihr Bündnis mit den Drachen gebrochen und so nebst der Zerstörung vieler ihrer so prächtigen Bauten auch Krieg und Unheil über das einst so friedliche Graue Gebirge gebracht hatten. Eigentlich sollte Gouri sogar froh sein, dass es bald einen Zwerg weniger auf dieser Erde gab. Aber sie konnte nicht. Der Mut dieses Kriegers faszinierte sie: Er hatte einen Drachen auf sich gehetzt, einen Mentalen Schrei erlitten, und er lebte immer noch. Er hatte seine Waffen verloren, sein Schild war verkohlt und sein eleganter Bart hatte sich im Flammeninferno von eben gerade verabschiedet. Dennoch stand er vorsichtig auf, aufrecht dem sich immer noch von der Anstrengung erholenden Drachen entgegenstellend, und blickte ihm entschlossen und ungerührt ins echsenartige Angesicht.

In ihrer Kindheit hatte Gouri stets fasziniert den Legenden von Zwergen und Drachen, von mächtigen Schilden und den Landen jenseits des Grauen Gebirges gelauscht. Und die Heroen in diesen Geschichten liefen nicht von einem unfairen Kampf davon, sondern setzten sich für Gerechtigkeit ein. Und dieser Kampf zwischen Schildzwerg und Drache war nicht gerecht.

In diesem Moment entschloss Gouri sich: Sie würde nicht wegrennen. Sie würde wenigstens diesen einen Streit zwischen Zwerg und Drache zu schlichten versuchen! Das fühlte sich richtig an! Und so richtete Gouri sich auf, ihr verfilztes Agrenhaar wehend im heissen Wind, und war schon kurz davor, zwischen den Schildzwerg und den Feuerdrachen zu springen, um diesem Unsinn Einhalt zu gebieten...

...da fasste sie sich wieder. Einfach so zwischen einen wütenden Drachen (insbesondere ein Jungtier) und einen Schildzwerg zu springen, war kein gescheiter Weg, sein Leben zu verlieren. Gab es vielleicht andere Wege, wie sie dem ungeschützten Schildzwerg helfen konnte?

So jung, wie Sagrak, der Drache, war, wollten seine Eltern bestimmt nicht, dass er sich alleine auf Zwergenjagd befand. Auf einem aufmüpfigen Ausflug war er also! Wahrscheinlich kannte er sich nicht im Geringsten mit der realen Welt aus. Er hatte wohl davon geträumt, bei seiner kleinen Exkursion eine Horde böser Zwerglein zu finden und problemlos zu flambieren, „ihrem gerechten Ende zuzuführen“, wie die Drachen es in ihren Gutenachtgeschichten ihren Kindern beibrachten. Und nun wunderte er sich, wie es sein konnte, dass er den Schrei genutzt und seinen Feueratem verbraucht hatte, während dieses dreckige Wichtlein vor ihm noch lebte. Viel brauchte es nicht mehr, damit Sagrak verunsichert das Weite ziehen müsste. Allerdings nichts, was die schwache Gouri selbst tun könnte. Aber vielleicht jemand anderes?

Gouri blickte wild um sich. Die meisten Wesen in ihrer Nähe waren geflohen, doch da! Dort im Felsen hinter Sagrak konnte sie einen Eingang zu einem Arpachenbau erkennen!

Die Arpachen, riesige Insektoide, hatten mit ihren mächtigen Kiefern einst Gänge durch das Graue Gebirge gegraben und wie die Zwerge ein System aus unterirdischen Gängen gebaut – wenn auch weit weniger elegant. Im Gegensatz zu den inzwischen aus taktischen Gründen zugeschütteten Eingängen zum Zwergenreich Cavern waren die Höhlen, die die Gänge zu den Arpachenbauten markierten, weder getarnt noch verschlossen, und so nutzen verschiedenste Spezies sie gerne als Aufenthaltsort.

Allen voran natürlich die Agren, die Gouri in so einer Situation nicht gross helfen könnten.

Ein Höhlenwicht wäre auch nicht wirklich hilfreich; diese schleimigen Gestalten waren kaum stärker als ein Agren und griffen bloss an, wenn ihr Gegner ihnen den Rücken zuwandte.

Eine Sporne wäre schon etwas besser; diese Spinnenwesen konnten zwar nichts gegen einen Drachen ausrichten, würden aber, wenn provoziert, wütend aus ihrer Höhle schiessen und durch aufgestellte Vorderbeine versuchen, ein möglich einschüchterndes Bild abzugeben, was vielleicht schon genug war, um Sagrak zu vertreiben.

Ein Hornbär wäre noch effektiver; diese gehörnten Riesenbären konnten es tatsächlich mit einem jungen Drachen aufnehmen, wenn sie aus ihrem Winterschlaf geweckt wurden.

Der Hauptgewinn wäre aber eine Trollhöhle. Die Trolle waren mit Abstand die stärksten flugunfähigen Wesen im Grauen Gebirge und eine Horde von ihnen hatte sogar schon einmal einen ausgewachsenen Drachen erlegt. Sie waren nicht mit hoher Intelligenz gesegnet – den besiegten Drachen hatten sie verspeist, ohne das Fleisch auch nur anzubraten zu versuchen, und diejenigen Trolle, die nicht an die Lebensmittelvergiftung ihr Leben gelassen hatten, waren kurz daraufhin von den übrigen Drachen dabei erwischt worden, wie sie triumphierend aus den skeletalen Überresten des Gefallenen eine Hütte zu bauten versuchten – aber man brauchte keine hohe Intelligenz, um einen kleinen Drache wie Sagrak zu verschüchtern.

Und so klaubte Gouri einen scharfen Stein aus dem felsigen Boden, wog die Distanz zur mysteriösen Höhle kurz ab und warf ihn dann energisch in die Richtung des Höhleneingangs.

Sie verfehlte um Längen und traf Sagrak am Kopf.

Rückblickend war das vielleicht sogar etwas Gutes, da es dem Schildzwerg einige wertvolle Sekunden schenkte. Aber in dem Moment, als Gouri erblickte, wie ihr Stein den Drachen an der Schläfe traf und dieser blitzschnell den Kopf drehte und sie aus seinen glühenden blutroten Augen direkt anstarrte – in diesem Moment hüpfte Gouris Herz in ihre Mooshose, und sie konnte nichts mehr, als „Spornendreck“ zu flüstern und innerlich mit ihrem Leben abzuschliessen. Dann fasste sie sich wieder, und als der Feuerdrache langsam auf sie zuschritt, den Kopf auf dem langen Hals weit erhoben (sie wusste, dass er sich so auf einen erneuten Feuerstrahl vorbereitete, und dieser wäre ihr Ende und das des Schildzwergs gewesen), der lange stachelbesetzte Schwanz zuckend vor Vorfreude, warf Gouri voller Adrenalin einen zweiten Stein.

Der spitze Gesteinssplitter drehte sich mehrmals um die eigene Achse, beschrieb eine elegante Flugparabel über Sagraks Kopf hinweg und verschwand im Dunkel der Höhle, wo ein leises Klack anzeigte, dass er ein Ziel gefunden hatte.

Ein tiefes, wütendes Brüllen erklang. Wie viel Glück konnte eine Agren haben? Es handelte sich tatsächlich eine Trollhöhle!

Sagraks Augen verengten sich, und er wandte sich ebenso blitzschnell, wie er seine Aufmerksamkeit auf Gouri gerichtet hatte, wieder von ihr ab. Gouri sah in ihm nicht mehr das feuerspeiende, bösartige Monster, dass sie noch eine Sekunde zuvor in ihm gesehen hatte. Stattdessen war er bloss ein armes Jungtier, hineingeführt in einen Konflikt, welchen es nicht begonnen hatte und welchen es nicht beenden konnte. Furcht stand in Sagraks Blick und seine Furcht brach aus seinem Kopf heraus, erfüllte Gouri und den Schildzwerg.

Dann trat der Troll aus der Höhle.

Es war ein furchterregendes Monstrum, Beine so dick wie Eichenstämme, Muskeln wie ein Bär und riesige gewundene Hörner, für die es keinen Vergleich im Tierreich gab – nicht mal ein Hornbärenhorn konnte mit der Stabilität und Tödlichkeit eines Trollhorns auch nur annähernd mithalten. Jetzt brüllte der Troll, sich immer noch das Auge reibend, welches Gouri anscheinend getroffen hatte: „WER WAGEN WERFEN DIESEN STEIN HIER?!“

Trolle brauchten nicht viele Worte, damit man sie verstand. Der breite Troll blickte sich wütend um und fokussierte seinen Blick natürlich auf den grössten potentiellen Gegner im Raum, Sagrak.

Ungeachtet der Tatsache, dass Sagrak mit seinen Tatzen wahrscheinlich nicht mal einen grossen Schild greifen konnte, also definitiv nicht den Steinbrocken geworfen hatte, stapfte der Troll furchtlos auf den Drachen zu und brüllte: „DIESER TROLL HIER KNACKEN DIESEM DRACHEN DORT SEIN HALS!“

Die beiden Riesenwesen machten Augenkontakt. Weisse, pupillenlose Augen trafen auf loderndes Rot. Dann war der Troll bei Sagrak angekommen. Nun spie Sagrak Feuer, doch der Troll war schneller und drückte ihm die Schnauze zu. Ein klein wenig Glut floss dem prustenden Drachen aus der Nase, und der Troll brüllte auf vor Schmerz, als es über seine Hand floss und seine Haut versengte. Er hielt jedoch weiterhin fest. Sagrak versuchte verzweifelt, sich loszustrampeln. Er wandte und drehte sich, und entschlüpfte dem Griff des Trolles. Seine Flügel bereits entknittert, sprang er hoch in die Luft, um entrüstet davonzufliegen.

Gouri war damit zufrieden. Sobald der Drache in den fliegenden Zustand gekommen war, würde ihm keine Gefahr mehr durch den Troll drohen. Sagrak würde verstört das Weite suchen, und der Troll würde ihm folgen, von der einfältigen Hoffnung erfüllt, ihn immer noch zu erwischen. Sie und der Schildzwerg wären in Sicherheit. Ende gut, alles gut.

Gouri stolperte den Hang hinunter auf den sich immer noch hinter Schildüberresten duckenden Zwerg zu und näherte sich ihm vorsichtig. Seine Augen weiteten sich, als er sie als Angehörige der Agren erkannte, und sein Blick schien sie anzuflehen, die Dummheiten sein zu lassen und sich nicht seinetwegen in Gefahr zu begeben. Gouri ignorierte sie und wollte ihm schon aufhelfen...

...da packte der Troll den wegfliegenden Sagrak am dornigen Schwanz und zog diesen zurück auf den Erdboden, wo er ungeschickt aufklatschte. Die Erde erbebte, sodass selbst Gouri Mühe hatte, auf ihren stämmigen Agrenbeinen stehen zu bleiben. Der Troll überquerte den Weg zu Sagraks Brustkorb mit einem gewaltigen Sprung und landete auf ihm, seine massigen Beine links und rechts auf die zerschmetterten Flügel der Echse stellend. Sagrak schrie auf. Sein Schrei ging allen Anwesenden durch Mark und Bein. Dann schnappte er mit seinen scharfen Zähnen nach der Nase des Trolles. Dieser lachte nur auf, klemmte Sagraks Schnauze erneut fest. Sagrak stach mit seinem Schwanz nach dem Rücken des Trolls. Dieser grunzte nur amüsiert und drückte noch fester zu.

Die kämpfenden Giganten waren Gouri definitiv zu nahe – wenn Sagrak eine ungeschickte Bewegung mit seinem Dornenschwanz machen würde, dann wären sowohl sie als auch der Schildzwerg Spiesschen. Ein Stossgebet sendend an die Geister des Waldes, des Steins und der Lüfte, näherte sich Gouri dem angekokelten Schildzwerg, welcher sich inzwischen erhoben hatte.

„Folge mir!“, flüsterte sie eindringlich. Der Zwerg tat wie geheissen und humpelte ihr nach, weg von den kämpfenden Biestern. Sie mussten zu einem sicheren Agrenbau, wo den Schildzwerg hoffentlich Verpflegung und Arzneien erwarten würde, und zwar schnell!

Als sie glaubte, genug weit weg von Troll und Drache zu sein, warf Gouri einen Blick zurück.

Ihr gefiel nicht, was sie erblickte.

Sagrak wand sich ein letztes Mal vergeblich. Verzweifelt sandte er einen leisen Mentalen Schrei aus, welcher nicht einmal den angeschlagenen Schildzwerg von den Füssen holte. Dann blubberte er einen letzten Rest Lava aus seinem Magen heraus und versuchte, diesen dem Troll anzuspucken. Nichts half ihm aus seiner misslichen Lage. Sein Leben zog vor seinem inneren Auge vorbei, und so offen, wie sein Geist momentan war, zog sein Leben auch vor den Augen von Agren, Schildzwerg und Troll vorbei. Kurze Eindrücke von verschiedensten Momenten schossen auf sie ein: Der erste Riss in Sagraks Eierschale, durch den das helle Licht der Aussenwelt schien; seine erste Jagd; seine erste Flamme, wie sie einen ganzen Baum mit nur einem Niesser in Brand setzte; ein Bad im See am Grunde der Krahal-Schlucht mit einer Drächin, die er sehr mochte; die reine Wut, als er von Kreatoks Verrat gehört hatte; die Genugtuung, mit welcher er Dutzende Zwerge durch die Winterburg jagte; die Furcht, als direkt neben ihm sein bester Freund mit einem riesigen Stein aus einem Wurfgeschoss der Riesen vom Himmel geschlagen wurde; die Rachegefühle, die er verspürte, als er die Zera tief im Lavafluss Enran begrub, gemeinsam mit den zwei Riesen, die das Katapult bedient hatten... dann wurde es still.

Äusserlich hatte sich am Drachen nichts verändert, doch das Feuer in seinen Augen war erloschen, und er wehrte sich nicht mehr. Sein Körper wurde schlaff und seine Klauen bohrten sich nicht mehr länger in die grobe Haut des Trolls, sondern fielen kraftlos auf die Seite.

Die letzten Eindrücke, die die Anwesenden aus Sagraks Geist empfangen hatten, waren ein Bild und ein einziges Wort gewesen.

Das letzte Bild, welches dem Drachen durch den Kopf geschossen war, war ein riesiger schwarzer Baum aus Stein, über und über mit Edelsteinen verziert, in einer Art Höhle, in die wohl die ganze Winterburg doppelt und dreifach gepasst hätte. Der Steinbaum war umgeben von einem sanften blauen Leuchten, und dutzende von Drachen sassen auf seinen Ästen und unterhielten sich. Weiter unten sah man kleine Kreaturen, die mit ihren Hornklauen die Rücken und Flügel einiger seufzender Drachen massierten. Eine friedliche Stimmung herrschte.

Das letzte Wort, welches der Drache gedachte hatte, war ‚Krahal’.

Gouri hatte den Verdacht, dass Sagraks Geist nicht mehr in diesem Körper weilte, als der Troll triumphierend den Hals des leblosen Drachen schnappte und mit einem breiten Grinsen einmal um die eigene Achse drehte, sodass die Wirbel nur so knackten. Ihr wurde übel und sie musste sich von der Szenerie abwenden. Ein Schleifgeräusch verriet ihr, dass der Troll, dessen Kampfeslust soeben befriedigt worden war, gerade stolz seinen Fang in die Höhle zurückzerrte, um vor seiner Familie damit anzugeben. Mögen sie doch dran ersticken!

Gouri war selbst über ihre Wut auf den Troll überrascht, die sie verspürte. Immerhin hatte der Troll ihr Leben und das des Zwerges gerettet. Aber er hatte auch ein Leben beendet, und er freute sich daran. Das hätte der Drache zwar auch getan, aber dieser hatte immerhin eine gute Begründung für seine Taten. Nun gut, gut war seine Begründung ja nicht, aber wenigstens verständlich. War das Verhalten des Trolles etwa nicht verständlich? Verwirrt liess Gouri sich zu Boden fallen, während ihre Gedanken weiter kreisten.

Jetzt war der Schildzwerg an ihrer Seite. Seinen gepolsterten Helm mit den vielen Runenverzierungen hatte er abgesetzt. Die Überreste seiner Rüstung, sein Bart und sein Haupthaar waren grösstenteils weggebrannt – unter seiner Rüstung erkannte Gouri eine isolierende Schicht schwarzer Stoffstreifen, welche er sich umgebunden hatte – doch strahlte er immer noch etwas Erhabenes aus, wie er vor ihr stand und sprach: „Der Mutter des Steins sei Dank. Nein, Euch sei Dank! Was Ihr getan habt... ich dachte, die Agren mischen sich nicht ein in... egal, ich will nur ausdrücken... wie kann ich das nur je wiedergutmachen?“

Gouri war immer noch schockiert von der Gewalt, die sie gerade erlebt hatte. Sie hatte die Geschichten gehört und einige Drachenangriffe aus der Ferne miterlebt, doch dies hier war ein Erlebnis ganz neuen Ausmasses gewesen.

„Diese Trolle sind eine elende Plage!“, stiess sie hervor. Das machte keinen Sinn, der Troll hatte soeben ihr und des Zwerges Leben bewahrt. Sie sollte nicht wütend auf ihn sein. Aber die sinnlose Grausamkeit, die das Wesen nonchalant an den Tag gelegt hatte, weckte überraschend viel mehr Emotion in ihr als die Grausamkeit, die die Zwerge und Drachen einander aufgrund ihrer langjährige Feindschaft zeigten. Nicht, dass der Krieg sinnvoll gewesen wäre.

Der Zwerg erwiderte: „Ihr meint, das Graue Gebirge hätte eine Trollplage? Ich komme gerade von einem Treffen aus dem Trollland in Norden. Die dort lebenden Hünen überragen den grössten Schildzwerg um über zwei Köpfe und leben dennoch im wahrsten Sinne des Wortes in den Baumwipfeln, weil auf dem Boden zu viele Angriffe durch Trolle drohen. Das ist doch verrückt! Das nenne ich eine wahre Trollplage!“

Er klang beinahe enerviert. Dann fiel ihm auf, dass so eine Reaktion nicht förderlich war, und fügte hinzu: „Verzeiht bitte. Es war ein... aufregender Tag.“

Gouri schwieg. Sie wusste nicht, was zu sagen war. Sie wusste nicht, wie es jetzt weitergehen sollte. Sie hatte stets davon geträumt, selbst Teil einer Sage zu werden und einen jungen Zwergenprinzen zu treffen. Aber die Realität war auf einmal überwältigender als ihre Fantasie.

„Ich bin Gouri“, sprach die Agren schlicht.

„Prip, Sohn des Aigar, Bote des Eisernen Stuhls von Cavern und Hüter des Casamatucs der Drei Wasser, zu Euren Diensten“, erwiderte der Schildzwerg mit einer angedeuteten Verbeugung.

Gouri wünschte sich, sie hätte auch einige Titel zu nennen. Sie streckte ihm ihre grüne Hand zur Begrüssung hin und Prip schüttelte sie mit seinen Eisernen Handschuhen. Gouri schrie auf, denn die Handschuhe waren immer noch heiss vom Odem des Feuerdrachen (offenbar waren sie innerlich isoliert, sodass Prip die Wärme nicht verspürt hatte). Prip entschuldigte sich vielmals. Beide tauschten ein nervöses Lachen aus. Dann besann Gouri sich: Ihr ging es immer noch prima, doch Prip war, auch wenn er es sich nicht anmerken lassen wollte, verwundet und verbrannt, und brauchte dringend einen Heiler. Sie mussten zurück zu den anderen Agren.

„Prip, Dir... Euch geht es nicht gut. Ich kenne den besten Heiler des Grauen Gebirges und kann Euch zu ihm führen. Folgt mir!“

Prip blickte sie kurz stolz an, als wollte er sie herausfordern, seinen gesundheitlichen Zustand nicht zu kritisieren. Dann sprach er: „So gerne ich auch die Dienste dieses Heilers in Anspruch nehmen würde, nehme ich doch an, dass er den Schildzwergen ein wenig zurückhaltender gegenübersteht als Ihr, und ich will die Gastfreundschaft der Agren nicht überstrapazieren. Zudem muss ich dringend zu der Zwergenfeste „Die Drei Wasser“, dem Fürsten von Cavern Bericht erstatten gehen.“

Gouri prustete beinahe los. Prip würde in seinem jetzigen Zustand doch keine drei Baumlängen weit kommen. Allerdings hatte er recht – die übrigen Agren würden sich nur ungern um einen Schildzwerg kümmern wollen, erst recht nicht, wenn dieser gerade erst eine Auseinandersetzung mit einem Drachen gehabt hatte. Auch wenn es die Drachen waren, die zwischendurch den einen oder anderen unvorsichtigen Agren brateten, so waren es in ihren Augen die Zwerge, die die Angriffe dieser Tiere provoziert hatten, und so waren Zwerge keine gern gesehenen Gäste.

Gorui fasste einen Entschluss: „In diesem Fall werde ich Euch begleiten. Unterwegs stossen wir bestimmt auf einige Büschel Wolfskraut, dann kann ich Euch immerhin mit einer behelfsmässigen Salbe aushelfen.“

„Wie tief soll ich denn noch in Eurer Schuld stehen?“, ächzte Prip.

Gouri bemerkte, dass er nicht protestierte. Sie verspürte einen Anflug kindlicher Freude. Sie würde diesen Zwerg zu den Hallen der Zwergenkönige begleiten! Sie würde auf ein Abenteuer gehen!

Erst als die beiden die Alte Zwergenstrasse erreichten und eine weitere Höhle passierten, fiel Gouri ein, dass die anderen Agren wohl gerade umkamen vor Sorge um sie.






\newpage
\section{Der Sturm auf die Drei Wasser}

„Obacht! Da vorne ist etwas!“, zischte Gouri zu Prip. Prip verharrte, wo er war, und packte die verkohlten Überreste seines Schildes fester. Schildzwerge, dachte Gouri belustigt, immer auf ihre Waffen fixiert. Dieses verbrannte Stück Holz war die einzige Waffe, die ihm geblieben war, und nun hielt er daran fest, als hinge sein Leben daran.

„Ein Lebewesen?“, wisperte Prip angespannt. Zwerge wie er konnten im Dunkel der Nacht keine fünf Hasensprünge weit blicken, wenn sie nicht eine ihrer genialen Kerzenbrillen trugen. Gouri richtete ihre erheblich besser an die Dunkelheit angepassten Agrenaugen nach vorne, in die Richtung, aus der das Geräusch gekommen war. Im dichten Nebel konnte sie aber auch nicht viel mehr einen grossen Schemen ausmachen, der sich kaum bewegte – natürlich ausgerechnet auf der anderen Seite der Hängerücke, die sie und Prip überqueren mussten.

„Es bewegt sich kaum, aber es bewegt sich. Das wird ein Lebewesen sein.“, flüsterte Gouri zurück.

„Was kannst du erkennen?“

„Es ragt aus dem Nebel heraus. Es ist gross.“

„Ein Troll bei einem Mitternachtsspaziergang vielleicht? Oder ein Drache?“

„Auf jeden Fall ist es kein gutes Zeichen.“

Gouri bedachte ihre Optionen. Sie konnten versuchen, die Brücke zu überqueren, aber das würde das Wesen auf jeden Fall auf sie aufmerksam machen. Und zu warten, könnte sinnlos sein, falls sich der Schemen letzten Endes nur als seltsame Felskonstruktion herausstellen sollte. Mühsam unterdrückte sie ein Gähnen. Agren brauchen ihre dreizehn Stunden Schlaf, um fit und munter zu sein, und Gouri war momentan weder das eine noch das andere.

„Gibt es andere Wege zu den Drei Wassern?“, fragte sie Prip.

„Es gibt einen Pfad der Korn-Schlucht entlang bis zu einem Felsvorsprung, von wo aus wir zum Noswald springen können. Aber dieser Umweg würde uns das einen ganzen Tag kosten. Mal ganz abgesehen davon, dass ich in meinem jetzigen Zustand den Sprung kaum schaffen werde.“

Gouri war überrascht, sonst versuchte Prip doch immer, so zu tun, als wäre er nicht am ganzen Körper versengt und verkratzt. Offenbar war es ihn wirklich wichtig, so schnell wie möglich die Drei Wasser zu erreichen. Wie um ihre Gedanken zu bestätigen, betonte Prip noch einmal: „Es ist wirklich dringend. Wir müssen diese Brücke jetzt überqueren.“

Gouri seufzte. Eine gute Mütze Schlaf kam wohl nicht in Frage. Dabei sahen diese Moosbüschel selbst im Dunkeln unglaublich einladend aus...

Als hätte Prip ihre Gedanken gelesen, fügte er an: „Von der anderen Seite der Brücke ist es ein Katzensprung bis zu den Drei Wassern. Dort wirst du die Gastfreundschaft des Zwergenvolkes erleben dürfen, erlesene Speisen, Federbette, warmes Kaminfeuer!“

Gouri antwortete nicht. Prip liess seine Schultern hängen und fuhr fort: „Du hast ja recht, du musst die Brücke nicht überqueren, wenn du nicht willst. Ich danke dir vielmals für alles, was du bislang schon für mich getan hast, ohne dich und deine Heilkünste wäre ich wohl keinen Hasensprung weit gekommen. Ich werde dich nicht vergessen und eines Tages werde ich zurückkommen und meine Schuld...“

Gouri atmete tief durch und stiess dann hervor: „Jetzt schweig schon stille, ich komme ja weiterhin mit.“

Prip hatte klargemacht, dass er auf jeden Fall versuchen würde, diese Brücke zu überqueren, und Gouri wollte den schwachen und angeschlagenen Schildzwerg nicht alleine durch diese düstere Nacht wandern lassen. Und Prips Worten nach lagen die Drei Wasser und weiche Zwergenbette nicht weit entfernt hinter dieser Schlucht. Ganz abgesehen davon wollte Gouri auch nicht alleine im Dunkeln zurück zu ihrer Agrenhöhle wandern, und darüber nachdenken, was für Sorgen sich ihre Familie gerade machen musste, wollte sie auch nicht. Von den Drei Wassern konnte sie bestimmt einen Falken mit einer Botschaft an die Agrengesellschaft schicken. Falls den Zwergen ihre Wünsche überhaupt wichtig waren. Aber darum konnte sie sich später kümmern, zunächst einmal hiess es, diese verflixte Hängebrücke zu überqueren.

Wenn sie nur nicht so viel Angst davor hätte, in die Tiefe zu stürzen.

Prip beobachtete Gouri besorgt, während er einige Schritte auf die Brücke zuging. Gouri folgte ihm nur langsam. Prip drehte sich zurück zu ihr und fragte: „Ist etwas nicht in Ordnung? Du musst mich wirklich nicht über diese Brücke begleiten.“

Gouri atmete etwas schneller. Grosse Höhen waren noch nie ihr Ding gewesen, und Hängebrücken erst jetzt nicht, aber hier im Dunkeln, da war noch etwas anderes. Diese grosse Gestalt im Nebel auf der anderen Seite der Brücke weckte in ihr eine zusätzliche Furcht, die Prip einfach nicht gleich zu spüren schien. Oder es gut versteckte. Gouri wollte diese Brücke einfach nicht überqueren. Sie murmelte etwas über Höhen und das Dunkel und Prip schien kurz abzuwägen, ob er sich trotzdem einfach von ihr verabschieden sollte und den Rest des Weges zu den Drei Wassern alleine zurücklegen. Dann sprach er:

„Wie wär’s damit: Wir gehen einige Schritte auf die Brücke zu, und schauen, ob sich die Gestalt bewegt. Wenn ja, dann rennen wir so rasch wie möglich zurück in den Nebel, wo uns niemand erwischen wird. Wenn nicht, dann können wir weitergehen.“

Gouri nickte stumm. Dann fügte sie leise hinzu: „Können wir jeweils alle paar Schritte warten, ob dieses Wesen, falls es denn eines ist, sich bewegt?“

Prip ergänzte sogar: „Wir können jederzeit warten, bis du dich sicher fühlst.“

Gouri war sich nicht sicher, wie sehr das stimmte. Der Zwerg kam ihr schon sehr ungeduldig vor.

Und so betraten die beiden die Brücke, der verletzte Schildzwerg stramm voranschreitend, die unverletzte Agren zaghaft hinterherschleichend.

Drei Holzplanken weiter blieb Gouri stehen, das dicke Brückentau fest im Griff, und lauschte. Sie hörte das Knarren der Brückenbretter und in weiter Entfernung den Schrei einer Horneule. Die Gestalt am anderen Ende der Brücke war weiterhin nur als undeutlicher Schemen zu erkennen, und wenn Gouris Augen sie in diesem angespannten Moment nicht täuschten, wurde der Nebel sogar noch dichter. Ihre unnatürliche Furcht vor diesem Wesen schwellte wieder an, und sie zwang sich, ruhig und langsam ein- und auszuatmen.

Prip, der schon einige Schritte weitergelaufen war, drehte sich um und sah Gouri dastehen. Er kehrte zurück zu ihr. „Die Agren, die mir das Leben gerettet hat, indem sie einen riesigen Troll erlegt hat, fürchtet sich vor ein wenig Höhe? Das ist echtes Zwergenhandwerk, diese Brücke wird in einem Jahrhundert noch stehen! Und im Dunkeln sieht man nicht einmal, wie tief es ist...“

Prips Stimme wurde leiser und setzte aus. Er wusste nichts mehr zu sagen, und wollte weiterziehen.

Gouri riss sich zusammen und setzte ein grimmiges Lächeln auf. Dann stapfte sie weiter, einen Schritt nach dem anderen, links, rechts, links, rechts. Ihre Hand huschte am Brückentau entlang, es immer noch als Stütze nutzend, aber nicht mehr festklammernd. Dann blieb sie erneut stehen. Einen Drittel des Weges über die Brücke hatten die beiden jetzt zurückgelegt, und der Nebel auf der anderen Seite der Brücke verdeckte den Schemen jetzt vollständig. Es bewegte sich nichts. Gouri glaubte, im Wind vereinzelte Stimmen zu hören. Ob diese aus dem unterirdischen Zwergenreich stammten oder nur die durch das Heulen des Windes entstehende „Musik der Toten“ war, vermochte sie nicht zu sagen.

„Was siehst du?“, flüsterte Prip angespannt.

„Nichts“, antwortete Gouri niedergeschlagen, „der Nebel ist zu dicht. Aber es bewegt sich wenigstens nicht.“

„Kannst du weitergehen?“

„Nur... gib mir noch einen Augenblick.“

Prip beobachtete Gouri wieder etwas besorgt, als diese in die Knie ging und tief durchatmete. Das Schwanken der Brücke im Wind wurde nicht weniger, das Holz war alt und morsch, und noch wäre es einfacher, sich umzudrehen und zurückzukehren, einen anderen Weg zu gehen. Aber sie mussten weiter.

Gouri richtete sich wieder auf und ging strammen Schrittes weiter. Sie versuchte, das Knirschen und Quietschen der Brückenbretter nicht zu beachten, und zog ihr Lauftempo noch an. Die Mitte der Brücke überquerte sie schnell, es gab nun kein rasches Zurück mehr. Und der Nebel war immer noch viel zu dicht, um die dunkle Gestalt zu erkennen. Bewegte sich da vorne nicht etwas? Gouri hörte nichts mehr als ihre Schritte auf den Holzplanken und ihren eigenen Atem, viel zu laut in der sonst so stillen Nacht, und sie blieb wieder stehen.

Prip rammte sie von hinten, zum Glück schmerzlos, aber nicht ohne ein für Gouri ohrenbetäubend wirkendes Scheppern seiner Rüstung auszulösen. Sie schnappte sich das Brückentau wieder und warf einen Blick in die Tiefen der Schlucht. Dafür schalt sie sich gleich wieder, denn es vervielfachte ihre Panik nur. Prip fluchte in der Zunge der Zwerge.

Beide blieben wachsam stehen und blickten auf die Nebelschwaden vor ihnen. Dann schauderte Gouri und blickte wieder zurück, in die Richtung, aus der sie soeben gekommen waren. Hatte sie nicht soeben gerade ein Geräusch von da gehört? Keine Seite der Brücke war jetzt definitiv sicher.

„Komm, weiter!“, drängte sie Prip, welcher sie auf die Füsse zog. Leise, leise schlichen Zwerg und Agren über die alte Zwergenbrücke.

Der Mond schien hell und beleuchtete die Szenerie in einem Spiel aus Licht und Schatten. Gouri schaute in den Himmel – nur weg vom Boden – und mochte die Ästhetik eigentlich noch. Wenn die Stimmung nicht so gruselig wäre, so würde man diesen nebligen Himmel und die ganze Lage bestimmt als schön bezeichnen. Schon bildete sich in ihrem Kopf die Anfänge eines Agrengedichts, und sie hielt inne. Diesmal sah Prip es kommen und hielt rechtzeitig an, denn der Nebel lichtete sich.

Der Nebel lichtete sich und gab den Blick frei auf einen riesigen Troll, welcher am Ende der Brücke sass und vor sich hin schnarchte.

Gouri zuckte zusammen. Sie hatte das Bild noch deutlich im Kopf, wie der namenlose Troll von vorhin den Drachen Sagrak vom Himmel zog und ihm ein rasches, gnadenloses Ende bereitete. Sie versuchte, gegen den Instinkt anzukämpfen, auf den Fersen umzudrehen und davonzurennen.

Prip fluchte erneut unter seinem Atem, als auch er den Troll am Brückenende entdeckte. Dann lief er an Gouri vorbei und mit den entschuldigenden Worten „Ich muss zu den Drei Wassern, so schnell wie möglich“ rannte er los, scheppernd und klimpernd, direkt auf den Troll zu.

Gouri schalt ihn innerlich, dann fiel ihr mit Schrecken auf, was der Schildzwerg schon viel früher erkannt haben musste: Der Troll regte sich. Sie hatten ihn bereits aufgeweckt, und nun galt es, an dem Ungetüm vorbeizukommen, ehe es sich vollends auf sie fokussieren konnte.

Prip hatte schon die halbe Distanz zurückgelegt, und Gouri überlegte kurz, ihn alleine weiterziehen zu lassen. Doch bei seinem verletzten Tempo hätte sie ihn im Nu eingeholt.

Gouri rannte los, versuchte, so gut wie möglich einfach nur nach vorne zu blicken. Prip humpelte voran, eine Hand an seinem Schild, die andere am Brückentau. Gouri war noch drei Schritte von ihm entfernt, da traf sie eine heftige Windbö, die sie zusammenzucken liess. Ihr Blick liess von Prip ab, welcher sich jetzt wieder weiter von ihr entfernte und nur noch einige Hasensprünge vom Ende der Brücke entfernt war, zum Troll. Sie zuckte zusammen.

Der Troll war vollends erwacht!

Dieser hier war etwas gedrungener als das Exemplar, welches sie heute bereits getroffen hatte, doch deswegen kein weniger ernst zu nehmender Gegner. Er räkelte sich und gähnte, seine gelblichen Augen mickrig aus seinem hervorspringenden Schädel starrend. Gouri konnte nicht erkennen, ob er sie anblickte. Nun hielt auch Prip inne, und rührte sich nicht mehr. Noch bestand die Chance, dass der Troll sich einfach wieder zu Boden sinken lassen würde.

Prip drehte sich zu Gouri um und gestikulierte, sich nicht zu bewegen.

Der Troll liess sich nicht wieder in sein Grasbett sinken, im Gegenteil, er streckte sich zu seiner vollen Grösse und packte das Brückengeländer mit seinen überlangen Armen. Langsam kam sein mächtiger Bauch auf den Rand der Brücke hinzu. Sein Kiefer knackte zweimal. Hatte er soeben sein Essen erspäht? Würde die Brücke seinem unglaublichen Gewicht standhalten können?

Gouri gestikulierte zurück an Prip, er solle Fersengold geben. Sah er sie überhaupt in der Dunkelheit der Nacht? Prip blieb ruhig und musterte das riesige Vieh, welches ihm den Weg zu den Drei Wassern versperrte.

Gouri fluchte. Verdammte Zwergenfeste, warum musste Prip auch so zwingend dorthin wollen?

Ihr hüpfte das Herz noch mehr in die Hose, als sie erkannte, wie Prip den Griff seines Schildüberrestes etwas stärker packte und leicht anhob. Er bereitete sich auf einen Angriff vor. Dieser Idiot! Dieser sture absolute Idiot! Noch ehe Gouri ein passenderes Schimpfwort einfiel, setzte sich Prip auch schon in Bewegung.

Prip rannte humpelnd und laut scheppernd über die wackelige Hängebrücke auf den Troll zu. Ohne zu überlegen, liess auch Gouri das Brückenseil los und setzte ihm nach, doch Prip legte ein für seinen Zustand erstaunliches Tempo an den Tag und Gouri war immer noch wackelig unterwegs, weswegen sich ihr Abstand nur vergrösserte.

Prip legte die letzten paar Schrittlängen in einem mächtigen Satz zurück und rammte den Troll mit den Überresten seines Schildes in den Bauch.

Prip prallte vom Trollbauch glatt ab und konnte sich gerade noch ein Seil der Hängebrücke fassen, ehe er in die Tiefen der Korn-Schlucht gefallen wäre.

Gouri schrie auf. Der Troll gluckste belustigt und rieb sich die Magengegend. Gouri blickte von der Kreatur zu Prip, welcher sich nun an der Aussenseite des Hängebrückengeländers festklammerte, nur wenige Hasensprünge vom sicheren Grasland auf der anderen Brückenseite entfernt... wenn er nur noch kurz durchhielt – doch war Prip sichtlich bedröppelt durch den Aufprall, den er soeben erhalten hatte.

Der Troll gluckste weiter und Gouri fiel viel zu spät auf, wie nahe das Wesen sich nun von ihr befand. Einen langen muskulösen Arm ausstreckend, griff der Troll nach ihr.

Aus dem starken Griff eines Trolls gab es kein Entrinnen. Er hatte die Agren mit einer Hand beinahe komplett umschlossen und hob sie in die Höhe, fast zärtlich, ohne sie zu zerquetschen. Wollte er sich einen Snack gönnen, mitsamt Haut, Haaren und Lederschuhen?

Gouri schrie auf, und Prip stimmte mit ein.

Da liess der Troll sich lachend rückwärts auf den Rücken fallen, sodass die Erde bebte und die Erdgeister rumorten. Wieder im Gras liegend, hielt er die durchgeschüttelte Gouri hoch in die Luft über seiner Körpermitte.

Gouri spürte, dass sich die rauen Finger des Trolls von ihr zu lösen begannen, ehe Prip es sehen konnte – Prip konnte im Dunkeln ohnehin so gut wie nichts sehen. Dann war es auch schon soweit.

Der Troll liess Gouri los.

Sie spürte, wie sie dem vorher so peinigend vorgekommenen Gefängnis entschlüpfte und nach unten glitt. Der freie Fall dauerte nicht lange, doch Gouri kam es wie eine Ewigkeit vor. Der Wind zerrte an ihren verfilzten Haaren und Mooskleidern. Noch im Fall drehte sie sich einmal um die eigene Achse und erreichte den molligen Trollbauch mit dem Kopf zuerst. Die Erfahrung hätte sie unter anderen Umständen als ekelerregend beschrieben, aber besser als ein Besuch in einem Trollmagen war sie allemal.

Gouri prallte mit einem dumpfen Geräusch vom Bauch des Trolls ab und wurde dadurch erneut in die Luft geschleudert, diesmal seitlich. Sich schon auf Knochenbrüche oder schlimmeres einstellend, drehte sie sich, diesmal kontrolliert, sodass sie wenigstens mit ihren Füssen am Boden einschlagen würde.

Sie erreichte den Boden nicht. Stattdessen war des Trolles erdige Hand erneut da, und fasste sie auf. Als er Gouri erneut anhob, erkannte diese an seinem friedlich grinsenden Gesichtsausdruck, dass er ihr nichts Böses wollte. Der tat nix, der wollte nur spielen!

Prip zog sich unter Ächzen über das Brückengeländer und stand auf wackeligen Beinen, mit offenem Mund auf die Szene schauend, die sich vor ihm abspielte.

Der Troll indes liess Gouri erneut los. Ihr Magen drehte sich um und der Trollmagen schlug ihr beim Aufprall die Luft weg, ehe sie erneut in die Höhe geworfen wurde. Diesmal erwartete sie den Griff des fröhlich glucksenden Trolls und bereitete sich innerlich vor, doch streifte der Troll sie unschön am rechten Arm, welcher unangenehm nach hinten gebogen wurden. Es war Zeit, diesem Spielchen ein Ende zu bereiten. Sie holte tief Luft, versuchte, die Schmerzen in ihrem Arm zu ignorieren, und...

„Hey!“, rief Prip von der Brücke aus. Der Troll hob seinen Kopf und starrte ihm grinsend entgegen.

„Hey!“, liess nun auch Gouri verlauten. Der Troll blickte wieder sie an, sichtlich verwirrt. Offenbar verstand er den drohenden Unterton in ihrer zitternden Stimme. Er rappelte sich auf, setzte sich auf seine kurze Unterschenkel und hielt Gouri vor sein Gesicht, welche unter den pochenden Schmerzen in ihrem rechten Arm Mühe hatte, einen wütenden Gesichtsausdruck beizubehalten. Der Troll öffnete sein Maul leicht und Gouri erblickte eine Reihe schiefer gelber Zähne. Wollte er dennoch einen Snack gönnen?

„Hier drüber! Schau hierher!“, hustete Prip hervor. Der lädierte Schildzwerg war von der wackeligen Brücke getreten und etwas näher gehumpelt. Von seiner ersten gewaltsamen Konfrontation mit dem Troll hatte er gelernt, dass er hier mit roher Kraft nichts ausrichten konnte.

Prip hielt die Überreste seines Schildes hoch – hätte genauso gut ein zwergengrosses Stück Holzkohle sein können – und sprach langsam, als würde er sich einem Kind nähern: „Tausch! Tauschen! Dieser Schild hier gehörte einst meinem Vater Aigar, und er war ein mutiger und ehrenvoller Mann, so wie ich voller Ehre bin.“

Der Troll lächelte dümmlich und sein Blick schweifte schon wieder zu Gouri hinüber.

„Schau her!“, warf Prip jetzt ein, „Dieser Schild hat die Feenwelt besucht und ist wieder zurückgekehrt! Feen! Kennst du Feen?“

„Feeeeeeeen“, wiederholte der Troll das Wort langsam. Er konnte ja doch sprechen!

„Feen, genau. Diese kleinen fliegenden Wesen, du kennst sie, oder? Sie sind ausgesprochen lustig, nicht? Dieser Schild ist genauso lustig. Und ich will ihn mit dir tauschen.“

Prip tauschte einen nervösen Blick mit Gouri. Wie viele von seinen Worten verstand der Troll?

„Tau – schen“, brummelte der Troll jetzt.

„Genau, tauschen“, versuchte Prip erneut sein Glück. Er näherte sich langsam dem Troll, während er weiterhin beruhigend auf ihn einredete, „Das heisst: Ich gebe dir was, und du gibst mir was. So haben wir beide etwas davon. Ich schenke dir, du schenkst mir.“

„Dieser Troll geben dir. Du geben etwas an dieser Troll“, brabbelte der Troll nach. Gouri zweifelte inzwischen daran, dass er Prip verstand. Doch der Griff des Trolls hatte sich etwas gelockert, weshalb ihr Arm weniger verrenkt wurde. Das war gut.

Prip pries weiterhin heiser seine kümmerlichen Schildreste an: „Schau nur auf diesen wunderschönen Schild. Die Feenkönigin höchstpersönlich hat ihn einst gesalbt, weil mein Vater die Feenwelt von einigen Dunklen Schemen gereinigt hat. Er...“ Prips Stimme versagte. Gouri sprang ein: „Man sagt diesem Schild nach, er könne bis zum heutigen Tag Übergänge zur Welt der Feen öffnen. Willst du einen Schild besitzen, der Tore zu einer anderen Welt öffnen kann?“

„Feeeeen“, sprach der Troll, und seine Mundwinkel zogen sich noch weiter nach oben, als Gouri je für möglich gehalten hätte.

Prip hatte seine Stimme wiedergefunden und lenkte die Aufmerksamkeit des Trolls wieder zu ihm: „Hier, ich lege den Schild auf diesem Grasstück ab. Aber damit du ihn haben kannst, musst du deine andere Hand öffnen. Dann kannst du zu den Feen reisen! Feen!“

Nun strahlte das Gesicht des Trolles, und er griff mit seiner freien Hand nach den Schildüberresten, die Prip auf dem Boden platziert hatte. Prip reagierte blitzschnell: „NEIN!“

Der Troll zuckte zurück und blickte ihn aus traurigen Augen an. Prip hatte das Tauschgeschäft schon lange aufgegeben, und sprach weiter: „Nein, warte! Zwei Hände! Zwei Hände an den Schild! Sonst keine Feen!“

„Keine Feen?“, sprach der Troll enttäuscht nach. Prip nickte und zeigte ihm beide offenen Hände. Und wie durch ein Wunder tat der Troll es ihm nach und öffnete seine Hand. Gouri purzelte von der Hand hinunter und schlug mit dem Rücken auf dem Pflaster der alten Zwergenstrasse auf. Es schmerzte schrecklich, doch jetzt war keine Zeit dafür, sie musste sich bewegen, schnell, sie war frei! Auf jetzt, auf zur rettenden Festung!

Gouri stellte sich auf ein Knie, und versuchte, sich zu erheben. Die Welt schwankte, doch sie blieb stehen – soll noch einer behaupten, Agren hätten ein schlechtes Gleichgewicht! Sie tastete ihre Gliedmassen und ihr Gesicht ab. Alles brannte unangenehm, und sie schmeckte ein wenig Blut, aber schlimmer als einige Schürfungen würde es nicht sein. Sie humpelte von der riesigen Gestalt des Trolles weg. Ein Blick zurück verriet ihr, dass der Troll die Schildüberreste freudig in die Hände genommen hatte und auf einem Finger zu balancieren versuchte.

Prip nahm nun auch seine Beine in die Hand und huschte hurtig humpelnd am grossen Troll vorbei. Dieser blickte ihm enttäuscht nach und brummelte: „Kein Feen... dieser Schild nicht sein toll! Schild von Troll sein besser. Troll haben Schild mit Händeschütteln drauf! Dieser Schild dort viel mehr toll!“

Prip rammte seine Fersen in den Steinboden der alten Zwergenstrasse und drehte sich rasant um.

„Schild mit Händeschütteln drauf?“, fragte er argwöhnisch nach. Gouri hätte ihm zugezischt, er solle doch einfach still sein, doch sie war noch zu erschöpft dafür, presste ihre Hände in ihre Seiten und verschnaufte tief. Noch schien ja keine imminente Gefahr vom gutmütigen Troll auszugehen.

„Schild mit Händeschütteln drauf!“, nickte der Troll. „Dieser Schild dort sein viel mehr gut!“

Prip verzog sein Gesicht zu einer schmerzverzerrten Grimasse: „Ein rotbrauner Schild?“, fragte er weiter, seine Stimme zitternd.

„Dieser Schild dort haben Farbe von Erdtroll, jawohl.“, nickte der Troll glücklich.

Prip fluchte in seine Bartstummel hinein und drehte sich wieder vom Troll ab, jetzt sogar ein wenig schneller rennend als zuvor. So jagte er an Gouri vorbei, welche sich wieder zu ihm umdrehte und ihm verwirrt hinterhersah.

„Freunde, wartet! Dieser Troll hier können erzählen mehr von dieser Schild dort! Dieser Zwerg dort rennen in Bauch von dieser Troll nochmal! Bitteeee!“

Sein Klagen verklang, als auch Gouri sich in Bewegung setzte und den Troll hinter sich liess. Keine weiteren Geräusche kamen vom Troll. Offenbar war er nicht in Stimmung, ihnen nachzujagen. Zumindest hoffte Gouri dies.

Prip rannte weiter vorne immer noch die Pflastersteine der alten Zwergenstrasse entlang, wurde allerdings etwas langsamer und liess Gouri zu sich aufholen. Gouri schmerze es inzwischen nicht nur an den Prellungen und Schürfungen, sondern auch ein Seitenstechen machte sich bemerkbar. Prip war eindeutig zäher als sie. Warum war er nur so in Eile?

„Warum bist du nur so in Eile?“, presste Gouri zwischen zwei Atemzügen hervor, „Wir könnten hier schnell anhalten, an den Büschen dort drüben wächst etwas Wolfskraut, das könnte ich zerreiben, und die Drei Wasser...“

„Die Drei Wasser sind gefallen, Gouri!“, sprach Prip. „Der Troll beschrieb einen rotbraunen Schild mit zwei verschränkten Händen. Das ist der Bruderschild, einer der vier legendären mächtigen Schilde, die der Meisterschmied Kreatok geschaffen hatte, ehe er seinen Verstand verlor.“

Gouri folgte ihm langsam: „Und wenn dieser Troll sagt, dass er diesen... Bruderschild besitzt...“

Prip vollendete ihren Satz: „...dann ist der Bruderschild sicherlich nicht mehr in den Drei Wassern, wo er sein sollte. Die Drei Wasser sind die Feste, wo die vier mächtigen Schilde Kreatoks aufbewahrt werden. Wenn der Bruderschild daraus errungen werden konnte...“ Er sprach nicht weiter.

„Prip. Wir wissen doch noch nicht mal, ob der Troll die Wahrheit sagt. Vielleicht hat er die Schilde auch nur von weitem gesehen.“

„Die Schilde werden nur in absoluten Notfällen hervorgebracht! Schon allein, dass dieser Troll darüber Bescheid weiss, heisst, dass die Feste angegriffen wurde. Nach dem Fall der Burg Karulzar und des Alten Brons sind die Drei Wasser der letzte oberirdische Zwergenstützpunkt, der uns noch bleibt. Wenn dieser auch noch an die Drachen fällt, werden wir uns endgültig in den Untergrund zurückziehen müssen... doch solange die Drachen da sind, werden wir die Besatzung der Feste nie abziehen können, dann stehen wir da ja wie auf dem Präsentierteller... und im Unterirdischen Kampf sind wir unterlegen, wir brauchen Ballisten, um eine Chance gegen die Drachen zu haben... wenn wir uns in den Untergrund zurückziehen, können die Drachen und die Riesen aus dem Süden sich weiter bekriegen, aber das Volk der Zwerge würde dabei so gut wie ausgelöscht... die Schlacht um die Drei Wasser wird das Schicksal der Schildzwerge bestimmen... ich muss zu den Drei Wassern, ich muss es wissen...“

Prip war zusammengesackt und murmelte ein wenig weiter: „Am Baum der Lieder in den nördlichen Trolllanden sagte man mir, die Riesen aus dem Süden wären kurz davor, die Magie der Toten freizusetzen. Die Toten zurückzuholen! Ich wollte den Fürst der Drei Wasser davor warnen, aber nun...“

Gouri betrachtete den verzweifelten Schildzwerg, nicht wissend, was nun zu tun sein. Auf der Suche nach einer Ablenkung warf sie ein: „Stimmte dies mit dem Schild deines Vaters und der Feenwelt?“

Prip guckte auf und sah sie entrüstet an, sein Blick sprach: „Wie könntest du annehmen, dass ich nicht die Wahrheit sagen würde!?“, während sein Mund sprach: „Es... es ist nur eine Legende, aber genug, um einen Troll abzulenken. Vielleicht erzähle ich dir eines Tages davon.“

Er fasste sich etwas und richtete sich wieder auf. „Ich muss es wissen! Auf zu den Drei Wassern!“

Gouri nickte zaghaft, rupfte ein wenig Wolfskraut aus einem Busch am Wegesrand, steckte es in die Tasche ihres Wams und setzte Prip nach. Gemeinsam huschten sie noch ein kurzes Stück der Zwergenstrasse entlang, welche nun im Mondeslicht deutlich zu erkennen war, da sie den Nebel verlassen hatten. Dann bog Prip nach rechts ab, eine Böschung hoch, und Gouri folgte ihm. Langsam krochen sie bis auf die Spitze des Hügels und blickten darüber.

Hitze schlug in ihre beiden Gesichter

Die Drei Wasser! Auch im nebeligen Dunkel der Nacht konnten Gouris Agrenaugen erkennen, dass es sich dabei um eine wunderschöne Zwergenfeste handelte: Hohe Kuppeltürme, Zwergenstatuen links und rechts, eine Quelle, die in einem riesigen Brunnen sprudelte. Doch das aus dem Quellenbrunnen weit in die Höhe schiessende Wasser konnte nicht verbergen, dass die Festung lichterloh brannte. Flammen schossen meterweit in die Höhe, Zwerge wuselten um den Stein herum und versuchten, das Zusammenfallen des Gebäudes zu verhindern.

Prip riss seine Augen auf: „Wie konnte dies geschehen?! Unser Gestein brennt nicht! Welche diabolischen Mächte – “

Dann fiel sein Blick in die Luft, und er sah die vielen Drachen, die die Feste weit in der Höhe’ umkreisten. Ein einzelner Drache lag weiter weg am Boden, von mehreren dicken Lanzen durchbohrt, doch die anderen hatten sich in die Lüfte geschwungen und blubberten nun fröhlich Lava auf die Drei Wasser hinunter.

Die Drei Wasser brannten lichterloh.

Die letzte Zwergenfeste war gefallen.

Prip schluchzte auf.

„Das... das ist nicht gut.“, murmelte Gouri, den Blick nicht loslösen könnend vom Inferno, welches über den Drei Wassern tobte. Die Flammen tobten, und die hunderte von Metern über den höchsten Zinnen kreisenden Drachen wirkten nicht, als hätten sie vor, den Angriff abzubrechen, ehe von den Drei Wassern mehr als ein verkohltes Gerippe übrig war.

Der gefallene Drache lag verrenkt neben einem der drei hohen Türme, Speere aus seinem Bauch ragend. Offenbar war es den Zwergen möglich gewesen, eine der riesigen Echsen vom Himmel zu holen. Warum versuchten sie es nicht erneut? Sicherlich konnten die Schildzwerge Ballisten fabrizieren, die höher reichten, als die Drachen fliegen konnten?!

Doch ehe Gouri Prip nach diesem Sachverhalt ausfragen konnte, hatte Prip sich erhoben und war den Hang hinuntergestolpert, direkt auf die sterbende Zwergenfeste zu.

Gouri fluchte „Spornendreck!“ und setzte ihm nach.

Glücklicherweise waren die Drachen ziemlich akkurat darin, die Drei Wasser mit Lava und Feuer zu treffen, sodass man ausserhalb der Feste in relativer Sicherheit war. Viele Schildzwerge standen etwas abseits in Gruppen zusammen. Es war nicht zu erkennen, ob sie sich berieten oder einfach nur litten unter dem Anblick ihrer Heimat, die vor ihren Augen davonschmolz.

Prip eilte zu einem solchen Trio. Sowohl auf seinem als auch auf ihren Gesichter erschien ein gegenseitiges Erkennen:

„Prip! Was ist mit deiner Rüstung geschehen?“

„Vergiss deine Rüstung, was ist mit deinem prächtigen Bart geschehen?!“

„Und wer ist denn deine Begleiterin?“

Prip verschnaufte und antwortete: „Das ist Gouri, die mutige Agren. Sie hat mein Leben vor einem dieser Mistviecher gerettet.“

Dann setzte er nach: „Kümmert euch doch nicht um mich! Was in allen Feuern der Tiefminen ist denn hier geschehen?!“

Einer der anderen Schildzwerge, ein grossgewachsener Bursche mit einer Glatze und einem mächtigen roten Schnurrbart, antwortete mürrisch: „Nach was sieht es denn aus? Den Drachen war Karulzar und der alte Bron noch nicht genug! Wir müssen unser Volk nach Cavern evakuieren, aber nicht mal dorthin können wir unbeschadet, da der nächste freie Eingang auf der anderen Seite der Korn-Schlucht liegt!“

„So ein Zufall, von da kommen wir gerade“ liess Gouri verlauten. Der Zwerg mit dem prächtigen roten Schnurrbart blickte sie finster an und fuhr fort:

„Fürst Bailor hat es erwischt. Hornheim aus dem Roteisenstein gibt jetzt den Ton an – wenn es nach diesem Tag überhaupt noch eine Zwergenarmee gibt, über die man herrschen kann. Wir sitzen hier auf dem Präsentierteller!“

Prip fragte entsetzt: „Abes was war mit den Schilden?! Was ist mit den Ballisten?“

Der Zwerg mit dem Schnurrbart murrte: „Du hast einiges verpasst. Die Drachen haben uns belagert, die Ballisten hervorgelockt, uns unsere Speere verschwenden lassen. Mit Verlaub, unser guter Kommandant war ein Trottel. Und die Nachschübe aus Cavern haben auf sich warten lassen. Wahrscheinlich haben die elenden Riesenreptilien Feuerstösse ausgelöst und uns vom Rest des Höhlennetzwerks abgeschnitten.“

Der Zwerg holte tief Luft und fuhr fort: „Sie haben die Wachtürme und Waffenkammern zuerst abgesengt. Das war schon einige Tage her. Als hätten diese steinernen Trottel es gerochen, ist eine Horde Trolle gekommen und hat sich den Bruderschild aus den Überresten der Waffenkammer gekrallt, ehe wir einen Gegenschlag ausführen konnten. Gut, wir waren auch anderweitig beschäftigt.“

Gouri wusste besser, als den Troll von vorhin mit dem Bruderschild zu erwähnen. Der Zwerg war schon enerviert genug.

Sein Schnurrbart erzitterte, als der Zwerg fortfuhr: „Den Sternenschild und den Dunkelschild konnten wir den Trollen entreissen, doch fiel der Sternenschild kurz darauf an die Drachen, die im Sturzflug Zwerg um Zwerg von unseren Zinnen picken. Es sind mehr gefallen als damals, als der Urtroll Tiefenfall vernichtete! Das ist das Ende der Schildzwerge!“

Nun sprach der Zwerg nicht mehr. Prip und Gouri warteten immer noch auf den Aufenthaltsort des Dunkelschilds.

Ein anderer Zwerg meldete sich zu Wort: „Tja, der Dunkelschild... der liegt dort drüben unter diesem Biest!“ und er zeigte zum gefallenen Drachen, welcher nahe der Mauer der Drei Wasser am Boden lag.

Die Zwerge fluchten gemeinsam. Unter diesem Vieh würde man den Schild nur sehr mühsam und unter grossem Zeitaufwand bergen können. Zeit, die sie schlichtweg nicht hatten.

Gouri, die keine Ahnung vom Armeegeschäft hatte, aber dutzende Legenden über die mächtigen Schilde kannte, konnte verstehen, was für ein riesiger Verlust das Fehlen der drei Schilde für die Schildzwerge war. Gute Güte, vielleicht würde das wirklich das Ende der Schildzwerge sein.

„Immerhin, wir sind dabei, einige Wurfmaschinen zu bauen. Wenn diese Sklavenschinder aus dem Süden Zwerge dazu anleiten können, ganze Zeras zu bauen, so können wir freien Schildzwerge das erst recht! Wurfmaschinen mögen weniger elegant als Ballisten sein, aber sie werden ihre Arbeit tun. Und die Drachen konzentrieren sich bloss auf die Drei Wasser, die werden das Ganze ganz sicher nicht kommen sehen.“

Da atmete Prip wieder ein wenig auf.\bigskip



Die Sonne erhob sich hinter den Bergen, und Gouri verspürte Hunger in ihrem Magen. Sie nahm nicht an, dass einer der an- und niedergeschlagenen Zwerge sich zu einem Morgenmahl begeben würde, und so kletterte sie einen Hügel hinunter und schnappte sie sich einen angrenzenden Busch und begann, ihn knabbernd zu verzehren. Dann blickte sie zu den Zwergen auf, die sie mit grossen Augen ansahen.

„Oh Herrin des Steins, das war doch hoffentlich kein heiliger Busch oder so was?“, rief Gouri ihnen entgegen. Die Zwerge starrten immer noch, und Gouri war sich immer noch sicher, dass es mit dem Verzehren ihres Asts zu tun hatte. Doch dann lachten sie auf und drehten sich weg, weiter an einem verzweifelten Rettungsplan für die überrannte Zwergenfeste schmiedend.

In diesem Moment vernahm Gouri ein Knacken aus dem Gebüsch hinter ihr.

Sie drehte sich alamiert um.

Eine verhutzelte Gestalt krabbelte kaum zehn Meter von ihr entfernt durch das Gestrüpp. Etwas weiter entfernt erkannte Gouri etwas, das an eine Ruine erinnerte – dort musste einst ein Aussenturm der Drei Wasser gestanden haben, welcher jetzt natürlich verbrannt und eingestürzt war.

Wahrscheinlich hatte die Person vor ihr sich im Turm befunden, als dieser gestürzt worden war, und nun hatte sie sich endlich in die Freiheit gekämpft.

Die Gestalt war ein Zwerg, zweifelsohne, mit einem langen grauen Bart. Er wäre bestimmt elegant gekleidet gewesen, wäre sein langer Mantel nicht an so vielen Stellen gerissen, verschmutzt und angebrannt.

Etwas an der Erscheinung dieses Zwergs liess Gouris Alarmglocken klingeln. Äusserlich konnte Gouri keine Verletzungen an ihm feststellen, doch ging er langsam und ungelenk, als hätte er seine Füsse für eine lange Zeit nicht mehr gebraucht. Seine Lippen bewegten sich rasch, als murmle er vor sich hin. Fahrig rieb er sich seine Arme.

Da fiel Gouri auf, dass seine Hände und Füsse in den Überresten goldener Ketten eingeschlossen waren. Handelte es sich bei diesem Zwerg um einen Gefangenen der Schildzwerge? Sie hatte gehört, dass die Schildzwerge manchmal spezielle Häuser für diejenigen einrichteten, die gegen allgemein anerkannte Regeln verstossen hatten, und wenn sie einmal drin waren, durften sie sie nicht mehr verlassen.

Gouri stellte fest, dass der entflohene Zwerg in den goldenen Ketten ihr eindeutig zu nahe war für ihren Geschmack. Zwar hatte er noch nicht zu erkennen geben, dass er Gouri bemerkt hatte, doch bewegte er sich in leichtem Zick-Zack auf sie zu, und wenn sie sich nicht irrte, wurden seine Schritte sogar noch schneller. Nein, jetzt war er noch fünf Schritte von ihr entfernt und kämpfte sich buchstäblich durch das Gebüsch auf sie zu. Gouri drehte sich ab und blickte oben an den Hügel, wo Prip und die übrigen Schildzwerge sich berieten. Sie musste so schnell wie möglich zu ihnen.

Die Schildzwerge standen immer noch oberhalb des Hügels, doch sie schienen sich nicht mehr zu beraten, stattdessen blickten sie allesamt auf Gouri, oder genauer gesagt auf einen Punkt ein wenig hinter Gouri. Sie mussten den entflohenen Zwerg ebenfalls gesehen haben.

Prips Gesichtsauszug zeigte ein erschrecktes Erkennen des Zwergs, und er rief aus: „WEG VON IHM, GOURI, WEG!“

Das war alles, was sie zu hören brauchte. Etwas war mit diesem geflohenen Zwerg nicht in Ordnung. Sie musste weg von ihm. Sie musste weg, JETZT!

Und Gouri rannte los, so schnell es ihre müden verkratzten Gliedmassen zuliessen. Nach nur einigen stolpernden Ansätzen wurde sie von hinten gepackt und zu Boden geworfen. Ihr Kopf schmerzte, doch sie konnte nicht ausruhen, nicht jetzt!

Ein überraschend starker Griff packte sie am Kragen ihrer Mooskleidung und zog sie nach oben. Der entflohene Zwerg hielt sie fest vor sich und blickte ihr in die Augen. Seine eigenen Augen zuckten unwillkürlich, und er rief aufgeregt: „Ich habe ihn nicht getötet, weisst du, ich habe ihn nicht getötet, und dass er nicht zurückkommt, nie mehr zurückkommt, das ist seine Schuld, nicht meine!“

Gouri versuchte zurückzuweichen, dem Griff zu entkommen, aber der Zwerg zerrte sie nur noch weiter in die Höhe, sodass ihre Füsse kurzzeitig den Boden verliessen.

„So versteh’ mich doch, ich wollte das nie! Die haben mir gesagt, das könne ich nicht mehr wiedergutmachen, nur dafür büssen! Büssen! Habe ich nicht schon genug gebüsst?! O Nehal, warum hast du mich nur verlassen?“

Dann lockerte sein Griff sich und Gouri plumpste zu Boden. Um sich blickend, erkannte sie Prip und die drei weiteren Schildzwerge, die näher getreten waren und einen Kreis um den entflohenen Zwerg und die am Boden liegende Gouri bildeten. Sie waren dem geflohenen Zwerg zahlenmässig weit überlegen, doch waren alle ihre Gesichter angespannt. Selbst Prip warf nur einen raschen Blick auf Gouri, um zu sehen, ob es ihr vergleichsweise gut ging, und blickte dann blitzschnell zurück zum Zwerg in den Ketten.

Dieser grinste: „Vorsicht ist die Mutter aller Tugend, das hat Nehal immer gesagt. Ich nehme dann an, ihr wisst, wer ich bin? Ich...“, er lachte nervös auf, „Ich bin mir nämlich gerade nicht so sicher, wer ich bin. Ich weiss natürlich, wer ihr denkt, dass ich bin...“

Die Zwerge tauschten unsichere Blicke aus. Der mit dem prächtigen Schnurrbart nickte grimmig.

„Aha!“, rief der Zwerg in den goldenen Ketten aus, „Ihr glaubt, zu wissen, wer ich bin! Dann solltet ihr ja auch wissen, dass ich bereits gebüsst habe, ihr Narren! Und ich gehe sicherlich nicht mehr zurück in dieses Drecksloch! Also... ihr werdet mich ziehen lassen müssen, oder ich sehe mich gezwungen, zu... drastischeren Massnahmen zu greifen.“

Während die Sprechweise des Zwergs kohärenter wurde, wurde auch sein Blick klarer, und er schien sich seiner Situation besser bewusst zu werden. Das war nicht beruhigend. Er drehte sich einmal langsam im Kreis und musterte jeden Zwerg kurz. Bei Prip blieb sein Blick hängen:

„Was ist denn mit dir passiert?“

„Das war natürlich ein Drache. Keiner von denen hier. Und er ist jetzt tot“, erwiderte Prip grimmig. Gouri konnte nicht erkennen, ob er versuchte, Eindruck zu schinden.

Beeindruckt wirkte der entflohene Zwerg nicht, stattdessen liess er die Überreste seiner goldenen Ketten locker um seinen Arm herumwirbeln, und bellte: „Ein Drache, soso. Wie hiess er denn?“

„Sagrak.“

Der entflohene Zwerg grinste: „Oho, das freut mich doch. Weil ich jemanden kenne, den das ganz und gar nicht freuen wird.“

Dann schrie er plötzlich: „HA!“ und griff an seinen linken Unterarm. An diesem Handgelenk war eine kleine, surrende, golden glänzende Apparatur befestigt, mit einer Sprungfeder, welche sich soeben von einem Haken löste – und ein goldener Bolzen schoss aus der Apparatur hervor und traf der schnurrbärtigen Zwerg an der Schulter. Er verzog das Gesicht schmerzverzerrt und fiel zu Boden. Prip und die beiden anderen Schildzwerge – ein breiter Krieger und eine hochgewachsene Kriegerin – brauchten einige Sekunden, um sich zu fassen. Sekunden, die der entflohene Zwerg gut zu wissen nutzte. Er rammte Prip in die Seite, warf ihn zu Boden und gab zünftig Fersengeld, den Hügel hinauf, weg von der Gruppe.

Überrumpelt, aber nicht weiter verletzt, rappelte Prip sich wieder auf und blickte hinüber zum schnurrbärtigen Zwerg, in dessen Schulter immer noch ein goldener Bolzen steckte – der entflohene Zwerg hatte genau in die Spalte der Rüstung getroffen.

Der breite Krieger und die hochgewachsene Kriegerin hatten sich bereits neben ihm niedergelassen:

„Wie schlimm ist es?”

„Meinst du, der Bolzen ist vergiftet?”

„Definitiv, sonst würde unser Belenor das doch mit links wegstecken.”

„Aber wie hätte er an Gift kommen können?”

„Wie er wohl an diese Waffe gekommen ist – ach, die hat er bestimmt wieder selbst entworfen. Wie immer! Dieser verflixte Genius.”

Der verletzte Zwerg mit dem Schnurrbart – Belenor – stöhnte auf und die Aufmerksamkeit fiel wieder auf ihn. Nun gelang auch Prip dazu. Gouri hatte bereits ein wenig Wolfskraut aus ihrem Mooskleid gezogen und fragte die übrigen Zwerge: „Will denn niemand den entflohenen Zwerg verflogen?”

Die übrigen blickten sich unsicher an. Dann ergriff Prip das Wort:

„Vielleicht… vielleicht ist das gerade das, was er will.”

„Wir können nicht wissen, was er vorhat. Wenn er in so kurzer Zeit eine kleine Giftmischung herstellen konnte, hat er inzwischen schon eine geniale Falle erbaut”, half ihm die Zwergin aus.

„Vielleicht ist es das Beste, ihm komplett aus dem Weg zu gehen”, hustete nun auch Belenor.

Gouri blickte den Schildzwergen einem nach dem anderen ins Gesicht, und lachte dann beinahe auf. Sie hatten Angst! Dann fiel ihr auf, dass sie demnach selbst auch Angst haben sollte.

„Wer... wer war dieser entflohene Zwerg denn?” wollte Gouri wissen.

Prip setzte zu einer Antwort an, doch da ertönte ein langgezogener Schrei von der anderen Seite des Hügels, und die Zwerge blickten einander alarmiert an. Der breite Zwerg fasste sich zuerst und hetzte den Hügel hinauf, gefolgt von Prip und der Kriegerin. Belenor blieb liegend zurück und schien inzwischen nicht einmal so unzufrieden mit seiner Position zu sein. Gouri sah, dass es ihm vergleichsweise gut ging, also liess sie ihr Kräuterbündel bei ihm und spurtete selbst den übrigen Schildzwergen hinterher.

An der Hügelkuppe angekommen, erblickte sie ein grauenvolles Bild. Ein weiterer Drache war vom Himmel gefallen und lag nun am Ende einer tiefen Schleifspur in der Erde, mitten in der Ebene vor der Festung. Er rührte sich nicht mehr. Er schaute auch nicht mehr wie ein lebendiger Drache drein – vielmehr war er steinern in einer fliegenden Position eingefroren.

Keine Zwergenballiste hatte den Drachen erledigt, und keine Lanze steckte in ihm. Gouri blickte wild um sich, konnte aber nichts erkennen, was das Wesen niedergestreckt haben könnte. War er einfach so im Flug versteinert worden?

Schliesslich ergriff Gouri das Wort: „Wer hat denn den da erledigt?“

Prip lachte auf: „Das war er selbst. Weißt du, die Drachen sind Seelen der Erde und des Feuers, zusammengeschmiedet während ihrer Entwicklung im Ei. Doch sie verhalten sich gerne wie Seelen der Lüfte und schwingen sich in ungeahnte Höhen. Werden sie zu übermütig und verbringen zu wenig Zeit in Kontakt mit der guten alten Mutter Erde, so holt diese sie sich auf ihre eigene Art zurück.“

Die hochgewachsene Zwergenkriegerin schnaubte auf: „Als ob die Mutter der Erde so aktiv in das Weltgeschehen eingreifen würde. Das ist doch eindeutig ein Fluch der Windgeister, der allzu aufmüpfigen Drachen zeigen soll, wo sie hingehören!“

Der breite Zwergenkrieger gluckste und meinte: „Mein Vater hat mir jeweils erzählt, dass die Trolle vom Stein kommen, und jedes Mal, wenn der Urtroll einen neuen erschafft, so verlangt die Balance der Natur, dass ein anderes Wesen wieder zu Stein wird, und das ist dann halt eben ein Drache wie dieser gefallene Kerl hier.“

Prip seufzte und argumentierte nicht weiter, sondern setzte einen poetischen Schlussstrich: „Vom Steine sind sie gekommen, und zu Steine sollen sie werden.“

Die Zwergenkriegerin nickte: „Mehr Gestein für unsere zukünftigen Katapulte. Das mag ich.“

Weiter vorne erblickte Gouri den entflohenen Zwerg, wie er mutig über das Schlachtfeld sprintete. Keiner beachtete ihn, alle Augen waren auf den versteinerten Überresten des gefallenen Drachen gerichtet. Sie machte die anderen Zwerge auf den fliehenden Zwerg aufmerksam.

Prip fluchte: „Schaut euch mal seine Trajektorie an. Der will zum Drachenkadaver.“

„Dort liegt der Dunkelschild!“, hauchte Gouri.

Der breite Zwerg nickte zustimmend: „Unter mehreren Tonnen Drachenfleisch sollte der Dunkelschild eigentlich auch vor ihm sicher sein – aber wenn irgendjemand darankommen kann, dann ist es er.“

„Dann müssen wir ihn aufhalten!“

Die Zwerge standen allen Anschein nach einer Konfrontation mit dem entflohenen Zwerg immer noch abweisend gegenüber. Überraschenderweise sahen sie aber nicht bedrückt aus. Die hochgewachsene Kriegerin grinste sogar schwach und meinte dann zu Gouri: „Er kommt nicht weit. Siehst du, wie die übrigen Zwerge aus dem Weg laufen? Die wissen, was da kommen wird.“

Gouri blickte, und tatsächlich! Die meisten Schildzwerge, die sich noch in der Nähe des Gemäuers befunden hatten, huschten von der Einschlagstelle des gefallenen versteinerten Drachen weg, als ginge es um ihr Leben – was darauf hindeutete, dass es tatsächlich um ihr Leben gehen könnte. Einzig der entflohene Zwerg kümmerte sich nicht darum und rannte schnurstracks an den versteinerten Überresten der Echse vorbei, auf den Leichnam des anderen Drachen hinzu.

Prip fuhr fort: „Das hat damit zu tun, dass die Drachen ihre versteinerten Kameraden nicht zurücklassen. Schliesslich können sie sie noch retten, wenn sie sie rasch genug in ihr stinkendes Nest zurückbringen. Warte darauf, was jetzt kommt.“

Erwartungsvoll blickte Prip nach oben. Eine dicke Wolkendecke schwebte inzwischen über den Drei Wassern, aber die immer noch von oben herabfallende Lava zeigte deutlich an, dass die Drachen weiterhin über dem Gemäuer schwebten und Feuer herabregnen liessen. Kam es ihr nur so vor, oder war die Lava weniger geworden?

Da wurde die Wolkenwand zerfetzt, und Gouri fühlte sich ungut an Sagraks Angriff auf Prip erinnert, als ein riesiger schwarzer Drache aus dem Himmel stob und ohne abzubremsen neben dem gefallenen Steindrachen auf die Erde prallte, sodass Schlamm und Gestein hunderte von Zwergenhöhen gen oben und zur Seite stieben. Dann erhob er sich langsam aus dem soeben verursachten Krater.

Sagrak war der erste Drache gewesen, den Gouri gesehen hatte, und schon er hatte ihr eine Todesangst eingejagt. Dieser hier war um ein Vielfaches schlimmer. Er hätte Sagrak um mehr als das fünffache überragt, seine Krallen alleine waren schon fast so lange wie ein mächtiger Zwerg. Sein Schwanz zuckte umher, als er seinen schlangenhaften Kopf um sich wandern liess, mit seinen riesigen rot glühenden Augen auf der Suche nach etwas, das ihm gefährlich werden konnte. Doch da war nichts. Dieser schwarze Drache war der sichere Tod für alle, die es wagten, sich ihm entgegenzustellen.


\newpage
\section{Kreatok gegen Tarok}

Der soeben gelandete schwarze Drache musterte seine Umgebung aufmerksam. Diejenigen Schildzwerge, die durch den Einschlag des Reptils in die Erde noch nicht von den Füssen geholt worden waren, duckten sich hinter ihre Schilde und begannen, alte Zwergenlieder zu rezitieren. Offenbar bereiteten sie sich auf den Mentalen Angriff des soeben gelandeten Riesendrachen vor. Langsam wurden die verschiedenen Stimmen synchron, und mit anschwellender Lautstärke sangen die Zwerge eine wortlose Melodie, in Bruderschaft tapfer vor einem übermächtigen Feind stehend und dessen hinterhältige geistige Attacke erwartend. Sie sangen die uralte Melodie des Steinsang.

Der entflohene Zwerg in den goldenen Ketten drehte nur kurz seinen Kopf, um die Anwesenheit des riesigen Gegners anzuerkennen, und rannte dann schnur stracks weiter auf den Drachenkadaver zu, wo der Dunkelschild lag.

Der riesige Drache verzog seinen Mund zu etwas, das man mit etwas Fantasie ein Grinsen hätte nennen können, und entblösste dabei gelbe Zähne, grösser als Langschwerter. Seine Stimme war tief und höhnisch, als sie in Gouris Kopf erscholl:

„WICHTE! WER DENKT IHR, DASS ICH BIN? ICH HABE ES NICHT NÖTIG, ZU SOLCH EHRENLOSEN MITTELN WIE EINEM GEISTIGEN ANGRIFF ZU GREIFEN. ICH BIN EIN DRACHE VON EHRE. DENJENIGEN UNTER EUCH, DIE IHREM GERECHTEN ENDE STANDHAFT GEGENÜBERTRETEN WOLLEN, GEWÄHRE ICH DEN EHRENHAFTEN TOD DURCH DAS LEUTERNDE FEUER. LASSET EURE SCHILDE FALLEN.“

In Gouris Kopf dröhnte es, und ein rascher Blick auf ihre Begleiter verriet ihr, dass es ihnen ebenfalls so ging. Kein einziger Schildzwerg liess seine Deckung fallen, vielmehr wurde der gemeinsame Gesang nur noch lauter. Der riesige schwarze Drache fuhr unberührt fort:

„WUSSTE ICH’S DOCH! NICHTS ALS WIRBELLOSER SPRONENDRECK SEID IHR, WÜRMER UNTER MEINEN TATZEN. IHR FÜRCHTET EUCH ZU RECHT, DENN ICH BIN TAROK, DER RÄCHER, UND ICH HABE GESCHWOREN, EUCH ALLE FÜR EURE VERBRECHEN BÜSSEN ZU LASSEN!“

Sein Hals richtete sich auf, und Gouri, die die Anzeichen inzwischen erkennen konnte, duckte sich hinter den Hügel. Tarok spie einen gewaltigen Feuerstrahl und tauchte seine Umgebung in ein Flammenmeer. Gouri fühlte die Hitze an ihr vorbeiwallen, und sie duckte sich noch tiefer in die kühle Erde. Geschrei ertönte irgendwo vor ihr, grauenvolles Geschrei. Ein abrupter Schmerz in ihrem Rücken verriet ihr, dass ihre Mooskleidung zu glühen begann. Rasch rollte sie über die Erde, um das Feuer zu löschen, und der Schmerz verklang etwas. Sie beschloss, noch etwas weiter den Hügel hinunterzurollen, weg von dem Flammeninferno.

Da prallte sie mit jemandem zusammen. Sie blickte auf. Hitze schlug ihr in die Augen, sodass sie diese gleich wieder schliessen musste, doch erkannte sie in diesem Lidschlag Prip und die übrigen Schildzwerge, die ebenfalls unten am Hügel das Flammenmeer ausharrten.

„Tarok, der Rächer?“, hauchte Gouri entsetzt. Sie hatte grausame Geschichten über diesen jähzornigen Drachen gehört.

Die umliegenden Zwerge nickten verzweifelt. Belenor war es schliesslich, der mit zitterndem Schnurrbart das Wort ergriff: „Es ist vorbei. Die Furcht vor ausgeklügelten Fallen im Erdboden war es, der die Drachen in der Luft behielt. Jetzt, wo Tarok hier gelandet ist, und unsere Machtlosigkeit demonstriert, wird ihnen klar werden, dass wir ihnen komplett unterlegen sind. Alle Drachen werden hier landen, und die Drei Wasser vernichten. Sie werden die Feste aus dem Erdboden reissen und in die Tiefen von Cavern eindringen. Wir haben verloren.“

Die übrigen Zwerge nickten traurig. Einzig Prip schien zwar fassungslos, den Mut aber noch nicht verloren zu haben: „Jetzt kommt schon! Tarok spricht hier zwar grosse Worte, aber das dient auch nur unserer Abschreckung. Er will sich bloss seinen versteinerten Freund schnappen. Noch ist nicht alles vorbei!“

Er rüttelte den breiten Krieger an den Schultern: „Wir sind viele! Die Drei Wasser werden fallen, doch wir Zwerge werden uns verstreuen, und wie der Phönix aus der Asche werden wir ein neues Reich errichten. Es gibt immer noch die Eier!“

Gouri blickte wieder auf: „Die Eier?“

Prip nickte mit einem schelmischen Grinsen: „Diese elenden Biester müssen ihre Eier tief unter der Erde vergraben, damit sie schlüpfen. Darum bauen sie grosse Bauten und Höhlen. Aber wir Zwerge sind die Herren des Gesteins, und selbst wenn diese zu gross gewachsenen Eidechsen unsere Gänge mit ihrem heissem Atem versengen, bis sie spucken müssen – solange noch ein einziger wahrer Schildzwerg übrig ist, so werden wir weiterhin die unterirdischen Nester der Drachen ausfindig machen und vernichten.“

Die hochgewachsene Kriegerin warf nun ein: „Wir können mit Stolz verkünden, dass seit dem Ausbruch des Kriegs kein einziges Drachenküken mehr das Licht der Welt erblickt hat. Die hätten uns halt nicht verraten und angreifen dürfen!“

Jetzt war Gouri verwirrt: „War es nicht ein Zwerg, welcher den Krieg ausgelöst hatte?“

Die hochgewachsene Kriegerin schluckte tief und meinte: „Theoretisch ja, es war der Fallenmeister und Meisterschmied Kreatok, der ein Massaker unter den Drachen anrichtete. Aber dem hat es ja auch den Verstand zugenebelt, und wir haben ihn vor Gericht gestellt und zünftig lange eingekerkert. Kreatok ist keiner von uns mehr. Die Drachen jedoch, diese elenden Biester, die haben das als Verrat der Schildzwerge an ihrem Volke interpretiert und sich gegen uns gewendet. Der Rest hier ist reine Selbstverteidigung. Die Schildzwerge und Verrat, was für eine Vorstellung.“ Sie spuckte angewidert auf den Boden.

Gouri blickte auf, überrascht bemerkend, dass sie die Hitze nicht mehr spürte. Das Flammenmeer war versiegt. Die Erde zitterte nicht mehr. Sie guckte nach oben, doch die Hügelkuppe verdeckte die Sicht auf das Schlachtfeld. Hatte Tarok bereits den versteinerten Drachen geschnappt und in Sicherheit gebracht? War etwas vorgefallen?

Da ertönte ein metallischer Klang, eine Mischung aus dem dumpfen Schallen eines Gongs und dem zischenden Erwachen einer Feuerlunte. Gouri schaute verwirrt um sich. Sie konnte nicht erkennen, von wo der Klang stammte.

Auf den Gesichtern der Zwerge tauchten jedoch breite Grinsen auf. Sie erkannten diesen Klang. Prip stemmte sich sogar ohne zu zögern hoch und rannte zur Spitze des Hügels. Gouri wollte ihn anschreien, er solle unten bleiben, doch dann zogen auch die übrigen Zwerge an ihr vorbei, ungeachtet der Gefahr, die immer noch von Tarok ausgehen könnte. Gouri haderte für einen Augenblick, und rappelte sich dann ebenfalls auf, den Hügel hinaufhastend. Oben stehend hatte sie einen guten Blick über das Schlachtfeld. Und was für ein Anblick das war!

Tarok, der riesige Drache, stand immer noch in dem von ihm verursachten Krater neben dem versteinerten Drachen. Rund um ihn war die Erde schwarz und verdorrt, die Gräser verschmort und die Steine gesplittert. Den meisten Zwergen, die sich hinter ihre Schilder geduckt hatten, war es nicht besser ergangen, doch hatte isolierende Kleidung, wie Prip sie trug, vielen das Leben gerettet, und nun sassen sie ächzend in den angeschmolzenen Überresten ihrer Rüstung fest.

Die Drei Wasser standen auch immer noch, und immer noch regnete es ein wenig Lava von der Drachenhorde oberhalb der Wolken.

Doch der Drachenkadaver war nicht mehr. Wo er gelegen hatte, wogte jetzt ein Meer aus schwarzen und silbernen Flammen, die den Drachenkörper bereits verzehrt hatten und an der umgebenden Erde zu nagen begannen. Dieses Flammenmeer war fast so gross wie Tarok selbst, und der Drache fixierte das fremdartige magische Gebilde mit verengten roten Augen, sein Schwanz immer stärker zuckend. War er wütend? War das Furcht? Gouri konnte es nicht sagen.

Und da erkannte sie endlich die Quelle des seltsamen metallenen Geräuschs. Vor dem silbernen Flammenkonstrukt stand ein Zwerg. Nicht irgendein Zwerg, nein, es war derselbe entflohene Zwerg mit den Überresten goldener Ketten an seinen Händen und Füssen, welcher vorhin ihre Gruppe angegriffen und Belenor verletzt hatte. Jetzt stand er mutig da, und stellte sich dem Drachen entgegen. In seiner Hand hielt einen spitzen schwarzen Schild mit wenigen Ornamenten am Rande, auf den er immer wieder mit seiner goldenen Kette schlug, was das metallische Klingen auslöste. Das silberne Feuer wogte und waberte im Einklang mit seinen Schlägen. Kein Zweifel: Das war der legendäre Dunkelschild, der den Unterirdischen Krieg ausgelöst hatte. Der entflohene Zwerg war der Träger des Dunkelschilds. Der Erschaffer des Dunkelschilds. Der entflohene Zwerg war der legendäre Meisterschmied Kreatok. Und er war soeben mit seiner gefährlichsten Kreation wiedervereint worden.

Wie der verwirrte Zwerg, welcher vor Kurzem auf Gouri zugestolpert war und wirr vor sich hin geredet hatte, wirkte Kreatok nun wahrlich nicht mehr. Sein Blick war wieder fest nach oben auf Tarok gerichtet, ein hämischen Grinsen unter dem mächtigen Rauschebart erkennbar. Dann hörte er auf, weiteren Lärm mit seinem Dunkelschild zu machen, und öffnete seinen Mund zu einem Triumphgegröle:

„Meine Brüder und Schwestern, wie sehr habe ich eure Gesellschaft vermisst! Ich bin der Kreatok, den ihr viel zu lange falsch behandelt habt – doch fürchtet euch nicht, ich bin nicht nachtragend. Unter meiner Abwesenheit hat das Volk der Schildzwerge gelitten, so wie auch ich gelitten habe, alleine eingesperrt in diesem Turm, weit weg von meinen schönen Kreationen. Doch frohlocket und jauchzet, denn ich bin zurück, und somit wird alles wieder in Ordnung sein!“

Es war eindeutig nicht alles in Ordnung. Gouri schluckte schwer. Sie wollte sich nach den übrigen Zwergen umsehen, insbesondere Prip, um zu sehen, was sie von dieser Entwicklung hielten. Klar, Kreatok war ein Verbrecher, und ein verrückter noch dazu, doch schien er in Verbindung mit den mächtigen Schilden eine wahre Chance gegen die übermächtigen Drachen zu bieten. Oder war das nur Farce? Er wirkte weniger wahnsinnig als vorhin, doch für wie lange würde das so bleiben? All diese Fragen erhoffte Gouri sich von Prip beantwortet zu kriegen.

Doch Prip war nicht mehr hier.

Die übrigen Zwerge waren nicht mehr hier.

Gouri erkannte sie weiter vorne, bereits tapfer aufs Schlachtfeld rennend, Prip ungeachtet seiner zahlreichen Wunden und blauer Flecken. Sie wollten ihren Freunden helfen.

Von den Rändern der Ebene und aus dem Innern der der Drei Wasser erkannte Gouri weitere Zwerge, die hervorströmten, zu den Gefallenen und Verwundeten, teils mit Tragen, teils mit Heilkräutern, teils mit nichts ausser ihren Kleidern und dem Willen, die verwundeten und angekokelten Schildzwerge in Sicherheit zu tragen.

Das war auch Kreatok aufgefallen, und er schrie weiter: „Ja, meine Brüder und Schwestern, schnappt euch die Verwundeten und bringt sie in Sicherheit. Ich bin hier, um mich wieder in den Dienst meines Volkes zu stellen. Wenn all das hier vorbei ist, dann werde ich hinunter in die Hallen von Cavern steigen, den Thron beanspruchen und die Zwerge der mächtigen Schilde in ein glorreiches Zeitalter der Innovation bringen! Ihr habt zu lange im Schatten der Riesenechsen gewartet, und zu viele von uns leiden IN DIESEM AUGENBLICK unter dem Joch der Riesen aus dem Süden! Wir werden sie alle befreien! Doch zunächst gibt es eine letzte Riesenechse, die mich enttäuscht hat, und mit der es fertigzuwerden gilt.“

Damit richtete Kreatok seine starren Augen wieder auf Tarok, welcher seit dem Erscheinen des Schwarzen Feuers keinen Wank getan hatte: „Tarok, du alter Waschlappen! Ich habe gehört, dein kleiner Sagrak ist nicht mehr nach Hause zurückgekehrt. Das muss schon tragisch sein, auch noch deinen letzten Sohn zu verlieren. Echt, wenn du lieber in dein heimeliges Nest gehen und dich dort ausweinen willst, lasse ich dich bereitwillig ziehen, mein Freund!“

Taroks Gesicht zeigte keine Regung, aber eine Welle der Wut und des Schmerzes brandete abrupt von ihm aus. Gouri war überrascht, wie gut Kreatok Tarok zu kennen schien. Und darüber, dass Sagrak, der Drache, welcher gestern Prip überfallen hatte und von einem Troll getötet worden war, offenbar Taroks Sohn gewesen war. Das machte die Situation auf keinen Fall besser.

„Hast du deine Worte verloren? Schon klar, es kommt nicht jeden Tag, dass man...“, setzte Kreatok erneut an. Doch diesmal unterbrach Tarok seinen Spott und sprach langsam und deutlich ein einziges Wort:

„KREATOK.“

Kurzzeitig flackerte Unsicherheit in Keratoks Stimme auf, ehe er zu einer Antwort ansetzte: „Ja, so haben mich meine Eltern benannt. Worauf willst du hin...“

Erneut wurde er von Tarok unterbrochen, welcher tonlos sprach: „ICH HÄTTE NICHT GEGLAUBT, DASS DU NOCH AM LEBEN BIST. DEINE ZUNGE IST SPITZER GEWORDEN.“

Er wartete einen kurzen Augenblick, und setzte dann nach: „NEHAL VERMISST DICH.“

Gouri konnte von ihrer entfernten Position Kreatoks Gesicht nicht erkennen, doch es kam keine Antwort von Seiten des Zwergs. Dafür flammte das Schwarze Feuer hinter ihm weiter auf, sodass es Tarok nun deutlich überragte. Nehal war der Name von Kreatoks besten Freund gewesen, einem Drachen. Kreatok hatte ihn der Legende nach mit ebenjenem Schild umgebracht, den er nun wieder in seiner Hand hielt.

„KREATOK. DEINE TATEN SIND UNVERZEIHLICH, DARUM SCHWÖRE ICH DIR GLEICH JETZT: SOBALD ICH DIE GELEGENHEIT DAZU KRIEGE, WERDE ICH DICH VERNICHTEN. ABER DEIN VOLK MUSS NICHT WEITER UNNÖTIG LEIDEN. ICH BIN EIN DRACHE VON EHRE. WENN DU DICH ERGIBST, SO LASSE ICH DIE ANDEREN ZIEHEN.“

Kreatok sagte nichts und die Flammen hinter ihm wuchsen weiter in die Höhe und Breite, erreichten die Drei Wasser und leckten an der bröckelnden Aussenmauer.

„KREATOK, SEI VERNÜNFTIG. DEINE ZEIT IST UM. NEHAL HAT DIR SCHON LANGE VERGEBEN. ER WARTET AUF DICH AUF DER ANDEREN SEITE.“

„Es war seine Schuld!“, schrie Kreatok jetzt plötzlich auf. „Er soll mir nicht verzeihen, das lag alles an ihm! Ihm hätte das Feuer nichts angehabt, hätte er sich nicht von mir abgewandt! Ich habe ihn nicht... ich wollte nicht!“

„WAS WOLLTEST DU NICHT, KREATOK?“

„Ich wollte keinen Krieg! Es gab nie ein geheimes Bündnis der Zwerge gegen die Drachen. Ich teilte mein Wissen über die Dracheneier nur aus Notwendigkeit... weil DU, Tarok, uns angegriffen hast. DU mit deiner ewigen Anfeindung und dem Schmerz, den du an uns auslässt. Ich war genauso entsetzt über die tragischen Tode wie du, aber DU musstest ja einen gleich eine ganze Horde der deinigen auf die Besatzung von Karulzar losschicken.“

Kreatok schien wieder etwas sicherer zu werden, aber er wirkte definitiv nicht gesünder, als er ausspuckte: „Ich lag falsch. Es war nicht Nehals Schuld. Es war aber auch nicht meine. Es war deine Schuld! Was nach diesem tragischen Unfall geschehen ist... all diese Tode hast DU auf dem Gewissen!“

„DU HAST IN DEN FLAMMEN GESUNGEN UND GETANZT! IN DEINEN KÜHNSTEN TRÄUMEN SAH DAS NICHT NACH EINEM UNFALL AUS!.“

Kreatok hielt inne. Er setzte zu einer Antwort an: „Das... das war nicht... der Schild, er trägt Dunkelheit in sich... sobald ich ihn einmal abgelegt hatte, da...“

Der Meisterschmied blickte den Schild an, den er immer noch festhielt, und hielt ihn etwas zur Seite, als würde er ihn jeden Augenblick fallen lassen.

Und Gouri sah, wie Tarok seinen Hals aufrichtete und tiefer ein- und ausatmete. Sie erkannte den Plan des Drachen. Sobald Kreatok den Dunkelschild loslassen würde, würde Tarok dessen Leben wie eine Kerze auspusten. Und dann wären die Seiten im Krieg wieder im Ungleichgewicht. Die Schildzwerge würden ausradiert werden. Prip, welcher gerade verzweifelt versuchte, eine am Boden liegende Kameradin vom Schlachtfeld wegzuzerren, würde sterben, wenn nicht durch das Feuer, dann durch die Trolle, welche nach jedem Scharmützel im Grauen Gebirge die Verletzten zusammensammeln kamen.

Das durfte sie nicht zulassen.

Doch was konnte sie schon tun?

Und in diesem Augenblick schien die Zeit sich zu verlangsamen, und Gouri erkannte etwas Glitzerndes auf Taroks Rücken. Sie schaute genauer hin. Zwischen zwei Stacheln, wie sie zu Dutzenden aus seiner Wirbelsäule herausstachen, sass sicher eingeklemmt ein bläulich glänzender Schild.

Belenor hatte vorhin erwähnt, dass der Sternenschild an die Drachen gefallen war. Konnte es sein, dass Tarok ihn mit sich gebracht hatte, als letzter Schutz gegen das Feuer des Dunkelschilds?

Wenn der Dunkelschild die Verkörperung des Finsteren und Verdorbenen war, so war der Sternenschild die Verkörperung des Guten und Reinen. Wenn nur jemand an den Sternenschild gelangen könnte, so könnte man damit das Schicksal vielleicht wieder ins Lot wenden. Doch dafür müsste jemand zu Tarok, auf dessen Rücken klettern, und den Sternenschild stibitzen, ehe die riesige Echse etwas merkte. Das war eine Tat, die an Unmöglichkeit angrenzte. Und dazu kam noch, dass niemand in der Nähe war, dem Gouri diesen Plan überhaupt mitteilen könnte. Sie selbst war wahrlich nicht dafür geeignet, so unerfahren und schwach, wie sie war. Gouri war kurz vor dem Verzweifeln – da tauchte plötzlich vor ihrem inneren Auge das Gesicht eines freundlich glucksenden Trolls auf.\bigskip



Der Troll erwachte aus seinem steinernen Schlaf und gähnte ausgiebige. Langsam rappelte er sich auf und räkelte sich im Licht der Morgensonne.

Er hatte keinen Namen.

Die niederen Trolle benötigten keinen Namen.

Er hatte nur seinen Clan, und die Bedürfnisse seines Clans kamen vor seinen eigenen.

Seltsam, dass er etwas abseits des Clan-Lagers eingenickt war. Sein Magen grummelte, aber das war es nicht, was ihn geweckt hatte. Als er auf seine Hand blickte, lagen darin die Überreste eines verkohlten Schildes.

Ach ja, genau! Der Troll hatte letzte Nacht zwei wirklich nette Gesellen getroffen, die seine Brücke überquert hatten, und diese Gesellen hatten ihm die verkohlten Überreste eines mythischen Schilds geschenkt. Der Troll grinste selig. Wahre Freunde waren das gewesen. Nicht weggerannt waren sie, wie es sonst alle taten, sondern auf ihn zugegangen und mit ihm gespielt hatten sie. Er gluckste schon allein von der Erinnerung an das Treffen. Leider waren sie davongerannt, ehe es Zeit fürs Essen geworden war – sie hatten so lecker ausgesehen! Und leider hatten ihre mythischen Schildüberreste nichts Tolles drauf gehabt.

Aber das machte nichts. Schliesslich besass sein Clan noch immer den braunen Schild mit dem Händeschütteln drauf. Und das war ein wahrlich toller Schild, das hatte der Troll schon von weitem gespürt.

Da tauchte vor dem inneren Auge des Trolls das Gesicht des einen dieser Gesellen auf, ein grünes Zwerglein mit einer Kleidung aus Moos war das gewesen. Und diese Gestalt schwebte nun in seiner Vorstellung rum und grinste ihn an. Das war ja ungewöhnlich. Was noch ungewöhnlicher war: Die Gestalt gestikulierte in eine gewisse Richtung – weg von der Brücke, und hin zu seiner Clanhöhle.
  
Fröhlich folgte der Troll dem grinsenden Gesicht der grünlichen Gestalt in die Höhle seines Clans. Der Wächter-Troll guckte ihn seltsam an, aber unserem Troll war das egal. Es fühlte sich gut an, diesem Gesicht zu folgen. Der Clan-Häuptling guckte noch dümmlicher drein, als der Troll einfach so in seinen Schlafraum stapfte und zum dort an der Wand hängenden mythischen Schild trat, aber auch das war unserem Troll egal. Unser Troll fühlte sich sicher und geleitet. Das Lächeln der grünen Person vor seinem inneren Auge bestätigte ihn in seinem Tun. Ein letztes Mal blickte er ihr in die Augen, diesmal fragend. Das Gesicht nickte, und so griff der Troll nach dem Schild und hob ihn in die Höhe und –

– da fühlte der Troll seine Knie weich werden und nachgeben. Was für finstere Hexenkunst steckte in diesem Schild? Der Troll wurde von einem Augenblick zum nächsten immer schwächer und schwächer! Schon mochte er den verfluchten Schild nicht mehr halten, und der Schild kullerte davon und rollte seinem ihn immer noch mit grossen Augen anstarrenden Clan-Häuptling vor die Füsse.

Unser Troll wollte nach dem Schild greifen, doch seine Arme waren zu schwer, viel zu schwer, und plumpsten nutzlos auf den Höhlenboden. Dann tat es ihnen sein restlicher Köper nach.

Da lag der Troll nun zusammengesunken am Boden. Selbst das Atmen kam ihm wie eine riesige Anstrengung vor. Was war nur los hier? Das immer noch freundlich lächelnde Gesicht der grünlichen Gestalt verblasste.\bigskip



Gouri spürte die Veränderung kommen, ehe sie eintraf. Wie die Spannung, die vor einem Gewitter in der Luft liegt, war ihr Körper aufgeladen und bereitete sich auf die einströmende Kraft vor. Breitbeinig stand Gouri oben auf der Hügelkuppe und starrte in die Ferne, das Gesicht des freundlichen Trolls immer noch im Kopf behaltend. Da spannten sich plötzlich ihre Muskeln unverhältnismässig an, und ein Blitz aus reiner Energie zuckte durch ihren Körper, drang in jede Faser ihres Wesens und erfüllte sie von Kopf bis Fuss. Gouri schrie schmerzerfüllt auf.

Dann war es vorbei.

Aber alles hatte sich verändert.

Gouri fühlte sich so leicht, so leicht, wie sie sich noch nie in ihrem Leben gefühlt hatte, leicht wie eine Feder, und doch so standhaft wie das Graue Gebirge selbst. Ihr Atem ging rasch, und sie fühlte, als könnte sie Bäume ausreissen. Konnte sie wahrscheinlich tatsächlich.

Ein Hoch darauf, dass sie sich an den Aufenthaltsort des Bruderschilds erinnert hatte. Ein Hoch auf den netten Troll, der sie in einem so positiven Licht sah, dass er ihr seine Stärke geliehen hatte. In gewissem Sinne auch ein Hoch auf Kreatok und Nehal, die diesen so mächtigen Bruderschild geschaffen hatten. Gouri jubelte. Sie war bärenstark, nein, sogar trollstark! Jetzt galt es, den Sternenschild von Taroks Rücken zu erobern.

Gouri duckte sich, holte Anlauf, und tat einen SPRUNG, welcher sie meterweit in die Luft katapultierte. Windstösse trafen sie im Gesicht, und sie ruderte ungeschickt mit den Armen, ehe die grasbewachsene Erde unter ihr allzu schnell wieder auf sie zukam und sie unschön zu Boden krachte. Gouri kullerte den Hügel hinunter, wobei sie sich noch mehr Schürfwunden an den Armen und Beinen zuzog.

Ächzend richtete Gouri sich wieder auf und strich sich Matsch aus dem Gesicht.

Ein grossartiger Start. Aber positiv bleiben: Sie war immer noch stark wie ein Troll, einige Schürfwunden würden ihr da nichts ausmachen.

Gouri versuchte es erneut, diesmal etwas vorsichtiger. Selbst das Laufen war schwer, da ihr Körper jede Bewegung viel schneller vollzog, als sie es erwartete. Aber nach einigen Schritten hatte Gouri den Dreh soweit raus, damit sie ein etwas schnelleres Tempo einlegen konnte. Und so begann Gouri zu rennen.

Gute Güte, und wie sie begann, zu rennen! Die Umgebung um sie verzerrte sich zu grünen und braunen Farbschlieren, als Gouri einen weiteren Zahn zulegte. Schemen von verletzten und einander helfenden Zwergen huschten links und rechts an ihr vorbei, einige drehten sich erstaunt nach ihr um. Sie mochten viel Schauriges erlebt haben in den letzten Tagen, doch eine Agren mit der Stärke eines Trolls und der daraus resultierenden Geschwindigkeit sah man auch nicht alle Tage.

Gouri wagte einen kleinen Hopser, und schaffte es diesmal, nach der Landung auf den Beinen zu bleiben. Sie verlor kurzzeitig den Rhythmus, wurde langsamer, fand die Kontrolle wieder und beschleunigte auf ein noch höheres Tempo als vorher, stetig auf den riesigen schwarzen Drachen zu.

Kreatok und Tarok warfen sich weiterhin Anschuldigungen an den Kopf, doch Gouri kümmerte sich nicht mehr gross um den Inhalt. Kreatok hatte den Dunkelschild immer noch nachdenklich in der Hand und Tarok war immer noch komplett auf ihn fokussiert. Soweit, so gut.

Gouri wagte einen noch grösseren Satz, dann noch einen, und diesmal erreichte sie wieder eine Höhe von mehreren Metern. Das machte beinahe Spass!

So näherte sich Gouri hüpfend, immer höher hüpfend, dem riesigen Drachen. Sie sprang in hohem Bogen über den einem anderen Schildzwerg helfenden Prip, welcher sich aufrichtete und sie mit grossen Augen anstarrte. Von weit hinten hörte Gouri ihn etwas rufen, doch der Wind verzerrte es, sodass sie es nicht hören konnte. Kurz überlegte Gouri, stehenzubleiben, und gelang ins Stolpern. Dann rief sie sich ins Gedächtnis: Der Sternenschild. Sie musste zum Sternenschild, solange Kreatok den Dunkelschild noch nicht losgelassen hatte.

Nun war Gouri kaum mehr eine Drachenlänge von dem gigantischen Tarok und dem noch viel gigantischeren schwarz-silbernen Feuersturm Kreatoks entfernt. Die Hitze schlug ihr entgegen und brannte in ihren Schürfwunden. Ihre Schuhe hatte sie längst abgelaufen, und ihre Füsse würden die Tortur auch nicht mehr lange mitmachen. Aber weit war es nicht mehr, nur noch zwei, drei Schritte –

– da riss Kreatok mit einem inbrünstigen Schrei den Dunkelschild in die Höhe und der schwarze Flammensturm setzte sich in Bewegung, umströmte ihn und raste auf Tarok zu, welcher laut aufbrüllte und seinerseits einen feuerroten Flammenstrahl in Richtung des Schildzwergs sandte – und funkelten da etwa Sterne in Taroks Strahl?

Die Feuerwände krachten zusammen in einem Schauspiel, welches zugleich unglaublich furchterregend und wunderschön majestätisch aussah. Silberne Flammen trafen auf orange, schwarze auf rote, und die Feuer vermischten sich in Wirbeln und Strudeln, verglühten die letzten Reste des Grases zwischen dem Schildzwerg und dem Feuerdrachen und tauchten die Umgebung in ein gespenstisches rotes Licht. Silbern glitzernder Rauch stieg in die Höhe und verknüpfte sich mit den schneeweissen Wolken, über denen immer noch Drachen flogen und Lava auf die Drei Wasser niederregnen liessen.

Gouri war zu spät, um auszuweichen. Sie flog mitten im Sprung auf die vielfarbige Feuerwand zu, während die Hitze sich auf ein unerträgliches Mass steigerte.

Sollte sie noch versuchen, ihre Geschwindigkeit zu bremsen? Falls sie das überlebte, könnte Gouri sich zurückziehen und später einen erneuten Versuch machen, den Sternenschild auf Taroks Rücken zu erringen. Doch würde das Schicksal ihr eine zweite Chance geben?

Sie durfte den Troll nicht vergessen, dessen Stärke sie im Moment führte. Dieser Troll lag gerade hilflos zusammengesunken auf einem Höhlenboden nicht weit von hier entfernt, und brauchte dringend seine Kraft zurück. Sobald Gouri ihm diese zurückgegeben hatte, würde sie sie wohl nicht wiedererhalten können. Es hiess jetzt oder nie.

Gouri dachte an Prip und seine Zwerge, die sich selbstlos auf das Schlachtfeld gewagt hatten, um ihre Mitzwerge zu unterstützen. Ihre Bemühungen wären alle für nichts, wenn Tarok und Kreatok jetzt einen Feuersturm über das Schlachtfeld entfachen würden. Prip würde sterben. Belenor mit seinem prächtigen Schnurrbart würde sterben. Gouri selbst würde sterben, wenn sie nicht schleunigst Leine zog.

Und so, mitten im freien Fall, traf Gouri ihre Entscheidung.

Sie prallte auf dem Boden auf und sprang erneut hoch, viel zu schnell, viel zu weit, viel zu nahe an den tobenden Kampf der Feuer. Ungehindert schoss Gouris Körper direkt auf das Flammenmeer zu. Sie schloss die Augen und bereitete sich auf ihr Ende vor.

Ein kühler Luftstoss traf Gouri im Gesicht und fegte die Hitze davon. Ungläubig öffnete sie die Augen, gerade rechtzeitig für den nächsten Sprung. Tarok sass nicht mehr auf der Erde, sondern hatte seine riesigen Flügel aufgespannt und schwang sie würdevoll, sich langsam in die Lüfte erhebend.

Das schwarze Feuer konnte ihm gefährlich werden. Der Feuerdrache wollte sich in Sicherheit begeben. Und durch das Bewegen seiner Flügel sorgte er für frische Luftströme, die Gouri ganz nebenbei die rettende Kühle verschafften. Sie dankte der Mutter des Steins, und legte die letzten paar Meter in einem mächtigen Sprung zurück.

Im Nachhinein hätte sie besser zielen können. Aber Gouri war sowohl geistig als auch körperlich an ihrem Ende, und hatte sich nur auf den Sternenschild fokussiert, darauf, dass der Schild gleich aus ihrer Reichweite verschwinden würde und dass sie ihn vorher erreichen musste. So hatte sie zum letzten Sprung angesetzt und sich weit in die Lüfte erhoben, an Taroks Beinen, Brust und selbst über seine Kopfhöhe hinweg, direkt auf seinen Rücken zu, wo zwischen zwei spitzen Stacheln weiterhin der Sternenschild eingeklemmt war. Aus der Nähe glitzerte er noch viel schöner.

Gouri lächelte.

Dann prallte sie auf den Rücken des Tarok und wurde gleich wieder meterhoch in die Luft geschleudert, knallte an einen seiner mächtigen Flügel, der sich gerade am Heben war, und rutschte daran herunter, bis sie am unteren Flügelansatz sass, beinahe nicht mehr bei Bewusstsein.

Gouri, die Agren, war auf Tarok gelandet.

Hatte er sie bemerkt? Ein rascher Blick verriet ihr, dass Tarok immer noch viel zu sehr damit beschäftigt war, einen Feuerstrahl nach dem anderen auf Kreatok abzusenden, als dass er sich darum kümmern konnte, was für mickrige Wesen da auf seinem Rücken landeten.

Gouri hatte Mühe, die Situation zu begreifen. Sie lag auf dem Rücken von Tarok. Sie lag auf dem Rücken eines vom abhebenden Drachen. Man könnte argumentieren, dass sie einen Drachen ritt – sie, eine einfache Agren aus dem Grauen Gebirge!

Durch den Rauch, welcher vom Drachenfeuer und vom Feuer des Dunkelschilds hoch in die Lüfte stieb, konnte Gouri den Boden nur noch schwer erkennen, aber es war klar, dass er sich von ihr entfernte. Mit mächtigen Schwüngen seiner breiten Flügel schraubte Tarok sich weiter in den Himmel hervor, weiter weg vom riesigen silbernen Feuer, welches Kreatok beschworen hatte. Gouris Höhenangst schlug rasant Alarm, also wandte sie sich vom Blick nach unten ab und drehte sich mühsam auf den Rücken.

Der Blick nach oben löste in ihr fast noch mehr Furcht aus. Über ihr kreisten mindestens fünf Drachen, welche aufgehört hatten, Lava auf die Drei Wasser regnen zu lassen, und stattdessen Taroks Kampf mit Kreatok aufmerksam beobachteten. Konnten die Drachen sie vielleicht erkennen?

In diesem Moment lief Gouri ein weiterer Schauer über den Rücken, und sie sackte etwas zusammen. Ihr Muskeln entspannten sich kurz und verspannten sich gleich wieder, als eine Welle reinen Schmerzes ihren Körper ergriff.

Gouri brauchte einige Augenblicke, um zu erkennen, dass es sich dabei nicht um einen telepathischen Angriff eines Drachen handelte. Nein, es war die Kraft, die sie vom Troll erhalten hatte. Die Stärke des Trolls verliess ihren Körper und ihre eigene Verletzlichkeit trat wieder zum Vorschein. Gouri schaute auf ihre Hände, welche stark zitterten und ihr nicht mehr zu gehorchen schienen. Ihr Blickfeld verengte sich und die Welt wurde dunkel. Blind warf Gouri sich nach vorne, auf den mächtigen Sternenschild zu, welcher nur wenige Schritte von ihr entfernt zwischen den Rückenstacheln des Tarok steckte.

Sie verlor das Bewusstsein nicht. Sobald ihr Körper den Schild berührte, ertönte ein leises metallisches Summen, und der Schleier vor ihren Augen lichtete sich. Die Schmerzen liessen nach. Der Sturm von Taroks Flügeln wurde leiser und leiser. Und Gouri fühlte nicht mehr Taroks erhitze Schuppen unter ihren Händen, sondern festen, kalten Stein.

Verwirrt blickte Gouri sich um. Sie erblickte dunkle Äste und Blätter auf allen Seiten, weiter vorne sogar einen riesigen Stamm. Offenbar befand sie sich unter einem steinernen Mammutbaum, höher als alle Burgen, die sie in ihrem Leben je gesehen hatte, so gross, dass seine Blätter den Himmel verdeckten. Schimmernde Edelsteine überzogen den steinernen Baum, und gleissendes bläuliches Licht strahlten diese aus. Dieses Licht umgab Gouri, und dieses Licht spiegelte sich im glänzenden Sternenschild, den sie immer noch in der Hand hielt.

Tatsächlich: Sie besass den Sternenschild! Eines der mächtigsten Artefakte, von denen sie aus zahlreichen Legenden und Sagen gehört hatte. Sagen und Legenden, denen sie als Kind begeistert gelauscht und die sie in sich aufgesogen hatte. Und nun war sie selbst mitten in einer Sage.

Weiter vorne, nahe beim Stamm, rührte sich etwas. Eine seltsame Kreatur mit Hornklauen anstelle von Händen löste sich aus einem Loch im Stamm des riesigen Steinbaums und krabbelte geschickt über den kalten Stein auf Gouri zu. Die Kreatur erreichte sie jedoch nie, denn in jenem Augenblick schoss eine gewaltige Gestalt an ihr vorbei und landete zwischen der Kreatur und Gouri.

Es war ein Drache, wie konnte es auch anders sein.

Gouri erkannte erst jetzt die übrigen fliegenden Drachen, wie sie weit über ihr den Steinbaum umkreisten. Viele von ihnen schenkten ihr keine Beachtung, aber manche verdrehten die langen Schlangenhälse, um sie zu erkennen. Einer der Drachen, welcher besonders schnell den Baum umkreiste, erinnerte Gouri ungut an Tarok – und wenn sie plötzlich an diesem Ort sein konnte, so war dieser Drache vielleicht auch Tarok selbst.

Als es vor ihr schnaubte, wandte Gouri sich rasch wieder dem Drachen zu, welcher sich auf dem Boden niedergelassen hatte und sie aufmerksam studierte:

„Der Sternenschild?“, fragte eine tiefe, aber nicht unfreundliche Stimme in ihrem Kopf, „Ich hätte nicht gedacht, dass ich ihn so bald schon wiedersehen würde. Du brauchst dich nicht zu fürchten, kleine Agren. Solange dein Geist hier in Krahal weilt, können wir dir nichts tun. Also setzen wir uns hin und reden über diese... Angelegenheit.“

Der Drache strahlte eine gewisse Ruhe und Gelassenheit aus, und Gouri spürte, dass er die Wahrheit sagte.

Tarok (falls es denn Tarok war) hatte sich Fleckchen Stein, auf dem Gouri und der fremde Drache sassen, nun ebenfalls angenähert, und schwebte elegant in der Luft, jedoch ohne sich niederzulassen. Er überragte den ruhigen Drachen um mindestens das dreifache.

„DU!“, zischte Tarok zu Gouri, „Du hast meinen Sohn Sagrak auf dem Gewissen!“ Doch zu einem physischen Angriff liess Tarok sich nicht hinreissen.

Der Sternenschild summte plötzlich auf, und Gouri spürte eine Welle des Schmerzes von Tarok ausgehen. Erlaubte der Schild ihr, die Emotionen der Drachen in Krahal zu teilen? Sie spürte einen vertrauten Geist aus der Höhe, und als sie ihren Hals verdrehte, erkannte sie Sagrak, wie er weit über ihr auf einem steinernen Ast sass und sie aufmerksam bespitzelte. Waren die meisten Drachen, die sie hier sah, nur Abbilder bereits gestorbener Drachen? Waren es ihre Geister? Gouri hatte Geschichten von Krahal gehört, doch jetzt, wo sie den Ort mit eigenen Augen sah, war es einfach absolut überwältigend. Was tat ihr Körper momentan in der realen Welt? Und was Tarok?

Als hätte er ihre Gedanken erraten, antwortete der ruhige Drache nun:

„Keine Sorge, du stürzt nicht gerade vom Himmel. Tarok vermag es, seine Flügel zu schwingen, auch wenn sein Geist gerade in Krahal ruht.“

Mit einer gekrümmten Kralle deutete der ruhige Drache in die Höhe. Was Gouri zunächst für den vom riesigen Steinbaum verdeckten Himmel gehalten hatte, wirkte nun wie eine Höhlendecke. Doch das konnte natürlich nicht sein, schliesslich hatte eine so grosse Höhle unmöglich unter der Erde Platz.

Wichtiger war jedoch, was sich an dieser möglicherweise-Höhlendecke abspielte: Darauf erkannte Gouri eine sich bewegende Wandmalerei, ein Abbild ihrer selbst, wie sie auf dem Rücken Taroks lag, den Sternenschild fest umklammert. Dieses Abbild von Tarok bewegte sich nicht gross, aber seine Flügel schlugen ununterbrochen. Durch die Decke konnte Gouri erblicken, was sich soeben in der realen Welt abspielte.

Der ruhige Drache seufzte, und sprach: „Solange du den Sternenschild trägst, gebietest du darüber, wer Krahal sonst noch zu betreten vermag. Ich bitte dich darum inbrünstig: Wärst du so lieb, Kreatok hierhin zu bringen?“

Gouri atmete tief ein und aus und streckte ihre Hand schützend aus: „Ich... ich brauche kurz einen Moment. Das... das ist alles etwas zu viel.“

Der ruhige Drache seufzte wieder und liess sich zu Boden sinken, die Vorderpfoten elegant gekreuzt: „Du hast alle Zeit der Welt. Naja, ‚alle Zeit‘ ist vielleicht etwas übertrieben. Aber jedenfalls genügend Zeit.“

Tarok, welcher vorhin noch unruhig über ihnen beiden hin- und hergetigert war, atmete tief ein und aus und liess sich dann auch zu Boden fallen. Die Erde zitterte unter Gouris Füssen und sie fühlte sich plötzlich ganz klein, direkt vor der Schnauzen zweier Drachen, die sie anstarrten.

Der ruhige Drache wandte sich Tarok zu und fragte vorsichtig: „Bist du sicher, dass hilfreich ist, wenn du dich so nahe befindest...“

Tarok drehte seinen Kopf dem ruhigen Drachen entgegen und öffnete seinen Mund leicht, ein furchterregendes Knurren ausstossend, woraufhin der viel kleinere ruhige Drache unterwürfig den Kopf senkte und Tarok gewähren liess. Hitzewellen entwichen Taroks Maul und trafen Gouri im Gesicht, die sich abzuwenden hatte. Erneut fragte sich, wie sicher sie hier in Krahal wirklich war. Was würde geschehen, wenn dieser Tarok – sein Geist – sich entscheiden würde, sie zu beissen? Würde sie in der echten Welt aufwachen oder konnte sie hier drin sterben?

Plötzlich glomm der Sternenschild in einem beruhigenden blauen Licht auf, und Gouri überkam die Gewissheit, dass sie im Moment nicht gefährdet war. Sie war sich absolut sicher, dass Tarok sie nicht verletzen konnte. Der Schild würde sie beschützen. Der Schild hatte Hoffnung in ihr geweckt.

Der ruhige Drache blickte vom wütenden Tarok auf und fast so etwas wie Stolz überkam ihn, als er den leuchtenden Schild ansah und sprach: „Ach... ist er nicht wunderschön? Von den vieren war er mir stets der liebste.“

 Gouri sah ihn überrascht an. Der ruhige Drache sprach weiter: „Verzeih mir, in diesem ganzen Klamauk hatte ich mich noch gar nicht vorgestellt: Ich bin Nehal. Ich war derjenige, der mit Kreatoks Hilfe die vier mächtigen Schilde hergestellt hatte. Und auch wenn Kreatok jeweils den Silberschild benutzt hatte, um nach Krahal zu gelangen, denke ich, dass mit geschickter Führung auch eine Trägerin des Sternenschilds seinen Geist hierhin bringen könnte. Kannst du das für uns tun, edle Dame?“

Gouri starrte Nehal an. Der Erschaffer der mächtigen Schilde! Kurz hatte sie das Verlangen, sich zu verbeugen – sie hatte bei Prip abgeschaut, wie das ging – doch dann liess sie es sein. Sie hatte einhundert Fragen und nochmals einhundert mehr, doch das war nicht der Zeitpunkt, erst recht nicht vor dem grimmig dreinschauenden Tarok. Somit begnügte sich Gouri damit, ein leises „Ich bin Gouri“ zu stammeln und dann zu fragen: „Wie... wie kommt es, dass du... dass Sie... hier...“

„Ah, wir Drachen sind Seelen der Erde und des Feuers, wir sind nicht einfach so auszulöschen, indem man unsere Körper vernichtet“, lächelte Nehal, „Solange noch einer einziger von uns übrig ist, leben wir in dessen Kopf weiter. Krahal besteht weiter, solange noch einer von uns bei klarem Verstand ist“ – bei diesen Worten schaute Nehal warnend auf Tarok – „und das wird wohl noch jahrtausendelang der Fall sein.“

„KOMMT KREATOK JETZT ODER NICHT?“, warf Tarok ungeduldig ein.

Gouri und Nehal verstummten umgehend.

„Wenn ich Kreatok hierhin bringe, könnt ihr ihm nichts tun?“, fragte Gouri nochmals nach.

„Ja, Träger der mächtigen Schilde sind hier sicher. Wir wollen nur reden.“, antwortete Nehal, und Gouri spürte abermals durch den Sternenschild in ihrer Hand, dass Nehal dies als die reine Wahrheit ansah.

So weit, so gut. Nun musste Gouri nur noch Kreatok nach Krahal holen. Sie hatte zwar keine Ahnung, wie sie das anstellen sollte, aber sie hatte Vertrauen in die mächtigen Schilde – bislang hatte sie ja auch nicht mehr Ahnung gehabt, wie die Schilde genau einzusetzen waren.

Und prompt summte der Sternenschild metallisch, glomm leuchtend auf und in Gouris Nähe verdichteten sich die Nebelschwaden. Heraus trat Kreatok, der Meisterschmied, den Dunkelschild immer noch in der Hand. Das lief ja wie am Schnürchen.

Im Gegensatz zu Gouri wirkte Kreatok nicht überrascht, in Krahal zu stehen. Sein gefasster, fast gelangweilter Gesichtsausdruck deutete darauf hin, dass er bereits einmal hier gewesen war. Das machte auch Sinn, schliesslich hatten er und Nehal die mächtigen Schilde geschaffen und wussten ziemlich sicher über alle Tricks Bescheid, die diese auf Lager hatten.

Da er nicht über den riesigen Baum inmitten Krahals und die Unmenge an herumfliegenden Drachen staunen musste, konnte Kreatok seine Aufmerksamkeit direkt auf Tarok richten. Tarok rührte sich nicht vom Fleck, aber sein Schwanz zuckte immer wieder und seine Augen verengten sich – wenn Gouri die Körpersprache des riesigen Reptils richtig deutete, stellte er sich gerade vor, wie nett es wäre, den kleinen Zwerg zu rösten.

Kreatoks Mine verfinsterte sich. Er nickte Gouri mit einem kalten Gesichtsausdruck kurz zu und setze zu einem Spruch an – wahrscheinlich wollte er schnippisch fragen, warum sie ihn nach Krahal gebracht hatte.

Dann fiel Kreatoks Blick auf Nehal, und er blieb mit halb geöffnetem Mund stehen. Beinahe sofort begannen seine Augen zu glänzen. Tiefe Furchen erschienen in seiner gerunzelten Stirn und sein Kinn zitterte, sodass sein ganzer Bart zu wackeln begann. Ja, Gouri fiel auf, dass sein Bart, welcher in der Welt da draussen schon längst versengt worden war, hier in Krahal wieder in voller Pracht vorhanden war. Was für ein mysteriöser Ort.

Nehal erwiderte Kreatoks Blick aus rot glühenden Augen, aber im Gegensatz zu Taroks leuchteten sie nicht bedrohlich, sondern warm und einladend, wie der Schein eines warmen Lagerfeuers. Nehal rief seinen Mörder beim Namen, in einer Tonlage, in welcher man sonst einen alten Freund begrüsste: „Kreatok...“

Eine einzelne Träne löste sich aus Kreatoks Augenwinkel und rollte seine Wange entlang. Sein Gesichtsausdruck zeigte nicht mehr den verwirrten entflohenen Gefangenen, als den Gouri ihn kennengelernt hatte. Und er war auch nicht mehr der kühle, spöttische Zwerg, der sich Tarok in den Weg gestellt hatte. Kreatok wirkte jetzt wie jemand, der jahrelang in der Dunkelheit gesessen hatte und nun einen Strahl Sonnenlichts erblickte. Wie jemand, dessen Kind an einer schweren Krankheit gelitten hatte und der nun von dessen Heilung erfuhr. Wie jemand, der seinen besten Freund im Affekt umgebracht hatte und ihn nun wieder erblickte, nicht wütend oder enttäuscht, sondern einfach... war das Vergebung, das Nehals Gesicht ihm versprach? Konnte ihm überhaupt vergeben werden?

Gouri spürte abermals durch den Sternenschild, dass Kreatok der Lage noch nicht traute. Er selbst hatte sich nie verziehen und hatte in den Jahren der Einsamkeit in den Kerkern der Drei Wasser gelitten, als er sich immer und immer wieder vorgestellt hatte, was Nehal nur von ihm denken musste. Kreatok hatte beim Schmieden des Dunkelschilds, von Nehal unbemerkt, machthungrig eine Dunkle Quelle angezapft, und diese hatte ihn erfüllt, begleitet, geleitet, bis er nicht mehr sie kontrollierte, sondern sie ihn. Erst als er Nehals Schreie gehört hatte, war Kreatok wieder aufgewacht, und hatte den Dunkelschild von sich gestossen. Doch zu diesem Zeitpunkt war es bereits zu spät gewesen. Nehals Körper war vernichtet und Kreatok hatte den Krieg zwischen Zwergen und Drachen ausgelöst.

Ein Bild blitzte vor Gouris innerem Auge auf: Die rauchenden Überreste Nehals, wie sie umgeben von silbernem Feuer in einer Höhle tief unter Cavern brannten. Sie wusste, dass dieses Bild in Kreatoks Erinnerung festgebrannt war, und wie schon seit vielen Jahren kämpften in seinem Innern zwei Seiten.

Die eine Seite Kreatoks wollte den Dunkelschild wieder von sich werfen, so weit weg, wie nur irgend möglich, und zu Nehal rennen, ihn umarmen, ihm sagen, dass es ihm so unglaublich leid tat. Doch Worte konnten dem, was er getan hatte, nicht gerecht werden.

Die andere Seite Kreatoks konnte den Schild nicht loslassen. Kreatok war nach so langer Zeit zum ersten Mal wieder mit einer seiner Kreationen vereint, und sie war seine Lebensversicherung. Ohne Schild würden die Schildzwerge ihn wieder gefangen nehmen, oder schlimmer, Tarok ihn direkt vernichten. Nein, schon die ganze Angelegenheit hier musste ein Trick sein. Nehal konnte in ihm nur den Mörder sehen, den er selbst in sich sah. Sobald er den Schild losliess, würden sich die beiden Echsen auf ihn stürzen und seinen Geist zerreissen. Er brauchte den Dunkelschild.

Und so blieb Kreatok stehen, den Dunkelschild immer noch fest gepackt, und starrte Nehal entgegen, immer mehr Tränen in den Augen.

Auch von Nehal gingen jetzt Wellen der Trauer aus, als er seinen alten Freund vor sich anblickte, wie er zitterte und den Dunkelschild umklammerte. Erneut sprach er seinen Namen, diesmal mit einem leichten Anschwung von Tadel in der Stimme: „Kreatok... du brauchst diesen Schild doch nicht.“

Immer noch strahlte Nehal Vergebung und Freude aus, als er hinzufügte: „Komm, alter Freund. Du musst nicht mehr leiden.“

Nehal hatte seine Emotionen nicht mehr vollständig unter Kontrolle und für einen kurzen Augenblick fühlte Gouri, wie viele Dinge Nehal noch sagen wollte und nicht auszudrücken vermochte, weil Worte nicht genug waren. Es muss dir nicht leidtun, Kreatok. Du bist nicht das Böse hier. Es wird alles in Ordnung sein.

Kreatok spürte diese Dinge auch, und er war so nahe davor, den Dunkelschild loszulassen. Der Sternenschild in Gouris Händen summte ein letztes Mal auf und strahlte heller denn je zuvor, als die Agren alle ihre Zuversicht auf den Schildzwerg lenkte, wie er da vor ihr, Nehal und Tarok stand und mit sich selbst haderte.

Etwas in Kreatok brach und er murmelte: „Was soll’s.“

Dann öffnete er seinen Griff und der Dunkelschild entschlüpfte ihm, zu Nebel verpuffend, sodass kein Abbild des Schildes mehr in Krahal blieb. Gouri fiel auf, dass Kreatok auch hier in Krahal noch die Überreste goldener Ketten trug. Sie brauchte nicht nach oben in die echte Welt zu blicken, um zu wissen, dass der wahre Kreatok soeben den Dunkelschild losgelassen hatte.

Die Schlacht um die Drei Wasser war vorüber.

Nehal lachte fröhlich auf.

Kreatok schluchzte auf und rannte auf Nehal zu, die Arme zu einer Umarmung ausgestreckt.

Er sollte nie dort ankommen.

Tarok, der sich die letzten Minuten unnatürlich still verhalten hatte, bäumte sich brüllend auf und spie einen mächtigen Feuerstrahl auf den noch rennenden Kreatok, welcher innerhalb eines Augenblicks Krahal verliess und zu Nebel verpuffte. Erneut musste Gouri nicht nach oben in die reale Welt sehen, um zu wissen, dass Tarok soeben das Feuer auf Kreatok eröffnet hatte und Kreatoks Körper keine Chance gehabt hatte.

Gouri schrie auf: „NEIN!“

Nehal blickte wie versteinert auf Tarok. Hatte er gewusst, dass dies Taroks Plan gewesen war? Hatte er dies etwa unterstützt?!

Sie sollte die Antwort nie erfahren, denn in diesem Augenblick schrie Tarok: „DAS IST ES, WAS IHR VERRÄTER VERDIENT!“

Und der riesige Tarok stürzte auf Nehal zu, packte den kleineren Drachen im Sprung und hob schleuderte ihn in die Höhe. Nehal prallte überrumpelt und ungelenk auf den Boden, wo er mit einem verblüfften Gesichtsausdruck so rasch zu Nebel verpuffte, wie es Kreatoks Geist soeben getan hatte. Gouri war geschockt. Konnten die Überreste der Drachen in Krahal immer noch vernichtet werden?

Kreatok und Nehal. Die Erschaffer der vier mächtigen Schilde. Figuren aus Legenden, die Gouri als kleine Agren geliebt hatte. Und nun, so kurz nachdem sie erfahren hatte, dass die beiden noch existierten, hatte Tarok ihnen ein gemeinsames Ende bereitet.

Tarok zuckte erregt und wandte sich Gouri zu, Wut und Rache immer noch sein Denken beherrschend. Gouri packte den Sternenschild fester und erwiderte Taroks bösartigen Blick. Die Sache durfte nicht noch weiter eskalieren. Sie war bereit, für das Ende dieses Kriegs zu kämpfen.\bigskip



Prip schwitzte Blut und Wasser. Eben noch hatte er gesehen, wie Gouri mit überzwerglicher Geschwindigkeit an ihm vorbeigehüpft war, dem Flammeninferno vor ihm entgegen, und sich auf den Rücken des Tarok geschwungen hatte. Was bei allen sieben Feuern der Tiefminen war ihr bloss in den Sinn gekommen? Und woher hatte sie diese Stärke errungen?

Prip hatte sich wieder seinen verwundeten und gefallenen Kameraden zuwenden müssen, doch als er die nächste Schildzwergin aus ihrer Rüstung befreit hatte und in eine möglichst sichere Zone abseits der Festung zerrte, fiel ihm die Begegnung mit Troll und dem Bruderschild wieder ein. Der Troll musste tatsächlich seine riesige Stärke an die Agren übergeben haben. Prip wurde warm ums Herz... war das Stolz?

Gleich danach hüpfte ihm das Herz in die Hose, als ihm einfiel, dass Gouri, kräftig wie ein Troll oder auch nicht, soeben auf einem abhebenden Drachen gelandet war. Verschreckt schaute er sich nach dem Kampf der Giganten um.

Das war ja seltsam! Die Feuer, sowohl das silberne des Dunkelschilds als auch das rote des Feuerdrachen, waren erloschen. Kreatok stand immer noch vor den Wänden der Drei Wasser und schien sich nicht zu bewegen. Tarok hing in der Luft und schlug mit seinen riesigen Flügeln, dass die umliegenden Bäume schwankten, doch abgesehen davon bewegte sich der Drache auch nicht. Und Gouri konnte Prip gar nicht mehr erkennen.

Lieferten sich Kreatok und Tarok ein geistiges Gefecht?

Da! Tarok öffnete seinen Schlund und liess Feuer auf Kreatok herunterregnen! Der Körper des Zwergs wurde innert Sekunden verschlungen und zu Asche. Prip knirschte mit den Zähnen. Klar, Kreatok selbst würde niemand vermissen. Aber ohne ihn und den Dunkelschild wären die Drei Wasser endgültig verloren.

Doch Tarok schlug weiterhin bloss mit den Flügeln und blieb ansonsten regungslos. Lieferte er sich einen weiteren geistigen Kampf? Das konnte doch nur mit Gouri sein? Wo befand sich die Agren? Ging es ihr gut?

Prip trotte hinüber zu Belenor, welcher sich auf einem behelfsmässigen Lager von Kreatoks Giftpfeil erholte. Dort griff Prip nach dem Fernrohr, welches Belenor an seinem Gürtel trug. Auch eine Erfindung, welche Kreatok perfektioniert hatte. Prip versuchte nicht, daran zu denken, als er das Fernrohr auf Tarok richtete.

Da! Auf dem Rücken von Tarok, eingekeilt zwischen zwei spitzen Stacheln, erkannte er etwas Glitzerndes. War das etwa... der Sternenschild? Und da unter dem Schild, lag da nicht... Gouri! Die Agren hatte den hell leuchtenden Schild umklammert und lag so auf Taroks Rücken, sich ebenfalls nicht weiter bewegend. Sie trat tatsächlich im Geist gegen den Feuerdrachen an. Prip wischte sich den Staub aus dem Gesicht und verfluchte sich dafür, dass er nichts ausrichten konnte.

In diesem Moment erlosch das Glitzern des Schildes. Tarok richtete sich in der Luft auf und stiess ein grauenvolles Geheul aus, welches Prip durch Mark und Bein ging. Dann schwang er sich weiter in die Lüfte, immer noch kreischend, und drehte sich einmal um die eigene Achse. Prip saugte tief Luft ein, als er sah, wie Gouris winziger Körper den Rücken des Drachen verliess und dicht gefolgt vom Sternenschild in den Himmel geschleudert wurde.

Tarok, der furchteinflössende riesige Feuerdrache, schrie immer noch und wedelte mit seinen gewaltigen Tatzen, als wolle er einen unsichtbaren Feind abzuwehren. Das konnte nur etwas heissen: Der geistige Kampf war nicht zu seinen Gunsten ausgegangen! Die übrigen Drachen, die immer noch über der Festung kreisten, stiegen mit ein in Taroks Geheul und zogen mit ihm ab.

Jubelschreie von Seiten der Schildzwerge wurden laut und auch in Prips Herz machte sich Erleichterung breit, als er erkannte, dass die Dachenbrigade die Flucht ergriffen hatte – für den Moment jedenfalls. Nur der versteinerte Drache neben den Drei Wassern war noch übrig.

Doch Prip freute sich nicht so sehr wie die anderen. Seine Gedanken schweiften rasch zurück zu Gouri, und so rannte er los, der Stelle entgegen, wo ihr Körper gelandet haben könnte.

Bereits von weitem konnte er den Sternenschild glänzen sehen, trotz der Menge an Schlamm und Dreck, die ihn umgab. Und darunter lag eine Gestalt... Prip versuchte, seinen Schritt zu beschleunigen, doch die Verletzungen, die er sich im Kampf gegen Sagrak zugezogen hatte, zeigten sich umso deutlich. Mehr humpelnd als rennend legte er die letzten Meter zur Einschlagstelle des Sternenschilds zurück. Die ganze Zeit hatte er sich eingeredet, dass niemand einen Sturz von dieser Höhe überlebte, auch nicht jemand mit der Stärke eines Trolls – falls Gouri diese überhaupt noch besessen hatte. Und doch hatte er die Hoffnung noch nicht aufgegeben, als er den Sternenschild erreichte, sich neben ihm zu Boden fallen liess und den leuchtenden Schild ungelenk von Gouris Körper zerrte.

Da lag Gouri nun, in einem Krater vor der Festung, ihre eigentlich grüne Mooskleidung braun vom Dreck und rot vom Blut, das Gesicht nach unten, die verzottelten Haare grösstenteils versengt. Sie hatte Tarok irgendwie verschreckt und damit die Überreste der Drei Wasser gerettet. Das Zwergenvolk würde Zeit haben, sich zu erholen und sich zu wappnen für den nächsten Angriff. Sie besassen noch drei mächtige Schilde, darunter den Sternenschild!

Gouri hatte sich geopfert, um einem ihr fremden Volk Hoffnung zu schenken.

Was für eine heldenhafte Tat. Darüber würden eines Tages Geschichten geschrieben werden.

Prip kniete sich neben Gouri nieder. Er machte sich keine Illusionen. Sie atmete nicht mehr.

Prip schluckte tief, einen dicken Kloss im Hals. Prip bereite sich darauf vor, den Letzten Segen zu sprechen. Er hustete sich etwas Dreck aus der Lunge, und griff nach Gouris Schultern, um den Körper der Agren umzudrehen.

Prip konnte es kaum glauben: Gouris Gesicht war verkratzt und verschmutzt, aber wie durch ein Wunder grösstenteils unversehrt. Dasselbe... ja, dasselbe galt auch für ihren restlichen Körper. Keinerlei Knochenbrüche? Konnte das sein?

Ein metallisches Summen lenkte Prips Blick auf den Sternenschild, welcher immer noch neben ihm lag. Er leuchtete hell und im Schein dieses Lichtes konnte Prip erkennen, wie sich langsam jeder Schnitt, jede Prellung, jeder Schürfung auf Gouris Gesicht auflöste und Farbe darin zurückkehrte. Ein angenehmes Prickeln unter seiner Haut verriet Prip, dass dasselbe mit ihm geschah. Dann plötzlich brach das Summen des Schildes ab und ein stechender Schmerz zeigte Prip, dass er längst nicht vollständig geheilt worden war. Der Sternenschild hatte seine Kraft für heute wohl endgültig aufgebraucht.

Gouri sog tief Luft ein und verschluckte sich ganz grässlich. Eine Hustenattacke überkam sie und Prip vergass alle eigenen Wehwehchen.

Dann, endlich, schlug Gouri die Augen auf und lachte beinahe auf, als sie Prips besorgten Gesichtsausdruck sah.

In dieses glockenhelle Lachen stimmte Prip mit ein.\bigskip



Viel gab es zu tun in den nächsten Tagen. Die Drei Wasser waren beinahe komplett zerstört worden und viele gute Schildzwerge verletzt oder gar gestorben. Die Drachen konnten jeden Augenblick zurückkehren. Und so wurde die gesamte restliche Zwergenarmee unter dem Grauen Gebirge innert Kürze mobilisiert, um einen Rückzug der Besetzung der Drei Wasser unter die Erde, in die riesigen Gänge und Höhlen Caverns, zu vollbringen.

Dort tüftelte man bereits an trickreichen Fallen und Geschossen, welche auch von unter der Erde aus Drachen erlegen könnten. Denn der Unterirdische Krieg zwischen den Zwergen und Drachen war noch lange nicht zu Ende, er hatte vielmehr erst begonnen. Und sowohl die Trolle als auch die Krahder aus dem Süden würden etwas zu sagen haben darin.

Der Sternen- und Dunkelschild wurden zurück nach Cavern gebracht. Beide würden in den dunklen Tagen, die erst noch kommen sollten, wieder an die Drachen fallen. Einzig der Silberschild, in welchem kein lebender Zwerg oder Drache einen Zweck sah, verblieb während den gesamten restlichen Unruhezeiten in seiner Waffenkammer.

Prip und Gouri bekamen davon nur noch wenig mit. Sie halfen bei Bergungen und Aufräumaktionen mit, so gut sie konnten, und kamen sich in dieser Zeit näher. Am ersten Abend nach Kreatoks Tod versammelten sie sich neben den Massengräbern, um dem Meisterschmied und seinem Drachen zu gedenken. Sie waren wahrscheinlich die einzigen, die um die beiden trauerten.

Als der Abzug der Zwerge aus den Drei Wassern glücklich verlaufen war, verliess Prip das Zwergenvolk, um Gouri auf ihrem Heimweg zu begleiten. Der Rückweg verging ebenso wie die Aufräumarbeiten überraschend ereignislos. Keine Trolle warteten bei der Brücke und keine Drachen versuchten, die beiden Wanderer zu rösten.

Gouri war besonders nervös, denn sie hatte in all dem Chaos keinen Falken ausfindig machen können, um ihrer Familie zu berichten, dass bei ihr den Umständen entsprechend alles in Ordnung war. Die übrigen Agren mussten annehmen, dass Gouri längst bei einem Drachenangriff umgekommen war. Doch als sie dann endlich ihre Heimathöhle erreichte... nun, ich will nicht vorweggreifen. Lest einfach selbst:\bigskip



Alle waren versammelt in der Höhle: Gouris Eltern, ihre Geschwister, sogar der Alte Grone, der unlängst in den Seniorenkreis der Agren aufgenommen worden war.

Ihre Blicke landeten zunächst auf Gouri, als diese durch den Höhleneingang trat und sich schüchtern umsah. Ein überraschtes Raunen lief durch die Menge.

Dann wanderten die Blicke zu Prip, der sich in allerbester Heldenmanier (samt vergoldeter Armschiene und elegantem Gehstock) in eine Pose warf. Das in die Agrenhöhle scheinende Sonnenlicht liess seine Rüstung glorreich erglänzen. Das Murmeln der Agren nahm an Lautstärke zu.

Dann kehrten die Blicke der versammelten Agrengesellschaft zurück zu Gouri.

Ihr kleiner Bruder Krul war der erste, der sich rührte. Mit einem glücklichen Glucksen warf er die Moospuppe von sich, die er gehalten hatte, rannte auf Gouri zu und umarmte das Bein seiner grossen Schwester.

Damit war der Bann gebrochen. Ein Lächeln (vielerorts mit Freudentränen garniert) bildete sich auf den Gesichtern ihrer Eltern und Geschwister, und auch sie rannten nach vorne und umarmten ihr zurückgekehrtes Familienmitglied. Die restlichen Agren grinsten ebenfalls und kamen nach vorne. Einige musterten Prip noch argwöhnisch, doch die meisten kümmerten sich primär um Gouri und umkreisten sie, fragten sie, wo sie gewesen war, was sie mit ihrem Besitztum tun sollten, das sie schon auf ihre Geschwister verteilt hatten, und wer denn dieser Begleiter sei.

Gouri schüttelte Hände links und rechts und tauschte, inzwischen selbst in Tränen ausgebrochen, Umarmungen aus.

Dann bat sie um Ruhe, damit der edle ‚Prip, Sohn des Aigar, Bote des Eisernen Stuhls von Cavern und Hüter des Casamatucs der Drei Wasser‘ seinen Bericht der Ereignisse abgeben könne.

Prip, der sich während der ganzen Zärtlichkeiten peinlich berührt im Hintergrund gehalten hatte, trat nach vorne und räusperte sich. Er hatte auf dem Weg bereits daran gefeilt, wie er diese ganze Geschichte am besten vortragen könnte.

Der Alte Grone blieb am Rande der Menge stehen und gluckste fröhlich.

Die Troubadoure suchten bereits ihre Flöten hervor und putzten sie eifrig.

Die verlorene Tochter war zurückgekehrt.

Heute Abend würde es ein Fest geben.







\begin{chapterbox}
    \chapter{Der Giftzwerg und die Sphäre (2020)}
    \label{Der Giftzwerg und die Sphäre (2020)}
    \az{0 bis 340}
    
    Im Anschluss an die Düsteren Zeiten sitzt ein Mechaniker in einem Kerker der Schildzwerge fest. Nun bleibt ihm bloß sein Verstand, um die Geschehnisse in Cavern (und einen neugierigen Fürsten) zu seinen Gunsten zu lenken. Kann Kjall seiner Zelle entkommen, den Lauf der Zeit ins Lot rücken und in seine Heimat zurückkehren? Will er das überhaupt?
\end{chapterbox}


\section{Prolog: Bragor verscherbelt den Bruderschild}

\az{Jahr 62}

Das Rietgras wogte sachte im Wind. Eine massige Gestalt wanderte über das Feld. Nur mit einem Lendenschurz aus Schaffell und großen Stiefeln war sie bekleidet. Auf ihrem Kopf saßen zwei mächtige Hörner. In ihrer einen Hand hielt sie einen Speer. Ihre andere Hand war über kunstvolle Knoten mit einem beschmutzten Schild verbunden, den die Gestalt hinter sich her schleifte. Das Sonnenlicht ließ die prallen Muskeln des Wesens glänzen. Es war ein Tarus.

Eine kleine Gestalt trat aus dem Schatten eines Baumes hervor und näherte sich dem Tarus. Sie trug einen großen Rucksack und benutzte einen schweren Schmiedehammer als Stock. In der Hand hielt sie eine Art silberner Kugel. Ein roter Haarschopf umrahmte ein kantiges Gesicht. Es war ein Zwerg.

„Seid herzlich gegrüßt!“, rief der Tarus mit tiefer Stimme fröhlich und hob dann seine Hand, an welcher der dreckige Schild festgeknotet war, „Ihr hättet nicht zufälligerweise Interesse daran, einen Schild zu kaufen?“

Der Zwerg lächelte, zeigte gelbe Zähne und sprach: „Zufälligerweise hätte ich Interesse daran.“

Der Tarus grinste breit zurück: „Wie viel wäre dieser Schild wert, was meint Ihr? Genug, damit man damit etwas Ausrüstung für eine Reise in ein Gebirge erwerben könnte?“

„Da bin ich mir nicht sicher. Wenn ich mir den so ansehe, würde ich nicht mehr als ein Goldstück dafür springen lassen.“

„Was kann ich denn für ein Goldstück erhalten?“, fragte der Tarus neugierig.

Der Zwerg stutzte kurz und schmunzelte dann: „Viel, äußerst viel. Genug für mehrere Wochen Proviant und die beste Kletterausrüstung östlich des Fahlen Gebirges. Nicht, dass Ihr letzteres nötig hättet, wenn ich Euch so ansehe.“

Der Zwerg holte einen kleinen Beutel hervor, in dem einige Münzen klirrten, und zog eine Goldmünze heraus. Sie war klein und die Prägung darauf saß schief, doch der Tarus betrachtete sie dennoch fasziniert. Dass ein schäbiger Schild gleich viel wert sein sollte wie ganze Säcke voller Apfelnüsse und eine wertvolle Kletterausrüstung, schien ihm nicht seltsam vorzukommen. Und so wurde der Tausch schnell abgehandelt.

Tarus und Zwerg waren kurz davor, ihrer eigenen Wege zu gehen, da hielt der Tarus nochmals inne und fragte: „Meister Zwerg, wisst ihr vielleicht, wo eine Pflanze namens ‚Sternkraut‘ wächst?“

Der Zwerg musterte den Tarus lange und meinte dann: „Ihr erinnert mich an jemanden, den ich einst kannte... sagt, wie lautet Euer Name?“

„Bragor, Meister Zwerg. Und eurer?“

„Ich sehe schon überall Geister...“, murmelte der Zwerg mehr zu sich selbst als zum Tarus, „Mein Name ist für Euch nicht von Belang. Und nein, werter Bragor, ich kenne mich leider nicht in der Kunde der Pflanzen aus. Ich würde Euch raten, zu einer Kräuterhexe zu gehen, oder vielleicht zum Baum der Lieder, da wird man Euch eher aushelfen können.“

Der Tarus ließ seine Schultern hängen: „Beim Baum der Lieder war ich schon“, murmelte er.

Der Zwerg kümmerte sich nicht weiter darum, sondern betrachtete den dreckigen Schild in seinen Händen ehrfürchtig. Dann drehte er sich um, winkte dem Tarus zum Abschied zu und lief dem Sonnenuntergang dagegen. Er war in Hochstimmung. Er hatte soeben den zweiten der vier mächtigen Schilde aus uralter Zeit errungen.








\newpage
\section{Gefangen}

\az{Jahr 0}



Kjall saß in seiner Kerkerzelle und schrieb. Die Schildzwerge waren so zuvorkommend gewesen, ihm ein wackeliges Pult, durchnässtes Pergament und vertrocknete Tinte zur Verfügung zu stellen, damit er sich den lieben langen Tag dort verwirklichen könnte. Natürlich war sich Kjall bewusst, dass die Wachen alle paar Tage seine beschriebenen Pergamente durchstöberten, um zu schauen, ob sie darin irgendwelche Hinweise auf seine Herkunft oder seine Tiefminen-Golems finden könnten. Dachten sie tatsächlich, dass er es nicht mitbekommen würde, wenn sie mitten in der Nacht laut polternd an sein Pult traten und in seinen Schriften rumwühlten?

Kjall hörte dennoch nicht mit Schreiben und Skizzieren auf. Außer dem Pult und seiner Pritsche befand sich in dieser Zelle kaum etwas, auch keine weiteren Vergnügungsmöglichkeiten. Ausgang hatte er nebst seiner halbtäglichen Toilettengänge auch kaum. Er musste nur darauf achten, keine wichtigen Informationen niederzuschreiben. Und so verbrachte Kjall die meiste Zeit damit, Schmiergedichte auf seine Wachen zu verfassen sowie Skizzen kreativer Konstruktionen, die abbildeten, was Kjall mit den Wachen gerne anstellen würde – natürlich gespickt mit furiosen Flüchen verschiedenster Art. Es war niederer Humor, aber dadurch nicht minder amüsant.\bigskip



Schwere Schritte hallten durch den dunklen Gang, an dessen Ende Kjalls Kerker lag. Es war unüblich für die Wachen, bereits zu dieser Zeit zurückzukehren. Kjalls Magen hatte noch nicht einmal zu knurren begonnen, was heißen musste, dass die nächste Mahlzeit noch lange nicht an der Reihe war. Wollten sie ihn etwa wieder befragen? Inzwischen musste doch selbst der einfältigste dieser Narren begriffen haben, dass er ihnen nicht verraten würde, von wo er kam. Erst recht nicht, nachdem sie die Überreste seiner Tiefminen-Golems eingeschmolzen hatten. Die Golems aus wertvollem Roteisen waren vor der Sphäre sein größter Schatz gewesen, und wenn er hier erst einmal raus war, würden sie es noch bereuen, diese Meisterwerke nicht als solche geachtet zu haben.

Die Schritte im Gang verstummten. Kjall setzte seine Feder ab und drehte sich zu den Besuchern um. Er erwartete, den üblichen missmutig gelaunten Weißbart zu erblicken. Stattdessen fiel sein Blick auf goldene Stiefel und einen verfilzten grauen Bart. Buschige Augenbrauen, die wirre, eisblaue Augen verdeckten. Gesichtszüge, die ihm nur allzu bekannt erschienen.

„Fürst Hallwort!“, hauchte Kjall. Hatte dieser neugierige Trottel sich tatsächlich dazu herabgelassen, einen einfachen Gefangenen wie Kjall zu besuchen, in der Hoffnung, mehr über seine Herkunft und sein Können zu erfahren? Kjall war natürlich nicht bereit, dem Fürsten das Geheimnis der Golemkerne zu verraten. Aber das musste er Hallwort nicht direkt auf die Nase binden. Schließlich besaß der Fürst nicht nur die Schlüssel zu seiner Zelle, sondern auch die Autorität, Kjall laufen zu lassen. Moment mal... Schlüssel! Kjalls Gedanken rasten. Fürst Hallwort besaß nicht nur die Schlüssel zu Kjalls Zelle, sondern auch den zur geheimen Fürstenkammer! Der geheimen Fürstenkammer, in welcher Kjall und seine Kumpanen in einigen Jahrzehnten die Sphäre finden würden, jenes Artefakt, welches Kjall überhaupt erst in die Vergangenheit versetzt hatte. Und das bedeutete, dass die jüngere Version der Sphäre vielleicht in diesem Augenblick in der geheimen Fürstenkammer lag. So nah! Wenn Kjall an diese Sphäre kommen könnte, wäre er nicht nur frei, sondern auch mächtiger als je zuvor. Das würde es ihm erlauben, noch weiter zurück in die Vergangenheit zu reisen, um die Gründung Andors zu verhindern und die Werte der Schildzwerge zu bewahren! Und dafür musste er nur eines hinkriegen: Den neugierigen Idiot vor sich geschickt zu manipulieren.

„Ich sehe mit Freude, dass du meinen Namen kennst“, plapperte Fürst Hallwort los, „Deine Notizen sind in modernen Zwergenrunen geschrieben. Deine Hände sind die vernarbten Hände eines Zwergs, der sein Leben in den Tiefminen verbracht hat. Und doch scheint dich keiner dort zu vermissen. Du bist ein Mysterium, o Zwerg, und du wolltest den Wächtern nicht einmal deinen Namen nennen. Nun frage ich dich, wie ein gewisser Zwergenfürst dich umzustimmen vermögen könnte. Ich spüre, dass an dir mehr steckt, als man auf den ersten Blick meinen mag.“

Kjall traute seinen Ohren kaum. Hatte Hallwort tatsächlich in den Tiefminen umherfragen lassen, ob ihn jemand erkannte? Hatten er und seine Tiefminen-Golems einen derart starken Eindruck hinterlassen? Ein Glück, dass sie nicht sein jüngeres Ich aufgespürt hatten, sonst hätte er einiges zu erklären gehabt.

„Mein Name ist... Jork, mein Fürst“, log Kjall, „Und wenn Ihr mir schon die Ehre erweist, mich hier zu besuchen, so will ich Euch ein Geheimnis verraten...“

Kjall senkte seine Stimme und Hallwort trat näher an sein Gitter. Kjall betrachtete die beiden Zwergenwachen, die auf weiter hinten im Schatten des Gangs warteten und die Interaktion beobachteten. Sie mochten nicht verstehen, was Kjall flüsterte, aber sie würden definitiv mitkriegen, wenn Kjall eine unüberlegte Bewegung tätigte. So blieb Kjall ruhig und wisperte weiter: „Mein Fürst, ich komme aus der Zukunft! Ich bin hier auf der Mission, Cavern vor dem Untergang zu bewahren!“

Gespannt wartete Kjall auf Hallworts Reaktion. Der Fürst ließ sich nichts anmerken, sondern starrte Kjall mit seinem eisblauen Blick geradewegs in die Augen, still darauf wartend, dass Kjall weitersprach. Er war nicht überzeugt, natürlich nicht, aber er hatte ihn auch noch nicht ausgelacht. Hallworts hatte bekanntlich ein offenes Ohr für jede Theorie, die man ihm auf die Nase binden wollte. Und das wusste Kjall zu seinem Vorteil zu nutzen. Jetzt musste er nur noch Argumente finden, die seine Behauptung untermauern würden.

„Mein Fürst, ich bin hier in Eurem Auftrag. Ich hatte das Ziel, mit den mechanischen Golems ins Rietland zu ziehen und einen gefährlichen Andori von einer finsteren Tat abzuhalten.“

„Ach“, sagte Hallwort. Er schwieg eine Zeit lang und streichelte seinen grauen Rauschebart.

„Hmm“, murmelte Hallwort dann, „Äußerst interessant. Aus welchem Jahr meint Ihr zu kommen, Jork? Was ist das für ein gefährlicher Andori und inwiefern würde er Caverns Untergang herbeiführen?“

Kjall sprach langsam und bedacht weiter, während sein Kopf ratterte: „Vielleicht ist es besser, wenn ich Euch so wenig wie möglich üblich die Zukunft verrate.“

Hallwort lachte auf: „Sehr geschickt! Das macht es äußerst schwer, zu überprüfen, ob Ihr die Wahrheit sprecht. Sagt, werter Jork, warum habt Ihr nie nach mir gerufen? Ihr schienet bereit, bis ans Ende Eurer Tage in dieser Zelle zu verstauben. Wenn Ihr tatsächlich in meinem Auftrag hierher geschickt worden wärt, so hättet Ihr doch sicherlich nach einer Audienz bei mir verlangt.“

Kjall biss sich auf die Unterlippe. Er war noch nie sonderlich gut im Improvisieren gewesen, erst recht nicht, wenn so viel auf dem Spiel stand. Wie lautete seine Geschichte? Er war von Fürst Hallwort aus einer Zukunft geschickt worden, in welcher Cavern dem Untergang geweiht war. Hallwort hatte ihn hierher geschickt. Warum hätte er sich still verhalten? Weil, weil...

„Ich kenne Euch doch, mein Fürst! Nichts weckt Euer Interesse so schnell wie ein ungelöstes Rätsel. Ihr seid schneller zu mir gekommen, als wenn ich, ein einfacher Gefangener, um eine Audienz gebeten hätte. Feuer und Spucke, dass Ihr hier vor mir steht, zeigt doch gerade, dass es funktioniert hat! Ich... ich weiß Dinge, die ich nicht wissen könnte, wenn ich löge! Ihr trinkt Euren Drachentee mit immer mit zwei Löffeln Honig drin, auch wenn Ihr das bei jeder passenden Gelegenheit abstreitet. Ihr führt eine geheime Tölplingzucht im Brauneisenstein. Als Kind seid Ihr einmal in den Geheimen See gefallen, als ihr heimlich und unerlaubt Cavern zu erkunden versuchtet, und eine von Wasser triefende, grünliche Gestalt rettete euch.“

Kjall dankte dem feurigen Gott für die Klatschtanten, die sich nach Hallworts Tod über ihn das Maul zerrissen hatten. So vieles war damals aufgedeckt worden, das die breite Bevölkerung Caverns zum jetzigen Zeitpunkt noch gar nicht wissen konnte. Hallwort legte seinen Kopf schief und schien zum ersten Mal ernsthaft zu bedenken, ob Kjall vielleicht die Wahrheit sagte. Nach einer weiteren Überlegungspause fragte er kleinlaut: „Werden meine Tölplinge entdeckt werden?“

Jetzt hab‘ ich dich, dachte Kjall, und sprach: „Arpachen-Überfall. Fragt nicht weiter nach. Niemand wird ernsthaft zu Schaden kommen.“

Stille.

„Ist es Euch denn gelungen, diesen gefährlichen Andori zu überwinden, ehe Ihr gefangen genommen wurdet?“

„Ich befürchte nicht, mein Fürst. Er hat weitere Andori auf mich gehetzt. Schade, dass Ihr meine Golems eingeschmolzen habt. Aber wir werden auch so eine Möglichkeit finden, an ihn heranzukommen.“

Hallwort tippte sich nachdenklich auf die Nase: „Selbst wenn Ihr mich überzeugen könnt, dass Ihr aus der Zukunft und in meinem eigenen Auftrag hier seid, so dürfte es noch einiges schwerer werden, den guten König Brandur davon zu überzeugen. Zu verlangen, dass er einen seiner Landleute herausrücke für etwaige Verbrechen, die noch gar nicht begangen wurden, ist kein geschicktes Vorgehen, um den Frieden zwischen unseren Völkern zu wahren. Und Frieden ist in diesen kriegerischen Zeiten das höchste Gut. Die Trolle...“

„Mein Fürst, deswegen versuchen wir ja auch gar nicht, Brandur davon zu überzeugen. Der fragliche Andori hat keine starke Beziehung zum Königshaus und ich habe in der Vergangenheit keinerlei Beziehung zu euch. Ich muss nur hier raus und kann dann ganz im Geheimen...“

Kjall riss seine Augen auf, als wäre ihm soeben etwas eingefallen: „Mein Fürst, die Apparatur, mit welcher ich in die Vergangenheit reisen konnte, stammt aus Cavern, aus der geheimen Fürstenkammer! Mit etwas Glück befindet sich das Artefakt in diesem Augenblick bereits dort! Wenn ich dessen Funktionsweise demonstrieren könnte, so würde Euch das den letzten Zweifel nehmen, dass ich tatsächlich aus der Zukunft komme.“

„...und da nur ich Zugang zur Kammer habe, wäre damit geklärt, dass ich Euch hierher gesandt habe. Genial!“, rief Hallwort aus, „Schnell, werter Jork, verratet mir, wonach ich Ausschau halten soll!“

Hatte er die Lüge tatsächlich abgekauft? Spielte der Fürst seine Begeisterung nur? Wollte er Kjall auf die Probe stellen? Kjall war sich nicht sicher, aber es konnte sicherlich nicht schaden, mitzuspielen.

„Das Artefakt ist eine metallene Sphäre aus einer Vielzahl ineinander verschlungener Ranken und Verstrebungen. Ihr werdet es wissen, wenn Ihr es seht. Findet diese Sphäre und bringt sie hierher, dann kann ich Euch ihre Funktionsweise demonstrieren.“

Hallwort sah Kjall prüfend in die Augen. Dann nickte er, drehte sich energetisch um und stapfte von Kjalls Zelle weg. Die beiden Zwergenwachen musterten Kjall finster und schritten dann im Einklang ihrem Fürsten hinterher. Kjall blieb alleine zurück und überlegte.\bigskip



Überraschenderweise wurde Kjalls Mediation nicht durch zurückkehrende Wachen oder gar einen neugierigen Fürsten gestört. Stattdessen glomm plötzlich ein blauer Funke in der Mitte der Zelle auf. Noch während Kjall sich aufrichtete, um den Funken genauer zu betrachten, erstrahlte jener und wurde zu einer unförmigen, wabernden blauen Masse. Die Masse wuchs langsam, bis sie beinahe die Größe eines Skrals hatte. Kjall hatte ein solches Phänomen bereits einmal gesehen: Als er und seine Komplizen das Portal in die Vergangenheit geöffnet hatten, das sie hierhergebracht hatte.

So war Kjall eher fasziniert als überrascht, als die blaue Masse zu rotieren begann, immer breiter und dünner wurde, bis sie als irrsinnig rasch entlang ihrer Symmetrieachse rotierende Scheibe in der Mitte seiner Zelle schwebte. Durch diese hellblaue Hülle konnte Kjall verschwommen erkennen, dass auf der anderen Seite des Portals drei Gestalten zu sehen waren, drei Zwerge. Ihre Proportionen waren verzerrt, als würde Kjall sie durch eine dicke Wasserschicht betrachten. Sie schienen sich zu unterhalten.

Kjall streckte seine Hand nach dem blauen Portal aus und spürte, wie dessen Hülle unter seinen Fingerspitzen nachgab. Auf der anderen Seite des Portals schien einer der anderen Zwerge es ihm gleichzutun. Schnell trat Kjall zurück, um Platz zu lassen, denn er antizipierte, was als nächsten geschehen könnte.

Tatsächlich traten die drei Zwerge kurz darauf synchron nach vorne. Blaue Lichtstrahlen brachen aus der Portalhülle hervor und ließen Schatten über die karge Zellenwand tanzen. Mit zugekniffenen Augen versuchte Kjall, die Portaloberfläche im Blick zu halten. Die Schemen der drei Zwerge wogen und waberten, zerfaserten und setzten sich wieder zusammen. Dann brachen sie aus der Oberfläche des Portals hervor und plumpsten mit einem lauten Knall auf den Zellenboden. Das blaue Leuchten ebbte ab und das Portal wurde wieder zur einer blassen hellblauen Scheibe, die in der Mitte des Raums schwebte. Doch Kjall achtete nicht auf sie. Was da vor ihm lag, war um einiges spannender. Der erste der drei Zwerge, die soeben durch das Portal getreten – oder besser gefallen – waren, war eine von Kjalls Wachen. Wie lautete sein Name schon wieder? Bort? Der zweite Zwerg war schon um einiges spannender, denn es handelte sich dabei um Fürst Hallwort. Und der dritte Zwerg... war Kjall selbst!\bigskip



Eine Zeitreise durch ein solches Portal war ein ganz und gar unangenehmes Geschäft, daran mochte sich Kjall noch erinnern. Man schwebte eine Zeit lang durch eine Art Limbo, einem endlosen Raum aus gleißendem blauen Licht, in dem es keine Schwerkraft zum Orientieren und keine Luft zum Atmen gab. Kjall hatte bei seiner ersten Reise gedacht, dass sein letztes Stündlein geschlagen hätte, als er nach rund einer Minute sein Bewusstsein verloren hatte. Stattdessen war er in der Vergangenheit aufgewacht, in den Armen eines finsteren Schattenwesens, welches ihn durchs Rietland trug. Den beiden verfluchten Schatten, dieser Shan und diesem Bleichen, hatte die luftlose Reise durch den Limbo offenbar nichts ausgemacht. Immerhin war Kjall damals nicht der einzige gewesen, der ohnmächtig geworden war, und der mächtige Tarus Thogger war noch um einiges schwerer zu tragen gewesen als er selbst. Dennoch hatte der Vorfall etwas an seinem Stolz gekratzt.

Genauso wie Kjall damals selbst ohnmächtig geworden war, lagen die drei zeitreisenden Zwerge nun allesamt ohnmächtig in einer Reihe vor ihm: Bort, Hallwort und ein zweiter Kjall. Es war ein wahrlich unwirkliches Gefühl, eine Kopie seiner selbst zu betrachten. Sah Kjall wirklich so gedrungen aus? Und was war mit diesen abstehenden Ohren los? Die Hände und Füße des zweiten Kjalls waren mit schweren Eisenketten verbunden, welche wiederum an Borts Gürtel befestigt waren. Hallwort hingegen war genauso ketten- wie bewusstlos. An seinem Gürtel hängte das übliche Sammelsurium an Täschchen und Ketten voller wertlosem Allerlei, wie man es von Fürst Hallwort erwarten würde – aber auch ein kleines, unscheinbares Stück Metall mit einigen Scharnieren und Drähten dran. Ein Casamatuc, ein Zwergenwerkzeug, mit dem man fast jedes Schloss aufkriegen konnte! Was für ein Geschenk des Schicksals!

Kjall fiel kaum auf, dass das durchscheinende Zeitportal seine Form verlor, wieder zu einem kleinen Funken wurde und dann vollends verpuffte. Stattdessen stürzte er sich auf den Gürtel des ohnmächtigen Hallwort, griff nach dem Casamatuc, rannte zur Zellentür und setzte das Werkzeug an. Schnell, schnell, schnell! Wie lange blieb ein kräftiger Zwerg üblicherweise bewusstlos? Kjalls geschickte Finger waren wie geschaffen für solche Tätigkeiten und der Casamatuc des Fürsten des Schildzwerge war wie erwartet von allerbester Qualität. Ein Klick, zwei Klicks, beim dritten klemmte es, etwas Rütteln, ein Klack, eine Drehung, und die Zellentür war offen. Kjall war frei!

Ein kurzer Kontrollblick zurück verriet Kjall, dass Hallwort die Fürstenkrone mit den vier eingemeißelten Schilden aus uralter Zeit zum Treffen mit seinem mysteriösen Gefangenen nicht aufgesetzt hatte. Zu schade, die hätte sich in Kjalls Museum wirklich prächtig gemacht. Aber das war nun wirklich nicht die Hauptsache. Kjall wollte sich bereits wieder abwenden und auf die Socken machen, da besann er sich eines Besseren. Er wusste zwar nicht, aus welcher Zeitlinie dieser zweite Kjall stammte, aber dessen Hallwort war gleich gekleidet wie derjenige Hallwort, welcher sich erst vor Kurzem noch mit Kjall unterhalten hatte. Es bestand bestimmt eine gewisse Chance, dass Kjall selbst bald in eine ähnliche Lage wie dieser zweite Kjall kommen würde, und dann würde er es bestimmt sehr schätzen, nicht angekettet an eine andere Wache zu Bewusstsein zu kommen. Und ohnehin sollte man doch immer Solidarität mit sich selbst zeigen.

Rasch huschte Kjall zurück zum zweiten Kjall, sorgsam darauf achtend, ob Hallwort oder Bort Zeichen des Erwachsens zeigten. Sie taten es nicht, und so kniete sich Kjall neben dem zweiten Kjall nieder und setzte den Casamatuc an dessen Fesseln. Er drehte und stieß, es klickte und knirschte, aber die Ketten blieben zu.

„Bei Bailas angekokelter Eiche!“, knurrte Kjall. Er hätte nicht gedacht, dass es solche Casamatuc-resistenten Schlösser bereits für einfache Eisenketten gab. Was nun? Sollte er den zweiten Kjall doch einfach zurücklassen? Was würde geschehen, wenn man die beiden anderen Zwerge hier finden würde? Was wäre das für eine Zeitlinie, in welcher fortan zwei Borts, und noch schlimmer, zwei Hallworts existierten?

In diesem Moment riss der zweite Kjall seine Augen auf und packte Kjall am Handgelenk. Er richtete sich geschmeidig auf – seine Ketten klirrten laut – und sah sich in der Zelle um.

„Schnell“, richte er dann das Wort an Kjall, „Hilf mir, die beiden anderen aus dieser Zelle zu schaffen. Keine Sorge, das war ihre erste Zeitreise. So schnell werden sie nicht wieder erwachen.“ Bei den Flammen des Zor, klang Kjalls Stimme immer so krächzend?

Ohne auf eine Antwort zu warten, packte der zweite Kjall Bort an den Schultern und begann, die Wache in Richtung der immer noch offen stehenden Zellentür zu bugsieren. Kjall starrte seinem Doppelgänger mit einem flauen Gefühl im Magen nach, fasste sich, beugte sich dann zum ohnmächtigen Fürst Hallwort hinunter und zerrte diesen aus der Zelle heraus. Was hatte dieser Zwerg nur gegessen, schweres Roteisen?! Es war absurd. Der Fürst der Schildzwerge in seinen Händen, ihm komplett ausgeliefert. Und doch konnte er nichts damit anfangen, da es sich um den Hallwort aus einer anderen Zeitlinie handelte!

Kjall folgte dem zweiten Kjall ächzend aus der Zelle hinaus in einen kleinen Seitengang, wo er Fürst Hallwort erleichtert zu Boden plumpsen ließ. Der zweite Kjall hatte Bort dort bereits auf eine leichte Decke gebettet und meinte: „Wir müssen noch mal schnell in die Zelle zurück.“

Die Kette, die die Hände und Füße des zweiten Kjall an Bort knüpfte, war so lang, dass der zweite Kjall bis fast in die Zelle zurücklaufen konnte, ehe seine Kette anschlug.

„Der Casamatuc!“, zischte der zweite Kjall, „Rasch!“

Kjall huschte an ihm vorbei in die Zelle zurück, griff nach dem dort liegenden Casamatuc und übergab das Werkzeug seinem Doppelgänger: „Er wird die Ketten leider nicht lösen können, das habe ich bereits versucht.“

Der zweite Kjall nickte: „Ich ebenfalls.“

Ehe Kjall über diese Worte nachsinnieren konnte, setzte sein Doppelgänger nach: „Aber wir wollen doch keine Spuren zurücklassen. Deswegen musst du auch in dieser Zelle bleiben.“

Da war Kjall aber nicht mehr einverstanden: „Bei Nehals verfaulten Zähnen, niemals bleibe ich hier zurück! Ich fliehe mit dir!“

„Das glaube ich nicht“, sprach der zweite Kjall und zog die schwere Zellentür zu, sodass Kjall in der Zelle feststeckte, während sein Doppelgänger außerhalb der Zelle stand. Ein leises Klonk verriet Kjall, dass sein Doppelgänger die Zellentür mit dem Casamatuc wieder verschlossen hatte. Kjall fluchte.

„Kjall! Lass mich hier nicht im Stich!“

Sein Doppelgänger entfernte sich ungerührt von der Zellentür.

„Wenn du mich nicht sofort hier herauslässt, schreie ich und alarmiere die Wachen!“

Der zweite Kjall drehte sich zurück: „Jetzt sei schon still. Wenn du dich geschickt anstellst, bist du schon bald in derselben Lage wie ich.“

Mit diesen Worten huschte der zweite Kjall in den dunklen, modrigen Seitengang zurück, in welchem die beiden Kjalls Bort und Hallwort zurückgelassen hatten. Kjall hörte seine Ketten ein letztes Mal klirren, dann war alles still.

Kjall trat lautstark gegen die Zellentür, einmal, zweimal, dreimal. Sie rührte sich keinen Millimeter. Zwergische Facharbeit. Es war zum Heulen. So nah an seiner Freiheit war er gewesen, und er hatte sich von einem Doppelgänger seiner selbst austricksen lassen! Nicht einmal sich selbst konnte man noch vertrauen. Konnte er sich vielleicht irgendwelche Vorteile davon verschaffen, das Versteck seines Doppelgängers zu verraten, sobald der Hallwort aus dieser Zeitlinie zurückkehrte? Und was hatte der zweite Kjall nur damit gemeint, dass Kjall selbst bald in seiner Situation sein könnte?


\newpage
\section{Geflohen}



Hektische Schritte kündigten die Rückkehr des Zwergenfürsten aus Kjalls Zeitlinie an. Kjall richtete sich erwartungsvoll auf. Hallworts breites Grinsen war durch das Gitterfenster in der Zellentür schon von weitem zu erkennen. Es verriet Kjall, dass der Fürst erfolgreich gewesen war. Triumphierend rannte dieser auf Kjalls Zelle zu und gab dann preis, was er in seinen Händen hielt: Eine etwa faustgroße Kugel aus umschlungenen Streben und kleinen Schaltern. Die Sphäre!

„Ich nehme an, das zeigt, dass meine Geschichte der Wahrheit entsprach“, setzte Kjall an, doch Hallwort unterbrach ihn:

„Mein lieber Jork, so sehr ich Eurem Wort auch Glauben schenken will, so fällt es mir nur schwer zu glauben, dass dieses abstruse Artefakt – so hübsch es auch aussieht – einen in eine andere Zeit bugsieren kann. Eine kleine Demonstration wäre bestimmt nicht zu viel verlangt. Welche Schalter muss ich betätigen?“

Kjall horchte auf. ‚Wenn du dich geschickt anstellst, bist du schon bald in derselben Lage wie ich...‘ Das hier könnte es gewesen sein, was der zweite Kjall damit gemeint hatte! Jetzt war Fingerspitzengefühl gefragt.

Kjall beugte sich eifrig vor und zeigte auf einige Querverstrebungen: „Wisst Ihr, dieses Gerät vermag es, ein Portal in Vergangenes zu öffnen. Dafür müsst ihr bloß die entsprechenden Querverstrebungen hier runterschieben, damit die Kontrollzelle in der Mitte freigegeben wird. Dort könnt ihr einstellen, wie weit in Vergangenes ihr reisen wollt... aber... aber nein, das ist eine Kunst, die sich nicht so mir nichts, dir nichts erlernen lässt. Nein, mein Fürst, haltet ein!“

Doch es war schon zu spät. Hallwort hatte in seiner Begeisterung den Haupthebel betätigt und die Maschinerie in Gang gesetzt. Die Sphäre begann, leise zu sirren. Blauer Dampf stieß aus einer Öffnung in ihrer Seite, waberte in die Luft und verformte sich zu einem immer schneller drehenden Portal nach Wer-weiß-wann. Hallwort ließ die Kugel erschrocken fallen und seine beiden Wachen stürzten nach vorne, um sich zwischen das Zeitportal und ihren Fürsten zu stellen. Sie sorgten sich umsonst, denn das Portal blieb nur wenige Augenblicke lang stabil, ehe es sich wieder zu einem kleinen, blau leuchtenden Funken zusammenzog und mit einem leisen „Puff“ verschwand.

„Mit Verlauf, mein Fürst, das war nicht geschickt!“, knurrte Kjall zwischen zusammengekniffenen Zähnen hervor. Wie viel Treibstoff befand sich wohl noch in der Sphäre? Falls die seltene Mischung aus flüssigem Feuer und Tiefsand vom Grund des Geheimen Sees, die der Sphäre ihre Kraft verlieh, wegen dieses neugierigen Trottels vollends verbraucht würde, würde Kjall ein für alle Mal hier feststecken. Und dabei war er so nah an seiner Flucht!

„Mein Fürst, ich würde vorschlagen, dass Ihr mich die Sphäre bedienen lässt“, schlug Kjall vor, „Ich könnte ein Portal wenige Minuten in die Vergangenheit öffnen, da kann bestimmt nichts schiefgehen.“

Hallwort blickte immer noch mit weit aufgerissenen Augen auf die Stelle, wo das Zeitportal verschwunden war. Dann fasste er sich und hob die Sphäre hoch, um sie an Kjall zu überreichen, als der Wächter Bort vortrat: „Nicht so schnell! Ihr werdet bestimmt nicht einfach so durch ein magisches Portal treten, welches von einem vorlauten Gefangenen geöffnet wurde.“

„Mein Fürst!“, flehte Kjall Hallwort an, „Ihr werdet durch das Portal hindurchsehen können und überprüfen, dass sich auf der anderen Seite nichts Gefährliches befindet. Spiegel in Vergangenes gibt es genug – habt Ihr nicht selbst einen solchen Hadrischen Spiegel in Euren Gemächern? Nur das Reisen durch das Portal wird beweisen, dass diese Kugel das ist, was ich behaupte. Und wolltet Ihr nicht schon immer die grundlegenden Gefüge der Welt erleben? Dies ist Eure Gelegenheit!“

Jetzt hab‘ ich dich, dachte Kjall, als Hallworts Augen zu glänzen begannen. Der alte Kindskopf. Stets auf der Suche nach einem Abenteuer.

Die zweite Zwergenwache trat nun ebenfalls hervor: „Ich muss doch tunlichst davon abraten. Und egal, was geschieht, ich verlange, dass der Gefangene während der Prozedur angekettet ist. Bort ist doch immer so stolz auf diese speziellen Eisenketten, die sein Vetter ihm geschenkt hatte. Das wäre endlich eine Gelegenheit, sie einzusetzen!“

„Genauso ist es“, setzte Bort gegenüber Kjall nach, „Ich halte mich nahe bei dir und lasse dich nicht aus den Augen, Freundchen. Und egal, was geschieht, diese Sphäre bleibt hier in diesem Raum in dieser Zeit! Nicht, dass du noch versuchst, mit ihr zu entkommen. “

Bort war nicht auf den Kopf gefallen, aber Hallworts Neugierde und Borts Respekt vor den Befehlen seines Fürsten würden dem entgegenwirken können. Dann blieb die Sphäre halt hier.

„Perfekt“, rief Kjall aus, „Dann kettet Bort mich an sich, ich öffne das Portal einige Minuten in die Vergangenheit und wir treten hindurch. Die Sphäre bleibt hier und ist somit außer Reichweite, falls ich sie wirklich stehlen wollte. Worauf warten wir noch?“

Jetzt musste es einfach schnell gehen, damit keiner sich Gedanken darüber machte, wie sich diese Zeitlinie hier weiterentwickeln würde, sobald das Trio durch das Portal getreten war. Hallwort war sicher nicht bereit, sein Reich spurlos zu verlassen. Ein Glück, dass zumindest die zweite Zwergenwache zu verwirrt schien, um sich einen Reim auf Kjalls Plan zu machen.\bigskip



Das Ganze war schnell abgewickelt. Kjall stand umgedreht an der Zellenwand, während Bort seine Hände und Füße mit Ketten verband und schließlich über eine längere Kette an seinem Gürtel befestigte. Der Schlüssel zu diesen Ketten legte er auf dem Tisch, weit von Kjall entfernt. Hallwort platzierte indes die Sphäre auf einem kleinen Strohhaufen auf dem Steinboden und trat dann zurück. Die zweite Schildwache musterte das ganze Geschehen verwirrt, während sie sich am Kopf kratzte.

Dann war es soweit. Kjall hantierte am Kontrollzentrum der Sphäre herum und stellte sicher, dass das Trio bloß eine Viertelstunde in die Vergangenheit reisen würden. Hallwort wollte sicherlich Kjalls Gesicht am anderen Ende des Portals erkennen, um sicherzugehen, dass es sie nicht in die tiefste Urzeit sandte, so aufregend ein Aufenthalt in der guten alten Urzeit Caverns auch wäre. Während bei Zeitreisen über Jahrzehnte oder gar Jahrhunderte hinweg zweifelsohne große Unsicherheiten ins Spiel kamen, konnte der Ankunftszeitpunkt für solche kurzen Zeitreisen glücklicherweise relativ klar bestimmt werden. Kjall überprüfte seine Einstellungen ein letztes Mal, seine Hände zitternd vor Vorfreude, und betätigte dann den Haupthebel.

Ein lautes Zischen ertönte, als flüssiges Feuer und Tiefsand sich mischten und blauer Dampf aus der Sphäre ausgestoßen wurde, welcher sich rasch zu einem Zeitportal formte. Diese Mal blieb das Zeitportal stabil. Auf der anderen Seite konnte man verschwommen erkennen, wie ein verschwommener früherer Kjall aufstand, sich dem Portal näherte und interessiert hineinblickte. Nun ja, genau genommen nicht der frühere Kjall, sondern ein Kjall aus einer anderen Zeitlinie. Oder?

Kjall blickte Bort und Hallwort an. Bort schaute finster zurück und brummelte etwas von wegen, dass Kjall es bereuen würde, wenn dem Fürsten auch nur ein Haar gekrümmt würde, während Hallwort nur noch Augen für das Portal übrig hatte.

„Auf drei“, flüsterte der Fürst.

„Bereit“, antwortete Bort. Kjall tat es ihm nach und begann dann, so unauffällig wie möglich seinen Atem zu verlangsamen. Er musste unbedingt als erster auf der anderen Seite des Portals erwachen.

„Eins!“, rief Hallwort.

Bort tatschte mit seinem Eisernen Handschuh nach dem Portal, welches unter seiner Berührung strahlend blau aufleuchtete.

„Zwei!“, rief Hallwort.

Kjall holte tief Luft.

„Drei!“

Die drei Zwerge traten synchron nach vorne und verschwanden ins Portal hinein.\bigskip



Kjall, Bort und Hallwort schwebten durch den endlosen Limbo, der von blauem Leuchten erfüllt war. Schwerelos hingen sie im leeren Raum, und doch bewegte sich die Gruppe langsam, unendlich langsam auf die helle Öffnung zu, die das andere Ende des Portals darstellte. Hallwort hatte den Fehler gemacht, zu versuchen einzuatmen, und keuchte nun verzweifelt, während er sich auf die Brust schlug. Bort hatte eine bessere Kondition und war ganz und gar nicht glücklich damit, wie die Situation sich entwickelte. Seine Handschuhe griffen nach Kjall, während er mit hochrotem Kopf versuchte, Kjall bei der Gurgel zu packen. Kjall fühlte sich, als würde sein Kopf explodieren, und kniff sich weiterhin stur die Nase zu. Hatten sie wirklich erst einen Drittel des Weges zurückgelegt? Hatte seine letzte Zeitreise durch den Limbo auch so lange gedauert?

Hallwort zuckte ein letztes Mal und erschlaffte. Bort ließ von Kjall ab und versuchte, sich durch den schwerelosen Raum auf seinen Fürsten zuzubewegen. Er schrie verzweifelt auf, ein Geräusch, welches stark verzerrt an Kjalls Ohren drang. Ohren, die mit schwerem Druck belegt waren. Nun zuckte auch Bort auf und erschlaffte. Kjalls Blickfeld wurde an den Rändern immer dunkler. Dann hielt er es nicht mehr aus, ließ seine Nase los und folgte seinen Instinkt, welcher ihn anschrie, einzuatmen. Glühender Schmerz stach in seine Lunge. Das gleißend blaue Licht wurde dunkler und verschwand dann endgültig. Das letzte, was Kjall wahrnahm, war ein metallisches Klonk, als die Kette, die ihn an Bort fesselte, auf Stein knallte.\bigskip



Ein metallischer Klirren war auch das erste Geräusch, das an Kjalls Bewusstsein drang, als er wieder aufwachte. Jemand hantierte hektisch an den Ketten zwischen seinen Füßen herum! Kjall riss seine Augen auf und blickte in sein eigenes Gesicht. Zwischen einem verfilzten roten Bart und Haarschopf blitzen kleine Augen hervor, welche ihn erstaunt und etwas erschrocken ansahen. Ein starkes Gefühl von Déjà-vu erfasste Kjall.

Kurzerhand packte Kjall den zweiten Kjall am Handgelenk, zog sich geschmeidig in die Höhe und blickte sich in der Zelle um. Das Zeitportal hatte sich bereits wieder geschlossen. Wie erwartet lagen Fürst Hallwort und Wächter Bort noch ohnmächtig neben ihm, während ein zweiter Kjall vor ihm stand und ihn anstarrte. Die Zellentür stand sperrangelweit offen und ein Casamatuc lag am Boden. Soweit, so gut. Jetzt musste es nur rasch gehen.

„Schnell“, rief Kjall seinem Doppelgänger entgegen, „Hilf mir, die beiden anderen aus dieser Zelle zu schaffen. Keine Sorge, das war ihre erste Zeitreise. So schnell werden sie nicht wieder erwachen.“

Ohne auf eine Antwort zu warten, packte der Kjall Bort an den Schultern und begann, ihn in Richtung der immer noch offen stehenden Zellentür zu bugsieren. Was hatte dieser Zwerg nur gegessen, schweres Roteisen?! Ein Schleifgeräusch und einige unterdrückte Flüche verrieten Kjall, dass sein Doppelgänger sich dem ohnmächtigen Hallwort zugewandt hatte und diesen aus der Zelle herauszerrte.

Kjall zog den ohnmächtigen Bort in einen dunklen Seitengang. Eine ehemalige Zelle? Ein Lagerraum? Jedenfalls lagen einige Decken am Boden. Kjall ließ Bort auf eine dieser Decken sinken und überlegte. Er musste so schnell wie möglich seine Ketten lösen und fliehen. Falls irgendwie möglich, im Besitz der Sphäre – so nahe würde er ihr so bald wohl nicht wieder kommen. In einigen Minuten würden sowohl der Schlüssel für seine Ketten als auch die Sphäre beinahe unbewacht in seiner Zelle liegen, wenn diese Zeitlinie sich gleich entwickelte wie diejenige, aus der er stammte. Und damit sich diese Zeitlinie gleich entwickelte, musste Kjall es nur seinem ersten Doppelgänger nachtun und diese zweiten Doppelgänger zurück in die Zelle sperren.

Ein lautes Ächzen zeigte Kjall, dass sein Doppelgänger soeben den Seitengang erreicht hatte und Hallwort erleichtert zu Boden plumpsen ließ. Kjall wandte sich an seinen Doppelgänger und sagte: „Wir müssen noch mal schnell in die Zelle zurück.“

Ohne auf eine Antwort zu warten, bewegte er sich in Richtung der Kerkerzelle zurück. Die Kette, die seine Hände und Füße an Bort knüpfte, war so lang, dass er gerade noch so die Zellentür erreichen konnte, ehe seine Kette anschlug.

„Der Casamatuc!“, zischte er seinem Doppelgänger zu, „Rasch!“

Der zweite Kjall huschte an ihm vorbei in die Zelle zurück, griff nach dem dort liegenden Casamatuc und übergab das Werkzeug an Kjall: „Er wird die Ketten leider nicht lösen können, das habe ich bereits versucht.“

Kjall nickte: „Ich ebenfalls. Aber wir wollen doch keine Spuren zurücklassen. Deswegen musst du auch in dieser Zelle bleiben.“

Auf dem Gesicht des zweiten Kjalls zeigte sich zunächst Verwirrung, dann rümpfte er die Nase: „Bei Nehals verfaulten Zähnen, niemals bleibe ich hier zurück! Ich fliehe mit dir!“

„Das glaube ich nicht“, sprach Kjall und zog die schwere Zellentür zu, sodass der zweite Kjall in der Zelle feststeckte, während er selbst außerhalb der Zelle stand. Geschickt setzte Kjall den Casamatuc an die Zellentür und drehte ihn leicht. Ein leises Klonk verriet ihm, dass die Tür sich erfolgreich verschlossen hatte. Sehr schön, alles lief nach Plan! Kjall drehte sich ungerührt um und entfernte sich wieder von der Zellentür, während sein Doppelgänger fluchte

„Kjall! Lass mich hier nicht im Stich! Wenn du mich nicht sofort hier herauslässt, schreie ich und alarmiere die Wachen!“

Dieses Wargorgehirn! Verstand er denn nicht, dass diese Situation ihm nicht schaden würde? Kjall seufzte. Natürlich verstand er es nicht. Noch nicht. Gereizt gab er zurück: „Jetzt sei schon still. Wenn du dich geschickt anstellst, bist du schon bald in derselben Lage wie ich.“

Mit diesen Worten huschte Kjall in den dunklen, modrigen Seitengang zurück, in welchem die beiden Kjalls Bort und Hallwort zurückgelassen hatten. Drei Knalle hallten durch den Gang. Sein Doppelgänger hatte soeben wütend gegen die Zellentür getreten.\bigskip



Bort wachte als nächster auf und war kurz davor, sich auf Kjall zu stürzen. Als er aber das kleine Messer sah, das Kjall von Hallworts Gürtel entfernt hatte, um es dem schlafenden Fürsten an den Hals zu halten, verhielt Bort sich plötzlich ganz ruhig. Kjall setzte dennoch warnend einen Zeigefinger auf die Lippen. Nicht, dass Bort noch auf den Gedanken kam, den Bort und den Hallwort aus dieser Zeitlinie mit seinem Geschrei zu warnen. Aber groß fürchtete sich Kjall nicht davor. Der Fürst bedeutete Bort viel zu viel, als dass er sein Leben so leichtfertig aufs Spiel setzen würde. Das Kjall noch nie mit einem solchen Messer umgegangen war und erst recht noch nie jemandem einen Todesstoß versetzt hatte, musste er dem Wächter ja nicht auf die Nase binden. Ein Mörder war Kjall bei weitem nicht. Aber einschüchternd wirken konnte er.

Hallwort regte sich gerade noch rechtzeitig, damit Kjall auch ihm wortlos bewusst machen konnte, in welcher Lage er sich befand. Laute Schritte aus dem dunklen Gang kündigten die Ankunft des Hallworts aus dieser Zeitlinie an, welcher gerade seine Sphäre zum zweiten Kjall in der Zelle brachte und am Lagerraum vorbeihastete, ohne die Zeitreisenden zu bemerken.

Hallwort sah sich interessiert um: „Ist das hier die Vergangenheit? Sieht nicht ganz so aus, wie ich es erwartet hätte.“

„Keinen Mucks mehr“, zischte Kjall den beiden Zwergen zu, „Wir wollen ja nicht eure Doppelgänger aus dieser Zeitlinie alarmieren.“.

Bort warf ihm nur einen hasserfüllten Blick zu und sah dann zu Boden. Hallwort hingegen gab sich völlig ungerührt angesichts des Messers an seiner Kehle und antwortete leise: „Wie kommt Ihr denn darauf, dass wir sie alarmieren könnten, werter Jork? Falls das überhaupt Euer Name ist – diese prekäre Situation lässt darauf schließen, dass Ihr uns in so einigen Dingen nicht die Wahrheit erzählt habt.“

Kjall ignorierte den Fürsten zunächst. Nach einigen Momenten der Stille überwiegte sein Interesse aber und er fragte: „Hallwort, wenn ich es wollte, was würde mich schon daran hindern können, jetzt hier rauszurennen und die Aufmerksamkeit deiner Wachen auf mich zu ziehen?“

Hallwort grinste ihn an, als er weiterflüsterte: „Nun, es gibt nichts, was Euch daran hindern würde, das zu tun. Aber ich weiß mit absoluter Sicherheit, dass Ihr es nicht tun werdet. Denn wenn Ihr es getan hättet, dann wären der gute Bort und ich ja alarmiert worden und befänden uns jetzt nicht hier. Es ist wie bei den echten Prophezeiungen – keine Kraft dieser Welt zwingt einen dazu, sie in Erfüllung zu bringen, und dennoch werden sie sich früher oder später bewahrheiten.“

„Kaulquappenquatsch!“, brummelte Kjall, „Das hier ist eine ganz andere Zeitlinie als die, aus der wir kommen.“

„Ach?“, meint Hallwort, „Und sie entwickelt sich zufälligerweise genau gleich wie die unsere, trotz unserem Einwirken? So lauscht doch...“

Von der Zelle am Gangende erklang soeben die laute Stimme Kjalls: „Ihr trinkt Euren Drachentee mit immer mit zwei Löffeln Honig drin, auch wenn Ihr das bei jeder passenden Gelegenheit abstreitet. Ihr führt eine geheime Tölplingzucht im Brauneisenstein. Als Kind seid Ihr einmal in den Geheimen See gefallen, als ihr heimlich und unerlaubt Cavern zu erkunden versuchtet, und eine von Wasser triefende, grünliche Gestalt rettete euch.“

Bort nickte nun auch: „Ich glaube, das sind dieselben Worte. Genau dieselben.“

Kjall schüttelte den Kopf: „Nein, nein, das muss eine andere Zeitlinie sein, sonst hätte ich das Vergangene ja gar nicht erst durch meinen Angriff auf die Andori verändern können!“

Hallwort antwortete nachdenklich: „Wie kommt Ihr darauf, dass Ihr sie verändert habt? Habt Ihr erreichen können, was Ihr erreichen wolltet?“

„Noch nicht“, knurrte Kjall, „Aber wartet nur, bis ich hier raus bin.“

Natürlich hatte Kjall sich schon die Möglichkeit überlegt, dass die Zeitlinie vorgeschrieben und unveränderlich sein könnte. Aber da er in keinem der unzähligen Geschichtsbücher und Schriftrollen, die er im Laufe seines Lebens verschlungen hatte, vom Angriff eines Schattens wie diesem Bleichen oder Shan oder von einem Angriff der Tiefminen-Golems auf die Brandurs Lager gelesen hatte, war er sich so sicher gewesen, dass die Sphäre ein Portal in eine andere Zeitlinie öffnen musste. Wie hätte er denn ahnen sollen, dass ihr Vorhaben so spektakulär scheitern würde? Nein, es konnte nicht sein! Es durfte nicht sein! Es musste eine Möglichkeit geben, Cavern vor dem Verfall zu bewahren, Xolls Tod zu verhindern und Radans Verbannung ungeschehen zu machen.

Hallwort unterbrach Kjalls Gedanken: „Wisst Ihr, ich habe mir bereits ausführlich Gedanken zu Zeitreisen gemacht. Einst kam ein Wanderer aus dem Hohen Hadria in Cavern zu Besuch, um seine Ausbildung zum vollwertigen Zauberer abzuschließen. Wie war sein Name schon wieder... Kohlaff? Er berichtete mir von Hadrischen Stundengläsern, die die Zeit verlangsamen konnten, und wir tratschten nächtelang darüber, was es denn bedeuten würde, wenn man die Zeit nicht nur komplett anhalten, sondern vielleicht sogar so sehr verlangsamen könnte, dass sie rückwärts abliefe. Wir kamen zum eindeutigen Schluss, dass Zeitportale, wie sie sich beispielsweise in den Tiefen des Horun zu manifestieren scheinen, einen nur in dieselbe Zeitlinie befördern können, aus der man stammt.“

Hallworts Augen glänzten schon wieder von Aufregung, und er schien das Messer an seiner Kehle komplett vergessen zu haben. Er strotzte vor Selbstbewusstsein, dabei musste er doch falsch liegen! Wenn Hallwort damit recht hatte, dass sie sich in derselben Zeitlinie befanden, aus der sie gekommen waren, dann könnten Kjall, der Fürst oder Bort ja mit jeder unbedachten Handlung Ereignisse in den Gang setzen, deren Auswirkungen früher oder später in einem Paradoxon enden würden! Was, wenn Kjall beispielsweise Xolls Tod verhinderte? Dann wäre Kjall selbst ja ganz anders aufgewachsen und hätte sich wohl nie dem Zeitreisen verschrieben – und der Kjall, der er in diesem Moment war, würde sich in Luft auflösen müssen, aufhören zu existieren!

„Und überhaupt“, fuhr Hallwort gedankenverloren fort, „Wenn Ihr mit diesem Portal in andere Zeitlinien reisen könntest, so würde das doch heißen, dass Ihr bei jeder Reise eine Zeitlinie zurücklässt. Wisst Ihr überhaupt, wie viele Zeitlinien dadurch bereits entstanden wären? Das Schicksal einer einzelnen Zeitlinie wäre doch komplett unbedeutend im See all dieser Möglichkeiten. Um das Beispiel vom Untergang Caverns zu nehmen: Ihr könntet vielleicht in eine Zeitlinie reisen, in welcher Cavern vor dem Untergang bewahrt wird, aber würdet dabei immer eine Zeitlinie zurücklassen, in welcher Cavern untergeht. Ich ging davon aus, dass mein zukünftiges Ich Euch solche Sachen erklärt hätte und dass der Untergang Caverns, aufgrund dessen ich Euch laut Eurer Behauptung hierhin zurück geschickt hätte, in der Zukunft bereits verhindert worden wäre...“

„Aber wenn es bereits verhindert worden wäre, warum würdet Ihr mich dann noch zurückschicken?!“, brauste Kjall auf. Sein Messer zitterte gefährlich nahe an Hallworts Adamsapfel, welcher erst jetzt wieder zu realisieren schien, dass er in Lebensgefahr schwebte.

„Nun, Jork, das müsst Ihr jetzt auch nicht sofort verstehen...“, versuchte Hallwort, Kjall zu beruhigen. Aber Kjall hatte genug. Er konnte seine angespannte Stimmung nicht mehr unterdrücken.

So leise wie möglich zischte er aufgebracht: „Weißt du, wie wir dich in der Zukunft nennen, wenn du erst einmal den Hammer abgegeben hast?! Hallwort den Nutzlosen! Hallwort den Pilzzüchter! Hallwort den Wahnwitzigen! Du hast nichts Großes an dir, du neugieriger Narr! Hältst dich für etwas Besseres als wir anderen. Lässt Cavern im Stich! Du steuerst unseren Untergang herbei, und du verstehst nicht einmal, dass es sich abwenden ließe! Ich sollte mich deiner hier und jetzt entledigen, immerhin würde Hallgard als Fürst nicht...“

„Das solltest du nicht tun!“, meldete sich Bort nun zum ersten Mal seit einigen Minuten wieder zu Wort, „Wenn du Fürst Hallwort etwas antust, wirst du diesen Lagerraum nicht mehr lebend verlassen, das schwöre ich bei der Axt meiner Urahnen!“

„Stimmt, das ist sehr spannend“, warf Hallwort ein. Seine Stirn war gefurcht. Offenbar war er nicht glücklich über die Spitznamen, die man ihm in der Zukunft verleihen würde, verkniff sich aber irgendwelche bissigen Antworten. „Offenbar wurden wir bei unserer Reise durch die Zeit auch örtlich versetzt, von Jorks Zelle in diesen Lagerraum, vermutlich aufgrund der natürlichen Drehung der Weltenscheibe...“

„Wieder falsch!“, fauchte Kjall nun, „Wir landeten am exakt selben Ort. Mein Doppelgänger und ich brachten euch ohnmächtige Zwerge in diesen Raum, ehe er in seine Zelle zurückkehrte!“

„Umso besser!“, rief Hallwort auf, „Das sollte meine Theorie doch bestätigen können. Habt Ihr bereits beide Rollen dieser Interaktion identisch durchlebt? Das wäre ein deutlicher Indikator, dass die Zeitlinie vorbestimmt...“

„Hallwort, wenn du nicht sofort den Mund hältst...“, setzte Kjall an.

„Drohe meinem Fürst noch ein einziges Mal, Verräter, und ich...“, fiel ihm Bort ins Wort.

„Vielleicht könnten wir alle einmal tief durchatmen“, schlug Hallwort vor, „Überlegen wir lieber, wie es weitergehen soll. Ich maße mir nicht an, Eure Motive zu verstehen, aber ich nehme an, dass Ihr gerne von hier fliehen würdet. Ich und Bort würden gerne weiterleben und könnten uns wahrscheinlich mit Eurer Flucht abfinden, solange Ihr fortan einen weiten Bogen um Cavern macht.“

Bort murmelte irgendetwas, das eindeutig nicht nach einer Zustimmung klang.

Kjall sammelte sich. Ja, er musste an seine Zukunft denken. Erst einmal weg von hier, raus aus Cavern, und dann könnte er noch lange genug überlegen, wie er den Verfall Caverns verhindern könnte.

„Das folgende wird nun geschehen“, setzte er an, „Wir bewegen uns alle drei rüber zur Zelle, wo inzwischen nur noch dieser zweite Schildwächter... wie lautet sein Name?“

„Ihr Name ist Ragga.“

„...wo sich inzwischen nur noch Ragga aufhält. Ihr entfernt meine Ketten, übergebt mir die Sphäre und ich ziehe von dannen. Falls ich irgendeine unbedachte Bewegung von Bort oder Ragga – oder Hallwort – sehe, so findet diese Messerklinge ein Ziel und Hallwort ein unschönes Ende. Wir wollen das alle nicht, also seht zu, dass es nicht soweit kommt. Alle einverstanden?“

Hallwort nickte ergeben, während Bort immer noch mit den Zähnen knirschte. Dann nickte er ebenfalls. Kjall empfand wieder Hoffnung. Bald hätte er das alles hinter sich.\bigskip



Ragga stand alleine inmitten der Zelle, hielt die Sphäre in ihrer Hand und versuchte wahrscheinlich gerade herauszufinden, wohin Hallwort, Bort und der Gefangene durch das Zeitportal verschwunden waren.

„Ragga, sei so lieb und roll die Sphäre zu mir rüber, willst du?“, fragte Kjall höflich, als sich seine Prozession der Zellentür näherte. Ragga blickte auf und erstarrte, als sie Kjalls Messerklinge an Hallworts Hals sah.

„Den Schlüssel zu meinen Ketten bitte ebenfalls. Ich will Bort nicht für den Rest meines Lebens hinter mir herschleppen.“

Eines musste man Ragga lassen, sie war erheblich schlauer als Bort und folgte Kjalls Anweisungen ohne Protest. Bort hingegen fluchte so sehr, dass Kjall sich gleich mehrere kreative Ausdrücke für spätere Gelegenheiten merkte, während Borts Ketten von Kjall gelöst und Bort damit an Kjalls Pritsche gekettet wurde.

Ein wenig Gerangel gab es tatsächlich auch noch, als Kjall die drei anderen Zwerge in der Zelle einzuschließen versuchte. Sein Messer an Hallworts Kehle war das einzige Druckmittel, das er hatte, und auch wenn das Leben des Fürsten der Schildzwerge einiges wert war, war sich Kjall natürlich bewusst, dass der kräftige Fürst ihn mit wenigen Faustschlägen zu Boden bringen konnte, wenn er ihm die Gelegenheit dazu gab. Also achtete Kjall ganz besonders auf Hallwort, als er diesen langsam losließ und ins Innere der Zelle dirigierte. Darum hätte Kjall beinahe Ragga übersehen, als diese sich überraschend auf ihn warf und ihn zu Boden drängen versuchte. Nur Kjalls schnelle Reflexe bewahrten ihn davor, für den Rest seines Lebens in einem Kerker der Schildzwerge zu versauern. So trat Kjall gerade noch rechtzeitig zurück, stieß Ragga von sich und riss – bereits zum zweiten Mal am heutigen Tag – die Zellentür zwischen sich und den anderen Zwergen zu. Mit der passenden Drehung des Casamatucs und einem dumpfen Klonk verschloss sich die Zellentür. Kjall trat zurück, Casamatuc in der linken, Sphäre in der rechten Hand.

Hallwort sah ihn nachdenklich an, Bort musterte ihn hasserfüllt und Ragga zeigte ihm eine unschöne Geste. Kjall hob seine Hand zum Gruß, drehte sich dann um und rannte den dunklen Gang entlang. Wahrscheinlich würde es einige Stunden dauern, bis die nächsten Wachen auf ihrer halbtäglichen Runde an dieser Zelle vorbeikommen und die gefangenen Zwerge bemerken würden, aber er wollte sein Glück nicht unnötig herausfordern. Erst recht nicht, als die Zwerge in der Zelle hinter ihm um Hilfe zu schreien begannen.

In der Stoffkleidung, mit denen die Schildzwerge ihren Gefangenen ausstatteten, schepperte Kjall nicht einmal, als er so schnell er konnte durch die finsteren Gänge und Stollen Caverns huschte, stets auf der Hut vor Patrouillen der Schildzwerge oder Minenarbeitern aus den Tiefminen. Obwohl sein Herz raste und er in jedem zweiten Schatten einen Zwerg erkannte, war Kjall in Hochstimmung. Er war nicht nur frei, er besaß sogar die Sphäre. Ihr kaltes Metall brannte in seiner Faust, und ein leises Summen verriet ihm, dass sie noch genug Saft für mindestens einen Zeitsprung besaß, wahrscheinlich sogar mehr. Nur wohin sollte er sich nun wenden? Die Stollen, die zu den Zellen führten, waren ihm unbekannt gewesen, aber diese Fackel von Cavern in der Halterung da vorne kannte er nur allzu gut! Kjalls Werkstatt und sein geheimes Museum lagen hier ganz in der Nähe!

Kurz überlegte Kjall, ob er seiner Werkstatt einen Besuch abstatten sollte, entschied sich dann aber dagegen. Wer weiß, wie sein jüngeres Ich darauf reagieren würde, wenn es einer faltigeren Version seiner selbst im Gefangenenoutfit gegenüberstehen würde? Lieber würde er zum Verlassenen Turm aufbrechen. Kjall wusste nicht, wie es seinen Verbündeten ergangen war, aber wenn der Bleiche, Shan oder Thogger die Begegnung mit den Helden besser überstanden hatten als er, würden sie sich wahrscheinlich ebenfalls dort einfinden. Die Schar hatte sich über längere Zeit in diesem Turm eingerichtet. Nur eine der unzähligen tragisch verlassenen Zwergenbauten im Grauen Gebirge. Das Zwergenreich war am Verenden und viele Zwerge kümmerte es nicht einmal! Kjall spuckte aus.\bigskip



Der Weg zum Verlassenen Turm verlief größtenteils ereignislos. Einmal musste Kjall einer Horde kichernder Zwergenkinder ausweichen, welche sich der Brücke zum Roteisenstein entlanghangelten. Wo waren nur deren Eltern?! Ein anderes Mal konnte er sich gerade noch rechtzeitig in einen Erker verdrücken, als eine Horde grölender Minenarbeiter an ihm vorbeizog. Und dieses Mal horchte er auf, denn einige dieser Leute erkannte er!

Da war der breite Radan, derjenige Zwerg, der für ihn wohl einem Freund am nächsten kam, eng versunken in eine energische Diskussion mit einem Spitzhackenträger. Auf dem Weg zu seinem Wachdienst? Noch hatte er keine Ahnung davon, dass er einst dafür verbannt werden würde, für seine Prinzipien einzustehen. Es war wirklich eine Schande.

Dort war der blonde Drak, der soeben prahlte vom köstlichen Waldpilz-Eintopf, den seine Gemahlin heute Abend für seine Familie zubereiten würde. Kram, einer von Draks unzähligen Kindern, war im Moment vielleicht sogar noch zu jung, um feste Nahrung zu kosten, sollte aber nach Hallworts Sohn Hallgard zum nächsten Fürsten Caverns werden und Radans Verbannung veranlassen... und... und steckte in diesem Moment ebenfalls in dieser Zeitlinie fest! Das hatte Kjall völlig vergessen! Der Kram aus der Zukunft befand sich ja mit den restlichen Helden auch hier in der Vergangenheit! Solange Kram ebenfalls in dieser Zeit festsaß, kümmerte es Kjall nicht einmal so sehr, falls er nicht aus dieser Zeit fliehen können würde. Ohne Kram, der noch keine Nachkommen, ja nicht einmal einen Lebensgefährten hatte, musste das Amt des Fürsten von Cavern doch an jemand anderes aus Hallworts Blutlinie fallen, der die Schildkrone zumindest ein bisschen mehr verdiente als ein dahergelaufener Zwerg aus den Tiefminen. Kjall musste nur darauf achten, dass die Helden mit ihrer Sphäre nicht seinen Plänen, Andor zu verhindern, einen Strich durch die Rechnung machen konnten.

Aber jeder Gedanke an Kram war vergessen, als Kjall das letzte Mitglied der Zwergenhorde erkannte: Da war Xoll Hammeraxt, Meisterschmied und Meisterkrieger – und außerdem Kjalls Vater. Kjalls Augen wurden feucht, als er sah, wie Xoll Hammeraxt den Stollen entlangstapfte und fröhlich über einen Witz Radans lachte. Erinnerungen an ihre gemeinsame Zeit schossen Kjall durch den Kopf. Konnte Kjall ihn davon abhalten, zur Befreiung der Rietburg aus den Händen dieses vermaledeiten Dunklen Magiers auszuziehen? Konnte Kjall Xolls unnötigen Tod verhindern? Was, wenn Hallwort recht gehabt hatte und Kjall immer noch in derselben Zeitlinie feststeckte – würde Kjall dadurch dann ein Paradoxon auslösen und seine eigene Existenz auslöschen? Nein, er hätte durch seine schiere Anwesenheit in dieser Zeitlinie doch bestimmt bereits mit irgendwelchen unbedachten kleinen Taten ein Paradoxon ausgelöst. Dass er noch hier war, bewies doch gerade, dass er, Hallwort und Bort sich in einer alternativen Zeitlinie befinden mussten. Und das hieß, dass er Xoll vielleicht vor seinem Tod bewahren können würde.

Diese und ähnliche Gedanken schossen durch Kjalls Kopf, während er durch die Gänge Caverns schlich, langsam, aber stetig dem Verlassenen Turm entgegen.\bigskip



Als Kjall an der großen Waffenkammer der Schildzwerge vorbeikam, hielt er inne und spähte vorsichtig hinein. Leer. Perfekt. Er schlich langsam im Schatten der Wände und Sockel durch den gewaltigen Raum, den die Zwerge einst ins Gebirge geschlagen hatten, und spähte umher. Feingearbeitete Kettenhemde, goldene Brustpanzer, große kantige Helme, alles fürstliche Rüstwaren, alles langweilig. Da fiel Kjalls Blick auf einen Schild in der Ecke. Er lief zu ihm, nahm ihn auf, betrachtete die reichen Verzierungen und dachte an Nehal, den Drachen und Kreatok, den Meisterschmied der Schildzwerge, die einst gemeinsam der Schmiedekunst nachgingen und die vier mächtigen Schilde erschufen. Dies hier war der Silberschild. Kjall lachte. „Dieses nutzlose Ding! Einer der vier ‚mächtigen‘ Schilde aus uralter Zeit, einfach so ungeschützt in einer Waffenkammer verstaubend! Ha, ha.“

Kjalls Sammeldrang überwog seine Furcht, durch den Diebstahl des Silberschilds ein Paradoxon auszulösen und seine eigene Existenz zu beenden. Diese Furcht legte sich endgültig, nachdem er den Schild angehoben und angelegt hatte, ohne sich in Luft aufzulösen. Als er den Verlassenen Turm erreichte, hatte er den Silberschild fest auf seinen Rücken geschnallt.\bigskip



Der Verlassene Turm war, nun ja, verlassen. Von Kjalls Kumpanen befand sich keiner dort, weder war die düstere Präsenz vom Bleichen oder Shan zu spüren, noch ließ sich der massige Tarus Thogger blicken. Kjall stand auf der Spitze des Turms und ließ seinen Blick über das Land schweifen, während die Sonne unterging und das Rietland in goldenen Farbtönen erglühen ließ. Dort drüben konnte er gerade noch den hohen Turm von Brandurs Lager erkennen. Nirgendwo war ein ungewöhnlich dunkler Schatten zu sehen. Waren seine Verbündeten gescheitert oder hatten sie sich nur an einen anderen Ort zurückgezogen? Es sah so aus, als wäre Kjall erstmals auf sich alleine gestellt. Aber das war in Ordnung. Die Schatten ließen ihm ohnehin immer Schauer über den Rücken laufen.

Was sollte Kjall als nächstes tun? Brandur entführen konnte er ohne seine Golems kaum, erst recht nicht, wenn die Helden aus der Zukunft ihn noch beschützten. Und weiter zurück in die Vergangenheit zu reisen, würde ihm auch kaum etwas bringen können, da er auf sich alleine gestellt nicht die Kampfkraft zur offenen Konfrontation besaß. Kjall wollte zurück in seine eigene Zeit, zu seiner Werkstatt, seinem Museum, zu seinen Tiefminen-Golems. Konnte er vielleicht seinem vergangenen Ich die Tiefminen-Golems stehlen, ehe dieses sie in einen zum Scheitern verdammten Angriff auf die proviorische Rietburg sandte? Wenn er diese erst einmal wieder hätte, könnte er noch tiefer ins Vergangene zurückreisen und Brandurs Schar vielleicht gleich nach ihrer Überquerung des Grauen Gebirges abpassen. Ironischerweise würde es Kjall sehr schwer fallen, die Kontrolle über die Tiefminen-Golems von seinem früheren Ich zu übernehmen, so lange die Tiefminen-Golems sein früheres Ich beschützen. Es war zum Verrücktwerden!

Kjall wollte im Moment nichts weiter, als sich auszuruhen, zu sammeln und an einem sicheren Ort einen neuen Plan auszuhecken. Aber einfach so zurückkehren in seine eigene Zeit konnte er auch nicht, da er nur wusste, wie er Portale in Vergangenes, aber nicht in die Zukunft öffnen konnte. Wobei... wenn er irgendwie wieder zurück in die Zukunft gelangen könnte, könnte sein zukünftiges Ich ein Portal in die Vergangenheit öffnen, an diesen Ort, zu diesem Zeitpunkt! Hoffnungsvoll blickte Kjall um sich, in der Erwartung, einen blauen Funken zu sehen, der sich zu einem Portal verformen würde. Kein Funke kam. Keine zukünftige Version seiner selbst öffnete ein Portal in die Vergangenheit, um Kjall vom Verlassenen Turm abzuholen. Was hatte das zu bedeuten?

Nach einer halben Stunde war die Sonne komplett untergegangen. Der blaue und der rote Mond standen beide hoch am Himmel, aber ihr Licht konnte gegen die Kälte der Nacht nicht ankommen. Aus der Ferne ertönte der kehlige Ruf eines Skrals. Kjall gab es auf, auf ein Portal aus der Zukunft zu warten, und zog sich ins Innere des Turms zurück, wo er mithilfe eines Glutstabs ein kleines Feuer entfachte. Dann wechselte er seine Gefängnismontur gegen eine verstaubte Wanderkleidung, die er nebst dem Silberschild und dem Glutstab in der Waffenkammer gefunden hatte. Der Silberschild! Kjall setzte sich neben den verzierten Schild mit den vier Stacheln und musterte ihn. Tastete ihn ab, versuchte, irgendeine magische Verbindung zu spüren. Nichts, außer dass der Nachtwind vielleicht noch etwas kühler durch das verlassene Gemäuer strich. Einige Zwerge nannten ihn den ‚Sturmschild‘ nach den mysteriösen Umständen um Hallworts Tod. Warum wollten diese Jungspunde für alles neue Namen finden? ‚Silberschild‘ war doch eine viel treffendere Bezeichnung.

Dieser Schild war für Kjall schon seit je her ein Rätsel gewesen. Was hatten sich Kreatok und Nehal nur dabei gedacht, als einen Schild ohne besondere Fähigkeiten erschaffen hatten? Eine Aussage über Ästhetik im Allgemeinen? Dass der Schild wunderschön war, ließ sich kaum bestreiten. Vielleicht waren die Fähigkeiten des Schildes einfach noch nicht entdeckt worden? Der Legende nach hatte er die Piraten mit Blitzen erschlagen, die ihn in einigen Jahren zu stehlen versuchen würden, aber ob das wirklich das Tun des Schildes gewesen war. Und ein letztes Mal reiterierte er: Kjall besaß jetzt den Silberschild. Demnach konnte Hallwort ihn in einigen Jahren nicht mit auf seine Reise in den Norden nehmen, und keine Piraten konnten versuchen, ihn zu stehlen. Kjall hatte durch den Diebstahl des Silberschilds nicht nur kein Paradoxon ausgelöst, sondern auch bewiesen, dass er mit seinen Handlungen die Zukunft ändern konnte und sich in einer neuen Zeitlinie befand. Oder? Hm... er musste sich morgen näher dazu Gedanken dazu machen. Jetzt wollte er einfach nur schlafen.

\newpage
\section{Gesehen}


Kjall konnte kaum schlafen. Er wälzte sich herum, auf den Bauch, auf den Rücken, auf die Seite, aber keine der Lagen war bequem. Nicht, dass er sich je groß um Bequemlichkeit geschert hätte. Er konnte sich jedoch nicht daran erinnern, derart heftige Schlafprobleme erlebt zu haben. Heute konnte er seine aufgewühlten Gedanken einfach nicht zur Ruhe bringen. Und so lag Kjall still da und ließ seine Gedanken schweifen.

Noch bevor die Sonne sich am nächsten Tag über die Berggipfel erhob, beschloss Kjall, diesem elenden Halbschlaf ein Ende zu bereiten und zum Baum der Lieder aufzubrechen. Wenn irgendwo weitere Schriftrollen zum Gebrauch dieser mystischen Sphäre zu finden waren, dann wohl in den Schwarzen Archiven. Ohne seine Tiefminen-Golems würde es für Kjall schwieriger werden, dort einzubrechen, aber das Risiko wollte er auf sich nehmen. Er wollte zurück in seine eigene Zeit, sein eigenes Museum, und wenn keine zweite Version seiner selbst ihn dorthin bringen wollte, musste er halt selbst herausfinden, ob diese Sphäre auch Portale in die Zukunft öffnen konnte. Schlimmstenfalls, falls er entdeckt würde, könnte er einfach ein kleines Stückchen weiter in die Vergangenheit fliehen.

Einige Apfelnuss-Sträucher am Rande des gewundenen Wegs verliehen Kjall etwas dringend gebrauchte Stärke, während er in seiner neuen Wanderkleidung durch den Wachsamen Wald strich, den Silberschild fest auf seinen Rücken gezurrt und mit einer dünnen Stoffdecke bedeckt, sodass er nicht allzu sehr auffiel. Kjall konnte nur hoffen, dass die Helden von Andor nicht ebenfalls beschlossen hatten, den Baum der Lieder aufzusuchen.\bigskip



Erstes Sonnenlicht ergoss sich über den Wachsamen Wald. Die Lichtung, über welcher der Baum der Lieder imposant thronte, lag vergleichsweise still da. Die meisten Bewahrer vom Baum der Lieder waren wahrscheinlich noch in ihren Betten oder kümmerten sich um ihre allmorgendlichen Rituale. Einige Dorfbewohner zogen bereits zum Brunnen und brachten Wasser zurück in ihre Häuser. Bogenschützen patrouillierten in Dreiergruppen am Rande der Waldlichtung entlang. Ein kleines Männchen saß auf einem Hocker am Rande der Lichtung und blickte Kjall aus zwei verschmitzten Augen unverwandt an. Wäre es vielleicht geschickter, umzudrehen und in der nächsten Nacht zurückkehren? Da fiel Kjalls Blick auf eine mächtige Gestalt, die in der Nähe des kleinen Männchens auf einem Stein saß, im Schneidersitz, die Hände auf die Knie gelegt und den Kopf mit den mächtigen Hörnern in Andacht gesenkt. Ein Tarus! Ein ihm wohlbekannter Tarus! Er schien etwas vor sich hin zu murmeln, und wenn Kjall sich nicht täuschte, wogte das Gras um den Stein herum, flimmerte die Luft um den Tarus leicht und knirschten die nahestehenden Bäume, als würde die Seele des Waldes dem Tarus antworten. Geisterhaft.

Um nicht den Verdacht der Wache haltenden Bogenschützen auf sich zu ziehen, näherte sich Kjall dem Tarus langsam, aber selbstbewusst aufgerichtet. Dass er abgesehen vom Schild auf seinem Rücken komplett unbewaffnet war, ließ ihn relativ ungefährlich erscheinen, dennoch wollte er keine Konfrontation auslösen. Das kleine bucklige Männchen betrachtete ihn weiterhin nachdenklich, während er näher schritt, nickte ihm dann aber freundlich zu und wandte sich von ihm ab. Er atmete erleichtert auf.

„Thogger“, zischte Kjall, und tippte dem Tarus auf die Schulter, die selbst im Sitzen die Schulter des Zwergs überragte.

Thogger schien nicht überrascht, sondern öffnete bloß die Augen: „Falls du dich anschleichen wolltest, so warst du eindeutig zu laut. Was willst du, Kjall?“

„Freust du dich nicht, mich zu sehen? Ich bin hier, um mehr vom Umgang mit der Sphäre zu erfahren. Ich will zurück in meine eigene Zeit reisen. Wie sieht es bei dir aus? Und wie steht es um die anderen?“

Thogger schnaubte: „Die haben versagt. Shan wurde in Brandurs Lager eingekerkert. Der Fahle hat tatsächlich einen Schleier der Nacht über das Land verhängt, aber dieser wurde kurz darauf wieder gebrochen, von einer Kräuterhexe an einem Feuerkreis. Ich wusste gar nicht, dass es einen in Andor gibt. Mit dieser Hexe und dem buckligen Hüter dort drüben habe ich mich schon ausführlich unterhalten. Sie verfügt über so viele Kenntnisse, dass sie eine gute Schamanin abgegeben hätte, und er ist weiser, als es selbst mein Vater Hogger war. Die Hexe scheint tatsächlich zu wissen, wo Sternkraut wächst, und könnte mich dorthin führen. Das ist gut, denn diese Helden, die mir dasselbe versprochen hatten, haben sich nach ihrem Sieg über den Fahlen in Luft aufgelöst. Ich vermute, dass sie einen Weg zurück in ihre eigene Zeit gefunden und uns hier zurückgelassen haben. Schöne Helden sind das. Ich weiß nicht, wie du ohne die Sphäre nach Hause zurückkehren willst.“

„Wenn es nur das ist...“, grinste Kjall, und ließ die schimmernde Sphäre aus einer seiner Taschen hervorblitzen. Thogger riss die Augen auf. Kjalls Grinsen wurde noch breiter: „Direkt aus der geheimen Fürstenkammer der Schildzwerge. Wenn du wüsstest, was ich alles erlebt habe... aber das ist jetzt nicht relevant. Kommst du mit, wenn ich herausfinde, wie wir uns zurück in unsere Zeitlinie transportieren können?“

Thogger zögerte: „Ich habe geschworen, mich Varatans Fluch hinzugeben, sobald das Sternkraut sicher in meiner Heimat angekommen ist. Diese Hexe vermag mich zum Kraut zu führen. Ich glaube, hier meine Bestimmung gefunden zu haben. Ich stärke mich nur noch für diese letzte Reise, dann breche ich auf. Meine Wünsche sind mit dir auf deinem weiteren Pfad, Kjall, aber ich glaube, hier trennen sich unsere Wege.“

Kjall nickte schwer. Er hatte nie ganz verstanden, was es mit diesen Flüchen auf sich hatte, die auf Thogger, Shan und dem Bleichen lagen, aber wenn Thogger hier seinen Frieden finden konnte, würde Kjall ihn nicht aufhalten. Interessanter war, dass die Helden offenbar einen Weg zurück in ihre eigene Zeit gefunden hatten. Eine Schande, dass Kram nicht in der Vergangenheit festsaß, aber diese Möglichkeit ließ in Kjall die Hoffnung wachsen, dass er dasselbe erreichen könnte. Zudem hieß das auch, dass die Helden ihm in dieser Zeit nicht mehr gefährlich werden konnten. Kjall musste nur den riesigen Baum betreten, die Treppe hochklettern, die Schwarzen Archive erreichen und die Schriftrollen finden, die ihm den Weg in die Zukunft zeigen würden. Kein Problem, er hatte bereits einem Fürsten und seinen zwei Wachen getrotzt, was sollten ihn da einige blasierte Schriftensammler auch aufhalten.

Kjall griff dem riesigen Tarus an den Unterarm und sprach ihm viel Glück und den Segen des feurigen Gottes für seine Zukunft zu. Der Tarus verabschiedete sich gerührt und Kjall bewegte sich schnurstracks über die Lichtung auf den Baum der Lieder zu. Vielleicht war es doch eine schlechte Idee gewesen, bei Tageslicht hierher zu kommen. Die Bogenschützen hatten ihn nicht aufgegriffen, vielleicht konnte er aufgrund seiner Waffenlosigkeit auch als harmloser Gelehrter durchgehen, aber sein Aussehen weckte dennoch die Aufmerksamkeit vieler Dorfbewohner. Zwerge sah man hier nicht alle Tage, und erst recht nicht welche, die sich mit dem Tarus unterhielten, der erst kürzlich noch Sturm und Unwetter über den Baum beschworen hatte. Und der Silberschild auf seinem Rücken sah selbst von einem Tuch verdeckt immer noch imposant aus.

Ich bin ein ganz unschuldiger, schriftrollenversessener Narr, ganz genau wie alle anderen hier, dachte Kjall, achtet einfach nicht auf mich, ich bin nur wegen der Schriftrollen hier. Er schaffte es nur bis kurz vor den Eingang zum großen Tor am Baum der Lieder, dann hörte er eine helle Stimme vom Rande der Lichtung: „Das ist Jork, dieser elende Zwerg, der in die Schwarzen Archive eingebrochen ist!“

Kjall sah sich unauffällig um – oder besser gesagt versuchte es zumindest – und erblickte eine junge Bewahrerin, die mit ausgestrecktem Finger auf ihn zeigte. Er hatte diese Frau noch nie in seinem Leben gesehen.

Thogger sah nun ebenfalls alarmiert auf und meinte beschwichtigend: „Aber was redest du denn da, Folla, das ist doch nur Kjall, ein alter Freund...“

Folla quittierte diese Aussage mit „Schweig schon stille, unseliger Unruhestifter!“ und beharrte weiterhin darauf: „Das ist Jork, der Mörder von Fanatos! Fasst ihn!“

Das brachte Leben in die drei Bogenschützen, die Folla am nächsten standen. Kjall blieb nicht lange genug stehen, um ihnen beim Anrennen zuzuschauen. Stattdessen nahm er selbst die Beine in die Hand und hechtete auf den Eingang zum Baum der Lieder zu. Links und rechts von ihm zischte je ein weiß befiederter Pfeil durch die Luft, und zwei dumpfe Geräusche an seinem Rücken verrieten ihm, dass zwei weitere Pfeile am Silberschild an seinem Rücken abgeprallt waren. So war dieser Schild zumindest für etwas gut. Dann hatte Kjall das Innere des Baums der Lieder erreicht.

Ein Glück, dass er den inneren Aufbau des Baums so gut studiert hatte, als er seinen ersten Einbruch mit den Tiefminen-Golems geplant hatte. Dieses Meisterwerk aus Architektur und Naturkunde, eine Vielzahl von Räumen und Wohnungen in einem lebendigen Baum, hatte Kjall stark an die Art und Weise erinnert, wie die Zwerge Gänge und Stollen durchs Graue Gebirge gruben. Auch hier fühlte er eine tiefe Verbundenheit mit der Natur, als er im Innern des riesigen Baums nach oben sah. Aber davon durfte er sich nicht ablenken lassen.

Kjall wusste genau, wie er die Schwarzen Archive erreichen konnte. Die Wendeltreppe mit den viel zu großen Stufen hoch, durch den Eingang auf die Balustrade hinaus, um die halbe Balustrade herum, die Raumflucht betreten, zweite Tür links. Auf seinem Weg traf er auf allerlei verschlafenere und wachere Menschen, welche ihm neugierig hinterhersahen. Novizen, Adepten, untere Bewahrer, selbst der Oberste Bewahrer... Kjall hatte für sie keine Zeit, denn sobald die Bogenschützen unter der Anführung Follas die Wendeltreppe erreicht hatten, zischten wieder Pfeile an ihm vorbei.

„Fasst ihn! Kallun, so schnapp‘ ihn dir doch! Er hat Fanatos auf dem Gewissen!“, ertönte erneut Follas Stimme, und die ersten Bewahrer versuchten, sich in Kjalls Weg zu stellen. Einer bekam ihn auch tatsächlich zu fassen. Als Kjall sich losriss, löste sich das Tuch von seinem Rücken und enthüllte den Silberschild in all seiner Pracht. Dabei kannte Kjall doch gar keinen Fanatos. Von was sprach diese Frau nur?

Endlich erreichte Kjall die Raumflucht, die die Schwarzen Archive enthielt. Gerade noch rechtzeitig schaffte er es, die Tür hinter sich ins Schloss fallen zu lassen, als von draußen auch schon die ersten Faustschläge zu hören waren.

„Gib auf, Jork!“, schrie Folla nun, „Nur wenn du dich freiwillig stellst, wird das Urteil des Rats der Bewahrer milder ausfallen! Und der Rat wird dich früher oder später richten, du Mörder, da kannst du dich darauf verlassen!“

Ein flaues Gefühl machte sich in Kjalls Magen breit, aber er achtete nicht darauf. So war das nun wirklich nicht geplant gewesen. Aber wie schon bei ihren Anschuldigungen, hatte diese Folla auch mit ihrer letzten Aussage hatte Unrecht. Kjall verfügte sehr wohl über die Möglichkeit, dieser Situation ungestraft zu entkommen.

Nachdem Kjall mithilfe des fürstlichen Casamatucs sicherstellte, dass die Tür zu den Schwarzen Archiven demnächst verschlossen blieb, beschloss er, auf Nummer sicher zu gehen, und wuchtete ein schweres Schreibpult vor die Tür. Viel Zeit würde ihm das nicht verschaffen, aber zumindest genug, um die passende Stelle im Schwarzen Archiv aufsuchen zu können.

„Zwergenartefakte, Zwergenartefakte...“, murmelte Kjall, während er die langen Regale auf, und abwanderte. Das Archiv war nach einer gewissen Logik sortiert, die sich ihm noch nicht zur Gänze erschloss. „Wo sind denn die verflammten Zwergenartefakte?!“

„Den Gang herunter, zweites Regal links“, ertönte eine heisere Stimme. Kjall schreckte auf. An einem der Schreibpulte zu seiner Linken saß eine uralte Frau, in ein schneeweißes Gewand gekleidet, welche mit einem langen Finger zitternd den Gang entlangzeigte. Kjall starrte die eigenartige Frau an. Sie war unverwechselbar eine Bewahrerin, und eine altehrwürdige dazu. Warum half sie ihm? Ihr Gesicht blieb abgewandt, als sie sanft sprach: „Nun zieht schon von hinnen, Störer des Friedens. Ihr werdet nicht zufrieden sein mit dem, was ihr vorfindet. Die Schriften sind nicht mehr hier.“

„Nicht mehr hier?“, fragte Kjall, „Wo sind sie dann?“

„Wenn wir das wüssten, mein Kind, wenn wir das wüssten. Sie wurden uns gestohlen.“

Dann drehte die Frau ihm endlich ihr Gesicht entgegen. Ihr beinahe zahnloser Mund verzog sich zu einem Lächeln und blinde Augen starrten ins Nichts, als sie prophetisch meinte: „Du wirst hier keinen Frieden bringen, Kjall, Sohn des Xoll, und du wirst keinen Frieden finden. Dein Ende wird langsam kommen und schmerzvoll sein, wie du es verdient haben wirst. Und doch... wenn du dich je dazu entschließen solltest, deine Geschichte zu Pergament zu bringen, so würden wir uns geehrt führen, sie hier zu bewahren.“

Kjall starrte die alte Frau entgeistert an und wich verängstigt zurück. Erst ein weiteres lautes Klopfen vom Eingangstor zu den Schwarzen Archiven holte ihn in die Gegenwart zurück. Jetzt hörte er sogar einen Schlüssel im Schloss knirschen.

„Alte Hexe“, fauchte er der seltsamen Frau entgegen, und rannte dann in Richtung des Ganges, auf den sie gezeigt hatte. Zweites Regal links, zweites Regal links... da war es!

„Möge Irlok dich holen!“, stieß Kjall hervor, als er endlich die Ablage fand, die in großen andorischen Buchstaben mit „Zwergenartefakte“ betitelt war. Hier hatte er gehofft, weitere Informationen zur Sphäre zu finden. Aber im Gegensatz zu sämtlichen sonstigen Ablagen in den überladenen Regalen dieses Archivs war diese Stelle vollkommen leer. Keine einzige Schriftrolle zu den Zwergenartefakten befand sich in diesem Regal. Die alte Bewahrerin hatte nicht gelogen. Aber das war jetzt nicht die Hauptsache. Wichtiger war, diesem verrückten Mob aus aufgebrachten Bewahrern zu entkommen.

Kjall packte seine Sphäre aus und legte sie vor sich auf den modrigen Archivboden. Wie weit sollte er in die Vergangenheit reisen? Eine Woche, einen Mond, gar ein ganzes Jahr? Falls er nicht herausfände, wie er in die Zukunft reisen könnte, würde ihn das alles nur tiefer in die Vergangenheit schicken. Und tiefer in der Vergangenheit... die Zeiten, in denen Cavern noch glorreich und voller Tradition gewesen war... ach, es wäre schon zu schön, diese Zeiten zu erleben. Aber es war wichtiger, sich um die Zukunft des Zwergenreichs zu kümmern, und das konnte er tief in der Vergangenheit nicht. Rund ein Mond zurück musste reichen. Das würde ihm nebenbei, falls er hier scheitern und nicht in die Zukunft reisen sollte, genug Zeit verschaffen, in Ruhe einen neuen Plan auszuhecken, ehe sein Doppelgänger aus der Zukunft mit Thogger, Shan, dem Bleichen und den Helden im Gepäck hier erscheinen würde.

Das Kontrollzentrum der Sphäre war rasch bedient und das Zeitportal entstand in Kürze. Kjall blickte hindurch und sah durch die blau schimmernde Scheibe – wenig überraschend – denselben Regalabschnitt, den er auch so hinter dem Portal sehen konnte. Personen waren keine zu sehen. In diesem Moment ertönte ein lautes Krachen von der Tür zu den Schwarzen Archiven und Follas Stimme: „Gut gemacht, Kallun!“

Die Stimme der alten Bewahrerin ertönte: „Wirklich toll gemacht, Kallun! Der Kerl ist doch schon so gut wie weg und neue Türen wachsen auch nicht auf Bäumen. Mhare, mein Kindchen, sei bitte so gut und begleite deine Großmutter zurück auf ihr Zimmer. Das war mehr als genug Aufregung für heute. Und ohnehin, Folla, dein Vorgehen...“

Kjall vernahm nicht mehr, was die alte Bewahrerin an Follas Vorgehen auszusetzen hatte, denn er hatte sich bereits die Sphäre gegriffen und war damit ins blaue Zeitportal gesprungen.\bigskip



Der Geruch nach altem Holz stieg in Kjalls Nase. Kjalls Nase, welche auf dem Archivboden flachgedrückt wurde. Der Archivboden des Schwarzen Archivs, auf dem Kjall ziemlich verrenkt ruhte und jeden Moment entdeckt werden konnte!

Rasch richtete Kjall sich auf und stieß prompt mit dem Kopf an eine Tischkante. Er unterdrückte einige Flüche und sah sich um. Aus einem vergitterten Fenster brach Sonnenlicht ins Archiv und tauchte den Raum in ein warmes, goldenes Licht. Staubfetzen tanzten durch die Luft und eine Fee zog sich laut meckernd vom Fensterrahmen zurück, Kjall habe mit seinem unordentlichen Magiedingsbums ihren Schönheitsschlaf gestört, ein Verbrechen, für das der Rat der Bewahrer bestimmt die Höchststrafe verhängen würde.

Kjall wusste besser, als sich mit einer Fee anzulegen, also brummelte er eine leise Entschuldigung und wandte sich ab. Sein Blick fiel auf die Ablage, die mit „Zwergenartefakte“ angeschrieben war. In der Zeitlinie, aus der er soeben gekommen war, war sie vollkommen leer gewesen. In dieser Zeitlinie, einen Mond früher, war sie allerdings nur so überfüllt mit Manuskripten, Schriftrollen, Pergamenten aller Art. Kjall hatte seine Tiefminen-Golems bereits zu einem höchst riskanten Einbruch in die Schwarzen Archive gesandt, aber diese Dokumente hier hatte er noch nicht studieren können. Diese Gelegenheit konnte er sich nicht entgehen lassen. Auf der Suche nach einem Gefäß drehte sich Kjall im Kreis. Eine alte Holzkiste lag da unter einem Pult, von Spornenweben übersät. Ihr Deckel fehlte und außer Staub schien sie nichts zu beinhalten. Das musste genügen.

Kjall wuchtete die Holzkiste unter dem Lesepult hervor und stand auf die Zehenspitzen, um die „Zwergenartefakte“-Ablage zu erreichen. Ein Glück, dass sie so tief unten stand! In Windeseile zerrte Kram wahllos Schriften und Berichte aus der Ablage in die verstaubte Holzkiste. Egal, was diese Berichte beinhielten, er konnte nur davon profitieren.

„Jetzt stört er nicht nur meine Ruhe, sondern stiehlt auch noch seltene Schriften. Ts, ts, was der Oberste Bewahrer nur davon halten würde?“, ertönte nun die neckische Stimme der nervigen Fee, die sich vom Fenstersims erhoben hatte und Kjalls Kopf umschwirrte. Sie fuchtelte mit ihren kleinen Händchen herum und Funken sprühten drohend aus ihren Fingern.

„Ich klau‘ doch nichts, ich bin nur hier, um zu studieren!“, erwiderte Kjall unwirsch, wedelte die glühenden Teilchen beiseite und kletterte demonstrativ auf den Stuhl vor einem Lesepult. Der Silberschild auf seinem Rücken drückte unangenehm gegen die Stuhllehne, aber das war jetzt nicht wichtig. Einer der Berichte, die Kjall von der Ablage in die Holzkiste geschaufelt hatte, hatte sein Interesse ganz besonders erweckt. Dieses grau-silberne Papier stammte eindeutig aus der Fürstenkammer Caverns... und er hatte bislang nur einen einzigen Bericht gelesen, welcher auf diesem Material geschrieben war.

Rasch kippte Kjall die restlichen Schriftrollen der „Zwergenartefakte“-Auslage in seine Holzkiste und wandte sich mit zitternden Fingern dem Bericht auf dem grau-silbernen Papier zu. Er traute seinen Augen kaum, als er eine weitere Skizze eines ihm so wohlbekannten Zwergenartefakts erkannte. Kein Zweifel, er war auf Gold gestoßen! Der zweite Teil eines Berichts zur geheimnisvollen Sphäre!

Wie schon der erste Teil des Berichts, den Kjall aber und abermals konsumierte hatte, war auch dieser hier in schlichten Lettern, kurz und bündig geschrieben. Wer weiß, ob hier vielleicht mehr Informationen zum Erschaffer der Sphäre zu finden waren. Im Moment musste Kjall nur eines: Herausfinden, ob und wie er zurück in die Zukunft reisen konnte.

„Meint der Zwerg etwa, er könne mich ignorieren?!“, erklang erneut die quengelnde Stimme der Fee, die sich auf Kjalls Lesepult niedergelassen hatte. Noch mehr Funken tanzten um die Fee herum und drohten, die nahegelegenen Schriftrollen zu entzünden.

Die Fee fuhr fort: „Ich weiß, ich weiß, ich soll‘ ja um Mutters Willen nicht die edlen Herrschaften stören, die sich hier um die Bewahrung der Geschichten und Legenden des Landes kümmern, das wurde mir oft genug eingebläut. Aber weiß er, ich hab‘ da so den gewissen Verdacht, dass er gar nicht zu den edlen Herrschaften gehört, die diesen Raum betreten sollten. Ich hab‘ einen Riecher für so etwas.“

Kram guckte die Fee enerviert an und erkannte zum ersten Mal, dass es sich um einen Feerich handelt. Der Feerich tippte sich auf die Nase und kicherte, als er Kjalls Gesichtsausdruck sah, und meinte: „Ertappt! Wie spannend, sonst geschieht hier fast nie etwas aufregendes! Was sucht er hier? Will er geheimes Wissen klauen? Wie gedenkt er, von hier zu entkommen? Soll ich die Wachen alarmieren?“

Kurz überlegte Kjall sich, einfach einen weiteren Mond in die Vergangenheit zu reisen, um diese Fee hinter sich zu lassen. Aber er wusste nicht, wie viel Treibstoff noch in der Sphäre lagerte, und er würde diesen sicher nicht wegen eines vorlauten Feerichs verbrauchen. Er musste nur diesen Bericht überfliegen... da! Da stand eindeutig etwas von einer Reise in ‚Zukünftiges‘. Und dafür musste man... oh, natürlich! Die Querverstrebung links und der dritte Schalter im Kontrollzentrum. Nun, jetzt war es offensichtlich! Warum war Kjall nicht von selbst darauf gekommen?

„Meint er etwa, er könne mich einfach ignorieren? Hallihallo, mich gibt es auch noch!“

Der Feerich flatterte vor Kjalls Gesicht und schnippte mit den Fingern. Der Bericht zur Sphäre, den Kjall ehrfürchtig in seinen Händen hielt, verging in einem Funkenregen. Kjall zog seine Hände abrupt zurück und betrachtete fassungslos die verkohlten Papierfetzen, die vor wenigen Augenblicken noch ein Bericht zur Sphäre gewesen waren. Dann wandte er seinen hasserfüllten Blick dem Feerich zu. Der Bericht zu einem der mächtigsten Artefakte aus aller Zeit, vernichtet durch die Willkür einer Fee! Funken und Feuer, was für ein nutzloses Wesen! Kjall sah rot.

„Ups!“, grinste der Feerich, „Habe ich nun seine Aufmerksamkeit? Die Wachen werden in wenigen Augenblicken hier oben sein. Ich glaube, damit ist seinen Schandtaten ein Ende...“

Kjall griff nach seinem Casamatuc und versetzte dem kleinen Feerich einen mächtigen Hieb. Der Feerich reagierte nicht rechtzeitig und wurde aus der Luft an das nebenstehende Regal gefegt, wo er gebrochen zu Boden sank. Kjall musste ihn nicht genauer ansehen, um zu wissen, dass er das nicht überlebt haben konnte.

‚Mörder‘, klang Follas Stimme aus seiner Erinnerung nach, ‚Du hast Fanatos auf dem Gewissen!‘

Schwer atmend stand Kjall von seinem Pult auf. Hatte Folla recht behalten? War er ein Mörder? Aber... aber es war doch nur eine elende Fee! Kjall hatte noch nie einem Zwerg oder selbst einem verflixten Menschen etwas angetan und vermochte dies wahrscheinlich auch gar nicht! Klar, Brandur hätte er mit Freuden entführt, aber das wäre für das größere Wohl Caverns gewesen. Und dieser Feerich, dieser Nervtöter, hatte nicht einmal ansatzweise eine Ahnung davon, welches Schriftstück er soeben vernichtet hatte. Nicht... nicht dass er deswegen verdient hätte, zu sterben.

Bei Kreatoks versengten Augenbrauen, Kjall wusste ja jetzt, wo sich der Bericht befand! Er würde einfach hierhin zurückreisen und den restlichen Bericht lesen können. Es war nicht von Bedeutung, weder die Vernichtung des Berichts noch den Tod dieses Störenfrieds hier. Es war einfach nur ärgerlich. Aber gut, nun musste Kjall definitiv von hier verschwinden. Zum Glück wusste er jetzt auch, wie. Die Querverstrebung links und der dritte Schalter im Kontrollzentrum. Ganz einfach.

Als wäre das ihr Stichwort gewesen, hörte Kjall, wie sich die Tür zu den Schwarzen Archiven mit einem lauten Knarren öffnete. Eine helle Stimme, die er als Follas erkannte, rief: „Fanatos, ist alles in Ordnung? Warum hast du uns gerufen?“

Die Besorgnis in Follas Stimme wurde zu Gereiztheit, als die Bewahrerin nachsetzte: „Ich schwöre, wenn das wieder einer deiner Scherze ist...“

Kjall brachte die Sphäre aus seiner Tasche zum Vorschein und verschob hastig Streben, Schalter und Hebel. Keine Zeit, den genauen Zeitpunkt zu bestimmen, einfach in die Zukunft zurück wollte er. Lieber etwas zu weit, wenn er bedachte, dass er sich soeben mit dem gesamten Bewahrerorden zu verfeinden schien. Kjall platzierte die Sphäre in der Holzkiste mit den gestohlenen Schriftrollen und drehte sie leicht, sodass das Portal sich seitlich um die Holzkiste bilden würde. Diese Kiste würde Kjall sofern möglich gerne in die Zukunft mitnehmen.

Die Sphäre begann zu sirren und blauer Dampf stob aus einer Öffnung zu ihrer Seite, während die drohenden Schritte Follas immer näher kamen. Zu wenig Zeit, zu wenig Zeit! Kurzerhand wandte sich Kjall vom sich bildenden Zeitportal ab und trat selbstbewusst zwischen den Regalen hervor, den Silberschild immer noch auf seinen Rücken geschnallt.

„Zum Gruße, werte Bewahrerin“, rief Kjall, „Ich bin Jork von den Schildzwergen“ – er tippte auf den Schild an seinem Rücken – „und ich würde hierher gesandt, um... um... um die Stabilität dieser Räume zu begutachten.“

Folla blieb überrascht stehen, nur wenige Schritte von Kjall, dem toten Feerich und dem sich bildenden Zeitportal entfernt. Es dauerte nur einige Augenblicke, bis sie zwei gekrümmte Klingen aus ihrem Gürtel gezogen hatte und Kjall entgegenstreckte. Immerhin hatte sie das Portal und Fanatos‘ Leichnam noch nicht entdeckt.

„Wie seid Ihr hier hereingekommen, Jork? Zu diesem Raum besitzen nur wenige Auserwählte Zugang...“

„Ich muss sagen, ich bin wirklich beeindruckt von den Abstützungen, die diese Deckenkonstruktion verstärken. Wer auch immer diese Räume in den Baum gebaut hat, die wussten wirklich, was sie taten“, plapperte Kjall munter weiter, immer wieder zum sich bildenden Zeitportal schielend. Inzwischen hatte es volle Größe erreicht und seine Rotation nahm stetig zu. Nicht mehr lange...

„Was wollt Ihr hier?“, fragte Folla nun, „Der Oberste Bewahrer würde bestimmt nicht einfach... nun, fragen wir ihn doch einfach selbst. Wenn Ihr mir bitte folgen würdet? Dem Geheimnis Eures Aufenthalts hier werden wir noch auf den Grund gehen, glaubt mir! Aber das müssen wir nicht in hier in diesen geheimen, zutrittsverbotenen Räumen tun.“ Den Worten „geheim“ und „zutrittsverboten“ verlieh sie eine besondere Betonung.

„Ich befürchte, dass ‚zutrittsverboten‘ kein wirkliches Wort ist. Und ich befürchte, dass ich euch nicht folgen werde“, antwortete Kjall vorsichtig. Aus dem Augenwinkel sah er, wie die Holzkiste mit den Schriftrollen und der Sphäre darin ins schief stehende Portal kippte. Der Durchgang war offen!

„Wenn ihr euch nicht gefügig verhaltet, so sehe ich mich gezwungen...“, begann Folla, doch Kjall achtete nicht mehr auf sie. Er drehte sich auf seinen Fersen um, rannte zwischen die Regale und stürzte auf das Portal zu. Ein scharfer Schmerz entflammte in seinem rechten Arm, dann hatte er das blau schimmernde Portal erreicht und ließ sich hineinfallen. Kalt glitt dessen Oberfläche über seine Haut, während die üblichen blauen Strahlen sein Blickfeld überdeckten. Das letzte, was Kjall wahrnahm, war ein gellender Schrei Follas.

„Fanatos!“\bigskip


\az{Jahr 340}


Kjall erwachte nießend in einem riesigen Berg aus Staub und zerfledderten Pergamenten. Kein Tageslicht und keine Fackeln erhellten das Innere des Raums. Nur schwaches, goldenes Mondlicht drang durch das Fenster zu seiner linken. Blind tastete Kjall seine Umgebung ab und ergriff die Holzkiste mit zahlreichen Schriften zu verschiedenen Zwergenartefakten. Er rang nach Luft. Warum war es hier so schwer zu atmen?! Natürlich war die Lage hier kein Vergleich zum schrecklichen Limbo zwischen der Zeit, aus dem er soeben gekommen war, aber dennoch kam Kjall mit der schlechten Luft nur schlecht klar. Endlich trafen seine Finger auf kühle, glatte, verschlungene Metallstreben! Kjall ergriff die Sphäre, steckte sie in seine Tasche und hechtete zum nahe gelegenen Fenster.

Kjall musste auf den Sims klettern, um das Fenster zu ergreifen. Er riss und zog, und unendlich langsam ruckelte das Glas auf. Kühle, frische Luft strömte in die Schwarzen Archive, und Kjall sog sie gierig ein. Gute Güte, wie lange hatte niemand diese Archive betreten? Das war falsch, das sollte so nicht sein. Kjall hatte es offenbar in eine weit entfernte Zukunft verschlagen, erheblich weiter, als er geplant hatte.

Ein frischer Schnitt brannte an Kjalls rechtem Arm. Offenbar hatte Folla ihn mit einem ihrer Dolche erwischt. Aber tief war er nicht und die Blutung hatte bereits gestoppt. Alles im grünen Bereich.

Zwischen den schneebedeckten Ästen des Baums der Lieder hindurch – es musste Winter sein, was auch die verdammte Kälte erklären würde – sah Kjall in der Ferne Lichter aufblitzen. Konnten das der Sternenhimmel über den nördlichen Ausläufern des Grauen Gebirges sein? Irrte er sich, oder waren weniger Sterne am Himmelszelt zu sehen als üblicherweise? Kjall schüttelte den Kopf und wandte sich wieder vom Fenster ab.

Am Lesepult fand er eine schon fast vollständig abgebrannte Wachskerze in einem Tonhalter, welche er mit einer beiläufigen Bewegung seines Glutholzes anzündete. Das Licht der Kerze enthüllte erst jetzt den Zustand der Schwarzen Archive. Viele Schriftrollen lagen am Boden umher, dort drüber war sogar ein Regal umgestürzt, und jede Oberfläche war überzogen mit einer dicken Schicht Staub, die Kjalls Niesreiz ganz gewaltig reizte. Offenbar war schon seit Monden niemand mehr hier gewesen und hatte sich um die Ordnung gekümmert.

Kjall versteckte seine Holzkiste mit den Berichten zu verschiedenen Zwergenartefakten unter einem Lesepult und überlegte. Diese letzte Begegnung mit Folla und dem Feerich Fanatos hatte ihn etwas aus der Bahn geworfen. Warum hatte Folla ihn erkannt, ehe er in die Vergangenheit gereist war? War eine solche Interaktion über verschiedene Zeitlinien hinweg möglich? Oder war die Zeitlinie doch vorgeschrieben und Andors Entstehung unvermeidlich? Konnte man dies überhaupt überprüfen?

Kjall lauschte an der Tür, die die Schwarzen Archive mit dem Rest des Baums der Lieder verband. Keine Stimmen waren zu hören. Schliefen die Bewahrer allesamt?

Die Tür erwies sich als hartnäckigeres Hindernis als erwartet, zumal Kjall seinen Casamatuc in der Vergangenheit liegen gelassen hatte. Beim Netz der Schicksalssporne, er würde nicht in diesem Raum verrotten! Endlich, mit etwas Gewalt und einem eisernen Kerzenhalter, schaffte Kjall es, das Schloss aufzubrechen. Die Tür schwang mit einem lauten Knarren auf und Kjall duckte sich zurück ins Dunkel, für den Fall, dass das Geräusch einige Bewahrer geweckt hatte. Nichts. Nach einigen Minuten brachte Kjall genug Mut auf, die Schwarzen Archive zu verlassen. Er huschte aus dem Türrahmen und blickte zurück auf die aufgebrochene Tür. Jemand hatte in großen schwarzen Lettern „Betreten verboten per Dekret des \st{Statthalters Ken Dorr} Obersten Bewahrers“ darüber gepinselt.

Zumindest sah der restliche Teil des Baums der Lieder nicht so verlassen aus wie die Schwarzen Archive. Der Boden war sauber, die Türen waren beschildert und drei Stockwerke weiter oben sah Kjall sogar noch ein Lichtlein brennen – wohl irgendein Mitglied des Bewahrerordens, dessen Enthusiasmus für alte Schriften keine Tageszeit kannte. Ein Gefühl, von dem Kjall selbst nur allzu gut wusste, wie es sich anfühlte.

Kjall wusste ganz genau, wo er hinwollte: Im zweiten Stock, in der großen Bibliothek, gab es eine Sammlung von Aufzeichnungen zu den vier mächtigen Schilden aus uralter Zeit, in welcher Kjall bereits geschmökert hatte. Zu manchen war erheblich mehr bekannt als zu anderen, Zum Feuerschild hatten die Bewahrer aus Kjalls Zeit etwa bloß eine kleine, mickrige Erwähnung aus der Zeit der Trollkriege gefunden. Oh, der mächtige Feuerschild! Kjall hatte seine Kraft erleben dürfen, als sich ihm in der Vergangenheit die Helden von Andor und dieser ach so ehrenvolle Schwertmeister Harthalt entgegengestellt hatten. Es gab keinen Zweifel daran, dass der magische Schild, den Harthalt geführt hatte, der Feuerschild gewesen war! Wie dieser Lackaffe ihn wohl errungen hatte? Und ob er überhaupt wusste, was für einen Schatz er da trug? Wahrscheinlich nicht, die Schildzwerge hatten die Geschichte des Feuerschilds gut unter Schutz gehalten. Nun gut, es brauchte Kjall nicht zu kümmern. Ein hämisches Grinsen zog sich über sein Gesicht: Vielleicht könnte er Harthalt sogar mithilfe der Sphäre einen Besuch abstatten gehen und ihm den Feuerschild abknöpfen. Das wäre doch was: Nicht nur einen, sondern gleich zwei mächtige Schilde zu führen. Kjall wurde ganz hibbelig bei der Vorstellung. Fokus!

Fokus! Jetzt ging es zunächst einmal darum, herauszufinden, ob Kjall durch seinen Diebstahl des Silberschilds wirklich die Vergangenheit verändert hatte. Die große Bibliothek war einfach genug zu finden, und bei dieser Beschilderung fand Kjall die große Sammlung zu den vier mächtigen Schilden noch viel einfacher. Wo war denn die Chronik? Ah, da! Silberschild, Silberschild, Silberschild... nichts!? Wie sah es unter ‚Sturmschild‘ aus? Ah, hier! „lagerte für hunderte von Jahren in der Waffenkammer der Schildzwerge, ehe er von Fürst Hallwort dem Großen nach Silberhall befördert wurde.“

„Feuer und Spucke“, murmelte Kjall, „Feuer und Spucke...“

Es war, als hätte er den Schild nie gestohlen. Hatte er die Vergangenheit gar nicht verändert? Wobei, vielleicht war das hier auch nur eine andere Zeitlinie. Zum Glück konnte Kjall zumindest das überprüfen. Er raste in eine andere Sektion der großen Bibliothek und suchte die Chronik zu den Unruhen am Baum der Lieder. Zur Zeit der Trollkriege... in einem Herbst... das Jahr, das Jahr... hier war es: „Jork (?), rothaariger Zwerg, Einbruch in die Schwarzen Archive, Diebstahl zahlreicher Schriften, Mord. Verfügt über unbekannte Portaltechnologie. Zwei Unruhen in einem Abstand von rund einem Mond. Beziehung zu Thogger? Entflohen aus den Kerkern der Schildzwerge gemäß Aussage von Wächter Bort.“

Sein Besuche am Baum der Lieder waren niedergeschrieben worden. Das machte es eindeutig. Dies war dieselbe Zeitlinie, in welcher Kjall den Silberschild gestohlen hatte. Und dennoch hatte er keine Veränderung am Verlauf der Geschichte ausgelöst. Kein Paradoxon hatte ihn verschlungen. Hatte Fürst Hallwort mit seinen Behauptungen recht gehabt? Standen das Vergangene und die Zukunft in Stein gemeißelt? Das würde doch bedeuten, dass irgendjemand irgendwann den Silberschild, den er immer noch auf seinem Rücken trug, zurück in seine Zeit bringen würde. Hieß dass, das Kjall den Schild wieder verlieren würde, verlieren musste?

Kjall sank neben dem Pult zu Boden. Wenn diese Überlegungen stimmten, so konnte er tun und lassen, was er wollte, und dennoch nicht verhindern, dass Hallwort in den Norden zog. Nicht verhindern, dass Hallgard Xoll in seinen Tod schickte. Nicht verhindern, dass Kram zum Fürsten ernannt wurde. Nicht verhindern, dass Radan verbannt wurde. Eine Schande war das! Wenn die Geschichte feststand, konnte Kjall geradesogut gar nicht erst versuchen, das Schicksal zum Guten zu wenden. Aber... Moment mal, er konnte das Ende der Geschichte herausfinden. Ja, er konnte die Zukunft Caverns erfahren, hier, in diesem Augenblick! Hier in dieser Bibliothek lagerten die Aufzeichnungen aller Jahrhunderte, die zwischen Kjalls Zeit und dieser Zukunft liegen mögen, und Kjall konnte sie studieren. Wobei, wenn er darüber nachdachte, dann gab es sogar noch einen schnelleren Weg. Einen visuelleren Weg.

Von neuem Tatendrang erfüllt, sprang Kjall auf. Er verließ die große Bibliothek und hetzte die Wendeltreppe mit den viel zu hohen Stufen hoch. Er wollte ganz oben an die Spitze. Dort, am höchsten Punkt des Baums der Lieder, befand sich eine kleine Aussichtsplattform, von der aus man den gesamten Wachsamen Wald und das Rietland überblicken konnte – und natürlich die Ausläufer des Grauen Gebirges, unter denen ein Teil des Zwergenreichs Cavern lag.\bigskip



Kühle Winterluft schlug Kjall entgegen, als er das Ende des Wendeltreppe erreichte und die Tür zur Aussichtsplattform des Baums der Lieder erreichte. Er rutschte beinahe auf der dünnen Schneeschicht aus, die sich hier angelagert hatte, konnte sich gerade noch fassen und rannte dann nach vorne zur Balustrade, wo er sich auf die Zehenspitzen stellte, um knapp darüberzusehen. Der Silberschild fiel klappernd zu Boden. Kjall riss ihn an sich und bemerkte kaum, dass der kalte Wind um ihn herum sich legte und Wärme in seine Glieder strömte. Er war zu beeindruckt von der Aussicht, die sich ihm zeigte. Beeindruckt – und erschrocken.\bigskip



Die Nacht lag über Andor, nur der rote Mond und einige Sterne standen am Himmel. Dennoch erglühte das Rietland in einer Lichterpracht. In der Ferne leuchteten die Fenster der Rietburg, dieses verhassten Gebäudes, als säße im Innern der Häuser und Türme immer noch der Tag gefangen. Seltsame, golden glitzernde Seile führten von den Türmen und Zinnen der Burg zu Masten und von da ins weitere Rietland... aber wo war denn das Rietland? Wo früher nichts als trockenes Gras im Wind geweht hatte, standen nun Häuser dicht an Häusern, von der Rietburg bis hin zum Südlichen Wald. Der südliche Wald hatte sogar etwas an Größe eingebüßt, um Platz für weitere Siedlungen zu lassen. Das waren keine einfachen Bauernkaten, jedes einzelne dieser Häuser ragte mehrstöckig in den Himmeln und hatte Anbauten, die die Taverne zum Trunkenen Troll vor Neid erblassen lassen würden. Noch dazu schien bei all diesen Häusern Licht aus den Fenstern hinaus, als hielte man Gevatter Tag höchstpersönlich gefangen! Es war absurd. Wo einst der freie Markt gelegen hatte, stand nun ein breiter Turm – zwergische Machart, das erkannte Kjall sofort. Immerhin schien die Taverne zum Trunkenen Troll, dieser winzige Punkt am Rande des Lichtermeeres, noch so auszusehen, wie Kjall sie kannte – aber auch zu diesem Gebäude führten eigenartige hängende Seile, an denen hin und wieder ein bläulicher Blitz entlangraste, und auch aus den Fenstern der Taverne glühte goldenes Tageslicht, wie es selbst die Fackeln von Cavern nicht zustande brachten.

Kjall wurde Angst und Bange, als ihm bewusst wurde, wie viele Menschen in einem einzelnen Haus leben konnten. Diese Unmengen an Behausungen im Rietland waren bestimmt noch nichts im Vergleich mit der schieren Anzahl an Menschen, die sich dort versammelt haben mussten und taten, was Menschen eben so tun.

Mit Schaudern erkannte Kjall, dass selbst das Fahle Gebirge, dessen Gipfel immer in den Wolken zu stecken schienen, von diesen seltsamen, golden leuchtenden Seilen überspannt war... und waren das etwa Häuser? Häuser auf dem Fahlen Gebirge?! Kjall rieb sich die Augen. Jawohl, da schienen tatsächlich noch weitere Hütten zu stehen. Der warme Schein aus ihren Fenstern beleuchtete den dunklen Smog, der aus ihren Schornsteinen dampfte und den Hang des Fahlen Gebirges entlang zog um sich in einer Schwarzen Wolke am Gipfel zu sammeln, die sich mit dem allgegenwärtigen Weiß des Schnees biss.

Das Gebiet südlich der Rietstadt und des Südlichen Waldes lag im Dunkeln der Nacht. Immerhin hatten die Menschen nicht auch dort noch Häuser hingepflanzt. Vermutlich mussten sie Felder anlegen. Kjall wurde schwindelig, als er sich überlegte, was dieser Moloch einer Stadt für Nahrungsbedürfnisse haben musste. Sein Blick schweifte zur Narne, auf der er... waren das Boote? Das konnten doch unmöglich Boote sein, die waren ja viel zu groß! Und was war dieser golden leuchtende Dampf, den sie ausstießen? Konnte es sein, dass sie von derselben Technologie angetrieben wurde, die diese goldenen Leitungen im Rietland verteilten?

Kjalls Blick folgte dem Lauf der Narne in Richtung des Hadrischen Meeres, und erstarrte erneut. Die Nebelinseln, so wenig er von ihnen auf diese Distanz auch erkennen konnte, waren ebenfalls von farbigem Licht erfüllt, das den Nebel und kleine Teile des umliegenden Meeres erglühen ließen. Ein weiteres Glühen erhellte eine unförmige dunkle Erhebung im Hadrischen Meer in der Mitte zwischen Sidra und Silberland. Kjall konnte sie nicht näher einordnen. Es sah fast aus, als hätte man einen verdammten Riesenkraken versteinert.

Kjall senkte seinen Blick auf den Wachsamen Wald, und auch hier, zwischen den immergrünen Bäumen, waren hier und da golden glänzende Leitungen gespannt worden, die immer wieder kurz aufglühten und dann wieder dunkler wurden. Immerhin lag der Wald abgesehen davon im Dunkeln. Die Bewahrer und Dorfbewohner schienen zumindest die Ruhe der Nacht zu achten.

Einen letzten Ort in diesem Panorama gab es, den Kjall noch nicht betrachtet hatte, weil er sich davor fürchtete, was er sehen würde. Cavern, die Heimat der Schildzwerge. Sein Heimat. Wie hatte das Zwergenreich den Zahn der Zeit überstanden? Kjall schlich vorsichtig zum anderen Ende der Balustrade und hievte sich hoch.

Das Graue Gebirge stand schwarz vor dem Sternenhimmel, und keinerlei goldene Leitungen überzogen die Berge. Licht brach dennoch aus ihm hervor, aus dutzenden Türmen und Turmgruppen, die scheinbar zufällig über die Berge verteilt waren. Meisterwerke der zwergischen Handwerkskunst, die zu Kjalls Zeiten noch nicht existiert hatten. Selbst der Verlassene Turm strahlte aus seinen Fenstern, und die bröckelige Treppe, die vom Fuße des Gebirges zum Turm führte, wirkte blitzblank, so gut wie neu geschaffen! Der südliche Eingang nach Cavern war besonders hell erleuchtet, und nicht nur im erwarteten Gelb, nein, da schwebten grüne, blaue und rote Sphären voller Licht, die den Eingang nach Cavern beleuchteten und die Dutzenden von Zwergen beschienen, die sich dort aufhielten. Sie bewegten sich rhythmisch und schienen eine Art Zeremonie abzuhalten. Der Wind trug von Zeit zu Zeit kurze Fetzen ihres Gegröles an Kjalls Ohr. Dessen Aufmerksamkeit schweifte aber schnell wieder zurück zu den eleganten Bauten, die das Graue Gebirge überzogen. Wenn es schon von außen so aussah, wie musste es dann erst innen sein? Oder auf der anderen Seite des Grauen Gebirges, das hier waren ja sogar nur die nördlichen Ausläufer!

Es gab keinen Zweifel: Das Cavern war aufs Neue erblüht und erstarkt, trotz aller Widrigkeiten, die sich ihm in den Weg gestellt hatten. Kjall konnte kaum genug kriegen vom Anblick der leuchtenden zwergischen Handwerkskunst unter dem Sternenhimmel. Er wischte sich eine Träne aus dem Auge. Die Zukunft Caverns... sie war gut.

\newpage
\section{Gestohlen}

Kjall wusste nicht, wie lange er dagestanden und die Pracht des erstarkten Caverns bestaunt hatte, als ihn ein nicht allzu ferner Ruf eines Skrals zurück in die Realität holte. Genauer gesagt wurde ihm plötzlich bewusst, dass er nicht als einziger auf der Aussichtsplattform oben am Baum der Lieder saß und die Umgebung begutachtete. Eine junge Bewahrerin in grauer Kleidung hatte sich zu ihm gesellt und musterte ihn neugierig.

„Zum Gruße“, setzte Kjall vorsichtig an, „Ich... bin Jork.“

„Ich bin Josella“, antwortete die Bewahrerin mit einem freundlichen Lächeln, „und es freut mich, dass Ihr zu uns gefunden habt. Dieser Ort ist wahrscheinlich mein Lieblingsplatz im gesamten Wachsamen Wald. Hier kann man viel Zeit verbringen und die Umgebung begutachten. In der Nacht ist es besonders schön. All diese Lichter...“

Kjall nickte. Die Aussicht war wirklich atemberaubend bezaubernd, wenn auch für ihn wohl nicht aus denselben Gründen wie für sie.

„Wie seid Ihr an den Wachen unten am Baum der Lieder vorbeigekommen?“, fragte Josella nun. Reine Neugierde sprach aus ihrem Gesicht, keine Feindseligkeit.

Unverfrorene Lügen hatten ihm noch selten geschadet, also meinte Kjall fröhlich: „Ich habe mit den Wachen einfach geredet, natürlich. Ihnen ist mein Abliegen sehr verständlich gewesen.“

„Na gut, ich beiße“, lächelte Josella, „Wollen Sie mir Ihr Anliegen verraten? Was suchen Sie des nachts oben auf einer Aussichtsplattform, welche eigentlich einzig und allein uns Bewahrern zugelassen wäre?“

„Nun, das ist leicht. Ich wollte das glorreiche Cavern bei Nacht erblicken. Ich hatte einst ein Gemälde gesehen, welches dieses Blick zeigte, ein Gemälde, welches mich sehr faszinierte. Bis eben konnte ich nicht glauben, dass es wirklich so schön aussieht.“

Josella nickte. Stille breite sich zwischen den beiden aus, wie sie gemeinsam an der Balustrade standen und das Graue Gebirge betrachteten. Der starker Duft von Wolfskraut drang an Kjalls verfrorene Nase. Hatte Josella welches zerrieben, ehe sie hier nach oben gekommen war? Kjall musste aufpassen, dass es ihm nicht die Sinne vernebelte. Die Bewahrerin schien weiterhin freundlich gesinnt zu sein, auch wenn Kjall sich sicher war, dass sie ihm seine Geschichte nicht abkaufte.

„Darf... darf ich Euch etwas fragen?“, fragte Kjall.

Josella nickte erwartungsvoll.

„Kennt Ihr Euch gut aus mit der Geschichte des Zwergenreichs? Genauer gesagt der Zeit unter der Führung von Fürst Kram?“

Josella druckste etwas herum: „Huch, das ist lange her... lange her, dass es geschehen ist, und auch schon lange her, dass ich die Geschichte Caverns gebüffelt hatte.“

Kjall blieb still und wartete darauf, dass Josella ausführte.

„Nun, gut, wenn Ihr unbedingt meine Kenntnisse überprüfen wollt... Fürst Kram herrschte über Cavern zwischen den Jahren 72 und 76 nach andorischer Zeitrechnung, die kürzeste Herrschaftsperiode aller Fürsten, die seither hier geherrscht hatten.“

Kjall horchte auf.

„Ich glaube nicht, dass ich näher auf Fürst Krams überraschenden Tod eingehen muss, diese Anekdote sollte inzwischen jedem bekannt sein...“

„Tut so, als hätte ich noch nie davon gehört.“

Josella warf ihm einen verwirrten Blick zu und meinte dann: „Nun... nun... Fürst Radan hatte ja bereits einmal versucht, einen Aufstand gegen Fürst Kram anzuzetteln, kurz nach der Krönung Fürst Krams, und war dafür verbannt worden. Nachdem sich Radan allerdings in der Schlacht gegen Krahd auf die Seite der Zwerge entschieden hatte, wurde dessen Verbannung aufgehoben und Radan mit aller Ehre zurück nach Cavern eingeladen.“

Kjall spitzte die Ohren und grinste innerlich schon über beide selbige hinaus.

Mit gequältem Gesichtsausdruck fuhr Josella fort: „Kaum drei Monde nach der Rückkehr aus dem Lande Krahd startete Radan einen weiteren Aufstand, diesmal erfolgreich. Fürst Kram wurde im Bad überrascht und hinterrücks... nun ja... Fürst Radan wurde Herrscher über Cavern und regierte mit strenger Faust, stets die Zukunft im Blick behaltend. So expandierte er Cavern, brachte alte Zwergenfesten wieder zum Aufbau und befahl die Errichtung neuer Meisterwerke. Den Kontakt zu Andor und dem Barbarenland – und den Zwergen aus den Silberminen, die ihm die Krone absprachen – brach er größtenteils ab, weswegen wir leider nur über wenige Berichte aus dieser Zeit verfügen. Aber es lässt sich nicht leugnen, dass Cavern unter seiner Führung zu alter Stärke zurückfand. Soll ich fortfahren?“

„Ich bitte darum“, nickte Kjall eifrig. Radan als Fürst, das hätte er sich nie träumen lassen!

Josella schien sich etwas zu entspannen, als sie fortfuhr: „Radans einziger Sohn und Nachfolger, Fürst Breggo, stärkte die Bunde Caverns mit seinen Nachbarn wieder, während er stets auf dessen Unabhängigkeit achtete, und dass bei den unzähligen Händel mit der Außenwelt Cavern stets den größten Profit zulegte. Hatte man sich zur Zeit von Radans Führung noch manchmal vor kriegierischen Auseinandersetzungen zwischen Cavern und den Reichen der Menschen gefürchtet, so konnte das gesamte Land unter Breggos Führung aufatmen.“

Kjall knirschte mit den Zähnen. Er hätte nur zu gerne gesehen, wie die Heere der Menschen unter dem Ansturm Caverns zurückgedrängt wurde. Aber es konnten ja nicht alle Wünsche in Erfüllung gehen.

„Und wer herrscht jetzt über das Zwergenreich?“

Josella runzelte ihre Stirn leicht, verlor aber nicht ihr Lächeln: „Woher kommt Ihr denn, dass Ihr nicht einmal wisst, wer... verzeiht, das war unhöflich. Die regierende Herrscherin ist Meria, Fürstin unter dem Berge und Trägerin der Schildkrone, Tochter des Breggo, Sohn des Radan, Sohn des Eidin. Sie ist jung, erst um die 90 Jahre alt. Ich habe den Zeitpunkt ihrer Krönung miterlebt, da war ich noch ein Kind. Damit meine ich, dass ich von ihrer Krönung gehört habe, als sie abgehandelt war. Menschen waren zu diesem festlichen Anlass natürlich keine eingeladen.“

Kjall nickte hocherfreut. Er mochte den Verlauf der Zeit nicht so steuern können, wie er es gedacht hatte, aber er konnte sich glücklich schätzen, dass die Zukunft Caverns so rosig aussah. Nun hatte er genug gesehen von dieser seltsamen Zukunft voller überfüllter Menschenhäuser und goldener Leitungen. Kjall konnte sich nun entspannen, zurück in seine eigene Zeit reisen und... und ja, was tun?

Was wollte Kjall denn überhaupt in seiner eigenen Zeit? Klar, es wäre schön, Radans Aufstieg hautnah mitzuerleben, aber noch schöner wäre es zum Beispiel, den Urzeiten Caverns einen Besuch abzustatten, vielleicht sogar an ihnen beteiligt zu sein. Mithilfe der Sphäre stand Kjall die gesamte Zeitlinie Andors offen, und nun wusste er auch mit ziemlicher Sicherheit, dass er nichts tun würde, was der Entwicklung Caverns schaden könnte – vielleicht war er sogar an deren Entstehung beteiligt. Die Blütezeit der Drachen, als es noch keine elende Burg im Rietland und keine Krahder in Krahd, ja sogar keine Bewahrer im Wald gab! Als Drachen durch die Lüfte flogen, im Bündnis mit den Zwergen, die die Tradition noch zu achten wussten. Dorthin wollte Kjall. Und dorthin würde er auch kommen.

Aber noch nicht sofort. Kjall betrachtete den Silberschild in seiner Hand und ein alter Instinkt erwachte wieder in ihm. Ein Instinkt, den er lange Zeit unterdrückt hatte. Der Sammlertrieb! Kjalls Lebenserwartung war noch lange. Solange die Sphäre genug Saft enthielt, konnte Kjall nach Belieben durch die Zeit hopsen. Und neuen Treibstoff herzustellen war auch nicht die unüberwindbare Aufgabe, für die er sie einst gehalten hatte. Flüssiges Feuer, Tiefsand vom Geheimen See... das konnte er alles hier in der Nähe finden. Die Zeit, die Herstellung zu perfektionieren, hatte Kjall ja. Die Urzeiten Caverns würden auf ihn warten, eines Tages. Zunächst einmal galt, es seine Sammlung vervollständigen. Die vier mächtigen Schilde. Ein Hadrischer Kompass. Die magischen Waffen. Kjalls Herz schlug immer schneller, als er sich erst so richtig dessen bewusst wurde, was für Möglichkeiten er nun hatte. Die ganze Welt stand ihm offen!\bigskip



„Seid Ihr noch bei uns, Jork?“, fragte Josella vorsichtig und stupste Kjalls Schulter. Kjall schüttelte sich, sein Blick klärte sich, und ihm wurde wieder bewusst, wo er sich befand.

„Natürlich, Josella. Aber nicht mehr lange, denn ich muss mich gleich wieder auf die Socken machen.“

Josella senkte den Kopf zur Verabschiedung: „Wenn Ihr meint... seid Ihr sicher, dass Ihr euch im dunklen Wald zurechtfinden werdet? Es streifen noch immer Mitglieder dieser Trollhorde aus dem Barbarenland durch die Gegend, die unseren Jägern wie durch Zauberhand durch die Maschen zu schlüpfen schienen. Wir habe einen Experten aus Werftheim gerufen, aber auch der meinte, wir müssten einen Zauberer... und weg ist er.“

Weg war er tatsächlich, denn während Josellas Monolog hatte Kjall sich von der Brüstung entfernt und war wieder im Innern des Baumes verschwunden, wo man ihn rasch die Treppe runterklappern hörte.

Ich lauschte Kjalls leiser werdenden Schritten und trat dann aus dem Schatten hervor.

„Ich muss dir von ganzem Herzen danken, Josella. Du weißt nicht, wie sehr du gerade unseren beiden Reichen geholfen hast.“

Josella blickte mit zusammengekniffenen Augen auf mich herunter. Jetzt lächelte sie nicht mehr: „Das hat sich falsch angefühlt. Und das war es auch. Wir Bewahrer vom Baum der Lieder sind der Tradierung der Wahrheit verpflichtet, nichts als der Wahrheit. Was bringt es, ihn jetzt zu belügen? Er wird es doch ohnehin herausfinden.“

Ich schüttelte meinen Kopf und sprach durch den Raureif in meinem Bart: „Du hast ihm das gegeben, was er wollte. Du hast ihm einen letzten Grund gegeben, die Zukunft Zukunft sein zu lassen und sich anderen, vergangenen Dingen zuzuwenden. Dingen, die er erwiesenermaßen nicht zerstören kann. Du hast ihm und unseren Ländern zu Frieden verholfen. Er wird sich wohl kaum die Mühe machen, deine Geschichte nachzuprüfen. Und so wird er nie auf den Gedanken kommen, sie doch noch zu den Ungunsten von Krams Blutlinie modifizieren zu wollen.“

Josella schien noch nicht überzeugt, bohrte aber auch nicht weiter nach. Stattdessen fragte sie: „Er trug einen Eurer mächtigen Schilde. Warum habt Ihr ihn ihm nicht einfach abgenommen?“

Ich grinste: „Um dasselbe gleich noch drei weitere Male tun zu müssen? Ich warte lieber, bis er alle vier gesammelt hat und greife sie mir dann.“

„Aber... aber wenn Ihr ihn heute geschnappt hättet, dann hätte er ja gar nicht alle vier...“

„Manchmal ist es besser, nicht zu sehr darüber nachzudenken. Ich weiß, dass er eines Tages alle vier finden und vereinen wird. Das ist das, was zählt.“

Josella schüttelte wieder ihren Kopf: „Wisst Ihr überhaupt, worauf Ihr Euch da einlässt?! Das sind die vier mächtigen Schilde aus der Urzeit! Wie könnt selbst Ihr auch nur davon träumen, gegen einen Träger der Vier anzukommen?“

Ich grinste: „Dieser Kjall ist ein kleiner Bengel, der keine Ahnung hat, mit welchen Mächten er da spielt.“

Ich wuchtete den schweren Sack von meinem Rücken, wo er mit einem Scheppern auf den Holzboden der Aussichtsplattform fiel. Ich öffnete den Sack und ließ Josella einen Blick hineinwerfen.

Meine Grinsen wurde breiter: „Davon abgesehen besitze ich die Vier doch auch.“

Josella machte große Augen und war dann still.\bigskip




\az{Jahr 72}

Ich war zu früh. Typisch.

Die Reise in und durch die Tiefminen war überraschend ereignislos verlaufen, auch wenn mich nur der geschickte Einsatz des Sternenschilds davon abgehalten hatte, gleich bei meinem Eintritt in die Tiefminen durch einen abrupten Feuerstoß gebrutzelt zu werden. Ich hätte vielleicht doch lieber den Zugang durch die alte Zwergentür am Baum der Lieder nehmen sollen, aber da ich es so sehr mochte, durch diese altbekannten Stollen zu spazieren und den Stein unter meinen Fingerspitzen zu spüren, wagte ich den Umweg über die östlichen Tiefminen.

Still und ruhig lag es da, das zerfallene Museum Kjalls. Eine große Zwergenaxt lag vor dem Eingang, aber sie war das einzige Artefakt, welches hier noch vollständig war. Das Innere des Museums war von einer großen Staubschicht überzogen, Steinbrocken lagen kreuz und quer über den Raum verteilt, durchmischt mit verkohlten Holzsplittern und verbogenen Metallplatten. Ein trauriger Anblick. Was auch immer hier gewütet hatte, es war eine äußerst destruktive Macht gewesen.

Die Kalibration für einen Zeitsprung von über einem Jahrhundert war äußerst komplex, und ich wollte lieber zu früh als zu spät auftauchen. So kam es, dass ich noch einen halben Tag im Jahr 72 zu warten hatte, ehe Kjall zur Tür hineinspazierte. Stunden, die ich unter anderem damit verbrachte, Kjalls Museum in seiner Blütezeit zu bestaunen.

Kjall hatte bereits vor dem Erringen der Sphäre eine beeindruckende Sammlung uralter Zwergenartefakte erreicht, aber nun war die schiere Anzahl von Ausstellungsstücken einfach absurd. Eine andorische Flöte, die Brandur selbst gespielt hatte. Ein hadrischer Kompass, der den Seekrieger Ruuf damals zur mythischen Insel Danwar geleitet hatte. Boords Hammer, mit dem er die große Statue in der Halle der vier Schilde aus dem Stein gehauen hatte. Selbst einer von Nehals Fangzähnen! Alle Artefakte waren penibel beschriftet, von Staub befreit, auf verschiedensten Sockeln zur Schau gestellt. Ich kam kaum aus dem Staunen heraus.

Das Hauptaugenmerk des geheimen Museums waren natürlich die Sockel der vier mächtigen Schilde aus uralter Zeit. Drei hatte Kjall bereits erbeutet, und sie standen auf Hochglanz poliert nebeneinander: Der Sternenschild, der Bruderschild, und der Sturmschild. Mein Herz pochte schneller, als ich sanft über ihre glatten Oberflächen strich und sie unter meiner Berührung leise aufsummten, als würden sie mich erkennen. Meine Meisterwerke. Nur der Feuerschild fehlte noch, stattdessen war auf seinem Sockel eine Skizze des Schwertmeisters Harthalt abgebildet, wie er den markanten dreieckigen Schild führte.

Der Sockel des Feuerschilds war nicht der einzige, der mit einer Skizze statt eines Artefakts belegt war. Wahrscheinlich erfuhr Kjall schneller von neuen Artefakten, als er die bekannten einsacken konnte, sodass es ihm gar nie möglich sein würde, eine vollständige Sammlung zu besitzen. Orweyns Hammer der Stärke und Varatans Helm der Macht hatte Kjall noch nicht errungen, ebenso fehlte ihm ein Amulett des roten Mondes und ein Hadrischer Spiegel. An ein vollständiges Hadrisches Stundenglas war Kjall offenbar auch noch nicht gekommen, der entsprechende Sockel war nur mit einigen Scherben, etwas Sand und einem halben Holzgestell belegt. Dafür bemerkte ich mit Staunen, dass Kjall eine der drei magischen Waffen aus Hadria errungen hatte! Varlion, das Flammenschwert, steckte in seiner Scheide in der hinteren linken Ecke des Museums. Bei den Hörnern des Urtrolls, wie war Kjall daran gekommen? Hatte er es der Zauberin Eara abgeknöpft, nachdem diese damit durch das Schwarze Portal nach Andor gereist war? Ich überlegte mir kurz, das Schwert einzustecken, entschied mich dann aber dagegen. Ich war wegen meiner Schilde hier, da musste ich mir nicht noch mehr Aufgaben aufbürden.\bigskip



Als Kjall endlich ins Museum trat, hatte ich es mir in einer Ecke des Museums gemütlich gemacht und mir die Kapuze des Hral übergeworfen, auf dass ich beinahe eins mit dem Schatten der Ecke wurde. Kjalls Kleidung war angekokelt und er murmelte fluchend etwas von einem elenden Sektenanführer, der diesen Dunklen Tempel ohnehin nicht verdient hatte. Dabei verrieten Kjalls blitzende Augen seine eigentlich prächtige Laune. Er übersah mich vollständig, ganz wie ich es erwartet hatte, und trat mit einem breiten Grinsen vor die Sockel der vier Schilde. In seiner linken Hand sah ich seine Sphäre aufblitzen, bevor er sie in seine Tasche wandern ließ. In seiner rechten Hand hielt er den Feuerschild, welchen er triumphierend auf seinem Sockel platzierte. Dann trat er zurück und stand einfach still da.

Ich hielt mich ebenfalls still und wartete auf eine Reaktion.

Kjall setzte sich andächtig auf den Boden und betrachtete seine vier Schätze. Ich hätte schwören könne, aus seinem Augenwinkel eine Träne laufen zu sehen. Dann war der Bann gebrochen und Kjall lachte auf, klatschte in seine Hände und sprang, nein, tanzte förmlich um die Sockel der vier Schilde herum. Ich ließ ihn einige Minuten des Triumphs genießen. Dann aber hielt ich es für angebracht, die Kapuze des Hral abzustreifen und in Kjalls Applaus einzustimmen.

Wie erwartet trat Kjall erschrocken zurück und stolperte beinahe über den Sockel mit dem frisch errungenen Feuerschild. Er fasste sich, richtete sich stolz auf und rief: „Wer... wie...“

Kjall blickte die vier Schilde an und schien zu verstehen: „Kreatok?!“

Ich nickte.

Kjall griff sich an seine gefurchte Stirn und meinte: „Drachenfeuer und Zwergenspucke, wie hast du mich gefunden?“

Ich hielt einige Fetzen uralten Pergaments in die Höhe, die mit einer sehr kleinen Handschrift überzogen waren: „Eine mysteriöse Nachricht aus der Vergangenheit. So relevant ist das nicht. Wichtiger ist, was du hier für eine wundervolle Sammlung angelegt hast. Ich habe mich etwas umgesehen, und ich bin echt beeindruckt!“

Kjalls Schultern sackten herunter und ein erleichterter Ausdruck flog über seine Züge: „Meister Kreatok, es ist mir eine unglaubliche Ehre... es gibt so viel, was ich dich fragen könnte. Wie... wie funktionieren diese Schilde? Könnte man weitere produzieren? Was hat dich dazu geritten, den Unterirdischen Krieg anzuzetteln? Was für Abenteuer hast du mit der Sphäre erlebt? Könnten wir die Vergangenheit irgendwie ändern?“

Ich gebot Kjall, seinen Redeschwall einzustellen. Mein Vater hatte mich gelehrt, schlechte Neuigkeiten nicht unnötig lange hinauszuzögen, also fuhr ich fort: „So sehr mir dieses Museum auch gefällt, so muss dir in der Zwischenzeit bestimmt bewusst geworden sein, dass diese Sammlung hier nicht auf ewig Bestand haben können wird. Es ist unklug, derart mächtige Gegenstände so nahe beieinander zu halten, und erst recht unklug, wenn eine solche Sphäre dabei ist. Falls jemand darauf stoßen sollte, der sich nicht so sehr aufs verborgene Vorgehen versteht wie du, Kjall, so wären die Folgen undenkbar. Jemand sollte diese Schilde und die Sphäre wieder an ihre rechtmäßigen Orte zurückbringen. Und dieser jemand bin ich.“

Mit diesen Worten trat ich vollständig aus dem Schatten und Kjall sog scharf Luft ein, als er meine Ausrüstung sah. Ich trug meinen Bruderschild in meiner linken und meinen Sturmschild in meiner rechten Hand. Meinen Sternenschild hatte ich mir auf den Rücken geschnürt, wo er strahlend hell zu leuchten begann und das geheime Museum Kjalls in einen türkisen Schein tauchte. Meinen Feuerschild hatte ich mir auf die Brust gebunden. Er war der einzige Schild, von dem ich mir geschworen hatte, ihn heute nicht einzusetzen.

Eine dunkle Macht lag über diesem Schild, eine dunkle Macht, die den Unterirdischen Krieg und den Zerfall des Zwergenreichs ausgelöst hatte. In meiner eigenen Zeit hatte ich diesen Schild noch gar nicht geschmiedet, aber ich wusste, dass ich ihn eines Tages schmieden würde, und ich wusste, dass ich danach den silbernen Feuerschlund entfesseln würde. Und Nehal würde brennen. Mein Nehal. Dafür würde es eine Erklärung geben müssen, da war ich mir sicher. Aber ich kannte sie noch nicht. Würden mir die ganzen Zeitreisen den Verstand verdreht haben? Würde mich ein Fluch befallen haben? Ich war ratlos, und ich fürchtete mich davor, die Wahrheit zu erfahren. Aber im Moment musste ich mich nicht darum kümmern. Im Moment zählte nur, dass ich nicht aus Versehen die finstere Macht des Schildes entfesselte.

„Ich hoffe, dass wir diese Angelegenheit friedlich klären können“, sprach ich weiter, „Wenn du mir die Sphäre und die Schilde übergibst, bringe ich sie weg von hier und kehre dann zurück, um mich in aller Ausführlichkeit mit dir über die Wirkungsweise der Schilde zu unterhalten.“

Ein wirrer Blick trat in Kjalls Augen: „Kreatok, deine Zeit ist vorüber. Du hast nie das volle Potential deiner Sphäre erkannt, oder sie nicht ausgiebig genug genutzt. Diese unermessliche Macht... ich könnte zum Todeszeitpunkt meines Vaters reisen und ihn in die Zukunft holen. Ich könnte ein Bündel Sternkraut aus dem Grauen Gebirge holen und es Thogger vor die Füße werfen, ehe er überhaupt in den Süden aufbräche. Ich könnte tun und lassen, was ich will, und ich tue es dennoch nicht!“

„...weil du weißt, dass du damit nie eine Veränderung erzielen würdest...“

„Nein, weil ich mich beherrschen kann! Mit dieser Sphäre bin ich ein Gott, und doch halte ich mich immer noch im Verborgenen, weil mir dieser große Rummel nichts bedeutet! Dieses Museum schadet niemandem, und niemand außer mir wird es je zu sehen bekommen. So lasse mir doch diese kleine Freude!“

„Ich frage nur nach der Sphäre und den Schilden, den restlichen Kram mögest du behalten.“

„Unterschätze mich nicht, Kreatok!“, fauchte Kjall nun, „Ich besitze dieselbe Macht, dieselben Artefakte wie du, aber im Gegensatz zu dir habe ich sie sinnvoll eingesetzt. Ich habe hart gearbeitet für diese Sammlung...“

Kjall schüttelte enttäuscht seinen Kopf.

„Ich hätte gedacht, dass wenigstens du erkennen würdest, wie großartig...“

„Das tue ich auch, Kjall, wirklich, aber dennoch muss ich darauf bestehen, dass du mir die Sphäre und die Schilde aushändigst.“

„Man sollte wohl nie seine Helden treffen.“

Kjall tat einen mächtigen Satz nach vorne, auf den Takuri-Spiegel zu, der an die Wand gelehnt war. Wenn Kjall durch diesen Spiegel trat, würde an seiner Stelle wohl in Kürze ein verdutzter Troll in dieser Höhle stehen, oder gar ein garstiger Wardrak. Nichts, mit dem ich nicht klarkommen könnte. Dennoch wäre es äußerst ärgerlich, wenn Kjall und seine Sphäre mir durch die Lappen gehen würden. Darum verstärkte ich den Griff um den Sturmschild an meiner rechten Hand und sandte eine gezielte Druckwelle durch die Luft, welche Kjall von den Füßen holte und einen gezackten Spalt in die Spiegeloberfläche schlug. Takuri-Asche stob daraus hervor und glühte leicht auf, während ein Partikel nach dem anderen an einen Ort gesogen wurde – ganz ähnlich, wie es der glühende Tiefsand aus dem Geheimen See mit den Zeitportalen der Sphäre tat.

Wie ich so abgelenkt die Takuri-Asche betrachtete, entging mir völlig, dass Kjall zurück zum Sockel des Feuerschilds hechtete und sich seinen Feuerschild schnappte. Plötzlich wurde mir mulmig zumute. Hoffnung, redete ich mir selbst zu, das wichtigste ist es, die Hoffnung nicht zu verlieren.

„Kjall“, setzte ich an, „Du weißt nicht, was es mit diesem Schild auf sich hat...“

Kjall brüllte irgendetwas Unverständliches und hielt den Feuerschild in die Höhe, über dessen ornamentierte Oberfläche ein unheilvoller schwarzer Glanz huschte. Schwarz-silbernes Feuer brach aus den Wänden des Museums hervor und leckte an den verschiedensten Artefakten, die da standen. Die Flammen fraßen sich in Sekundenschnelle durch den Hadrischen Spiegel, die andorische Flöte, die Überreste des Hadrischen Stundenglases, dutzende Notizen und Artefakte, Holz und Stahl.

Die Hitze schlug mir entgegen und versengte mir den Bart. Ich drehte Kjall den Rücken zu, in der Hoffnung, dass mein eigener Feuerschild das Feuer ein wenig absorbieren würde. Wo waren die Roteisenschilde, wenn man sie brauchte?! Dann griff ich erneut nach der Macht des Sturmschilds und fühlte den vertrauten Druck in meinen Ohren, als ich die Luft auseinanderzog und eine Barriere zwischen dem flammenden Inferno im Museum und der Luftblase, die mich umgab, schuf.

Kjall stand inmitten des wütenden Feuers und sah zu, wie es sich ins Gestein fraß, seine Artefakte verzehrte und seine Arbeit in Funken aufgehen ließ. Er stand einfach da, ein höllisches Lachen seiner Kehle entspringend, während sein Lebenswerk in Flammen aufging. Eine Gänsehaut lief mir über den Rücken. Ich musste dem ein Ende setzen, sofort.

Das silberne Feuer brandete immer noch an die luftleere Barriere, die meine kleine Luftblase umschloss. Ein drittes und letztes Mal griff ich nach der Macht meines Sturmschildes, spürte die Wassertropfen in der Luft, die Spannungen in der Erde und den Druck auf den Stollen. Während die Barriere um mich herum sich auflöste und das silberne Feuer auf mich zustürzte, zog ich die Luft um Kjalls Kopf auseinander und hüllte ihn in eine luftleere Blase. Kjall japste auf, aber er hatte in all seinen Reisen durch den Limbo der Zeitportale gelernt, seinen Atem auszusetzen. Er griff nach einem der wenigen noch intakten Sockeln, schnappte sich einen schwelenden Hadrischen Kompass und drehte mit seiner freien Hand an der Kompassnadel.

Ein leises Kichern entschlüpfte mir. Wie wollte Kjall mit einem kleinen Kompass, den man an jedem halbwegs besiedelten Meeresufer finden konnte, gegen die Macht des Sturmschilds ausrichten? Leider unterschätzte ich die Reserven des Schilds, welche beinahe entladen waren. Schon ertönte ein blechernes Knirschen und der helle Glanz auf der Oberfläche meines Sturmschilds wurde stumpf. Die Blase um Kjalls Kopf löste sich auf, Wind wirbelte wieder umher und frische Luft strömte in Kjalls Kehle. Purer Hass strahlte aus seinen Augen, als er erneute nach seinem Feuerschild griff und eine silberne Flammenwand aus der Decke des Raumes brach.

Varlion das Flammenschwert, dessen Scheide soeben in Asche aufgegangen war, erhob sich wie von Zauberhand in die Luft und flog Kjall geradewegs in die ausgestreckte freie Hand. Nun mischten sich neben dem silbernen Flammen auch orange Schlieren in die Luft. Kjalls Mund verzerrte sich zu einer wahnsinnigen Fratze, welche aus ihrem Schlund tiefschwarze Flammen direkt in meine Richtung spuckte. Der eiserne Rand meines Bruderschilds verformte sich leicht und der Knubbel in der Schildmitte glühte feuerrot auf, als ich den Schild zwischen die tiefschwarzen Flammen und meine Wenigkeit wuchtete und er die Hitze absorbierte. Ich wusste nicht, wie lange er noch durchhalten könnte.

In einem letzten Effort, gegen die schiere Dunkelheit anzukommen, die mir entgegengeschleudert wurde, griff ich im Geiste gleichzeitig nach dem Sternenschild und dem Bruderschild, verwob Fäden der Hoffnung und Fäden der Stärke und ließ das Geflecht auf Kjall zufliegen. Der Sternenschild flackerte kurz. Lange würden auch die Kraftreserven dieses Schildes nicht mehr halten, aber viel Zeit brauchte ich nicht mehr. Die Verbindung zu Kjall war geschaffen, und so griff ich hoffnungsvoll nach Kjalls Stärke, ließ sie durch mich fließen und vertauschte sie dann mit der Stärke der kleinen Feuermaus, welche zwei Seitengänge von uns entfernt ahnungslos durch die Stollen huschte.

Kjall klappte zusammen wie eine Puppe, der man die Fäden durchgeschnitten hatte. Der Feuerschild und Varlion das Flammenschwert fielen aus seinen schwachen Händen und das Flammeninferno versiegte. Seine Sphäre rollte aus seiner Manteltasche. Ich rannte zu Kjall herüber und trat den Feuerschild zur Seite. Mein Sternenschild flackerte ein letztes Mal auf und sein Brummen verstummte, woraufhin Kjalls Kraft zurück in seinen Körper floss, aber ich war schneller und drückte Kjall mit einem Stiefel zu Boden. Fast wie nebenbei steckte ich seine Sphäre ein.

Kjalls Blick blieb noch einige Augenblicke lang von Hass erfüllt, dann kam er wieder zur Besinnung und schlug sich die Hände vor den Kopf. Ich ließ von ihm ab und trat zurück. Kjall drehte seinen Kopf zur Seite und betrachtete die Zerstörung, die der Feuerschild angerichtet hatte. Das Innere des Museums war nun von einer dicken Ascheschicht überzogen. Steinbrocken lagen kreuz und quer über den Raum verteilt, durchmischt mit verkohlten Holzsplittern und verbogenen Metallplatten. Ein trauriger Anblick.

„Ich... ich weiß nicht, was über mich gekommen ist“, wimmerte Kjall zitternd, mehr zu sich selbst als zu mir. Kraftlos griff er nach einem Stück Kohle – den Überresten des Hadrischen Kompasses – welches unter seinen Fingern zerstob. „Nein, nein, das kann nicht sein, das darf nicht sein.“

Ich ignorierte Kjall fürs erste und begutachtete die zerstörten Artefakte. Die meisten konnte man nicht mal mehr erkennen. Varlion brannte immer noch, aber seine Flammen sahen beinahe süß aus im Vergleich zum Inferno, welches hier soeben gewütet hatte. Kjall und seine Kleidung waren unversehrt und ich war mit einigen hässlichen Rötungen weggekommen. Wir hatten beide unglaubliches Glück gehabt.

Kjall schlug mit der Faust auf den Boden und blickte mir wütend entgegen. Ich konnte es ihm nachfühlen, aber ihm nicht helfen. Nicht, dass er meinen Trost überhaupt angenommen hätte. Und so wandte ich mich mit einem mulmigen Gefühl im Magen wieder ab.

Vor mir lagen die vier mächtigen Schilde aus uralter Zeit gleich in doppelter Ausführung. Die linken vier hatte ich zusammen mit der Sphäre in meiner linken Manteltasche erst vor einigen Tagen von meinem zukünftigen Ich überreicht bekommen und in diesen Kampf mitgebracht. Die rechten vier hatte Kjall aus einer Vielzahl von Stellen in der Vergangenheit erbeutet. Nun war es wohl an der Zeit, mich auf die Socken zu machen. Ich musste die linken vier Schilde an ihre rechtmäßige Stelle in der Geschichte zurückbringen. Aber vorher sollte ich lieber noch die rechten vier Schilde und Kjalls Sphäre meinem früheren Ich überbringen gehen.

Ich ergriff meinen großen Sack und packte alle acht Schilde sorgfältig darin ein. Stolz überkam mich, als ich meine Meisterwerke ein letztes Mal betrachtete, auch wenn ich beim Anblick der Feuerschilde leicht erschauderte. Kjalls Sphäre steckte ich lieber in meine Jackentasche als in den Sack. Die rechte, damit sie ja nicht mit der Sphäre in meiner linken Manteltasche verwechselt wurde. Ich schulterte den Sack unter erheblichem Gerumpel und wandte mich zurück an Kjall:

„Ich schätze, dass es für mich an der Zeit ist zu gehen. Es ist bald Weihnachten und ich habe eine Menge Geschenke auszutragen. Lebewohl, Sohn des Xoll. Es war mir eine Freude, dich kennenzulernen, und es wäre mir eine Freude gewesen, dich noch näher kennenzulernen. Aber das Schicksal scheint andere Pläne für uns zu haben. Mein Beileid für deine Sammlung. Sie war wirklich außergewöhnlich, und das Gewissen, sie erreicht zu haben, kann dir niemand mehr wegnehmen.“

Kjall blickte mich an, nun ganz und gar nicht mehr wütend oder schockiert, nicht einmal mehr enttäuscht. Seine Augen hatten stattdessen einen flehenden Eindruck angenommen:

„Kreatok?“, fragte er stockend.

„Hm?“

„Wie steht es um den Treibstoff in der Sphäre? Wäre... wäre es dir möglich, mich auf eine letzte Reise zu schicken?“, Kjall wies schwach auf sein zerstörtes Museum, „Hier hält mich nichts mehr. Ich träumte ohnehin schon jede Nacht von den Urzeiten Caverns, den bemannten Festen, den Drachen...“

Ich hatte ihn wirklich falsch eingeschätzt. Sein ganzes Lebenswerk war in Flammen aufgegangen und es wäre nur allzu verständlich gewesen, wenn er mich als Auslöser dieses Untergangs angesehen und in die Tiefen des Enran verdammt hätte. Aber hier saß er und bat mich darum, ihm einen letzten Wunsch zu erfüllen. Er tat mir leid.

Ich entgegnete: „Bist du sicher, dass das die beste Entscheidung ist? Was ist mit deinen Freunden hier? Radan, seine Schar?“

„Was soll schon mit ihnen sein, Kreatok?“

Kjalls Blick wurde wieder wässerig, und ich fragte mich erneut, ob ich Kjall unterschätzt hatte. Konnte es sein, dass er herausgefunden hatte, dass ich ihm von Josella hatte einen Bären aufbinden lassen, was Caverns Zukunft anging? Hatte er versucht, die drei ihm damals noch fehlenden mächtigen Schilde zunächst nach der Rückkehr der Helden aus Krahd zu erringen? Hatte er erfahren, welche Schildzwerge gemeinsam mit den vier Schilden im Enran versinken würden?

„Mich hält hier nichts mehr. Bitte.“, wiederholte Kjall, diesmal gefasster. Er schnäuzte sich in den rußigen Bart.

Ich blickte ihn nachdenklich an. Ärger anrichten konnte er in dieser Zeitlinie nicht mehr, das wäre mir doch sicherlich schon aufgefallen. Und in den Urzeiten gab es weit und breit keine Sphäre, mit der Kjall zurückkommen könnte.

„Sprichst du denn überhaupt die Alte Sprache?“, fragte ich den immer noch vor mir knienden Zwerg.

Kjall antwortete ohne zu zögern: „Isklja Dugandr!“

Ich nickte beeindruckt. Ich unterschätzte ihn immer noch. Leiste murmelnd zog ich Kjalls Sphäre hervor, klappte die Hauptstreben zur Seite, legte den linken Schalter um und drehte das kleine Zahnrad im Kontrollzentrum zurück, immer und immer weiter, bis es anschlug. Dann stellte ich die Sphäre auf den Boden und vernahm mit Ehrfurcht, wie es darin klickte, ratterte und surrte, flüssiges Feuer mit Tiefsand vermischt wurde und fröhlich blubberte, bis der vertraute blaue Dampf aus seiner Öffnung gepustet wurde und sich im verkohlten Raum ausbreitete. Was für eine meisterhafte Konstruktion diese Sphäre doch war! Ich konnte es immer noch kaum fassen, dass ich es war, der sie erschaffen hatte.

Kjall blickte mich weiterhin unverändert an. Aus seinem Blick sprach Akzeptanz, und ja, selbst Dankbarkeit. Das zerstörte Museum schien vergessen. Dachte er an glorreiche Zwergenfesten und feuerspuckende Drachen, Meisterwerke der Schmiedekunst und die wackere Bezwingung des unberührten Felsgesteins? Dieses romantische Bild würde ihm noch früh genug vergehen.

Ich klopfte ihm auf die Schulter.

„Tief durchatmen, Kjall. Das wird eine lange Reise für dich.“


\begin{center}
    Weiter geht es in \hypref{Die Legende der entzweiten Zeit (2023)} und \hypref{Der Giftzwerg und das Drachenherz (2023)}.
\end{center}








\newpage
\section{Epilog: Bragor erhält den Bruderschild zurück}

\az{Jahr 62}


Das Rietgras wogte sachte im Wind. Eine massige Gestalt wanderte über das Feld. Nur mit einem Lendenschurz aus Schaffell und großen Stiefeln war sie bekleidet. In ihrer einen Hand hielt sie einen Speer. Auf dem Kopf saßen zwei mächtige Hörner. Das Sonnenlicht ließ die prallen Muskeln des Wesens glänzen. Es war ein Tarus.

Eine kleine Gestalt trat aus dem Schatten eines Baumes hervor und näherte sich dem Tarus. Sie hatte einen großen Sack über die Schulter geworfen. In der ein Hand hielt sie eine Art silbernen Kugel, in der zweiten einen runden Schild. Ein grauer Bart war in den Gürtel der Gestalt gesteckt. Es war ein Zwerg.

„Seid herzlich gegrüßt!“, rief der Zwerg fröhlich und hob dann seine Hand, die den Rundschild festhielt, „Ich bin hier, um unser Geschäft rückgängig zu machen! Der Schild, den Ihr mir verkauft hat, funktioniert gar nicht“

Der Tarus runzelte seine Stirn: „Was meint Ihr damit, dass der Schild nicht funktioniere? Und Ihr seid doch gar nicht derselbe Zwerg, dem ich diesen Schild gerade eben verkauft hatte?“

Der Zwerg lachte schallend auf: „Nicht derselbe Zwerg?! Das ist ein guter Trick, den muss ich mir merken!“

Der Tarus blickte den Zwerg weiterhin verwirrt an. Dieser klopfte dem Tarus gegen den Bauch und meinte: „Nun, ist das hier Euer Schild oder nicht?“

„Nun ja, er sieht schon so aus, wie derjenige, den ich...“

„Wunderbar, dass wäre das ja geklärt. Ich bin nicht zufrieden mit dem Schild und will ihn gerne zurückgeben!“

Der Tarus kniff die Augen zusammen: „Nein, Ihr seht wirklich nicht so aus wie der Zwerg, mit dem ich soeben einen Handel abgeschlossen habe. Und auch wenn ich nicht viel vom Handel verstehen mag... wenn Ihr mir den Schild zurückgebt, dann muss ich Euch das Goldstück wiedergeben, und das würde ich gerne behalten.“

Den Zwerg machte eine wegwerfende Handbewegung: „Dann behaltet das Gold halt. Ich will einfach nur den Schild loswerden!“

Der Tarus kratzte sich am Kopf, kniete sich dann aber brav hin und streckte seinen Arm aus, damit der Zwerg ihm den Rundschild wieder anbinden konnte. Sogar ein besonderes silbernes Seil benutzte der Zwerg. Verstehe einer diese Kontinentalbewohner und ihre Gepflogenheiten.

Als der Zwerg sich aufrichtete und bereits wieder weglaufen wollte, streckte der Tarus seine Hand aus und meinte: „Ihr... nein, dieser andere Zwerg, dem ich diesen Schild verkauft habe... mir wurde geraten, mich an den Baum der Lieder oder eine Kräuterhexe zu wenden. Ich war bereits am Baum der Lieder, und die wollten mir nur eine schwere Ziege aufbinden. Sagt, wisst Ihr, wo ich vielleicht eine Kräuterhexe finden könnte, die sich mit der Verbreitung von Sternkraut auskennt?“

Der Zwerg kratzte nachdenklich seinen bauschigen Bart, zeigte dann mit seinem Finger in den Südwesten und sagte: „An Eurer Stelle würde ich mich an die Hütte hinter dem Hügel dort drüben wenden. Ich habe gehört, dass sich dort erst kürzlich eine Hexe hat blicken lassen, und dass dort eine vorzügliche Fischsuppe mit Lauch und Wernersilie aufgetischt wird. Auch wenn Ihr die reine Wernersilie wohl lieber hättet, wenn ich Euch so ansehe. Vielleicht werdet Ihr dort Eure Auskunft erhalten – oder jemanden finden, der Euch den Schild abnimmt.“

Der Zwerge drehte und schraubte an der seltsamen silbernen Kugel in seiner Hand herum und ein großes blaues Portal bildete sich aus hellblauem Dampf. Der Zwerg blickte prüfend hinein. Dann holte er Anlauf und sprang in das Portal, welches sich zu einem kleinen blauen Funken zusammenzog und mit einem leisen Plopp verschwand.

Der Tarus blieb alleine übrig und kratzte sich an den Hörnern. Was hier abging, war eindeutig nicht alltäglich. Er drehte sich um und sah nachdenklich den Hügel an, hinter dem angeblich eine Hütte lag, in welcher gelegentlich eine Hexe einkehrte.

Noch war der Tag jung.

Zeit, dieser Hütte einen Besuch abzustatten.








\begin{chapterbox}
    \chapter{Der Steppennomade, der große See und das große Feuer (2022)}
    \label{Der Steppennomade, der große See und das große Feuer (2022)}
    \az{62 bis 63}

    \begin{center}
        Teil I der Magischen Abenteuer
    \end{center}
    
    Die Steppennomaden Barz und Nabib, zwei Mitglieder des Iquar-Stammes, kehren nach einer langen Reise in ihre Heimat zurück. Doch auf ihrer Pfahlbausiedlung Thakkum im großen See Ava wird gerade der Barbarenkönig vorstellig, um Unterstützung für die Invasion eines nahe gelegenen Königsreichs einzufordern. Denn der Krieg der Barbaren gegen die finstere Skelettarmee aus dem Süden wirkt immer aussichtsloser.
\end{chapterbox}






\section{Der Angriff der Barbaren}

\az{Jahr 62}



Sonnenaufgang, 42 Tage vor dem großen Unheil.\bigskip



Barz rupfte das Büschel Mondbeeren sorgfältig aus dem Boden und verstaute es in einem kleinen Beutel an seiner Kleidung, in dem sich bereits einige weitere getrocknete Mondbeeren neben einigen Heilkräutern aufhielten. In der Ferne konnte er den großen See immer deutlicher erkennen. Ava. Seine Heimat. Ruhig und still lag der See vor hohen Berggipfeln und glitzerte – wie die Mondbeeren – verheißungsvoll im Licht der aufgehenden Sonne.

„Guten Morgen und nichts wie raus aus den Federn! Die frühe Echse fängt den Fisch! Wenn wir uns beeilen, sind wir noch vor Sonnenhoch zuhause“, rief Barz und rüttelte Nabib beinahe sanft wach. Nabib rieb sein Gesicht und nuschelte ungehalten etwas von unterbrochenen Abenteuern im Reich der Träume. Doch auch Nabib konnte sich ein Grinsen nicht verkneifen, als er müde blinzelnd die Silhouette das Avas am Horizont erspähte. Lautstark gähnend reckte und streckte er sich und wurde dafür von Barz mit einem Kuss belohnt.

Jirisa war erheblich schwerer in die Gänge zu kriegen. Barz‘ treue, aber auch träge Steppenechse hatte sich ein wenig unterhalb der Anhöhe zum Schlafen niedergelassen und weigerte sich vehement, sich auch nur um eine Handbreit zu verschieben.

„Komm schon, Jirisa, die Zeit für deine Winterstarre ist noch nicht gekommen!“, protestierte Barz und kraulte Jirisas ledrige Haut an einer ganz bestimmten Stelle hinter einem Ohr. Die Beine der Echse zuckten wohlig und ein dumpfes Brummen drang aus ihrem Hals. Als Barz mit Kraulen aufhörte, öffnete Jirisa eines ihrer Augen und schielte empört nach ihm. Ihre lange blaue Zunge schlängelte sich an einem schief aus ihrem Unterkiefer ragenden Fangzahn vorbei und leckte nach Barz. Als dieser stoisch nicht darauf reagierte, hievte die Steppenechse widerwillig ihren massigen Körper auf ihre stämmigen Beine, woraufhin Barz sie brav weiterkraulte.

„Na also, geht doch. Wenn wir vor der Mittagshitze den Ava erreichen, besorge ich dir eine Extraladung Fische zum Abendessen. Das ganze Gras muss dir doch schon lange zum Halse raushängen.“

„Wenn du Jirisa fertig geliebkost hast, kannst du mir dann helfen, das Lager zusammenzuräumen?“, ertönte Nabibs Stimme vom Lagerplatz.

„Spüre ich da Eifersucht?“, erwiderte Barz lachend und kehrte zu seinem Freund zurück. Nach allerlei Liebkosungen und einem kurzen Frühstück aus faden, doch nahrhaften Sträuchern (mit einem rötlichen Pulver aus Barz‘ Gewürztasche schmackhaft gewürzt) kamen Barz und Nabib tatsächlich dazu, ihr Gepäck auf Jirisas Rücken zu verstauen und mit Jirisa im Schlepptau in Richtung des großen Sees Ava aufzubrechen.

Sie alle waren erschöpft von der langen Reise, die hinter ihnen lag, doch zumindest die beiden Menschen blickten zufrieden zurück auf einige Monate voller aufregender Abenteuer und fantastischer Entdeckungen. Barz hatte Proben von allerlei ihm unbekannten Substanzen gesammelt und konnte es kaum erwarten, diese gemeinsam mit seiner Schamanin zu untersuchen. Nabib hatte einen ganzen Rucksack mit Skizzen der Landschaft und ihrer Bewohner gefüllt und hatte vor, dieses Wissen in den kommenden Winterwochen an die nächste Generation weiterzugeben versuchen, ehe Barz und Nabib im Frühjahr zu ihrer nächsten Reise aufbrechen würden. Vielleicht wäre dann sogar Yafka wieder einmal mit von der Partie.

Dieses Jahr hatten Barz und Nabib es auf ihrer Reise bis an einen salzigen See im Norden geschafft, der so groß war, dass sie dessen anderes Ufer mit ihrem besten Fernrohr nicht hatten erspähen können. Wollten sie beim nächsten Mal vielleicht das südliche Gebirge zu erklimmen versuchen? Nun, sofern die Lage dies dann überhaupt noch erlauben würde. Die Gerüchte von verlustreichen Schlachten des Yetohe-Stammes gegen grausame Skelettarmeen aus dem Süden gelangten bereits seit einigen Jahren bis an den friedlichen Ava... und die Gerüchte wurden tendenziell immer düsterer.

Barz hatte keine Ahnung, wie die Lage im südlichen Gebirge im Moment aktuell aussah. Schuld daran waren wie so oft die elenden Krarks. Barz ließ seinen Blick über die weite Steppe schweifen. Es kreisten gleich zwei dieser riesigen Raubvögel majestätisch im Aufwind über der Ebene. Nicht sonderlich verwunderlich, aber doch ärgerlich. Krarks waren schon seit jeher eine Plage für die Bewohner dieses Landes gewesen, doch in diesem Jahr war ihre Population wie aus dem Nichts in die Höhe geschossen. Und mit diesen Bestien war nicht zu spaßen, falls sie der Hunger überkam.

Einen einzelnen Falken hatten Barz und Nabib zum Antritt ihrer Reise aus Thakkum mitgenommen. Dieser war, mit einer schönen Nachricht an ihre Heimat im Gepäck, kaum außer Sichtweite geflogen, da hatte das triumphierende Jagdgeheul eines Krarks ihnen schon verraten, dass ihr Brief nie den Ava erreichen würde. Und kein einziger Falke hatte Barz und Nabib während ihrer ganzen Reise erreicht, obwohl sie sonst förmlich mit solchen überschüttet wurden. Was das bedeutete, war ihnen klar: Der ganze Himmel war zu einem Jagdrevier für Krarks geworden und kein Falke war mehr sicher. Immerhin konnte Barz die nun fast allgegenwärtig zu findenden schwertschneidescharfen Krarkfedern verbrennen und das daraus entstehende Pulver kauen, um seinen Geist in einen meditativen Zustand bringen, in welchen es ihn nicht störte, dass keiner am Ava mit ihnen in Kontakt treten konnte. Manchmal schien es ihm gar, als würde er unter dem Einfluss dieses Meditationspulvers wie durch einen Nebel hindurch Schemen seiner Bekannten und seiner Familie erhaschen, und bis er seine Augen wieder öffnete, schien ihm jegliches Zeitgefühl abhandengekommen zu sein. Faszinierende Effekte, die er zurück in Thakkum gerne weiter untersuchen würde.

Doch früher oder später musste Barz seine Augen immer wieder öffnen. Und dann störten ihn die Krarks wieder, und Jirisas Sturheit, und Nabibs Schnarchen. Dann störte es ihn wieder, den Kontakt zum Ava verloren zu haben, und dass sie erst mit zwei Wochen Verspätung zurückkehren würden. Doch nun würde es kaum mehr einige Stunden dauern, bis Barz seinen lange vermissten Familienmitgliedern wieder in die Arme fallen konnte. Seine Schritte beschleunigten sich.

Nabib grinste und tat es ihm nach.

Jirisa trotte gemächlich hinter den beiden her und wurde keinen Deut schneller.\bigskip







Sonnenhoch. 42 Tage vor dem großen Unheil.\bigskip



Die Holzdielen knarrten wie eh und je unter Barz Stiefeln, als er einen ersten Fuß vom kleinen Böötchen auf die Pfahlbausiedlung Thakkum inmitten des großen Sees Ava setzte. Die Wasseroberfläche plätscherte ruhig und verheißungsvoll. Der große See! Seine Heimat! Nach seinen langen Reisen kehrte er immer wieder gerne an diesen Ort zurück. Er gab ihm neue Kraft. Und nicht nur ihm. Hier, auf der Pfahlbausiedlung Thakkum siedelte der Iquar-Stamm schon seit Jahrhunderten. Genauer gesagt, seitdem die drei legendären Barbaren-Brüder Yetohe, Iquar und Jpaxo sich miteinander verstritten und ihre Stämme allesamt andere Gebiete des Landes für sich beansprucht hatten: Yetohe die weite Steppe, Iquar den großen See und Jpaxo das hohe Plateau im südwestlichen Gebirge. Der Iquar-Stamm war längst nicht so zahlreich, wie der Jpaxo-Stamm es einst gewesen und der Yetohe-Stamm es immer noch waren. Doch im Gegensatz zu den anderen beiden Stämmen lebten die Iquar in einer Pfahlbausiedlung auf dem großen See, und so mussten sie weder Hunger noch Furcht vor Raubtieren oder feindlichen Armeen erleiden.

Das einzige Manko war der begrenzte Platz auf den Pfahlbauten. Barz und Nabib hatten die Steppenechse Jirisa bei den Echsenstallungen neben einer Weide mit ihren Artgenossen gleich außerhalb des Sees zurücklassen müssen. Glücklicherweise hatten sie rasch einige unbeschäftigte Burschen gefunden, die ihnen bereitwillig beim Transport all ihres Gepäcks von Jirisas Rücken ins Innere der Siedlung ausgeholfen hatten.

Tatsächlich hatten sich außergewöhnlich viele unbeschäftigte Menschen am Seeufer getummelt. Viel mehr Steppenechsen als sonst waren in und um die Stallungen gestanden, die meisten von ihnen mit schön verzierten Gurten und Sätteln bestückt. Und gleich dutzende schlichte Jurten waren daneben aufgestellt worden, zwischen denen es von Menschen nur so wimmelte. Durch die Öffnung einer Jurte hatte Nabib in deren Innerem eine hölzerne Statue erkannt und Barz darauf aufmerksam gemacht. Die Statue stellte wohl eine Art Rind dar, einen Büffel – auch wenn es dem Schnitzer an Talent zu mangeln schien. Barz und Nabib sahen sie nicht zum ersten Mal und wussten um ihre Bedeutung. Offenbar war sogar der Häuptling der Büffel-Sippe höchstpersönlich hier anwesend. Wenn sie es nicht besser wüssten, hätten Barz und Nabib anhand der schieren Anzahl Anwesender angenommen, dass der halbe Yetohe-Stamm auf Besuch in Thakkum war.

Nun trennte sich der Weg der beiden Freunde. Nabib konnte es kaum erwarten, seine Familie wiederzusehen. Die zahlreichen Zeichnungen, die er auf seiner Reise angefertigt hatte, würde er erst nach ausgiebigen Plaudereien mit seinen Eltern bei den Hütern des Wissens abliefern. Barz wollte hingegen lieber sofort seine Substanzproben ins Haus seiner Schamanin bringen und sich danach voll und ganz aufs Wiedersehen mit seiner Freundin Yafka konzentrieren können, mit der er auf dieser Reise wegen der Krarks nicht einmal Briefkontakt hatte halten können.

Beim Gedanken an Yafka tastete Barz unbewusst nach seiner Halskette, die er stets unter seiner wärmenden Kleidung trug. Zwei Ringe hingen daran: Ein geschnitzter Ring aus Holz, zu dem ein identischer auf Yafkas Halskette aufgefädelt war, und ein gehauener Ring aus Stein, zu dem Nabib einen identischen um seinen Hals trug.

„Heute Abend sehe ich dich wieder, zur sechsten Glocke vor dem großen Versammlungssaal“, versprach Nabib nach einer letzten Umarmung. Ganz gemäß Tradition würden heute Abend viele Iquar draußen rund um den Versammlungssaal speisen, und Nabib und Barz würden der Stammesleiterin Naquila und den versammelten Stammesmitgliedern vom abenteuerlichen Verlauf ihrer langen Reise berichten. Barz freute sich bereits ungemein auf den Anlass.\bigskip







Die Pfahlbausiedlung Thakkum war auf Pfählen inmitten des großen Sees Ava gebaut worden, an einer Stelle, wo der Seegrund vergleichsweise nahe unter der Wasseroberfläche lag. Barz hatte natürlich schon von mehrstöckigen Häusern gehört und auf einer Reise in die nördlichen Ausläufer des Grauen Gebirges tatsächlich bereits einige mehrstöckige Steintürme der Zwerge zu sehen gekriegt, doch verfügte Thakkum selbst bloß über ein einziges hohes Gebilde: Der dünne Zeitturm hinter dem Versammlungssaal, an dem die treuen Zeitmeister der Iquar zu jeder Stunde eine andersfarbige Fahne hissten. So konnte man von überallher in Thakkum in die Höhe sehen und die Zeit erkennen, auch wenn die Sonne gerade nicht mitspielte.

An diesem Zeitturm konnte sich Barz auch im üblichen Gewusel auf den Planken Thakkums gut orientieren und sein Haus rasch finden. Nabib wohnte gleich gegenüber, vermutlich war er schon lange eingetroffen. Barz atmete tief durch, zügelte seine Vorfreude und schritt näher.

Barz‘ und Yafkas Heim hatte aus der Ferne wie immer ausgesehen. Beim Nähertreten bemerkte Barz nun wie üblich einige Veränderungen, die in seiner Abwesenheit stattgefunden hatten. Hier war ein Fensterrahmen angeknackst, dort drüben wuchsen frische farbige Blumen aus einem Topf und die Haustür war neu bemalt worden, mit einer hübschen Rune, die vermutlich für Frieden oder Gesundheit stand. Runen waren noch nie Barz‘ Spezialität gewesen.

Die Tür schwang unter seinem sanften Druck auf und gab überraschenderweise den Blick frei auf ein kleines Kind, welches im Eingangsbereich saß und mit einem umherhüpfenden Ball spielte. Es warf Barz einen trotzigen Blick zu und fragte: „Wer bist du denn?“

„Dasselbe könnte ich dich fragen“, gluckste Barz und betrat sein Zuhause. Er verstaute seinen Bogen in einem Fach neben der Tür und hängte seinen langen Mantel mit den vielen Pulvertaschen daneben an die Wand.

„Mama! Mamaaaaa! Mama Yafka! Da ist ein Mann einfach so reingekommen!“, schrie das kleine Kind und rannte davon.

„Barz! Na endlich! Ich wusste doch, dass ich aus dem Fenster einen heimkehrenden Nabib erspäht habe“, rief eine Barz wohlbekannte Stimme von weiter hinten, und Barz wurde warm ums Herz. Yafka trat ins Licht, wischte sich ihre Hände an ihrer Schürze ab und beugte sich zum kleinen Kind herunter.

„Karyz, das ist Barz. Mein Freund, der auf Entdeckungsreise im ganzen Land war. Der mit der mächtigen Steppenechse Jirisa, ich habe dir doch Bilder von ihnen gezeigt.“

Während das Kind nickte und umgehend seinem Springball in ein anderes Zimmer nachhechtete, wandte Yafka sich Barz zu und erdrückte ihn fast in ihrer herzlichen Umarmung: „Endlich bist du zurück! Wir hatten uns schon solche Sorgen gemacht, als ihr nicht wie geplant vor zwei Wochen den Ava erreicht hattet. Mein tapferer Steppennomade. Komm her, lass dich herzen!“

„Papperlapapp, die richtigen Steppennomaden sind die Yetohe, die auch den Winter in der Kälte da draußen verbringen. Nabib und ich konnten viel von ihnen lernen, doch ist der Ava noch immer unsere feste Heimat, und per Definition muss ein No...“

Barz wurde von einem herzlichen Willkommenskuss unterbrochen.

„Ach, rede dich nicht runter. Willkommen zuhause! Schön, dass du es doch noch geschafft hast! Wegen dieser Krarks konnten wir nicht einmal Falken zu euch aussenden, um zu hören, ob es euch gut geht. Die alte Resi hat großspurig geweissagt, dass du und Nabib endlich in ein fremdes Land durchgebrannt wärt.“

„Ich würde dich nie freiwillig hier sitzen lassen. Ich hoffe, das weißt du.“

„Natürlich weiß ich das. Wie ist es euch ergangen?“

„Wir haben den großen Salzsee im Norden berührt!“, berichtete Barz mit leuchtenden Augen, „Selbst der Ava ist nichts als ein kleiner Tropfen dagegen.“

Yafka machte selbst große Augen und ratterte dann in beachtlichem Tempo herunter: „Ach, wirklich? Unglaublich! Kannst du mir alles erzählen? Bist du erschöpft? Magst du mir noch in der Küche helfen? Ich muss bis Sonnenhoch zwei Dutzend Rüben rüsten.“

Barz bejahte alle ihre Fragen und folgte Yafka in die Küche, wo ein beträchtlicher Stapel Goldrüben auf ihre Zubereitung warteten.

„Ich habe ein Geschenk für dich“, flüstere Barz verschwörerisch, „Sieh her!“

Er griff in eine seiner unzähligen Manteltaschen und warf daraus eine Prise glitzernden Staubs in die Höhe, welcher ohne Effekt zu Boden rieselte.

„Hübsch. Was war das?“, fragte Yafka, immer noch dem glitzernden Pulver nachsehend.

„Ein Ablenkungspulver!“, grinste Barz. Er hatte urplötzlich eine silbern schimmernde Blume in der Hand.

„Und das hier ist eine äußerst seltene Silberblume. Sie wächst am Ufer des großen Salzsees und ist ungenießbar, aber spür nur mal ihre Struktur. Überraschend robust für ein Gewächs, aber auch sehr selten. Ah, das Funkeln des roten Mondlichts auf ihren Blättern... ich musste sofort an deine Augen denken. Ich habe noch einige mehr gepflückt. Wenn irgendjemand es schafft, daraus Ableger zu ziehen, dann...“

„...dann wäre das meine Mutter selig gewesen.“

„Willst du zumindest versuchen, in ihre Fußstapfen zu treten? Die Wassermänner an der Küste schworen bei der Sauberkeit ihrer Schuppen, dass man die Zukunft schmecken kann, wenn man sich ein Blatt dieser Blume unter die Zunge legt.“

„Na, selbstverständlich. Ich kümmere mich gern darum. Sofern deine andere Schwiegermutter sie mir nicht abknöpft, sobald sie davon erfährt. Vielen Dank, Barz. Es ist schön, dass du wieder hier bist.“

Die beiden fielen sich erneut in die Arme.

„Apropos Mütter: ‚Mama‘?“, ahmte Barz die Stimme des kleinen Kindes belustigt nach, „Na, so lange bin ich aber nicht fortgeblieben, oder?“

Yafka gluckste und zog als Antwort ihre Halskette unter ihrem Kleid hervor. Neben Barz‘ Holzring baumelte nun ein zweiter, dünner Ring daran, der nach hellem Metall ansah.

„Oh Yafka, hast du wieder jemanden gefunden? Mein Glückwunsch! Wer kann sich denn so glücklich schätzen?“

„Ihr Name ist Zanyitaz und sie ist eine Fischerin vom Süddock. Du hast sie sicher auch schon gesehen. Karyz ist ihr Kind. Die beiden haben frisch das Gästezimmer bezogen. Zanyitaz ist gerade am Markt, doch sie wollte zum Mittagsmahl wieder hier sein. Ich kann es kaum erwarten, euch einander vorzustellen. Es würde mich sehr wundern, wenn ihr euch nicht verstündet.“

Yafka beäugte Barz ein bisschen besorgt, vermutlich auf seine Reaktion wartend. Sie ergänzte:

„Verzeih, wir wollten mit Austausch der Ringe und ihrem Einzug hier mindestens warten, bis du wieder hier bist. Oder zumindest bis du deinen Segen geben kannst. Aber dann konnten wir uns wegen der Krarks nicht mit euch austauschen und dann hattet du und Nabib solche Verspätung und dann musste Zanyitaz ihr Haus aufgeben – das ist eine spannende Geschichte für ein anderes Mal – und da dachten wir, wenn die Götter ihren Willen wollen, so sollen sie den halt kriegen. Du meintest doch immer wieder, dass dieses Haus zu groß sei für uns beide und dass ich, wenn du fort bist...“

Yafkas Stimme verlor sich. Barz, in dem plötzlich wieder ein Gefühl unglaublicher Zuneigung anschwoll, lächelte ihr zu, drückte sie fest an sich und küsste sie auf die Stirn.

„Keine Sorge, du kennst mich doch. Wir werden suchen, und wir werden Wege finden, mit denen wir allesamt zufrieden sind. Ich freue mich sehr für dich. Und bin natürlich gespannt darauf, diese Zanyitaz kennenzulernen. Und Karyz. Bei den Göttern, ein Kind in diesem Haushalt, wer hätte es für möglich gehalten!“

Yafka entspannte sich etwas und widmete sich wieder ihren Rüben. Barz folgte ihrem Vorbild und setzte nach: „Und da wirst du von Karyz bereits Mama genannt?“

Yafka grinste.

„Das wäre nicht meine erste Entscheidung gewesen, aber es hat sich so ergeben und.... ich mag’s irgendwie doch. Magst du mir jetzt vom riesigen Salzsee im Norden erzählen?“

„Klar, könnte ich, aber noch nicht zu viel. Sonst birgt der Reisebericht bei der Versammlung heute Abend ja gar keine Überraschungen mehr für dich.“

Yafka versteifte sich abrupt: „Oh, Barz, hast du es noch nicht vernommen? Die Versammlung heute Abend wird sich leider nicht um eure Rückkehr drehen können.“

„Ach? Ist etwas vorgefallen?“

„Hah! So könnte man es ausdrücken. Der König kommt zu Besuch.“\bigskip







Sonnentief. 42 Tage vor dem großen Unheil.\bigskip



Wie es bei Feiern üblich war, räumten an diesem Abend dutzende von Familien ihre Tische und Bänke nach draußen und verteilten Unmengen an schmackhaften Esswaren darauf. Als wollten sie sich alle gegenseitig dabei überbieten, die kreativsten Kreationen zum Konsumieren zu kredenzen, die man aus den kargen Pflanzen der Steppe und dem, was der See ihnen schenkte, kreieren konnte.

Ein einzelner Feuerzauberer aus einer fernen Eisinsel (Hadda? Harda? Der Name wollte Barz gerade nicht mehr in den Sinn kommen, es war jedenfalls diejenige Insel, von der er sich das seltene magische Cantharis-Pulver zusenden ließ) war während Barz‘ und Nabibs Abwesenheit an den Ava gelangt. Offenbar gehörte es zur Ausbildung der Zauberer, eine lange Reise in ein weit entferntes Reich zu machen, mit ihrer Magie zu helfen und Erfahrungen zu sammeln. Eine Einstellung, die die reiselustigen Barz und Nabib nur allzu gut nachvollziehen konnten. Dieser Feuerzauberer war möglicherweise ein wenig eitel (er nannte sich selbst stolz „den großen Lifornus“, ein Name, der nicht zuletzt mit seinem großen spitzen Zaubererhut zusammenhing, ohne den er nur selten gesehen wurde) verfügte aber auch über außergewöhnliche besondere Fähigkeiten, deren Demonstrationen ihn besonders beim Nachwuchs der Iquar eine begeisterte Beliebtheit verschaffte. Sobald Barz mit seiner Familie zur Versammlung trat, rannte Karyz zu Lifornus und bat ihn, Feuer zu spucken. Lifornus winkte lächelnd ab, doch während er sich abwandte, ging plötzlich sein Bart in Flammen auf, bloß für einen Augenblick. Karyz machte große Augen und Lifornus schlenderte davon, als wäre nichts geschehen.

Barz hatte einen wundervollen Nachmittag im Kreise seiner neu erweiterten Familie verbracht. Zanyitaz hatte sich als etwas schüchterne, aber gewitzte Person herausgestellt. Die Frauen hatten Barz in ein altes Strategiespiel eingeweiht, das im Süddock offenbar gerade der letzte Schrei war. Dann war Karyz dazugekommen und die Spielrunde war durch einen Spaziergang durch die Pfahlbausiedlung ersetzt worden. Barz hatte endlich wieder das Gefühl überkommen, zuhause zu sein. Ganz melancholisch war ihm geworden. Karyz hingegen war aufgedreht gewesen und war fröhlich den unter den Holzplanken hindurchschwimmenden Nixenkindern nachgerannt.

Nun, gegen Abend, floss Barz‘ Hochstimmung in einen düsteren Mischmasch aus Enttäuschung und Sorge über. Üblicherweise durften von Reisen zurückkehrende Iquar am Langtisch der Stammesleiterin Naquila neben dem Versammlungssaal speisen und von ihren Erlebnissen berichten, während die Nahestehenden aufmerksam lauschten und die Erzählungen nur leicht verfälscht an die weiter weg stehenden Tische weitergaben. Nun allerdings mussten Barz, Nabib und ihre jeweiligen Familien weiter weg Platz nehmen, der Versammlungssaal gerade noch knapp in Sichtweite. An ihrer statt saß neben der Stammesleiterin ein groß gewachsener Mann mit dichtem schwarzem Bart und einer imposanten goldenen Krone auf dem Kopf, deren breite Zacken selbst aus der Ferne noch stark glitzerten. Der Barbarenkönig. Begleitet wurde der König von gleich vier Häuptlingen aus verschiedenen Yetohe-Sippen. Je zwei Häuptlinge hatten links und zwei rechts des Königs Platz genommen hatten. Wenn Barz und Nabib näher säßen, könnten sie bestimmt die geschnitzten Büffel, Einhörner und dergleichen ausmachen, die die Kleidung der Häuptlinge zierten und ihre Clan-Mitgliedschaft deklarierten.

„Ich habe noch nie so viele Häuptlinge auf einmal gesehen“, raunte Yafka.

„Die Lage muss ernst sein“, nickte Zanyitaz.

„Wir waren einmal bei einem Treffen des Rats der Häuptlinge anwesend. Selbst da waren nur vier von denen dort, oder?“, flüsterte Nabib zu Barz. An die anderen gewandt führte er aus: „‚Anwesend‘ soll heißen, dass Barz und ich planlos in der Nähe der Zelte herumstanden, in denen die wichtigen Gespräche stattfanden. Die Yetohe sind bei politischen Diskussionen erheblich weniger offen als wir.“

Barz beendete seinen Gedankengang nachdenklich: „Mit der Anwesenheit seiner Häuptlinge will der König mächtig wirken, aber dass er überhaupt zugestimmt hat, sein Anliegen so öffentlich vorzutragen, zeugt von seiner Verzweiflung. Er ist auf die Iquar angewiesen. Ich werde reißen wie eine Bogensehne vor Spannung, wenn er nicht bald spricht. Will jemand noch Goldrüben?“

„Jaaa, ich will!“, rief Karyz und langte zu.\bigskip







Sonnenuntergang. 42 Tage for dem großen Unheil.\bigskip



„Hochverehrte Stammesleiterin, ich kann Euch nicht genug danken für die Gastfreundschaft, die die Iquar ihrem König entgegengebracht haben“, erscholl die tiefe, theatralische Stimme des Barbarenkönigs, woraufhin die in der Nähe sitzenden allesamt hastig von ihrem Geschirr aufsahen. Der große Lifornus verschluckte sich theatralisch und hustete zur großen Freude der umstehenden Jugendlichen einige Flämmchen hervor, ehe er ihnen zuzwinkerte und sich ebenfalls aufmerksam dem Barbarenkönig zuwandte.

„Es ist uns Ehre und Freude zugleich“, antwortete Stammesleiterin Naquila mit einem betont gelangweilten Unterton, der verriet, dass sie endlich zum Geschäftlichen kommen wollte.

„Euer König kommt mit schlechten Neuigkeiten aus dem Reich“, kam der König brav auf den Punkt, „Die Gefechte des Yetohe-Stammes gegen die dunkle Armee aus dem Gebirge im Süden verliefen lange Zeit triumphierend, und weiterhin kann blanker Knochen wenig gegen guten Stahl ausrichten. Doch scheinen die Reserven des Feinds wahrlich unerschöpflich, und mit jedem unserer gefallenen Krieger schenken wir seiner Sache einen weiteren tapferen Kämpfer.“

„Na, wenn der weitere tapfere Kämpfer für diesen Krieg anheuern will, dann wählt er die falschen Worte“, brummelte Nabib und füllte seinen Metkrug wieder bis zum Rand auf. Bereits zum zu vielten Mal, wie Barz an Nabibs leicht lallender Aussprache zu erkennen glaubte.

Der König ließ sich vom Getuschel nicht unterbrechen und verkündete mit von Bitterkeit triefender Stimme: „Es schmerzt mich, euch mitteilen zu müssen, dass wir das südliche Gebirge nicht bis in alle Ewigkeit werden halten können. Wir werden diese Lande verlassen und uns ein neues Zuhause suchen müssen. Häuptling Absorak von der Büffel-Sippe hat seine besten Krieger dem Wind Frais in den Westen folgen und das Königreich Andor hinter den Bergen ausspionieren lassen. Mit den zahlreichen Informationen, die Absorak nun zur Verfügung stehen, werden wir ihre kleinen Bauerndörfer im Nu überrennen, ihre baufällige Burg erstürmen und unseren Anspruch auf ihr fruchtbares Land geltend machen.“

Der Barbarenkönig schlug dem rechts neben ihm sitzenden Häuptling, vermutlich war das Absorak, wohlwollend auf den Rücken. Dieser verschluckte sich an seinem Eintopf, hustete heftig und blickte betreten zu Boden. Barz fragte sich, ob hinter seinem Blick mehr steckte als nur Scham über die verschüttete Suppe. Der Barbarenkönig ließ sich davon jedenfalls nicht den Wind aus den Segeln nehmen und fuhr schallend fort:

„Höre, o Stamm des Iquar! Dein König ruft dich an, sich den Yetohe beim Aufbruch in ein fremdes Land anzuschließen! Lange habt ihr Stämme zweier Brüder einander ignoriert, doch möget ihr nun wieder gemeinsame Sache machen. Mögen deine Krieger an vorderster Front bei der Eroberung unseres zukünftigen Reichs mitkämpfen! Denn die Gefahr aus dem Süden betrifft uns alle.“

Stille herrschte, nur unterbrochen vom Plätschern des großen Sees. Die unausgesprochene Warnung des Barbarenkönigs war unüberhörbar. Hätten sich die Yetohe erst einmal von der Front im südlichen Gebirge zurückgezogen, würde es nicht mehr lange dauern, bis die Skelette und Riesen aus dem Süden das Barbarenland gestürmt hätten. Der König glaubte offenbar nicht, dass die Pfahlbausiedlung Thakkum auf sich alleine gestellt dem Ansturm standhalten könnte. Oder er wollte sie das zumindest glauben lassen, damit die Krieger der Iquar ihn bei seiner Eroberung dieses Königreichs Andor unterstützen würden.

Der Appell des Königs war gleichzeitig als heroischer Aufruf und als Befehl formuliert gewesen. Doch schon die reine Anwesenheit des Königs und seiner Häuptlinge hier strafte seine selbstsicheren Worte Lügen. Wäre der Angriff auf Andor eine längst so sichere Sache, wie er das gerne den Anschein lassen wollte, so hätte er bloß einen Boten geschickt und sich nicht groß um die Antwort der Iquar geschert. Dass er hier war, zeigte doch, dass er verzweifelt auf Unterstützung hoffte. War der König auch bereits beim dritten Stamm, den Drachenkultisten der Jpaxo im westlichen Gebirge, vorstellig geworden? Hatten sie ihn abgewiesen? Technisch gesehen hatte der König Befehlsgewalt über alle drei Barbaren-Stämme, doch in der Realität folgten ihm nur die Sippen der Yetohe beim Wort, während die anderen beiden sich in allem außer dem Namen von seiner Regentschaft losgelöst hatten.

Dies wurde umso deutlicher, als Stammesleiterin Naquila das Wort ergriff und mit unverhohlener Abneigung sprach: „Dieses fremde Königreich, von dem ihr sprecht, dieses Andor... erinnere ich mich falsch, oder haben unsere Vorfahren nicht mit den dortigen Bewahrern einen Friedenspakt eingegangen?“

Der Barbarenkönig brummelte etwas in seinen Bart und antwortete gereizt: „Pakte sind ehrenhaft, ja, doch auch alte Pakte müssen manchmal gebrochen werden, wenn es ums reine Überleben geht. Das war den Bewahrern ebenso klar wie unseren Vorfahren, als sie damals den Frieden schlossen.“

„Und wie sieht es um die Landschaft in Andor aus? Selbst wenn Euer kühner Eroberungsplan aufgeht: Kann das Land nicht nur ein, sondern gleich zwei zusätzliche Völker ernähren? Sollen wir Iquar unsere sichere Heimat von Jahrhunderten einfach aufgeben, für eine unsichere Zukunft voller Kriege und Gefahren?“

„Kriege und Gefahren stehen den Iquar bevor, ganz gleich, wie der Stamm handelt. Die Yetohe ziehen Bauerngesindel und friedliche Bewahrer als Gegner einer Armee von Skeletten und Riesen vor. Wenn sie erst einmal vor euren Stegen stehen, würdet ihr euch nicht wünschen, andere Prioritäten gesetzt zu haben?“

„Mit diesem Krieg, den die Yetohe durch ihren unvorsichtigen Feldzug ins Land der Riesen begannen, haben wir Iquar nichts zu tun. Wir spüren keinen Druck durch diese Armee. Der große See ernährt uns, und er wird uns beschützen, wie er schon seit Anbeginn tat.“

„Wollt ihr euch etwa eurem König widersetzen?!“, brüllte Häuptling Absorak ungehalten auf, ehe er unter dem strengen Blick seines Königs verstummte.

Wildes Getuschel waberte durch die Reihen der Zuhörer. Selten wurde die Autorität des Barbarenkönigs und des Rats der Häuptlinge offen angezweifelt, und auch der König selbst schien diese Möglichkeit lieber nicht offen angesprochen zu haben.

Stammesleiterin Naquila richtete sich auf und alle Augen auf sie. Sie blickte die restlichen hochrangigen Tiere der Iquar der Reihe nach an. Barz konnte deren Gesichter nicht genau erkennen, vermutete aber kalte Entschlossenheit auf ihnen. Naquila nickte und sprach langsam, deutlich und bedacht, aber nicht ohne einen gewissen genüsslichen Unterton in ihrer Stimme: „Die Iquar kennen keinen König. Wir folgen einzig dem Gebot der Götter. Und die Götter haben nicht gesprochen. Wir sind mitfühlend gegenüber der hoffnungsarmen Lage der Yetohe und wünschen ihnen den Segen der Götter bei der Eroberung Andors. Doch wir werden keinen Teil daran haben.“

Die Herausforderung in ihren Worten konnte Barz nur schwer übersehen. Der König hatte keine Befehlsgewalt über die Iquar, solange sie die seine nicht anerkannten, und er konnte es sich nicht leisten, in einem offenen Gefecht gegen die Iquar seine Macht zu demonstrieren. Nicht, solange er alle seine Krieger fit für eine Invasion brauchte.

Häuptling Absorak schlug mit seiner Faust auf den Tisch und verteilte gute Suppe über seine Nachbarn. Der Barbarenkönig hingegen nickte ergeben. Als er wieder aufstand, schien alle Hochmütigkeit von ihm abzufallen. Leise sagte er:

„Dann bleibt mir bloß noch, eine Bitte an den Stamm der Iquar zu richten. So rasch wie möglich werden wir mit der Invasion Andors beginnen, doch nicht alle Mitglieder der Sippen können oder sollen in der ersten Welle mitreisen. Es gibt Alte und Schwache in unseren Reihen. Bitte, gestattet ihnen, ihre Jurten neben dem Ava aufzuschlagen, bis die Lage in Andor nicht mehr so gefährlich für sie ist. Unterstützt sie mit Nahrung und Heilmitteln. Und falls die Armee der Skelette zuvor bereits den Ava erreicht, mögen uns die Götter davor bewahren, so weist sie bitte nicht ab, sondern bietet ihnen Zuflucht auf den sicheren Pfählen von Thakkum.“

Stammesleiterin Naquila trat einige Schritte zurück und beriet sich mit den hochrangigen Tieren des Stammes. Dann verkündete sie kalt: „Ihre Jurten mögen die Zurückbleibenden in der Steppe aufschlagen, wo sie wollen. Falls dies neben dem Ava sein soll, so sei es so. Doch unsere Vorräte und unser Platz sind begrenzt. Falls es zu einer Belagerung durch diese Armee der Untoten kommen sollte, so werden die Iquar zwei Dutzend Mitglieder Yetohe in die Pfahlbausiedlung lassen, ehe wir die Stege kappen.“

„Zwei Dutzend?!“, schrie Absorak ungehalten, „Das reicht nicht im Geringsten! Wollt ihr uns etwa ausrotten?“

Der Barbarenkönig zischte ihm etwas zu und bedankte sich beinahe unterwürfig bei den Iquar für ihre Unterstützung in den düsteren Zeiten, die da kamen. Weiteres Getuschel ertönte in den Reihen der Iquar. Hier und da sah man Kopfschütteln. Auch Häuptling Absorak schüttelte entschieden seinen Kopf. Er richtete sich auf und richtete sein Wort an die gesamte Versammlung:

„O Iquar! Lasst ihr etwa zu, dass eure Stammesleiterin eure Geschwister so behandelt? Mir scheint, es wird hier über euren Kopf hinweg entschieden. Es ist nicht möglich, dass ihr alle wie sie gleichsam feige und egoistisch seid. Es geht hier ums Überleben eines ganzen Volkes!“

Zwei weitere Häuptlinge johlten Beistimmung. Der Barbarenkönig, obschon er seinen Kopf noch gesenkt hielt, beobachtete die Lage aus wachsamen Augen. Stammesleiterin Naquila versuchte, Absorak das Wort abzuschneiden, doch Absorak schrie nur noch lauter:

„Macht euch nichts vor in eurer ‚sicheren‘ Siedlung! Nicht nur unser Überleben steht auf der Kippe, sondern auch das eure. Ich habe die Riesen mit eigenen Augen gesehen. Das sind keine moralischen Wesen, es sind Monster. Sie werden diesen See überrennen, und alle, die sich bis dann noch hier aufhalten, werden Teil ihrer untoten Armee werden. Denkt an eure Geliebten und eure Kinder. Seht dieses Mädchen da vorne, wie es unbedacht mit seinem Essen spielt. Wenn ihr uns nicht auf der Suche nach einer neuen Heimat unterstützt, wird sein Skelett in wenigen Monden mit dem Feind marschieren, einen rostigen Säbel in der kleinen Hand!“

Rufe von Seiten der Zuhörer wurden laut, Absorak solle gefälligst seine Klappe halten. Stammesleiterin Naquila blickte sich hilflos um. Auch Barz blickte besorgt um sich. Er zweifelte nicht daran, dass die Iquar auf ihren Pfahlbauten sicher waren vor einer Armee aus dem Süden, die nicht einmal Boote besaß. Doch konnte es keine guten Folgen haben, wenn Absorak die Angelegenheit persönlich machte und Ängste weiter schürte. Die Yetohe waren verzweifelt und im Gegensatz zu den Iquar mussten sie jetzt handeln, also war ihr Verhalten durchaus verständlich. Doch der Plan des Barbarenkönigs war hirnrissig. Barz war noch nie selbst in Andor gewesen, doch hatte er genug über dieses Königreich gehört, um die Unwahrscheinlichkeit dieses Vorhabens abzuschätzen. Die Andori hatten fast die gesamte Trollpopulation des Landes ausgemetzelt. Trolle! Selbst eine gepanzerte Steppenechse konnte einem Troll nicht das Wasser reichen. Sich den Yetohe in diesem Kriegszug anzuschließen, glich Selbstmord.

„Gibt es denn niemanden unter euch, der sich noch um seine Familie kümmert?! Gibt es keinen, in dessen Adern noch das ehrbare Blut der ersten Brüder fließt? Wir Yetohe brechen morgen auf nach Andor. Zeigt, dass ihr besser seid als eure feige Stammesleiterin. Jeder, der sich uns bei der Invasion anschließt, wird fürstlich belohnt werden!“, beendete Absorak seine Tirade.

Hier und da ertönte zustimmendes Gejohle aus dem Publikum. Einige Iquar standen gar auf und hoben ihre Metkrüge. Am nächsten Morgen würden diese sich wohl den Yetohe anschließen und ihre Leben im Kampf gegen einen überlegenen Gegner riskieren. Barz schüttelte bloß seinen Kopf.

Dann drehte er sich um und sein Herz hüpfte in seine Hose. Nabib war auch aufgestanden, hatte seine Hand zur Faust geballt und klopfte sich damit auf die Brust.

„Setz dich wieder hin!“, zischte Nabibs Schwester ihm zu und zupfte an seinem Umhang, doch Nabib blieb standhaft stehen und johlte gemeinsam mit den übrigen aufgerichteten Iquar. Immer mehr erhoben sich, doch Barz kümmerte sich nicht mehr darum.

„Nabib! Du bist angetrunken, jetzt solltest du lieber keine wichtigen Entscheidungen treffen!“

In diesem Moment verkündete die tiefe Stimme des Barbarenkönigs: „Yetohe! Iquar! Im Morgengrauen ziehen wir los. Wer so tapfer ist, sich den Invasoren Andors anzuschließen, möge sich dann vor dem Steg versammeln. Ich danke jedem von euch bereits jetzt. Eure Kinder und Kindeskinder werden das in der Zukunft dann auch tun.“

Nabib nickte Barz mit einem traurigen Grinsen zu: „Du siehst: Es heißt jetzt oder nie.“\bigskip







Sternenhoch. 42 Tage vor dem großen Unheil.\bigskip



Erst nachdem die Nacht schon lange eingebrochen war und das Sternenband sich über den Himmel gelegt hatte, löste sich die Versammlung auf. Tische und Stühle wurden zurück ins Innere der Häuser verfrachtet und die Yetohe zogen sich wieder in ihre Jurten draußen auf dem festen Steppenboden zurück.

Barz hatte Nabib den Großteil des Abends sich selbst überlassen. Nun bewegte Nabib sich von seiner Schwester geleitet in Richtung seines Zuhauses und Barz fühlte, dass dies seine letzte Chance werden würde, ihn umzustimmen. Er langte hinter seinem Rücken nach einem Pulvergürtel.

„Nabib! So warte doch!“, rief Barz. Er erreichte ihn gerade noch, ehe er im Inneren seines Hauses verschwunden wäre.

„Ich hab’s ihm schon den ganzen Heimweg auszureden versucht“, sprach Nabibs Schwester, „Tu, was du nicht lassen kannst, aber lass ihm lieber seinen Willen, als dass ihr euch im Zwiste voneinander verabschiedet.“

Mit diesen Worten wurde Nabib stehen gelassen. Sofort legte Barz auf:

„Nabib, bitte, halte ein und sinniere nach über das, was du tun willst. Du hast Familie hier. Willst du sie wirklich im Stich lassen?“

„Oh, Barz, wenn ich hier dringend gebraucht würde, wäre ich doch gar nicht erst auf so lange Reisen aufgebrochen. Das gilt auch für dich, Barz. Du könntest mitkommen nach Andor.“

Barz hielt für einen Augenblick inne.

„Wir sind doch gerade erst heimgekommen. Ich will Yafka nicht gleich wieder verlassen.“

„Ich doch auch niemanden hier zurücklassen. Aber im Gegensatz zu dir sehe ich über meine aktuellen Wünsche hinweg und sehe, dass die einzige Art, wie ich meiner Familie und meinen Freunden tatsächlich helfen kann, darin besteht, dafür zu sorgen, dass für sie eine Zukunft... garantiert... ooh, dieser Satz ist zu lange für meinen müden Schädel. Jedenfalls ist es das, was ein wahrer Held tun würde.“

„Dann bin ich offensichtlich kein wahrer Held. Doch ist euer Vorgehen doch nichts als eine Verzweiflungstat.“

„Es ist zweifelsohne heldenhaft, für eine sichere Zukunft seiner Geliebten zu sorgen.“

„In ein fremdes Land einzudringen und einen aussichtslosen Kampf anzufangen, das nennst du eine sichere Zukunft?“

„Sicherer, als wenn wir hier bleiben und nichts tun, während eine finstere Armee aus dem Süden anrückt.“

„Das kannst du nicht mit Sicherheit wissen! Der Ava hat uns schon seit Jahrhunderten vor allen möglichen Gegnern bewahrt und kann es wieder tun.“

„Das kannst du ebensowenig mit Sicherheit wissen. Diese Skelettarmee ist nicht wie die Gegner, mit denen die Barbaren zuvor bereits zu kämpfen hatten. “

Stille machte sich zwischen den beiden breit. Dann...

„Nabib, ich werde dich vermissen. Ich werde mir solche Sorgen um dich machen. Wir waren noch nie so lange getrennt...“

Barz zog mit seiner freien Hand aus einem der dutzend kleinen Täschchen in seinem Mantel ein reich verziertes Amulett hervor. Seine Großmutter hatte es ihm zu seiner Volljährigkeit geschenkt. Nun führte er es zu Nabibs warmer Hand und legte es sanft hinein.

„Nimm dieses Amulett, auf dass es dich beschützen möge. Und dass du mich nicht vergisst.“

„Oh, Barz, du alter Romantiker. Wenn du mir einen Antrag machen willst... nicht jetzt. Glaub mir, es ist schon so schwer genug. Ich habe mich entschieden, und ich werde diese Entscheidung durchziehen. Und glaube ja nicht, dass du mir nicht fehlen wirst.“

Nabibs Stimme zitterte leicht, als seine Finger sich von Barz‘ lösten und er das reich verzierte Amulett an sich nahm. Im Dunkeln war sein Gesicht nur schwer zu erkennen, aber Barz glaubte, Tränenspuren glitzern zu sehen.

„Na, komm schon her, du sturer Hornbär“, brachte Barz hervor. Dann fielen die beiden Steppennomaden einander in die Arme. Barz drückte Nabib fest an sich und ließ für Minuten nicht mehr los. Nabib erwiderte die Umarmung. Barz atmete Nabibs Geruch ein letztes Mal ein. Dann ließ er ihn los. Nabib winkte ihm ein letztes Mal zu, betrat sein Haus und machte sich vermutlich umgehend ans Packen.

Barz trat einige Schritte zurück und betrachtete das Sternenband über dem klaren Himmel, die glitzernden Sterne vor dem schwarzen Hintergrund.

Dann öffnete er seine Hand und die Prise Bannpulver, die er darin bereitgehalten hatte, verstreute sich grünlich glitzernd im Wind, rieselte durch die Holzplanken und versank im schwarzen Ava. Barz‘ Hand kribbelte magisch, als er den letzten Rest des schimmernden Pulvers davon abrieb.

Es war eine gute Entscheidung gewesen, das Pulver nicht zu benutzen, sprach er zu sich. Es war gut, Nabib seinen eigenen Weg gehen zu lassen, statt ihn entgegen seines unbrechbaren Willens hier zu behalten. Doch glaubte Barz das noch nicht wirklich.

























\newpage
\section{Der Angriff der Riesen}


Sonnenaufgang. 41 Tage vor dem großen Unheil.\bigskip



Am Morgen flossen am Ufer des Avas noch einmal die Tränen, als sich Nabib endgültig von seiner Familie und von Barz verabschiedete. Er strubbelte seiner Nichte durch die Haare, umarmte Yafka, küsste Barz und hastete dann so schnell es sein Stolz ihm erlaubte davon, ehe er noch länger mit seiner Entscheidung hadern konnte.

Mit ihm brach der Großteil der Yetohe und eine Vielzahl an freiwilligen Iquar auf. Auf in Richtung Andor. Ein schier endloser Zug von Barbaren, einige von ihnen auf riesigen Steppenechsen reitend, und noch mehr Steppenechsen mit Jurten, Nahrung und Waffen beladen. Ganz vorne ritt der Barbarenkönig, umringt von seinen treusten Häuptlingen und einigen tapferen Kriegern. Barz kniff die Augen zusammen und versuchte zu erkennen, ob Nabib sich gar zu diesen Elitekämpfern gesellt hatte. Er hoffte es nicht. Nabib war ein Zeichner, ein Kartograph. Dass er außergewöhnlich gut mit einer Axt umgehen konnte, machte ihn noch lange nicht zu einem Meisterkämpfer.

Nachdem Barz zum letzten Mal sichergestellt hatte, dass kein Yetohe seine geliebte Steppenechse Jirisa mit den anderen mitgeführt hatte (und nachdem Barz, an Jirisas Seite gekuschelt, ein letztes imaginäres Streitgespräch mit Nabib zu dessen Abziehen durchgekaut hatte), schweiften seine Gedanken zu denjenigen Yetohe, die nicht fortgezogen waren.

So einige Jurten waren zurückgeblieben. Die Alten und Schwachen, die Jungen und Friedliebenden, sie alle warteten nun neben dem Ava auf eine gute Nachricht aus Andor. Ebenso wie diejenigen Iquar, deren Familie und Freunde nun auf einem Kriegszug waren. Waren vielleicht gar mehr Iquar mit dem Barbarenkönig aufgebrochen, als sie Yetohe hier abgeladen hatten? Schwer zu sagen.

Barz verbrachte den Rest des Sonnenlichts damit, Pfeile in Zielscheiben zu versenken und seine Rückkehr an den Ava zu verfluchen. Er und Nabib hatten bereits solche Verspätung gehabt. Wären sie nur einen Tag später heimgekommen, so hätten sie den Abzug der Yetohe verpasst. Und Nabib wäre noch hier.

Barz jagte einen letzten Pfeil ins Schwarze einer Zielscheibe. Dann kehrte er nach Hause zurück. Immerhin hatte Yafka die Vernunft besessen, in Thakkum zu verblieben. Und Zanyitaz und Karyz. Es war an der Zeit, sich an seine neue Familiendynamik zu gewöhnen. Und an der Zeit, mit seiner Schamanin die gefundenen Substanzen der letzten Reise zu analysieren. Wer weiß, vielleicht würde er ja gar etwas herausfinden, was gegen die unweigerlich in dieser Gegend aufkreuzende Armee der Riesen aus dem Süden helfen werden könnte.\bigskip







Sonnenaufgang. 23 Tage vor dem großen Unheil.\bigskip



Die Tage waren ins Land geflossen und zu Wochen geworden. Barz hatte viel Zeit mit seiner Familie verbracht, und viel Zeit bei seiner Schamanin für die Analyse der Stoffproben aufgewendet, welche er auf seiner langen Reise gesammelt hatte. Das konnte ihn zwar nicht immer von seinen Sorgen um Nabib ablenken, doch oft genug.

Nun kam der Tag, an dem die Armee des Riesen am Horizont erspäht werden konnte. Barz hatte sie als einer der ersten erblickt, als er mit Jirisa im Schlepptau und Karyz auf deren Rücken einen Spaziergang in der Nähe des großen Sees Ava gewagt hatte. Sie hatten gesehen, wie sich aus der Ferne eine ungeheure Streitmacht dem Ava näherte. Wie sie sich wie ein großer grauer Wurm über die weite Ebene der Barbarensteppe wälzte. Und man musste kein Held, kein Krieger und kein Feldherr sein, um zu wissen, was das hieß: Der letzte standhafte Trupp der Barbaren, der sich nicht der Invasion Andor angeschlossen, sondern sich weiterhin dem Heer der Riesen entgegengestellt hatte, war schlussendlich überrannt worden. Niemand stand mehr zwischen den Skeletten und dem großen See Ava. Und niemand stand mehr zwischen den Skeletten und dem Königreich Andor, wenn die Riesen beschlossen sollten, den Spuren der Yetohe in den Norden zu folgen.

Die übriggebliebenen Yetohe brachen ihre Jurten ab und flohen nach Thakkum, auf die schützenden Pfähle. Die Stammesleiterin Naquila versuchte, nach drei Dutzend Hilfesuchenden einen Schlussstrich zu ziehen, doch schien kein Iquar wirklich erpicht darauf, diese Entscheidung durchzusetzen, und so wurde sie in stiller Übereinkunft missachtet.

Die letzten Jurten und Totems, die nicht mit auf den See genommen werden konnten, wurden feierlich in Brand gesetzt, auf dass sie nicht dem Feind in die Hände fallen mochten. Die letzten Steppenechsen, die nicht gen Andor gezogen waren, wurden freigelassen. Barz blickte Jirisa lange in die Augen, ehe er sie mit einem Klaps zum Forteilen anhielt. Natürlich hielt Jirisa sich nicht daran und blieb störrisch neben den Stallungen stehen, egal, wie sehr Barz weit weg von hier zeigte. Am Ende musste er sie gewahren lassen.

Alle Boote, die noch am Ufer befestigt gewesen waren, wurden auf den See geschifft und mit der Siedlung verbunden. Die Pfahlbausiedlung lag einsam im großen See.

Nachdem sich so viele Iquar dem Angriff auf Andor angeschlossen hatten, gab es theoretisch ausreichend Platz auf der Pfahlbausiedlung für die übriggebliebenen Yetohe, auch wenn die Verteilung auf die verschiedenen Haushalte ein organisatorischer Albtraum werden würde. Barz, Yafka und Zanyitaz nahmen ein uraltes Ehepaar bei sich auf, welches sich oft zankte, insgeheim aber wohl beide füreinander durchs Feuer springen würden. Die beiden konnten gute Geschichten ohne Ende erzählen. Karyz mochte sie besonders und Zanyitaz fand in ihnen würdige Gegner in so manchen Brettspielen.

Doch war niemandem in Thakkum wirklich wohl zu Mute.

Denn rund um den See herum befanden sich nun sie.

Die Untoten.

Das grusligste an ihnen war die Stille. Jede andere marschierende Armee wäre schon von fern an Stimmen, Gebrüll, vielleicht gar Marschmusik erkannt worden. Doch die Skelettkrieger der Riesen liefen pausenlos im Gleichschritt, während nichts das leise Klappern ihrer Rüstungen und Waffen sowie das rhythmische Trommeln ihrer knöchernen Füße auf die weiche Erde zu vernehmen waren.

Zunächst hatten sie sich dem Ava kaum genähert, als wollten sie einfach nur am See vorbeiziehen und weiter in Richtung Andor marschieren, auf den Spuren der Armee des Barbarenkönigs. Dann aber schien sich etwas zu ändern. Vielleicht hatte eine der Kommandanten die Siedlung Thakkum inmitten des großen Sees erspäht. Egal, was es gewesen war, die Armee der Skelette hatte jedenfalls Halt gemacht und Stellung rund um den Ava bezogen. Doch standen sie nicht still. Stattdessen umrundeten die Skelette das Seeufer. Stetig marschierten sie, Tag und Nacht, ohne zu ermüden, ohne vom Pfad abzukommen. Sonst taten sie nichts. Selbst die herumstehenden Steppenechsen am Seeufer ließen sie in Ruhe, wie Barz überrascht (und erfreut) feststellte, als er seine eigene Steppenechse Jirisa ungestört nur einen Steinwurf entfernt von der vorbeimarschierenden Armee weiden sah. Was wollten diese Skelette nur?\bigskip







Sonnenhoch. 20 Tage vor dem großen Unheil.\bigskip



Am ersten Tag nach der Ankunft der Armee am großen See ertönten einige dumpfe Trommeln gerade dann, als die Sonne am höchsten Punkt am Himmelszelt stand. Eine riesengroße Gestalt trat nach ans Ufer des Avas. Still stand der Riese dort, während seine Armee hinter ihm vorbeimarschierte. Dann hob er seinen Arm und zeigte auf die Siedlung. Er brüllte etwas in einer tiefen, kehligen Sprache, das keiner der Anwesenden verstand. Doch blieb er weiter stehen, als wartete er auf eine Antwort. Hin und wieder wiederholte er seinen Ruf:

„Ungadahr Ferntahr! Parahnu Krahder! Darack Ambacus Berand Impalahr Tugabahr! Konthra Impalahr Rah!“

Glücklicherweise verfügte Barz‘ Schamanin über einen Trank, der ihr erlaubte, Strukturen in gesprochenen Worten zu erkennen und sie besser sortieren zu können, sodass es ihr gelang, die Sprache der Riesen anhand der wenigen Worte, die sie aus der Ferne gehört hatte, zu erkennen. Das Rezept für diesen Trank hatte sie einst von einem Schamanen der Yetohe erlernt, der ihn wiederum vom Großen Büffel höchstpersönlich gelernt bekommen hatte. Zumindest hatte dieser Schamane der Yetohe bei seinem Bart darauf geschworen, und den Yetohe waren ihre Bärte durchaus wichtig.

Nun konnte Barz‘ Schamanin ein bisschen von diesem außergewöhnlichen Trank schlucken, um danach für die restliche Barbarengesellschaft zu übersetzen:

„Der Riese schreit: ,Name Ferntahr, Prinz Unsterbliche.‘ Dieses eine Wort, ‘Krahder‘, ‚Unsterbliche‘, bedeutet noch mehr. Mir scheint, das könnte auch der Name dieser Riesen für ihr eigenes Volk sein.“

„Er hat sich uns also vorgestellt. Wie höflich“, spottete Stammesleiterin Naquila, „Wie geht es weiter?“

„Gebt mir einen Augenblick, diese Interpretation ist kein Kinderspiel... ‚Menschen... Ambacus... Fügsamkeit... Folge... Belohnung! Widerstand... Folge... Tod!‘ Mir scheint, er fordert uns auf, sich ihm zu ergeben. Das Wort ‚Ambacus‘ ist interessant. Ich spüre eine Assoziation mit ‚Folgsamen‘, aber mit einem unschönen Unterton dahinter. Vielleicht wäre ‚Unfreie‘ passender? Oder gar ‚Sklaven‘?“

„Ja, dass die Riesen aus dem Süden nicht nur Skelette, sondern auch Sklaven befehligen, ist schon seit längerem bekannt. Meint dieser Prinz wirklich, wir würden uns so leichtfertig ergeben? Wer würde einem solchen Feind einfach die Tore öffnen, nur weil er darum bittet?“

„Niemand, der doch Hoffnung hat. Und über die verfügen wir zum Glück ja.“\bigskip







Sonnenaufgang. 19 Tage vor dem großen Unheil.\bigskip



Am nächsten Morgen wurde die ganze Siedlung von den kehligen Rufen eines Krahders geweckt. Dieses Mal war es nicht Prinz Ferntahr, der sich ihnen stellte, sondern eine vergleichsweise kleine Riesin mit krummen Rücken, die sich auf einen beinernen Stab stützte, an dem mehrere gehörnte Schädel einer unbekannten Spezies hingen. Die Riesin hielt irgendetwas grün dampfendes an ihre Kehle und ihre Stimme drang viel lauter über den See, als Prinz Ferntahrs es geschafft hatte.

„Ungadahr Nahrack! Kosdivahr Krahder! Cohuiron Impalahr Vahnur! Zeruhor Gorb-Palacs Ihra!“

Wieder versammelten sich einige Schaulustige und Stammesleiter um die Schamanin, als sie die Worte zu übersetzen versuchte:

„Name... Nahrack! Hexe... Krahder! Kampf... Folge... Verlust! Waffen... Niederlage... Befehl!“

„Die hoffen immer noch darauf, dass sie uns zermürben können und uns ergeben, ehe sie zu aufwendigeren Mitteln greifen müssen. Wollen wir ihnen langsam eine Antwort geben, ehe sie zum Schluss kommen könnten, dass wir sie nicht verstünden?“, meinte der große Lifornus nachdenklich. Dieser Feuerzauberer, der eigentlich nur zum Abschluss seiner Ausbildung an den Ava gekommen war, war lieber in Thakkum geblieben, statt sich den Invasoren des Barbarenkönigs nach Andor anzuschließen oder alleine in die Steppe zu wandern. Es war möglich, dass er nun daran zweifelte, ob das die richtige Entscheidung gewesen war. Sein einst so schneidiger Bart war ungepflegt und man sah Lifornus immer häufiger in seinem Schlafmantel umherlaufen.

„Was wissen wir über diese Krahder?“, fragte Stammesleiterin Naquila in die Runde.

„Es gibt tatsächlich bloß zwei, die die Skelettarmee befehligen“, antwortete Barz. Seitdem er mithilfe seiner Schamanin aus einigen seltenen Materialen, die er in seiner letzten Reise mit Nabib gesammelt hatte, ein einzigartiges Umwandlungspulver geschaffen hatte, war er quasi über Nacht zu einer Koryphäe in Sachen Fernrohrproduktion geworden und hatte so irgendwie die Verantwortung über die Beobachtung der feindlichen Armee erhalten.

„Es mag einen anderen Anschein haben, weil wir verteilt über die ganze Armee immer wieder größere Riesengestalten sehen, aber unsere Fernrohre enthüllten uns, dass die meisten von ihnen Untote sind, nichts als puppetierte Knochenhaufen. Tatsächlich sahen wir bislang bloß diese zwei lebenden Krahder, diesen Prinz Ferntahr und diese Hexe Nahrack. Nur zwei Personen in einem riesigen Heer aus leblosen Puppen. Es wäre lächerlich, wenn unsere Lage nicht so prekär wäre.“

Naquila schmunzelte tatsächlich leicht, ehe sie nachhakte: „Und was könnt Ihr uns über diese beiden Krahder sagen?“

„Prinz Ferntahr bewegte sich gestern immer wieder rastlos umher und musterte die Umgebung. Er hat sich sogar schon einmal vorsichtig einer Steppenechse genähert und sie gestreichelt. Er scheint beinahe... aufgeregt? Wir können sein Gesicht selten sehen, er trägt oft einen Schädel darüber. Wie eine Art Krone vielleicht? Ich bin mir nicht ganz sicher über die Spezies, zu der dieser Schädel einst gehörte. Es könnte gut ein gefallener Krahder sein. Oder ein kleiner Drache? Und dann ist da diese Nahrack, diese Hexe. Sie verfügt über irgendwelche düsteren magischen Kräfte. Sie verbringt die meiste Zeit zwar im Zelt, das die Krahder am Ufer aufgebaut haben, doch einmal wurde sie gesehen, wie sie nach vorne trat und einen grünen Blitz auf einen zu tief fliegenden Krark schleuderte.“

„Sie ist eine Dunkle Hexe“, murmelte der große Lifornus, „In Hadria haben wir davon gehört. Einige unserer älteren Adepten hatten sich in ihren Reisen bis ins Graue Gebirge und darüber hinaus gewagt. Und unangenehme Begegnungen gemacht. Es ist zweifelsohne Dunkle Hexerei, die die toten Knochen dieser Skelette umherwanken lässt. Die Frage ist, ob die Dunkle Hexe sie aktiv steuert oder ob sie weiterhin agieren könnten, nachdem man die Hexe stürzte.“

„Wie stellt Ihr es Euch vor, die Dunkle Hexe von hier aus zu stürzen?“, warf Naquila ein.

„Jedenfalls einfacher als eine riesige Skelettarmee.“

Lifornus und Naquila funkelten sich gegenseitig an, während sie beide verstummten.\bigskip







Sonnenhoch. 18 Tage vor dem großen Unheil.\bigskip



Am nächsten Tag war die Lage noch nicht besser geworden. Die Bewohner Thakkums waren relativ ratlos, und den Krahdern schien es nicht besser zu entgegen. An diesem Tag stand pünktlich zum Sonnenhoch wieder die Krahder-Hexe Nahrack am Ufer, doch dieses Mal hatte sie eine Begleiterin neben sich stehen. Eine großgewachsene, doch neben Nahrack immer noch mickrig erscheinende Menschenfrau, die allem Anschein nach noch lebte. Eine Unfreie, eine Ambacu. Sie war mit seltsamen dunklen Tuchstreifen bekleidet und ihr Gesicht war noch weißer als das eines Skelettkriegers. Vermutlich Gesichtsfarbe. Kriegsbemalung? War dies nur Tradition oder steckte eine magische Bedeutung dahinter?

Gebückt und unterwürfig stand die Menschenfrau da, doch als Nahrack ihr ein grünlich dampfendes Etwas vors Gesicht drückte, klangen zunächst ihr stockender Atem und dann auch ihre stockenden Worte gellend laut über den Ava. Die Ambacu sprach eine andere Sprache als die Krahder, doch leider ebenfalls keine, die die Iquar ohne Schamanentrank verstanden hätten. Barz‘ Schamanin übersetzte, und als die Botschaft wieder als eine Mischung aus Appellen zum Ergeben und Versprechen von Unversehrtheit bei raschem Aufgeben ihrer Position bestand, wurde auch dieser Versuch der Kontaktaufnahme der Krahder verworfen.

Stammesleiterin Naquila sprach feierlich: „Der große See schützt uns und schenkt uns Nahrung. Wir können theoretisch beliebig lange im Ava aushalten. Eigentlich ist es sogar besser, wenn die Krahder uns belagern, statt die Invasoren nach Andor zu verfolgen.“

Die Anwesenden nickten allesamt zufrieden. Dass die Krahder und ihre Skelette nicht einfach so vom Ufer verschwinden würden, war allen klar, doch solange sie keine akute Gefahr darstellten, wirkte plötzlich niemand mehr so erpicht darauf, nach einer Lösung zu suchen.

Und so schien die Sache fürs Erste gegessen. Es traten auch keine weiteren Ambacus oder Krahder ans Seeufer, nur die stetig umhermarschierenden Skelettkrieger verblieben. Ob die Riesen glaubten, dass die Bewohner Thakkums sie nicht verstünden, oder ob sie verstanden, dass sie vermutlich keine Antwort erhalten würden, war nicht klar.

Klar war somit nur, dass der nächste Zug in diesem Spiel von den Krahdern gemacht werden würde.\bigskip







Kurz vor Sonnenaufgang. 14 Tage vor dem großen Unheil.\bigskip



Barz wurde von einem leisen Rumpeln geweckt. Yafka regte ich leise neben ihm.

„Hast du das auch gehört?“, zischte Barz ihr zu. Yafka nuschelte etwas und vergrub sich wieder in ihrem Kissen.

Nun ertönte gar ein Klirren von der Küche. Barz richtete sich auf und schlich vorsichtig aus dem Raum. Eigentlich gab es keinen Grund, besorgt zu sein. Der Ava schützte die Pfahlbausiedlung vor feindlich gesinnten Personen und Einbrüche waren eine ausgesprochene Seltenheit. Doch seit der Ankunft der Krahder lagen seine Nerven wie die von vielen Einwohnern Thakkums blank. Auch wenn sich vermutlich nur Karyz in die Küche geschlichen hatte, wollte er lieber doppelt nachprüfen, statt etwas Übles zu übersehen.

Barz lugte um die Ecke und erkannte: Es hatte sich tatsächlich jemand in die Küche geschlichen! Doch war es kein Kind. Zanyitaz stand mit einem Topf bewaffnet vor dem Esstisch und rührte sich nicht.

„Zan? Ist alles in Ord...“

Weiter kam Barz nicht, denn Zanyitaz zuckte bei seinen Worten zusammen. Ihr Blick glitt hinüber zu Barz und ihre Ablenkung wurde gnadenlos ausgenutzt. Aus einer aus Barz‘ Blickfeld nicht sichtbaren Ecke der Küche jagte ein Skelettkrieger mit einer rostigen Wurfaxt in der Hand hervor und ließ selbige auf Zanyitaz niedergehen. Diese wuchtete ihren Topf gerade noch rechtzeitig nach oben, damit die Axt sie nur an der Schulter streifte. Einen üblen Schnitt hinterließ sie dennoch.

Nahkampf war noch nie Barz‘ Stärke gewesen, hatte es auch nicht sein müssen, da man in der Steppe jeden Gegner in Pfeilreichweite schon lange sah, ehe eine Situation derart brenzlig werden konnte. Wenn Barz auf seinen Reisen in ein unschönes Gefecht geraten war, war es also mit Pfeil und Bogen ausgetragen worden. Mit dem Bogen, welcher zwei Räume weiter drüben unnütz an einem Haken hing, während Zanyitaz hier in akuter Lebensgefahr war. Also tat Barz das einzige Vernünftige: Er raste auf den Skelettkrieger zu und packte dessen Faust, welche die Wurfaxt umschlossen hatte.

Da sah Barz ein, wie unvernünftig das gewesen war. Der Skelettschädel drehte sich auf seiner losen Wirbelsäule, bis er Barz aus leeren Augenhöhlen einen feurigen Blick entgegenwerfen konnte. Dann vergrößerte sich irgendwie der Abstand zwischen den Knochen bei den Gelenken des Skelettkriegers. Dieser tat einen mächtigen Schritt und wanderte buchstäblich durch Barz hindurch. Barz spürte das Kratzen von Knochen auf allen Seiten seines Körpers, während das Skelett sich über ihn stülpte. Und plötzlich war es nicht mehr Barz, der von hinten den Arm des Skeletts fixierte, sondern das Skelett, welches von hinten Barz‘ Arm auf seinen Rücken verdrehte. Immerhin hatte Barz weiterhin die Wurfaxt im Griff.

Das war wirklich faszinierend. Es schien, als wären die Fäden Dunkler Hexerei, die den Skelettkrieger zusammenhielten und seine Bewegungen ermöglichten, völlig durchlässig für gewöhnliche Materie. Eine solche Verbindung wäre für zahlreiche andere Anwendungen überaus nützlich. Wobei, zuerst müsste man vermutlich testen, wie stabil die Verbindung war. Barz hatte schon davon gehört, wie starke Kämpfer der Yetohe ein Skelett in seine Einzelknochen zerrissen hatten, woraufhin es sich auch nicht mehr zusammensetzen hatte können. Und selbst die außergewöhnlichste Verbindung nützte kaum etwas, wenn sie schnell riss.

Leider war im Moment gerade keine Zeit für solcherlei Experimente. Stattdessen rangelte Barz noch kurz mit dem Skelettkrieger, bis dieser ihn in sein Ohr biss und Barz die Wurfaxt fallen ließ. Innerlich schloss er bereits mit seinem Leben ab.

Da ertönte ein lautes Klonk direkt neben Barz‘ Ohr. Zanyitaz hatte mit ihrem schweren Kochtopf den Schädel des Skelettkriegers glatt von dessen Wirbelsäule gefegt und das Skelett wankte zurück. Barz dankte den Göttern, ließ die Axt fallen, rannte zu einer seiner Substanzschublade, griff nach dem erstbesten Pulver, welches dort gelagert wurde, und warf eine gehörige Portion davon in Richtung des Skeletts.

Leider war das das Meditationspulver gewesen, und folglich wenig hilfreich. Braunroter Rauch stieg auf und das Skelett wankte ungestört weiter herum. Sein Fuß stieß auf die fallen gelassene Axt und hob sie hoch.

Barz griff erneut in die Schublade und nahm sich diesmal die Zeit, nach dem grünlich schimmernden Bannpulver Ausschau zu halten. Er griff sich eine Handvoll davon, ballte seine Faust fest und schüttelte sie. Als er sie wieder öffnete, glitzerte das Pulver rasch immer heller und Barz‘ Hand begann, magisch zu kribbeln. Rasch drehte er sich um und schleuderte die Faustvoll dem nun kopflosen Skelettkrieger entgegen, welcher gerade Zanyitaz mit erhobener Axt entgegensprang.

Das Bannpulver traf das schädellose Skelett mitten in die Rippen, und auch wenn einiges davon wirkungslos zwischen selbigen durchflog, hatte Barz genug der seltenen Substanz geschleudert, damit es seine Wirkung entfalten konnte. Magisches Knattern ertönte und ein hellgrünes Glühen baute sich um das Skelett auf, dessen Sprungangriff sich verlangsamte, bis es mitten in seinem Schlag festgefroren in der Luft hing.

Der Schädel des Skeletts klapperte ein bisschen entfernt vom restlichen Geschehen auf dem Küchenboden herum, doch konnte dieser auf sich alleine gestellt bloß seinen Unterkiefer auf- und zuklappen und sich dadurch nicht fortbewegen. Der war also keine Gefahr.

Barz schnappte sich ein Tuch und wedelte den Rest des Bannpulver-Dampfs aus dem zerstörten Fenster hinaus, ehe sich dessen magische temporäre Wirkung weiter ausbreiten konnte. Ah, da war der Skelettkrieger also eingedrungen. Dann wandte Barz sich Zanyitaz zu und half ihr auf. Die Wunde an ihrer Schulter sah unschön aus.

„Danke für die Hilfe“, sagte er. Zanyitaz verzog ihr Gesicht und nickte schwach.

„Yafka! Schnapp dir ein Messer, greif dir Karyz und verbarrikadier dich im Schlafzimmer!“, rief Barz laut, während er seine Pulverschubladen nach heilendem Cirathans-Pulver durchsuchte.

„Was ist los?“, erklang Yafkas verschlafene, doch alarmierte Stimme dumpf durch zwei Wände hindurch.

„Skelettkrieger in der Küche!“

„Skelettkrieger in der Küche?! Was ist der über den See gekommen?“

„Woher sollen wir das wissen?“

„Geht es euch gut?“

„Uns geht’s soweit gut. Die akute Gefahr ist gebannt!“

„Jetzt hört auf zu schwafeln, schnapp dir endlich Karyz und versperre Tür und Fenster, wir kommen dann nach!“, brüllte Zanyitaz überraschend laut auf und kippte zur Seite. Barz zuckte zusammen und wandte sich wieder seiner Suche nach dem stärkenden Cirathans-Pulver zu, während polternde Schritte aus dem Nebenzimmer von Yafkas raschem Handeln zeugten. Dann endlich hatte Barz das Säcklein mit Cirathans gefunden. Nun, ob es wirklich ein Pulver war, war debattierbar, für manch peniblen Beobachter (darunter immer wieder gerne Barz‘ Schamanin) mochte die grobkörnige Substanz aus zu großen Körnern bestehen, um noch als echtes Pulver gelten zu dürfen. Aber das war jetzt nicht relevant. Barz pickte drei mittelgroße blaue Cirathans-Körner heraus, rannte zu seinem Regal der eingesperrten Winde und griff sich eine Phiole mit luftdicht verschlossenem grünem Cantharis-Pulver daraus.

Keine Minute zu früh. Kaum hatte Barz die Phiole ergriffen, erglühte die kleine Staude Mondbeeren auf dem Schrank strahlend weiß. Barz wusste genau, was das bedeutete: Die ersten Sonnenstrahlen der aufgehenden Sonne fielen durch die Fenster ins Haus und ließen die Mondbeeren magisch aufleuchten. Dies war ein sehr nützlicher Effekt, um die sonst sehr unscheinbaren Beeren aufzuspüren. Dies war allerdings alles andere als nützlich, wenn man die temporären Anomalien, die Mondbeeren auslösen konnten, ausnutzen wollte, um eine Gefahr zu bannen. Und Bannpulver bestand bekanntlich zu einem Großteil aus getrockneten Mondbeeren.

Barz wandte sich zurück zum gebannten Skelett und sah gerade noch, wie auch der einst grüne Schimmer des Bannpulvers um es leuchtend weiß wurde, ehe er sich vollständig verflüchtigte. Der Skelettkrieger beendete seinen langen Sprung durch die Küche. Seine Axt grub sich in die Wand, wo zuvor noch Zanyitaz gestanden war.

Der Schädel des Skeletts klapperte weiterhin nutzlos am Boden, während der Krieger ziellos seine Axt umherschwang und im Zimmer umhertorkelte. Brauchten Skelette etwa ihre leeren Augenhöhlen, um zu sehen? Faszinierend.

Doch diesmal war Barz vorbereitet. Er schleuderte die Phiole mit dem giftigen Cantharis dem Skelett entgegen. Während er zwar nie der allerbeste Schütze des Iquar-Stammes gewesen war, so hatte er auch noch nie sein Ziel vollkommen verfehlt. So kam es auch hier: Die Phiole traf das Skelett an der Schulter, zerbrach und setzte das Cantharis-Pulver frei, welches sich auf der Schulter des Skelettkriegers verteilte. Im Nu erschienen Verätzungserscheinungen auf seinem Schulterblatt und grünlich schimmernder Giftdampf rauchte hervor.

Zeitgleich mit dem Freisetzen des Cantharis-Dampfs begannen die drei großen blauen Cirathans-Körner in Barz‘ Hand zu magisch zu glänzen. Rasch kniete sich Barz neben Zanyitaz und presste die Pulverkörner in deren verletzte Schulter. Augenblicklich breiteten sich schwach fluoreszierende bläuliche Ranken von jedem Korn aus, mehrere Zentimeter lang über Zanyitaz‘ Schulterwunde und ins verwundete Fleisch hinein.

Im selben Masse, wie das Skelett durch das Gift geschwächt wurde, heilte Zanyitaz‘ Schulterwunde. Diese Verbindung von Giftpulver aus Cantharis und Stärkungspulver aus Cirathans war ein wunderbares Beispiel für das Gleichgewicht der Magie, welches Barz mit seinen Werken stets zu wahren versuchte.

Als Zanyitaz‘ Wunde größtenteils verkrustet war und sie erleichtert aufatmete, verdunkelten sich die farbigen Ranken der blauen Körner und fielen zu Boden. Barz packte den großen Kochtopf und schleuderte ihn ein letztes Mal auf das umhertorkelnde kopflose Skelett, welches zusammenbrach. Barz packte den Topf erneut und ließ ihn immer wieder auf das Skelett niedergehen, bis kein einziger seiner langen Knochen sich mehr rührte. Gruslige Viecher waren das, bäh!

Erst da wurde Barz sich bewusst, dass er soeben die sterblichen Überreste eines Menschen zerschlagen hatte. Das war eine Person mit Gefühlen und Wünschen gewesen, ein tapferer Krieger der Yetohe oder ein unter unglaublichen Widrigkeiten durchhaltender Sklave der Riesen. Barz legte seine offene Hand auf seine Brust, Handfläche nach außen, und sandte ein Stoßgebet an die Götter. Neben ihm tat Zanyitaz es ihm gleich.

Dann rannte Barz zwei Räume weiter, packte seinen Köcher und seinen Bogen und rannte nach draußen, um nach weiteren Skelettkriegern Ausschau zu halten.\bigskip







Kurz vor Sonnenaufgang. 14 Tage vor dem großen Unheil.\bigskip



Die Pfahlbausiedlung der Iquar war mitten im Ava gebaut worden, an einer Stelle, an der der Seegrund beinahe die Wasseroberfläche erreichte, sodass die Holzpfähle nicht allzu lange sein mussten. Es war in der Steppe schon schwer genug, stabile Holzstämme zu finden. Manche behaupteten gar, dass damals, als die ersten Pfähle der Siedlung gesetzt worden waren, die Siedlung noch am Ufer einer Insel inmitten eines damals über einen erheblich tieferen Wasserstand verfügt habenden Avas gelegen hatte. Damals, ehe der unterirdische Fluss, der den Ava speiste, plötzlich viel mehr Wasser geführt hatte.

Der springende Punkt sollte sein, dass nicht der ganze Ava so flach war wie unter der Siedlung der Iquar. Tatsächlich fiel der Seegrund rund um die Siedlung relativ rasch und tief ab. In diesen Tiefen lebten zahlreiche Lebewesen. Fische und Seetang, die von den Iquar gezüchtet wurden. Doch auch Nixen und Seerösser. Ja, manche erzählen gar von Riesenkraken und Spornwalen, auch wenn seit Jahrzehnten keiner von diesen mehr gesehen wurde. Doch uns geht es nicht um Riesenkraken oder Spornwale. Wir wollen uns lieber einer Unterwasserstadt der Nixen zuwenden.

Denn zeitgleich mit dem Angriff eines einzelnen Skelettkriegers auf Barz‘ Haus gingen, nur einige hundert Meter von Thakkum entfernt, ähnliche finstere Dinge vor sich.

Die Nixen des Avas standen in engem Kontakt mit den Bewohnern der Pfahlbausiedlung, wie es sich zwischen einander gut gesinnten Nachbarn gehörte. Und auch sie fanden die Anwesenheit einer fremden Skelettarmee vor ihren Ufern äußerst beunruhigend und hatten einige Wachen aufgestellt. Diese Wachen der Nixen patrouillierten oft die Ufer und sollten sofort Alarm schlagen, wenn plötzlich etwas Unvorhergesehenes geschehen würde.

Heute wurde Alarm geschlagen. Denn eine Nixe namens Jarna hatte ein totes Seeross erspäht. Die seltenen Seerösser galten den Nixen als sehr wertvoll, und einen Leichnam zu finden, sorgte immer für Unruhe. Doch besonders schlimm war, dass das Seeross unzweifelhaft an Stichwunden verstorben war. Welche Nixe würde es wagen? Die Antwort war klar: Keine Nixe. Knöcherne Fußspuren im Seeschlamm zeugten vom Übertäter. Ein Skelettkrieger der Krahder hatte sich unter den See gewagt! Natürlich, fiel es ihnen plötzlich wie Schuppen vom Fischschwanz: Skelettkrieger mussten nicht einmal immer wieder an der Wasseroberfläche Luft holen gehen, wie es die Nixen taten. Sie konnten theoretisch beliebig lange unter halb der Wasseroberfläche umherspazieren und Schaden anrichten.

Jarna und zwei weitere Wächterinnen jagten den Fußspuren nach. Sie zerlegten den einzelnen Skelettkrieger, welcher in den Ava vorgestoßen war, ohne große Mühe, und brachten seinen immer noch tapfer klappernden Schädel zu Ilfara, der Königin der Nixen.

Ilfara war natürlich außer sich.\bigskip







Sonnenhoch. 14 Tage vor dem großen Unheil.\bigskip



Das nächste Treffen der oberen Stammesmitglieder der Iquar war um einiges angespannter als das letzte. Glücklicherweise waren nur einige wenige, vereinzelte Skelettkrieger an verschiedenen Orten in Thakkum eingedrungen, kaum ein halbes Dutzend an der Zahl, und die meisten waren gar unschädlich gemacht worden, ehe sie ein Leben hatten nehmen können. Dies war kein Angriff auf die Siedlung gewesen, sondern nur eine Machtdemonstration. Ein Zeichen der Krahder, dass Thakkum nicht so sicher war, wie die Iquar es gerne hätten. Ein Zeichen, dass der so für Sicherheit gesorgt habende große See für ihre Skelettkrieger kaum mehr als ein leicht durchquerbares Sumpfland war. Ein Zeichen, dass sie sich besser bald ergeben sollten.

Nicht nur die menschlichen Bewohner des Sees litten plötzlich wieder mehr Kummer und Sorgen. Bei dieser Versammlung der oberen Stammesmitglieder waren nun auch Vertreter der Nixen anwesend. Eine Linguistin der Iquar kniete neben einigen Löchern am Holzboden. Hin und wieder tauchte ein Nixenkopf mit grimmiger Miene aus einem Loch herauf, um der Linguistin eine gluckernde Nachricht in der blubbernden Nixensprache zuzugurgeln, woraufhin die Linguistin heftig nickte und den weiter vorne liegenden Stammesmitgliedern etwas zuzischte. Barz zweifelte daran, dass auch die beste Linguistin je Nixisch aussprechen könnte, doch schien das auch nicht nötig, denn allem Anschein nach verstanden die Nixen die Barbaren bestens. Und der Lautstärke ihrer Beschwerden über die angreifenden Horden der Krahder nach würden viele Stammesmitgliedern wohl liebend gerne so tun, als verstünden sie weniger von den Nixen, als sie tatsächlich taten.

Das Nixenvolk sah die Schuld für diesen Angriff klar auf Thakkums Schultern. Barz teilte diese Meinung nicht zwingend. Klar, ohne die Pfahlbausiedlung hätten die Krahder wohl kaum Stellung um den Ava herum bezogen. Doch jetzt, wo die Krahder von ihrer aller wussten, könnten sich die Nixen doch nicht einmal eines zukünftigen Friedens sicher sein, wenn die Barbaren sich augenblicklich den Horden ergeben und in den Süden verschleppt werden würden. Nein, sie saßen alle im selben Boot, und sollten sich besser alle gemeinsam um diese Lage kümmern.

Durch den Angriff hatte sich die Stimmung der Obersten gegenüber der Gefahr gewandelt. So sprach Stammesleiterin Naquila gerade: „Wir haben uns geirrt. Der See schützt uns nicht auf ewig vor den Skeletten. Wir werden natürlich Vorsichtsmaßnahmen treffen. Wachen aufstellen und Patrouillen umhermarschieren lassen. Die Pfähle, das Fundament unserer Siedlung, mit Stacheln ausstatten oder aber sie so glitschig machen, dass kein Skelett je daran hochsteigen könnte.“

Unabhängig davon, ob Naquila tatsächlich an die Möglichkeit ihrer Worte glaubte oder nur Hoffnung verbreiten wollte, sprach sie rasch weiter, ehe eine der anwesenden Nixen über diese nur die Menschen schützenden Maßnahmen protestieren konnten.

„Doch wenn selbst die Seerösser der Nixen an den Skeletten zugrunde gehen, so können wir nicht darauf hoffen, auf ewig hier in Thakkum ausharren zu können. So sehr es mich wurmt, das zugeben zu müssen: Wir müssen evakuieren.“

Eine der Nixen hob ihre Hand, vermutlich um zu fragen, wie es den Nixen helfen sollte, wenn Thakkum evakuiert würde und sie in einem von Skeletten belagerten See übrig blieben. Dann senkte sie sie wieder. Vielleicht war ihr aufgefallen, dass es ohnehin so gut wie keine Möglichkeit zur Evakuation gab.

Dies schien auch Naquila einzusehen, als sie niedergeschlagen fortfuhr: „Doch wenn nicht einmal die Armee des Barbarenkönigs etwas gegen die Krahder und ihre Armee ausrichten kann, hege ich keine großen Hoffnungen für uns.“

Bei diesen Worten erhob sich der große Lifornus von seinem Sitz, breitete theatralisch seine Arme aus und sprach, als wäre sein Moment gekommen: „Nun, soweit ich weiß, hatte die Armee des Barbarenkönigs keinen Magier bei sich.“

Naquilas Miene verdüsterte sich, doch schien sie zu erschöpft, um eine bissige Bemerkung zurückzugeben. Auch Barz fragte sich, ob es Lifornus ins Hirn geschneit hatte. Wollte er etwa andeuten, dass seine Macht der dutzender gefallener Schamanen an den südlichen Fronten überlegen war?

„Was soll das? Habt ihr irgendwelche versteckte Kräfte, mit denen ihr die ganze feindliche Armee verjagen könnt?“, fragte die Dolmetscherin der Nixen spontan.

„Ich wünschte, es wäre so“, fuhr Lifornus melancholisch fort, „Doch verfüge ich über etwas fast gleichsam Hilfreiches: Die Macht, Kraft meiner Magie Hilfe zu rufen.“

„Um Hilfe wollt ihr rufen?“, meldete sich Stammesleiterin Naquila nun wieder, „Wer würde uns denn noch helfen wollen? Der Stamm der Yetohe, die womöglich bereits in ihrer Invasion Andors gescheitert sind oder durch sie zumindest unglaublich erschöpft wurden? Der Stamm der Jpaxo, der sich tatenlos im Gebirge verkrochen hält? Deine hochwohlgeborenen Zaubererfreunde aus dem hohen Norden, die sich einen feuchten...“

„Beruhigt Euch bitte, ich bin auf Eurer Seite. Lasst mich ausholen.“

Lifornus räusperte sich und erzählte mit einer Baritonstimme, die den besten reisenden Barden ansehend nicken lassen würde:

„Schon seit jeher brechen junge Zauberer in meiner Heimat Hadria zum Abschluss ihrer Ausbildung in die weite Welt auf. Viele kehren mit allerlei neugewonnenem Wissen zurück, manche gar mit seltenen Substanzen oder Erfindungen. Und einst gab es eine Zauberin, die bei ihrer Rückkehr aus dem Königreich Andor – welches damals noch nicht Andor hieß und auch noch kein Königreich war – von einem brennenden Vogel berichtete, einem sogenannten Phoenix.“

Lifornus legte eine Kunstpause ein.

„Dieser Phoenix war angeblich vor Generationen von einer Einsiedlerin gefunden worden, welche versucht hatte, ein unüberwindbares hohes Gebirge im Westen zu überwinden, dessen Gipfel stets in Wolken gehüllt waren. Manche munkelten gar, das Gebirge sei unendlich hoch, obwohl diese These anhand der endlichen Länge des Schattens des Gebirges natürlich unzweifelhaft widerlegt werden kann.

Doch zurück zu unserer bergsteigenden Einsiedlerin. Sie kehrte um, als sie dieses kurz vor dem Tode stehende magische Feuerwesen entdeckte, und nährte es, bis es zu einem stattlichen feurigen Vogel anwuchs, größer als jeder Hornfalke. Sie dachte, dass sie durch ihre Hilfe den Vogel vor dem sicheren Tod bewahrt hätte. Doch wie es sich herausstellte, ist der Tod für diesen Phoenix nicht dasselbe Ärgernis wie für uns. Wann immer sich der Lebenszyklus des Vogels seinem Ende näherte, verging er in Flammen und hinterließ ein kleines Feuerküken, welches von neuem heranwuchs.“

Eine Nixe blubberte etwas, und die Dolmetscherin übersetzte leise: „Magst du mal zum Punkt kommen?“

Etwas verärgert, doch nicht weniger pathetisch, fuhr Lifornus fort: „Nun, dieses Feuertier konnte nicht nur sein Leben immer wieder aufs Neue beginnen. Nein, er schlug nicht nur der Zeit, sondern auch dem Ort ein Schnippchen: Der Phoenix konnte sich auch von Ort zu Ort teleportieren, wenn er nur so wollte: Er würde an einem Ort in einem Flammenbausch verschwinden und an einem anderen in einem ebensolchen wieder erscheinen. Zurück blieb jeweils nur ein kleines Häuflein Asche. Unsere Zauberin, welche auf ihrer Reisen in Andor in Kontakt mit dem Phoenix gekommen war, brachte ein Säcklein dieser Phoenix-Asche nach Hadria zurück. Dort studierte sie es und schaffte es, mithilfe der Asche kleine Münzen zwischen Orten hin und her zu teleportieren. Daraufhin ließ sie immer mehr Phoenix-Asche von Andor nach Hadria transportieren. Es gelang ihr, mithilfe dieser Phoenix-Asche immer größere Objekte und später auch kleine Lebewesen zu teleportieren. Ihr Durchbruch kam, als sie aus der Asche einen Zauberspiegel der Teleportation zu formen vermochte.

Die Zauberin ist inzwischen schon lange verstorben, doch steht zu vermuten, dass der Phoenix noch immer irgendwo in Andor gehalten wird, weitergereicht von Generation zu Generation. Wenn es uns gelingt, den Phoenix hierher zu rufen und aus seiner Asche einen Zauberspiegel zu formen, so könnten wir alle auf dieser Siedlung festsitzenden Menschen in Sicherheit teleportieren. Und ich kenne ein Ritual, welches den Phoenix rufen könnte. Doch ich muss euch warnen! Es handelt sich um ein gefährliches Ritual, und niemand kann das volle Ausmaß der Konsequenzen einschätzen, sollte es schiefgehen.“

Es kam zu einer Abstimmung. Die knappe Mehrheit der Stammesleiter sprach sich für Lifornus‘ Ritual aus. Stammesleiterin Naquila war prominent dagegen, ebenso alle anwesenden Nixen, denen Naquila daraufhin plötzlich Stimmrechte zuschrieben wollte. Auch wenn nur wenige Nixen auf einmal durch die Löcher im Boden erscheinen und der Linguistin ihre Wünsche vortragen konnte, zeugte zahlreiches wütendes Plätschern, Blubbern und Gurgeln von unterhalb des Holzbodens von der Anwesenheit zahlreicher anderer Wasserbewohner unter der Versammlung. Barz schlich sich zur Linguistin und bat sie, den Nixen zu vermitteln, dass auch sie von diesem Teleportationsspiegel Gebrauch machen könnten und dass es auch andere, sicherere Gewässer für sie gab. Vielleicht irrte er sich, doch glaubte er, dass daraufhin das wütende Gluckern unterhalb des Bodens etwas nachließ.

Lifornus fuhr indes ungeachtet dessen fort: „Sehr schön, die Mehrheit hat wieder einmal die beste Entscheidung getroffen. Dann können wir uns ja an die Arbeit machen und diesen Phoenix herbeirufen! Doch lassen wir uns lieber genügend Zeit, dieses Ritual vorbereiten. Wisst Ihr, es ist besser, wenn ich mir einige Tage Zeit lasse mit dem genauen Ausformulieren der Zauberformel. Die Welt beugt sich dem Willen der Zauberer, und oftmals reicht ein starker Glaube in die Magie und ein Zauberstab, um unsere Kräfte wie gewünscht zu leiten. Doch für große, weitreichende Wirkungen kann es geschickter sein, die Kräfte der Zauberei direkt anzurufen. Nun, wir stellen es uns so vor, als ob wir mit ihnen reden würden, doch ob die Essenz der Magie tatsächlich ein Bewusstsein hat, ist eine ganz andere Frage...“

Naquila verdrehte bereits wieder ihre Augen und Lifornus unterbrach sich:

„Nun, um unseren Willen den Kräften der Zauberei zu kommunizieren, sprechen wir die uralte althadrische Sprache. Doch es ist manchmal vertrackt, die passenden Worte zu finden. Das Althadrische kennt zum Beispiel kein eigenes Wort für ‚Phoenix‘ und wenn ich die Mächte der Zauberei dazu aufrufen würde, mir einen ‚Vartum‘, also einen ‚Feuervogel‘, zu bringen, dann könnte es übel enden. Denn die Vokabel ‚Tum‘ steht sowohl für ‚Vogel‘ als auch für ‚Sturm‘. Sprachen sind nun mal kompliziert. Und wir wollen definitiv keinen ‚Feuersturm‘ in unserer Mitte beschwören. Lasst mich darum... oh, oh, wartet!“

Lifornus schlug sich an den Kopf – sein spitzer Zauberhut wackelte bedrohlich – und murmelte dann beschämt: „Mir fällt gerade auf, dass die Formulierung der Zauberformel unser geringstes Problem ist. Das Ritual bedarf verschiedenster Zutaten, die wichtigste darunter ein gewisses Kraut, welches gar wundersame Wirkungen hat. Es vermehrt sich in der Blütezeit rasch, eine einzelne Staude nährt einen gestandenen Zwerg für einen ganzen Tag, gar seltene Fieber vermag es zu heilen. Doch heimisch ist es in diesen Landen nicht, und manch ein Zauberer brach auf der Suche danach in den Süden auf, nur um mit leeren Händen zurückzukehren. Eine Zucht davon in Nordgard, was gäbe ich nicht dafür. Ich befürchte, dass ich mir selbst und auch euch ohne dieses Kraut nicht werde helfen können.“

Da erhob sich Barz. Das Auffinden seltener Pulver und Tränke, Steine und Pflanzen war schließlich sein Spezialgebiet. Sein Moment war gekommen. Und der seiner Schamanin, die wohl über die außergewöhnlichste Kräutersammlung diesseits des Grauen Gebirges verfügte.

„Noch ist nicht aller Tage Abend, lieber Lifornus. Meine Schamanin verfügt über die wohl außergewöhnliche Kräutersammlung diesseits des Grauen Gebirges und ich bin auf meinen Reisen in den Kontakt mit Kräuterkundler der verschiedensten Völker gekommen. Nennt uns den Namen.“

„Sternkraut. Die Pflanze heißt Sternkraut. Sie soll laut den Chroniken aus dem Oktron im Grauen Gebirge wachsen.“

Barz suchte unter den Anwesenden nach seiner Schamanin, doch diese antwortete auf seinen fragenden Blick nur mit betrübtem Kopfschütteln. Doch so rasch würde Barz nicht aufgeben.

„Dann werden wir weitersuchen. Boten können wir in dieser Zeit der Belagerung wohl kaum welche aussenden. Doch ich werde umgehend einen Brief an den Stamm der Jpaxo im südwestlichen Gebirge verfassen. Ich kenne eine ihrer erleuchteten Drachenkultisten ziemlich gut. Wenn das Kraut dort in der Nähe wächst, wird sie uns das mitteilen können.“

Lifornus nickte und lehnte sich zurück.

Die Stammesleiterin Naquila hingegen warf ein: „Mit Verlaub, Pulvermeister Barz, ich will uns diese Hoffnung nicht streitig machen, doch wie im Namen der Götter gedenkst du, einen Brief an unseren Gegnern vorbeizuschleusen?“

„Die Krahder und Skelette? Keiner von ihnen scheint einen funktionsträchtigen Bogen bei sich zu haben. Von ihnen droht einem Nachrichtenvogel keine Gefahr.“

„Halte mich nicht zum Narren, Barz, ich meine natürlich die Krarks. Diese Raubvögel machen schon seit langem unseren Himmel unsicher. Kein Falke wird durch die Steppe bis ins Gebirge reisen können, ohne von einem Krark angefallen zu werden.“

Barz grinste schelmisch: „Keiner hat gesagt, dass wir die Nachricht mit einem Falken übermitteln würden.“\bigskip







Sonnenhoch. 13 Tage vor dem großen Unheil.\bigskip



„Das ist Wahnsinn“, murmelte Zanyitaz.

„Das ist Hoffnung“, widersprach Barz. Er schüttete lieber noch ein bisschen mehr orangefarbenes Schlafpulver in den Ködersack und band diesen zu, so fest er konnte.

„Krarks können fast unbegrenzt lange im Himmel kreisen, wenn sie so wünschen, und so überqueren sie auf der Suche nach Nahrung riesige Gebiete. Doch ihre Nester liegen oftmals im Gebirge.“

„Da vorne fliegt einer!“, rief Yafka Barz zu. Barz unterbrach seine Erklärung und blickte durch das Fernrohr, das ihm seine Freundin entgegenhielt. Tatsächlich! Ein braunroter Fleck schwebte hoch oben über ihnen am hellblauen Himmel. Jetzt musste er nur noch näher kommen. Er bereitete seinen Bogen vor.

„Die Drachenkultisten der Jpaxo leben an vereinzelten Stellen im Gebirge. Sie versuchen hin und wieder, Krarks abzurichten, und manchmal gelingt es ihnen sogar“, meinte Barz an Zanyitaz gewandt, „Wenn wir einem Krark eine Nachricht ans Bein binden, so klein und so leicht, dass sie ihm selbst kaum auffallen wird, dann wird früher oder später ein Mitglied der Jpaxo es bemerken und den Brief an die gewünschte Kultistin weiterleiten. Und sie kann uns über einen brav abgerichteten Krark Sternkraut zurücksenden.“

Zanyitaz hatte diese Erklärung schon mindestens zweimal gehört, doch nickte sie brav, ehe sie bedrückt anfügte: „Sofern diese Kultistin überhaupt Sternkraut besitzt. Und sofern der Brief noch rechtzeitig bei ihr landet. Sofern unser Krark überhaupt in die Nähe der Jpaxo kommt.“

Yafka deutete auf den Stapel von kleinen, fein zusammengerollten Pergamentrollen, die neben ihnen lag: „Und wegen all dieser Eventualitäten senden wir ja nicht bloß eine Nachricht fort, sondern so viele wir können. Los, Barz, er ist tief genug!“

Barz spürte, wie sein Herz laut gegen seine Brust pochte, als er den Pfeil aus dem Köcher zog. Er spannte seinen Bogen und richtete den Pfeil auf eine Stelle oberhalb des Avas, doch weit unterhalb des Krarks. Nicht, dass der Vogel sich bei seinem Sturz noch verletzte.

Er atmete tief durch und ließ los. Sirrend zischte der Pfeil von der Bogensehne und zog den Ködersack hinter sich her. Dieser Sack war in Stahlfischöl getränkt worden, bis er so richtig nach Fisch stank, und noch dazu mit Barz‘ glitzerndem Ablenkungspulver bestreut. Kein Krark konnte einem hell funkelnden Edelstein widerstehen, irgendetwas daran schien sie wie magisch anzuziehen. Sie sammelten solche Klunker gerne in ihren Nestern. Vielleicht, um Brutpartner zu beeindrucken? Und da die Iquar keine Edelsteinsammlung gehabt hatten, hatte Barz sich mit seinem glitzernden Ablenkungspulver aushelfen müssen. Ob Geruch oder Geblinke, irgendetwas würde den Krark hoffentlich auf den Ködersack aufmerksam machen.

Der geschossene Ködersack vollführte eine schöne Parabel von der Pfahlbausiedlung Thakkum weit über den großen See Ava. Barz, Yafka und Zanyitaz blickten dem schwächer werdenden Glitzern wie gebannt hinterher, während es seinen Flug vollführte. Bestimmt ging es einigen anderen Bewohnern auf der Pfahlbausiedlung ähnlich.

Da! Der rotbraune Fleck am Himmel hatte urplötzlich seine Flügel zusammengezogen und ließ sich fallen. Beinahe pfeilschnell stürzte der Krark in Richtung des Sees, in Richtung des immer langsamer fliegenden Köders. Dann, als der Köder den Höhepunkt seiner Flugbahn erreicht hatte, erreichten ihn die Klauen des Krarks, scharf wie eine Schwertschneide. Das ferne Glitzern wurde kaum merklich größer und färbte sich leuchtend orange, als der Krark den Köder streifte und der Köder mit einem Knall zerbarst. Sein Inhalt verteilte sich in seiner Umgebung. Dieses Schlafpulver, gewonnen aus dem beruhigenden Gift der Steppensporne, löste seine Wirkung zwar nicht so schnell aus wie das grüne Bannpulver aus Mondbeeren, doch dafür würde der Krark nicht mitten in der Luft stehen bleiben.

Tatsächlich konnte der Krark dem Schlafpulver nicht lange widerstehen. Er flatterte noch einige Male ungelenk mit seinen Flügeln und stürzte dann der Wasseroberfläche entgegen.

Platsch.

„Jetzt!“, rief Barz, doch das wäre nicht nötig gewesen. Zwei, drei, nein, gar vier Nixen hatten bereits den sachte strampelnden Krark ergriffen und schleppten ihn in Richtung Thakkum. Auf halbem Wege nahmen ihn zwei Mitglieder der Iquar auf einem Boot entgegen. Sie würden eine von Barz‘ Nachrichten an seinem Bein befestigen und darauf hoffen, dass der Krark sie nicht bemerkte und entfernte, ehe er in die Nähe der Jpaxo erreichte.

Dies war nur der erste Krark gewesen, doch dass es so fabelhaft funktioniert hatte, erfüllte Barz mit Hoffnung.

Seine Vorräte an Schlafpulver waren bereits ziemlich limitiert, doch für ein halbes Dutzend weiterer Krarks sollten die Reste noch reichen. Hoffentlich würden diese genau so leicht zu ködern sein.\bigskip







Sonnenaufgang. Zwei Tage vor dem großen Unheil.\bigskip



Im Laufe der letzten beiden Wochen hatten die Krahder hin und wieder weitere Skelette in den Ava gesandt, doch inzwischen waren die Nixen vorgewarnt und hatten anrückende Übeltäter aufhalten können, ehe sie eine der Nixen-Städte oder Thakkum erreicht hätten. Irgendetwas schien die Riesen davon abzuhalten, einen vollen Angriff auf den See und die Pfahlbausiedlung zu starten, der einen signifikanten Teil ihrer Truppen aufs Los setzen würde. Fürchteten sie sich vor etwas? Oder war einfach kein Kriegsherr oder geborener Taktiker unter ihnen anwesend? Wie Barz Tag um Tag Prinz Ferntahr in der Gegend umherspazieren und die Landschaft und Bewohner der Steppe untersuchen sah, oder wie Barz immer wieder die Hexe Nahrack vor ihrem Zelt beim Rühren in finster rauchenden Kesseln beobachtete, da hatte er das Gefühl, dass die beiden Krahder diesem Konflikt nicht annähernd dieselbe Signifikanz zuordneten wie die Bewohner des großen Sees.

Wenn Barz an die wenigen Augenzeugenberichte aus dem finsteren Land der Krahder und an ihren von Lavaflüssen und Feuerseen übersäten Boden dachte, war ihm klar, dass diese Situation auch für die Krahder Neuland war, hatten sie sich doch zuvor wohl kaum um durchquerbare Flüsse und Seen gekümmert, ganz zu schweigen von Belagerungen. Aber selbst ein unerfahrener, doch zielstrebiger Kriegsherr hätte inzwischen wohl bereits gehandelt und Thakkum von Untoten überrennen lassen. Es folgte der Schluss, dass Ferntahr und Nahrack keinen Wert darauf legten, diese Belagerung rasch hinter sich zu bringen. Vielleicht war eine Reise in ein schöneres Land wie dieses ja gar eine angenehme Erfahrung für sie. Falls sie überhaupt einen Sinn für Schönheit besaßen.

Es machte auch Sinn, dass die Krahder, welche sich ursprünglich durch einen Einfall der Yetohe in ihr Reich zu einem mächtigen Gegenangriff provoziert gesehen hatten, nach der Flucht der Yetohe keinen Druck zum raschen Handeln spürten. Dass sie nur zwei Vertreter ihrer Spezies so tief in den Norden gesandt hatten, zeugte ja gerade von ihrer Unbekümmertheit... oder ihrer Furcht, es könnte sich hinter Thakkum eine Falle verstecken?

Was es auch war, dass die Krahder von einem raschen Angriff abhielt, Barz schlief etwas ruhiger im Wissen, dass die Krahder, mochte ihre Armee auch viel gefährlicher sein für Thakkum als ursprünglich angenommen, nicht handeln können würden, ohne dass im ganzen See von den Nixen Alarm geschlagen würde.

An diesem Tag wurde Barz‘ ruhiger Schlaf von einem Kratzen an seinem Fenster gestört.

Barz wachte im Glauben auf, sein Zimmer wäre leer, denn dies war eine der Nächte gewesen, die Yafka in Zanyitaz‘ Raum verbrachte. Umso überraschter war Barz also, als er bemerkte, dass er nicht alleine war.

Statt der Sonne glotzte ein riesiger Vogelkopf durch sein Fenster herein. Ein Krark! Ein ganz friedlicher, seinem Halsband nach zu urteilen. Bei den Göttern, wie hatte dieser nur sein Haus aufspüren können? Er hatte vom Intellekt dieser Raubvögel gehört, doch diese Zielfindung war ihm erst recht alles andere als geheuer.

Rasch, noch im Nachthemd, hatte Barz sein Fenster geöffnet. Der Krark streckte ihm brav sein Bein entgegen und erlaubte ihm, ein daran befestigtes türkises Säcklein abzunehmen. Es schien überraschend stabil, trotz seiner Leichtigkeit, und enthielt ein kompliziert zusammengefaltetes Stück Papier. Barz steckte das Säcklein grinsend ein. Er hatte schon so eine Ahnung, dass er eines Tages eine Verwendung dafür finden würde. Ein Horder zu sein, war für einen Pulvermeister keine zwingend schlechte Eigenschaft.

Barz hatte natürlich schon einen Krark so nahe gesehen, aber noch nie einen wachen. Schaudernd musterte Barz dessen Schnabel und dessen unterarmlangen, gebogenen Krallen, die fast so scharf waren wie die Schneide eines Schwerts. Der Vogel beobachtete Barz aus viel zu wachen, aufmerksamen Augen. Dies war ein intelligentes Wesen, wurde Barz bewusst.

Wartete er auf eine Belohnung? Vorsichtig strich Barz dem Krark über den Kopf, immer von oben nach unten, um sich nicht an den scharfen Federn zu verletzen – das hatte er beim Umgang mit den bewusstlosen Krarks rasch und schmerzhaft gelernt.

Als der Krark sich auch nach der Streicheleinheit nicht gerührt hatte, rannte Barz in die Küche (und beinahe Karyz über den Haufen), griff sich eine Handvoll Seetang und kehrte damit in sein Zimmer zurück. Der Krark schnappte sich den Tang aus Barz‘ Handfläche – sein Schnabel kratzte sie nur leicht, doch Barz zuckte umgehend zurück – und würgte ein bisschen am glitschigen Mahl, ehe er sich abwandte und wieder davonflatterte. Barz blickte ihm immer noch etwas verdattert hinterher.

Erst jetzt wandte sich Barz der Botschaft zu, die der Krark hinterlassen hatte. Mit vor Aufregung zitternden Fingern entfaltete er das Papier. Einige flach gepresste Kräuterbüschel fielen heraus. Ein kurzer Spruch aus nur drei Wörtern stand auf dem Papier:

„Mögen die Götter mit euch sein.

– Sagramak“\bigskip







Sonnenhoch. Der Tag des großen Unheils.\bigskip



Barz trat an den großen Ritualkreis. Zwei Runenzeichner der Iquar hatten ihn in den letzten Tagen sorgfältig auf ein Tuch gemalt und dieses über eine Lücke zwischen den Häusern Thakkums gespannt. Nur so konnte die benötigte Fläche erzielt werden, derart große freie Gebiete gab es in der Pfahlbausiedlung sonst nicht, und das von Skeletten patrouillierte Seeufer bot sich kaum als Ritualstandort an.

Der große Lifornus streckte gebieterisch seine Hand aus und Barz überreichte ihm das Büschel gepressten Sternkrauts, das er von Sagramak, einer Schamanin und Drachenkultistin der Jpaxo, erhalten hatte. Vermutlich nicht ansatzweise so theatralisch, wie Lifornus es gewollt hatte. Aber das sollte schon in Ordnung sein.

Lifornus räusperte sich und breitete bedacht seine Arme aus. Er steckte einen kleinen Teil des Sternkrauts in seine Tasche und hob das restliche Bündel so hoch er konnte in die Höhe. Mit geschlossenen Augen zerrieb er es energisch. Leise murmelte er dazu in einer fremden Sprache. Leichter Rauch stieg von seinen Handflächen auf. Dann entflammten die Kräuter. Lifornus warf sie elegant in die Mitte des Ritualkreises. Das Tuch war von Barz‘ Schamanin mit einem speziellen Mittelchen präpariert worden, auf dass es nicht selbst Feuer fing. Barz bemerkte sie, wie sie angespannt auf das Tuch starrte und hoffte, dass es lange genug durchhielt.

Lifornus betrachtete den Ritualkreis mit dem brennenden Sternkraut darin aufmerksam und nickte dann.

Episch rief er: „Narbi gicein, prento varafera! Dirqo on te bini earim!“

Plötzlich war ein fernes Grollen zu hören, wie von einem Gewitter. Lifornus zuckte zusammen. Auch die restlichen Beobachter sahen sich unruhig an.

„Narbi gicein, prento varafera! Dirqo on te bini earim!“

Ein Windstoß fuhr durch das Steppengras und über die Wasseroberfläche. Das Ritualtuch flatterte.

„Narbi gicein, prento varafera! Dirqo on te bini earim!“

Der Himmel wurde immer düsterer, und Lifornus schrie jetzt beinahe.

„Narbi gicein, prento varafera! Dirqo on te bini earim!“

Sekunden nachdem das letzte Wort verklungen war, ertönte aus der Ferne ein tiefes Brüllen, das den Boden erzittern ließ. Im gleichen Moment war im Westen kurz eine helle Flamme am Himmel zu sehen.

Barz blickte durch sein Fernrohr und erspähte eine Rauchfahne vom westlichen Gebirge aus aufsteigen.

„Dort drüben, seht!“

Die Geräusche waren das erste, was die Beobachter beunruhigte. Ein Knarren und Knirschen, ein Knacken und Bersten, als würde das Gebirge selbst in sich zusammenstürzen. Die Rauchfahne nahm an Größe zu und eine weitere helle Flamme stieß in den Himmel. Dann erklang ein tiefes Röhren, das nicht vom Berg stammen konnte.

Ein Lebewesen.

Ein äußerst lautes und äußerst großes Lebewesen.

„Ist das etwa dein Phoenix?“, fragte Naquila erbleicht.

„Nein, nein, das kann nicht sein“, murmelte Lifornus, „Die Zauberin war sich sicher darin, dass auch der Phoenix in seinem nie endenden Wachstum eine maximale Körpergröße hatte. Spätestens bei dieser würde er in Flammen vergehen. Und diese Größe konnte laut ihr von einem Krark bei weitem übertroffen werden. Was auch immer das für ein riesiges Wesen ist, welches im westlichen Gebirge tobt, es ist nicht der Phoenix!“

„Was genau habt ihr für eine Zauberformel verwendet?“, fragte Naquila, „Könnte das hier ein weiterer ‚vartum‘-Versprecher sein?“

„Ich hieß die Kräfte der Zauberei an, mir ein ‚varafera‘, ein Feuertier, hierher zu leiten.“, erklärte Lifornus aufgebracht, „Wie viele Feuertiere außer den Phoenix gibt es wohl schon in dieser Umgebung?“

Obwohl das zuvor unmöglich erschienen war, erbleichte Naquila noch mehr.

„Du Narr! Weißt du etwa nicht, wer in diesem Gebirge lebt?! Weißt du etwa nicht, welche Spezies die Kultisten des dritten Barbaren-Stammes anbeten?!“

Lifornus schüttelte zitternd seinen Kopf.

Erneut erklang ein tiefes Brüllen aus dem Gebirge. Eine gewaltige schwarze Pranke schwang hinter einer Bergspitze hervor und knallte darauf herunter. Die Bergspitze bröckelte und gab den Blick frei auf die schreckliche Echse, welche sich dahinter in die Höhe stemmte. Ein riesiges Geschöpf. Monströse, messerscharfe Schuppen bedeckten den gigantischen Körper. Riesenhafte Flügel ragten aus dem Rupf und der Kopf saß auf einem schlangenartigen Hals. Blutrot und gemein blickten seine Augen. Und diese glutroten Augen dieses Feuerdämons waren direkt auf den Ava gerichtet.

Direkt auf Thakkum.

Barz‘ Herz hüpfte in seine Hose.

Ein Drache.






















\newpage
\section{Der Angriff des Drachen}


Sonnenhoch. Der Tag des großen Unheils.\bigskip



Die Drachenkultisten der Jpaxo mussten gerade außer sich sein vor Freude. Seit Jahrzehnten hatte man am Ava nichts mehr von Drachensichtungen gehört. Barz fragte sich unweigerlich, ob wohl Nabib gerade irgendwo weit im Westen gen Himmel blickte und den Ursprung dieser schaurigen Geräusche zu ergründen versuchte.

„Das... das ist nicht möglich!“, stammelte Lifornus, „Die Drachen sind doch schon seit Jahrhunderten ausgestorben! Dieser Krieg mit den Zwergen und Riesen... die Aufzeichnungen im Oktron besagten eindeutig...“

Barz sandte ein Stoßgebet an die Götter. Er hatte schon von Drachen gehört und Zeichnungen gesehen, aber es war noch einmal eine ganz andere Angelegenheit, einen mit eigenen Augen zu erblicken.

Der Drache kraxelte über einen weiteren Berghang auf den Ava zu, den Kopf weiterhin direkt auf die Pfahlbausiedlung Thakkum gerichtet. Barz beobachtete die Lage durch sein Fernrohr und erkannte Bergziegen, welche verzweifelt aus dem Pfad des Drachen hüpften. Als er seinen Blick auf den wolkenverhangenen Himmel warf, konnte er keinen einzigen Krark mehr darin erblicken. Kluge Vögel.

Nun hatte der Drache ein letztes großes Gebirgsplateau erreicht, welches steil in die Steppe abfiel. Langsam entfalteten sich zerknitterte Flügel, welche vielleicht seit Jahrhunderten nicht mehr in Gebrauch gewesen waren. Der Drache reckte und streckte sich, schüttelte seine Schultern, tat einen gewaltigen Satz und warf seinen bestimmt tonnenschweren Körper in die Tiefe.

Majestätisch rauschte die Riesenechse über die trockenen Steppengräser, hin und wieder mit ihren riesigen Flederflügeln schlagend und sich selbst wieder etwas in die Höhe katapultieren. Der Wind heulte um sie herum und Barz sah, wie sich das dürre Steppengras dem Sturmwind hinter den Drachenflügeln ergab, ja, zum Teil gar durch orkangleiche Böen der Erde entrissen wurden.

Dann richtete das Wesen seinen schlangenartigen Hals in die Höhe und riss seinen Mund weit auf. Barz hörte etwas gluckern und sprühen. Dann spie der Drache einen gewaltigen Feuerstrahl. Die trockenen Büschchen und Gräser der Steppe brauchten kaum mehr als das, und schon standen sie lichterloh in Brand.

Da erklang eine Stimme in Barz Kopf, grollend und laut, so laut, dass er verzweifelt seine Hände an seine Ohren presste, ohne dass ihm dies Linderung schenken würde:

„WER WAGTE ES, EINEN URTITANEN AUS SEINEM SCHLUMMER ZU REISSEN?!“

Barz blickte um sich und sah erheblich viele weitere Bewohner der Iquar auf ihren Knien, mit den Händen über ihren Ohren. Yafka kauerte neben ihm auf dem Holzboden, die Zähne zusammengebissen. Wie ging es Zanyitaz und Karyz? Ein Glück, dass die beiden im Haus zurückgeblieben waren. Konnte diese schreckliche geistige Stimme auch dort eindringen?

„WER WAGTE ES, DIE VERKÖRPERUNG DES ZORNS BEFEHLIGEN ZU WOLLEN?!“, brüllte der Drache.

Der große Lifornus setzte zu einer neuen Zauberformel an, doch seine auf einmal krächzende Stimme gehorchte ihm nicht mehr.

Der Drache hatte nun den Ava erreicht, schwebte einige Haushöhen über dem Wasser, während er unweigerlich auf Thakkum zuhielt und das Wasser sich unter seinen Flügelschlägen teilte. Dann rauschte er blitzschnell über die ersten Häuser und schon befand er sich über dem Ritualkreis.

„WER WAGTE ES... TAROK ZU RUFEN?!“, erklang die schaurige Stimme des Drachen ein drittes Mal. Mit mächtigen Schlägen seiner gewaltigen Schwingen bremste Tarok seinen Anflug, während unter ihm Menschen umgeworfen und Holzbretter umgeweht wurden. Der hohe Zeitturm mit der 12-Uhr-Flagge daran rumorte und wackelte bedrohlich. Barz sprang schützend über Yafka und blickte furchtsam in die Höhe. Taroks rote Augen glühten auf ihn herunter. Der lange Schlangenhals richtete sich zu seiner vollen Größe auf und erneut erklang dieses schreckliche Gluckern daraus, welches Feuer und Tod versprach.

Barz riss Yafka mit sich, weg von der Holzsiedlung, nur hinein ins kühle Seewasser. Er sprang. Augenblicklich wurde sein Mantel durchnässt und schwer und zog ihn in die Tiefe. Barz versuchte, nicht an all die wertvollen Pulver zu denken, die er in den vielen Taschen und angesteckten Säcklein aufbewahrt hatte. All die Stunden, die er mit Mörser und Häcksel, Schalen und Gläsern, kühlendem Schnee und wärmenden Heizsand in der Hütte seiner Schamanin verbracht hatte, um die magischen Bestandmittel in genau der richtigen Reihenfolge und Menge zu mischen... Die meisten waren nun sicherlich verdorben. Aber das war aktuell nicht wichtig.

Tarok spie einen weiteren gewaltigen Feuerstrahl, vermutlich heißer als alle Feuer, mit denen Barz bislang experimentiert hatte. Freilich konnte Barz diesen bloß verschwommen von unterhalb der Wasseroberfläche erkennen, denn sein Mantel zog ihn immer noch in die Tiefe. Verzweifelt versuchte er, seine Stiefel abzustreifen und seinen Mantel zu lösen, aber zuerst musste er seinen Köcher lösen und seine Lunge schrie schon jetzt wieder nach Luft. Sein Gesichtsfeld wurde enger.

Da packen ihn aus dem Nichts rettende Arme, die geschwind seinen Köcher lösten und den Mantel abstreiften. Während sein Blickfeld sich weiter verdunkelte, sah Barz verschwommen unter ihm seinen Mantel, wie er auf den Grund des Avas sank. Dann durchbrach Barz‘ Kopf wieder die Wasseroberfläche. Er japste nach Luft und blickte sich um, halb in der Erwartung, eine Nixe habe ihn gerettet. Stattdessen tätschelte ihm Yafka besorgt die Wange.

Chaos herrschte. Taroks Anflug und seine Feuerstrahlen hatten ein mächtiges Loch in die Pfahlbausiedlung gerissen. Der hohe Zeitturm war niedergestürzt und hatte eine Bresche in eine Häuserreihe gerissen. Links und rechts um Barz und Yafka schwammen dutzende von Holzbalken und -planken, zum großen Teil noch brennend. Weiter vorne verbrannte das eigentlich feuerfeste Tuch mit dem Ritualkreis drauf soeben zu grauer Asche.

Zahlreiche Iquar, obschon eigentlich exzellente Schwimmer, strampelten voller Panik im See nach Luft. Und die zahlreichen Yetohe, von denen nur die wenigsten zu schwimmen vermochten, waren keinesfalls besser dran. Hier und dort tauchten die ersten helfenden Nixen auf, doch von vielen anderen sah man nur noch sich rasch vom Chaos entfernende Fischschwänze.

Barz trat Wasser, griff ziellos nach Holzstücken und zog sie drehend durch das Seewasser, um das Drachenfeuer auf ihnen zu löschen, ehe er sie verzweifelt strampelnden Menschen zuschob. Neben ihm tat Yafka es ihm nach.

„Wir müssen hier weg“, erklang die Stimme des großen Lifornus, eine Oktave höher als sonst. Der Zauberer hatte es irgendwie geschafft, auf der Siedlung im Trockenen zu bleiben, auch wenn er seinen spitzen Zaubererhut verloren hatte. Er besänftigte soeben mit großen Gesten ein loderndes Feuer an einer Seitenwand des Versammlungssaals.

„Wohin?! Willst du etwa ans ‚sichere‘ Ufer schwimmen?“, antwortete Stammesleiterin Naquila von irgendwoher, auch ihre Stimme um einiges schriller als gewohnt.

Barz blickte sich verzweifelt um. War die immer noch am Ufer wartende Skelettarmee der Krahder einem die Siedlung angreifenden Drachen vorzuziehen? Sein Blick richtete sich nach oben. Der Drache Tarok schwebte noch immer über der Siedlung und wirbelte mit seinen Flügeln Holz und Wassermassen, ja gar einige Menschen umher. Doch war sein glühend roter Blick nicht mehr auf den Ritualkreis oder die Siedlung gerichtet. Stattdessen zeigte sein Kopf auf seinem Schlangenhals direkt auf das Zelt der Krahder am Ufer.

Gerade kam Prinz Ferntahr aus dem Zelt gestürmt. Um ihn herum versammelten sich einige seiner menschlichen und zwergischen Gehilfen mit der eigenartigen schwarzen Bänderkleidung. Barz konnte ihre weiß bemalten Gesichter nicht genau erkennen, erst recht nicht jetzt, wo sein Fernrohr ihm abhanden gekommen war. Vermutlich dümpelte es friedlich auf dem Grund des Avas.

Dem hektischen Umhereilen Prinz Ferntahrs und seiner Gefolgsleute nach waren die Krahder ebenfalls nicht adäquat auf einen angreifenden Drachen vorbereitet. Sie hatten ja nicht einmal Bögen bei sich. Irrte Barz sich, oder sah er dort drüben drei Unfreie die Gelegenheit beim Schopf packen und zu fliehen versuchen? Ob vor der Knechtschaft der Krahder oder vor Taroks Zorn, war schwer zu sagen. Prinz Ferntahr fackelte jedenfalls nicht lange, sondern zeigte auf die Flüchtigen und brüllte etwas. Eine Ambacu an seiner Seite senkte ihre Arme in den Boden, blaues Licht umspülte sie und eine riesige, unförmige Knochengestalt mit mehreren Schädeln rannte den Flüchtigen nach. Der Rest der Skelettarmee trottete weiterhin ungestört das Ufer entlang.

Ein lautes Rauschen oberhalb von Barz brachte ihn wieder in den Moment zurück. Taroks Flügel schlugen noch mächtiger als zuvor. Wie durch ein Wunder entfernte sich die Echse von der Pfahlbausiedlung und wandte sich stattdessen den Krahdern zu. Mit Lifornus‘ Hilfe kletterte Barz aus dem Ava zurück in die Siedlung und half Yafka, dasselbe zu tun. Stammesleiterin Naquila war nirgendwo mehr zu sehen, dafür umso mehr Schreie von anderen Barbaren zu hören, die unter Feuer, Wind oder Holzsplittern gelitten hatten oder immer noch litten.

Barz, Yafka und ein immer noch ziemlich verdattert den davonfliegenden Tarok anstarrenden Lifornus machten sich daran, erste Hilfe zu leisten, wo sie konnten. Tun konnten sie allerdings nicht viel, ehe die geistige Stimme Taroks erneut erscholl, diesmal ein wenig dumpfer, vermutlich distanzbedingt:

„IHR RIESEN DACHTET WOHL, IHR KÖNNTET EUCH AN MIR VORBEISCHLEICHEN? IHR DACHTET, DER OLLE TAROK VERSCHLIEFE EURE DREISTE INVASION? LASST MICH EUCH ZEIGEN, WIE SEHR IHR EUCH GEIRRT HABT. IHR LAUFT DURCH MEIN GEBIRGE. MEIN REICH. MEIN DRACHENLAND. UND KEIN VERDAMMTER RIESE WIRD ES BETRETEN UND ÜBERLEBEN, SOLANGE MEIN BLUT NOCH FLÜSSIG IST!“

Dann hatte Tarok das Zelt der Krahder erreicht. Er ließ einen mächtigen Feuerschwall darauf niederregnen und setzte es in Brand. Prinz Ferntahr hechtete heldenhaft zur Seite und plantschte in den Ava. Verzweifelt versuchte der Riese, seinen massigen Körper ganz unter Wasser zu befördern. Barz hätte aufgelacht, wenn die Lage nicht auch für ihn und seine Liebsten so prekär gewesen wäre.

Da erlosch das Drachenfeuer um das Krahderzelt in einem Schlag. Das Zelt fiel in sich zusammen und löste sich in farbige Fussel auf. Die gen Boden sinkenden Tücher enthüllten die Krahder-Hexe Nahrack, umgeben von grün leuchtenden Fäden, Stränge reiner Magie. Sie zielte mit zwei Fingern auf Tarok und kreischte irgendeinen Krahder-Fluch. Tarok zuckte zusammen. Unterhalb von ihm kraxelten Skelettkrieger übereinander und aufeinander, bildeten einen riesigen Haufen, versuchten, Taroks in der Luft schwebende Pfoten zu erreichen.

Erneut ließ ein tiefes, rhythmisches Geräusch alle Anwesenden zusammenzucken. Barz‘ Bauch kribbelte unpassend wohlig und seine Mundwinkel zuckten nach oben. Er brauchte einige Momente, um das Geräusch und diese Emotionen zu verstehen.

Tarok lachte.

Tarok lachte von ganzem Herzen.

Dann ließ der Drache sich vom Himmel fallen und pulverisierte bei seinem Aufprall mit schierer Masse den Haufen Skelettkrieger, der ihn zu erreichen versucht hatte. Ein weiteres Mal ließ Tarok Feuer auf die Überreste des Krahderzelts niederregnen. Barz konnte nur hoffen, dass egal, was für Ressourcen Tarok auch für sein Feuer benötigte, diese irgendwann bald aufgebraucht wären.

Die Dunkle Hexe Nahrack kreischte auf, als das Drachenfeuer sie erreichte, und verging in einem Flammenwirbel. Tarok klappte seinen Mund zu. In der grauen Asche, die einst Nahrack gewesen war, regte sich etwas. Ein bläuliches Glühen flammte auf, als sich eine blasse Gestalt aus den schwelenden Überresten erhob.

„OH NEIN, HEUTE NICHT!“, brüllte Tarok und trampelte mit seinen Pranken auf dem Boden herum, bis kein bläuliches Schimmern und keine blasse Gestalt mehr zu sehen waren. Dann wandte sich sein Schlangenhals dem Prinzen zu. Ferntahr war inzwischen weiter in den Ava hineingewatet und senkte gerade in dem Augenblick seinen kleinen Kopf unter die Wasseroberfläche, als Taroks nächster Flammenstrahl ihn erreichte.

Das Wasser um ihn herum begann zu kochen.

Ehe Ferntahr wieder zum Atmen auftauchen musste, hatten die an Tarok hochkletternden Skelettkrieger seine Flügel erreicht und begannen, mit ihren Schwerten und Äxten hineinzustechen. Leider schien sie das Ableben der Dunklen Hexe nicht in ihrer Aktivität einzuschränken.

Tarok brüllte auf, diesmal physisch, nicht geistig, und erhob sich wieder in die Lüfte. Er schien widerspenstig. Dass diese kleinen Wesen ihm tatsächlich schaden konnten, musste an seinem Selbstbild rütteln. Tarok drehte eine Pirouette, die auch die letzten Skelette von ihm abschüttelte und zum Teil bis weit in die Barbarensteppe oder den Ava schleuderte. Er legte seine Flügel an und schoss das Ufer des Avas entlang. Wie in einem wilden Wahn wütete Tarok durch die Reihen der Skelettkrieger, biss und kratzte, zermalmte und schleuderte Überreste weit, weit weg.

Prinz Ferntahr tauchte gerade rechtzeitig wieder aus dem Ava auf, um zu sehen, wie drei weitere seiner Ambacus das Weite suchten. Diesmal war niemand an seiner Seite, denen er befehlen konnte, sie aufzuhalten.

Barz wandte seine Aufmerksamkeit von diesen fesselnden Geschehnissen ab und wieder seiner Umgebung zu. Inzwischen hatten einige Heiler und Schamanen das Loch erreicht, welches Tarok in die Pfahlbausiedlung Thakkum gerissen hatte. Die meisten im Wasser strampelnden Barbaren waren ins Trockene gebracht worden, die meisten Brandherde eingedämmt und die wenigen Skelettkrieger, deren Überreste in die Nähe der Siedlung geschleudert worden waren, wurden rasch beseitig. Doch noch immer befand sich eine zornige Riesenechse gefährlich nahe. Gerade war Tarok damit beschäftigt, die letzten ordentlichen Reihen der Skelettkrieger um den Ava zu Knochenstaub zu zerschlagen, doch wenn er sich danach entscheiden sollte, dass Thakkum wie eine ideale Lagerfeuerstätte aussah, würde dies das Ende von dessen Bewohnern sein. Was tun, was tun?

Im Geiste ging Barz die wenigen Mittelchen durch, von denen er immer noch Vorräte in seinem Haus hatte.

Das grüne Bannpulver aus getrockneten Mondbeeren war sehr nützlich gegen kleinere Gegner, doch gegen jemanden dieser Größe und Stärke konnte es kaum etwas anrichten. Wenn Barz mehr als nur eine Tatze Taroks bannen wollte, bräuchte er Unmengen dieses Materials. Mehr, als es aktuell wohl auf der gesamten Welt gab.

Fürs orangefarbene Schlafpulver aus dem Schlafgift der Steppensporne galt das genau gleiche Problem. Zu großes Ziel, zu kleine Wirkung. Ganz zu schweigen davon, dass Barz seine letzten Vorräte davon vor zwei Wochen beim Anlocken der Krarks verbraucht hatte.

Das braune Meditationspulver aus Krarkfedernasche könnte Barz sicherlich beruhigen, ihm vielleicht gar etwas Zeit zum weiteren Überlegen schenken, doch mehr nicht.

Das graue Nebelpulver aus dem fernen Narkon, welches... nein, Barz besaß doch gar kein... Barz blieb einige Sekunden an einem Gedanken hängen, ehe dieser sich verflüchtigte. Worüber hatte er gerade nachgedacht? Egal, wichtiger als ein angreifender Drache konnte es kaum sein.

Das goldene Umwandlungspulver war Barz neuste Kreation, mit welcher er bereits ziemlich routiniert gewisse Alltagsgegenstände beschwören konnte, so etwa die Fernrohre, die in den letzten Tagen die Krahder ausspioniert hatte. Aber nur gewisse Gegenstände. Und nur manchmal. Noch war Barz sich nicht sicher, ob er diese Gegenstände tatsächlich erschuf, oder ob er sie nur aus z.B. einem nahe gelegenen Ausrüstungslager der Yetohe zu sich teleportierte. Er hatte bereits einige Briefe in Fernrohre umgewandelt, um zu gucken, ob plötzlich Falken mit Antworten aufkreuzen würden, doch noch war dies nicht der Fall gewesen. Aktuell hatte Barz noch nicht einmal herausgefunden, dass er mit dem Umwandlungspulver auch lebendige Falken ins Spiel bringen konnte. Faszinierendes Zeug, dieses Umwandlungspulver, aber dieser Drache brauchte mehr als ein paar frische Bögen, um erlegt zu werden.

Das silbern glitzernde Ablenkungspulver war vor zwei Wochen auch ganz verbraucht worden beim Anlocken der Krarks. Ganz zu schweigen davon, dass sich auch Tarok kaum davon ablenken lassen würde.

Dann war da nur noch das bläuliche Schwächungspulver, welches Barz erst kürzlich aus grünem Cantharis-Giftpulver und tiefblauen Cirathans-Körnern zu einem einzigen Pulver statt einer Zwei-Komponenten-Verbindung hatte optimieren können. Doch auch dieses würde dem Drachen wohl kaum mehr als eine Schuppe wegätzen können, und das kleine bisschen Heilkraft, welches es einem Verwundeten verleihen könnten, würde wohl kaum einen signifikanten Unterschied mehr machen.

Barz schüttelte resigniert seinen Kopf. Diese Überlegungen brachten ihm nichts. Gegen Tarok konnte er nichts anrichten. Dafür brauchte es größere Konstruktionen, so etwas wie riesige Katapulte. Oder mächtigere Magie. Doch der einzige Magier hier, der die Kraft oder Kenntnisse dafür besitzen konnte, machte soeben durch hektisches Herumgefuchtel mit seinem Zauberstab deutlich klar, dass er nichts ausrichten konnte. Barz ließ sich auf die knarzenden Planken sinken und betete zu den Göttern. Doch nur kurz, denn dann riss Yafka ihn wieder in die Höhe und deutete auf zahlreiche weitere Hilfesuchende in ihrer Umgebung. Barz machte sich auf, Heilkräuter aus der Hütte seiner Schamanin zu holen. Um Tarok würden sie sich sorgen, wenn es an der Zeit war.

„IHR MACHT ES MIR SCHON FAST ZU EINFACH!“, erscholl Taroks dumpfe Stimme ein weiteres Mal in seinem Kopf. Diesmal hatte Barz sich schon beinahe daran gewöhnt und stolperte nur kurz, ehe er weiterrannte. Er hätte schwören können, dass die Stimme einen amüsierten Unterton gehabt hatte. Da er aktuell zwischen zwei Hausreihen durchpreschte, war sein Blick darauf versperrt, was mit dem Drachen am Ufer abging. Barz lieferte die drei Büschel Heilkräuter, die er in seinen Hosenbund gestopft hatte, der nächsten Heilerin ab, rannte zwischen zwei Häuser und versuchte, durch die Lücke zu erspähen, was Tarok derart amüsierte.

Da, ganz in der Ferne, glaubte Barz einen weiteren Feuerstrahl zu erkennen. Er kniff seine Augen zusammen und kapierte plötzlich. Die Armee der Skelettkrieger war noch nicht vollständig an den Ava angetreten. Noch immer wälzte sich ein Wurm von Skelettkriegern aus dem Land der Krahder durch die Steppe der Barbaren an den Ava. Wie eine lange dünne Linie zog sich die feindliche Armee durch die Steppe und durch die südlichen Berge... vermutlich bis in ihre finstere Heimat. Ein Anblick, der Barz bereits seit einiger Zeit Albträume bereitet hatte – hatte diese langsam anrückende Armee denn wahrlich kein Ende? Doch nun musste er grinsen. Der Wurm von Skeletten konnte auch eine Lunte sein. Nun flog Tarok rasch und gezielt die Reihen der anrückenden Skelette an und setzte sie eines nach dem anderen in Brand. Und er lachte. Oh, wie er lachte. Das alles musste Erinnerungen zurückbringen an Jahrhunderte lang zurückliegende Kriege, in denen Tarok Seite an Seite mit anderen Drachen gekämpft und gesiegt hatte. Der Drache schwelgte kurz unbefreit in blutiger Nostalgie, ehe ihn seine Trauer wieder einholte.

Barz verdrückte eine Träne.

Dann brach der Strom fremder Emotionen abrupt ab und Barz besann sich wieder auf sich selbst. Was für einen außergewöhnlichen Geist diese Drachen doch hatten, dass sie andere an ihren Gedanken und Gefühlen teilhaben lassen konnten, ob diese es wollten oder nicht. Oder ob sie es selbst wollten oder nicht.

Nun, im Kontext von Taroks Erinnerungsfetzen, sah Barz ein, dass sie großes Glück im Unheil gehabt hatten, welches Lifornus‘ Ritual ausgelöst hatte. Wenn Barz die verwirrenden Eindrücke des Drachen aus der Zeit des unterirdischen Krieges richtig einschätzen konnte, so waren die Drachen und Krahder zutiefst verfeindet. Kaum etwas anderes als eine Krahderarmee hätte Tarok von Thakkum ablenken können. Und kaum etwas anderes als ein Drache hätte die Krahder zur Flucht zwingen können. Das Schicksal konnte einen miesen Sinn für Humor haben.

Weiter vorne sah Barz den Krahderprinzen Ferntahr vor mit Speeren und Dreizacken bewaffneten Nixen aus dem Wasser fliehen und sich fassungslos umsehen. Hier und da krabbelten und hüpften noch einige seiner Skelettkrieger mit fehlenden Gliedmaßen umher, dort drüben standen gar noch zwei zitternde Ambacus und warteten auf Ferntahrs Befehle. Doch was einst rund um den Ava herum eine schier unüberwindbare Armee aus wandelnden Leichnamen und Knochen-Golems gewesen war, war nun größtenteils bloß noch ein desorientierter Haufen aus Knochensplittern und teils verrosteten, teils geschmolzenen Waffen. Und die Linie des Skelett-Nachschubs, die sich vom Ava weit in den Süden zog, wurde soeben von einem mächtigen Drachen abgeflogen, der keinen einzigen Grashalm im Schatten seiner Flügel stehen ließ, und erst recht keinen Skelettkrieger.

Ferntahr brüllte panisch auf und suchte das Weite in der weiten Barbarensteppe.

Barz jubelte innerlich auf.\bigskip







Kurz nach Sonnenhoch. Der Tag des großen Unheils.\bigskip



Barz‘ Boot fuhr hart aufs Seeufer auf, doch Barz kümmerte sich nicht darum. Er schwang sich über die Bordwand, zog einen Pfeil mit einer rosa Spitze aus seinem Köcher und versenkte ihn im Schädel eines Skelettkriegers, welcher nach ihm zu grabschen versuchte. Neben ihm stach Yafka energisch mit ihrem Schwert auf einige sich schwach regende Beinknochen ein.

Eine Hitzewelle rauschte an Barz vorbei. Der große Lifornus trat an Barz‘ Seite, drei weitere Feuerkugeln seinen Zauberstab umkreisend.

„Na, das hat ja wie gewünscht geklappt, nicht?“, rief Barz Lifornus zu. Dieser unterdrückte ein Schluchzen, riss sich dann zusammen und sprach: „Rumuno kerkum!“

Zwei weitere Skelettkrieger ohne Arme und ohne Waffen wurden in ein nahestehendes Drachenfeuer geschubst und fielen in sich zusammen.

Hinter Barz, Yafka und Lifornus kletterten weitere Kämpfer aus Booten ans Seeufer und machten sich ebenfalls daran, die wenigen „überlebenden“ Überreste von Skelettkriegern gekonnt zu zerteilen. Wenn sie weiterhin solche Fortschritte machen könnten, würde der Ava noch vor Sonnenuntergang wieder völlig von Feinden befreit sein. Doch mussten sie schnell handeln, ehe die Überbleibsel der Skelettkrieger sich zu einer Horde zusammenrotten und einen gezielten Gegenangriff starten konnten. Falls sie nach Taroks Angriff und dem Tod ihrer Dunklen Hexe überhaupt noch dazu in der Lage waren.

Der Drache Tarok war inzwischen hinter dem südlichen Gebirge verschwunden und hatte eine Spur aus brennendem Steppengras und eingeäscherten Skeletten hinterlassen. Barz blickte immer wieder sorgenvoll in diese Richtung. Es war immer noch möglich, dass Tarok plötzlich genug davon hatte, die Armee seiner uralten Feinde zu zertrümmern, und sich wieder denen zuwenden wollte, die es gewagt hatten, ihn zu rufen. Dann müsste Thakkum so schnell wie möglich evakuiert werden. Es schauderte ihn beim Gedanken an seine Familie und Freunde. Zumindest waren Zanyitaz und Karyz aktuell in Sicherheit. Ehe Yafka sich Barz beim Landgang angeschlossen hatte, hatte sie ihm mitgeteilt, dass Zanyitaz sich einige andere Fischer, Karyz und einige weitere Kinder geschnappt hatte und mit diesen auf einem Fischerboot in den See gestochen waren. Dort draußen glaubten sie sich aktuell sicherer als in den brennbaren Pfahlbauten Thakkums. Und Barz stimmte ihnen zu.

Zum ersten Mal seit dem Anfang von Taroks Angriff erlaubte sich Barz einen Moment des Verschnaufens. Sein Blick driftete locker über seine Umgebung. Viele Skelettkrieger waren nicht mehr übrig. Doch dann erschrak er, als ihm bewusst wurde, dass die größte Gefahr nicht mehr von der Armee der Krahder ausging.

Die Steppe brannte lichterloh.

Kleine Flicken des trockenen Grases, welches Tarok in Brand gesteckt hatte, waren zu großen Flammenwirbeln geworden, deren Rauchsäulen weit in den Himmel reichten.

Sowohl als Bewohner einer Holzsiedlung als auch als Bereisender einer teils trockenen Steppe waren Barz die Gefahren des Feuers oft genug eingebläut worden, damit er großen Respekt vor ihnen hatte. Es gab einen Grund, warum sich die Schamanen der meisten Yetohe-Sippen aus Prinzip weigerten, mit Feuergeistern in Kontakt zu treten.

Und nun brannte die Steppe lichterloh, und es gab nicht wirklich etwas, das sie tun könnten. Vermutlich hätten zu diesem Zeitpunkt nicht einmal mehr die legendären Löschzwerge aus dem Westen groß etwas ausrichten können.

Im ersten Augenblick dachte Barz nur erleichtert daran, dass die meisten Steppenbewohner zu ihrer verzweifelten Invasion Andors aufgebrochen waren. Dort hatten sie wenigstens die Chance, von ihren Gegnern verschont zu werden. Das Feuer war hingegen gnadenlos.

Im zweiten Augenblick war Barz erleichtert darüber, dass die zurückgebliebenen Yetohe mitten im Ava sicher sein sollten vor der Feuerhölle, welche dort draußen herrschte.

Im dritten Augenblick freute sich Barz gar darüber, dass Taroks Rückkehr umso unwahrscheinlicher war, wenn die halbe Steppe seinetwegen brannte. Was wollte er noch schlimmer machen?

Erst im vierten Augenblick erinnerte sich Barz an die weiteren Bewohner der Steppe. Die ganzen nichtmenschlichen Tiere, welche aktuell vor dem Flammenmeer wegzurennen versuchen mussten. Die Büffel und die Hornbären. Die Einhörner und die Steppensporne. Die Flederkatzen und allgemein alle anderen Totemtiere.

Und dann gab es auch noch diejenigen, die sicherlich nicht schnell genug waren, um dem Feuer davonzurennen.

Die Steppenechsen.

„Jirisa!“

Noch während Yafka ihren Blick auf Barz richtete und zu verstehen versuchte, was er mit seinem Ausruf gemeint haben könnte, rannte Barz los. Ehe die Skelettkrieger der Krahder den Ava erreicht hatten, waren die wenigen Steppenechsen der Yetohe, die nicht als Reit-Echsen beim Angriff auf Andor zum Einsatz gekommen waren, feierlich freigelassen worden. Doch bewegten sich Steppenechsen nur selten, wenn man sie nicht dazu drängte. Und die Skelettkrieger der Krahder hatten die Echsen in Ruhe gelassen, das hatte Barz bei ihrer Ankunft überrascht beobachtet. Somit stand es zu vermuten, dass einige Steppenechsen sich immer noch in der Nähe ihrer einstigen Ställe, Felder und Nester aufhielten, wenige hundert Meter von hier entfernt. Und mit ihnen auch Barz‘ Steppenechse Jirisa.

Ein althadrischer Spruch ertönte hinter Barz. Vor ihm teilte sich das Drachenfeuer kurzzeitig. Er rief Lifornus ein Danke zu, hechtete durch die Feuerlücke und erreichte die ersten Stallungen der Steppenechsen. Diese standen bereits in Flammen. Der Haupteingang war gar in sich zusammengestürzt. Hitze und Rauchen schlug Barz entgegen. Schweiß rann seine Stirn hinunter und die feurige Luft kratzte in seinem Hals. Doch davon ließ er sich nicht aufhalten.

Zwei mächtige Steppenechsen standen in den Stallungen und röhrten panisch. Barz öffnete den Türverschluss der Hintertür und winkte sie ins Freie. Hinter sich vernahm er erneut Lifornus‘ tiefe Stimme einen Zauberspruch sprechen. Ein kleiner Spalt öffnete sich im Drachenfeuer, durch den gerade noch so das Blau des rettenden großen Sees Ava erblickt werden konnte. Barz klatschte den an ihm vorbeieilenden Steppenechsen an die Seiten, in der Hoffnung, sie noch etwas anzutreiben. Keine von ihnen war Jirisa.

Barz eilte weiter ins von Brandherden übersäte Feld, auf denen die Steppenechsen üblicherweise rasteten und grasten. Anstelle des Himmels sah er bloß noch schwarze Rauchschwaden über sich. Dort drüben lag der kokelnde schwarze Kadaver einer Steppenechse, an dessen Flanke immer noch Taroks Flammen leckten. Hier vorne sah er zwei weitere massige Echsen davonrennen, ebenfalls in Richtung des Seeufers. Zumindest hoffte Barz das. Es wurde immer schwerer, sich zu orientieren. Er spielte mit dem Gedanken, ebenfalls ans Ufer zurückzukehren. Hier gab es nicht viel, was er tun konnte, außer sein Leben aufs Spiel zu setzen.

Doch zumindest an einer letzten Stelle wollte Barz noch nach Jirisa oder weiteren Steppenechsen sehen, ehe er guten Herzens umkehren konnte. Die Nester der Steppenechsen lagen in der Nähe der Ställe. Eine kleine Senke, in der die Echsen unter guten Umständen getrocknetes Gras zusammenscharren, ihre Eier legen und diese wärmen konnten, bis ihr Nachwuchs schlüpfte. Barz musste schwer schlucken, als ihm bewusst wurde, wie brennbar diese Nester waren. Und wie die treuen Steppenechsen-Mütter ihren zukünftigen Nachwuchs dennoch kaum aufgeben würden.

Barz hechtete über eine Anhöhe und wischte sich das schweißnasse Gesicht ab. Er hustete und hielt seinen Kopf gesenkt, um dem hitzigen Rauch zu entgehen, doch das half kaum, wenn überall um ihn herum Brandquellen in den Himmel rauchten. Er blinzelte und versuchte, die Nester der Steppenechsen zu erspähen. Glücklicherweise waren die meisten leer. Doch wie befürchtet erkannte Barz am unteren Ende des Abhangs eine schwarz verfärbte Steppenechse, welche auf einem brennenden Flecken Steppengrases zusammengebrochen war. Noch war sie am Leben, und noch streckte sie ihren Kopf verzweifelt zu einem lichterloh brennenden Strohnests, in deren Mitte Barz drei schimmernde große Eier erblickte.

Jirisa hatte schon seit einigen Jahren keine Eier mehr gelegt, versuchte Barz sich einzureden, wie groß war da schon die Wahrscheinlichkeit, dass sie gleich bei ihrer Rückkehr an den See welche gelegt hätte? Doch er glaubte sich selbst nicht. Barz wusste, dass bei viel umherreisenden Steppenechsen das Legen von Eiern vertagt werden konnte, bis die Echsen eine Zeit lang am selben Ort verbrachten. Und dass sie nicht zwingend das beste Gespür dafür hatten, ob die Zeit vor der Winterstarre fürs Schlüpfen ihres Nachwuchses reichen würde.

Barz schlidderte den Abhang herunter und sein Herz sank erst recht in seine Hose, als die Steppenechse ihn erblickte, laut röhrte und ihr Gesicht ihm zuwandte. Diese Stimme kam ihm derart bekannt vor und diesen schiefstehenden Fangzahn in ihrem Unterkiefer hätte er im Schlaf erkannt. Das war Jirisa.

Jirisa röhrte ein weiteres Mal auf und stemmte ihren massigen Körper mit zitternden Beinen in die Höhe. Sie schien Barz nun auch erkannt zu haben. Drei Schritte weit kam sie auf ihn zu, ehe sie wieder zu Boden plumpste und schwach brummte. Die Flammen des brennenden Nests leckten weiterhin an ihrem Schwanz, doch zumindest war der Rest von ihr nun frei. Barz stürzte auf sie zu und streichelte ihre raue, viel zu warme Stirn.

„Du Dussel!“, fluchte er, „Du treuherziger Dussel, hast du etwa versucht, die Eier aus dem Feuer zu retten?“

Ein dumpfes Tröten drang aus Jirisas Kehle. Es klang beruhigend. Barz konnte nicht sagen, ob sie sich selbst oder ihn zu beruhigen versuchte. Bildete er sich ein, oder wirkte die Echse entspannter? Barz hatte ihr schon so oft geholfen, als sie sich einen Dorn in den Fuß getreten hatte, als sie in einem Sumpf stecken geblieben war, als sie sich im Netz einer Steppensporne verfangen hatte. Glaubte sie, dass Barz die Lage noch zum Guten wenden konnte? Verband sie Barz mit Gefühlen der Sicherheit und Geborgenheit? Gefühlen, denen er nun einfach nicht gerecht werden konnte?

Barz blickte an Jirisa herunter. Erschöpft und verbrannt, wie die Echse war, würde sie kaum mehr laufen können, und Barz konnte sie nicht transportieren. Er stand auf und versuchte, das Feuer hinter der Echse auszustampfen. Nicht einmal das gelang ihm vollständig. Er tastete seinen Mantel nach Heilmitteln ab, ehe ihm einfiel, dass dieser weiterhin versunken auf dem Grund des Avas lag. Er griff in seinen Gürtel nach Heilkräutern, doch diese hatte er einer Heilerin abgegeben, ehe er ans Ufer geschifft war.

Barz blickte verzweifelt zum Nest mit Jirisas drei weiterhin brennenden Eiern. Diese mussten inzwischen völlig gebraten sein. Auch dort konnte Barz nicht mehr helfen. Er verfluchte das Feuer, er verfluchte Tarok und er verfluchte den großen Lifornus, der Tarok gerufen hatte. Dann fiel sein Blick auf ein viertes, einzelnes Ei, welches ein wenig abseits des brennenden Nests lag. An einer Steppengras- und damit auch feuerfreien Stelle. Ein einzelnes dunkles Ei, in dessen schimmernden Schale sich das Flammeninferno seiner Umgebung spiegelte. Jirisa musste es rechtzeitig aus dem Nest bugsiert haben. Es war unversehrt.

Der Abstand zum Ei war in Windeseile überwunden. Es war klein für ein Steppenechsen-Ei, und doch größer als Barz Hand. Vorsichtig hob er es in die Höhe und sah davon sein eigenes verzerrtes Gesicht zurückspiegeln, ein dunkler Schatten vor rotem Hintergrund. Er hastete zurück zu Jirisa, kniete neben seiner langjährigen Begleiterin auf den heißen Grund und streichelte sie ein letztes Mal, während er ihr versichere:

„Oh Jirisa, es tut mir so leid, so leid, dass ich zu spät gekommen bin. Ich... ich kann nichts mehr für dich tun, alte Freundin. Oh, Jirisa. Tapfer hast du uns lange Zeit begleitet, und tapfer hast du versucht, deine Nachkommen vor diesem Sturm zu bewahren, der auch meine Schuld ist. Und zumindest eines davon wird dank dir fortbestehen können. Dein letztes Ei werde ich mit meinem Leben hüten, das verspreche ich dir. Aus diesem Ei wird eine stattliche Echse wachsen, und ich werde sie hegen und pflegen, so es in meiner Macht stehe. Ich werde sie... Sabri nennen. Kannst du sie dir vorstellen? Wie sie über die weiten Weiden der Steppe ziehen wird, stoisch, störrisch und stark? Das wird sie von dir haben. Oh, Jirisa, erinnere dich nicht an diese Flammen und die Hitze. Denke zurück an die guten Zeiten, an saftiges Steppengras, fliegende Fische, an das erquickende Wasser des Avas. So stelle ich mir das gute Leben nach dem Tode vor, und so eines hast du definitiv verdient. Bald wirst du dort sein und mit den anderen Steppenechsen frei herumtraben können. Zuvor musst du noch durch dieses Leiden durchstehen, doch danach... “

Ein Hustenanfall unterbrach Barz‘ Monolog. Er wischte sich die von der Hitze schweißnasse Stirn ab. Seine Hand zitterte und sah unschön rot aus. Doch noch ging Jirisas rasselnder Atem, und so kuschelte sich Barz erneut an seine Steppenechse und fuhr fort, redete von der unendlichen Steppe und den farbigen Flüssen im Paradies der Götter, auch wenn er im Hinterkopf ahnte, dass er damit vermutlich mehr sich selbst beruhigte als die Echse. Auch wenn er im Hinterkopf wusste, dass im Moment um ihn herum noch einiges Schlimmeres geschah, mit dem er sich später befassen musste..

Doch nicht jetzt.

Dieser Moment galt ihm und Jirisa.

Ein letztes Mal leckte ihre raue Zunge Barz‘ wunde Hände.

Ein letztes Mal atmete sie rasselnd aus.

Dann war es vorbei.

Tränen stiegen in Barz auf, ob vom beißenden Rauch oder von Trauer, konnte er nicht genau sagen. Er umhüllte Sabris Ei sorgfältig in ein Stück Stoff, kam endlich auf die Idee, ein anderes über seinen Mund zu stülpen, und rannte aus dem Flammeninferno fort, weg von Jirisas Leichnam, weg von allem Unheil. Das Unheil, das sein Sternkraut zu verantworten hatte.

Später würde Barz nicht mehr genau wissen, wie er es zurück ins rettende Wasser des Avas geschafft hatte. Irgendwann war Yafka plötzlich da und zog ihn mit sich, dann sanft die Böschung hinab. Seine wunden Hände wurden im Seewasser gekühlt. Von irgendwoher wurde ein bitteres Heilkraut in seinen Mund gelegt. Dann war da auf einmal ein Boot, welches ihn zurück zur Pfahlbausiedlung brachte. Yafka saß an seiner Seite, streichelte seinen Rücken und seinen von Feuersalven lädierten Bart. Sie schalt ihn nicht, doch Barz spürte ihre Sorge und dass sie alles andere als zufrieden war, dass er blindlings ins Feuer gerannt war. Heldenhaft, aber auch unvorsichtig. Dass er sich für Yafka gleich verhalten würde, musste er nicht sagen, das war ihr klar. Schien auch nicht zwingend beruhigend zu sein für sie. Barz konnte das auch nachvollziehen, war er auch selbst etwas erschrocken über seinen Wagemut. Und doch konnte oder wollte er das Gefühl nicht abschütteln, dass er richtig gehandelt hatte. Ohne ihn wäre Sabris Ei im Steppenfeuer verbrannt. Und ohne Jirisa auch. Das gab ihm Halt. Das gab ihm das Gefühl, dass Jirisas Tod nicht völlig unnötig gewesen war. Auch wenn er das gewesen war.

So hielt Barz während der ganzen Heimfahrt Sabris Ei fest umklammert.\bigskip







Sonnentief. Einen Tag nach dem großen Unheil.\bigskip



Einen kurzen Moment des Schreckens gab es noch, als Tarok bei seinem Rückflug aus Krahd erspäht wurde. Doch offenbar hatte er gerade für seinen Geschmack genügend Chaos angerichtet. Späher der Iquar berichteten, dass Tarok sich in seine Höhle ins Graue Gebirge zurückzog. Vermutlich würde das riesige Wesen nun wieder für Jahre oder Jahrzehnte in einen tiefen Schlaf zurücksinken, bis jemand so töricht war, es zu wecken.

Der große Lifornus lachte verzweifelt über die Nachfrage, ob er als Feuerzauberer denn nicht etwas gegen die wütenden Flammen draußen in der Steppe tun könnte. Selbst die besten Wassermagier aus dem fernen Danwar aus hätten keine Chance, dieses Feuer zu löschen, spottete er.

Die Schamanen beteten zu den Göttern, und zu den Wassergeistern des großen Sees, und zu den Wassergeistern aller kleinen Seen, die über die Steppe verteilt sein mögen, ja, sie beteten selbst zu den Drachen aus uralter Zeit. Und als ob ihre Gebete erhört worden wären, öffneten sich tatsächlich bald darauf die Schleusen des Himmels. Drei Tage lang strömte Wasser aus dunklen Wolken und kämpfte gegen die letzten Reste des Drachenfeuers draußen in der Steppe, doch auch dieses konnte nicht ewig durchhalten.

Der einst staubtrockene Boden wurde zu Schlamm aufgeweicht. Graue Asche wurde in die Flüsse und in den Ava gespült, womit die Nixen, Seerösser und Spornwale für einige Zeit zu kämpfen hatten.

Steppenechsen und weitere Bewohner der Steppe wurden gesucht und gepflegt. Barz stellte beim Nest der Steppenechsen einen Todesstein auf in Jirisas Namen und konzentrierte sich dann mit doppeltem Elan darauf, Sabris Ei zu hegen und zu pflegen.

Lifornus wurde für sein misslungenes Ritual vor ein Gericht gestellt, doch in Anbetracht seiner Reue, seines guten Willens und seines tatenkräftigen Einsatzes bei der Zerstörung der letzten Skelettkrieger ließen die Iquar von einer Strafe ab.

Lifornus hatte hoffentlich aus seinem Missgeschick gelernt, doch vom Abhalten von Ritualen konnte ihn ihn nicht abbringen: Eines Morgens wanderte er geheimnisvoll tuend mit dem letzten Rest des getrockneten Sternkrauts, einem Tuch und einem Kohlestift in die verbrannte Steppe hinaus, nur um am nächsten Abend mit einem brennenden Vogelküken auf seiner Schulter zurückzukommen. Quasi als Beweis, dass sein Ritual doch wie gewünscht funktionieren und einen Phoenix herbeirufen konnte, wenn er nur die richtigen Worte wählte.

Barz sah den kleinen Phoenix gerade lange genug, um sich von dessen Existenz zu überzeugen (und um eine bissige Bemerkung über Lifornus‘ Wagemut runterzuschlucken), als das Vögelchen auch schon wieder mit einem leisen Plopp in einem Feuerbausch verschwand und nichts als ein kleines Häufchen Asche auf Lifornus‘ edlem Gewand hinterließ. Irrte Barz sich, oder hatte er dabei leise Flötentöne vernommen? Er sammelte das Häufchen Phoenix-Asche sorgfältig ein und riet Lifornus, lieber keinem der höheren Stammesmitglieder davon zu berichten, dass er sich erneut an so einem Ritual versucht hatte, erst recht nun, da die Iquar keinen Bedarf mehr daran hatten. Denn solange die Riesen aus dem Süden einen weiteren Vorstoß ins Barbarenland mit einem Angriff Taroks verbanden, würden sie sich so bald nicht mehr in die Steppe trauen.

Langsam, aber sicher wurden die Schrecken der vergangenen Tage zu nichts weiter als unangenehmen Erinnerungen. Eines Tages waren sie gar nur noch Erinnerungen. Trupps der Iquar zogen aus in die wenigen Wälder an den nahe gelegenen Berghängen und kehrten mit einer angemessenen Menge an Holz zurück. Das Loch in der Siedlung wurde wieder zu- und ausgebaut. Der Zeitturm wurde wieder aufgestellt, höher und stabiler denn je. Der große Lifornus konnte wieder Feuertricks vorführen, ohne böse Blicke zu erfahren. Barz‘ Bart wuchs wieder nach. Und die Kinder blickten nicht mehr furchtsam gen Himmel, sondern bastelten fröhlich kleine Drachenfigürchen in der Schule, von welchen Karyz ganz stolz je eine an Zanyitaz, Yafka und Barz überreichte. Barz stellte seine in seinen Arbeitsraum und richtete immer wieder ein paar dankbare Gedanken an die Götter, wenn sein Blick während des Analysierens der Phoenix-Asche an der Figur hängen blieb.\bigskip







Sonnenaufgang. 15 Tage nach dem großen Unheil.\bigskip



Über Nacht war der erste Schnee gefallen und hatte die Steppe in ein weißes Glitzerfeld verwandelt. Aufgeregte Kinder wurden von ihren Eltern ans Ufer gefahren, damit sie dort Schneeballschlachten ausfechten konnten. So manch ein Erwachsener gesellte sich ebenfalls dazu. Und während die Steppenechsen in ihre reglosen Starren verfielen, verschwanden auch immer mehr Krarks aus dem Himmel, wie sie es die meisten Winter taten. So kam es, dass kurz nach dem ersten Schneefall endlich ein Brief mit einem Falken nach Andor ausgesandt werden konnte.\bigskip







Sonnenhoch. 22 Tage nach dem großen Unheil.\bigskip



Ein eleganter Falke erreichte den großen See Ava und präsentierte stolz einen langen Brief an seinem Bein. Die fröhliche Antwort aus Andor hatte nicht lange auf sich warten lassen, und wurde bald darauf vor dem Versammlungssaal von einer nicht unzufrieden dreinblickenden Stammesleiterin Naquila verlesen: Der Angriff der Barbaren auf Andor war gescheitert, und dennoch großartig gewesen!

Eine Gruppe von Andori hatte sich der anstürmenden Vorhut der Barbaren gestellt, kaum dass diese über die nördlichen Ausläufer des Grauen Gebirges ins Königreich Andor eingedrungen waren. Eine Gruppe bestehend aus Kriegern, Bogenschützen, ja, gar Zwergen und Zauberern. Wenn man den Gerüchten trauen durfte, hatten sich auch ein abtrünniger Spion aus der Büffel-Sippe und selbst der Fleisch gewordene große Büffel höchstpersönlich den Barbaren entgegen gestellt. Ab dann hatten die meisten gewusst, dass diese Invasion eine törichte, verlorene Sache gewesen war.

Doch hatten die Andori den Barbarenkönig leben lassen, und mit ihm sein Volk. Der Barbarenkönig war vor den König von Andor getreten und hatte ihm seine Krone dargeboten. Dem König von Andor war das Riesenvolk, die Krahder, wohl bekannt, war er doch selbst als junger Mann aus ihrer Gefangenschaft geflohen. Er hatte Verständnis für die Barbaren gehabt und sie im östlichen Rietland willkommen geheißen.

Der Mehrheit der Barbaren ging es nun offenbar gut im fremden Königreich. Sie hatten ihre Jurten im goldenen Rietgras aufgebaut. Manche waren gar in Kontakt mit den bereits dort ansässigen Bauern getreten und waren daran, Freundschaften aufzubauen. Als geschickte Schmiede, gute Holzwerker und eine willkommene Unterstützung gegen die bösen Kreaturen wurden die Barbaren offenbar gut geschätzt. Das Rietland war groß und Platz hatte es genug. Nur die Zwerge aus Cavern schienen von den neuen Nachbarn nicht viel zu halten und hatten die Wachen an den Minenausgängen verdreifacht. Mit dem Winter hatten die Yetohe in der Steppe schon oft gekämpft, und nun konnten sie gar mit Nachbarn über Vorräte von Korn, Brot und dergleichen verhandeln. Manche munkelten, Häuptling Absorak habe gar die dünne Plörre, die die Andori Met nannten, zu schmecken gelernt.

Weitere Falken wurden ausgetauscht zwischen dem Ava und Andor. Für die Iquar schien die ganze Sache größtenteils vorbei zu sein, doch die verschiedenen Sippen der Yetohe hatten zum Teil ganz unterschiedliche Vorstellungen, wie sie fortan leben sollten. Manche wollten im sicheren Rietland verbleiben, vielleicht gar dort ansässig werden. Manche wollten lieber im nächsten Frühjahr, sobald ihre Reit-Echsen aus der Winterstarre erwacht wären, in die verbrannte Steppe zurückkehren, die bereits so lange ihre Heimat gewesen war. Irgendwie musste es sich doch auch dort wieder überleben lassen. Und natürlich gab es auch einige Krieger der Iquar, die plötzlich mit dem Gedanken spielen mussten, ob sie im Rietland verbleiben und vielleicht ihre Familien dorthin einladen oder an den Ava zurückkehren wollten.

Barz freute sich über alle guten Nachrichten, und doch füllte sich sein Herz mit Sorge. Einige tapfere Krieger der Barbaren im verzweifelten Angriff ihr Leben gelassen oder sich üble Verletzungen zugezogen. Und Barz hatte noch nichts von Nabib gehört.

Immerhin wusste er, dass Nabib noch am Leben war. Schon seit einiger Zeit hatte Barz immer wieder mal die Wirkung seines Meditationspulvers genossen, um für einige Stunden vor sich hin zu sinnieren und sich zu entspannen. Wenn er dabei nur fest genug an Nabib dachte, schien es ihm, als könne er aus der Ferne dessen Stimme vernehmen. Dann sprach Barz zu sich selbst, und auch zu Nabib, immer wieder, wie ein Mantra: „Nabib, wo auch immer du gerade sein magst: Wir werden uns wiedersehen. Ich werde dich finden.“

Und auch wenn Barz nichts Genaueres empfand als die reine Existenz dieser meditativen Verbindung zu Nabib, so wurde ihm dabei immer warm ums Herz und er spürte, dass Nabib noch da war, irgendwo weit weg von ihm in dieser wilden Welt.

\az{Jahr 63}

Dann kam der Jahreswechsel, und dann zog Winter auch schon wieder vorüber und der Schnee wich Blumen, welche rund um den Ava aus Skelettschädeln sprossen. Inzwischen war eine Normalität in Barz’ Leben zurückgekehrt, welche er lange vermisst hatte. Das Leben mit Yafka, Zanyitaz und Karyz, ganz zu schweigen vom zankenden Ehepaar der Yetohe, die einander tatsächlich heroisch aus dem von Tarok ausgelösten Feuer gerettet hatten, war abwechslungsreicher und aufregender, als es Barz‘ Aufenthalte in Thakkum über die Winterwochen üblicherweise gewesen waren. Doch das empfand er nicht als Nachteil.

Als das Tauwetter gekommen war, wurde es für viele Bewohner Thakkums so langsam Zeit, zu Reisen aufzubrechen. Manche der dort aufgenommenen Yetohe und auch einige Iquar wollten über den Wachsamen Wald nach Andor reisen und sich wieder mit dem Rest ihrer Familien vereinen. Lifornus hatte offenbar genug vom Ava gesehen und wollte mit dem großen Zug zum Baum der Lieder reisen, um seine aufgefrischten Erkenntnisse zum angeblichen Aussterben der Drachen mit den dortigen Chroniken abzugleichen. Und auch für Barz war es wieder an der Zeit, auf der Suche nach seltenen Substanzen und außergewöhnlichen Pulvern zu einer Reise aufzubrechen.

Diesmal hatte er eine klare Idee, wohin es ihn verschlagen sollte: Nach Andor! Das Königreich, in dem viele Barbaren inzwischen ein Leben im Luxus führten. Dort endeten Nabibs Spuren, und dort würde Barz mit seiner Suche nach Nabib beginnen können.

Yafka würde auch dieses Jahr zuhause im Ava bleiben, erst recht nun mit Zanyitaz und Karyz an ihrer Seite. Und auch die Steppenechse Jirisa würde bei dieser Reise nicht mehr an Barz‘ Seite sein. Falls es ihm nicht gelingen würde, Nabib aufzuspüren, wäre er ganz alleine unterwegs. Barz unterdrückte die aufwallende Melancholie. Er würde nicht ganz alleine sein, er hatte ja stets noch Sabris Ei an seiner Seite. Bald wäre es für die kleine Sabri an der Zeit, zu schlüpfen, und Barz würde dabei sein.

Doch wollte Barz nicht mit dem restlichen Tross der Barbaren nach Andor aufbrechen. Nein, nicht ohne Grund hatte er in den letzten Wochen ausgiebig das Häufchen Phoenix-Asche studiert, welches er von Lifornus erhalten hatte.

Er war sich nun beinahe vollständig sicher, eine Verbindung einzigartiger Materialien gefunden zu haben, welche ihm erlauben sollte, seine eigene Position und die des Phoenixes zu vertauschen – und da der Phoenix bekanntlich vermutlich wahrscheinlich aus Andor stammte, würde dieser Trick es Barz erlauben, direkt im fremden Königreich aufzutauchen, ohne vorher eine lange Reise im Tross der Barbaren auf sich zu nehmen. Den Phoenix selbst sollte das nicht stören, der hatte ja schon bewiesen, selbst im Nu in seine Heimat zurückteleportieren zu können, falls er von dort versetzt wurde. Und Barz würde mit etwas Glück schon in kürzester Zeit Nabib wieder in die Augen blicken.

Ja. Das war ein guter Plan.\bigskip







Sonnenaufgang. 72 Tage nach dem großen Unheil.\bigskip



Barz legte seinen neuen langen Mantel mit dem vielen Taschen an, den er von Yafka geschenkt bekommen hatte. Diese unzähligen Taschen waren vollgestopft mit verschiedensten kleinen Pülverchen und Mittelchen, die ihm da draußen in der weiten Welt behilflich sein konnten. Neu darunter war das Vorhersehungspulver, welches er in den letzten Wochen mithilfe seiner Schamanin zusammengestellt hatte. Eine Hauptzutat waren natürlich Silberblumenblüten gewesen. Yafka war es tatsächlich gelungen, Ableger dieser seltenen Blumensorte vom großen Salzsee zu ziehen. Eingepackt war die visionäre Mischung im kleinen türkisen Säcklein, welches einst am Bein eines Krarks Sternkraut von den Jpaxo nach Thakkum transportiert hatte. Sternkraut, welches das große Unheil ausgelöst und doch Thakkum von den Krahdern befreit hatte. Barz verdrängte den Gedanken daran.

Eine weitere seiner Manteltaschen enthielt die kleine Drachenfigur, welche er von Karyz geschenkt bekommen hatte. Zahlreiche Taschen waren natürlich auch leer, damit Barz sie bei seinen Reisen im fremden Land erst füllen konnte. Zusätzlich hatte Barz einige Taschen mit Nahrung, Trinkschläuchen, Lagermaterial und weiteren vollen und leeren Probensäcklein umgeschnallt, und ebenso seinen eleganten Köcher. Er vermisste es, eine treue Lastechse mit sich zu führen. Für den Moment begnügte er sich mit Sabris Ei, welches er in einer besonders gut gepolsterten Umhängetasche vor seinem Bauch transportierte und mit einer gehörigen Portion Heizsand wärmte.

Zu guter Letzt schnürte er seine Stiefel, schwang sich seinen Bogen um und fühlte sich bereit für die Abreise.

„Ich werde zurückkommen, mit oder ohne Nabib, und mit allerlei neuen Abenteuergeschichten“, versprach er. Bei seiner Verabschiedung von seiner Familie flossen wieder Tränen. Karyz weigerte sich fast länger als Yafka, ihn loszulassen. Barz blickte ihnen allen ein letztes Mal tief in die Augen, inbesondere Yafka. Stolz und Freude wallten in ihm auf, hier in dieser Runde dazuzugehören.

Kurz schwankte sein Entschluss, quasi alleine zu dieser Reise ins Ungewisse aufzubrechen. Niemand zwang ihn, auf eine Reise zu gehen. Er könnte auch einfach in Thakkum verbleiben und weiterforschen. Dann aber schweiften seine Gedanken wieder zu Nabib. Noch lebte dieser, doch musste ihn irgendetwas vom Schreiben aufgehalten haben. Wo er auch war, vielleicht war er auf Barz angewiesen. Und dann würde ihn Barz auf keinen Fall im Stich lassen wollen.

So stand Barz breitbeinig in der Mitte seines Wohnzimmers und gebot allen Anwesenden, ausreichend Abstand zu halten. Er ergriff die Prise Phoenix-Asche, welche er mit zahlreichen weiteren Pülverchen aus dem Hause seiner Schamanin aufgepeppt hatte, ließ drei Tropfen Wasser aus einem Trinkschlauch darauf fallen, zerrieb die entstehende Masse, pustete darauf und ließ zu guter Letzt einen Funken darauf sprühen. Seine Handfläche glühte rosa glimmernd auf, bis es ihn beinahe blendete. Karyz machte große Augen, in denen sich das rosa Glühen spiegelte.

Barz dachte ganz fest an den Phoenix, den er gesehen hatte, führte das pink glühende Pulver vor sein Gesicht, ballte seine Faust und...\bigskip



...benommen richtete sich Barz auf. Sein Umfeld nahm er nur schemenhaft wahr. Stark pulsierend floss das Blut durch seinen Körper. Was war geschehen? Wie war er hierhergekommen? Er erinnerte sich einzig allein daran, dass er gerade dabei gewesen war, diese neue Pulvermischung auszuprobieren. Dann war alles verschwommen und auf einmal hatte er sich in dieser ihm unbekannten Umgebung wiedergefunden. Hatte er es diesmal übertrieben? Hätte er lieber auf konventionellem Wege nach Andor reisen sollen? Befand er sich überhaupt in Andor?

Vorsichtig richtete Barz sich auf. Es war überraschend dunkel. Nur von einer Seite fiel helles Sonnenlicht auf den Boden und zeichnete dort einen hellen Fleck auf einen glatten Steinboden mit einigen Strohnestern darauf. Aber das war ja gar kein Stroh! Es waren helle, weiße Stränge eines Barz unbekannten biegsamen Materials. Schnell steckte er einen ein, für spätere Analyse.

Allem Anschein nach befand sich Barz in einer kleinen Höhle, deren Decke gerade hoch genug lag, damit ein Mensch aufrecht stehen konnte. Das Sonnenlicht schien durch den kleinen Höhleneingang auf das eigenartige weiße Material, das den Boden bedeckte. Barz huschte zum Eingang, blickte hinaus und staunte. Das war ganz und gar nicht so, wie er sich Andor vorgestellt hatte.

Barz‘ Höhle befand sich nicht auf dem Boden, sondern weit oben in der Höhe! Außerhalb des Eingangs fiel eine steile Steilwand in die Tiefe, und sowohl weiter oben als auch weiter unten erspähte Barz weitere Einbuchtungen, die zu Höhlen führen konnten. Dieser Ort sah aus wie ein riesiger löcheriger Baum, doch aus purem Stein. Eine hohe Konstruktion aus Plattformen und Stangen reichte um den Baum herum. Menschen und kleinere Wesen wuselten auf dem Gerüst herum. Bauten sie irgendetwas?

Am für Barz‘ Geschmack viel zu tief unter ihm liegenden Boden reichten die Wurzeln des Steinbaums weit in die Breite und wurden von sattem Gras bedeckt – in die Richtung, die Barz erspähen konnte, breitete sich eine weite Steppe aus. Doch war das Gras nicht golden, wie das Rietgras Andors beschrieben worden war, sondern feuerrot. Wo befand er sich hier nur? Stammte der von Lifornus gerufene Phoenix etwa gar nicht aus Andor?

Die Steppe schien endlos, doch es gab viele Dörfer, und an verschiedenen Stellen glaubte Barz, Schafe und Ziegen zu sehen, die in kleinen Herden über das flache Land getrieben wurden.

Ein hoher freudiger Schrei erklang, und eine riesige Gestalt huschte am Eingang von Barz Höhle vorbei, feuerrot, Wärme ausstrahlend, ja, gar buchstäblich in Flammen stehend. Barz stolperte zurück, das Bild von Jirisas verbranntem Körper urplötzlich wieder vor seinem inneren Auge aufblitzend. Der Phoenix vor der Höhle rauschte indes in die Tiefe, breitete seine riesigen Flügel aus und schoss von einem Flammenschweif verfolgt wieder in die Höhe, während er einen melodischen Schrei ausstieß. Drei weitere, kleine Phoenixe verfolgten den großen, doch flogen sie noch weitaus weniger elegant und weniger kontrolliert. Während das gesamte Gefieder des großen Phoenixes in Flammen zu stehen schien, konnte er bei den drei kleineren höchstens ein paar Funken am Ende ihrer Schwanzfedern erkennen.

Nur langsam traute sich Barz wieder nach vorne. Gefährlich, wie das Feuer auch war, war es doch auch wunderschön anzusehen. Barz hätte aus dieser sicheren Entfernung noch lange den majestätischen Flug der Feuervögel verfolgen können, doch in diesem Augenblick schwang sich wie aus dem Nichts ein kleines buckliges Männlein von einer erhöhten Plattform in Barz‘ Höhle, erschrak gewaltig, als es Barz sah und hob daraufhin bedrohlich seine Hände, in welchen ein grünliches magisches Licht aufglomm.

„Ich will dir nichts tun. Ich habe mich hier verirrt“, sagte Barz rasch und versuchte, möglichst ungefährlich dreinzuschauen. Er hatte keine Ahnung, ob das Kerlchen ihn tatsächlich verstehen konnte.

Der kleine Wichtel musterte ihn weiterhin argwöhnisch, nickte dann aber grimmig. Offenbar hielt er Barz für ungefährlich genug, dass er sich traute, seinen Kopf wieder aus der Höhle rauszustrecken und laut zu rufen:

„Aćh! Aaaaaćh! Hüterin Aćh! Da sitzt ein Mann einfach so im Nest der Takuri!“

Barz, der der Sprache der Tulgori damals noch nicht mächtig war, verstand den Temm natürlich nicht. Ebensowenig verstand er die Antwort, welche von weiter unten im Steinbaum erschall, und da sie nicht sonderlich freundlich klang, rechnete er lieber mit einer unangenehmen Konfrontation und schob seine Hand in einen Pulversack mit Schwächungspulver.

Die nächste Person, die in Barz‘ Höhle kletterte, war kein Wichtel, sondern Menschenfrau mit einem leuchtend roten langen Umhang, welcher durch zwei gekreuzte Phoenix-Federn eindeutig die Verbindung der Frau zu diesen mythischen Feuervögeln signalisierte. Sie blickte ziemlich verwirrt drein, als sie den über und über mit Taschen bedeckten Barz vor einem leeren Nest stehen sah. Nichtsdestotrotz hatte sie rasch ein goldenes Schwert gezückt und vor Barz‘ Kehle gehalten.

„Wer bist du? Woher kommst du? Was suchst du hier?“

Was für eine faszinierende fremde Sprache das war, mit einer Menge abgehackter Laute und einem ‚s‘, das stark nach Lispeln klang. Barz versuchte, die passenden Worte und Gesten zu finden, um möglichst friedlich zu kommunizieren, dass er die Worte seines Gegenübers nicht verstand. Dann gab er auf und wechselte Kurs.

„Barz“, sprach er, auf sich selbst zeigend.

Die Phoenix-Hüterin nickte vorsichtig, legte sich die eigene Hand auf die Brust und sprach ihm nach: „Barth.“

Barz schüttelte freundlich lächelnd seinen Kopf. Zumindest das Lächeln und einfache Kopfbewegungen schienen hier dieselben zu sein wie die Barz altbekannten.

„Barz“, wiederholte er, auf sich selbst zeigend, „Barz.“

Dann zeigte er auf die Phoenix-Hüterin und machte ein möglichst verwirrtes Gesicht, um sie nach ihrem Namen zu fragen.

„Barbarth“, wiederholte sie, seinen verwirrten Gesichtsausdruck imitierend, „Barbarth Barth.“

Barz musste grinsen, denn bei diesen Worten kam ihm ganz unpassenderweise der alte Kinderreim vom Barbaren Barz in den Sinn. Dieser Namensvetter von Barz, ein legendärer bärtiger Barbier aus dem Barbarenland, war eigentlich ein Bierbrauer gewesen und hatte auch eine eigene Bierbar zum Ausschank seines berühmten Barbarenbiers besessen, doch hatte seine Passion halt eben dem eleganten Stutzen von Barbarenbärten gegolten. Ein Barbarenbierbrauer-Barbarenbierbarbesitzer-Barbarenbartbarbier-Barbar Barz halt. Köstliche Wortkonstellationen konnte man da zusammenstellen, unabhängig davon, ob ein solcher Barbar Barz je wirklich existiert hatte. Als Kind hatten Barz‘ Eltern ihm immer wieder versprochen, dass sie ihn nicht aufgrund des Reims benannt hatten.

Fokus! Barz unterbrach sein Grinsen, blickte der Phoenix-Hüterin in die dunklen Augen und zeigte auf verschiedenste Dinge, die er sehen konnte und anschließend laut benannte – das Nest – „Nest“ –, den Wichtel – „Wichtel“–, einen vorbeirauschenden Phoenix – „Phoenix“–, und zu guter Letzt sich selbst – „Barz“.

Nun schien der Hüterin ein Lichtlein aufzugehen. Auch sie zeigte nun auf einen vorbeirauschenden Phoenix – „Takuri“ –, Barz – „Barth“ – und zu guter Letzt auf sich selbst – „Ack“.

Ack, die Takuri-Hüterin. Sehr schön, damit war der Startstein für ihre Kommunikation gelegt. Barz wiederholte Acks Namen. Nun grinste sie und schüttelte ihren Kopf, vermutlich wegen Barz‘ grässlicher Betonung. Er trat einen Schritt näher. Sofort zückte Ack wieder ihr elegantes goldenes Schwert.

Da vernahm Barz ein Geräusch, das verdächtig nach dem Knacken einer Eierschale klang. Für einen kurzen Augenblick dachte er, dass er soeben irgendein Takuri-Ei zertreten hatte, und nun Acks Zorn erleiden müsste.

Dann blickte er an sich hinunter. Er glaubte, eine Bewegung in Sabris Tragetasche zu erkennen. Und da verstand er.

Er hatte tatsächlich eine knackende Eierschale gehört. Sabri! Seine Steppenechse schlüpfte!

Erschrocken und auch etwas hilflos blickte er wieder zurück zu Ack.

Und in ihren Augen erkannte er Verstehen.

\begin{center}
    Weiter geht es in \hypref{Der verschwundene Feuertakuri (2022)}.
\end{center}












\newpage
\section{Epiloge}

\az{Jahr 62}

„Du hast nicht nur den Tod einer unserer mächtigsten Hexen und der Verlust unserer halben Armee auf deinen Schultern zu tragen, sondern auch, dass du einen waschechten Drachen in unser Reich gelockt hast. Hätten wir nicht diese letzte Zera aus dem unterirdischen Krieg noch hier stehen gehabt, so frage ich dich: Wie stünde es um Borghorn?! Wie stünde es um unsere Familie?!“

Prinz Ferntahr schluckte tief und wagte es nicht, aufzublicken. Er verbarg sein vom Drachenfeuer entstelltes Gesicht hinter dem Schädel seiner Mutter.

König Gonhar, sein Vater, schlug ihm den Schädel aus dem Gesicht.

„Du bist es nicht wert, ihr Antlitz zu tragen! Du bist es nicht wert, unseren Namen zu tragen! Wenn du Undankbarer das nächste Mal diese ‚große Leere‘ oder ‚Lust auf mehr‘ in dir verspürst, wenn du dich das nächste Mal in diesen Hallen nicht heimisch fühlst, so verschwinde doch einfach und erlöse uns von deiner Dummheit! Cuhor hatte dich gewarnt, im Norden Vorsicht walten zu lassen!“

Jetzt wurde es Ferntahr doch zu viel: „Aber Vater! Cuhors Warnungen sind weniger verlässlich als der Tratsch eines altes Waschweibs! Und doch habe ich alle Vorsicht walten lassen, die man von mir verlangen könnte. Wir sind langsam vorgegangen, haben nie viel riskiert, ja, Nahrack hat gar jeden Tag die Geister der Gefallenen gefragt, ob...“

Gonhar beachtete seinen Sohn nicht einmal, sondern redete sich weiter in Rage: „Großzügig erfülle ich dir deinen Wunsch, diese fremde Land zu bereisen und diese törichten Norderländler für ihren Einfall in unser Königreich zu strafen, und das ist dein Dank?!“

Neben dem Thorn lehnte sich Prinzessin Ennevahr gegen einen Stein. Ein hässliches Grinsen verzerrte ihr Gesicht, während sie die Demütigung ihres Bruders beobachtete. Nun reckte sie sich zu Gonhar hoch und fragte betont unschuldig:

„Sag, Vati, darf Ferni nach diesem Debakel weiterhin seinen Teil der Armee der Toten führen? Oder müsste dieses Kommando ihm vielleicht entzogen...“

Ferntahr biss seine Zähne zusammen und knurrte: „Enn, halte dich da raus! Du kämst ja nicht einmal mit der Führung von zehn Skeletten klar, da...“

„Ferntahr, guck mich an! Du redest hier mit mir. Mit deinem König! Nicht mit deiner Schwester! Und die Antwort lautet: Nie wieder! Kein einziger unserer Hexer wird dir je wieder irgendwohin folgen! Kein einziges unserer Skelette wird sich deinem Befehl beugen!“

Ferntahr stöhnte auf: „Vater, habt Erbarmen. Wer konnte schon wissen, dass diese Barbaren einen verflammten Drachen auf ihrer Seite hatten? Und Nahracks Tod unterliegt nicht meiner Verantwortung, sie selbst legte sich mit Tarok an, anstatt zu fliehen. Das kann doch nicht meine...“

„Schweig, Unseliger!“, donnerte Gonhar, „Ich bin deine Ausreden satt! Ein fliehender Feigling bist du also auch noch. Was habe ich nur für einen Sohn erzogen?!“

Ferntahr versuchte, Ennevahrs Selbstgefälligkeit auszublenden und presste sich vor dem Thron noch tiefer auf den Boden, während er flehende Worte an seinen Vater richtete:

„Ich bitte Euch, lasst mir wenigstens das Kommando über das dritte Bataillon der Toten. Ich kann besser sein. Ich werde besser sein.“

„Das Kommando über deine Handvoll Ambacus magst du behalten, doch die mickrigen Überreste deines Heeres wirst du so bald nicht mehr sehen. Dabei bleibt es! Ich habe gesprochen!“

Gonhar atmete tief durch und ließ seinen massigen Körper ächzend zurück auf den grob gehauenen Felsen sinken. Diese Tirade hatte ihn ermüdet.

Ennevahr streckte Ferntahr die Zunge raus.

Ferntahr trat gegen einen Felsen. Er wusste, dass seines Vaters Gemüt rasch aufbrauste und dass er seinen Worten dann nicht allzu viel Glauben schenken sollte. Ferntahr konnte wieder in seinem Ansehen aufsteigen. Doch vergessen würde König Gonhar das Geschehene nicht. Wenn Ferntahr das nächste Mal irgendwo aufbrechen würde, würde er dies alleine tun müssen. Alleine, ohne Hexer, ohne Skelette. Höchstens mit einigen Ambacus.

Er knurrte frustriert und schlug erneut gegen einen Felsen.

„Darh, Rahha, kommt!“, rief er zweien seiner Ambacu-Hexen ungehalten zu, „Ruft mir zwei Golems und lasst sie meine Sänfte in meinen Turm tragen!“

„Mein Prinz“, druckste Darh herum, „Meister Corion verlangte nach mir, um...“

„Scheiß auf Corion, dein Prinz befiehlt es dir!“, brüllte Ferntahr auf. Selbst Ambacus wagten es, ihm zu widersprechen. Seine Augen wurden wässerig vor Schande.

Der Thron Borghorns würde ihm verwehrt blieben, solange Undavahr, dem erstgeborenen Prinzen, nichts geschah. Undavahr, der Feigling, der Ferntahr davon abgeraten hatte, die elenden Barbaren in den Norden zu verfolgen. Undavahr, der perfekte Sohn, der nie in die weite Welt aufbrechen wollte und zufrieden mit seinem Platz in dieser Ödnis schien.

Ferntahr seufzte.

Was wollte er hier in Krahd überhaupt noch?\bigskip



Nabib hustete und prustete, als der andorische Wassergeist ihn endlich aus seiner Gewalt (und aus seinem ertränkenden Körper) entließ. Die schimmernde Wassergestalt trat einen Schritt zurück und betrachtete ihn mit einem traurigen Blick aus blauen Augen hinter durchscheinenden blauen Haaren. Nabib hustete und prustete nach Luft, versuchte verzweifelt, sich aufzurichten, seine Waffe zu finden, schnell, rasch, ehe...

Ein stumpfer Schlag traf Nabibs Seite, woraufhin er endgültig auf den Boden klatschte und dort liegen blieb. Ängstlich starrte er nach oben und ins breite Gesicht eines Zwergs mit kurzem blonden Bart, welcher ihm eine aufwändig verzierte Doppelaxt unters Kinn hielt.

„Hübsch unten bleiben“, brummte der Zwerg. Er sprach die Sprache der Bewahrer, welche jedem reisenden Nomaden zumindest im Ansatz beigebracht wurde, wenn er in Kontakt mit anderen Völkern treten wollte. Immerhin bedeutete das, dass Nabib mit ihm verhandeln konnte.

Nabib betastete vorsichtig seine schmerzenden Rippen und erschrak, als er seine Finger hochhielt und helles Blut darauf sah. Er presste seine Hände auf die Wunde, während sein Geist raste. Bislang hatten Barz und er nur ein einziges Mal mit Zwergen zu tun gehabt, damals, als sie in die nördlichen Ausläufer des Grauen Gebirges aufgebrochen waren, um mit den Jpaxo zu reden. Die Zwerge waren aus dem Nichts aufgetaucht und hatten Barz und Nabib erst ziehen lassen, nachdem diese ihnen einen „angemessenen Zoll“ in Form ihrer goldenen Armbänder und Ketten überlassen hatten. Damals hatten die beiden Steppennomaden gelernt, dass Zwerge äußerst gierig sein konnten.

„Verschont mein Leben“, rief Nabib dem Zwerg zu. Dieser hielt ihm weiterhin seine Axt unters Kinn und legte seinen Kopf schief. Jetzt trat sogar noch ein weiterer Krieger der Andori an seine Seite, ein Mensch mit einem roten Haarschopf und einem Körperbau, den Nabib ungut an einige Krieger der Yetohe erinnert, an deren Seite er in den letzten Tagen gekämpft hatte. War es denn wirklich wahr, dass einige Barbaren gar auf die andere Seite übergelaufen waren? Nun, das war jetzt nicht relevant.

Nabib presste die Hände an seine Wunde. „Verschont mein Leben“, rief er erneut. „Es soll euer Schaden nicht sein“, ergänzte er und hielt dem Zwerg das reich verzierte Amulett hin, welches Barz ihm zu seiner Abreise aus Thakkum geschenkt hatte. Es schmerzte ihn, die Aufgabe dieses Andenkens überhaupt in Betracht zu ziehen, aber letzten Endes war sein Leben um einiges wichtiger als Barz‘ Erbstück. Barz würde das gut verstehen können.

„Das ist nicht nötig“, brummelte der rothaarige Mensch. Ja, Nabib glaubte eindeutig, einen Hauch barbarischen Akzents in seiner Stimme mitschwingen zu hören.

„Kein Held von Andor würde jemanden, der sich ergibt, angreifen“, bekräftigte nun auch der stämmige Zwerg. Dann traten er und der rothaarige Mensch an Nabibs Seite und halfen ihm auf.

Ein schwarzer Rabe ließ sich auf der Schulter des rothaarigen Menschen nieder, plusterte sein Gefieder und wirkte überaus zufrieden mit sich selbst. Er hielt seinen Schnabel an das Ohr des Rothaarigen und krächzte leise vor sich hin. Nabib hätte schwören können, dass die Stimme des Raben menschlich klang. Der rothaarige Mensch blickte den Zwerg und Nabib noch einmal prüfend an, nickte dann und wandte sich von den beiden ab, anderen Verletzten zu.

Während Nabib vom Zwerg in ein Lazarett in der Rietburg begleitet wurde, sah er aus dem Augenwinkel, wie der Barbarenkönig von seiner Reit-Echse stieg und seine Krone zu Füßen eines riesigen Büffelwesens mit großen Hörnern auf einem ansonsten relativ menschlichen Körper legte. War dies etwa der Fleisch gewordene Große Büffel, aufgrund dessen Auftritts vorgestern Absorak und der gesamte Büffel-Clan die Waffen niedergelegt hatte?

Nun, wenn selbst der Barbarenkönig sich den Helden von Andor ergab, war die Sache gelaufen. Was für ein Debakel die Invasion Andors doch gewesen war. All diese Toten und Verletzten, nur für nichts und wieder nichts. Mit Furcht dachte Nabib an die Zurückgebliebenen in Thakkum. Egal, ob die Riesen aus dem Süden mit ihrer Untotenarmee am großen See Ava Halt machten und die Pfahlbausiedlung belagerten, oder ob sie auf den Spuren der Barbarenarmee in Richtung Andor zogen, nahm diese Geschichte kein gutes Ende. Es schüttelte ihn.

Und dennoch... während der Zwerg ihn an einem riesigen orangen Feuer in einer eisernen Schale vorbei und durch ein riesiges Tor in die mächtige Rietburg eskortierte, konnte Nabib das Gefühl nicht abschütteln, dass seine Zukunft nicht allzu düster aussah. Denn urplötzlich trat Barz‘ Gesicht vor sein inneres Auge. Urplötzlich erfüllte Nabib eine innere Ruhe und Gelassenheit, die er nicht einmal für möglich gehalten hatte. Als hätte er bereits seit Stunden meditiert. Und es war ihm, als höre er die leise Stimme Barz‘ in seinem Ohr.

„Nabib, wo auch immer du gerade sein magst: Wir werden uns wiedersehen. Ich werde dich finden.“\bigskip



„Bitte, haltet ein mit Euren Lobpreisungen, ich bin wahrlich keine Gottheit. Ich bin nur Bragor, ein Tarus aus dem fernen Sturmtal. Werter Häuptling Absorak, es ehrt mich, wenn Ihr in mir etwas Besonderes seht. Und natürlich will ich, dass diese Streitigkeiten hier beigelegt werden. Keine Königreiche profitieren von einem solchen Krieg, weder Andor noch die Barbaren. Doch will ich Euch nicht täuschen. Mit Eurem heiligen Büffel...

... verzeiht, mit Eurem \textit{Großen} Büffel habe ich nichts zu tun. Ich habe überhaupt noch nie einen Büffel gesehen in meinem kurzen Leben.

Na ja, keinen lebendigen jedenfalls. Und diese Statue zeigt doch überhaupt erst, wie unterschiedlich wir sind. Guckt mich doch an, ich habe Finger! Und Daumen! Welcher vernünftige Büffel besäße denn Daumen?

Nein, da müsst Ihr euch irren, ich war noch nie...

Nein, ich habe auch noch nie einen Bart getragen. Wisst ihr, bei uns Taren dauert das ein bisschen länger als bei Euch Menschen.

Was?! Mögt ihr das noch einmal wiederholen, werter Häuptling?

Ein Tarus wie ich? Seid Ihr Euch ganz sicher?

Beim Wüten der Sturmgeister! Wie lange ist das her? Wo befindet sich dieser Tarus jetzt?

Das Graue Gebirge? Auf der Suche nach Sternkraut?! Das ist er! Das muss es sein!

Wer? Na, mein Opa. Wegen ihm bin ich überhaupt erst nach Andor gekommen. Und dank Euch habe ich nun endlich einen Hinweis darauf, wo ich weiter nach ihm suchen könnte.

Nein, nicht zwingend gerade jetzt. Nicht, solange hier noch Helden benötigt werden. Seht Ihr den Schild dort drüben? Das ist der mächtige Bruderschild, bei dem haben wir uns der Hilfe der Hilfsbedürftigen hier verschrieben, und dieses Versprechen werde ich nicht leichtfertig brechen. Aber irgendwann mal, wenn die Lage hier ruhiger ist.

Ihr habt mir soeben Hoffnung geschenkt. Ich danke Euch von ganzem Herzen, werter Absorak.

Opa, wo auch immer du gerade sein magst: Wir werden uns wiedersehen. Ich werde dich finden.“




\begin{chapterbox}
    \chapter{Der verschwundene Feuertakuri (2022)}
    \label{Der verschwundene Feuertakuri (2022)}
    \az{Jahr 63}

    \begin{center}
        Teil II der Magischen Abenteuer
        
        Fortsetzung von \hypref{Der Steppennomade, der große See und das große Feuer (2022)}
    \end{center}
    
    Takuri-Hüterin Aćh hat einen seltsamen Tag. Ein fremdsprachiger Nomade aus einem fernen Land erscheint wie aus dem Nichts an ihrem Nestbaum in den westlichen Ausläufern des Kuolema-Gebirges. Zeitgleich verschwindet ihr liebster Feuertakuri Turr. Sie kann ihn nicht einmal mehr mit ihrer Steinflöte zurückrufen. Hängen die beiden Vorfälle zusammen? Kann vielleicht der vielwissende Hüter der Zeit in der Metropole Agarb am anderen Ende von Tulgor weiterhelfen?
\end{chapterbox}


\section{Prolog}

\az{Jahr 62}

Die Stimmung am Nestbaum war angespannt. Schon seit Tagen lag Saarćhan, die älteste und riesigste aller hier angesiedelten Feuertakuri, hustend und keuchend in ihrem gewaltigen Nest. Sie lag im Sterben, und sie würde bald in einem riesigen Feuerwirbel ihr Leben beenden.

Der Lebenszyklus größerer Takuri dauerte erheblich länger als der von kleineren, und es war eine wahre Seltenheit, dass derart riesige Takuri ihr Leben beendeten. Sämtliche Hüter sahen solchen Momenten immer mit Spannung entgegen. Manchmal, ganz selten, kam es vor, dass ein Takuri im Feuer verging, ohne ein kleines Küken in der Asche zu hinterlassen. Ob dieser Takuri dann wirklich vom Angesicht dieser Welt verschwunden war, oder ob das Küken sich schlicht in diesem Augenblick an einen anderen Ort teleportiert hatte, war schwer zu sagen. Die Unsicherheit blieb jedenfalls. Und so warteten die Hüter allesamt Saarćhans letzten Atemzug ab. Doch bedeutete dies nicht, dass sie deswegen die Pflege der dutzenden anderen aufmüpfigen Feuervögel am Nestbaum vernachlässigen durften.

Takuri-Hüterin Aćh kümmerte sich beispielsweise gerade darum, die Gewichtsverläufe dreier schwächerer Takuri über die letzten Wochen zusammenzusammeln und möglichst übersichtlich auf einem Stück feuerfesten Pergaments gegeneinander zu plotten. Ihre umherschweifenden Gedanken wurden allerdings abrupt unterbrochen.

„Aćh! Aaaaćh! Hüterin Aćh!“, erklang eine hohe Stimme. Das klang nach einem Temm. Aćh verdächtigte Yrbstschly. Was für ein Name! Als hätte die Kleine eine mit Absicht besonders schwer auszusprechende Zeichenkombination gewählt.

Yrbstschly war im Gegensatz zu den meisten anderen hier ansässigen Temm nicht am Bau von Takuri-Spiegeln oder ähnlich magisch aufwendigen Aufgaben beteiligt, sondern hatte sich dem profanen Orden der Hüter angeschlossen. Mit ihren für eine Temm mickrigen 53 Lebensjahren war sie ziemlich jung und relativ unerfahren. Sie hatte die Tendenz, stets um Hilfe zu rufen, sobald ihr auch nur eine Brotkrume zu viel zu Boden fiel. Und stets rief sie nach Aćh. Wohl weil die beiden im selben Zimmer ruhten. Es war nicht so, dass Aćh sich nicht freute, aushelfen zu können. Aber manchmal konnte es auch einfach etwas viel werden.

„Aaaaćh! Hilfe!“, erklang erneut der hohe Ruf. Aćh seufzte, legte ihre Schreibfeder beiseite und eilte aus ihrem Zimmer hinaus.

Ihr und Yrbstschlys gemeinsames Zimmer lag in einem eleganten Gebäudekomplex, der auf den Wurzeln des Nestbaums der Takuri erbaut worden war. Der Nestbaum der Takuri wiederum war ein gigantischer Baum aus purem Gestein, der sich dutzende Meter in den Himmel erhob. Er stand am Übergang von den Gebirgsausläufern in die rote Steppe Tulgors. Man könnte denken, jemand hätte einen gigantischen Mammutbaum versteinert. Dass der Baum direkt aus einem Berg gehauen worden war, schien den hier wohnenden Hütern unwahrscheinlich, denn sie schafften es heutzutage nicht mal mit ihren besten Pickeln, mehr als einige Kratzer in die Oberfläche des Nestbaums zu schlagen. Allzu oft zerbrachen stattdessen die Pickel. Darum hatten die Hüter auch keine Behausungen und Treppen im Baum selbst gebaut. Stattdessen führten aufwendige Konstruktionen aus Plattformen, Kränen und dergleichen um den Baum herum und erlaubten den Hütern, die darin eingelassenen Nisthöhlen zu erreichen.

Diese Baumhöhlen waren ebenso wie der riesige Nestbaum selbst schon so lange hier gewesen, wie die schriftlichen Chroniken der Hüter zurückreichten. Und die Takuri hatten schon damals diesen Baum so sehr geliebt, dass der Orden der Takuri-Hüter fast keine andere Wahl gehabt hatte, als sich hier niederzulassen.

Auch heute umkreisten dutzende Feuertakuri glücklich den riesigen Steinbaum und regneten Funken auf die Gebäude der Hüter auf den Wurzeln des Baums herab. Die aufmüpfigen Feuervögel waren die gute Seele des Nestbaums. Ohne sie gäbe es den gesamten hier ansässigen Orden ebenso wenig wie die Spiegelschmiede eine halbe Meile weiter nördlich. Nicht, dass Aćh sich groß auf die wunderschönen Vögel achten konnte. Denn kaum war Aćh ins Freie getreten, lief sie schnurstracks in eine herbeieilende Temm hinein.

Es war tatsächlich Yrbstschly, die nach ihr gerufen hatte. Die Kleine begann prompt zu sprechen: „Endlich erreiche ich dich! Aćh, es ist schrecklich. Ich hatte Turr zu den Minenarbeitern eskortiert...“

Aćh seufzte. Die Hüter sollten eigentlich keine Takuri allein zur Mine bringen. Die Sprengungen in den Minenstollen waren sehr kontrolliert und eine von einem übereifrigen Feuertakuri entzündete Lunte konnte unschöne Konsequenzen haben. Aćh setzte einen tadelnden Blick auf, wurde jedoch von Yrbstschly unterbrochen, ehe sie überhaupt etwas sagen konnte.

„Beschwere dich bitte nicht. Ich habe deine Oma gefragt, und sie meinte, ausnahmsweise sei es in Ordnung. Die Minenmeisterin höchstpersönlich brauchte dringend einen Takuri, um mit seiner Hilfe die nach einem Bergbeben verstellten Spiegel in den Stollen neu zu kalibrieren. Bei dieser ganzen Bürokratie, wenn sie einen offiziellen Antrag gestellt hätte, dann wäre Turr erst in einer Woche bei ihnen gewesen. Die Menge an Mera, die dadurch verloren gegangen wäre...“

„...und Turr ist klein und vergleichsweise brav, darum nahmst du ihn mit“, beendete Aćh den Satz, „Ich verstehe. Was ist geschehen?“

Angesichts der Tatsache, dass Yrbstschly sich nicht mehr in Begleitung eines kleinen Takuri befand, bot sich eine gewisse üble Vermutung an. Immerhin sah Aćh in den letzten Sonnenstrahlen der untergehenden Sonne keinen Rauch von den Stolleneingängen an den Hängen des Kuolema-Gebirges aufsteigen. Zu einer Explosion schien es nicht gekommen zu sein. Doch wo war Turr nun?

„Ich tat wie geboten und brachte Turr tief in die Stollen. Hatte gehörig Sufarsaft zur Sicherheit mitgenommen, und sein liebstes Knabberseil. Doch als kurz bevor wir den ersten verstellten Spiegel erreicht hätten, hat sich Turr einfach so in Luft aufgelöst!“, berichtete Yrbstschly schlotternd und bestätigte so Aćhs Vermutung. Sie kniete sich zur Temm nieder.

„Etwas erzählst du mir noch nicht, oder? Warum verängstigt dich sein Verschwinden derart?“

Feuertakuri waren nun mal bekannt dafür, dass sie sowohl Zeit als auch Raum bezwingen konnten. Die Zeit, indem sie nach jedem Tod ein neues Küken wurden. Den Raum, indem sie in einem Flammenbausch verschwinden und an einem anderen Ort wieder auftauchen konnten. Diese Teleportationseigenschaften waren schließlich der Grund, warum diese Niststätte hier auch als Spiegelproduktionsstätte fungierte: Aus der Asche der Feuertakuri wurden in der Spiegelschmiede eine halbe Meile weiter nördlich wertvolle Spiegel mit geheimnisvollen Kräften der Teleportation gefertigt. Insofern sollte ein verschwundener Feuertakuri eine Hüterin nicht derart in Furcht versetzen, ja, nicht einmal überraschen.

Yrbstschly biss sich auf die Lippen, atmete dreimal tief durch und plapperte dann los:

„Nun, Hüterin Aćh, Turr ist nicht einfach nur in einem Flammenbausch verschwunden. Nein, plötzlich ging so ein bläuliches Glitzern über ihn, und glaube mir, ich habe ein leises Zittern im Boden und ein fernes Donnern vernommen, und eine Stimme, die etwas schrie.“

Yrbstschly räusperte sich und sprach mit einer theatralischen, volltönenden, tiefen Stimme einige Aćh völlig fremd erscheinende Worte: „Narbi gicein, prento varafera tarkaran nuo dragos! Dirqo on te bini earim!“

Mit ihrer gewöhnlichen, hohen Stimme fuhr Yrbstschly fort:

„Und dann verschwand Turr in einem Flammenwirbel. Einem blauen! Das ist unnatürlich, sage ich dir, unnatürlich! Ich habe mich so schnell wie möglich hierher zurück gemacht, aber der Weg aus den Minenstollen ins Freie war ein langer, und dann auch noch den Hang hinunter... es sind mehrere Stunden seit Turrs Verschwinden vergangen. Und meine Sorgen wachsen mit jeder Minute. Bläuliches Glitzern!“

Aćh kratzt sich am Kopf. Das war tatsächlich äußerst mysteriös. Nun war sie auch beunruhigt.

Aćh zog ihre tulgorische Steinflöte aus einer Seitentasche hervor. Die Flöte hatte vier Löcher und ähnelte mit etwas Fantasie einem sitzenden Takuri. Das Äußere ihrer Steinflöte hatte Aćh selbst zu bearbeiten gehabt, als sie sie erhalten hatte. Damals, als sie höchstoffiziell zu einer Takuri-Hüterin ernannt worden war.

Die Traditionen von Tulgor sagten, dass zum Trainieren und Führen eines Takuri außergewöhnliche Fähigkeiten nötig waren, die man durch eine lange und harte Ausbildung erwerben musste. Ein Takuri-Hüter zu sein war eine große Ehre. Und zum Abschluss ihrer Ausbildung erhielten neue Hüter eben ihre eigenen tulgorischen Steinflöten, um auch über lange Strecken mit den ihnen anvertrauten Feuervögeln kommunizieren zu können.

Das Äußere der Flöte mochte von Aćh selbst geschlagen worden sein und nicht von der größten Handwerkskunst zeugen. Das Innere der Flöte hingegen war sehr exakt gefertigt worden, von einer Temm, die beinahe ihr ganzes langes Leben hier am Nestbaum verbracht und die Kunst des Steinformens perfektioniert hatte. Sie wusste genau, wie man die Gänge durch den Stein zu ziehen hatte und wo man die Löcher hervorkommen lassen musste, damit das fertige Produkt die gewünschten Melodien hervorbringen konnte. Und in solchen Situationen war Aćh äußerst froh darüber.

Aćh setzte die Flöte an ihre Lippen und blies sanft hinein. Ein langsam anschwellender Ton, der dem Heulen des Windes durch die Schluchten des Kuolema-Gebirges ähnelte. Dann ließ sie ihre Finger geübt über die Löcher springen. Eine kurze, wilde Melodie, die abrupt endete. Turrs Melodie. Wie ein Name war sie, und vielleicht war sie gar effektiver als Turrs tatsächlicher Name. Manchmal weigerte sich der kleine Schlingel doch aus Prinzip, auf „Turr“ zu hören.

Aćh wiederholte Turrs Melodie noch einige Male, jeweils lauter und besorgter, und hängte einige dissonante Töne an, die Dringlichkeit und Sorge ausdrückten.

Kein Turr erschien. War er zu weit weg, um die Musik zu hören? War ihm vielleicht gar etwas zugestoßen? Yrbstschly betrachtete Aćh mit von Sorgen faltigem Gesicht. Aćh setzte eine betont gelassene Miene auf und verkündete: „Na, dann bleibt uns nur eines übrig: Ab ins Aschelager mit uns.“

Temm waren erheblich kleiner als Menschen und konnten im Rennen kaum mit ihnen mithalten. Darum ließ Aćh Yrbstschly auf ihre Schultern klettern. Gemeinsam eilten sie so ins Aschelager.

Das Aschelager war gemeinsam mit einigen anderen Gebäuden der Hüter und Spiegelbauer am Fuße des Nistbaums gebaut worden. Lange, hohe Regale mit verschiedensten Kisten und Schachteln bedeckten die kreisrunden Wände. Zielstrebig eilte Aćh zu einer Holzdose mit der Aufschrift „Turr“. Sie leerte ein Stückchen von Turrs Asche in eine speziell dafür eingebaute Einbuchtung an der Steinflöte. Ein wenig der Asche stieb in ihre Nase, die unangenehm zu kribbeln begann.

Aćh kannte die Anzeichen ihrer Niesreizattacken gut genug, um sich rechtzeitig abzuwenden und zu verhindern, die Aschen dutzender Takuri aus ihren Schachteln zu niesen. Leider hatte sie nicht bedacht, dass aktuell eine Temm auf ihren Schultern saß. Bei ihrem rasanten Umdrehen streifte Yrbstschly das Regal und rempelte Turrs Aschedose zu Boden.

Nachdem das letzte „Hatschi“ Aćhs Kopf durchgeschüttelt hatte, erkannte sie Turrs Dose umgedreht am Boden, während ein Windhauch den größten Teil seiner Asche effizient zur Tür hinauswehte. Und auch die Einbuchtung ihrer Steinflöte war wieder aschefrei. Aćh unterdrückte einen Fluch. Zerbrochenen Spiegeln soll man bekanntlich nicht nachtrauern, sondern stattdessen neue anfertigen.

Yrbstschly kletterte von Aćhs Rücken und hob die fast leere Dose von Turrs Asche auf. Mit großen Augen blickte sie zu Aćh hoch:

„Oh nein, oh nein, das wollte ich nicht! Verzeih mir, Hüterin Aćh. Was sollen wir nun nur...“

„Das ist nicht deine Schuld. Ich war das Trampeltier hier. Hast du dir wehgetan?“

Yrbstschly betastete ihren Glatzkopf sorgfältig und schüttelte selbigen dann. Aćh atmete auf.

„Dann ist doch alles gut. Ich habe noch genug Asche übrig, um ihn zu rufen.“

Sie griff an ihre bronzene Halskette und löste das linkste Element daran. Eine kleine Dose mit einer Prise von Turrs Asche darin. Sie hatte doch gewusst, dass dies eines Tages nützlich sein könnte!

„Ich habe kein gutes Gefühl dabei, Aćh“, murmelte Yrbstschly.

„Noch können wir hoffen, dass wir nicht mehr brauchen werden“, versuchte Aćh, optimistisch zu bleiben. Entschieden kippte sie die Dosis in ihre Steinflöte. Dann lief sie nach draußen und winkte den ersten vorbeifliegenden Takuri zu sich. Sie kraulte diesen am Nacken, bis er wohlig gurrend einige Funken aus seinem Gefieder verstreute.

Die Funken genügten, um die Turrs Asche in der Steinflöte zu entzünden. Rauch stieg auf und immer mehr Ascheflocken wurden mit leisem Plopp an den fremden Ort gesogen, an dem sich Turr aufhielt. Wo auch immer das sein mochte. Die Verbindung stand.

Erneut blies Aćh in ihre Flöte und spielte Turrs Melodie, betont besorgt und dringlich.

Kaum waren die letzten Flötentöne leise verklungen, entzündete sich aus dem Nichts ein Flammenwirbel in der Luft über Aćhs Kopf.

Ein kleiner Feuervogel fiel aus dem Feuerbausch herunter, prallte ungelenk auf Aćhs Kopf, krallte sich in ihre langen Haare und rollte in ihre Arme. Leise gurrend rieb der Takuri seinen Kopf an Aćhs Schulter. Turr. Aćh atmete erleichtert auf. Mit ihm hatte sie schon eine besondere Bindung gehabt, als sie noch ein kleines Kind gewesen war. Vorsichtig streichelte Aćh die feinen Federn an Turrs Nacken. Flammen flackerten über Turrs Gefieder, und Aćh zog ihre Hand rasch wieder zurück.

Das flackernde Gefieder und der Blick in Turrs Augen verrieten Aćh, dass er aufgeregt war, ja, gar ängstlich. Yrbstschly hatte gesagt, ein seltsames blaues Glitzern habe ihn überzogen, ehe er verschwunden war? Magie? Aber nicht Magie eines Temm, das hätte Yrbstschly spüren können. Ein Höhlenwicht? Ein menschlicher Magier?

Hin und wieder tauchten Pilgerer von nah und fern am Nestbaum auf, die sich etwas von den Takuri erhofften. Asche, Federn, Knochen, Zutaten für irgendwelche Tränke oder Pulver. Solange die Takuri fürs Stiften dieser besonderen Ingredienzien nicht leiden mussten, folgten die Hüter den Wünschen meistens. Gegen entsprechendes, angemessenes Entgelt, verstand sich. Konnte es sein, dass ein solcher Hexer hinter Turrs Verschwinden steckte?

Aćh und Yrbstschly informierten ihre Vorgesetzten über den Vorfall und den Unfall mit Turrs Aschedose. Diese ließen die gesamte Umgebung des Nestbaums nach einem Übeltäter absuchen, ohne Erfolg. Turr wurde beruhigt und flackerte eine Stunde nach seiner Rückkehr kaum mehr auf, nur noch einige Funken sprühten hin und wieder von seinen Schwanzfeder.

Dennoch machte sich Aćh nicht gleich wieder daran, neue Asche für Turrs Aschelager zu sammeln. Die Sammelprozedur war anstrengend und nicht immer angenehm für die Takuri, wie auch das Teleportieren kräftezehrend für sie war. Nachdem Turr zweimal eine unbekannte, potenziell sehr große Strecke gesprungen war, wollte sie ihn sicher nicht überstrapazieren. So bald würden sie die Asche ja hoffentlich nicht brauchen.\bigskip

***\bigskip

Barz starrte verwirrt auf die Schulter des großen Lifornus. Noch vor wenigen Sekunden hatte dort ein kleines brennendes Vögelchen gesessen. Nun, eine spontane Stichflamme später, war von diesem Vogel nichts mehr zu sehen bis auf ein kleines Häufchen Asche auf dem edlen Gewand des Zauberers. Leise Flötentöne schwangen einen Augenblick lang noch in der Luft, dann waren auch sie verklungen.

„Du hast ihn gesehen, oder?“, rief Lifornus beinahe flehend, „Mein Ritual funktioniert!“

Barz schluckte eine bissige Bemerkung über Lifornus‘ Wagemut runter und nickte stumm. Ein waschechter Phoenix. Es gab sie tatsächlich.

Während der große Lifornus stolz seinen spitzen Zaubererhut richtete und sich den Schweiß von der braunen Stirn wischte, streckte Barz seine Hand nach den Brandflecken auf Lifornus‘ Gewand aus.

„Dürfte ich vielleicht seine Asche einsammeln?“


\newpage
\section{Turr verschwindet und ein Fremder erscheint}

\az{Jahr 63}

Zwei Monate später\bigskip



„Aćh! Aaaaaćh! Hüterin Aćh! Da sitzt ein Mann einfach so im Nest der Takuri!“, erklang eine hohe Stimme aus einer Nesthöhle über ihr. Klang nach einem Temm. Aćh verdächtigte Yrbstschly. Sie unterbrach das Putzen der Ascheflecken am Boden vor ihr, streckte ihren Kopf aus der Nisthöhle und blickte am Stamm des Nestbaums in die Höhe.

„Was ist denn jetzt?“, rief sie hoch.

„Ein Fremder! Da sitzt ein Fremder in Turrs Nest!“, kam die Antwort abrupt.

Die Stimme klang wirklich verängstigt. Aćh ließ ihr Putzzeug fallen und schwang sich zwischen Plattformen am Nestbaum in die Höhe, bis sie die Baumhöhle erreichte, aus der eine Temm eifrig winkte. Es war tatsächlich Yrbstschly. Und Yrbstschly war nicht allein.

Ein in einen langen braunen Mantel gekleideter Mann stand zwischen verstreutem Nestmaterial weiter innen in der Nisthöhle und beäugte Yrbstschly vorsichtig. Seine eine Hand hatte der Fremde beschwichtigend in die Höhe gehoben, die andere langte an einen mit vielen kleinen Säckchen behängten Gürtel. Die Temm hatte wiederum ihre Hände gehoben und ließ ein grünliches magisches Glühen von ihnen ausgehen.

Absurderweise war der erste Gedanke, der Aćh durch den Kopf schoss, ob man den Mantel des Fremden wirklich einen Mantel nennen konnte. Die Schulterpartie erinnerte sie eher an mehrere übereinandergeschichtete ornamentierte Tücher. Ein bisschen wie die die Kleidung der Nomaden aus der Wilden Wüste, die sie damals gesehen hatte, als ihre Mutter sie auf eine diplomatische Reise ans andere Ende des Landes mitgenommen hatte.

Am Rücken des Fremden, unterhalb seines Köchers, hing ein schon leicht zerfleddertes Fell im selben Braun wie sein Mantel. Nebst den mysteriösen Säckchen an seinen Gurten waren auch zahlreiche Taschen in Mantel eingelassen, vorne saß gar eine große Bauchtasche, und weiter hinten in der Höhle sah Aćh einen weiteren beladenen Rucksack herumliegen. Der Mann war vollkommen überladen. So konnte man doch nicht umherreisen.

Wie war er nur hierher gelangt? Sie befanden sich einer Höhle im Nestbaum der Takuri, dutzende Meter über dem Boden. Bestimmte hätte jemand gesehen, wenn dieser Fremde das Gerüst erklommen hätte. Sie blickte sich um, sah jedoch nirgends einen Takuri-Spiegel liegen. Nun, es war nicht relevant. Zunächst einmal war wichtig, sicherzustellen, dass vom Neuankömmling keine Gefahr ausging. Er hatte einen Bogen umgeschnallt und trug einen Köcher auf seinem Rücken. Die Pfeile darin könnte er im Notfall auch im Nahkampf nutzen.

Aćh zückte ihr goldenes Schwert und hielt es dem Fremden vor die Kehle, auf dass er nicht auf dumme Ideen komme.

Laut fragte sie: „Wer bist du? Woher kommst du? Was suchst du hier?“

Der Fremde kniff seine Augen zusammen, verzog seine Miene, wartete einen Augenblick und begann dann, mit seinen Händen zu gestikulieren. War er stumm?

Nein, stellte sich heraus. Denn nachdem Aćh ihn eine Zeit lang unverständig angeguckt hatte, ließ der Fremde ernüchtert seine Schultern sinken, legte seine Hand an seine Brust – Handfläche nach außen – und sprach: „Barz.“

Wie eine Begrüßung.

Aćh konnte den Akzent nicht einordnen, ebenso wenig die Sprache, und erst recht nicht die Herkunft des Fremden. Aber feindlich eingestellt wirkte er nicht. Sie senkte ihr Schwert wieder, legte ihre eigene Hand auf ihre Brust und sprach ihm nach:

„Barz.“

Sie versuchte, das fremde Wort zu wiederholen, doch ihre Zunge stolperte über den Zischlaut am Ende. Der Fremde schüttelte freundlich lächelnd seinen Kopf und wiederholte: „Barz. Barz.“

Dabei zeigte er zunächst auf sich, ehe er seine Nase rümpfte und dann auf Aćh zeigte. Jetzt war Aćh noch verwirrter. Sie rümpfte ebenfalls ihre Nase und sprach: „Barzbarz. Barzbarz Barz.“

Der Zischlaut machte ihr immer noch zu schaffen, doch glaubte sie, die fremde Floskel immer besser imitieren zu können. Der Fremde grinste plötzlich breit und gluckste auf. Aćh wünschte sich, ihre Mutter wäre hier. Die war schon immer in diplomatischen Angelegenheiten besser gewesen. Aber nein, Nelímar musste ausgerechnet jetzt auf diplomatischer Mission in der Metropole Agarb sein.

Nun wurde das Gesicht des Fremden plötzlich wieder ernst und er starrte Aćh in die Augen. Dann tat er einige Schritte innerhalb der Höhle und zeigte nacheinander auf alle möglichen Dinge. Dazu gab er Laute von sich, die vermutlich Wörter in seiner Muttersprache waren – was für eine faszinierende fremde Sprache das war, mit einem besonderen Singsang in der Stimme und so wenigen harten Lauten. Yrbstschly ließ ihr bedrohliches magisches Leuchten wieder anschwellen, als der Fremde auf sie zeigte, ansonsten rührte sie sich nicht vom Fleck.

Am Ende zeigte der Fremde auf einen vor der Höhle vorbeirauschenden Takuri – „Fönićs“ – und dann noch einmal auf sich selbst – „Barz.“

Ach so, das war ein Name! Blut schoss in ihre Wangen. Etwas beschämt wiederholte Aćh Barz‘ Prozedere. Auch sie zeigte auf den Takuri vor der Baumhöhle – „Takuri“ und auf Barz – „Barz.“ Dann endete sie mit ihrem eigenen Namen – „Aćh.“

„Agk“, wiederholte Barz. Aćh grinste. Nun war es an ihr, den Kopf wegen einer Aussprache zu schütteln.

Noch ehe Aćh Yrbstschly vorstellen konnte, trat Barz auf einmal ihren Schritt näher. Rasch hatte Aćh ihr goldenes Schwert wieder gehoben, und Barz stand wieder stockstill da.

Da vernahm Aćh ein Geräusch, das verdächtig nach dem Knacken einer Eierschale klang.

Das war seltsam. In Turrs Nest sollten gar keine Eier liegen. Ohnehin war es unglaublich selten, dass ein Takuri ein Ei legte, aktuell gab es am gesamten Nestbaum wohl nur eines. Was war hier los? An Yrbstschlys und Barz‘ verwirrten Blicken erkannte Aćh, dass sie nicht die Einzige war, die das Knackgeräusch nicht einordnen konnte.

Dann blickte Barz an sich herunter und Aćh folgte seinem Blick. Sie glaubte, eine Bewegung in der Tragetasche vor seinem Bauch zu erkennen. Konnte es sein...

Erschrocken und etwas hilflos blickte Barz wieder zurück zu Aćh. Und sie glaubte zu verstehen.

Mit einem vorsichtigen Seitenblick auf Yrbstschly steckte Aćh ihr goldenes Schwert weg und gestikulierte das Abziehen der Tragetasche. In Barz‘ Augen blitzte Erleichterung. Er stellte seine Tragetasche ab, öffnete den Verschluss und zog vorsichtig, ganz vorsichtig, ein knackendes Ei heraus.

Das Ei war wirklich schön, selbst im Vergleich zu den leuchtenden Takuri-Eiern, die Aćh hier am Nestbaum schon gesehen hatte. Seine Oberfläche war relativ glatt und dunkel. In der schimmernden Schale spiegelte sich das in die Baumhöhle fallende Licht. Keine Ahnung, zu welcher Spezies es gehörte, und was dieser Barz damit vorhatte. So groß, wie es war, konnte er es kaum mit einer Hand halten. Unwahrscheinlich, dass er noch weitere bei sich trug.

Aćhs Faszination wurde abrupt von ihrem praktisch orientierteren Geist überdeckt. Das Ei schlüpfte! Was brauchte es? Was brauchte Barz? Sie scharrte ein wenig vom feuerfesten Nistmaterial der Takuri zusammen und gebot Barz, sein Ei dorthin zu legen.

Sorgfältig, wie man ein Neugeborenes in sein Bettchen legte, platzierte Barz das Ei im Nest und trat einige Schritte zurück. Dann blickte er fragend zu Aćh.

„Was? Was brauchst du?“, fragte sie, in der Hoffnung, ihr Tonfall könne irgendwie kommunizieren, was ihre Worte allein nicht konnten.

„Kann sich das Tier von selbst befreien oder müssen wir ihm helfen?“, mischte sich nun auch Yrbstschly ein. Aćh nutzte den Moment, um auf Yrbstschly zu zeigen und ihren Namen auszusprechen, da sie zuvor deren Vorstellung vergessen hatte. Barz kümmerte sich nicht sonderlich darum, sondern blickte sie weiterhin verständnislos an. Aćh versuchte, das Öffnen einer Eierschale mit ihren Händen darzustellen, wurde jedoch von einem lauten Knacken der tatsächlichen Eierschale unterbrochen.

Eine regelmäßige Reihe langer, dünner Nadeln stach synchron durch die Schale des Eis und ließ einen Teil davon abblättern. Barz fiel auf die Knie und beobachtete ganz genau, was da vorging. Er hielt jedoch weiterhin seine Distanz.

Erneut stach die Nadelreihe durch die Schale und blätterte einen Teil davon ab. Gute Güte, waren das Zähne? Aćh erhaschte einen ersten Blick auf einen Ausschnitt des Wesens, welches hier das Licht der Welt erblickte. Graue, faltige Haut. Ein winziges Auge umgeben von einer gräulichen Fleischmasse. Bei die sieben Feuern des Himmels, was war das nur für ein Wesen?!

Dann platzte die Schale komplett auf und etwas purzelte aus dem Ei heraus.

Ein überdimensionierter Kopf, der über einen dicken, mehrgliedrigen Hals in einen massigen Körper überging. Dünne Beinchen, ein langer Schwanz und ein überlanger Unterkiefer. Und dann erst die spitzen Zähne. Ein Raubtier? Eine abartige Echse? Bei ihrem Anblick schüttelte es Aćh innerlich.

Doch als sie Barz‘ Gesicht sah, erkannte sie darin den liebevollen Blick, den Aćh Turr und den anderen Takuri schenkte. Diese groteske Echse bedeutete ihm viel.

Aktuell ging es der Echse alles andere als gut. Ihr überproportionierter Magen warf Wellen und ein leises Ächzen war daraus zu vernehmen, aber keine Atmung. Barz stürzte zur Echse und drehte sie auf ihren Bauch, auf ihren Rücken, im Kreise, klopfte sie ab, tat alles Mögliche, ohne dass es ihr besser ginge. Noch immer röchelte sie mehr, als dass sie Luft holte. Hilflos blickte Barz um sich.

Aćh hatte noch nie ein Lebewesen belebt, geschweige denn eine Echse, von der sie nicht einmal wusste, wo in ihrem Körper sich Herzen befinden könnten. Aber sie war schon einmal dabei gewesen, als eine Temm vom Nestbaum gestürzt war und ihre Großmutter erste Hilfe geleistet hatte. Sie versuchte, die unangenehmen Erinnerungen zurückzuholen. So klein war der gefallene Körper gewesen. Was hatte ihre Großmutter schon wieder gemacht?

Barz tat etwas, was Aćh ganz und gar nicht gut erschien: Er schüttelte die Echse verzweifelt. Jetzt fühlte sie, dass sie in Aktion treten sollte. Sie stürzte nach vorne, hielt Barz‘ Hände fest und blickte ihn fragend an. In Barz‘ Augen schimmerten Tränen, und er nickte hastig. Das war für den Moment eine ausreichende Erlaubnis.

Aćh nahm ihm die Echse ab und hielt sie vorsichtig in ihren Händen. Ihre Haut fühlte sich viel rauer an, als sie erwartet hätte. Und sie war so zerbrechlich, so filigran. Aćh tastete die Seite der Echse ab und spürte Muskeln, weichere Teile, dann etwas, das Rippen sein könnten. Einen Brustkorb? Sie legte ihre Hände an eine gewisse Stelle und wartete einen kurzen Augenblick, ob ihr wirklich nichts Besseres einfiel. Dann drückte sie zu. Und nochmal. Und nochmal. Immer und immer wieder. Und auf einmal begannen die dürren Beinchen der Echse zu strampeln. Ein jämmerliches Geschrei entfloh ihrem winzigen Mund. Dann wurde es still. Ihre Brust hob und senkte sich, zwar unregelmäßig, aber immerhin.

„Es atmet, es atmet!“, rief Aćh, „Doch es scheint schwach. Sehr schwach.“

Barz konnte natürlich kein Wort davon verstanden haben, aber plötzlich war er wieder neben ihr. Er zückte ein kleines helles Säcklein aus seinem Mantel, öffnete es und zog eine Prise glitzernden Pulvers hervor, welches er der Echse ins Gesicht pustete. Ein magisches Glitzern breitete sich über die Echse aus und ihr Strampeln wurde wieder kräftiger. Ihre Seite hob und senkte sich regelmäßig.

Barz‘ sorgenvolles Gesicht wurde zu einer Miene der Freude. Er stieß seine Faust in die Luft und setzte an, Aćh zu umarmen, ehe er innehielt und ihr nur dankbar zuzwinkerte. Dann nahm er die Echse an sich und streichelte ihren massigen Kopf. Ihre lange Zunge leckte sein Gesicht.

„Sabri“, sprach Barz, und zeigte auf die Echse, „Sabri.“

Aćh fragte sich, ob das ein persönlicher Name oder der Name der Spezies war. Dann fiel ihr auf, dass es aktuell komplett irrelevant war. Und auch so bleiben würde, bis sie je auf andere Vertreter dieser Spezies stoßen sollte.

„Sabri“, sprach Aćh nach.

Sie sah den Fremden anderen Augen als noch vor ein paar Minuten. Doch blieb da immer noch ein ungutes Gefühl. Dieses magische Glitzern, das die Echse bei ihrer Heilung überzogen hatte... Damals, vor zwei Monaten, als Turr verschwunden war, hatte Yrbstschly auch etwas von einem magischen Glitzern gesagt. Heute war der Fremde augenscheinlich einfach so in Turrs Nest aufgetaucht. Und wo befand sich Turr?!

Ein noch schlechteres Gefühl breitete sich in Aćhs Magen aus. Was, wenn Turr wieder verschwunden war? Sie hätte Turrs Dose im Aschelager seit dem Missgeschick inzwischen auffüllen sollen. Doch hatte sie das immer weiter hinausgeschoben. Mist! Sie hatte keine Möglichkeit mehr, ihn zu rufen!

Sie befahl sich selbst im Stillen, Ruhe zu bewahren. Beinahe beiläufig zog sie ihre Steinflöte hervor und spielte Turrs Melodie, doch leider erschien der Takuri nicht. Doppelmist.

Naja, nun, wo die akute Notlage der Echse geklärt schien, war es an ihnen, zu ihren Vorgesetzten zu gehen und um Rat zu fragen.\bigskip







Aćh geleitete Barz aus dem Nest und vom Baum runter. Yrbstschly krabbelte auf Aćhs Schultern und warf dem Fremden hin und wieder misstrauische Blicke zu. Doch Barz kümmerte sich nicht groß um sie. Es war ein Wunder, dass er lange genug vom Blick auf die Echse in seinen Armen abließ, um seine Umgebung zu erkennen.

Zunächst warf er einen beunruhigten Blick auf die filigran wirkende Holzkonstruktion, die sich um den steinernen Nestbaum wand und es den Hütern erlaubte, rasch die verschiedenen Nester zu erreichen.

Während Aćh ihn auf die andere Seite des Baums führte, auf dass sie den dortigen Aufzug erreichen konnten, öffnete Barz staunend seinen Mund.

Aćh konnte es ihm nicht verübeln. Für sie war der Blick auf die steilen Felswände des Kuolema-Gebirges ein alltäglicher, und selbst sie war stets beeindruckt davon. Bis hoch in die Wolken ragten die Gipfel des Gebirges, dessen Name ja buchstäblich „Berge in den Wolken“ bedeutete. Das Gestein des Kuolema war in vergleichsweise hellen, fahlen Tönen gehalten. So konnte man kaum erkennen, wo Stein in Schnee und Eis überging, und wo in wabernde Wolken.

An manchen Tagen reichten die Wolken so tief, dass sie bis an den Nestbaum der Takuri reichte. Dur Nestbaum stand stolz und steinern auf den westlichen Ausläufern des Kuolema, gerade an der Grenze zur roten Steppe Tulgors.

Heute aber standen die Wolken hoch, und der Blick auf den Großteil des Gebirges war klar. Nur die Gipfel waren nicht zu sehen. Die Gipfel waren nie zu sehen. Manch einer mochte behaupten, es gäbe nicht einmal welche, und man könnte ihn nicht widerlegen. Aćh vermutete aber eher, dass ein magischer Fluch die Wolken am Auflösen hinderte.

Hier und da waren an den Hängen des Kuolema auch dunkle Eingänge in tiefe Höhlensysteme und Stollen zu erkennen. Darin arbeiteten die Minenarbeiter und entrangen dem gnadenlosen Gestein seine gut gehüteten Geheimnisse. Sie förderten neben Gold, Silber und Eisen auch uralte magische Mera-Steine. Geschickt in die Stollen gelenkte Strahlen roten Mondlichts ließen die Mera-Steine in den Stollenwänden aufglühen, woraufhin sie grob aus dem Felsen geschlagen werden konnten. Die Brocken wurden dann weiter inland über komplizierte Prozesse ganz aus dem Gestein befreit, geschleift und etwa in mächtige magische Amulette und Knochenhelme eingesetzt.

Aćh sah, wie Barz‘ Blick zu dem riesigen Gebäude eine halbe Meile weiter nördlich schweiften, aus welchem Dampf aufstieg.

Die Spiegelschmiede.

Dort wurden neben gewöhnlichen Spiegeln hin und wieder aus der Asche von Feuertakuri wertvolle Takuri-Spiegel hergestellt. Auch diese wurden später weiter ins Landinnere Tulgors geliefert, wo sie elegante Rahmen verpasst kriegten und an hohe Fürsten verkauft wurden.

Während die drei das Gerüst am Nestbaum herunterkletterten, ließ Yrbstschly weiterhin hin und wieder ein bedrohlich wirken sollendes grünliches Glühen über ihre Handflächen wabern, doch wirkte Barz davon eher interessiert als beunruhigt. Insbesondere musterte er fasziniert den goldenen Sand, der aus dem grünlich glühenden Ball der Magie herunterrieselte. Doch vor allem kümmerte er sich um das kleine Echsenwesen, das in seiner Bauchtausche ruhte und leise wimmerte.

Aćh vermutete nicht, dass aktuell eine Gefahr von ihm ausging, ließ ihre Hand aber zur Sicherheit demonstrativ auf dem Griff ihres goldenen Schwerts ruhen. Mit ihrem Daumen fuhr sie über die gemusterte Mondsichel, die als Parierstange diente. Der Mond, insbesondere sein rotes Licht, waren für die Tulgori von hoher Wichtigkeit, sowohl für uralte Traditionen als auch fürs Fördern von Mera-Steinen. Insofern ergab es natürlich Sinn, dass ein zeremonielles Objekt wie dieses goldene Schwert eine Mondsichel enthielt.

Hier am Nestbaum der Takuri wurden solche goldenen Schwerter, Speere und ähnliche Waffen eigentlich nur zu zeremoniellen Zwecken benutzt. Sie waren aus gewöhnlichem Metall gefertigt und nur mit einer dünnen Schicht Gold überzogen, die in einem tatsächlichen Kampf zerbröckelen würde. Doch aktuell waren die goldenen Überzüge der Waffen der Hüter allesamt intakt, und die in die zeremoniellen Knochenhelme eingelassenen Mera-Steine funkelten allesamt in mattem Blau – ein Zeichen, dass sie schon seit langem nicht mehr zum Blutvergießen genutzt worden waren. Die intakte goldene Farbe und die blauen Steine zeugten von den friedlichen Bedingungen hier am Nestbaum. Selbst die Höhlenwichte aus den nahegelegenen Höhlengängen im Kuolema stritten sich höchstens mit den Bergleuten in den Stollen und ließen die Takuri-Hüter in Ruhe.

Dennoch hatte Aćh viel Zeit damit verbracht, die Schwertkunst zu erlernen, um in einem Ernstfall zur Verteidigung des Baumes beitragen zu können. Und weil die Schwertübungen etwas Entspannendes, Anmutiges hatten, wie eine Art Tanz.

Echte Konflikte waren alles andere als das, sondern dreckig und unangenehm, davor hatte sie ihre Großmutter oft genug gewarnt. Oma Òkôkó hatte noch die blutigen Konflikte mit den marodierenden Banditenbanden aus dem Südwesten miterlebt. Sie hatte danach geschworen, nie wieder ein Schwert zu führen. Den Schwur hatte sie bislang halten können. Nun war sie eine der Obersten des Ordens der Hüter und eine gute Ratgeberin. Sie würde wissen, was mit Barz zu tun war.\bigskip







Òkôkó runzelte ihre Stirn und murmelte leise, sie wünschte sich, Aćhs Mutter wäre hier. Als Diplomatin kam Nelímar viel umher in den umliegenden Gebieten, und sie kannte viele Dialekte. Vielleicht hätte sie auch den von diesem Barz‘ erkennen können.

Und ein solches Tier wie diese Sabri hatte Òkôkó noch nie gesehen.

Sie kratzte sich am Kinn.

„Und dein Turr ist immer noch verschwunden?“

Aćh nickte angespannt: „Ich vermute, dass sich der Vorfall von vor zwei Monaten wiederholt hat. Und dass Barz irgendwie involviert ist. Auch wenn er jetzt gerade nicht so wirkt, als würde er einer Fliege etwas zu Leid tun wollen.“

Aćh deutete hinüber zu Barz, welcher gerade mit seiner kleinen Echse umhertollte und ihnen beiden keine Aufmerksamkeit schenkte.

Oma Òkôkó legte ihren Kopf schief: „Nun, es gibt einen großen Unterschied dazwischen, Schaden anzurichten und Schaden anzurichten zu wollen.“

„Verzeih mir“, nickte Aćh, „Ich hätte inzwischen neue Asche von Turr sammeln sollen. Nur wegen mir können wir ihn nun nicht zurückrufen.“

„Ach so? Hätte wirklich kein anderer Hüter die Asche sammeln können?“, meinte Òkôkó schnippisch.

„Das meinte ich nicht. Turr wurde mir anvertraut. Seine Sicherheit und Pflege sollte meine Verantwortung sein. Wenn ihm etwas zustößt, ja, vielleicht gar etwas zugestoßen ist, dann geht das auf meine Kappe.“

„Wir werden klarkommen, was auch immer genau hier los ist. Konnten wir bislang stets“, sprach Òkôkó zuversichtlich, „Doch was wollen wir nun tun? Was tun, was tun?“

„Wir könnten Turr suchen gehen, doch die Welt ist klein für einen Takuri und groß für uns. Ich fürchte mich um ihn. Er hat bislang noch selten von selbst zum Nestbaum zurückgefunden, wenn er einmal zu weit wegflog.“

„Wir bräuchten etwas, das ihn zu uns führt. Oder uns zu ihm führt. Versuche, Kontakt mit diesem Barz aufzunehmen. Finde heraus, ob er weiß, wo sich Turr aufhält.“

Aćh warf einen Blick zurück zu Barz. Dieser hatte aufgehört, mit seiner Echse zu spielen, und musterte fasziniert die vielen eleganten goldenen Reifen um Oma Òkôkós Hals.

„Ich werde mein Bestes geben, doch bin ich nicht zuversichtlich“, sagte Aćh, „Aber wer sonst könnte wissen, wie wir Turr wiederfinden können?“

Oma Òkôkó schnippte mit ihren Fingern: „Der Hüter der Zeit würde es wissen.“

„Ja, der Hüter der Zeit würde es wissen“, nickte Aćh nachdenklich, „Und Mama ist gerade in der Metropole Agarb unterwegs! Wir können ihr einen Falken schicken. Und dann könnte sie für uns mit dem Hüter konversieren, sofern er sich die Zeit nehmen kann.“

„Natürlich, sende den Falken an deine Mutter. Doch Angelegenheiten mit dem Hüter der Zeit sollte man meiner Erfahrung nach lieber persönlich klären. Vier Nächte wollen wir darüber schlafen. Falls Turr bis dann noch nicht zurück ist, schicke ich dich auf dem nächsten Hängeschiff nach Agarb. Wenn Nelímar bis dahin ein bisschen ihre Verbindungen zu den Stadtwachen spielen lässt, befindest du dich zwei, drei Tage später bereits in einer Unterredung mit ihm.“

„Oma, vier ganze Nächte? Turr kann in dieser Zeit alles Mögliche geschehen.“

„Er ist ein Takuri, er wird es überleben. Egal, was für ein Leid ihm zustößt, er kann es wieder vergessen, solange wir ihn zeitig finden. Doch wollen wir auch den Hüter nicht unbedacht stören. Mir scheint, vier Tage des Wartens sind eine angemessene Zeit.“

„Zwei Nächte!“

„Drei.“

„Wie du willst, Oma. Nach drei Nächten brechen wir zum Hüter auf, falls Turr bis dahin nicht zurück ist.“

Aćh neigte ihren Kopf und wandte sich zum Gehen.

„Ah, und kümmere dich um unseren seltsamen Neuankömmling“, meinte Òkôkó beiläufig, „Er hat vermutlich vielleicht mit Turrs Verschwinden zu tun. Wenn er dir irgendetwas verraten kann... nun, wahrscheinlich ist es am geschicktesten, mit ihm zu sprechen zu versuchen. Oder noch besser, nimm ihn gleich mit, wenn du zum Hüter der Zeit pilgerst. Der Hüter würde wissen, was mit ihm anzufangen ist.“

Aćh nickte ebenfalls und winkte Barz zu sich. Barz, der zuvor gerade eine Zeit lang Òkôkós schnell sprechenden Mund angestarrt hatte, vermutlich ohne auch nur ein Wort zu verstehen, eilte geschwind zu ihr. Wenigstens schien er Aćhs Handzeichen zu verstehen.

Bösartig wirkte er auch nicht. Stattdessen folgte er Aćh brav wieder ins Freie, während er mit einer Hand Sabri in seiner Bauchtasche streichelte.

Tja, jetzt hatte sie den Fremden am Hals, bis sie Turr ausfindig machen konnten. Das könnte interessant werden.\bigskip







Aćh hatte Barz rasch in ihrem und Yrbstschlys gemeinsamen Zimmer ein behelfsmäßiges Bett eingerichtet und ihm im hiesigen Chaos etwas Platz freigeräumt, auf dass seinen riesigen Rucksack und einige seiner vielen Taschen ablegen konnte. Nebendran richtete Aćh ein Nest für die kleine Echse Sabri ein, die Barz aber nicht aus seiner Nähe lassen wollte. Immer wieder schielte er vorsichtig zu den Takuri, die von Zeit zu Zeit um den Nestbaum schwirrten. Insbesondere wenn sie Funken sprühten oder ihr Gefieder aufflammte, kniff er beunruhigt seine Augen zusammen. Und immer wieder blickte er auch vorsichtig auf Aćhs Arme, denn unter ihren Armschonern lugte die eine oder andere Brandnarbe hervor. Das Hüten der Takuri war nun mal kein Kinderspiel, insbesondere das Hüten von jungen, wilden, rasch aufgeregten Exemplaren.

Aćh ahnte, dass das viele Feuer ihm Sorge bereitete, und führte ihn darum als nächstes zur Schutzanlage auf der anderen Seite des Nestbaums. Dort könnte er seine Haare und seine Kleidung mit Sufar einreiben. Sufar war eine klare, süßlich riechende Substanz, die einen zumindest eine Zeit lang vor dem Feuer der Takuri schützte.

Barz betrachtete mit äußerst großem Interesse, wie Aćh ihm die Feuerresistenz von Sufar demonstrierte, indem sie ihre Hand ins Sufarbecken tauchte und danach schmerzlos in einen Kessel voller glimmenden Takuri-Kots langte.

Als Aćh ihren eigenen Umhang demonstrativ ins Sufarbecken eintauchte und Barz gebot, dasselbe zu tun, schien dieser jedoch etwas dagegen zu haben, seinen Mantel einzutauchen. Wild gestikulierend öffnete er verschiedene Manteltaschen und daran befestigte Säckchen und zeigte ihr verschiedenste Pülverchen, die er darin aufbewahrte. War er ein Kräutersammler? Würde das Sufar diesen schaden? Aćh entschied, ihn gewähren zu lassen. Er wusste am besten, was sein Mantel brauchte, und nun, wo er das Sufarbecken kannte, könnte er sich in Zukunft selbst damit schützen, wenn er es denn wollen würde.

Die beiden verließen die Schutzstation wieder. Aćh nickte im Vorbeigehen einer alten Temm zu, die mit einem komplizierten Seilmechanismus das Sufar im Becken durchmischte und so vor dem Kristallisieren abhielt. Die Temm nickte zurück und blickte Barz unverhohlen neugierig an. Aćh zuckte nur die Schultern. Beim Abendessen würde es noch genug Zeit geben, Barz vorzustellen.

Die nächsten paar Stunden verbrachte Aćh mit ihrer üblichen Tagesroutine. Takuri füttern und bespaßen, Größen und Gewichte notieren, Dreck wegräumen, alles eigentlich wie gewohnt. Doch konnte sie sich kaum darauf konzentrieren, denn ihre Gedanken schweiften immer wieder zu Turr. Dem Takuri, zu dem sie von allen die tiefste Verbindung aufgebaut hatte. Dem Takuri, der nun schon zum zweiten Mal verschwunden war. Oh, wenn sie nur nicht seine Aschedose verschüttet hätte. Wo befand Turr sich wohl? Ging es ihm gut?

Und stets war Barz dabei, der Aćh auf den Fuß folgte und äußerst interessiert alles begutachtete, was sie tat und ließ, und sei es das Wegschütten feurigen Kots im stinkenden Abfalllager.\bigskip







Während des Abendessens war Barz wie erwartet die Attraktion der Woche, wenn nicht gar des Monats. Hüter um Hüter und Arbeiter um Arbeiter tröpfelte zur Essenszeit in die große Esshalle, wo ein ungewöhnlich großwüchsiger Temm aus einer noch größeren Suppenschüssel schöpfte. Sobald Barz entdeckt wurde, wie er gemeinsam mit Aćh am Rand des Saals saß und seine Suppe löffelte, bewegten sich Hüter um Hüter und Arbeiter um Arbeiter in seine Nähe und versuchten zu erhaschen, worüber er und Aćh sich unterhielten. Irgendetwas, was ihnen vielleicht verraten könnte, wer der mysteriöse Fremde mit der rundlichen Echse war, ohne dass sie ihre Privatsphäre ganz offen brächen.

Leider konnten sich Aćh und Barz überhaupt nicht unterhalten. Aćh versuchte Barz zu fragen, wer er sei, woher er komme und ob er etwas mit Turrs Verschwinden zu tun habe.

Barz hingegen sprach Worte, die ihr nichts sagten, und gestikulierte auf verschiedenste Dinge, mit denen sie wiederum nichts anfangen konnte.

Am Ende beschränkte sich Aćh darauf, auf verschiedene, einfache Dinge zu zeigen und ihre tulgorischen Namen zu nennen, während Barz die Namen wiederholte. Eine gemeinsame Sprachbasis musste etabliert werden, ehe sie sich auch nur halbwegs unterhalten konnten. Im Laufe des Essens gelang es Barz bereits, „Suppe gut“ und „viel Salz“ zu sagen.

Nach dem Abendessen wurden wie üblich Musikinstrumente hervorgeholt, von Flöten über Trommeln bis hin zu doppelhalsigen Streichinstrumenten, die sich über einen komplizierten Mechanismus selbst bezupften. Dann begannen die Feiern und Tanzereien des Abends. Barz schaute mit großen Augen zu, wie die Hüter und restlichen Arbeiter über den Tanzboden wirbelten, während über ihnen am Himmel Feuertakuri vorbeirauschten und fröhliche Laute von sich gaben.

Der Rhythmus der Tänze wurde schneller und einige Personen wagten es, zu Barz zu treten und ihn zum Tanz aufzufordern. Schon beim ersten Mal erhob er sich bereitwillig und ließ sich anleiten. Rasch wurde klar, dass er weder ein Gefühl für den Rhythmus noch für die Bewegungsabläufe hatte, doch ließ er sich davon nicht einschüchtern. Schon bald stolperte er leider über seinen Aufforderer, und obwohl dieser beim Weghumpeln mit zusammengebissenen Zähnen betonte, dass alles in Ordnung und er nicht böse sei, winkte Barz den restlichen Abend alle weiteren Tanzaufforderungen höflich ab. Er verließ die Versammlung, ehe die Musik verklang.

Aćh folgte ihm.\bigskip






Barz lief einige Minuten vom Nestbaum und den Feiern weg. Er beobachtete aufmerksam seine Umgebung, wiederum streng beobachtet von Aćh.

Im Westen führte ein Abhang von den westlichen Ausläufern des Kuolema-Gebirges weit in die rote Steppe Tulgors hinaus. Die Steppe breitete sich von hier bis zum Horizont aus, soweit das Auge reichte. Unterbrochen wurde das allgegenwärtige Rot des Steppengrases nur von der goldenen Farbe der weiten Kornfelder der Bauern, die hier und da angebaut waren, besonders in der Nähe des Gebirges. Hier und da konnte man auch kleine Häuser mit strohgedeckten Dächern und schwarzen Schornsteinen erspähen. Weiter im Norden gab es größere Häuseransammlungen, höhere Burgen und elegantere Brückenbauten geschickterer Baumeister über wildere Flüsse, doch waren diese alle von hier aus höchstens als weit entfernte Silhouetten zu erahnen.

Doch im Gegensatz zu all dem schien sich Barz vor allem für die Position der Sonne zu interessieren. Er suchte aus einer seiner unzähligen Taschen einen Fetzen Papier hervor und versuchte vergebens, mit einem dicken Kohlestift darauf etwas zu notieren.

Nachdem Aćh ihm einen besseren Stift und eine Unterlage besorgt hatte, gelang es Barz tatsächlich, eine rudimentäre Karte zu zeichnen. Rote Steppe, Baum, Gebirge. Sonnenverlauf (es dauerte einige Versuche, bis Aćh diese gekritzelte Kugel und den Pfeil mit der soeben hinter dem Kuolema verschwundenen Sonne verband), Osten und Westen. Die Zeichen, die Barz auf die Karte kritzelte, sagten Aćh nichts, doch mit den Bildern konnte sie etwas anfangen.

Dann zückte Barz ein neues Papier und zeichnete eine zweite rudimentäre Karte. See, Haus im See, Steppe drumrum. Sonnenverlauf, Osten und Westen. Große Wassermassen im Norden, Berge im Süden und welche im Westen, dahinter noch einmal eine Steppe, dahinter noch einmal Berge. Barz tippte wiederholt auf die Siedlung im See und auf sich selbst, dann auf die Karte des Nestbaums und auf Aćh.

Okay, damit konnten sie arbeiten. Die grobe Karte von Barz‘ Heimat sagte Aćh zwar noch nichts, sie war aber sicher ein guter Start.

Aćh ergänzte noch einige Details auf der Karte des Nestbaums. Von hier aus mochte die rote Steppe Tulgors schier endlos scheinen, doch war sie das in Realität natürlich nicht der Fall. Im Südwesten der Steppe lag die Wilde Wüste, welche sich seit Jahrhunderten langsam, aber stetig immer mehr der Steppe einverleibte. Im Süden lag ein weiteres Gebirge. Im Norden die Steilklippen des Ozeans. Und auch die Grenzen Tulgors vermochte Aćh einzuzeichnen.

Das Land Tulgor beanspruchte relativ klare Grenzen für sich. Aller Boden vom Kuolema bis zum Ende der Steppe galt als Reich des weisen Rats der Fürsten. Was weiter draußen lag, kümmerte sie ebenso wenig wie der weite Ozean hinter den steilen Klippen im Norden oder die Reiche jenseits der Berge und der Wüste.

Ob die Tulgori hier in diesen Landen entstanden oder aus Steppe, Wüste, Bergen oder Meer angereist waren, das war eine Frage, die sich nicht so leicht beantworten ließ. Die ersten Schriften, die die Erforscher der Geschichte hatten aufspüren können, stammten bereits aus einer Zeit, als Menschen schon seit mehreren Generationen in den hiesigen Dörfern lebten und sich nicht groß um ihre Herkunft scherten.

Die wenigen Entdecker, die in der niedergeschriebenen Vergangenheit in fremde Lande aufgebrochen waren, berichteten bloß von Riesen. Hinter den Bergen im Osten gab es welche mit großen Hörnern und langen Kinnen. Hinter den Bergen im Süden gab es welche mit kleinen Köpfen, die sich mit Skeletten schmückten. In der Wilden Wüste im Westen gab es welche mit langen Stoßzähnen und Bauchtaschen in ihrer Haut. Und der Ozean im Norden wimmelte ohnehin nur so von gehörnten Riesen, ungeheuren Kreaturen und Gefahren. So waren die meisten Tulgori zufrieden, sich in ihrem fruchtbaren, riesenlosen Fleckchen der Welt niederzulassen und das Schicksal durch weite Reisen nicht allzu sehr herauszufordern.

Aćh ergänzte auf Barz‘ Karte die Metropole Agarb an der Grenze zwischen Wüste und Steppe, die wohl bekannteste Stadt von ganz Tulgor. Den Ozean im Norden. Die breite Gebirgskette im Süden. Dieser Ozean und das Gebirge schienen auch mit der großen Wassermasse und den südlichen Bergen von Barz‘ Karte zusammenzupassen. Zufall, oder lag da mehr dahinter? Lag Barz’ Heimat etwa direkt im Osten oder im Westen von Tulgor? Oder vielleicht gleich weit in beiden Richtungen entfernt? Stammte er von einem Ort gerade gegenüberliegend auf der großen Weltenkugel?

Barz selbst schien die Antwort auf diese Fragen auch nicht zu kennen, denn er vertauschte mehrmals demonstrativ verwirrt die beiden Landkarten, um zu zeigen, dass er ihre relative Anordnung nicht kannte. Anschließend faltete er erstaunlich geschickt zwei kleine Figürchen aus Pergamentfetzen, einen Menschen und einen Vogel. Er setzte den Vogel auf die westlichere Steppe auf seiner eigenen Karte und tippte auf sich selbst, während er den Menschen auf der Karte seines Sees und seiner Steppe setzte. Dann verschob er seine Figur von seinem See auf die Nestbaum-Karte und machte: „Puff.“

Es dauerte noch einige Versuche, bis es ihm und Aćh gelang, einander ihre Ideen zu übermitteln.

Offenbar hatte Barz sich irgendwie hierher teleportiert. Möglicherweise hatte es mit den magischen Pulvern zu tun, die er in Säckchen bei sich trug und von denen er immer wieder welche vorzeigte. Es schien Aćh, als wäre sein Sprung hierher nicht das gewünschte Ergebnis gewesen. Seine wirkliche Absicht konnte sie zum jetzigen Zeitpunkt noch nicht erahnen. Ziemlich sicher war, dass Barz hinter der Verschwinden Turrs steckte. Wobei Barz zu glauben schien, dass Turr sich vor seinem Verschwinden bereits in der Nähe von Barz‘ Heimat befunden habe, in der westlicheren der beiden Steppen. „Andor“, sprach Barz immer, wenn er darauf zeigte.

Zudem schien Barz zu glauben, dass sich Turr nun in Barz‘ Heimat befand. Nun, selbst wenn dies stimmte, so würde dies nur so lange stimmen, bis Turr sich zurückzuteleportieren versuchte, und mangels leitender Flötenklänge womöglich irgendwo weit entfernt vom Nestbaum landete.

Sie hatten keine Ahnung, wo Turr sich aufhielt.

Aćh schüttelte resigniert ihren Kopf. Wie konnte man nur so fahrlässig sein?! Barz hatte Turr unbedacht von hier weggerissen und sich selbst an einen Ort versetzt, den er nicht kannte und an dem er nicht sein wollte. Und Aćh hatte Turrs Asche vor zwei Monaten verschüttet und konnte ihn somit nicht mehr zurückrufen. Eklig, eklig, eklig, diese Angelegenheit.

Als nächstes versuchte sie, Barz zu fragen, ob der Vorfall mit Turr vor zwei Monaten auch sein Werk gewesen war. Sie hantierte mit Turrs Papierfigur auf den Karten herum und spielte ihre Flöte vor – diese Klänge schienen Barz irgendwie bekannt vorzukommen, seine Augen zeigten jedenfalls Erkennen – doch scheiterte die Kommunikation schon daran, dass es Aćh nicht gelang, das Konzept von „Vergangenheit“ zu kommunizieren. Wie sollte das auch gehen, die herkömmlichen Monduhren Tulgors schienen Barz überhaupt nichts zu sagen. Wie man in seiner Heimat wohl die Zeit maß?

Resigniert gab Aćh irgendwann auf und brachte Barz zurück zu seinem provisorischen Bett.

Zeit, „Gute Nacht“ auf Tulgorisch zu lernen. Auch diese Worte schien er sich rasch merken zu können. Er antwortete mit einem Gruß in seiner eigenen Sprache, den Aćh sich wiederum einprägte.

Was für ein Tag!\bigskip







Die nächsten beiden Tage zogen überraschend rasch vorüber. Barz folgte Aćh nicht mehr auf Schritt und Tritt, zeigte sich aber immer noch äußerst interessiert an den Vorgängen rund um den Nestbaum der Takuri. Zum Glück für die beiden war er relativ gut darin, sich neue Begriffe zu merken, und schon bald wurden seine und Aćhs limitierte Konversationen weniger zu einem Spiel der Pantomime und des Zeichnungserratens, sondern ein Spiel der Assoziationen simpler Wortkombinationen.

Barz zeigte Aćh, wie man aus Papierfetzen kleine Papiermenschen, -temm und -vögel zu falten vermochte, und Aćh führte Barz in ein simples, doch in Tulgor sehr beliebtes Spielchen ein, in dem es darum ging, über Würfelergebnisse unter Bechern zu flunkern. Nebenbei war dies auch eine gute Gelegenheit für Barz, die tulgorischen Zahlenbezeichnungen zu repetieren. Auch wenn Aćh manchmal den Verdacht hatte, Barz würde die Würfel irgendwie mit magischen Kräften beeinflussen. Zu oft fühlte Aćh sich bei seinen Ergebnissen von den Gesetzen der Wahrscheinlichkeit hintergangen.

Yrbstschly, die Temm, hielt sich in den überraschend fern von den beiden. Insbesondere war ihr Barz‘ Echse Sabri nicht ganz geheuer, nachdem diese einmal nach ihren Beinen geschnappt hatte.

Umgekehrt war Barz hingegen überaus interessiert an Yrbstschly und den sonstigen Temm. Insbesondere schien er fasziniert am sandartigen goldenen Pulver, das bei vielen magischen Akten der Temm als Nebenprodukt freigesetzt wurde. Aćh ertappte ihn einmal dabei, wie er ein wenig Sand der Temm vom Boden zusammenkratzte und in eines seiner vielen Pulversäckchen füllte.

Ohnehin legte Barz einige ungewöhnliche Verhaltensweisen an den Tag. So demonstrierte er etwa auch großes Interesse am feuerfesten Nistmaterial der Takuri, welches von den Tulgori aus einer Pflanze hergestellt wurde, aus deren Fasern auch feuer- und reißfeste Kleidungsstücke und sogar Fischernetze hergestellt wurden. Eines Nachmittags fand Aćh ihn neben seinem Bett, wie er ein stark riechendes Pulver über eine Probe des Nistmaterials streute und daneben Notizen in fremden Zeichen auf einem Stück Papier machte. Aćh hatte nicht einmal mitgekriegt, dass Barz ein Stück vom Nistmaterial hatte mitgehen lassen.\bigskip







Ein anderes Mal fand Aćh Barz unansprechbar im Schneidersitz in seinem Bett sitzend vor, eine bräunliche Pulverpaste um seine Nase geschmiert. Mit geschlossenen Augen saß er da, reagierte nicht auf Aćhs Rufe und murmelte etwas Unverständliches vor sich hin. Etwas erschrocken und vermutlich unnötigerweise besorgt, rüttelte Aćh an seiner Schulter, bis Barz wieder seine Augen öffnete. Leichte Tränen glitzerten darin, als er sie erneut anstarrte.

Mit seiner Hand umfasste er wie unbewusst eine Halskette, an der zwei klobige Ringe hingen. Einer aus dunklem Holz und einer aus hellem Stein. Während Barz traurig vor sich hinblickte, zeigte Aćh fragend auf die Ringkette. Barz überlegte lange, lief dann zum Fenster, zeigte auf Aćhs Großmutter, welche in einiger Entfernung gerade einen jungen Hüter fürs Verschütten von Sufar schalt, und machte eine umarmende Geste.

Familie?

Familie.

Barz vermisste seine Familie? Baute er durch dieses bräunliche Pulver eine Verbindung zu ihnen auf, und Aćh hatte ihn dabei unterbrochen? Oder half es ihm dabei, sich an sie zu erinnern? Aćh war verwirrt. Konnte Barz sich etwa nicht einfach wieder dorthin zurückteleportieren, woher er gekommen war? Brauchte er Turr dafür? Brauchte er mehr magische Materialien, als er bei sich trug? Steckte er nun in Tulgor fest?

Die Fragen wurden nicht weniger.\bigskip







Aćh hatte einen Durchbruch, als es ihr gelang, Barz die Worte für „Vergangenheit“ zu übermitteln und ihn in ihr Zeitrechnungssystem einzuweihen. Dann endlich konnte sie auf die Ereignisse um Turr vor zwei Monaten zu sprechen kommen. Barz zeigte sich geständig, auch bereits in diese Ereignisse verwickelt zu sein, bastelte aber demonstrativ eine andere Papierfigur mit einem hohen spitzen Hut – oder einer besonders extravaganten Frisur? – mit der er auf den Landkarten etwas darstellte. Der Hinweis schien klar. Für das damalige Verschwinden war eine andere Person verantwortlich gewesen.\bigskip







Einmal wurde Aćh mitten in der Nacht geweckt, weil ein Takuri ihr die kokelnden Überreste einer Steppenmaus aufs Bett erbrach. Aćh verdrehte nur ihre Augen und schlurfte müde zur Putzstation. Barz hingegen machte große Augen und wälzte sich die restliche Nacht lang unruhig auf seinem behelfsmäßigen Bett herum, wobei er immer wieder mal aufstand und vorsichtig nach seiner kleinen Echse Sabri sah.

Als sie am nächsten Tag den Raum verließen und die noch friedlich ruhende Echse zurückließen, versuchte Barz gar, einige lose Holzstücke vor den Eingang zu packen. Vermutlich wollte er verhindern, dass ein Takuri dort eindrang.

Wann würde ihm wohl auffallen, dass es keine Absperrungen oder Zäune um den Nestbaum gab, obwohl die nicht allzu fern liegende rote Steppe Tulgors nur allzu leicht von einem übereifrigen Takuri entzündet werden könnte? Dass die Hüter sich nicht mal die Mühe machten, ihre Futtervorräte vor den Takuri zu verschließen? Die Takuri waren Meister der Teleportation, und die einzige bestätigte Möglichkeit, Takuri an einem Ort zu behalten, bestand darin, diesen Ort so spannend zu gestalten, dass sie von selbst dort bleiben wollten. Darum boten die Hüter den Takuri so viele Streicheleinheiten und Spielmöglichkeiten an. Und darum musste sich Barz keine großen Sorgen machen, dass ein Takuri zu Sabri eindringen würde. Eine schlafende Echse war für die Takuri wohl viel langweiliger, als einander über den Himmel zu jagen und sich gegenseitig Knabberseile abspenstig zu machen.

Das sollte nicht heißen, dass die Hüter gar keine Möglichkeiten hatten, die Takuri im Zaum zu halten. Einmal auf einige Hüter geprägt, hörten die meisten Takuri auf den Klang derer tulgorischer Steinflöten. Und natürlich waren ausgeklügelte Löschanlagen am Rande der Steppe eingerichtet worden, auf dass man etwaige durch jagende Takuri ausgelöste Steppenbrände durchs Öffnen einiger passenden Schleusen im Keim ersticken konnte.

Manchmal fragte sich Aćh wirklich, was die Takuri vor den Hütern gemacht hatten. Wenn sie schon damals so brandfreudig wie heute gewesen waren, hätte die Steppe wohl kaum überlebt. War vielleicht ein sehr mächtiger Temm so freundlich gewesen, ihren großen Nestbaum in Stein zu verwandeln, auf dass er von ihnen nicht stetig niedergebrannt wurde? War dies der Grund, dass der Nestbaum so hart war, dass Pickel und Bohrer an ihm zerbrachen, und dass die Takuri-Hüter ihre Kletterkonstruktionen außerhalb des Baumes hatten erbauen müssen? Fragen über Fragen...\bigskip






Zwischendurch suchten Aćh und Barz auch den besten Kartographen der Minenarbeiter auf, doch auch auf einer professionell hergestellten Karte Tulgors konnte Barz keine Verbindung zu seiner Heimat machen.

Während sie schon in der Mine waren, versuchte Aćh erneut ihr Glück, Turr mit der tulgorischen Steinflöte zu rufen. Leider keine Antwort.\bigskip







Dann, auf einmal, waren die drei Nächte des Wartens auch schon um. Aćhs Mutter Nelímar hatte einen Falken mit einer positiven Nachricht zurückgesandt. Wenn nicht irgendeine abrupte Fürstenkonferenz angesagt würde, konnte sie ihnen eine Audienz beim Hüter verschaffen. Sie konnten aufbrechen. Von den westlichen Ausläufern des Kuolema, wo sich der Nestbaum der Takuri befand, tief ins Land Tulgor hinein.

Auf, nach Agarb!

Auf, zum Hüter der Zeit!

Er würde ihnen sagen können, was sie wissen wollten. Sofern er sie für würdig befände. Nicht, dass er gemein oder hochnäsig wäre, im Gegenteil, er wurde immer als sehr fröhlich und freundlich beschrieben. Doch war seine Zeit nun mal sehr kostbar, und für jeden verschwundenen Schlüsselbund, dessen Position er seinem Besitzer verriet, verstarb möglicherweise ein erfolgreicher Heiler, der in seinem Leben noch zahlreiche Gebrechen hätte heilen können.

Aćh gruselte es etwas, darüber nachzudenken, über wie viel Macht der Hüter der Zeit verfügte. Macht korrumpierte, das sagte Òkôkó immer, und doch schien sie dem Hüter der Zeit nicht zum kleinen Kopfe zu steigen.

Barz, der in den letzten Tagen einmal die Funktionsweise der Takuri-Spiegel hatte aus der Ferne beobachten können, blickte fragend hinüber zur Spiegelschmiede, als Aćh ihn schnurstracks daran vorbeiführte. Dachte er vielleicht, dass sie per Spiegel reisen durften?

Aćh lächelte schief und winkte ab. Für die Tulgori waren Takuri-Spiegel von großem Wert. Selbst ihre Scherben bargen noch besondere Kräfte und ihr Besitz wurde, wie die Benutzung der Spiegel, strengstens reguliert durch die großen Spiegelmagnaten. Per Takuri-Spiegel zu reisen war viel zu teuer für eine einfache Takuri-Hüterin wie Aćh, die für ihre Arbeit am Nestbaum mehr in Kost und Logis denn in Sold entlohnt wurde.

Aćh und Barz würden stattdessen ein Hängeschiff nehmen, dessen Strecke Tulgor vom Osten bis in den Westen durchzog, von ihrem jetzigen Aufenthaltsort bis zur prächtigen Stadt Agarb. Nicht so schnell wie die Spiegel, doch immer noch schneller als die Reise per Kutsche oder Trampeltier. Nur wie würde sie das Barz alles erklären können?

„Geld? Gold?”, fragte Aćh Barz, und hielt ihm einige Goldstücke aus ihrer Tasche entgegen, die sie bei den Zwischenstopps für Verpflegung und Unterkunft ausgeben werden würden. Vielleicht auch für einige Souvenirs.

Barz schüttelte seinen Kopf. Kannte man überhaupt das Konzept von Goldstücken, dort, woher er kam? Oder gar Währungen im Allgemeinen?

Aćh steckte das Gold wieder ein und zeigte auf den gewundenen Weg, der vom Nestbaum der Takuri die bergigen Ausläufer hinunterführte und in ein kleines Dörfchen am Rande der roten Steppe führte.

Dort lebten und ruhten die meisten Minenarbeiter, die von dort aus jeweils für mehrere Tage in die Berge zogen und dort Stollen tief ins Innere des Kuolema-Gebirges gruben.

Die Spiegelschmiede neben dem Nestbaum der Takuri produzierte auch zahlreiche gewöhnliche Spiegel, und diese wurden oftmals eingesetzt, um das Licht des roten Mondes tiefer in die Stollen zu lenken, wenn die Felswände mal dafür nicht glatt genug geschliffen werden konnten.

Heute war es wieder einmal soweit. Der rote Mond stand am Tageshimmel und sein roter Schein wurde über mehrere Prismen vom restlichen Tageslicht getrennt und in die Stollen in die Tiefen des Berges geleitet, wo er Mera-Steine erstrahlen ließ.

Doch das brauchte Aćh und Barz nicht zu kümmern. Aćh streckte ihre Hand aus und lenkte Barz‘ Blick vom gewundenen Weg und vom kleinen Dörfchen am Fuße des Berges zum hohen Halteturm der Hängeschiffe, und von dort über eine regelmäßige Reihe hoher Masten, die vom Halteturm bis weit in die Steppe führten, so weit das Auge reichte. Zwischen den Masten waren stabile Transportseile gespannt, und in weiter Ferne war an einer Stelle ein Konstrukt mit großen weißen Segeln zu erkennen, welches über die Transportseile stetig näher glitt.

Ein Hängeschiff.

Tagein, tagaus, fuhren die Hängeschiffe die Transportseile entlang und transportierten aus den Minen geförderte Mera-Steine und in der Spiegelschmiede produzierte Spiegel vom Gebirge tief in Tulgor hinein, wo sie etwa von reichen Nomaden aus der Wilden Wüste gekauft wurden. Takuri-Spiegel bargen geheimnisvolle Kräfte und waren den Tulgori sehr wertvoll. Und selbst unreine, noch halb im Felsbrocken verborgene Mera-Steine waren als profane Gegenstände in den Sippen der wilden Wüste ungemein gefragt, ob als Stärkebeweis, Willensbeweis oder einfach nur als wertvolle Handelsgegenstände. Ganz zu schweigen von ihrer Beliebtheit als magisch potente Mittel. Jeder Hexer im Lande wollte wenigstens einige wenige dieser Steine bei sich haben, um mit ihnen magische Ströme zu leiten. Man munkelte, in den südlichen Bergen des Kuolema lebe ein alter Magier, der mithilfe einer horrenden Menge an Mera-Steinen jegliche Person ausfindig machen konnte. Dieser könnte Aćhs nächstes Ziel sein, falls der Hüter der Zeit ihnen mit Turr nicht aushelfen konnte.

Auf jeden Fall sorgte die enorme Beliebtheit der Mera-Steine und Spiegel im Lande dafür, dass viele Hängeschiffe die Transportseile entlangfuhren. Für ein geringes Entgelt lieferten sie auch Gäste von Dorf zu Dorf, von Stadt zu Stadt, von Burg zu Burg. Oder von den westlichen Ausläufern des Kuolema bis hin zur Metropole Agarb am anderen Ende Tulgors, wo der Hüter der Zeit residierte. Und dieser würde ihnen hoffentlich verraten können, wo sich Turr aufhielt.

Auweia. Aćhs Kopf schmerzte schon jetzt, wenn sie daran dachte, Barz auch nur einen Ansatz davon hauptsächlich mit Karten und Gesten erklären zu wollen.\bigskip







Die Zeit war gut getroffen. Noch während Aćh und Barz die Treppen zur Spitze des Halteturms erklommen – Barz wunderte sich eine Zeit lang über die kleinen Zwischenstufen, die jeweils ganz links der Treppe angebracht waren, bis er eine Temm beim Aufstieg überholte und wissend auflächelte –, sahen sie über sich ein Hängeschiff einfahren. Auf der obersten Plattform des Halteturms angekommen, reihten sie sich hinter den wenigen anderen Fahrgästen ein und spazierten über einen wackeligen Steg aufs Hängeschiff, während unter ihnen geschäftige Arbeiter Esswaren für die Minenarbeiter und Takuri-Hüter aus- sowie Mera-Steine und elegante Spiegel einluden.

Ein gut gebauter junger Mann mit einem prächtigen dunklen Haarschopf stellte sich wichtigtuend vor Aćh und Barz. Er überragte Aćh um einen ganzen Kopf und streckte ihr gebietend einen zweiteiligen Metallgegenstand hin, den er mehrmals klacken ließ. Einen Locher.

Während der Tulgori Aćhs und Barz’ Fahrkarten lochte, ließ der Mann seine helle Stimme über die anstehenden Fahrgäste erschallen: „Willkommen auf dem Hängeschiff ARCTOR, meine sehr verehrten Herrschaften, Damenschaften und allerlei anderen Mitglieder der Gesellschaft. Mein Name ist Ijs, und da der übliche Kapitän an diesem sonnigen Tag leider mit Magenbeschwerden an sein Bett gefesselt ist, werde ich auf der heutigen Reise nicht nur Ihr liebster Schaffner, sondern auch Ihr liebster stellvertretender Kapitän sein. Gleichzeitig werde ich mich auch als Aussichtsleiter versuchen. Ich kenne die Gegend hier wie meine Westentasche und vermag zu jedem der auf Ihrer Fahrt zu sehenden Bauwerken mindestens drei lustige Fakten abzugeben. Ganz zu schweigen von meinen Kenntnissen über die verschiedensten Brücken, die es aus von hier aus zu sehen geben wird. Freut euch!“

Aćh bedachte den enthusiastischen Schaffner und stellvertretenden Kapitän mit einem betont gelangweilten Blick. Barz hing ihm hingegen fasziniert an den sich schnell bewegenden Lippen, ohne dass er wohl auch nur das geringste Wort verstand. Aćh bugsierte Barz am Kapitän vorbei ins Innere. So jung, wie er aussah, konnte Aćh sich nicht einmal seiner Volljährigkeit sicher sein, geschweige denn seiner Kompetenz. Wobei, wenn die Schiffsgesellschaft einen so jungen Mann bereits mit dem Kommando über ein Schiff überließ, sprach das durchaus für ihn.

„Nicht von der gesprächigen Sorte, sind wir?”, fragte Ijs mit einem übertriebenen Schmollmund.

Aćh winkte mit einem höflichen Lächeln ab. Sie kannte vielleicht gar mehr lustige Fakten über die Bauwerke dieses Landes als er. Und Barz konnte sie beide ohnehin kaum verstehen.

Ijs las ihre Stimmung richtig und wandte sich den nächsten Fahrgästen zu, doch nicht ohne Aćh und ihrem Begleiter freundlich zuzuzwinkern.

Barz blickte immer wieder strahlend in die Ferne. Die Aussicht war wirklich wunderschön. Hin und wieder starrte Barz aber auch unsicher auf den doch ziemlich dünnen, schwankenden Holzboden, auf der er stand, und auf die filigran aussehende Konstruktion von Holz, Seilen und Metallbeschlägen, mit der das Hängeschiff an den gewaltigen Transportkabeln über ihnen hing. Aćh bemerkte den starken Griff, mit dem Barz sich an die Reling des Hängeschiffes klammerte. Beruhigend legte sie ihm eine Hand auf die Schulter. Sie hatte auch einige Reisen mit den Hängeschiffen gebraucht, ehe sie ihnen hatte vertrauen können. Doch dem ersten Eindruck zum Trotz war das System sicher entwickelt worden. Komplikationen auf Reisen waren selten und Abstürze noch seltener.

Dann stellte sich Ijs hinter die Kommandozentrale in der Mitte des Schiffs und klappte einige Hebel um. Gewaltige Segel entrollten sich links, rechts und unterhalb des Hängeschiffs. Ijs warf einen Kontrollblick darauf und hielt seinen Finger in die Luft. Aćh deutete sein Grinsen als gutes Zeichen. Der Wind stand günstig. Sie würden vorerst keinen zusätzlichen Antrieb nötig haben. Und keine einzige Wolke stand am Himmel. Nass würden sie auch nicht werden.

„Nehmt bitte alle Platz, die Anfahrt kann etwas schaukeln“, rief Ijs. Die wenigen Gäste, die noch nicht auf den ordentlichen Sitzreihen Platz genommen hatten, folgten seiner Anleitung. Barz folgte Aćhs Beispiel.

Ijs betätigte zwei weitere Hebel und drehte an einer Kurbel. Die Blockaden an den Rollen am Transportseil über ihnen lösten sich.

Und das Hängeschiff setzte sich ruckelnd in Bewegung.\bigskip



***\bigskip


Die Sonne ging auf über der Roten Pyramide. Der Hüter der Zeit rollte sich von seinem Schlafkissen, befestigte seinen mächtigen Turban auf seinem Kopf und blickte von der Spitze aus die vielen Treppenstufen der Pyramide hinab. An ihrem Fuße hatte sich bereits eine kleine Traube von Besuchern versammelt, doch der Hüter sah sie nicht wirklich. Während er gedankenverloren sein Frühstücksbrot verzehrte und zur Zahnpflege auf einem Kauast herumknabberte, starrten seine Augen ins Leere. Sein Geist forstete durch Jahre von Eindrücken, Bildern und Gesprächsfetzen, die er noch nicht erlebt hatte. Sein Herz raste vor Aufregung, wie schon seit Jahren nicht mehr. Nur noch wenige Tage. Dann würde er seine Worte ganz mit Bedacht wählen müssen. Die Zeitlinie stand auf der Kippe. Und nicht nur die von Tulgor. Die gesamte Welt war in Gefahr, in eine Eiswüste verwandelt zu werden.





\newpage
\section{Aćh und Barz reisen zur roten Pyramide}



Aćh mochte das Reisen sehr. Nichts gegen ihre Zeit am Nestbaum der Takuri, das Arbeiten mit den Feuervögeln war eine sehr lohnenswerte Erfahrung, doch wer konnte schon etwas gegen eine gelegentliche Tour durch die verschiedensten Orte des Landes haben? Wenn es diesmal doch nur unter glücklicheren Umständen hätte stattfinden können.

An Barz’ begeistertem Blick auf alles Temm- und Menschenmögliche, an dem sie vorbeifuhren, von dem elegantesten Kuppelbau eines hohen Lerninstituts bis hin zum simplen Design der öffentlichen Toiletten, sah Aćh wieder einmal, wie großartig die tulgorische Zivilisation doch sein konnte. Manchmal war das leicht zu vergessen in ihrem relativ abgeschiedenen Alltag am Nestbaum. Hier wurde sie auf eine schöne Weise daran erinnert.

Nun gut, eigentlich sah Barz auch alles andere mit Begeisterung an. Wilde Steppenblumen, deren blaue Blüten durch das rote Steppengras tief unter dem Hängeschiff stachen. Sechsbeinige Katzen, die auf der Suche nach Steppenmäusen durchs Gras huschten. Fledervögel, die in Schwärmen über den blauen Himmel zogen.

Schiffe konnten Menschen entlang von Flüssen transportieren, doch nur wenige Flüsse durchquerten die Steppe, und diese waren oft sehr kurvig.

Einst mussten die Urahnen der Tulgori Waren über Transportseile quer durch Tulgor geschickt haben. Diese Seile, gedreht aus derselben hochstabilen Faser, die auch für das feuerfeste Nestmaterial der Takuri verantwortlich war, waren auch heute noch zwischen hohen Masten über der roten Steppe gespannt. Und einst musste jemand mit einem besonders stabilen Schiff auf die Idee gekommen sein, selbiges entlang dieser Masten zu schicken. Die Hängeschiffe, große, bauchige Holz- und Metallkonstrukte mit weiten, in alle Richtungen abgespreizten Segeln, die den Transportseilen entlang über die Steppe huschten, wirkten auf den ersten Blick wie eine realitätsfremde, quirlige Idee, die einem uralten Märchenbuch entsprungen war. Doch die klugen Köpfe und Baumeister Tulgors hatten sie seit ihrem ersten Auftreten stetig verbessert. Heute waren sie ausgestattet mit modernsten Bautechniken und einer Prise magischen Unterstützung, und somit komfortabel, sicher, und vor allen Dingen eines: Schnell.

Bei günstigstem Wind dauerte die Reise von den westlichen Kuolema-Ausläufern zur Metropole Agarb mit den Hängeschiffen nur zwei Tage. Die Nacht würden Aćh und Barz in einer Taverne in Thelot verbringen, einer florierenden Stadt an der Handelsstrecke, wo sich die Transportseillinie der Hängeschiffe mit einem breiten Fluss kreuzten. Dieser Fluss floss aus dem südlichen Tulgor bis hierher und führte von hier aus bis weit in den Norden, wo die rote Steppe in grüne Felder überging. Dort verzweigte der Fluss sich und fiel an der felsigen Steilküste Tulgors in einem prächtigen Wasserfall in die Tiefe.

Aćh war schon einmal dort im Norden gewesen, als ihre Mutter Nelímar im Namen des hohen Rats der Fürsten mit einigen Schafshirten an der Küste hatte verhandeln müsse. Irgendein Landnutzungsdisput, den heute niemand mehr interessierte. Um die grünen Felder des Nordens, wo vielfältige Feldfrüchte und Obstbäume gediehen, lebten heuer nur noch Bauern, Fischer, Schmiede, Baumeister und Weber, relativ auf sich allein gestellt. Der hohe Rat der Fürsten hatte am Ende lieber andere Felder weiter westlich für sich beansprucht, und sein Einflussgebiet ohnehin eher in die Gebiete direkt außerhalb seiner Burgen und Städte zurückgezogen. Städte wie der Touristenmagnet Thelot, an dem Aćh und Barz die Nacht verbringen würden. Zahlreiche gewöhnliche Schiffe und gut ausgebaute Wege führten Passagiere von Thelot in die weiter Inlands liegenden Dörfchen, die alle ihren eigenen Charme hatten. Doch Aćh und Barz waren nicht als Touristen hierhergekommen.\bigskip







Kapitän Ijs verschloss den Zugang zum Hängeschiff hinter den letzten Passagieren mit einem überkompliziert aussehenden Schlüssel, winkte ihnen zum Abschied noch einmal zu und bewegte sich dann in Richtung einer Unterkunft für Hängeschiffmatrosen.

Barz hatte schon kurz nach ihrer Ankunft bereits einen seiner vielen Rucksäcke abgezogen, geöffnet und Aćh ein dünnes Stück Stoff gezeigt, welches wohl als Schlafsack fungieren sollte. Fragend deute er auf ein Stück Grünfläche neben einer hohen Brücke. Aćh lachte und gestikulierte ihm, er solle sein Schlafmaterial wieder wegpacken. Diese Nacht würden sie in einer Wirtschaft in Thelot verbringen. Das Konzept einer Wirtschaft schien Barz eher unbekannt zu sein.

Die Taverne zum Tauchenden Takuri war nicht das edelste Gasthaus, das Thelot zu bieten hatte, doch war sie sicher, gut besucht und das Essen ausgesprochen lecker. Der Name rührte von einer alten Legende her. Angeblich hatte es einst (und eigentlich immer noch) einen Takuri gegeben, der das Wasser geliebt hatte wie kein anderer, und immer wieder in den großen Ozean im Norden vorgedrungen war, trotz aller bösartiger Kreaturen, Seeriesen, Riesenkraken und sonstiger Ungetüme, die sich darin verbargen.

Eines Tages hatte Olrećh, ein gewiefter tulgorischer Safthändler, bei einer spätabendlichen Saftlieferung die falsche Richtung eingeschlagen und war mitsamt seiner gesamten Wagenladung in den Fluss gefallen. Da sei der Tauchende Takuri aufgetaucht und habe ihn aus dem Wasser gezogen. Olrećh sei überglücklich gewesen über sein Überleben und habe dieses Erlebnis als Schicksalssignal gewertet, dass er nicht für ein Leben als fahrender Safthändler geschaffen sei. So habe Olrećh am nächsten Tag mit der Hilfe einiger Fischer die Überreste seines versunkenen Wagens geborgen und daraus einen Stand gezimmert, wo er seinen gloriosen Saft verkaufte. Und der war so beliebt, dass der Stand bald zu einer kleinen Hütte wurde. Viele Reisende kamen vorbei und erzählten, was hier und dort im Lande geschah. Und so wurde bald aus den Überresten des alten Karrens die Taverne zum Tauchenden Takuri.\bigskip







Aćh und Barz konnten sich in einem großen Zimmer im oberen Stock der Taverne Platz einrichten und die Nacht dort verbringen. Es war ein Massenverschlag, doch die anderen Gäste waren angenehm still. Stiller als der Schankraum unter ihren Füßen, aus dem noch bis tief in die Nacht grölende Gesänge erklangen.

Auf einem der vielen Bette lag eine mysteriöse Person in einem grünen Gewand. Sie mampfte geistesabwesend an einer blauen Staude mit kleinen runden Beeren. Blaubachbeeren? Eine Sternkrautblüte war elegant in ihrem langen goldenen Haar platziert. Als sie Aćhs Blick bemerkte, schenkte sie ihr ein Lächeln und sprach schmatzend: „Ach, mich musst du nicht beachten. Ich war nie hier.“ Aćh beachtete sie nicht weiter. Als ihr Blick das nächste Mal auf dieses Bett fiel, war jenes menschenleer.

Barz kniete sich neben seinem Bett hin, öffnete einen Beutel mit einem braunen Pulver und rieb sich eine Prise davon um die Nase, ehe er sich seufzend aufs Bett sinken ließ. Seine Hand umklammerte wieder die zwei Ringe an seiner Halskette, die Barz selbst zum Schlafen nicht abzog.

Am nächsten Morgen erwachte Aćh zum sanften Klingeln einer Weckglocke. Barz lag nicht mehr in seinem Bett, sein Gepäck war bereits ordentlich neben seinem Bett zusammengepackt. Aćh fand ihren Begleiter draußen vor der Wirtschaft, wie er seine Echse Sabri dazu zu bewegen versuchte, ein Büschel Gras zu essen. Das war ja faszinierend, zuvor hatte Aćh nur gesehen, wie Barz sie mit totem Takuri-Futter zu füttern versucht hatte. War die Echse etwa ein Allesfresser?

Frohen Mutes, wenn auch mit der üblichen Stille zwischen ihnen, brachen Aćh, Barz und Sabri auf zum Halteturm eines Hängeschiffs nach Agarb.

An Bord des Schiffs erwartete sie bereits ein breit grinsender stellvertretender Kapitän Ijs.

„Noch seid ihr mich nicht los! Wie es sich herausstellt, suchen die in der Metropole einen exzellenten Schiffslenker und ich will mich bewerben. Nun führe ich dieses Hängeschiff bis nach Agarb weiter.“

„Na, da können wir uns aber glücklich schätzen“, grinste Aćh.\bigskip







Zwei Drittel der Reise von Thelot nach Agarb verliefen relativ ereignislos. Barz blickte immer wieder begeistert in die Umgebung hinaus und sorgenvoll zu Sabri, die in seiner Transporttasche schlief. Aćh wurde plötzlich siedend heiß bewusst, dass sie nie nachgeforscht hatte, ob Tiere auf den Hängeschiffen erlaubt waren, und blickte verstohlen zu Ijs herüber.

Nachdem der sich aber überhaupt nicht um Sabri kümmerte, beruhigte sich Aćh wieder, zückte ein Rätselpergament, welches sie sich an einem Touristenstand besorgt hatte, und machte sich daran, es zu lösen.

Immer wieder fiel ihr eine andere Passagierin auf. Eine junge Temm, die interessiert auf Aćhs Rätselpergament schielte. Als Aćh ihr anbot, sie könne auch einen Blick darauf werfen, wandte sich die Temm jedoch wortlos ab und verzog sich hinter einen übergroßen Koffer, der wohl zu ihr gehörte.

Hin und wieder zückte Aćh auch ihre Takuri-Flöte und spielte hoffnungsvoll Turrs Lockmelodie, leider wie üblich ohne Folgen.

Doch dann, am späteren Nachmittag, zogen plötzlich Wolken auf. Unnatürlich dunkle. Und unnatürlich schnelle. Barz und Aćh warfen einander besorgte Blicke zu. Ijs blickte auf einen sich wild drehenden Sturmhahn und schraubte mit gerunzelter Stirn an einem Hebel, der das Segel des Hängeschiffs etwas einklappen ließ. Die restlichen Passagiere rückten enger aneinander.

Dann ruckelte das Hängeschiff, stockte und stoppte gar vollständig. Gemurmel machte sich breit, während Ijs vergeblich an zwei anderen Hebeln herumrüttelte.

„Mir scheint, dass unser Schiffsmast irgendwie blockiert wurde. Verzeiht die Unannehmlichkeiten und macht Euch keine Sorgen, wir werden die Lage bald gelöst haben“, versprach er den Reisenden.

Aćh warf einen Blick über die Reling des Hängeschiffs. Zwischen dem schaukelnden Hängeschiff und dem Boden lagen mehrere Dutzend Meter freier Fall. Der rote Steppenboden darunter war viel weiter entfernt als üblicherweise. Es war eigentlich ein Vorteil der Hängeschiffe, dass sie auch über hügelige Gefilde reisen konnten, wo eine Kutsche einen großen Umweg schlagen müsste. Doch dass das Schiff nun ausgerechnet über einem tiefen Tal hatten sie stecken bleiben müssen, gefiel ihr ganz und gar nicht.

Der anziehende Sturm, der ihr durch die Haare rauschte, stärkte ihr Vertrauen in die Sicherheit ihres Gefährts auch nicht. Sie fluchte.

Kapitän Ijs lief zu einem komplexen Kasten und tippte in bestimmten Mustern auf einen kleinen Schalter. Aćh kannte die Muster. Der Code des Seefahrers Morseus. Damit konnten Nachrichten rasch entlang der Transportseile geschickt werden, wenn ein Hängeschiff in Problemen steckte. Und ebenso rasch würde eine Antwort aus Agarb zurück zu Ijs gelangen.

Ijs richtete sich auf und versuchte offensichtlich, Professionalität auszustrahlen. Laut proklamierte er: „Bitte bewahren Sie Ruhe. Verstärkung aus Agarb ist auf dem Weg hierher und wird uns hier abholen. Und selbstverständlich werden Sie allesamt für die Unannehmlichkeiten kompensiert werden.“

Die Sturmwolken wurden dunkler und dichter. Die ersten Regentropfen fielen aufs Holzdeck. Passagiere zogen ihre Kapuzen hervor und murmelten etwas davon, dass es doch noch vor so kurzem nach strahlend schönem Wetter ausgesehen hatte.

„Arb“, zupfte Barz an Aćhs Ärmel und zeigte beunruhigt auf den sich verdunkelnden Himmel, „Arboduk.“

Aćh blickte ihn verständnislos an. Barz zückte einen Pergamentfetzen und skizzierte eine Gewitterwolke. Dann zeichnete er ein wütendes Gesicht darin.

„Arboduk.“

„Ein bösartiges Gewitter?“

„Leben. Arb.“

„Ein lebendiges Gewitt... oh. Oh. Uh-oh. Ein Sturmgeist? Bist du dir sicher?“

Ein Blitz zuckte über den wolkenverhangenen Himmel. Wind wehte über das steckengebliebene Hängeschiffdeck und riss Barz den Papierfetzen aus der Hand.

Barz nickte bestürzt. Er zeigte auf verschiedene Stellen im Himmel, wo Aćh nur dieselben grauen Wolken sehen konnte. Nun, sie konnte ihm wohl vertrauen, dass er sich mit Sturmgeistern besser auskannte als sie. Naturgeister waren in Tulgor früher häufiger anzutreffen gewesen als heute, und so kannte Aćh sie vor allem als Akteure in alten tulgorischen Märchen, die sie auch heute noch gerne verschlang. Schwer zu verstehen waren die Naturgeister, und noch schwerer zu besänftigen, wenn sie etwas aufgescheucht hatte. Wenn das hier wirklich ein wütender Sturmgeist war, war ihre Lage alles andere als rosig.

Aćh quetschte sich nach vorne zu Kapitän Ijs durch und raunte ihm zu: „Was könntet Ihr tun, wenn ein wütender Sturmgeist kurz davor wäre, zu wüten?“

Ijs riss seine Augen auf: „Seid Ihr sicher?“

In diesem Moment zuckte eine weitere Serie von Blitzen über den Himmel. Diesmal konnte auch Aćh darin zwei wütende Augen erkennen. Der Donner folgte beinahe augenblicklich auf das helle Licht und brüllte etwas über das Hängeschiff, das mit etwas Fantasie nach „trOpfEn prAssEln AUf fEUchtE ErdE“ klang. Sprach der Geist? Was hatte dies zu bedeuten?

Ijs wich einen Augenblick verschreckt zurück. Aćh erinnerte sich daran, wie jung er doch noch sein musste, und zweifelte daran, ob sie ihn hätte behelligen sollen. Doch dann wurde sie eines Besseren belehrt, als Ijs sich räusperte, sich aufrecht hinstellte und gestikulierte: „In den Laderaum! Alle, rasch! Bringt Euch unter Deck in Sicherheit!“

Mit einem Zug an einem Kontrollhebel fielen die Segel des Hängeschiffs vollends in sich zusammen. Mit einem zweiten Zug eines Hebels öffnete Ijs am Heck eine versteckte Luke in den Laderaum voller Mera-Steine, wertvoller Spiegel und dergleichen teurer Ware.

„Es dürfte etwas eng werden, aber dafür seid ihr sicher vor der Witterung“, rief Ijs und scheuchte die meisten Passagiere zur Luke. An Aćh gewandt fuhr er fort: „Hüterin vom Nestbaum, wisst Ihr, wie man einen Takuri-Spiegel bedient? Wir haben einen an Bord. Damit könnten wir im Notfall vielleicht von hier verschwinden.“

Aćh schüttelte traurig ihren Kopf: „Die sind sehr wertvoll und nicht unser Fachgebiet. Ich habe ja noch nicht mal je einen berühren dürfen.“

Der Regen wurde ebenso stärker wie der Wind. Aćhs Stimme wurde übertönt von einem tiefen Grummeln des Sturmgeists: „blItz sprIngt zwIschEn dUnklEn wOlkEn!“

Seine Stimme klang zugleich nach Donnergrollen und einem protestierenden Menschen.

Barz bewegte sich nicht zum Heck des Schiffs, wo der Eingang zum trockenen Lagerraum wartete, sondern packte Sabri sicher in seine Bauchtasche und eilte zum Bug. Er warf einen Blick in die Tiefe unterhalb des Hängeschiffs und klammerte sich an die Reling. Dann griff er in eines seiner Pulversäckchen. Er warf eine Prise eines dunkelblauen Pulvers in die Luft, welches sich im magisch glitzernd im umherwirbelnden Wind verteilte. Das Heulen des Windes ließ etwas nach.

„Was tut er da?!“, fragte Ijs Aćh, die letzten Passagiere unter Deck winkend.

„Keine Ahnung, aber besser, als den Sturmgeist auszusitzen, dürfte es allemal sein“, meinte Aćh und rannte ebenfalls nach vorne zu Barz. Ihr goldenes Schwert zückte sie nicht. Wenn Barz‘ Bogen nichts dagegen ausrichten konnte, konnte ein Schwert das wohl erst recht nicht.

Ijs rannte ihr kopfschüttelnd hinterher. „Kommt nicht in Frage, dass der Kapitän sich unter Deck versteckt, während einige Passagiere einen Sturmgeist weiter reizen!“

„Wart Ihr nicht bloß ‚stellvertretender Kapitän‘?“

„Ist diese Unterscheidung jetzt wirklich von Relevanz?!“

„Sie ist, wenn Ihr wegen falschem Stolz etwas Unüberlegtes tut.“

Da blieb Ijs still.

„Was tust du da?“, fragte Aćh Barz, welcher immer noch Pulver in die Sturmluft warf, trotz des peitschenden Regens und der immer heftiger zuckenden Blitze.

„Gute Nacht. Gute Nacht Sturmgeist. Pulver nicht gut“, suchte jener nach Worten.

„Du willst ihn zum Schlafen bringen? Oder zumindest beruhigen?“

„In den Märchen war es auch immer eine bestimmte Tat oder ein Umstand, der die Sturmgeister aufregte“, meldete sich der stellvertretende Kapitän Ijs zu Wort, „Und man konnte sie dadurch besänftigen. Rasch! Was regt einen Sturmgeist auf?“

„Das ist die Seele eines Sturmwindes. Sie ist ungestüm und unberechenbar. Wer kann schon wissen, was genau in ihrem Geist vorgeht? Was sie aufregen könnte?!“, rief Aćh gegen den prasselnden Regen.

„Pulver? Magie? Böse?“, schrie Barz einige ihm bekannte tulgorische Worte in den Wind. Er konnte sich nicht genau ausdrücken, doch war das wohl auch nicht wichtig. Sie hatten keine Ahnung, was den Sturmgeist aufgeregt hatte. Und was auch immer Barz‘ Pulver tun sollten, sie taten es nicht. Oder nicht genug.

„Mag er vielleicht keine Takuri-Spiegel oder Mera-Steine?“, wandte sich Aćh an Ijs.

„Die vergangenen Fuhren wurden nie angegriffen!“, protestierte Ijs, „Wenn schon, dann sollten sich lieber Euer Freund seiner Pulver entledigen.“

„Zufall?“, warf Barz nun ein.

Aćh kauerte sich auf Deck.

„Ich sehe ihn nun auch!“, rief Ijs und zeigte in den Himmel. An einer Stelle hatten sich die Wolken zu einem besonders dunklen Fleck zusammengezogen, und die ununterbrochen umherzuckenden Blitze legten stets dieselben Bahnen zurück. Eine lange Blitzspur, die wie ein großer Mund wirkte, und darüber zwei kurze, die wie zwei Augen waren. Ein Gesicht im Sturm. Es schien wütend.

„Was willst du von uns?!“, schrie Aćh ins Gesicht des Sturmgeistes. Der Wind heulte nichts Verständliches, doch verstärkte er sich urplötzlich, zerrte an ihrem flatternden roten Umhang, riss sie von den Füssen und ließ sie einmal quer übers Deck schliddern, bis sie unschön gegen eine Sitzbank krachte.

Aćh blieb einen Augenblick benommen liegen und kauerte sich gegen das nasse Holz, während sie ihre Gedanken zu ordnen versuchte. Dann fiel ihr Blick auf einen großen Koffer an der Reling, der von den fliehenden Passagieren nicht mit unter Deck genommen worden war. Und auf die kleine Temm, welche sich dahinter verstecken versuchte.

„Hier rüber!“, rief sie Ijs und Barz zu und hangelte sich an der Reling entlang zur kleinen Temm.

„Hallo“, begrüßte Aćh die Temm vorsichtig.

Die Temm antwortete nicht und starrte entschieden an Aćh vorbei. Diese erkannte ein Namensschild, das mit einem Band an der Kleidung der Temm befestigt war und im Wind flatterte. „Trortra“ stand in großen Buchstaben darauf.

„Hallo, Trortra“, sprach Aćh. „Komm unter Deck, hier oben ist es gefährlich.“ Trortra die Temm antwortete immer noch nicht und drehte nun sogar ihren Kopf von Aćh weg. Doch Aćh hatte die Furcht in ihrem Gesicht bereits gesehen.

Hinter ihnen heulte der Sturmgeist noch heftiger aus. Der Koffer, hinter dem sich Trortra versteckte, wackelte gefährlich. Ja, das gesamte Hängeschiff wurde immer heftiger durchgeschüttelt.

„Verzeihung, Trortra, doch hier bist du nicht sicher“, sprach Aćh und packte die Temm mit beiden Händen, um sie in Richtung des sichereren Laderaums zu befördern. Da strampelte Trortra plötzlich auf, gab protestierende Laute von sich und streckte ihre kleinen Arme nach ihrem riesigen Koffer aus. Irrte sich Aćh, oder ertönten aus dem Koffer ebenfalls leise Laute?

„Bring sie unter Deck!“, rief Aćh Ijs zu und übergab ihm Trortra.

„Was macht sie denn noch hier oben?!“, antwortete der Kapitän. Er übernahm die protestierende Trortra von Aćh und hechtete mit ihr zur trockenen Luke.

Indes wandte sich Aćh ihrem mysteriösen Koffer zu. Durch einen verschwommenen Vorhang des vom Himmel runtergeschleuderten Wassers hindurch erkannte sie mehrere Verschlüsse an der Seite des Koffers, als bestünde er aus mehreren übereinander gestellten Teilen, die einzeln geöffnet werden konnten.

Aćh zog den Verschluss am obersten Kofferteil beiseite und enthüllte... einen Käfig. In dessen hinterster Ecke kauerte eine kleine, braunfellige Gestalt, die mit einem Schwanz nach Aćh stieß. Ein Schwanz, dessen harte Spitze dunkel schimmerte. Ein junger Skorpionfuchs aus der Wilden Wüste. Äußerst giftig, und sicher nicht ohne Deklaration transportierbar. Eine Skofu-Schmugglerin? Aćh blickte Trortra anschuldigend hinterher, doch die Temm war bereits von Ijs unter Deck befördert worden.

Konnte es sein, dass der Sturmgeist sich über den Skofu ärgerte? Hatten Sturmgeister etwas gegen Gifte? Oder gegen die typischerweise wasserlose Wüste, die einen Regensturm sicherlich schreckte?

Aćh verschloss das oberste Kofferfach wieder fest und zog das nächste auf. Sie erblickte einen giftigen Sumpffrosch in einem Glasgefäß. Das wurde ja immer heiterer. War die Temm eine Gifthändlerin? Arbeitete sie für eine?

Aćh zog den Vorhang vor einem weiteren Kofferteil zur Seite. Kaum hatte sie das sich darin befindende wieselähnliche Wesen mit blau blitzenden Schuppen gesehen, warf der blauschuppige Wühler auch schon eine Schar Stachelhaare von seinem Kopf in Richtung von Aćhs Hand. Rasch zog Aćh selbige zurück und verdeckte den Blick auf das Wesen wieder, doch hatte sie genug gesehen. Ein Ferun von den Feldern im Norden Tulgors. An der dortigen Steilküste waren sie äußerst ungern gesehen, doch hier schien offenbar jemand Interesse an ihm zu haben. Aćh ächzte, während sie einige harte Stachelhaare aus ihrer Hand klaubte. Nicht ohne Grund hatte man diese Spezies den Kopfstachlern zugeordnet.

Eins musste sie der Schmugglerin lassen, sie schien über ein äußerst breites Angebot zu verfügen.

Ijs winkte ihr von der Luke zum Laderaum her zu, und Aćh schlidderte ihm den Koffer übers rutschige Hängeschiffdeck entgegen. Er verfehlte einige Sitzbänke nur knapp.

„Pack den Koffer sicher unter Deck! Da sind Tiere drin! Aber pass auf, sie sind giftig!“

„Was?“, rief Ijs durch das Heulen des Sturmgeists zurück.

„Gifttiere!“

„Was?“

„Gift!“

„Saft?!“

„Gift!!!“

Ijs schüttelte seinen Kopf unverständig und reichte den Koffer einem anderen Fahrgast ins Hängeschiff hinein. Das war wohl in Ordnung, es würde schon niemand auf die Idee kommen, den Koffer unbedacht zu öffnen.

Nach dem Fund von Trortra wollte Aćh doppelt und dreifach sicher gehen, dass sie keinen anderen Fahrgast auf Deck übersehen hatten. So kämpfte sie sich den Wogen des Windes entgegen die Sitzreihen entlang und suchte nach weiteren versteckenden Fahrgästen. Sie fand niemanden. Alle waren sicher unter Deck. Alle außer ihr, Barz und Ijs.

Nicht, dass es den anderen besser ergehen würde, wenn das Hängeschiff sich vom Transportseil lösen und in die Tiefe stürzen sollte. Schon jetzt knarzten die Verbindungsmasten gefährlich stark. Und noch immer steckte das Hängeschiff über einem tiefen Steppental fest. Ihr Fall würde lang dauern und schmerzhaft enden. Wie lange konnten die Masten des Hängeschiffs halten?

Barz blickte ebenfalls besorgt hoch auf die knarzenden Masten. Er zückte ein grünes Pulversäckchen, rieb eine Prise daraus zwischen zwei Fingerspitzen, bis es knisterte, und schleuderte die Pulvermenge in die Höhe, an einen der beiden Schiffsmasten des Hängeschiffs.

Magisches Knattern ertönte und ein hellgrünes Glühen breitete sich von der Zielstelle aus. Dann gab es einen Ruck, und der Mast stand auf einmal stockstill in der Luft fest. Rund um die grünlich dampfende Stelle, an der das Pulver den Mast getroffen hatte, wirkte es, als fielen die Regentropfen langsamer. Ja, manche Tropfen ganz nah am glühenden Mast schienen gar völlig stillzustehen. Egal, wie stark der Sturmgeist am Rest des Schiffs oder an den Transportseilen ruckelte, der grün glühende Teil des Masts wich keinen Fingerbreit. Er stand buchstäblich in der Zeit eingefroren in der Luft fest. Der Sturm tobte, und doch hing ihr Schiff plötzlich wieder ruhig da.

Mit einem triumphierenden Grinsen auf dem Gesicht drehte sich Barz zu Aćh zurück. Sie wollte gerade ihre Daumen hochstrecken, da knarzte es unschön hinter Barz. Ein langer gezackter Spalt zog sich durch den unteren Teil des gebannten Masts, wo kein magisches Glühen herrschte. Er war angebrochen! Obwohl der Mast selbst noch still in der Luft stand, schaukelte das Hängeschiff darunter wieder im Sturm. Aćh und Barz fielen auf ihre Knie. Plötzlich war Kapitän Ijs wieder auf Deck, packte Aćh bei den Schultern und schrie ihr entgegen: „Dein Begleiter soll nicht den Mast bannen, so bricht er nur das Schiff auseinander! Aber wenn er stattdessen dieses magische Pulver auf ...“

„Ja, ja, sag es einfach ihm selbst!“, rief Aćh zurück und schubste den Kapitän weiter in Richtung Barz. Sie selbst huschte zur Luke zurück und warf einen Blick unter Deck.

Große Mengen der Ladefläche war mit Holzkisten bedeckt, welche sicher einmal schön aufeinander gestapelt gewesen waren. Nun herrschte das reine Chaos, doch war dies ebenfalls insignifikant. Regen tröpftelte durchs Deck hindurch und befeuchtigte die Passagiere, doch war dies auch nicht wichtig. Wichtig war, dass sie allesamt in relativer Sicherheit und unverletzt waren.

Auch Trortra, die Temm, befand sich im Laderaum, und versteckte sich mit einem schuldbewussten Blick hinter ihrem großen Koffer. Aćh stellte erleichtert fest, dass der Koffer immer noch verschlossen war. Aćh winkte ihr zu. Trortra schüttelte ihren Kopf.

Aćh blickte von der Luke zurück zum regnerischen Äußeren des Hängeschiffs. Regen und Hagel peitschte wie eh und je über das Deck und ließen das Holz bedrohlich knarzen. Ijs und Barz schrien einander einige Dutzend Schritte von der Luke an. Ijs deutete in eine bestimmte Richtung. Barz griff in sein grünes Pulversäcklein, zog eine weitere Prise Bannpulver hervor, setzte zum Wurf an und...

Eine weitere Blitzsalve jagte über den düsteren Himmel. Donner grollte. Der Sturmgeist riss sein gewaltiges Maul auf und brüllte: „wInd jAgt gEtIEr dUrchs stEppEngrAs!“

Ein unvergleichlicher Windstoß fuhr auf das Hängeschiff herunter. Er riss Sitzbänke beiseite. Ijs schlidderte gegen einen Mast und sank bewusstlos zu Boden. Sein langes im Sturm wehendes Haar war das einzige, das sich noch an ihm rührte.

Doch schien sich der Windstoß hauptsächlich auf Barz konzentriert zu haben. Der Nomade wurde über Bord geschleudert. Aćh schrie auf.

Der Sturmgeist riss Barz in die Höhe. Barz‘ Mund öffnete sich zu einem Schrei, den nur er selbst über den Sturm hinweg hören konnte. Sein Mantel bauschte sich auf, Taschen und Säckchen öffneten sich, verschiedenfarbig glitzernde Pulver verteilten sich durch die Luft und wirbelten in anmutigen Mustern durch die Regenschleier. Hier und dort knisterte und knatterte es, ein Glühen oder eine magische Flamme erwachte, nur um gleich wieder zu erlöschen.

Es sah wunderschön aus.

Dann hatte der Sturmgeist genug von Barz. Der Nomade wurde in hohem Bogen vom Hängeschiff weggeschleudert. Im freien Fall stürzte er der roten Steppe im Tal tief unter ihnen entgegen.

Aćh zögerte einen Augenblick. Dafür, dass Barz fürs Turr Verschwinden verantwortlich war, hatte sie ihn in den letzten Tagen überraschend ins Herz geschlossen. Doch hieß dies nicht, dass wegen ihm leichtfertig ihr Leben wegwerfen würde.

In diesem Augenblick wurde Barz wieder von einem Windstoß erfasst und in die Höhe geweht. Er jagte schreiend am Hängeschiff vorbei, und etwas löste sich von ihm. Eine kleine, graue, halslose Echse rutschte aus Barz‘ Bauchtasche, an seinen nach ihr greifenden Fingern vorbei, und rauschte selbst schreiend durch die Luft. Eine Echse, der Aćh schon einmal das Leben gerettet hatte.

Etwas in ihrem Kopf klickte, und ohne weiter zu überlegen zückte sie ihre Steinflöte und sprang über die Reling des Hängeschiffs.

Regen und Hagel schlugen ihr von allen Richtungen zugleich ins Gesicht, als sie im freien Fall auf Sabri zuhielt.

Sabris ledrigen Körper bekam sie sanft zu fassen.

Aćh drehte sich umher und suchte die Sturmschwaden unter ihr nach Barz ab. Sie sah seinen rasch fallenden Körper tief unter ihr dem Boden entgegenstürzen. Es wirkte nicht so, als könne sie ihn noch rechtzeitig erreichen. Und hatte er etwa seinen Körper zu einer Kugel zusammengerollt? Nein, damit würde er doch nur noch schneller fallen!

Da hörte Aćh über den Sturm hinweg ein bekanntes magisches Knattern. Ein grünliches Glühen breitete sich über Barz‘ zu einer Kugel zusammengerollten Körper aus. Sein Fall wurde verlangsamt, bis Barz mitten in der Luft erstarrt hing. Bewegungslos hing er da, mitten im Fall eingefroren in der Zeit.

Sein Bannpulver.

Barz hatte sich selbst gebannt und so seinen Fall aufgehalten.

Kaum war Aćh bewusst geworden, dass Barz‘ Rettung nicht mehr ihre höchste Priorität war und sie mit der kleinen Sabri aufs Hängeschiff zurückkehren konnte, fiel ihr auf, dass dies keineswegs so einfach sein würde wie gedacht. Regen und Wind zausten an ihren Haaren und ihrem langen Umhang, über sich hörte sie Passagiere schreien und den Sturmgeist heulen, unter ihr knatterte Barz‘ Bannpulver, und in ihrer Hand krächzte eine hilflose kleine Echse.

Aćh ließ alle Geräusche auf sich einfließen und zu einem dumpfen Hintergrundrauschen werden. Sie atmete tief durch, einmal, zweimal, bis sie ihren meditativen Mittelpunkt gefunden hatte. Dann handelte sie. Rasch, aber nicht unüberlegt.

Aćh griff an ihren Hals und löste das mittlere Element ihrer bronzenen Halskette auf. Im Behältnis glitzerte eine Prise Takuri-Asche. Asche von Saarćhan, der ältesten und riesigsten aller Feuertakuri, die schon seit Wochen am Nestbaum im Sterben lag. Nun musste Aćh hoffen, dass sie nicht in den zwei Tagen seit ihrer Abreise ihr Leben gelassen hatte und zu einem Küken geworden war. Und dass sie noch kräftig genug für einen letzten Teleport war.

Mit der einen Hand zog sie ihre Steinflöte hervor, mit der anderen kippte sie die kleine Dosis von Saarćhans Asche in die speziell dafür angefertigte Einbuchtung der Flöte. Jetzt musste es rasch gehen, ehe der Sturmgeist die Asche in alle Himmelsrichtungen verstreute. Oder ehe Aćh auf dem immer näher kommenden Boden einschlug. Aćh griff nach einer der beiden zeremoniellen Takuri-Federn, die ihren Umhang zierten, löste sie und brach sie mit einer Hand in zwei Stücke. Ihr Umhang, nicht mehr länger an ihrer Rüstung gehalten, flatterte davon. Die gebrochene Feder hingegen glühte leuchtend auf und sprühte Funken. Aćh presste sie in die Steinflöteneinbuchtung.

Die Funken genügten, um Saarćhans Asche in der Steinflöte zu entzünden. Rauch stieg auf und immer mehr Ascheflocken wurden mit leisem Plopp an den weit entfernten Ort gesogen, an dem sich Saarćhan aufhielt. Die Verbindung stand.

Verzweifelt blies Aćh in die Flöte und spielte eine Lockmelodie, betont besorgt und dringlich. Sie hängte auch einige dissonante Töne an, die ihre Not und Lebensgefahr ausdrückten.

Die letzten Flötentöne waren noch nicht einmal verklungen, da entzündete sich aus dem Nichts ein riesiger Flammenwirbel in der Luft unter Aćh. Ein dumpfes Tröten erklang. Mächtige Schwingen, jede beinahe so lang wie Aćh selbst, entrollten sich aus dem Wirbel. Die Flügel begannen, dem Wetter entgegenzuschlagen. Wind zupfte an ihren Federn und Regen prasselte auf ihr Feuer ein, doch die riesige Takuri ließ sich davon nicht beeindrucken. Saarćhan war gekommen!

Aćh stürzte der Takuri entgegen, die sich ihrerseits fallen ließ, um ihre Geschwindigkeit an Aćhs anzugleichen. Aćh prallte ungelenk auf Saarćhans Rücken, krallte sich an ihren Hals und zupfte Federn aus. Jedes Stück entblößter Haut brannte und schmerzte, denn Saarćhan stand buchstäblich in Flammen und Aćh hatte keinerlei schützendes Sufar aufgetragen. Aber es würde gehen müssen.

„Danke, danke, Saarćhan, alte Heldin“, rief Aćh durch zusammengebissene Zähne hindurch.

„Kurjo, Saarćhan!“, wies sie die Takuri an und deutete auf das Hängeschiff über ihnen. „Kurjo!“

Wie zur Bestätigung stieß Saarćhan ein melodisches Kreischen aus, dann schlug sie mit ihren mächtigen Schwingen und erhob sich in die Luft. Das golden schimmernde Vogelwesen brauchte nur wenige Flügelschläge, um die Höhe des Hängeschiffs zu erreichen. Langsam umschwirrte sie das Schiff, während der Sturm ihre brennenden Federn zerzauste.

Oben angekommen, rollte sich Aćh vom Rücken der uralten Takuri aufs nasse Holzdeck des Hängeschiffes. Ihre verbrannte Haut protestierte, doch Aćh zwang sich, aufzustehen. Noch war es nicht vorbei. Sie suchte den Himmel nach dem Sturmgeist ab, und sah sein blitzendes Gesicht an dem magisch gebannten Mast vorbeizucken.

Aćh legte die Hände wie einen Trichter an den Mund und rief dem Tier ein neues Kommando zu: „Saarćhan?! Advaria! Advaria meza!“

Dann zeigte sie auf das Gesicht des Sturmgeists.

Saarćhan hob kurz den Kopf, als wolle sie zeigen, dass sie verstanden hatte. Majestätisch schwang sie sich höher in die Luft und jagte direkt auf den Sturmgeist zu. Sie breitete ihre gewaltigen Schwingen weit aus und ließ sie dann mit aller Kraft vor ihrem Körper zusammenklatschen. Im selben Moment schien ein großer Ball feurigen Lichts vor der Takuri zu explodieren! Der gewaltige Feuerball hüllte den Sturmgeist ein und war so hell, dass Aćh geblendet wegsehen musste.

Der Sturmgeist schrie auf und donnerte in seiner tiefen Stimme: „schAttEn flÜchtEn vOr vErschlIngEndEm flAmmEnmEEr!“

Als die Flammen sich legten, plumpste etwas vom Himmel und knallte hart aufs Holzdeck. Etwas belämmert vom Aufprall hockte dort nun ein kleines, piepsendes Küken, das mit großen Kugelaugen in die Welt schaute.

„Oh, Saarćhan, bist du verletzt?“, rief Aćh.

Als die Takuri ihren Namen hörte, watschelte sie eilig zu Aćh. Diese blickte ängstlich nach oben. Falls der Sturmgeist von Saarćhans Feuerball nicht vertrieben worden war, hatte sie wahrlich keine Ahnung, was sie nun noch gegen ihn tun könnten.

Doch es hatte gereicht.

Die dunklen Wolken wichen wieder strahlend blauem Himmel, als hätte dieser Vorfall nie stattgefunden.

Der Sturmgeist war davongezogen, so schnell wie er erschienen war.

Aćh sah mit der Sonne im Rücken in die rote Steppe hinaus und erblickte einen wunderschönen doppelten Regenbogen.

Die Passagiere trauten sich langsam wieder aufs Deck hervor. Selbst Kapitän Ijs, der unschön gegen den Mast gestürzt war, hob benommen seinen Kopf und murmelte: „Was ist geschehen?“

Aćh rannte zu ihm und tastete seinen Kopf und Körper nach Verletzungen ab. Als er zudem meinte, dass ihm weder schlecht sei noch dass er verschwommen sähe oder sich ihm die Welt drehe, atmete Aćh erleichtert auf.

„Das war unglaublich!“, rief sie aus und schlug ihm begeistert auf die Schulter. Diese Aufregung! Die Spannung! So etwas habe ich schon seit langem nicht mehr gefühlt!“

„Ooookay“, murmelte Ijs und rubbelte seine Schulter, „Schön, dass du etwas Schönes draus ziehen konntest. Das Hängeschiff steckt immer noch fest. Und wo befindet sich dein pulveriger Kumpel? Wir werden hier ausharren müssen, bis Verstärkung aus Agarb eintrifft. Mann, was hat den Sturmgeist nur so in Rage versetzt?“

Ijs blickte unruhig über die Reling des Hängeschiffs hinaus, ob der Sturmgeist nicht plötzlich doch noch zurückkehren wollte.

„Wer kann das schon so genau sagen? Wir haben ihn vertrieben! Wir sind mächtig!“

Aćh jubelte glücklich. Sie balancierte ein Takuri-Küken und eine junge Steppenechse auf sich. Dann blickte sie über Bord und begann sich Sorgen darum zu machen, was geschehen würde, wenn Barz‘ Bannpulver zu wirken aufhörte.\bigskip







Tatsächlich hing Barz die gesamte Zeit bis zum nächsten Sonnenaufgang in der Schwebe zwischen dem Hängeschiff und dem Erdboden. Er befand sich nur einige Meter über dem Steppengras. sein Bannpulver hatte gerade noch rechtzeitig zu wirken begonnen.

In der Zwischenzeit erreichten andere Schiffe von Agarb aus die Stelle, wo die ARCTOR feststeckte. Die restlichen Passagiere wurden mit langen Leitern aus dem Hängeschiff in die Steppe gerettet und Ijs versprach ihnen allen einen Drink in Agarb auf seine Rechnung. Zwei äußerst offiziell und äußerst grimmig dreinblickende Speerträger warteten bereits auf den Boden und versprachen „weitgehende Untersuchungen“ dieses Unglücks. Die Ladung des Hängeschiffs wurde mit einem mobilen Kran in ein zweites eingetroffenes Hängeschiff umgeladen und weiterverschifft. Und ein großes Luftkissen wurde unter den eingefrorenen Barz geschoben, für wenn er wieder zu fallen begänne.

Die sehr offiziell wirkenden Speerträger stellten sich als stolze Ordnungshüter heraus, die umherliefen und Anwesende nach dem Vorfall ausfragten, während sie sich Notizen in langen Schriftrollen machten.

Schon bald erschienen über weitere Hängeschiffe von Agarb auch die ersten sonstigen Personen. Verwandte und Bekannte von Reisenden, Helfende und Schaulustige.

Die kleine Temm, Trortra, wurde abgeholt von einer in einen langen Mantel gehüllten Temm, die sie schluchzend umarmte und herzte.

„Ach du liebe Güte, was ist denn hier geschehen? Geht es dir gut? Tut dir etwas weh?“

Aćh fiel aber auch auf, wie rasch die abholende Temm einen größtmöglichen Abstand zwischen die Ordnungshüter und Trortra brachte. Und als Aćh sich einige Minuten später erneut nach ihnen umsah, waren die beiden plötzlich wie vom Erdboden verschluckt. Trortras Koffer voller seltener und giftiger Tiere stand hingegen immer noch da und wurde kurz darauf von den Ordnungshütern konfisziert.

Nachdem alle Zeugenberichte des Vorfalls aufgenommen worden waren und nachdem Aćh berichtete, dass sie leider nicht wusste, ob das Bannpulver den Schiffsmast und Barz von selbst wieder freigeben würde, war Aćh die Einzige, die noch bei Barz blieb. Die restlichen Passagiere zogen mit einem anderen Hängeschiff davon. Ijs klopfte ihr zum Abschied auf die Schulter und versprach, dass er ihr diese Heldentat nie vergessen würde.

„Mein Vater Saro meint immer, dass man keine gute Tat unbelohnt lassen soll. Ich verschaffe euch beiden Gratisfahrten auf den Hängeschiffen auf Lebenszeit“, meinte Ijs zwinkernd, während er mit den restlichen Passagieren davonzog. Aus der Ferne hörte Aćh noch, wie er erneut alle, die es wollten, auf eine Runde Bier in Agarb einlud.

Eine Ordnungshüterin blieb eine längere Zeit bei Aćh und unterhielt sich mit ihr über dies und das, bis auch sie sich entschuldigte und meinte, sie müsse nach Agarb zurück. Wer könne schon wissen, wie sich dieser Bann lösen ließe? Mit etwas Pech säßen sie noch eine ganze Woche hier, ohne dass etwas geschah. Aćh versprach ihr, dass sie sich spätestens beim Sonnenuntergang von der Szene lösen würde.

Zwischendurch tauchten einige Techniker der Hängeschiffgesellschaft auf, untersuchten das von Barz magisch gebannte Schiff, das immer noch regungslos an seinem Transportseil hing, kratzten sich an ihren Kinnen und reisten dann wieder zurück.

Und Aćh saß einfach da, betrachtete die Umgebung, streichelte das Saarćhan-Küken und Sabri und hoffte darauf, dass Barz‘ Bannpulver sich irgendwann in Luft auflösen würde. Sie konnte nicht einfach von hier weg. Sie könnte zwar den Hüter der Zeit auch ohne ihn aufsuchen, doch wer würde sich dann um Barz kümmern?\bigskip







Nelímar, Aćhs Mutter, reiste kurz vor Sonnenuntergang in einem Hängeschiff an. Sie brachte Verbandszeug, Decken und einen Riesenkorb voller Vorräte mit sich. Ihre sonst stets makellose Frisur saß etwas schief und ihr langes festliches Gewand in elegantem Schnitt war vorne falsch geschnürt. Hastig sprach sie: „Ich bin sofort aufgebrochen, als ich davon gehört hatte! Du Arme! Ihr alle Arme! Einen wütenden Sturmgeist, das gab es schon seit Jahren nicht mehr.“

Sie schüttelte ihren Kopf und machte sich sofort daran, die Brandspuren auf den Oberarmen ihrer Tochter zu verarzten.

„Du kommst mit mir, junge Dame!“, meinte Nelímar, „Ich lasse zwei meiner Sekretäre antreten und Wache halten. Sie werden berichten, wenn sich bei diesem Barz etwas tun sollte, und könnten ihn danach sofort zu uns führen. Falls er morgen immer noch hier feststeckt, lasse ich einen Druiden rufen, der sich mit solchen Zeitanomalien auskennt. Zunächst einmal bringe ich dich jedenfalls in die Diplomatengemächer, päpple dich auf und gucke, ob wir trotz der heiklen politischen Lage eine Audienz beim Hüter der Zeit bekommen können. Wie klingt das?“

Abgesehen von der „heiklen politischen Lage“ klang das großartig, musste Aćh zugeben. Sie protestierte noch etwas, sie wolle bei Barz bleiben, sah dann aber ein, dass sie hier kaum mehr erreichen konnte als zwei erheblich wachere Menschen.

So wurde Aćh per Hängeschiff nach Agarb gebracht und von Nelímar in ein Gästezimmer im Diplomatenquartier geführt. Sie konnte ihre durchnässte zeremonielle Rüstung in trockene, vorgewärmte Diplomatenkleider eintauschen. Und kaum war sie unter weiche, schwere Decken geschlüpft, fiel sie auch bereits in einen tiefen, traumlosen Schlaf.\bigskip







Aćh erwachte erst wieder zur Mittagszeit, als die Sonne schon hoch am Himmel stand. Ihre Gedanken liefen nur langsam wieder an. Das war doch nicht ihre Decke... das war doch nicht ihr Bett... das war auch nicht ihr Zimmer... der Sturmgeist! Agarb! Barz!

Erschreckt aus dem Bett strampelnd, stolperte Aćh beinahe über das Saarćhan-Küken und die kleine Steppenechse Sabri, die sich balgten. Oder besser gesagt: Saarćhan hopste Sabri nach und versuchte, auf ihren Rücken zu klettern, während die Echse sich wälzte und das Küken abzuschütteln versuchte.

So fasziniert vom Blick auf die beiden Kleinen war Aćh gewesen, dass sie beinahe übersehen hätte, dass Barz im nächstgelegenen Bett schnarchte. Sein durchnässter Mantel hing an einem eleganten Stuhl daneben. Offenbar hatte man auch ihm ein modisches langes Diplomatenkleid angeboten. Offenbar hatte sich sein Bannpulver doch noch gelöst. Oder war gelöst worden.

Am Frühstückstisch – nun, wohl eher zum Mittagsmahl – erzählte ihre Mutter ihr, was ihre Sekretäre miterlebt hatten.

Im Licht der ersten Sonnenstrahlen über dem Horizont hatte sich der grüne Schimmer des Bannpulvers leuchtend weiß verfärbt und vollständig verflüchtigt. Barz’ Fall hatte sich wieder beschleunigt und er war sanft ins Luftkissen geplumpst. Prompt habe er nach Sturmgeist, Schiff und Sabri gefragt, aber Kommunikation hatte sich als schwierig herausgestellt. Am Ende war er ihnen folgsam, doch nervös, nach Agarb gefolgt und sei überglücklich gewesen, als er erfahren hatte, dass Aćh seine Sabri gerettet hatte.

Sie hatten nicht einmal einen Druiden den Bann untersuchen und lösen lassen müssen.\bigskip






Bald trat der wache Barz auch schon selbst aus seinem Zimmer. In seinen Händen spielte er mit einem goldenen Armreif und zwei goldenen Ohrringen, die vermutlich zu seinem Gewand passen sollten. Stach man sich denn keine Ohrlöcher in seiner Heimat?

Interessiert betrachtete Barz die Augengläser auf Nelímars Nase, durch die sie irgendein diplomatisches Dokument musterte. Dann, noch etwas wackelig auf den Beinen, trat Barz an Nelímar vorbei und stellte sich vor Aćh.

Nur noch drei kleine Pulversäcklein hatte er mit etwas Seil an Knöpfen seines Diplomatengewands befestigt: Ein grünes, ein dunkelbraunes und ein weißes. Waren dies alle Pulver, die der Sturmgeist gestern nicht vernichtet hatte? Das grüne kannte sie bereits, das war dasjenige Pulver, das den Mast des Hängeschiffs und auch Barz‘ Fall selbst gebannt hatte. Die anderen beiden waren ihr noch unbekannt.

Barz ignorierte die leckeren Speisen auf dem Tisch, zückte das weiße Pulversäcklein und legte es sanft in Aćhs Hände.

„Danke“, sagte Barz, „Danke Sabri. Und Verzeihung. Verzeihung Takuri. Turr. Unglück nicht wollen. Turr finden. Helfen Turr finden. Verzeihung.“

Das waren vermutlich die meisten tulgorischen Worte, die Barz je auf einmal gesprochen hatte.

Ein Dank dafür, Sabri gerettet zu haben. Eine Entschuldigung dafür, Turrs Verschwinden ausgelöst zu haben. In Form... eines seiner wenigen übrigen Pulversäcklein? Aćh öffnete das Säcklein und nieste, als ein bisschen goldener Pulverstaub in ihre Nase aufstieg.

Barz schwenkte seine Hand zu seinem grünen Pulver und meinte: „Bannpulver Gefahr. Schwierig. Viele Menschen Pulver nicht gut. Nicht das da. Das gut. Gleichgewicht Magie.“

„Ich... öhm... danke. Danke, Barz, für dieses großzügige Geschenk. Ein magisches Pulver. Wie edel. Und mach dir keine Sorgen. Wir werden Turr finden.“

„Du böse ich. Verstehen. Gut. Turr finden.“

Was sollte das nun heißen?

„Was ist das für Pulver?“, fragte Aćh, auch um das Thema zu wechseln.

Barz legte seinen Kopf schief und schien nach Worten zu suchen. Nun, das war absolut verständlich, es hätte Aćh gewundert, wenn er es einfach so hätte ausdrücken können. Barz kratzte seinen Barz, zückte ein Stück Papier und skizzierte rasch ein Bild eines Wesens, das Aćh unbekannt war. Halb Fischschwanz, halb Strichmännlein. Ein Gestaltwandler? Vermutlich lebte es in Barz‘ Heimat.

„Nixe“, sprach Barz, und zeigte auf das Wesen.

„Nixe“, wiederholte Aćh.

„Nixe“, wiederholte Barz.

„Nixe“, wiederholte Aćh. Sie war sich nicht sicher, wie gut ihre Aussprache war, ebenso wie Barz sich vermutlich nicht bewusst war, wie gut die seine war. Sie mussten einander nur verstehen können. Der Rest würde sich mit der Zeit ergeben. Falls sie tatsächlich mehr als diese paar Tage miteinander zu tun haben würden.

Das war also die Nixe.

Wenn Aćh die Zeichnung richtig interpretierte, sonderte dieses Wesen offenbar Staub ab, den man einsammeln konnte und... nun, so richtig appetitlich wirkte das nun wahrlich nicht. Nixenstaub?

Barz gestikulierte eine Wurfbewegung und sprach: „Pulver. Stark Feind. Du schwach. Du stark. Muss man mögen.“

Er nickte mit einem breiten Grinsen. Aćh grinste ebenfalls, ein wenig aus Reflex. Sie mochte nicht ganz verstehen, was das Pulver, aber „wirf gegen starke Feinde zum Schwächen“ reichte ihr für den Moment. Wobei, wie stark konnte der Nixenstaub schon sein, wenn Barz es ihr einfach so anvertraute? Vielleicht war es viel mehr eine symbolische Geste.

Egal. Es war an der Zeit, beim Hüter der Zeit eine Audienz zu ersuchen.\bigskip




Die Metropole Agarb war ein riesiger Flickenteppich von Behausungen und Grünflächen, umzogen von einer hohen dunkelroten Mauer, die die Steppe draußen hielt.

Verschiedene Kulturen stießen hier aufeinander, von den Steppenwanderern über verschiedene soziale Schichten von Stadtbewohnern bis hin zu Karawanenführer aus der Wilden Wüste weiter westlich.

Inmitten des kunterbunten Sammelsuriums, das die Metropole Agarb war, stand eine hohe Steinpyramide, ziegelrot und uralt. Sie war eine der vielen architektonischen Besonderheiten der Metropole. Einer der Gründe, warum viele Interessenten der Architektur von nah und fern hierher strömten, um diese Besonderheiten zu studieren, die Baumeister der neueren Werke zu treffen und allgemein von ihnen zu lernen. Und diese Pyramide war der Sitz des Hüters der Zeit.

Sie hatte keine Spitze, sondern war oben flach. Vor einigen Jahrhunderten hatte der Rat der Fürsten von Tulgor beschlossen, dort eine Sitzsäule darauf einzurichten, eigens für den tulgorischen Hüter der Zeit. Dieses hohe Amt war extra erstellt worden, um einen gewissen, damals noch sehr jungen Temm offiziell in die bürokratische Anlage der Großstadt einzuordnen. Denn der Rat der Fürsten war nicht zufrieden gewesen damit, dass jemand einen derart großen Einfluss auf das Stadtgeschehen ausübte, ohne einen offiziellen Platz in ihrer politischen Hierarchie zu haben.

Heute waren nur noch sehr wenige der damaligen Fürsten am Leben, und auch den Hüter der Zeit zierten viel mehr Falten als damals. Doch noch immer ruhte er den Großteil seiner Zeit meditierend auf der Spitze der Säule der Roten Pyramide, und noch immer übten seine Ratschläge einen enormen Einfluss auf das politische Geschehen in der Metropole, ja, in ganz Tulgor aus.

Der Hüter der Zeit hatte seit Geburt eine besondere Gabe gehabt. Sein Geist erinnerte sich nicht an das, was bereits geschehen war, sondern an das, was erst noch kam. Und so ersuchte jeder, von Fürsten im Zoff bis hin zu Kindern, deren ihr Kätzchen davongesprungen war, eine Audienz beim Hüter. Und der Hüter widmete jedem, der bis zur Spitze der Roten Pyramide kletterte, einen angemessenen Moment, um ihnen auszuhelfen.

Freilich, schon seit Jahrzehnten konnte nicht einfach so jeder Dahergelaufene die Pyramide erklettern. Die Zeit des Hüters war schließlich kostbar, und so kümmerten sich Stadtwachen darum, sicherzustellen, dass Würdige als Erste die Pyramide erklimmen durften. Leider bedeutete das in Realität, dass die Reichen und Mächtigen für ihre teils kleinlichen Bedürfnisse zuerst zum Hüter vorgelassen wurden als eine Bettlerin, für deren Familie es um Leben und Tod gehen mochte.

Als Protestreaktion auf diese Ungerechtigkeiten gab es nun schon seit einigen Jahrzehnten den Brauch, zum Fuße der Roten Pyramide zu pilgern, aber sich nicht bei den Stadtwachen für Zulass zu bewerben, sondern schlicht einige Tage beim Fuße der Pyramide zu ruhen, in der Hoffnung, der Hüter der Zeit möge einen von selbst zur Spitze rufen. Denn wessen Urteil als das des Hüters selbst könnte schon besser abschätzen, wer seiner Zeit am würdigsten wäre?

So kam es, dass der Hüter immer häufiger jemanden vom Fuße der Pyramide nach oben rief, um seinen Rat zu empfangen. Doch wenn man sich mit den Stadtwachen gut verstand, waren diese immer noch eine sicherere Möglichkeit, vor den Hüter treten zu dürfen.

Grundsätzlich hatten es Diplomaten wie Nelímar auch wirklich gut mit den Stadtwachen. Leider war aktuell gerade einiges an politischem Geschehen im Gange. Der Rat der Fürsten hatte vorgestern gleich zwei Mitglieder an ein heißes Fieber verloren, und nun buhlte halb Tulgor um ihre Positionen.

Soeben war Fürstin Regilda, ein hochrangiges Ratsmitglied, mit einer ganzen Kolonne an Bediensteten aus ihrem Schloss in der roten Steppe zur Roten Pyramide angereist. In Anbetracht ihrer vergangenen Besuche war abzuschätzen, dass ihr Aufenthalt den Hüter der Zeit für mindestens zwei volle Tage in Anspruch nehmen würde.

Aćh hätte es natürlich nicht viel ausgemacht, zwei Tage zu warten, doch bis dahin würden höchstwahrscheinlich schon die nächsten Ratsmitglieder für eine Unterredung eingetroffen sein. Und wenn Ratsmitglieder den Hüter für sich beanspruchten, rief er so gut wie nie Bittsteller vom Fuße der Pyramide zu sich hoch. Die Lage sah nicht gut aus für ihr Anliegen, Turrs Aufenthaltsort zu erfahren

Nichtsdestotrotz führte Nelímar Aćh und Barz an diesem Nachmittag zur Roten Pyramide. Nachdem sie sich einige Souvenirs gekauft hatten, darunter eine kleine Statue des buckligen Hüters und ein kleines Holzmodell der Roten Pyramide, das man wie ein Puzzle auseinandernehmen und – unter größerer Denkleistung – wieder zusammensetzen konnte, machten sie sich auf den Weg zum Vorplatz am Fuße des Gebildes.

Wie zu erwarten hatte Fürstin Regildas Ankunft die meisten Hilfesucher abgeschreckt. Nur wenige Tulgori hatten sich auf dem großen Vorplatz verteilt, aßen vor Marktständen und spielten Kartenspiele. Keiner sah hingebungsvoll zur Spitze der Pyramide hoch, wie es sonst oft zu sehen war. Hin und wieder blickte jemand nervös zu den golden eingekleideten Bediensteten der Fürstin, die vor der Treppe Wache standen.

Und da war sie, Fürstin Regilda höchstpersönlich. Die Goldene Zentaurin. Soeben kämpfte sie ihren schweren Pferdeunterkörper unter größter Mühe die Treppe zur Pyramidenspitze hoch. Immerhin musste sie nicht getragen werden. Warum der Hüter der Zeit wohl all jene, die ihn besuchten, die Pyramide erklimmen ließ, selbst wenn es für sie eine Tortur darstellte? Manch ein armer Tropf hatte in der Vergangenheit gar hochgetragen werden müssen.

Die Sonne glänzte auf Fürstin Regildas Körper und Kleidung. Beide sahen aus, als wären sie in ein Bad aus Goldfarbe getaucht worden. Und wie üblich wirkte es, als sähe Regilda Aćh direkt in die Augen, ihr aus dieser Entfernung nur verschwommen erkennbarer Kopf in einem unnatürlichen Winkel verdreht. Die Fürstin hatte eine unheimliche Präsenz, so viel stand fest.

Barz blickte verschreckt drein, während er die Fürstin betrachtete.

„Augen. Sehen. Ich. Ich.”, sprach er flüsternd. Auch er musste die Fürstin so sehen, als würde sie ihm stets direkt ins Gesicht blicken.

„Illusion”, schüttelte Aćh ihren Kopf beruhigend, „Es sieht für alle nur so aus, als würde sie sie direkt ansehen. Weiß der Himmel, wie oder warum sie das tut, aber es ist nur eine Illusion.”

In diesem Augenblick ertönten hastige Schritte von der Spitze der Pyramide. Eine ältere Frau – vermutlich hatte sie dem Hüter vorhin seinen Nachmittagssnack gebracht – eilte die Treppenstufen hinunter, hielt vor der goldenen Fürstin an und verneigte sich tief. Von unten an der Pyramide war es unmöglich zu vernehmen, worüber die beiden sich unterhielten. Hier und dort sah Aćh Wartende von Spiel und Speise aufsehen, um dem eigenartigen Geschehen einen verstohlenen Blick zuzuwerfen.

Dann öffnete die Fürstin ihren Kopf. Ihre tiefe Stimme hallte laut über die Ebene, als sie sprach:

„Sind hier eine Aćh und ein Barz anwesend?! Wer auch immer ihr sein mögt, tretet vor und sprecht mit dem Hüter der Zeit! Und macht gefälligst schnell!“

Aćhs Herz machte einen Purzelbaum und rutschte in ihre Magengegend zugleich. Fürstin Regilda hatte sie angesprochen! Fürstin Regilda klang verärgert! Fürstin Regilda wurde vom Hüter der Zeit vertagt, damit er mit ihr und Barz sprechen konnte!

Sie packte den verwirrt dreinblickenden Barz am Ärmel und eilte nach vorne, auf dass sie so rasch wie möglich die Pyramide erklimmen könnten. Sie wagte es nicht, einen Blick auf die Fürstin zu werfen, als die beiden an ihr vorbeikletterten.

Der Hüter der Zeit hatte sie zu sich gerufen.\bigskip

***\bigskip

Ein kleines Vögelchen lag zitternd auf einer meterdicken Eisschicht und fror. Hier herrschte überall Schatten. In diese Schluchten fiel nie direktes Sonnenlicht, unter anderem wegen der dicken Wolkendecke, die stets den Blick auf den Himmel verdeckte. Die hellste Lichtquelle waren die leuchtend goldenen Stichflammen, welche hin und wieder aus dem schwachen Körper des Vogels hervorbrachen und einen Teil des Eises schmolzen, auf dem er lag. Doch dadurch sank der unterkühlte Takuri nur noch ein bisschen tiefer in seine Eiswasserlache ein.

Leise Schritte waren zu vernehmen. Sie verstummten, als sich eine Gestalt über die Lache beugte. Eine klirrende, weibliche Stimme ohne jegliche Gefühlsregung bohrte sich in den Kopf des Feuervogels:

„Na, was haben wir denn hier?“









\newpage
\section{Der Hüter der Zeit weist zum Ewigen Eis}

Aćh hatte den Hüter der Zeit noch nie persönlich gesehen, bloß von ihm gehört. Auf den ersten Blick sah er wie ein ganz gewöhnlicher Temm aus, ein kleines buckliges Männchen halt. Er trug einen hohen Turban, der seinen ohnehin schon überproportionierten Kopf noch viel größer erschienen lässt. Darunter lugten zwei verschmitzte Augen hervor. Der Blick aus diesen Augen wirkte auf Aćh, als würde der Hüter direkt in ihre Seele blicken.

„Wir sind uns noch nicht begegnet und doch kenne ich euch bereits“, sagte der Hüter mit krächzender Stimme. „Man nennt mich den Hüter der Zeit und dieser Aufgabe gehe ich nach. Und ich spüre, dass ich durch euch beide die Zeitlinie ganz besonders hüten kann.“

Aćh machte große Augen. Sie und Barz, sie beide waren besonders relevant für den Verlauf der Zeitlinie?!

Der Hüter der Zeit, vielleicht im Glauben, Aćh wolle seiner Aussage widersprechen, fuhr fort: „Da wirst du mir einfach glauben müssen. Ich nehme die Zeit anders war als ihr euch vorstellen könnt.“

Das hatte Aćh bereits gewusst und gedacht. Nichts läge ihr ferner, als dem Hüter nicht zu glauben. Dass er das nicht erkannte, bedeutete, dass er nicht allwissend war, ja, nicht einmal unfehlbar. Aber das hatte sie eigentlich schon gewusst. Der Hüter der Zeit erinnerte sich an die Zukunft, und sorgte dafür, dass sich die Zeitlinie möglichst wünschenswert entwickelte. Nicht mehr, nicht weniger.

„Verzeiht, dass wir Eure Zeit in Anspruch nehmen, und danke, dass Ihr...“, setzte Aćh an, doch der Hüter unterbrach sie.

„Verzeih mir, aber diese Unterredung wird kürzer dauern als diejenige mit der edlen Fürstin, und ist von ähnlicher Relevanz. Lassen wir das höfische Geplapper sein und fallen wir mit der Tür ins Haus. Was ist Euer Belangen?“

Aćh legte ihren Kopf schief: „Wisst Ihr das nicht bereits?“

Der Hüter gluckste: „Nun, selbst wenn mein Geist sich meiner Antwort bereits bewusst wäre, müsste die Frage nicht immer noch gestellt werden, damit ich mich an sie erinnern konnte, nicht?“

Was sollte denn das nun heißen? Nun, der Appell zur Eile fiel nicht auf taube Ohren. So sprach Aćh:

„Nun, wir sind auf der Suche nach meinen verschwundenen Feuertakuri, Turr. Barz hier... komm her, Barz! ... Barz hier hat sich irgendwie nach Tulgor teleportiert, von einem fernen Land namens... Andor? Seit seiner Ankunft hier ist Turr verschwunden, und wir wollten um Rat fragen kommen. Wo hält er sich auf?“

Ganz entgegen seiner ursprünglichen Bitte, ihre Unterredung kurz zu halten, lehnte sich der Hüter der Zeit zurück, schloss seine Augen und bewegte seine Lippen leise.

„Andor... Andor... ja, das sagt mir etwas. Andor, das Königreich hinter dem Kuolema. Ja, ich war sogar schon mal da.“

Aćh stockte: „Aber der Kuolema ist unüberquerbar. Die ganzen Geister und Dämonen. Keiner...“

„Ich sagte auch nichts davon, dass ich über die Berge gereist wäre. Die Dämonen der Berge wurden vor Urzeiten von den Temm ins Gestein zurückgetrieben und in die Täler des ewigen Eises verbannt. Wir wissen es besser, als dass wir es wagen würden, dort einzudringen. Doch es gibt andere Pfade zwischen Tulgor und Andor. Uralte, brüchige Stollen, die tief unter dem Kuolema hindurchführen. Die Wege wandeln sich stetig, und immer mehr schließen sich, doch hin und wieder öffnet sich eine Gelegenheit zur Durchreise. Früher schloss ich mich einst den reisenden Temm an. Wir unterquerten das Kuolema-Gebirge und gelangten in dieses fremde Land namens Andor. Ich half, wo ich konnte, und ich holte einen kleinen Feuertakuri ab, der sich im Land verirrt hatte. Wenn ich mich nicht irre, könnte dies gar derjenige Takuri sein, wegen dessen Verschollenheit ihr mich hier und heute aufsuchen kamt. Auf jeden Fall kehrte ich mit dem Takuri wieder hierhin zurück, und nahm meine Rolle als Hüter der Zeit auf der roten Pyramide wieder auf. Eine kurze Zeit danach erschütterte ein Beben den Kuolema und verschloss den Gang, den ich begangen hatte. Einer meiner Begleiter, Wrort, steckt sich heute immer noch in Andor fest. Doch bleiben solche unterirdischen Gänge nie bis in alle Ewigkeit verschlossen.“

Der Hüter der Zeit zwinkerte Aćh zu, welcher aktuell keine Antwort in den Sinn kommen wollte. Zu viele Informationen auf einmal hatte der Hüter ihr zu vermitteln versucht.

So wandte sich der Hüter der Zeit Barz zu, der wie üblich mit leerem Blick nebenan gestanden und an seiner neuen tulgorischen Kleidung herumgefingert hatte.

„Ronda tatt Hieron“, sprach der Hüter der Zeit. Während Aćh die Stirn runzelte und die fremden Worte zu verstehen versuchte, glänzten Barz‘ Augen auf.\bigskip







„Du sprichst die Sprache der Bewahrer?!“, rief Barz freudig auf. Er war kein Meister der andorischen Sprache, doch waren die Bewahrer die engsten Handelspartner der Barbaren und so wurde ihre Sprache jedem reisenden Nomaden zumindest im Ansatz beigebracht.

Während kaum ein Bewahrer sich darum kümmerte, die Sprache der Barbarenstämme zu erlernen, die für sie nur die „wilden Völker des Ostens“ waren. Barz wusste, dass das Wort der Bewahrer für die Barbarenvölker eine negative Konnotation hatte. Und dass „barbarisch” dort beinahe eine Beleidigung war und für „roh“, „grausam“ und „unkultiviert“ stand. Barz hatte schon seit jeher seinen Kopf schütteln müssen über diese Tatsache. Er war ein stolzer Barbar. Ihre Kultur mochte anders sein als die der Bewahrer. Auf Thakkum wurde nur selten mit Gold gehandelt, öfters in Gütern und Dienstleistungen, was Handel mit den Bewahrern schwierig machten. Und die Barbaren mochten nicht so tief in die Vergangenheit zurückblicken können wie die Bewahrer. Doch sagte dies doch nichts aus über ihre Kultiviertheit, ja, ihren Wert als Personen.

Doch all dies brauchte ihn nun wahrlich nichts zu kümmern. Er konnte sich mit diesem mysteriösen Hüter der Zeit unterhalten. Endlich! Er räusperte sich eilig. Seine Sprache war ganz eingerostet in den letzten Tagen.

Barz hatte schon vor einigen Tagen am Stand der Sonne ablesen können, wo Osten und Westen, Norden und Süden lagen. Nun konnte dieser bucklige Wichtel ihm vielleicht auch verraten, in welcher Richtung Andor lag. Wo Nabib sich aufhielt. Und wie Barz in seine Heimat zurückkehren könnte.

Der Wichtel grinste und sprach schnell: „Ich spreche die Sprache der Bewahrer vermutlich besser als du. Man nennt mich den Hüter der Zeit, weil ich einen besonderen Blick auf die Zeitlinie habe. Ich erinnere mich nicht an das, was bereits geschehen ist, sondern an das, was erst noch kommt. Ich weiß, weil man es mir sagen wir, dass ich mich schon mal in Gesellschaft der Bewahrer befand, einige Jahrzehnte ist es her. Und ich sehe, dass ich mich erneut in der Gesellschaft der Bewahrer befinden werde.“

Barz schluckte seine Freude darüber hinunter, sich endlich wieder mit jemandem austauschen zu können. Aćh hatte ihm trotz der limitierten Kommunikationsmöglichkeiten eintrichtern können, was für eine wichtige Person dieser Hüter war. Und wenn er sich tatsächlich an die Zukunft zu erinnern vermochte, dann war das auch berechtigt. Ohne seine Gedanken lange zu sortieren, schwafelte Barz dutzende Fragen herunter:

„Oh, die Götter seien gepriesen! Wo sind wir hier? Wo liegt die große Steppe meiner Heimat von hier aus? Wo liegt Andor? Wo liegen die Grenzen Eures Wissens? Wie geht es Nabib? Werde ich meine Familie wiedersehen?“

„Immer mit der Ruhe!“, lachte der Hüter der Zeit, „Dann gucken wir uns doch mal diese Fragen eine nach der anderen an. Wo sind wir hier? Wir befinden uns hier in einem Land, das von seinen Einwohnern Tulgor genannt wird. Und es wird auch von den Andori so genannt werden, wenn sie erst einmal von seiner Existenz erfahren. Andor liegt im Osten von hier, gleich hinter dem riesigen Gebirge, deren Gipfel stets in den Wolken liegen. Und deine große Steppe liegt noch weitaus weiter östlich.“

Der Hüter der Zeit schloss seine Augen und runzelte seine Stirn, als versuche er angestrengt, sich an etwas zu erinnern.

„Ich glaube, die Andori nannten dieses Gebirge ‚Fahles Gebirge‘. Sagt dir das etwas?“

Das sagte Barz tatsächlich etwas. In den Briefen aus Andor hatten die invadierenden Barbaren etwas von den unglaublich hohen Bergen im Westen berichtet, dem Fahlen Gebirge eben. Und Barz erinnerte sich auch an die Geschichte, die dieser Feuerzauberer aus dem fernen Hadria ihm damals in Thakkum erzählt hatte:

„Dieser Phoenix war angeblich vor Generationen von einer Einsiedlerin gefunden worden, welche versucht hatte, ein unüberwindbares hohes Gebirge im Westen zu überwinden, dessen Gipfel stets in Wolken gehüllt waren. Manche munkelten gar, das Gebirge sei unendlich hoch, obwohl diese These anhand der endlichen Länge des Schattens des Gebirges natürlich unzweifelhaft widerlegt werden kann.“

Damit war es klar. Das Kuolema-Gebirge war das hohe Gebirge im Westen Andors. Als Barz versucht hatte, sich nach Andor zu teleportieren, hatte er sein Ziel um einiges überschossen. Und das vermutlich, weil dieser Takuri, Turr, gar nicht aus Andor gekommen war, sondern eben vom Nestbaum! Nun ergab alles Sinn.

Doch, ehe Barz sich vollständig darüber freuen konnte, dass er nun wusste, wo seine Heimat lag, fuhr der Hüter schon mit gerunzelter Stirn fort:

„Und wie geht es Nabib? Wer ist denn... Nabib... Nabib... lass mich überlegen... ja, ja, ihr beide werdet euch wiedersehen, nachdem du Andor erreicht hast. Es geht ihm gut. Er war krank? Eine Wunde. Ich glaube, da wird eine entzündete Wunde sein. Oder da war eine. Doch sie heilt gut, er verbrachte nur einige Zeit lang ohne Bewusstsein.“

„Ich werde ihn wiedersehen? Wann? Wie?“

„Oh, mit Zeitangaben war ich noch nie gut“, brummelte der Hüter der Zeit und kratzte sich an seiner breiten Nase.

„Viele Monate werden vergehen... Ein... zwei... ja, mehr als zwei Jahre.“

Der Hüter der Zeit streckte seine Hand mit zweieinhalb von vier Fingern ausgetreckt vor.

„Zwischen zwei und drei Jahren, dann wird etwas geschehen, was dir erlaubt, nach Andor zu reisen. Dort wird dich Nabib wiederfinden.“

Die Erleichterung ob Nabibs Überleben wich einem dumpfen Schrecken.

Zwei bis drei Jahre, bis er Nabib wiedersehen konnte?! Zwei bis drei Jahre, die er noch hier verbringen müsste?! Nicht, dass er es in Tulgor nicht mochte, aber das war einfach zu viel Zeit weit weg von seinen Geliebten und Bekannten. Barz langte sich an den Kopf und sank zu Boden.

Wie hatte er nur so unüberlegt sein können?! Mit einem Experimentierpulver umherzureisen, statt einfach mit den restlichen Barbaren nach Andor auszuziehen! Mit Mächten zu spielen, von denen er weniger verstand. Mit einer Pulvermischung, die er nun nicht mehr reproduzieren konnte, und die ihm vielleicht nicht mal nützte, selbst wenn er es könnte. Denn es gab keine Takuri mehr in Andor, zu denen er sich teleportieren könnte. Er saß hier fest.\bigskip







Aćh sah Barz zu Boden sinken. Besorgt schielte sie zu ihm.

„Was habt Ihr ihm gesagt?“

„Das ist wohl an ihm, dir mitzuteilen, wenn er es denn will“, sprach der Hüter entschuldigend.

Verständnisvoll nickte Aćh und fragte:

„Was ist denn nun mit Turr? Was ist mit ihm geschehen? Können wir ihn finden? Wo können wir ihn finden?“

Sich am Kopf kratzend überlegte der Hüter der Zeit eine Zeit lang mit geschlossenen Augen. Dann hüpfte sein Kopf heftig auf und ab, als er endlich die erlösenden Worte sprach:

„Ja, ich glaube, mich zu erinnern. Du hast mir davon erzählt, wie ihr Turr wiedergefunden habt. Der Takuri verirrte sich wohl auf seinem Rückweg von Barz‘ Heimat nach Tulgor. Er landete dort, wo kein feuriges Wesen je sein sollte. Ihr beide werdet ihn finden in einer Schlucht hoch oben im Kuolema-Gebirge. Im ewigen Eis gleich hinter dem uralten Felsentor. Ihr Takuri-Hüter solltet irgendwo Aufzeichnungen haben darüber, wo sich das befindet.“

Aćh verpasste die Implikation, dass sie und der Hüter der Zeit sich eines Tages wiedersehen und ausgiebig unterhalten würden. Eine Vielzahl von Emotionen durchströmte sie, sowohl Freude über die Information als auch vor allem Sorge um Turr. Der Arme saß nun schon seit mehreren Tagen im ewigen Eis fest. Welche Tortur hatte er erlitten? Wie oft war er bereits erfroren?

Barz am Arm packend sprach Aćh: „Wir müssen zurück! Wir müssen sofort zurück – ins Kuolema-Gebirge!“\bigskip







Die Rückreise von Agarb zum Nestbaum verlief ereignislos. Kein Sturmgeist überfiel das Hängeschiff. Das Wetter war großartig. Das Essen war gut. Die Aussicht war bombastisch.

So rasch wie möglich hatte Aćh Falken ausgesandt, damit die anderen Hüter bei ihrer Ankunft bereits alle bekannten Notizen und Karten zu den Schluchten des ewigen Eises und dem uralten Felsentor zusammengesucht hatten. Diesen Aufzeichnungen nach befand sich das Felsentor in der Nähe eines Waldes, am oberen Ende eines geschlängelten Pfades durch ein Meer aus Steinen den Berg hinauf. Diesem Pfad würden sie folgen.

Das Küken von Saarćhan wurde gemeinsam mit der kleinen Sabri am Nestbaum in Obhut gegeben. Barz trennte sich nur äußerst schweren Herzens von seiner Echse, doch meinte er, dass die Kälte der Berge der Kleinen alles andere als guttun würde. Sie würde ohnehin nur in eine Winterstarre fallen, wenn sie sie mitnähmen.

Aćh war zwar nicht überrascht, doch erfreut darüber, dass Barz sie bereitwillig auf ihrer Reise in den Kuolema begleiten würde. Klar, Turrs Verschwinden war seine Schuld, doch hätte er sich geradesogut von dieser Verantwortung drücken können wie alle anderen Takuri-Hüter am Nestbaum.

„Wenn der Hüter der Zeit meinte, dass ihr beide den Takuri zurückholen werdet, dann werdet ihr beide das tun müssen“, hatte Oma Òkôkó bestimmend gesprochen, „Der Kuolema ist tödlich, und wir können nicht für die Sicherheit weiterer Begleiter garantieren. Doch unsere Wünsche und Hoffnungen werden bei euch sein. Und die beste Bergsteigerausrüstung diesseits des Ozeans.“

Der Weg begann, wie so oft im Gebirge, mit einem weiten offenen Pfad, der mit zunehmender Höhe schmaler wurde. Es folgten steile Passagen und schmale Schluchten. Die beiden kamen gut voran und die Laune war den Umständen entsprechend prächtig. Je höher sie kamen, desto kühler und windiger wurde es. Jetzt waren sie froh um die schneefeste Ausrüstung, die sie am Nestbaum erhalten hatten. Aćh und Barz schlossen ihre Kleidung, so gut es ging, und zogen die Kapuzen ihrer Umhänge über ihre Köpfe.

Im Laufe des Aufstiegs nahm der Wind noch zu und die Temperatur sank stark ab. Sie waren an der Schneegrenze angekommen. Jetzt hieß es noch vorsichtiger sein, denn der Pfad war unter der Schneedecke nur zu erahnen. Während einer Rast setzte Schneefall ein. Mit noch enger zusammengebundenen Kapuzen bahnten sie sich ihren weiteren Weg durch den knietiefen Schnee, bis sie endlich das Felsentor erreichten.

Das legendäre Felsentor, hinter dem der Legende nach die Eis-Dämonen und Geister des Gebirges ihr Unheil trieben.

Bereits aus der Ferne sah das Tor gigantisch aus. Wie ein Torbogen spannte es sich über den Weg.

„Schau her, wir haben es geschafft!“, rief Aćh freudig und zeigte auf das Felsentor. Barz schüttelte nur seinen Kopf und deutete an eine Stelle zwischen ihrem Pfad und dem Tor, wo selbst durch das dichte Schneetreiben hindurch eine dunklere Stelle zu erkennen war.

Ein tiefer, gezackter, breiter Spalt trennte Aćh und Barz von dem Felsentor, hinter dem sich Turr laut dem Hüter der Zeit befand. So ein Spalt war auf keiner Karte eingezeichnet gewesen. Würden sie ihn überqueren können?

„Was tun?“, fragte Barz. Er zückte ein Seil und blickte Aćh fragend an.

„Wo willst du das befestigen?“

„Nicht. Springen. Tod. Gefahr.“

Aćh nickte ergeben.

„Gehen wir mal näher heran. Vielleicht finden wir irgendeinen Ort, an dem wir hinübersteigen können. Schlimmstenfalls nutzen wir dein Bannpulver, um ein bisschen Schnee in der Luft schweben zu lassen und darüberzusteigen...“

„Bannpulver nicht viel. Nicht so machen. Rutschen. Fallen. Tod.“

„Es funktioniert nicht so? Tja, du musst es wissen. Lass uns dennoch mal näher treten.“

Aćh und Barz schritten durch den tiefen Schnee so nahe an die Schlucht vor dem Felsentor, wie sie konnten. Von nahe sah das Tor noch beeindruckender aus. Es war nicht natürlichen Ursprungs. Auch wenn es auf den ersten Blick wie eine zufällige Felsenformation wirken mochte, erkannte Aćh von nahe zwei mächtige Säulen, die das Tor stützten.

Durch das Tor hindurch erblickten sie eine riesige glatte Eisfläche dahinter. Schimmerndes Weiß und kaltes Blau, soweit das Auge reichte. Das ewige Eis. Einst war es vielleicht ein See in einem komplexen System von tiefen Schluchten gewesen, doch nun war es eine legendäre Eismasse, die sich quer durch den Kuolema zog.

Während Jahrhunderte alte, verbleichte Karten noch angaben, dass das ewige Eis am Grunde des großflächigen Tals und tief hinter dem Felsentor lag, reichte die Eisfläche nun bis direkt ans Felsentor. Als hätte sie sich in den vergangenen Jahrhunderten stetig die Talhänge entlang ausgebreitet und wäre nur vom Felsentor aufgehalten worden.

Jahr um Jahr zogen wagemutige Jugendliche aus, um Wege durch die Berge zu finden. Niemand, der das ewige Eis erreicht hatte und weitergezogen war, kehrte je wieder zurück, um davon zu berichten. Und wenn die ersten Sonnenstrahlen des Frühlings den Schnee an den felsigen Hängen schmolzen, gab ihnen das Gebirge die Toten zurück. Oft sprachen die Alten von Geistern und Dämonen, die in den Bergen hausten. Und auch Aćh, die eigentlich nicht viel von solchen Gruselgeschichten hielt, langte beinahe unbewusst an ihren Schwertgriff.

„Da!“, rief Barz, aufs Felsentor zeigend. Seine nackte Hand war bereits bläulich angelaufen. Er hätte Handschuhe anziehen sollen.

Aćh kniff ihre Augen zusammen und hatte ziemlich Mühe damit, durch die Schwaden fallender Schneeflocken hindurch zu erspähen, was Barz‘ Augen erkannt hatten.

Vor einer der beiden imposanten Säulen des Felsentors stand eine über und über mit Schnee und Eis bedeckte Statue einer Frau mit einem Schwert. Wie ein Wachposten wirkte sie. Es hätte Aćh nicht gewundert, wenn sie sich aus der Säule gelöst hätte und mit gezücktem Schwert auf sie und Barz zugekommen wäre.

Da löste sich die Frau aus der Säule, schüttelte ein wenig Schnee von sich und trat zwei Schritte zur Seite, doch stand sie immer noch auf der anderen Seite des Felsentors. Ihr Schwert hielt sie hingegen locker an ihrer Seite. Es war ebenso eisblau wie ihr Körper und ihre Kleidung. Was für ein Material war das?

Die Frau war tatsächlich keine Statue. Aber auch kein Mensch. Sie wirkte, aus wäre sie aus purem Schnee geformt. Aus ihren pupillenlosen Augen glühte ein weißes Licht. Und unter ihrem langen Haarschopf lugten zwei kurze Geweihe hervor.

„Ängstigt euch nicht, ich will euch kein Unheil“, sprach die Frau mit einer klirrenden Stimme. Sie zückte aus dem Schnee hinter sich einen mächtigen Eisblitz und warf ihn in die Schlucht zwischen ihr und den beiden Neuankömmlingen. Unter lautem Getöse wie von berstendem Eis wuchs unter Aćhs und Barz‘ staunenden Blicken ein Steg aus Eis. Ihr Weg zum Felsentor.

„Seid ihr hier, um diesen Feuervogel abzuholen?“, fragte die seltsame Frau

Aćh nickte, immer noch misstrauisch.

„Dann kommt und nehmt das elende Vieh mit!“, rief die Schneefrau, „Es schmilzt uns schon seit Tagen ein Loch ins ewige Eis!“

„Wer sein?“, fragte Barz nun, der von Aćh unbemerkt seinen Bogen gezückt hatte und ihn locker an seiner Seite hielt.

„Hui, jetzt macht mir mal keine Angst!“, rief die Schneefrau überspitzt, „Wir Eis-Dämonen halten nicht so viel aus, wie ihr wagemutigen Menschen immer meint.“

„Wer bist du?“, wiederholte Aćh Barz‘ Frage.

„Mein Name ist Nesdora, und ich bin eine einsame Bewohnerin des ewigen Eises. Eine der wenigen. Und möglicherweise die Einzige, die noch bei klarem Verstand ist. Die lange Zeit im Eis tut den wenigsten langfristig gut. Auch mir wird es nicht auf ewig guttun.“

Aćh ging in Gedanken die Geschichten über die Eis-Dämonen des Kuolema durch, die sie schon seit ihrer Kindheit vernommen hatte. Sie glaubte, sich daran zu erinnern, dass vor einem Jahrzehnt eine Tulgori namens Nes im Gebirge verschollen gegangen war. Und dass im Gegensatz zu den Leichen ihrer Begleiter die von Nes nie gefunden worden war.

Vorsichtig marschierte Aćh über die Eisbrücke auf die Fremde zu. Hinter ihr hörte sie Barz zunächst einen protestierenden Laut produzieren, ehe er ihr folgte.

Vorsichtig trat Aćh näher zum Felsentor. Nun konnte sie auch die kleinen Runenkritzeleien erkennen, die auf den Torsäulen zu sehen waren. Faszinierend.

„Sieh her, da hinten sitzt er, der elende Feuervogel“, zischte Nesdora und zeigte aufs ewige Eis hinaus. Kaum hundert Meter vom Felsentor entfernt war eine kleine Senke im Eis zu erkennen, in der etwas brannte.

Ja, das musste Turr sein! Von hier aus konnte sie nur eine kleine Gestalt in einer Wasserlache erkennen, die vor sich hin flackerte, doch passte dies so gut zur Prophezeiung des Hüters der Zeit. Der Arme! Nun war seine Rettung nur noch Schritte entfernt. Aćh setzte an, zu ihm zu rennen.

„Wartet!“, rief Nesdora, „Wenn ihr einfach so durch das Felsentor zu schreiten versucht, werdet ihr unsägliche Schmerzen erleiden. Es wurde einst von den Temm geschaffen, um unsere beiden Völker zu trennen. Hier, zieht diese Eiskristall-Ketten an. Sie gewähren euch Durchgang.“

Nesdora zog wie aus dem Nichts zwei durchscheinende Halsketten hinter ihrem Rücken hervor. An beiden hingen jeweils drei rautenförmige Eiskristalle, ähnlich zu den kleinen Aschedöschen, die Aćh an ihrer eigenen Halskette trug. Und als Aćhs Blick Nesdoras Hals streifte, erkannte sie, dass der dünne Umgang der Eis-Dämonin von einer identischen Eiskristallkette zusammengehalten wurde.

Nesdora holte aus und warf ihnen beiden je eine Eiskristallkette zu. Während Aćh die ihre mit einem Handschuh fing, fasste Barz die seine geschickt in einer nackten Hand auf.

„Huch! Kalt. Viel kalt.“, beschwerte er sich.

Nesdora verdrehte die Augen, doch wirkte die Geste irgendwie unnatürlich. Oder zumindest ungewohnt.

Mit klirrender Stimme fuhr sie fort: „Zieht die Ketten über euren Nacken, so, dass sie direkten Kontakt mit eurer Haut haben. Sonst hält die Magie im Tor euch für Eindringlinge und hält euch zurück. Oder Schlimmeres. Ich habe es zum Glück noch nie miterlebt.“

Barz tat wie geboten, öffnete seinen Wintermantel so weit, dass sein Hals entblößt wurde, und legte sich seine Eiskristallkette um. Oder besser gesagt, er versuchte es. Er trug bereits eine Kette mit zwei klobigen Ringen um seinen Hals, und diese kollidierten nun mit den Eiskristallen von Nesdoras Kette, welche deswegen nicht schön anliegen wollte. Frustriert riss Barz sich die Ringkette von seinem Hals und ließ selbige achtlos zu Boden fallen.

„Du auch“, mahnte Nesdora Aćh. Diese hatte bislang ihre Kette nur in ihrem Handschuh gehalten und tat nichts dergleichen. Ihr war die Situation nicht geheuer, auch wenn sie nicht in Worte zu fassen vermochte, warum.

Aćh warf einen Blick auf Nesdora, welche sie aus ausdruckslosen Augen anstarrte. Zurück zu Barz, welcher die Eiskristallkette an seinem Hals gerade rückte. Zurück zum Felsentor, dessen Säulen von zahlreichen Runen übersät waren. Zu Turr, welcher kaum hundert Meter davon entfernt am Boden saß und litt.

Dann schloss sie ihre Augen, und schritt energisch durchs Felsentor.

Keine unsäglichen Schmerzen überfielen sie.

Keine Magie hielt sie davon ab, hindurchzumarschieren.

Obwohl ihre Kette keinen Kontakt zu ihrer Haut hatte, hatte das Tor sie nicht aufgehalten.

Nesdora hatte gelogen.

Anschuldigend blickte Aćh die Eis-Dämonin an und schleuderte ihre Eiskristallkette demonstrativ zu Boden.

„Okay, das sieht jetzt böse aus...”, begann Nesdora, beschwichtigend ihre Hände hebend.

„Barz, zieh deine Kette aus!”, bellte Aćh, den Blick weiterhin auf Nesdora gerichtet.

Nesdora versuchte es erneut: „Es gibt eine ganz logische Erklärung. Ich habe euch nicht ganz angelogen. Und ich will euch nichts Böses.”

„Barz, deine Kette!”, rief Aćh erneut. Barz stand weiterhin reglos auf der anderen Seite des Felsentors und blickte sie an. Dann legte er seinen Kopf schief und sprach in gebrochenem Tulgor ganz unschuldig: „Nicht abnehmen wollen.”

Jetzt hatte Aćh aber genug. Sie langte zum goldenen Schwert, welches immer noch in seiner Scheide an ihrer Seite steckte.

Gleichzeitig geschahen zwei Dinge: Zum einen langte Barz hastig nach seinem Pulvergürtel. Zum anderen sprang Nesdora Aćh an und hinderte sie mit einem eiskalten Griff am Ziehen ihres goldenen Schwertes.

Nesdora hauchte aus. Eisiger Dampf schwallte aus ihrem Mund auf Aćhs Kleidung herunter und überzog sie mit einem Reif aus Forst. Aćh schrie verärgert auf, doch war ihre Kleidung natürlich am Nestbaum mit feuerfestem Sufarsaft überzogen worden. Was gegen das Feuer der Takuri isolierte, half auch gegen die Eiskälte des Kuolema.

Von Nesdoras Kälteangriff unbetroffen, ließ sich Aćh nach vorne fallen. Nesdora, welche darauf nicht vorbereitet gewesen war, stürzte in den Schnee und ließ Aćh los. Diese verpasste ihr einen Faustschlag und stolperte von ihr weg.

Erneut machte sich Aćh daran, ihr goldenes Schwert zu ziehen. Mit Blick auf Barz bemerkte sie jedoch, dass ihr kaum die Zeit dafür blieb. Barz hatte seinen grünen Bannpulversack ergriffen und rieb nun eine Prise des gefährlichen Materials in seiner Handfläche warm. Dabei blickte er starr auf Aćh. Das konnte kein gutes Zeichen sein. Gleich würde er das Pulver werfen.

Mit einem mächtigen Schritt hopste Aćh zurück durchs Felsentor, weg von Nesdora und hin zu Barz, und packte dessen Faust mit ihren Handschuhen.

„Lass los, Barz“, redete sie ruhig auf ihn ein, „Ich bin es, Aćh! Ich bin nicht dein Feind. Diese Dämonin hat etwas mit dir gemacht, doch du kannst stärker sein als ihre Zauber!“

„Kenne dich. Aćh, Takuri-Hüterin“, keuchte Barz, „Nicht wünschen Kampf. Eis Hilfe. Siantari Hilfe.“

„Barz! Lass das Pulver fallen! Was soll das überhaupt heißen?“

„Bald. Bald verstehen.“

Einen Augenblick lang rangelten Aćh und Barz noch um das Bannpulver in seiner Faust. Im Geschüttel lösten sich einige Körnchen und gelangten in Aćhs Nase. Aćh kannte die Anzeichen ihrer Niesreizattacken gut genug, um zu wissen, was nun folgte, doch wusste nicht, wie sie hier und jetzt darauf reagieren sollte.

Nieser um Nieser schüttelte ihren Körper durch, und da sie Barz‘ Hand festzuhalten versuchte, auch diese. Aćh fühlte, wie ihr Barz‘ Pulverfaust entglitt, ja, wie sie selbige gar durch den Schwung eines Niesers schwungvoll irgendwohin stieß. Sie stolperte ängstlich zurück, während sie das inzwischen vertraute magische Knattern des sich entfaltenden Bannpulvers hörte.

Nachdem das letzte „Hatschi“ Aćhs Kopf durchgeschüttelt hatte, öffnete sie ihre Augen, in der Hoffnung, das Bannpulver möge sie selbst verfehlt haben.

Es hatte.

Vor ihre drehte und wand sich Barz und versuchte, vom Fleck zu gelangen. Doch seine Faust, welche vorhin eine Prise Bannpulver gehalten hatte, steckte wie angegossen in der Luft fest, die Finger weit gespreizt, von einem grünlichen Glühen umgeben. Egal, wie sehr sich Barz bemühte und den Rest seines Körpers bewegte, seine gebannte Faust regte sich kein bisschen und hielt ihn zurück.

Freilich konnte sich Aćh nicht lange darüber freuen. Barz war ihr noch nahe genug, dass er sie am Umgang packen und nach vorne ziehen konnte.

Ihr Kämpferinstinkt erwachte. Aćh ließ sich von Barz‘ Schwung mitziehen und stieß sich gerade rechtzeitig vom Boden ab, sodass ein gewöhnlicher Gegner von ihr mitgerissen und zu Boden geworfen worden wäre. Leider war Barz‘ gebannte Faust kein gewöhnlicher Gegner.

Sie stieß sich im richtigen Moment vom Boden ab und rollte über Barz hinweg, doch statt dass dieser zu Boden ging, ertönte ein grausiges Knirschgeräusch und Aćh wurde allein in den Schnee geschleudert.

Als sie hinter sich blickte, drehte sich ihr Magen um. Barz‘ Faust, weiterhin vom grünlichen Glühen des Bannpulvers umgeben, steckte weiterhin in der genau selben Position in der Luft fest. Der Rest von Barz‘ Körper befand sich nun allerdings in einem sehr unnatürlichen Winkel dazu. Barz‘ Unterarm sah aus, als hätte er neben seinem Ellbogen plötzlich ein zweites Gelenk entwickelt.

Es schmerzte Aćh bereits, Barz’ gebrochenen Arm so zu sehen. Dieser biss hingegen nur seine klappernden Zähne zusammen und ließ keinen Schmerzenslaut vernehmen.

Jetzt hatte Aćh endlich die Zeit, ihr goldenes Schwert zücken und auf Barz richten. Nicht, dass dies nun noch viel nützen würde. Sie blickte zurück zum Felsentor und erkannte mit Freude, dass Nesdora auf der anderen Seite stand und ihr bloß einen eisigen Blick zuwarf. Offenbar konnte die Eis-Dämonin das Felsentor nicht durchschreiten. Turr mochte noch nicht gerettet sein, doch die akute Gefahr war gebannt.

Da ertönte ein Ächzen hinter ihr. Aćh drehte sich zurück und erblickte, wie Barz mit seiner gesunden Hand einen violetten Stein in seinen Mund schob und darauf biss. Weiß der Himmel, woher er diesen Stein gezogen hatte. Es knackte, und als Barz seinen Mund wieder öffnete und den Stein ausspuckte, war dieser in mehrere spitze Splitter zerfallen. Von ihnen allen ging ein immer heller werdendes Licht aus, welches Aćh in den Augen schmerzte. Und wo diese Lichtstrahlen auf das hellgrüne Leuchten des Bannpulvers trafen, verschwand das grünliche Glimmen in der Luft. Das Bannpulver löste sich!

Barz‘ gebrochener Arm wurde nicht länger in der Luft gehalten, sondern fiel an seine Seite. Er war wieder frei. Und schon wieder griff er in ein Pulversäcklein. Was sollte sie tun? Sie konnte Barz doch nicht einfach abstechen!

Da dachte Aćh zurück an den Nixenstaub, das Barz ihr geschenkt hatte. Gegen einen starken Feind solle sie es einsetzen, um ihn zu schwächen, hatte Barz gesagt. Die Gelegenheit schien unglaublich passend. Sein eigenes Geschenk würde den von irgendeiner dämonischen Magie belegten Nomaden schwächen, auf dass Aćh ihm hoffentlich helfen konnte. Wie poetisch.

Sie griff an ihren Wintermantel und zückte das weiße Nixenstaub-Säcklein. Die Schnur war schnell geöffnet, und ebenso eine Prise des Pulvers ergriffen.

Barz blieb einfach nur stehen und blickte sie regungslos an.

Aćh pustete ihm den Nixenstaub ins Gesicht.

Barz grinste.

Urplötzlich knackte es in Aćhs Arm. Unsägliche Schmerzen strahlten ihn entlang. Das goldene Schwert entglitt ihr und stürzte – wie Aćh – selbst in den Schnee. Verkrampft hielt sie ihren Unterarm an ihren Körper, während es darin knirschte und schmerzte, als pulverisierten sich all ihre Armknochen. Ihr Gesichtsfeld verengte sich. Als sie zurück zu Barz blickte, erkannte sie, dass dessen gebrochener Arm auf einmal wieder ganz gerade und gesund aussah.

Bei den sieben Feuern des Himmels, was war soeben geschehen?!

Barz fackelte nicht lange, sondern stürzte sich auf Aćh und drückte sie zu Boden. Er ergriff Aćhs gesunden Arm, zerrte ihren Handschuh zur Seite und führte ihre Finger mit seinen eiskalten Händen zu seinem Hals. Dann presste er Aćhs Hand auf seine Eiskristallkette.

Sie war eiskalt.

Und noch so viel mehr.\bigskip







Alle Wut und alle Sorgen, ja, sogar aller Schmerz aus ihrem gebrochenen Arm flossen aus Aćhdora wie aus einem löchrigen Sieb. Barzdur blickte ihr in die Augen. Sie verstanden einander.

Aćhdora fühlte, wie starke Magie aus den Eiskristallen von Barzdurs Kette seinen und ihren Körper einhüllten. Die magischen Kristalle waren der Ursprung von Barzdurs aktuellen Gedanken und seines Willens. Und nun waren sie auch der Ursprung von Aćhdoras Gedanken und Willen.

Plötzlich wurde Aćhdora bewusst, dass sie beobachtet wurde. Aus weiter Ferne spürte sie einen forschenden Geist auf sie blicken. Wie etwas Fremdes ihre Erinnerungen durchforschte und selbst welche mit ihr teilte.

Vor ihrem inneren Auge sah sie das Antlitz einer Frau, weiß und durchscheinend, langhaarig und gehörnt. Die Eis-Dämonin sprach, ohne ihren Mund zu öffnen, und ihre klirrende Stimme bohrte sich tief in Aćhdoras Geist.

„Ich bin Siantari. In den tiefen Schluchten des Kuolema scheint niemals die Sonne auf das ewige Eis, und dort ist mein Reich.“

Ihr wurde bewusst, als hätte sie es schon lange gewusst und sich erst jetzt wieder daran erinnert, dass Siantari nun schon seit beinahe 500 Jahren die Dämonin des ewigen Eises war, die über die schattigen Täler hinter dem Felsentor herrschte und das ewige Eis nährte. Und ihr wurde bewusst, dass das schlecht war. Sie wurde wie das ewige Eis im Kuolema eingesperrt, während es sich doch eigentlich über die gesamte Welt ausbreiten sollte. Glücklicherweise gab es nun sie und Barzdur.

Ja, Aćhdora und Barzdur befanden sich auf der anderen Seite des Felsentors. Sie waren frei. Wenn sie einige Stunden warteten, würden die Eiskristallketten auch ihre Körper erfrieren lassen und ihnen ein Leben als Eis-Dämonen schenken. 500 Jahre lang könnten sie so bestehen, bis auch ihre Schneekörper nicht mehr halten würden und sie einen Nachfolger für ihre Ketten zu finden hätten.

Aber das war in weiter Zukunft. Siantari hatte ein dringlicheres Problem. Ihr eigener Schneekörper war schon beinahe 500 Jahre alt. Lange würde er nicht mehr halten. Sie musste einen Nachfolger finden. Und Siantari wusste, dass die abenteuerlustigen Tulgori, die immer wieder dieses Gebirge zu erklimmen versuchen, nicht als Nachfolger taugten. Schwächliche Jungen, die meinten, das Eis der Berge würde sie zu Männern formen…

Glücklicherweise war es Siantari als erster Eis-Dämonin des Kuolema gelungen, Eiskristallketten zu formen und ihre Kräfte so auf weitere Menschen zu übertragen, ohne selbst zu vergehen. Leider hielt der schwächere Geist dieser neuen Eis-Dämonen die Einsamkeit des Gebirges kaum aus, und mit der Zeit wurden sie alle von einem Wahn befallen.

Doch Aćhdora und Barzdur befanden sich im Gegensatz zu den anderen Durs und Doras außerhalb des Felsentors. Sie konnten nach Tulgor zurückkehren und einen würdigen Nachfolger für Siantari finden. Vielleicht gar einen Temm, der wusste, wie man das Felsentor öffnete. Dann wäre es endlich soweit. Das Felsentor könnte sich öffnen. Siantari könnte wieder frei sein. Das ewige Eis könnte sich weiter ausbreiten. Die gesamten umliegenden Lande könnten in eine Eiswüste verwandelt werden.

Ja, Aćhdora und Barzdur waren bereit, in Siantaris Dienst zu treten. Vielleicht, wer weiß, würde ihnen außerhalb des Felsentors auch das Schicksal aller anderen Durs und Doras des Gebirges erspart blieben. Und selbst wenn sie eines Tages der Wahn befallen sollte, würde es das wert gewesen sein.

Alles war gut.\bigskip







In diesem Augenblick wurde Aćh an ihrem langen roten Umhang nach hinten gerissen. Ihr Gesichtsfeld verfärbte sich schneeweiß, als sie über den verschneiten Boden rutschte und ihre Kapuze sich mit kaltem Nass füllte. Ihr Geist wurde wieder klar. Leider kehrten auch die pochenden Schmerzen in ihrem Arm zurück. Sie hatte Mühe, die Eindrücke der letzten Augenblicke einzusortieren. Und die Zeit drängte.

Nesdora rupfte weiterhin an Aćhs Umhang und näherte sich ihr langsam. Die Eis-Dämonin hatte die zweite Eiskristallkette in den Händen, die für Aćh bestimmt war. Offenbar hatte sie sie eingesammelt und hatte nun vor, sie Aćh aufzuzwingen. Doch hatte sie wohl nicht erkannt, dass Barz Aćh bereits erwischt hatte, und ihr nun ungewollt einen Moment der Klarheit geschenkt. Einen Moment, um sich zu wehren. Ihre Gedanken rasten.

Aćh lag gerade mitten unter dem Felsentor, ein bösartiger Barz zu ihrer Linken und eine nicht nette Nesdora zu ihrer Rechten. Doch war sie nicht allein. Sie hatte Turr! Noch mochte der Takuri keine Ahnung haben, dass sie hier war, doch würde er ihr sicherlich zu Hilfe kommen, wenn sie ihn rief. Warum hatte sie nicht früher an ihn gedacht?!

„Herjo! Turr, Herjo!“, brüllte Aćh aus voller Kehle. Irgendwie schaffte sie es, ihr Schwert zu packen und von Barz zurückzustolpern. Sie wankte durchs Felsentor auf Nesdoras Seite. Zum ewigen Eis hin. Und zu Turr. Mit Nesdora konnte sie es vielleicht selbst in diesem geschwächten Zustand aufnehmen, wenn sie nur die andere Eiskristallkette zerstören konnte. Um Barz würde sie sich später kümmern müssen.

Ein lautes Knattern ertönte hinter ihr. Als sie zurückblickte, sah sie, wie dutzender weiße Lichtkugeln vor Barz herumwirbelten und ihn zurückdrängten, während er unter großer Anstrengung vergeblich versuchte, das Felsentor zu durchschreiten. Wie vermutet ließ das Tor keine Eis-Dämonen durch, auch keine Angehenden.

Das hieß, dass sie nur noch mit Nesdora zu kämpfen hatte. Und sie war nicht allein.

Ein melodisches Kreischen kündigte Turrs Ankunft an. Der Takuri war schwach, und er schlidderte mehr übers Eis, als dass er darüber flog, doch bewegte er sich stetig auf Aćh und Nesdora zu.

Aćh und Nesdora hatten beide ihre jeweiligen Schwerter erhoben und hielten sie vorsichtig vor sich. Aćh versuchte, sich an die Kunst des Schwerttanzes zurückzuerinnern, doch sie war erschöpft und ihr dominanter Arm war gebrochen. So musste sie sich mit einigen uneleganten Schwüngen ihres Schwerts begnügen. Hauptsache, sie hielten Nesdora noch einige Momente von sich selbst weg. Einige Momente, bis...

Jetzt!

„Turr?! Advaria! Advaria meza!”

Aćh wartete nicht ab, zu sehen, ob Turr sie verstanden hatte. Sie drehte sich ab. Ein lautes Klatschen von goldenen Flügeln war zu hören. Eine Hitzewelle rauschte über Aćh hinweg. Als sie sich wieder zurückdrehte, drehte Turr stolz Kreise über einer am Boden zusammengekugelten, reglosen Eis-Dämonin. Turr sah größer und kräftiger aus als vor dem Kampf. Stichflammen flackerten über sein wunderschönes Gefieder.

Nesdora war vorerst ausgeknockt. Zeit, sich um Barz zu kümmern.

„Turr?! Kurjo!“, rief Aćh, und zeigte mit ihrem gesunden Arm auf Barz. Dieser hatte aufgehört, sich durchs Felsentor zu kämpfen versuchen, und starrte ausdruckslos auf Aćh. Sein Problem hatte mit diesen Eiskristallketten zu tun. Da war doch vernünftig, anzunehmen, dass man mit genügend Hitze Abhilfe schaffen konnte.

Turr ließ nicht lange auf sich warten, sondern stieß einen melodischen Ruf aus, breitete seine Schwingen aus und hielt rasend schnell auf Barz zu. Aćh humpelte ein wenig langsamer hinterher.

Doch auch Barz war nicht untätig. In Windeseile hatte er seinen Bogen erhoben, einen Pfeil aus seinem Köcher gezückt und angelegt. Dann zielte er. Aćh ließ sich zu Boden fallen und duckte sich hinter eine Schneewehe. Dann blickte sie panisch wieder auf, als ihr bewusst wurde, dass vermutlich nicht sie das Ziel war.

„Turr! Herjo! Nepom...“

Ein lautes Kreischen unterbrach ihren Ruf. Barz hatte seine Bogensehne losgelassen und den Pfeil abgeschossen. Turr, der nur noch weniger Meter von ihm entfernt gewesen war und sich wie ein Seeadler mit ausgestreckten Krallen auf ihn gestürzt hatte, wurde von Barz‘ Pfeil mitten in der Brust getroffen. Der Feuervogel schrie. Sein Schrei wurde von einem Knistern und Brodeln überdeckt. Rund um den Pfeil in seiner Brust brachen Flammen hervor und verzehrten seinen Körper. Doch während Turr starb und zu Asche verglühte, stürzte sein sich wild umherdrehender Körper weiterhin auf Barz zu.

Der Steppennomade blinzelte nicht einmal, während der sterbende Takuri ihm mit Höchstgeschwindigkeit ins Gesicht klatschte und es in einen Feuerball tauchte.

Aćh zwang ihren protestierenden Körper, sich aufzurichten und zu Barz zu stolpern. Ihren gebrochenen Arm an die Seite gepresst, humpelte sie so schnell sie konnte durchs Felsentor auf ihn zu.

Barz hatte seinen Bogen losgelassen und war zu Boden gestürzt. Sein Bart war angekokelt und seine Miene eine Maske der Furcht. Neben ihm lag ein halb geschmolzener Pfeil und ein unversehrtes, zitterndes Takuri-Küken in einem Häufchen Takuri-Asche, welches langsam ein Loch in den Schnee schmolz.

Aćh kniete auf Barz Brust nieder und hinderte den zappelnden Nomaden so am Aufstehen. Barz‘ Wintermantel war oben weiterhin so weit geöffnet, dass sie gut auf seinen Hals sehen konnte. Es sah alles andere als gut aus. Die Eiskristallkette klebte weiterhin wie angeleimt auf seiner Haut, und schlimmer noch, rund um die Kette hatte sich sein Hals bläulich-weiß verfärbt, als wäre er aus Eis geformt.

„Weg!“, hauchte Barz schwach, „Feuer... nicht Feuer... Takuri... weg!“

Sein Knie traf Aćhs Rücken, doch so leicht ließ sie sich nicht von ihm schubsen. Aćh langte neben Barz, unter das Turr-Küken, und ergriff mit ihrem Handschuh eine Handvoll glühender Takuri-Asche. Dann presste sie den feurigen Mischmasch auf Barz‘ Hals. Barz schrie auf und strampelte noch heftiger, doch sie ließ sie nicht ab.

Endlich wurde Barz still. Etwas knirschte. Als Aćh ihre Hand von seinem Hals ließ, war von der Eiskristallkette nichts mehr übrig außer dahinschmelzende Eissplitter. Barz‘ Hals sah übel aus, ein Mischmasch aus eisblauen Flecken und rötlichen Verbrennungen, doch seine Augen blickten wacher drein als zuvor.

„Geht es wieder?“, fragte Aćh vorsichtig. Barz murmelte etwas Unverständliches. Dass er sich nicht mehr wehrte, wertete Aćh als gutes Zeichen. Ihre eigene Ergebenheit gegenüber Siantari hatte ja auch aufgehört, sobald sie keinen direkten Kontakt mehr zur Eiskristallkette gehabt hatte. Alles war wieder gut.

Ächzend rollte Aćh von Barz runter, heulte auf vor lauter Schmerzen in ihrem gebrochenen Arm und tastete nach dem kleinen Turr. Mit großen Kugelaugen blickte das Küken sie an und wimmerte leise. Aćh steckte ihn in eine Tasche ihres Mantels, streichelte sein Gefieder und redete ihm gut zu.

„Gefahr!“, rief Barz.

Aćh blickte seinem ausgestreckten Finger entlang und erblickte Nesdora, die Eis-Dämonin, welche sich wieder aufgerichtet hatte und auf sie beide zuhumpelte. In einer Hand hielt sie ihr bläulich schimmerndes Eisschwert, die andere war leer.

Noch befanden sie sich auf der anderen Seite des Felsentors. Nesdora konnte sie technisch gesehen nicht erreichen. Sollten sie versuchen zu fliehen? Da manifestierte Nesdora wie aus dem Nichts einen gewaltigen Eisblitz in ihrer leeren Hand und hob diese, als wolle sie den Blitz auf Aćh und Barz schleudern. Davor konnten sie nicht wegrennen.

Doch das war auch nicht nötig. Barz, Wut in seinen Augen, griff in eine seiner Manteltaschen und zückte einen dunkelbraunen Beutel. Noch während Nesdora zum Wurf ausholte, öffnete er es und pustete Nesdora eine Wolke von etwas Goldenem entgegen, was für Aćh verdächtig nach zusammengescharrtem Sand der Temm aussah. Wie viel Sand hatte Barz am Nestbaum gesammelt?!

Auf jeden Fall schien es zu wirken, denn die Eis-Dämonin nieste, stolperte und rutschte auf der glatten Fläche des ewigen Eises nach vorne. Ihr Eisblitz fiel ihr aus der Hand und verdampfte, als hätte er nie existiert. Ihr Eisschwert schlidderte durchs Felsentor zu Aćh. Diese fackelte nicht lange, sondern packte das Schwert mit ihrer nichtdominanten Hand, schritt durchs Felsentor nach vorne und ließ das Schwert auf die Eis-Dämonin niedergehen.

Es schnitt glatt durch die Dämonin, wie ein warmes Messer durch Butter. Nur an der Eiskristallkette um ihren Hals blieb es kurz stecken, doch nach einem Ruck zersplitterte auch die Kette. Während der Kopf der Dämonin zur Seite rollte und mit dem Geweih an einer Säule des Felsentors stecken blieb, löste sich der Körper der Eis-Dämonin langsam in viele einzelne Eiskristalle auf. Fast schien es, als versinke sie im ewigen Eis.

Aćh verharrte noch einen Augenblick, um sicherzugehen, dass Nesdora sich nicht irgendwo um sich erneut manifestierte. Wer konnte schon so genau wissen, über welche Fähigkeiten diese Dämonen verfügte. Dann traf sie die Realisation.

„Ich... ich habe sie getötet“, flüsterte sie, „Sie... sie war doch nur...“

Barz hob Aćhs Schwert auf und verpasste Nesdoras Kopf einen verächtlichen Tritt, sodass auch dieser aufs ewige Eis rollte und sich darin auflöste. Er betastete mit einer Hand sorgfältig seinen wunden Hals, während er ihr mit der anderen Aćhs goldenes Schwert überreichte. Noch immer war dessen Oberfläche unzerkratzt.

Aćh fuhr mit einem zitternden Daumen über die Mondsichel über dem Schwertgriff und schob das Schwert dann wieder in seine Scheide.

Sie schaute ein weiteres Mal sorgsam nach dem kleinen Turr-Küken und stolperte dann aus dem Felsentor hinaus, dicht gefolgt von Barz. Nur weg von hier.

Als sie einen vorsichtigen Blick zurück auf die ewige Eisfläche warf, erstarrte sie. Eine großgewachsene Gestalt kam rasch auf sie zu. Sie lief nicht über das ewige Eis, sondern schwebte in einem Wirbel aus Nebel und Schnee, die Arme auf beiden Seiten ausgestreckt. Noch war nur ihre Silhouette am Horizont zu erkennen. Doch kam die Eis-Dämonin rasend schnell näher. Bald würde man ihren Kopf genau erkennen können, und Aćh hätte sich stark gewundert, wenn dort kein Geweih zu sehen wäre. Und ein Blick, so kalt, dass man meinte, sofort zu Eis zu erstarren.

Siantari. Die oberste Dämonin des ewigen Eises.

Eine Eiseskälte befiel ihre Glieder. In ihr breitete sich ein Gefühl zunehmender Erstarrung aus.

Ein Ruck fuhr durch Aćhs Körper, als Barz sie packte, umdrehte und weiterzog.

„Turr hier. Gehen, gehen, gehen, jetzt!“, intonierte er panisch.

Mit zusammengebissenen Zähnen drehte sich Aćh um und stolperte ebenfalls den Weg die Berge hinab.

Als sie einen letzten Blick zurückwarf, sah sie mit Erleichterung, dass auch Siantari vom Felsentor zurückgehalten wurde. Noch immer hatte diese ihre Arme ausgebreitet und gebot dem Sturm um sie herum, zu wüten. Eine Wolke aus Schnee und Kälte jagte den Berg hinab auf Aćh und Barz zu und hatte sie rasch eingehüllt. Hätte Barz nicht schon so Aćh gestützt, hätten sie einander verloren.

Dann brachen sie aus dem Sturm heraus ins Freie. Noch immer war es bitterkalt, doch jagte ihnen kein peitschender Wind mehr ins Gesicht. Siantaris Macht schien ihre Grenzen zu haben. Sie hatten es geschafft.\bigskip







Etwas langsamer setzen Aćh und Barz ihren Weg zum Nestbaum zurück fort. Erleichtert, dass Aćhdora und Barzdur nun doch keine Realität würden.

„Verzeihung“, war alles, was Barz hervorbrachte, immer und immer wieder, „Verzeihung.“

Er tastete immer wieder nach seinem Hals und nach seinen Pulversäckchen, schien mit dem Ergebnis allerdings nicht zufrieden. Insbesondere hatte er keinen weiteren Nixenstaub mehr übrig.

„Nicht Nixenstaub geben sollen. Doch Gefahr. Viel Gefahr. Pulver Gefahr. Dumm. Dumm! Verzeihung.“

Aćh sah schon kommen, dass Barz sie nie wieder in die Nähe seiner Pulver lassen kommen würde. Nun, eigentlich störte sie das auch nicht zwingend. Ohnehin war nun vor allem wichtig, dass sie beide zum Nestbaum zurückkehrten, ehe sie erfroren. Und ehe ihnen die Nahrungsvorräte ausgingen.

Einmal blieb Barz abrupt stehen, als er nach seinem verletzten Hals langte und wohl erkannte, dass seine Ringkette immer noch fehlte. Die war nun wohl in Siantaris Gewahrsam. Er verzog sein Gesicht, schritt dann aber hastig weiter.

So humpelten Aćh und Barz den Berg hinunter. Die untergehende Sonne warf nur noch ein schwaches Licht auf das Aćh so wohlbekannte Land Tulgor. Sie betrachtete Turr in der Dämmerung. Noch immer kuschelte er sich schwach an ihre Seite, und nur ein leichtes Glimmen schimmerte über sein Gefieder. Nichtsdestotrotz wusste Aćh, dass das Leuchten seiner Federn ihr Licht sein würde in den dunklen Stunden, die vor ihnen lagen.

Bis zurück an den Nestbaum der Takuri schafften sie es vor dem Einbruch der Nacht nicht, und so suchten sie Schutz im erstbesten verlassenen Mineneingang. Barz fand eine kleine Holzkonstruktion, welche früher wohl mal einen Spiegel in sich verankert gehabt hatte, um das Licht des roten Mondes tiefer ins Gebirge zu lenken.

Mithilfe von Turrs Federn gelang es Aćh, das Konstrukt in Brand zu setzen. Dann ließ sie sich ächzend zu Boden sinken und umklammerte ihren gebrochenen Arm.

Barz blickte sie schuldbewusst an.

„Ich Wächter. Du Ruhe. Zukunft Baum Heiler“, sprach er ruhig, auch wenn Aćh nicht ganz klar war, ob er eher sie oder sich selbst beruhigen wollte. Es war egal. Schon bald hatte sich ihr unterkühlter Körper am Lagerfeuer etwas aufwärmen können, und trotz der stechenden Schmerzen in ihrem Arm sank sie bald darauf in einen unruhigen Schlaf.\bigskip







„Nicht Gute Nacht! Nicht Gute Nacht! Gefahr!“, erklang Barz Stimme viel zu laut. Aćh schlug ihre Augen auf. Sie fröstelte. Ihr Arm pochte höllisch. Das Lagerfeuer war erloschen und ihre Decke war zur Seite gerutscht. Barz stand ein wenig von ihr entfernt und blickte angespannt ins Dunkel der Mine hinein. Laute Geräusche waren von dort zu hören, ein Schlurfen und Stampfen, ein Brabbeln und Lärmen. Lebewesen. Waren das Arpachen, Sporne, etwa gar Lumiwürmer? Das wäre alles NICHT gut!

Aćh stemmte sich in die Höhe und zückte ihre Steinflöte. Eine rasche Melodie und das Kommando „Kurjo!“ genügte, und schon schwirrte der kleine Turr den Gang entlang, tiefer in die Mine hinein.

Im schwachen goldenen Licht, das von seinem Gefieder ausging, erkannte Aćh eine Vielzahl bleicher Gestalten in rudimentärer Kleidung. Unter ihren langen Kapuzen lugten breitnasige Gesichter hervor. Höhlenwichte! Mindestens vier! Sie grabschten mit klammen Klauen nach dem vorbeifliegenden Turr, bekamen ihn aber nicht zu fassen.

„Menschenfresser! Mensch essen!“, rief Aćh.

Barz verstand, machte große Augen, zückte seinen Bogen und legte einen Pfeil an.

„Weg! Husch!“ rief er den Höhlenwichten zu. Turr war inzwischen wieder bei Aćh angelangt, weswegen die Wichte aktuell nur an ihren glänzenden Augen im Schatten zu erkennen waren.

Ein Schrei erklang hinter Aćh. Sie wirbelte herum und erblickte, wie Barz von zwei weiteren Höhlenwichten angesprungen und umgeworfen wurde. Vor dem Mineneingang befanden sich weitere Wichte. Ihr Ausgang war versperrt!

Während Barz mit den beiden Wichten auf ihm rang, schlidderte sein Bogen hinüber zu Aćh. Doch selbst mit zwei gesunden Armen hätte sie diesen nicht beherrscht. Fluchend zückte sie ihr goldenes Schwert und fuchtelte damit in Richtung der vier Höhlenwichte tiefer in der Mine. Diese sollten bloß nicht näherkommen.

Ein Höhlenwicht saß frech auf Barz‘ Brust, während ein anderer ihm nach den Augen klaubte. Barz wehrte sich vehement, doch ohne großen Erfolg. Die Wichte waren überraschend kräftig für ihr Aussehen.

Aćh humpelte näher und zog ihr goldenes Schwert dem einen Höhlenwicht über den Schädel. Dieser klappte zusammen, rollte von Barz runter und ein schwaches magisches Glitzern breitete sich über seinen Körper aus.

Ehe Aćh sich dem zweiten Höhlenwicht auf Barz zuwenden konnte, kündigte ein Schlurfen und wirres Brabbeln hinter ihr von der Ankunft der restlichen Höhlenwichte. Sie wandte sich ihnen zu und schwang ihr Schwert, trotz ihres protestierenden Arms.

Sie blickte Turr an. War er schon stark genug, wieder einen Angriff auszuführen?

Ehe sie Turr rufen konnte, wurde auch sie von kalten, feuchten Händen gepackt. Ein Höhlenwicht rang Aćh zu Boden. Sie schrie, sowohl wegen der Schmerzen als auch, um die Wichte zurückzuscheuchen. Mit aller Kraft versuchte sie, sich dem Griff zu entziehen. Erfolglos.

Ein einzelner Höhlenwicht schob sein grimmig grinsendes grün-graues Gesicht vor Aćhs und brabbelte etwas vor sich hin, das bei genauem Hinhören Wortfetzen sein könnten. Dann biss der Wicht ihr in die Nase.

Neben sich hörte sie Barz erneut verzweifelt aufschreien.

Sie waren geliefert.

Da erklang eine tonlose aber nicht unfreundliche Stimme in einem fremden Akzent: „Das Besetzen unserer Minen ist in diesen Landen nicht gern gesehen. Und nächtliche Überfälle erst recht nicht. Ich muss euch bitten, euch unverzüglich zu ergeben, Unruhestifter!“

Ein grünlicher Blitz zuckte durch den dunklen Stollen. Aćhs Gesichtsfeld wurde von blendend hellem Grün geblendet. Ein Hitzeschwall rauschte über sie hinweg. Der Höhlenwicht kreischte auf und ließ von ihrer Nase ab. Als ihre Sicht langsam wieder zurückkehrte, erkannte Aćh die Silhouette eines Menschen vor dem Höhleneingang. In seiner Hand glühte etwas Grünliches, von dem immer wieder Strahlen in Richtung der ins Dunkel der Mine zurückweichenden Höhlenwichte ausschlugen. Nur der von Aćh ausgeknockte Höhlenwicht blieb liegen, sowie der über Barz zappelnde, der sich immer noch labernd an sein Opfer klammerte.

Grünes Feuer.

Magisches Feuer.

Ein Hexer.

Der Hexer griff an seinen Gürtel, zückte ein Säcklein, öffnete es und streute ein hellblaues Pulver daraus über den strampelnden Höhlenwicht.

Der Höhlenwicht wankte und stöhnte, rollte von Barz runter und wandelte seine Gestalt, während magisches grünes Feuer rund um ihn herum aus dem Boden brach und sich um seine Gliedmaßen wand. Als das Feuer verschwand, lag da nicht mehr ein Höhlenwicht, sondern eine hässliche kleine Schuppenkreatur mit zwei Köpfen und Flossen anstelle von Gliedmaßen. Sie erinnerte an einen Sumpffisch. Und offenbar brauchte sie wie ein Fisch Wasser zum Atmen. Der verwandelte Höhlenwicht japste einige Male vergeblich nach Luft. Dann wurde sein Körper totenstill.

Barz hörte auch auf zu strampeln und hielt sich stöhnend seinen Magen. Zugleich blickte er jedoch staunend zum Hexer und dessen blauen Pulversäckchen hoch.

„In der Steppe Tulgors wächst ein besonderes Kraut, welches im Licht der aufgehenden Sonne in hellem Blau erstrahlt. Dessen getrocknete Blüten können zu einem Pulver zermahlen werden, welches neuen Mut verleiht und Feinde verwandeln kann“, erklärte der Hexer mit einer gelassenen Selbstverständlichkeit. Als wäre dies in der aktuellen Lage die wichtigste Information.

„Doch müssen wir Acht geben. Der Einsatz des Pulvers kann auch dafür sorgen, dass mehr Kreaturen auftauchen. Hexerei ist nun mal nicht komplett kontrollierbar, und jeder Einsatz kann ungewollte Konsequenzen zur Folge haben.“

Dann blickte der Hexer zum kleinen Takuri und meinte: „Nun, der wäre natürlich noch praktischer. Aus der Asche von Feuertakuri lässt sich ein Pulver erstellen, welches Kreaturen völlig verschwinden lassen kann. Zumindest eine Zeit lang.“

Barz blickte noch interessierter zu ihm hoch, auch wenn er wohl nur die Wörter „Takuri“ und natürlich „Pulver“ verstanden hatte.

Der Hexer blickte hinter sich und rief laut: „He, Leute, kommt euch ansehen, was ich gefunden habe.“

Rufe von anderen Menschen außerhalb der Mine wurden laut.

Dann streckte der Hexer Aćh und Barz seine Hände entgegen und half ihnen auf.

„Los, nichts wie ins Freie mit uns. Dieser Gegend wimmelt ohnehin nur so von Wichten. Wir haben schon vor Wochen den Stollen hier aufgegeben. Wir graben jetzt weiter westlich.“

„Danke. Vielen Dank, o Fremder“, war alles, was die beiden als Antwort herausbrachten.

„Ich bin doch kein Fremder“, grinste der Fremde schwach, „Mein Name ist Haamun und ich bin meines Zeichens Minenarbeiter in den hiesigen Mera-Minen.“

„Aćh, Takuri-Hüterin vom Nestbaum.“

„Letzteres hätte ich mir doch fast denken können, wenn ich mir den Takuri so ansehe.“

„Das ergibt natürlich Sinn! Nun, was tut Ihr hier, ‚Haamun‘? ‚Licht des Morgens‘, hä? Da können wir doch von Glück reden, dass Ihr nicht erst zum Sonnenaufgang erschienen seid, sonst wären wir beide wohl Geschichte.“

Haamun gluckste leise: „Das war kein Glück und auch kein Zufall. Zufälle sind sehr unwahrscheinlich. Nein, wir haben euer Lagerfeuer von weiter unten aus gesehen und gedacht, dass das Wichte anlocken könnte. Einfach so in einem Minengang zu nächtigen, ohne auf Gefahren zu achten, ist schon fahrlässig.“

Sein tadelnder Ton verschwand, als Haamun Aćhs gebrochenen Arm sah.

„Was suchtet ihr überhaupt hier? Alles in Ordnung mit euch?“

„Es könnte schlimmer sein“, meinte Aćh, „Wir suchten diesen Takuri, und wollen nun nur noch zurück an den Nestbaum. Das hier ist übrigens Barz, ein Fremder aus einem weit entfernten Land. Er spricht unsere Sprache kaum.“

Haamun nickte Barz höflich zu. Dieser starrte weiterhin fasziniert auf das Pulversäcklein an Haamuns Gürtel.

Während Haamun sie ins Freie geleitete, stießen weitere Minenarbeiter zu ihnen, die wissen wollten, was diese hirnverbrannten Reisenden denn in dieser Mine zu suchen hatten. Als der Suchtrupp vollständig war, begaben sie sich alle gemeinsam zum Lager der Minenarbeiter weiter westlich. Dort wurden Aćh und Barz verpflegt und verarztet, ehe sich alle Arbeiter um ein Lagerfeuer versammelten. Der Morgen war bald da, und manchmal war kein Schlaf bekanntlich besser als gar kein Schlaf.

Haamun entzündete das Lagerfeuer mit magischen Mitteln, doch brauchte er einige Versuche. Immer wieder schnippte er seine Hand gegen die Scheite und murmelte etwas vor sich hin, bis endlich ein grüner Funke übersprang. Erschöpfte sich seine Kraft mit der Zeit?

Barz rückte näher an Haamun und fragte interessiert: „Alle Minenarbeiter. Magie?“

Allgemeines Gelächter von den Arbeitern war zu vernehmen.

„Ich wünschte, es wäre so!“, rief eine Arbeiterin, „Stell dir vor, wir alle könnten derartig Kräfte führen. Aber unser Haamun hier ist etwas Besonderes. Seine Fähigkeiten sind so vielseitig wie unberechenbar. Und das ist nicht alles. Aus einem fernen fremden Land will er kommen, ja, gar den Kuolema allein überquert haben! Erzähl ihnen davon, Haamun!“

Haamun schüttelte seinen Kopf und meinte leise: „Das ist eine Geschichte für ein andermal. Ich denke nicht gerne daran zurück. Damals war ich noch nicht Haamun. Und ihr glaubt mir doch ohnehin nicht.“

„Vielleicht glauben wir dir schon“, warf Aćh ein, „Wir haben kürzlich einiges erlebt, was aus Legenden kommen könnte. Eine Eis-Dämonin wollte uns rekrutieren und Sturmgeist hat unser Hängeschiff angegriffen.“

Haamun grinste schief: „Ah, mit diesen Geistern kann ich auch nichts anfangen. Sie mögen meine Hexerei nicht. Glaubt mir, wegen ihres langen Lebens liegt ihnen Vergebung sehr lange fern.“

„Barz hier will auch aus einem weit entfernten Land stammen. Und wir haben soeben den Kuolema erklommen und das ewige Eis hinter dem Felsentor mit eigenen Augen gesehen. Wir würden dir vermutlich einige verrückte Geschichten glauben.“

„Ihr habt das Felsentor gesehen?“, fragte Haamun erstaunt, „Ich spürte damals bei meiner Überquerung des ewigen Eises einen dunklen Dämon, der mich aus seinem Versteck beobachtete. Und ich spürte mächtige Magie über dem Felsentor liegen, als ich daran vorbeischritt. Es schien irgendetwas dahinter einzusperren. Und die Wolken, die das ewige Eis stets in Schatten hüllen, werden von einer Magie festgehalten, an der die Spuren der Drachen hing. Ich war froh darüber, das ewige Eis hinter mir gelassen zu haben. Magische Flammen an meiner Seite hin oder her, diese Schluchten jagen mir einen Schauer ein.“

Barz murmelte etwas Unverständliches vor sich hin. Dann, als wäre ihm plötzlich ein Licht aufgegangen, wandte er sich erneut an Haamun: „Du sagen was Land du kommen?”

Haamuns vom grünlichen Licht gruselig beleuchtetes Grinsen erstarb, als der Hexer tonlos flüsterte: „Ich stamme aus Andor.”

Barz klatschte triumphierend in seine Hände.

Aćh zog ihre Augenbraue hoch.

Turr gurrte.\bigskip

***\bigskip

Der Mond stand hoch am Himmel und schien sein silbriges Licht über Silberland. Im Norden der Insel heulten Nordskrale in die Nacht hinein, doch Iril war so stark auf ihre Arbeit fokussiert, dass sie die Rufe nicht wahrnahm. Die Runenschülerin saß vor dem Eingang zur Silbermine und blickte durch eine vor ihr linkes Auge geklemmte Lupe auf ihr Werk. Sorgfältig ritzte sie feinste Runenfolgen auf den Rand des silbernen Rundschilds in ihrem Schoß. Hin und wieder blickte sie vom Schild auf, um ihre Zeichnungen mit einem der dutzend Notizblätter abzugleichen, welche sie um sich herum verteilt hatte.

Jemand klopfte ihr auf die Schulter.

„Iril, komm rein. Heute wirst du nicht mehr fertig. Kreatok hat seine Schilde auch nicht in einer Nacht erschaffen. Selbst die besten Meister brauchen ihren Schlaf.“

Es war ihre alte Runenmeisterin. Jede noch so kleine Fläche ihrer Haut war mit einer Runentätowierung überzogen. Ihr kahl rasierter Schädel glänzte im Licht der Nacht in allen Regenbogenfarben.

„Noch nicht“, protestierte Iril, „Nur noch den dritten Runenreif, dann sind die Gravierungen abgeschlossen.“

„Wie du meinst“, grinste die Meisterin, „Lass mich dir wenigstens einen heißen Cocoa bringen.“

Während sie sich von Iril entfernte, spielte sie gedankenverloren mit dem grünlich schimmernden Hammer an ihrem Gürtel.






\newpage
\section{Epiloge}


„Hey, Ijs, kommt dein älterer Bruder nicht mit?“

„Eforas? Ne, der hat Schiss. Mein Vater hat ihm zu viele Gruselgeschichten über den Kuolema erzählt.“

„Der olle Saro soll mal nicht so tun. Wie steht es um dich?“

„Ich?! Ich habe mich noch nie vor einer Herausforderung gefürchtet! Außerdem, hast du nicht von Aćh und Barz gehört?“

„Von wem?“

„Die beiden haben sich hoch ins Gebirge gewagt und sind völlig unversehrt, ja, gar mit einem Takuri an ihrer Seite, zurückgekommen.“

„Diese Hüterin und dieser Fremde, der sich bei den Bergleuten verdingt? Ja, natürlich habe ich von ihnen gehört.“

„‚Unversehrt‘? Hatte nicht einer von ihnen seinen Arm verloren?“

„Gebrochen, nicht verloren. Das macht einen Riesenunterschied.“

„Und, glaubt ihr ihre Geschichte? Dass sie wirklich Eisdämonen getroffen haben?“

„Ja, logo. Die sind ehrenhafte Helden, die lügen nicht. Die beiden waren auf der Fahrt zwischen Thelot und Agarb dabei, als unser Hängeschiff vom Sturmgeist angegriffen worden. Haben uns gerettet. Heldenhaft, sage ich euch.“

„Na bitte! Dann ist es ja definitiv möglich, bis ins Gebirge vorzudringen und zurückzukommen.“

„Aber nur weil die beiden es schaffen, heißt das nicht, dass wir das auch können.“

„Ijs, kriegst du etwa kalte Füße?“

„Niemals.“\bigskip







Drei Tage und eine Lawine später\bigskip



Ijs schlidderte ein weiteres Mal auf der riesigen Eisfläche aus. Diesmal blieb er auf dem ewigen Eis liegen. Er hatte einfach nicht mehr die Kraft dazu, auch nur einen seiner gefühlslosen Finger zu heben. Wirre Gedanken und Erinnerungen schwirrten durch seinen unterkühlten Kopf. Bilder des Todes, die kein fühlendes Wesen je sehen müssen sollte. Dazwischen hallten Gesprächsfetzen aus einer gefühlt Jahre zurückliegenden Vergangenheit durch seinen Kopf. Sein Vater Saro, der ihn mit gerunzelter Stirn und Furcht in den Augen vor dem Kuolema warnte. Sein älterer Bruder Eforas, der mit gesenktem Blick meinte, dass er auf Ijs‘ wagemutigen Ausflug lieber nicht mitkommen würde.

Kaltes, glattes, ewiges Eis presste gegen Ijs‘ Backe, die so kurz vor dem Erfrieren war, dass sie nicht einmal mehr kribbelte. Ijs spürte keinen Hunger mehr, und keinen Schmerz, nur noch eine wohlige Wärme, die sich in seinem Körper ausbreitete. Er war der letzte Überlebende, da war er sich sicher. Er hatte seine Freunde in den Tod geführt. Da war es doch nur fair, dass er ihnen nun in den Tod folgte. Ijs hatte gehört, dass man vor seinem Tod sein Leben noch einmal vorbeiziehen sah, doch nichts dergleichen geschah hier. Es wurde einfach nur alles dunkler, und immer dunkler.

„Junge!“, erklang eine eisige Stimme, die Ijs nicht näher einordnen konnte. War sie eine Illusion? Realität?

Ein eiskalter Griff packte Ijs und drehte ihn unsanft auf den Rücken. Weiße Schemen schwirrten in seinem Gesichtsfeld. Dann, plötzlich, erkannte er verschwommen ein runzliges schneeweißes Gesicht mit einem langen verfilzten Bart, das sich auf ihn niederbeugte. Es war kein menschliches Gesicht. Die stahlblauen Augen zeigten keine Emotionen, und die abgebrochenen Überreste eines Geweihs ragten links und rechts aus dem Schädel seines Gegenübers hervor.

„Junge!“, wiederholte der Eisdämon mit rasselnder Stimme, „Du sollst hier nicht sterben. Das wäre eine Verschwendung. Du kannst Tari dienen. Dein Körper ist noch frisch. Ganze fünfhundert Jahre lang kannst du noch Tari dienen. Erst danach sollst du diese Welt verlassen, und zwar, indem du einen Nachfolger wählst. Indem du all deine Kraft an einen Nachkommen übergibst. So, wie ich es hier mit dir tue. Ich, Beriandur, der hiermit dein Vater werde.“

Ijs verstand die Worte seines Gegenübers nicht mehr, aber das musste er auch nicht. Das würde er später tun.

Beriandur der Eisdämon grabschte nach Ijs‘ Hand und führte dessen klamme Finger zu etwas Scharfem, Kantigem, Eiskalten. Wie Ijs‘ Hand die Eiskristallkette streifte, so begann sein Geist, sich zu klären. All die Furcht und das Trauma fielen von ihm ab. Tiefe Ruhe erfüllte ihn ebenso wie fremde Eindrücke und Erinnerungen, die nicht die seinen waren.

„Ijs heißt du also. Willkommen, mein Sohn Ijsdur, im ewigen Eis“, erklang erneut die klirrende Stimme des Eisdämons. Ijs‘ Blick fokussierte sich. Nun konnte er wieder den wolkenverhangenen Himmel über sich erkennen, und die Schneeflocken, die ihm entgegengewirbelt wurden. Und den Eisdämon, der sich über ihn beugte und ihn mit ausdruckslosem Blick betrachtete.

Urplötzlich ergriff der Eisdämon die Eiskristallkette, die um seinen eigenen Hals hing, und riss sie gewaltsam von seinem schneeigen Körper. Er krümmte sich und ächzte, während sich von den zurückgelassenen Einbuchtungen der Eiskristalle in seinem Hals Splitter und Spalte in alle Richtungen seinen Körper entlang ausbreiteten. Ijs wollte zurückweichen, doch sein unterkühlter Körper weigerte sich zu gehorchen.

Erleichterung machte sich auf dem Gesicht des Eisdämons breit.

„Ich bin alt, mein Sohn. Mein Geist ist nicht mehr klar. Ich fürchtete schon, dass ich keinen würdigen Nachfolger mehr finden würde, ehe dieser Körper vergehe. Doch dem war nicht so. Nun kann ich in Frieden diese Welt verlassen.“

Mit diesen Worten riss der bärtige Eisdämon Ijs‘ Mantel auf und presste ihm die Eiskristallkette aufs Schlüsselbein. Ijs hörte etwas knacken und knirschen, während sich eine Taubheit von seinem Hals aus über seinen ganzen Körper ausbreitete.

„Willkommen, Ijsdur, zu Deinem zukünftigen Dasein in den Diensten Taris.“

Das waren die letzten Worte des Eisdämons. Er trat einige Schritte zurück, stolperte dann und verschmolz mit der Eisfläche, versank buchstäblich darin.

Ijsdur versuchte vergeblich, die fremden Eindrücke und Erinnerungen in seinem Geist zu sortieren und zu vorstehen. Dann, langsam, gehorchte ihm sein Körper wieder. Er hob seine zitternden Hände, doch waren diese schneeweiß geworden, als wäre er schon längst tot.

Langsam erhob er sich. Es schien, als ob er dem ewigen Eis des Kuolema-Gebirges entwuchs. Er schaute auf die Gestalt, die vor ihm lag. Die Gestalt seines "Vaters", des Eisdämons Beriandur, die mehr und mehr vor seinen Augen verschwamm und mit der endlosen Eisfläche eins wurde, bis nur noch einige wenige Eiskristalle übrig blieben.

Als er an sein eigenes Spiegelbild erblickte, waren an seinen Schläfen kleine Geweihe gewachsen. Und seine Ohren hatten sich zugespitzt.

Doch fürchtete er sich nicht.

Denn Ijsdur verstand nun, warum er hier war.

Dies war nun seine Heimat.

Und es war gut so.\bigskip





\az{Jahr 65}



Zwei Jahre später\bigskip



Ein Kopf, der den Unterirdischen Krieg mit eigenen, blutroten Augen miterlebt hatte, sank tief in den blutgetränken Boden Andors ein. Er schmeckte Matsch, Dreck und Blut. Ein letztes Ächzen drang aus dem riesigen Maul.

Damit schied Tarok, der letzte Drache, für immer aus dem Leben. Seine Flamme des Zorns erlosch.

Südlich davon, tief, tief unter den Bergen des Grauen Gebirges, wogte in einer gewaltigen Höhle ein Geflecht aus Düsternis umher. In der Mitte ragte ein riesiger Baum empor. Schimmernde Edelsteine ragten hier und da aus seiner längst versteinerten Oberfläche. Doch klar erkennbar unter dem durchscheinenden hellen Gestein floss in seinen dicken Adern stetig rotes Blut durch den Baum. Schon seit Jahrtausenden, seit der Erschaffung durch die uralten Drachen, hatte der Baum des Blutes hier in Krahal gestanden, den Drachen Leben gespendet und ihnen Kraft und Energie verliehen.

Nun stockte der Herzschlag des Baumes. Das rote Blut gerann. Ein Glimmen lief über die Rinde des Baums und verwandelte helles, lebendiges Gestein in tote, schwarze Asche. Der Baum verödete innert Minuten. Die Schreie tausender Drachenseelen erklangen ein letztes Mal und verstummten dann für immer.

Einen Augenblick lang, regte sich nichts mehr, als wäre die Zeit selbst in Krahal stehen geblieben.

Dann, urplötzlich, wurde Krahal zutiefst erschüttert. Das Geflecht der Düsternis wirbelte wild umher und zerriss die erstarrten Überreste des blutigen Baumes. Krahal verschlang sich selbst. Die gewaltige Höhle stürzte in sich zusammen.

Das Splittern von Stein hallte in den Bergtälern wider und übertönte die Musik der Toten in den Schluchten des Grauen Gebirges. Felsen barsten und Lawinen krachten. Erschütterungen gingen vom Epizentrum aus. Sie zogen sich in Wellen durchs Graue Gebirge, jagten durch Fels und Stein, zwischen Schluchten und durch die unterirdischen Gänge der Schildzwerge.

An der Ruine der Winterburg stürzte eine weitere Mauer unter lautem Gekrache in sich zusammen.

Rhona und Grone, die Stammesältesten der Agren, fielen in eine stille Umarmung, während ihre Höhle um sie herum bebte.

Schildzwerge fielen auf ihre Knie und sandten Stoßgebete aus, während ihre Umgebung zitterte und die Tiefminen Flammen spuckten.

Der Mechaniker Kjall raste in seinem kleinen Museum umher und versuchte unter Einsatz eines seiner Tiefminen-Golems verzweifelt, seine Schätze – Fackeln von Cavern, schöne Schilde, ein zersplittertes Hadrisches Stundenglas – auf ihren angestammten Sockeln zu behalten.

Der Riesenkrake Irlok zuckte unruhig umher, während gewaltige Wellen ans Ufer des Geheimen Sees brandeten und Stalaktiten von der Decke ins Wasser klatschten.

Die von Krahal ausgehenden Beben wogten auch bis ins Kuolema-Gebirge.

Tulgorische Höhlenwichte sprangen beiseite, als ihre Gänge sich teilten.

Ein tiefer Spalt tat sich im ewigen Eis auf.

Siantari, die oberste Dämonin des ewigen Eises, breitete ihre Arme aus und rief eine wirbelnde Wolke aus Schnee und Eis zu sich. Der Wirbel hob sie in die Höhe und ließ sie einige Meter über der zitternden Eisfläche schweben, weg von den Vibrationen der Erde. Ihre eisblauen Augen starrten regungslos auf das Felsentor, das ihr nun schon seit 49 Dekaden den Zugang zur restlichen Welt verwehrte.

Das Felsentor erzitterte wie der gesamte Rest des Kuolema-Gebirges. Staub rieselte herab und Splitter bröckelten. Doch die Magie der Temm, die das stabile Tor vor Jahrtausenden vor dem ersten Dämon des ewigen Eises verschlossen hatten, ließ nicht nach.

Dann war es vorbei. Die Welt, das Gebirge, das ewige Eis, sie alle standen wieder ruhig da. Krahal war nicht mehr. Shan, die Schattenhexe, blickte verwirrt um sich, als sie die gewaltige Seelenkraft Krahals tief unter der Erde auf einmal nicht mehr fühlen konnte.

Doch das Felsentor vor dem ewigen Eis des Kuolema hatte standgehalten. Die Eisdämonen waren immer noch in der Schlucht dahinter gefangen.

„Fećht!“, fluchte Siantari. So schnell, wie der Ärger in ihr aufgeschwellt war, wurde er auch schon wieder von der allgegenwärtigen Ruhe des Eises überdeckt. Schon lange hatte sie nicht mehr so stark gehofft. Wer nicht hoffte, konnte auch nicht enttäuscht werden. Und dennoch war Siantari nun enttäuscht, dass das Felsentor dem Beben standgehalten hatte.

Dafür hatte, zunächst unbemerkt von Siantari, etwas Anderes dem Einsturz Krahals nicht standgehalten. Mochten die Temm vor Jahrtausenden ihre eigene Magiequelle für das Felsentor genutzt haben, so war die unnatürliche Wolkenkuppel, die sie über die Gipfel des Kuolema gelegt hatten, durch uralte Drachenmagie gespeist worden. Denn auch die Drachen der Urzeiten hatten ein Interesse daran gehabt, dass die Eisdämonen im Fahlen Gebirge und ihre tiefe Schlucht abgeschirmt vom Rest der Welt blieben.

Nun, wo Krahal eingestürzt und die magischen Ströme der Drachen versiegt waren, lichteten sich die Wolken am Gipfel des Kuolema langsam im beißenden Wind. Der Himmel über der ewigen Eisfläche war wieder zu sehen. Zum ersten Mal in beinahe fünf Jahrhunderten erblickte Siantari wieder das Licht der Sonne. Ungewohnt prickelte es auf ihrer eisigen Haut.\bigskip







Es gab auch noch einen anderen, der verstand, warum sich die Wolken von den Gipfeln des Kuolema verzogen. Haamun, der Hexer aus Andor. Er hatte bei seiner Überquerung des Gebirges die Magie der Temm gespürt, die das Felsentor für die Eisdämonen undurchdringlich machte. Und er hatte die Drachenmagie gespürt, die die Wolken auf den Gipfeln des Kuolema-Gebirges gehalten hatte.

Haamun trat vor seine Hütte und blickte zum Kuolema hinauf. Nun waren die schneebedeckten Gipfel der Berge sichtbar geworden. Der Himmel jenseits des Gebirges erglühte in rotem Licht. Da wusste Haamun, dass der letzte Drache besiegt worden war. Und wenn der Drache angegriffen hatte, so hieß das, dass der alte König tot war. Und dass es an der Zeit war, nach Hause zurückzukehren.

Haamun wurde von einem Hilferuf aus seinen Gedanken gerissen.

„Haamun!“, erklang die Stimme einer Minenarbeiterin, die aus einem Stollen zu rennen kam, „Haamun! Die Erde hat gebebt. Der Stollen hat nachgegeben. Seram hat eine Platzwunde am Kopf. Kannst du...“

Haamun hatte bereits eine Handvoll Heilkräuter aus der Steppe ergriffen und rannte los. In den Jahren, seit er hierhergekommen war, hatte er viele der seltenen und oft unbekannten Kräuter der Steppe gesammelt, Tränke gebraut und die verschiedensten Salben, Pulver und Gifte hergestellt. Wenn es einen Notfall gab, war er zur Stelle, um zu helfen. Selbst wenn den Geholfenen am Ende etwas Unvorhergesehenes geschah, wenn sie seit der Heilung grüne Haut hatten oder nur noch im Versmaß sprechen konnten... nun, solange Haamuns Hexerkunst ihr Leben rettete, war es das weitaus wert.

„Der Stollen ist nicht zusammengestürzt“, behauptete Seram trotzig, während Haamun seinen Kopf verarztete, „Im Gegenteil! Aus dem Felsen hat sich ein alter Stollen wieder befreit. Während ihr alle euch vor niederrieselnden Kieseln versteckt habt, habe ich gesehen, wie der Berg sich mir gegenüber öffnete! Einen uralten Gang mit Wänden voller Runen und ähnlicher Kritzeleien sah ich. Das Beben hat einen der uralten Gänge der Temm freigelegt!“

Haamun hielt inne. Er hatte schon seit langem gespürt, dass die Stollen unterhalb des Kuolema-Gebirges sein Weg zurück nach Andor sein könnten. Nun, mit dem Tod des alten Königs und des Drachen, begann eine neue Ära für Andor. Eine, an der Meres wieder teilhaben konnte. Und ausgerechnet jetzt wurde ein uralter Temm-Stollen freigelegt? Alle Spielsteine fielen an ihre Stelle. Dieser Gang würde sie nach Andor führen.

Es war an der Zeit. Nun musste er rasch handeln. Mitreisende auftreiben für die Rückkehr in seine Heimat. Haamun selbst, seine Tränke und seine Geschichten waren beliebt bei einigen Bauern nahe des Gebirges, insbesondere den jüngeren. Vielleicht würden sich einige davon ihm anschließen. Und in den nahe gelegenen Dörfern konnte er sicherlich einige Händler, Baumeister und Kartographen zusammentreiben, deren Neugier stärker war als die Furcht vor den Geistern der Berge. Mit einer ganzen Gesellschaft würde er in Andor aufkreuzen und in den unruhigen Zeiten nach dem Tod von König und Drache aushelfen. Sie würden ihm verzeihen und den längst vergangenen Zwist beilegen. Er konnte wieder Meres sein.\bigskip







Aćh war gerade unterwegs in der Metropole Agarb, als sie ein Falke mit einer Nachricht von Barz erreichte.\bigskip



„Hochverehrte Hüterin Aćh,



Ein Beben hat am Ende eines Stollens der Bergarbeiter eine Verbindung zu einem uralten Gang der Temm freigelegt. Haamun, der Hexer, ist sich absolut sicher, dass er uns nach Andor führen kann. Er trommelt eine Gruppe Reisender zusammen, die sich für das Königreich auf der anderen Seite des Kuolema interessieren. Und der Gang scheint gerade groß genug für Sabri zu sein. Ich habe vor, mich ihm anzuschließen. Bist du dabei? Wir brechen in wenigen Tagen auf.



Beste Grüße

Barz“\bigskip



Wie weit Barz‘ Sprachkenntnisse doch in den letzten Jahren gewachsen waren!

Aćh schluckte schwer und ließ ihre Optionen Revue passieren. Endlich war es soweit. Seit dem Gespräch mit dem Hüter der Zeit damals hatte Barz gewusst, dass es einen Weg für ihn nach Andor und zurück zu seiner Familie geben würde. Und in den Jahren seit damals, in denen sie und Barz sich näher gekommen waren, war mehrmals die Frage aufgekommen, ob sie ihn begleiten würde.

Sie mochte ihre Arbeit am Nestbaum der Takuri, und sie mochte ihre Freunde und Bekannte dort. Doch sie mochte auch Barz, und sie mochte Abenteuer. Sturmgeister und Eisdämonen, auf so etwas stieß man nicht, wenn man sein Leben lang am selben Baum verbrachte. Andor, das fremde Land, konnte bestimmt von diplomatischem Kontakt mit Tulgor profitieren, und als Tochter einer Diplomatin war Aćh für eine solche Rolle vermutlich besser geeignet als ein Hexer und seine zusammengewürfelte Reisegruppe.

Noch während Aćh ihre Gedanken sortierte, klopfte es an ihrer Tür. Als sie sie öffnete, blickte sie überrascht in das Gesicht ihrer Mutter... und in die verschmitzten Augen des Hüters der Zeit, der auf ihrem Rücken saß. Seinen Turban hatte er gegen eine braune Kutte eingetauscht.

„Na, schon reisefertig gemacht?“, krächzte der Temm grinsend, „Ich schließe mich dir gleich an, wenn das für dich in Ordnung ist. Schließlich will ich auch mit nach Andor!“\bigskip







Gemeinsam hatte Aćh mit dem Takuri das Land durchquert und vor wenigen Tagen das Gebirge erreicht. Die Sonne war hinter den schneebedeckten Bergspitzen untergegangen, und ein roter Mond stand am Himmel. Nun sah der Takuri Aćh noch einmal an, und in einem kurzen Augenblick rotglühenden Feuers beendete er sein Leben, so wie er es schon so oft beendet hatte. Als kleines Küken watschelte er in Aćhs Arme und kuschelte sich an sie.

Der Hüter der Zeit betrachtete Turr mit zusammengekniffenen Augen:

„Ist alles in Ordnung mit dem Takuri? Er ist nun schon zum zweiten Mal in zwei Wochen verglüht.“

Aćh schüttelte ihren Kopf.

„Nein, irgendetwas ist besonders an ihm. Seit seinem Verschwinden und seiner Zeit im ewigen Eis altert er viel zu schnell. Manchmal durchlebt er seinen Zyklus in wenigen Tagen!“

„Vielleicht finden wir ja in Andor etwas, das ihm helfen kann“, meinte Barz optimistisch, und setzte sich neben die beiden ans Lagerfeuer, „Und daran zu leiden scheint er auch nicht.“

Mit einem nachdenklichen Blick setzte Barz nach: „Ist das eigentlich nach jedem Zyklus immer noch derselbe Takuri?“

Aćh grinste. Sie hatte sich eine solche Frage schon oft gestellt. War ein Takuri nach seiner Regeneration noch derselbe Takuri? Wie sollte man sie nur beantworten? Der neue Takuri würde nie identisch aussehen wie der alte, aber dennoch würde der neue Takuri dem alten erheblich ähnlicher sehen als allen anderen. Vielleicht mit einer anderen Federanordnung oder sonstigen Kleinigkeiten. Auch wenn das Küken noch auf den alten Namen reagieren würde, stellte sich die Frage, ob esdieselben Erinnerungen wie der frühere Takuri trug. Aćh hatte oft genug mit einem Töpfchentraining von vorne anfangen müssen. Gewisse Eigenheiten, bestimmte Verhaltensmuster, blieben dennoch nach jeder Regeneration, soweit sie sagen konnte. Wobei ihre Einschätzung natürlich vielen Vorurteilen unterlag. Wie wahrheitsgetreu sie war, war schwer einzuschätzen.

Aćh beschloss, Barz eine Gegenfrage zu stellen: „Was würdest du sagen: Wenn du am Morgen aufwachst, bist du dann derselbe Mensch, der eingeschlafen ist?”

„Natürlich”, meint Barz, „Wenn wir uns nach jedem Schlaf als andere Person sehen würden, gäbe es ja kaum einen Grund, uns groß um die nächsten Tage zu sorgen. Wir würden doch einfach unseren einzelnen Tag mit kurzweiligen Genüssen verschwenden und die Menschheit hätte sich gar nie erst zu einer Gesellschaft entwickelt.”

Das war nicht die Sorte Antwort, die Aćh erwartet hatte.

„Pfff. Da kann ich ja nicht einmal eine Analogie zur Regeneration des Takuri ziehen. Außerdem verzichten wir doch für unsere Nachkommen auf gewisse Genüsse, könnten wir das dann nicht auch hier tun? Und dass etwas ohne den Glauben an etwas nicht entstanden wäre, sagt ohnehin nichts darüber aus, wie wahr dieses etwas wirklich ist, oder etwa nicht?”

Barz legte den Kopf schief: „Da magst du Recht haben. Ich weiß, dass du diesen Takuri hier nach jeder Regeneration immer noch Turr nennst. Macht ihr das normalerweise immer so, oder gebt ihr den Älteren jeweils neue Namen, wenn sie sich regenerieren?”

„Wir sehen von Fall zu Fall, ob wir einen neuen Namen verleihen wollen. Oftmals ist es praktischer, einfach dieselben Namen zu verwenden. Manchmal ist ihre Persönlichkeit nach einem Zyklus aber einfach zu anders, und dann wählen manche Hüter neue Namen. Dann gibt es oft eine Namenszeremonie, wie wenn einer von uns den seinen wechselt.“

„Tulgori wechseln ihre Namen?”

„Ihr Barbaren etwa nicht?“

„Natürlich nicht!“

„Hast du denn in all deiner Zeit hier keine Namenszeremonie mitgekriegt? Heißt du als Kind etwa gleich wie als Jugendlicher? Wie als Volljähriger? Wie als alter Erwachsener?”

„Natürlich heiße ich immer gleich.”

„Aber ein Kind kann doch noch keinen Namen wählen. Das Kind kennt die Bedeutung ja gar nicht.”

„Das muss es auch nicht. Die Eltern wählen ja für das Kind.“

„Und was, wenn die Eltern sich nicht einig sind?“

„Dann diskutieren sie das halt aus.“

„Und was, wenn das Kind den Namen später nicht mag?“

„Jeder, den ich kenne, mag seinen Namen. Irgendwie.“

„Was, wenn nicht? Ich mochte meinen Geburtsnamen nicht.“

„Du wusstest ja auch, dass du ihn ändern kannst. Wir gewöhnten uns dran.“

„Na eben.“

„Dann weiß man ja nicht mal, von wem man redet, wenn Leute später immer anders heißen.“

„So oft ändert sich der Name ja auch wieder nicht. Und man gewöhnt sich daran.“

Hinter Barz grunzte es, als seine Echse Sabri ihn endlich erreichte (sie hatte sich in der letzten Stunde unglaublich gemächlich auf ihn zubewegt) und sich ächzend auf den Boden sinken ließ. In den letzten Jahren war sie beachtlich gewachsen, sodass sie ihm im Stehen nun schon bis zur Schulter reichte. Sie war über und über beladen mit verschiedenen Pulversäcklein. Barz‘ Vorräte mochten beim Kampf gegen den Sturmgeist erheblich reduziert worden sein, doch hatte er in der Steppe Tulgors und in zahlreichen Gesprächen mit dem Hexer Haamun seine Pulversammlung teils aufstocken (Mondbeeren für sein Bannpulver schienen beispielsweise buchstäblich überall zu wachsen) und teils gar alternative Pulvermischungen zusammenstellen können.

Haamun und sein Trupp waren gleich zwei Tage nach dem großen Beben, dem Verschwinden der Wolken über den Gipfeln und dem Freilegen des Temm-Stollens in der Mine verschwunden, um die Lage mit eigenen Augen auszukundschaften. Schon am ersten Abend waren sie mit glänzenden Augen und einer Handvoll wertvoller Edelsteine in seiner Tasche wieder beim Stolleneingang aufgetaucht und berichteten von einer „lohnenswerten Sackgasse“.

Nun, eine ganze Woche nach dem großen Beben, war die Kompanie schon zum dritten Mal an den Startpunkt zurückgekehrt. Ihre gute Laune hatte sich gelegt und Haamun verbrachte viel Zeit damit, mit einer Art Wünschelrute in seinen Händen im Eingangsbereich verschiedener Stollenabzweigungen umherzulaufen. Die unterirdischen Gänge der Temm hatten sich als wahres Labyrinth herausgestellt, welches nur mit der nötigen Magiekunde navigiert werden konnte. Und Haamun rang mit seinem Stolz.

Eforas, einer seiner engsten Vertrauten, unterhielt sich soeben etwas abseits des Lagerfeuers mit ihm. Man konnte von außen nicht viel vernehmen, doch Aćh glaubte eindeutig, ein „Du sagtest, du würdest den Weg durch den Berg kennen!“ von Eforas zu vernehmen, sowie eine zischende Antwort von Haamun.

Ein Raunen ging durch die Dutzend Händler, Baumeister und Kartographen, die ebenfalls in der Nähe saßen.

Aćh wisperte zum Hüter der Zeit: „Ich weiß, dass diese Temm-Gänge viele Verzweigungen besitzen, aber du bist ein Temm und kannst noch dazu die Zukunft sehen! Kannst nicht einfach du uns durch die Gänge führen?“

„Wer kann schon wissen, was genau ich alles tun kann?“, meinte der Hüter mit einem verlegenen Grinsen, „Aber das würde Haamun wohl nicht gefallen. Bald schon wird er von selbst den richtigen Einfall haben. Und ganz abgesehen davon mag ich die Gesellschaft hier.“

Aćh und Barz blickten sich an und grinsten. In Barz‘ Blick war aber auch eine lange unterdrückte Sehnsucht zu erkennen. Gedankenverloren griff er an seinen Hals und spielte mit einer goldenen Kette, die Aćh ihm vor einem Jahr geschenkt hatte. Barz‘ Ringkette war, soweit sie wussten, noch immer vor dem Felsentor der Eisdämonen eingeschneit.\bigskip







„Was... was ist das?“, stammelte Haamun leise.

„Es ist wunderschön“, murmelte Aćh.

Haamun hatte endlich – so hofften alle – die richtigen Abzweigungen gefunden. Schon seit einigen Tagen waren sie nun die Stollen unter dem Berg entlanggewandert, ohne sich in Sackgassen zu verirren. So langsam wünschten sich alle Reisenden, wieder das Licht der Sonne zu sehen und den Wind in den Haaren zu spüren. Doch der Anblick, der sich ihnen hier unter der Erde bot, war beinahe ein Ausgleich dafür.

Sie hatten eine Höhle unter der Erde erreicht, aus deren Wände dutzende Edel- und Mera-Steine ragten. Ein leises Summen erfüllte den Raum, und wie überall sonst waren auch hier zahlreiche Runen am Boden zu sehen. Turr entfaltete seine Flügel und erhob sich zum ersten Mal in einigen Tagen zu einem kurzen Rundflug in der geräumigen Höhle. Er stieß einen Freudenschrei aus. Seine Schwanzfedern glommen glücklich.

Barz klopfte der dem Trupp langsam nachtrottenden Sabri auf den Rücken. Sie folgte dem Angebot bereitwillig und ließ sich zu Boden plumpsen. Schon bald ertönte ein Schnarchen aus ihrem breiten Mund.

Der Hüter der Zeit kniff seine Augen zusammen und runzelte seine Stirn. Nach einiger Zeit öffnete er sie wieder und sprach:

„Tut mir leid, werte Reisende. Ich habe keine Ahnung, um was für eine magische Höhle es sich hier handelt. Ich werde dies mein ganzes Leben lang nicht in Erfahrung bringen. Und ihr demnach wohl ebenfalls nicht.“

Murmeln war unter den Reisenden zu hören. Dann sprach Haamun leise und monton:

„Es wäre schon fast schade, dieses schöne Bild zu zerstören und die Mera-Steine mitzunehmen.“

Eforas nickte: „Fast.“

Energisch trat er nach vorne an die Wand und untersuchte einen Mera-Stein, der darin verankert war. Wie üblich sah es aus, als wäre der farbige Stein mit dem umliegenden Felsen verschmolzen.

„Sitzt fest wie angeleimt.“

Eforas zückte einen Pickel und ließ ihn auf den Felsen rund um den Stein niederfahren. Der Pickel zerbrach mit einem hellen Klingen. Es erinnerte Aćh an daran, wie die Werkzeuge der Minenarbeiter auch am Nestbaum der Takuri zerschellten.

„Hier wollte jemand nicht, dass diese Schätze entwendet werden“, meinte Eforas schicksalsergeben.

„Schaut mal her!“, rief Barz da. Er hatte in den Rucksack auf Sabris Rücken gelangt, den die Reisenden mit allerlei Schätzen und Gaben beladen hatten, und ein rautenförmiges Felsstück herausgezogen. Aus dem Felsstück ragte ein schwach grünlich leuchtender Mera-Stein. In dieser Form wurden die Steine üblicherweise weiter versandt, um weiter Inlands komplett aus dem Felsen geschält zu werden. Ein komplexer Prozess, der den Minenarbeitern abgenommen wurde.

Barz hob den grünen Mera-Stein in die Höhe und brachte ihn in die Nähe eines in die Höhlenwand eingelassenen grünen Mera-Steins. Je näher Barz trat, desto heller leuchteten die beiden magischen Steine, und ein immer lauteres Summen war zu vernehmen. Während Barz die Distanz zwischen den Mera-Steinen verringerte, wurde das Summen zudem immer dumpfer und tiefer, bis die Reisenden es nicht mehr hören konnten, doch die Vibrationen immer noch in ihrem Körper zu fühlen glaubten.

Dann machte es „klick“, als die beiden Mera-Steine einander berührten. Das helle Leuchten erlosch – und der Mera-Stein aus der Wand rutschte herunter und kullerte über den Boden, als hätte ihn nie etwas in der Wand gehalten.

Da waren die Reisenden aber außer Rand und Band. Innert Minuten hatten sie alle Mera-Steine mithilfe anderer Mera-Steine aus der Wand gelöst und eingepackt. Nachdem jemand feststellte, dass sich die Edelsteine ebenfalls so lösen ließen, dauerte es sogar noch kürzer, bis auch diese allesamt ordentlich in einem Sack versammelt waren. Außer die ein, zwei Steinchen, die vielleicht jemand lieber in seine eigene Hosentasche hatte wandern lassen. Sabri wurde eine weitere Tragetasche voller Mera-Steine übergehängt. Die Echse ächzte, wehrte sich aber nicht.

In vielen Augen blitzte Begeisterung über den unerwarteten Fund. Der Hüter der Zeit blickte hingegen nur nachdenklich drein. Und auch Haamun fiel nicht in Freudenschreie ein. Ausgerechnet Meres machte sich Gedanken darüber, ob das Lösen der Mera-Steine ungewollte Konsequenzen haben könnte.\bigskip







Siantari tigerte hinter dem Felsentor über das ewige Eis. Hin und her, hin und her. Ihre Augen waren starr auf den Himmel gerichtet. Die Wolken hatten sich verzogen, und die Sterne zogen ihren Blick magisch an. Die klirrende Kälte, die Siantari unerbittlich entgegenschlug, konnte ihr nichts anhaben. Sie war die frostigen Tage und Nächte im Kuolema-Gebirge gewohnt.

So stand sie da und blickte in den weiten Kosmos, weiß auf Blau. Ein Kosmos, in dem sie und ihre Eisdämonen doch nur ein so kleiner, unwichtiger Teil waren. Noch. Denn eines Tages würde das ewige Eis über alles unter dem Himmelszelt gewachsen sein. Und Siantari würde dabei geholfen haben.

Sie verspürte Hoffnung. Ja, die Sterne erfüllten sie mit Hoffnung.

Die uralte Drachenmagie, die die Gipfel des Kuolema stets in Wolken verhüllt hatte, war verklungen. Alles verfiel mit der Zeit. Auch uralte Magien. Und so konnte es nicht mehr unendlich lange dauern, bis auch ihr Felsentor geöffnet wurde und sie oder einer ihrer Nachfolger das ewige Eis verlassen konnte.

Etwas regte sich in ihrem Hinterkopf. Eine Wahrnehmung, klein und fein, und doch so unglaublich relevant.

Sie fühlte, dass der Fremde zurückgekehrt war. Der Hexer mit den grünen Flammen, der das ewige Eis vor einigen Jahren, die für Siantari kaum mehr als ein Wimpernschlag her waren, vom Osten her überquert hatte. Doch überquerte er den Kuolema nicht. Er ging unter den Bergen hindurch, und er war nicht allein. Und sie trugen Mera-Steine bei sich, diese Narren!

Siantari spürte, wie sich etwas grundlegend änderte. Damals, als sie zur Eisdämonin geworden war, hatte sie auf einmal eine Erkenntnis ergriffen. Dass sie nie wieder durchs Felsentor gehen konnte. Dass sie nun eine Gefangene hinter dem Felsentor war, dazu verdammt, 500 Jahre lang das ewige Eis zu nähren, um am Ende einen neuen Eisdämon zu erschaffen und dann für immer im Eis zu versinken. Diese Erkenntnis, magisch auferlegt durch das Felsentor und alle anderen uralten Vorrichtungen der Temm, die die Eisdämonen von allen Seiten in diesen Schluchten einsperrten... noch vor wenigen Augenblicken war diese Erkenntnis Siantari noch bewusst gewesen, wie eine leise Last, die ihr seit Jahrhunderten auf den Schultern lag. Nun war sie einfach nicht mehr da. Sie fühlte sich so unbeschwert. Der Bann war verschwunden.

Misstrauisch schritt Siantari auf das Felsentor zu. Konnte es sein, dass das Tor geöffnet worden war? Es sah noch genau gleich aus wie zuvor. Doch fühlte es sich ganz anders an.

Triumphierend glitt Siantari nach vorne und marschierte durch das Felsentor. Keine uralte Magie hinderte sie daran. Die Ströme aus dem Innern des Bergs waren versiegt.

Der Fremde und seine Begleiter mussten die Quelle des Felsentors gefunden haben. Und ihre Mera-Steine hatten den Mechanismus ausgehebelt. Sie hatten das Felsentor geöffnet. Siantaris Felsentor war offen. Sie war frei!

Ein kurzes Lächeln huschte über Siantaris Lippen.

Und es kam noch besser. Die Reisenden tief unter dem Gebirge trugen weiterhin Mera-Steine mit sich. Siantari konnte die Steine und ihre Lage spüren, so wie sie den gesamten Berg spüren konnte. Schließlich war das ihr Berg. Das waren ihre Steine. Und so würde sie stets wissen, wohin diese Reisenden reisten. Sie konnte ihnen folgen. Sie würden ihr den Weg durch den Berg zeigen. Der Weg in ein fremdes Land. Sollte sie ihnen folgen?

Siantari blickte sich um. Links oder rechts? Wollte sie zurück nach Tulgor, in das Land ihrer Kindheit, oder lieber in den Osten, wohin diese Narren reisten und woher dieser mysteriöse Hexer gekommen war?

Leise Wehmut übermannte Siantari, als sie an ihre Heimat dachte. Ein überaus menschliches Gefühl, wie sie überrascht feststellte. Wollte sie dorthin zurückkehren? Sehen, was aus der Hütte ihrer Familie geworden war?

Nein, sprach sie zu sich. Tulgor bedeutete ihr nichts. Zeit, neue Wege zu beschreiten. Zeit, sich der Zukunft zuzuwenden. Zeit, in das Land im Osten aufzubrechen.

Siantari erhob sich und blickte ein letztes Mal über das ewige Eis des Kuolema. Hier und dort glaubte sie, ferne Silhouetten anderer Eisdämonen ausmachen zu können. Einsame, verwirrte Gestalten.

Siantari war es als erster Dämonin des ewigen Eises gelungen, Eiskristallketten zu fertigen, die ihre Gabe und ihre Mission an andere weitergeben können. Hin und wieder hatte sie aus verirrten Bergsteigern Durs und Doras geschaffen. Sie hatte sie einsetzen wollen, um aus dem Felsentor auszubrechen. Doch hatte die lange Zeit im ewigen Eis den menschlicheren Geistern der Durs und Doras nicht gutgetan. Und nun brauchte Siantari brauchte sie nicht mehr. Das ewige Eis konnte sie auch allein bis in die umliegenden Länder ausbreiten.

„Meine Kinder“, erhob sie ihre kalte Stimme, und das Echo ihrer Worte schallte über die riesige Eisfläche, „Ihr seid nun endlich frei, und frei sollt ihr nun sein. Ich entlasse euch aus meinen Diensten. Geht, wohin euch eure Schicksale ziehen. Verbreitet das ewige Eis. Wenn es so sein soll, werden wir uns wiedersehen.“

Keine Reaktion kam aus dem ewigen Eis. Siantari wusste, dass ihre Worte angekommen waren. Doch wie viele der Eisdämonen hier besaßen überhaupt noch genug Willenskraft, um sich zu irgendeiner Handlung aufzurappeln? Sie war allein. Aber das war in Ordnung. Sie war schon so lange allein gewesen. Sie wandte sich nach Osten und verließ das ewige Eis und das Fahle Gebirge. Zeit, alles Land in eine Eiswüste zu verwandeln.

\begin{center}
    Weiter geht es in \hypref{Runen im Schnee (2022)}.
\end{center}










\begin{chapterbox}
    \chapter{Die Legende der entzweiten Zeit (2023)}
    \label{Die Legende der entzweiten Zeit (2023)}
    \az{63 und 64}

    \begin{center}
        Fortsetzung von \hypref{Der Giftzwerg und die Sphäre (2020)}
    \end{center}
    
    Instabile Zeitportale verbinden zwei Zeiten: Das herbstliche Andor, dessen Bewohner noch nichts von der ewigen Kälte ahnen, und das winterliche Andor, wo einige letzte Helden die Rietburg tapfer verteidigen.
\end{chapterbox}


\az{Jahr 64}

Es war ein eiskalter Tag in Andor. Bauern bibberten in der Rietburg, Kreaturen schlichen durch den Schnee und die Vorratskammern der Andori wurden leerer und leerer. \bigskip

Die Ewige Kälte lag über Andor. Alle Helden waren ins Land der Steppe gezogen, um rettende Kräuter für die Eisschlafenden zu sammeln. Alle? Nein! Einige wenige Helden waren in Andor verblieben und wehrten eine Vielzahl von Kreaturen ab. So viele von ihnen auch dem Eisschlaf anfielen, stetig rückten neue Dunkle Gegner nach. Die Verteidiger Andors wurden immer weniger, ohne dass gute Neuigkeiten aus dem Osten kämen. 

Und noch wussten die Helden nicht, dass eine finstere Macht weitere Gefahren nach Andor rief. Gefahren, die sie mit Dunkler Magie in einem festen Bann hielt. \bigskip

Fürst Hallgard hatte eine Notkonferenz einberufen. Der Herrscher über Cavern spürte bereits die Kälte in seinen Knochen aufsteigen und befürchtete, dass es zu spät sei. „Unsere klügsten Köpfe können dem Winterstein mit ihrem komplexesten Konstruktionen keinen Kratzer zufügen, ohne zu Eisstatuen zu erstarren! Keiner von Rekas Tränken vermag die Kälte zu lindern! Ist dies unser Ende?“ „Warum schicken wir den Winterstein nicht woandershin, weit weg?“, fragte der grimmige Radan. 

„Keine Lehre aus Thoralds Taten gezogen?“, lachte Hallgard trocken. „Wir riskieren damit ohnehin nur, dass dieses mächtige Artefakt in feindliche Finger fällt.“

„Es fehlt uns an der Zeit“, murmelte Mart, „Wir würden bestimmt eine Lösung finden, wenn wir den Effekt des Artefakts nur für eine gewisse Zeit eindämmen könnten.“ „Das ist es! ZEIT!“, rief Hallgard und riss seine Augen auf. Der Zwergenfürst mahlte mit seinem Unterkiefer und schien mit sich selbst zu hadern. Dann rief er bestimmt: „Alle raus aus diesem Raum! Alle ... außer du, Mart!“ \bigskip

„Mart, höre mir gut zu“, flüsterte Hallgard, sobald alle anderen den Raum verlassen hatten. „Ich werde bald der Ewigen Kälte zum Opfer fallen. Meine Beine sind bereits taub geworden. Doch deine sind es nicht. Nimm diesen Schlüssel und eile zur geheimen Fürstenkammer. Die dort gesammelten Artefakte offenbaren Antworten auf das Leben, das Universum und den ganzen Rest. Ein bestimmtes Artefakt kann uns hoffentlich in der hiesigen Lage weiterhelfen. Reise zur Geheimen Kammer, öffne sie mit meinem Fürstenschlüssel und bringe mir die Sphäre zurück. Eine faustgroße metallene Kugel aus einem Geflecht ineinander verschlungener Streben. Du wirst sie erkennen, wenn du sie siehst.“ 

„Und was soll ich mit dieser Sphäre tun?“, fragte Mart. „Du? Nichts!“, keuchte Hallgard, „Das Wissen um diese Apparatur ist zu gefährlich, als dass es irgendjemandem mehr anvertraut werden sollte, als zwingend nötig. Ich selbst werde die Sphäre handhaben. Bringe sie mir, ehe auch ich dem Eisschlaf anheimfalle!“ \bigskip

Während Mart mit dem Schlüssel zur geheimen Fürstenkammer davonhastete, ließ Hallgard sich auf seinen Thron zurücksinken. Seine Gedanken rasten. Hätte er Mart doch vom Geheimnis der Sphäre erzählten sollen? Hallgard spürte, wie die Ewige Kälte durch sein Fleisch kroch und seine Muskeln verhärtete.

Wohl ahnend, dass er nicht mehr atmen würde, wenn Mart zurückkehrte, stolperte der Fürst zu seinem Schreibpult. Hektisch tauchte er eine silberne Feder in ein Fässchen Geheimtinte, welche in wenigen Tagen verblassen würde. Dann suchte er in seinem Gedächtnis die Instruktionen hervor, die ihm sein Vater Hallwort einst hinterlassen hatte, ehe dieser neugierig in den Norden gereist war. Seinem Tod entgegen.

Hallgard wünschte sich, er hätte damals besser zugehört. Doch so viele andere Gedanken waren zu dieser Zeit durch seinen noch nicht einmal 80 Jahre jungen Kopf geschossen, allen voran Wut auf seinen Vater und dessen Missachtung der Zwergentradition.

Was hatte Hallwort schon wieder gesagt, wie man das Ding aktivierte? Und worauf sollte man unbedingt achten, wenn man es einsetzte? \bigskip

Mart hatte die Hoffnung nie aufgegeben. Die schwach summende silberne Sphäre in seiner Manteltasche versprach Rettung für Cavern und die umliegenden Reiche. Doch die Ansammlung trauernder Zwerge vor dem Thronsaal verriet Mart, dass er nicht schnell genug gewesen war. Auch den weisen Fürsten hatte es erwischt. Man ließ Mart nicht einmal den erstarrten Hallgard sehen. Mit gesenktem Kopf reichte ihm seine Schwester Ragga, eine Wächterin Hallgards, ein versiegeltes Pergament. „Vom Fürsten. Nur für deine Augen bestimmt“, sprach sie. Mit zitternden Fingern entrollte Mart die Botschaft. Zittrige Runen und hastige Skizzen übersäten sie. Instruktionen von Hallgard. In schlichten Lettern stand dort beschrieben, wie man die Streben und Schalter zu betätigen hatte, um die Sphäre zu nutzen und – nein, Hallgard schien Mart immer noch nicht verraten zu wollen, was die Apparatur denn tat, wenn sie aktiviert worden war. Doch hatte er ihm hoffentlich genug mitgeteilt, dass Mart es selbst herausfinden konnte. 

„Das Schicksal liegt in deinen Händen, Mart“, endete die Nachricht des Fürsten. „Ich vertraue auf deine Diskretion. Denn, wenn diese Macht in die falschen Hände gerät, ist noch so viel mehr als nur das jetzige Cavern in Gefahr.“ \bigskip

Ohne zu verstehen, kehrte Mart rasch in seine Wohnkammer zurück, an seinen zwei Eulenküken vorbei, in seinen kleinen Schlafraum. Dort versuchte er im flackernden Licht eines Glutholzes, Hallgards zu folgen. Doch waren sie unvollständig, widersprüchlich, unverständlich. Dazu kam, dass die Zeichen auf dem Dokument stetig verblassten. Wollte er sie abschreiben? Durfte er dies überhaupt?

Mart schob die letzte Querverstrebung runter und legte ein Kontrollzelle in der Mitte der Kugel frei. Doch konnte er nicht mehr entziffern, welchen Hebel er nun umlegen musste. Wagte er es, zu improvisieren? \bigskip

Da klopfte es an der Tür. Herein trat Marts treuer Lebensgefährte Bort, mit einer Tasse warmen Tees in der Hand und bestimmt einigen beruhigenden Worten auf der Zungenspitze. Doch statt diese Worte zu teilen, erstarrte Bort bloß, als er die Sphäre in Marts Hand erblickte.
„Bei Nehals feurigem Kot! Woher hast du die?“

„Aus der geheimen Kammer unseres Fürsten. Weißt du etwa, was sie ist? Was sie kann?“

„Ein wenig. Ich bin ihr bloß einmal begegnet, und das ist lange her.“ Bort starrte die Sphäre ängstlich an, ehe er anhängte: „Es verwundert mich, dass sie ihren Weg zurück nach Cavern fand. Sei vorsichtig mit ihr, Mart.“

„Bin ich doch stets“, versuchte Mart, selbstsicher zu erscheinen.

„Kaulquappenquatsch!“, entgegnete Bort. 

„Hallwort nannte die Sphäre unsere letzte Hoffnung.“ „Dem mag so sein“, murmelte Bort, „Doch habe ich keine guten Erinnerungen an sie. Wir mussten beim feurigen Gott schwören, niemandem von der Wirkungsweise dieser Apparatur zu erzählen, damit die Details nicht in die falschen Hände gelangen konnten.“

„Du musst mir ja nichts erzählen, nur helfen. Erinnerst du dich zufälligerweise noch daran, wie die Sphäre ausgelöst wird? Die letzten Aufzeichnungen unseres Fürsten sind leider bloß lückenhaft – und werden immer blasser.“

Bort kratzte sich am Bart. „Kaum. Ich war noch ein frischgebackener Wächter und löste die Sphäre nie selbst aus. Darf ich?“

Vorsichtig überreichte Mart seinem Partner die verblassenden Pläne. Bort verglich die Verstrebungen der Sphäre mit den Skizzen und bemängelte einige Unterschiede, welche auch Mart schon aufgefallen waren. „Nein, das kommt mir alles unbekannt vor. Warte, ich hole Papier“, meinte Bort, „Besser, wir zeichnen ab, was wir an Anweisungen haben, ehe sie völlig verschwinden.“

Schon war er wieder zum Zimmer hinausgehuscht. Mart blickte angestrengt auf Hallgards Notizen und versuchte sie sich einzuprägen. \bigskip

„Kreatok, steh‘ mir bei!“, erklang Borts fluchende Stimme aus dem nächsten Raum, gefolgt von einem lauten Scheppern. Vermutlich war Bort mal wieder zu hastig gerannt und hatte einen Tisch gestreift. 

„Alles grün?“, erkundigte sich Mart. Keine Antwort ertönte. Das war leicht besorgniserregend.

„Was machst du da?“, erklang eine heisere Stimme direkt hinter Mart, die nicht zu Bort gehörte. Mart fuhr herum und erstarrte. Das war definitiv besorgniserregend. Vor ihm schwebte die bläulich schimmernde Gestalt eines Greises, der griesgrämig auf ihn herunterblickte. Ein Mensch, allem Anschein nach. Oder zumindest war er einst einer gewesen. Sein gebückter Kopf streifte den Türrahmen. Hinter ihm konnte Mart knapp eine am Boden liegende Gestalt ausmachen, die unschön wie Bort aussah. „Nein, nein, hör nicht auf! Das sieht ja höchst interessant aus, was du hier tust“, sprach der Greis. „Ich kann mich nur nicht entscheiden, ob ich dieses Artefakt aufhalten oder studieren sollte.“ 

„Wer ... wer seid Ihr?“, fragte Mart.

„Das ist kaum von Belang. Ich bin hier, um euch zu helfen“, nickte der Greis, „Und diese Sphäre sieht so aus, als könnte sie mir sehr gut helfen. Wenn sie nicht falsch ausgelöst wird. Ich vermute ...
“
Er starrte in die Kugel hinein.

„... ja, ich vermute, du musst nur noch dieses Zahnrad dahinten zwei Speichen weiterdrehen. Dann kommt alles gut.“ 

In Gegenwart dieses eigenartigen Wesens weiter an diesem streng geheimen und höchst gefährlichen Artefakt zu werkeln, erschien Mart alles andere als angebracht. Welche anderen Optionen standen ihm offen? 

„Dann mache ich es halt selbst“, grummelte der Greis. Er langte nach der Sphäre. Da hatte Mart aber genug, sprang selbst vor und schnappte dem Greis die Sphäre weg.


Ein wenig zu energisch. 

Ein lautes Klicken ertönte.

Mart hatte einen Mechanismus ausgelöst.

Die Sphäre begann zu surren und zu dampfen. Blauer Dampf stieß aus einer Öffnung in ihrer Seite, waberte in die Luft und wogte, festigte sich jedoch nicht. Ein Knall ertönte. Alles wurde dunkel. Mart fühlte, wie die Welt sich um ihn herum drehte, ihm die Luft ausging ... \bigskip

\az{Jahr 63}

... und frische Luft in seine Lungen strömte, während er nach Atem japste. Er blickte sich ängstlich um. Er lag auf dem Boden seiner Wohnhöhle, doch etwas war falsch. Der flauschige Teppich, auf dem Marts Gesicht ruhte, den sollte es eigentlich nicht mehr geben. Der war bei einem feurigen Unfall vor einigen Wochen verbrannt, und Mart hatte sich bislang nicht um einen Ersatz gekümmert. 

Vorsichtig erhob sich Mart und blickte sich in seiner Wohnhöhle um. Da war noch mehr, das nicht stimmte. Sein Schaukelstuhl und sein Ecktisch standen beide am falschen Ort, noch wie von vor seiner kürzlichen großen Höhlenumstellungsaktion. Der mächtige rote Edelstein, den Mart unlängst erworben hatte, hing nicht an der Wand. Bort war nirgends mehr zu sehen. Dafür war da immer noch dieser bläulich glühende Greis, der auf ihn herunterstarrte.

„Was hast du getan?!“, rief der Greis. Wütend grabschte er die Sphäre aus Marts Griff und starrte sie an. Das Ding stotterte und zuckte vor sich hin und reagierte nicht auf die Versuche des Greises, etwas an ihr einzustellen. Sein Mund verzog sich und er brüllte erneut: „Was hast du getan?! Die Schatten der Gegenwart und der Zukunft ... sie sind alle falsch! Verzerrt! Verändert!“

Mart wusste selbst nicht, was genau er getan hatte. Musste er auch nicht. Denn in diesem Augenblick beruhigte sich die Gestalt schlagartig. Breit grinsend murmelte sie: „Ah, natürlich! Ich machte mir umsonst Sorgen. Wie konnte ich bloß so blind sein? Wie konnte ich das nur vergessen?“ Ohne weitere Erklärungen breitete der Greis gebieterisch seine Arme aus und rief einige finstere Worte in einer fremden Sprache. Dunkelheit überkam Mart und er sank in einen tiefen, traumlosen Schlaf. \bigskip

Ja, Mart war wahrlich durch diese mysteriöse Sphäre in die Vergangenheit gereist, in einen Herbst, in dem noch niemand etwas von der Ewigen Kälte ahnte. Und die stotternde Sphäre sorgte nicht nur beim glühenden Greis für Sorgen. Schwirrende, instabile bläuliche Portale öffneten sich spontan an verschiedenen Stellen im Lande und verbanden die zwei Zeiten: Das herbstliche Andor, in das Mart und der Greis gereist waren, dessen Bewohner noch nichts von der ewigen Kälte ahnen, und das winterliche Andor, aus dem Mart und der Greis gereist waren, wo einige letzte Helden die Rietburg tapfer verteidigen. Nachdem Mart wieder erwacht war, suchte er zögerlich die herbstlichen Helden und berichtete ihnen: 

Ein gruseliger, bläulicher Greis habe ihn überwältigt und ein wertvolles Artefakt der Schildzwerge gestohlen! Es war nicht auszuschließen, dass dieser Vorfall für diese eigenartigen Zeitportale verantwortlich war, welche durch Andor waberten. Obwohl Mart sich geheimnisvoll gab und keine weiteren Details zu diesem gestohlenen Artefakt herausrückte, war klar: Die Helden wollten ihm helfen und den Greis aufspüren. 

\az{63 und 64}

So kam es, dass, während die meisten noch nicht eisschlafenden Helden im Land der drei Brüder den Dunklen Magier Orweyn suchten, von Orweyn prophezeite Bedrohungen überwanden und Wintersteine verwandelten – ein Abenteuer, das später als Legende der Pranken des Winters in die Chroniken der Bewahrer vom Baum der Lieder eingehen würde – auch in Andor nicht eisschlafende Helden den glühenden Greis jagten, und von ihm nach Andor geschickte Bedrohungen bannten. Dieses Abenteuer würde eines Tages als Legende der entzweiten Zeit in die Chroniken der Bewahrer vom Baum der Lieder eingehen. 

Mart würde später erfahren, wie sich Herbst-Helden aus der Vergangenheit mit Winter-Helden aus der Zukunft zusammentaten. Wie die Helden durch fünf in der Vergangenheit am Krallenfelsen versteckte und in der Zukunft am Krallenfelsen gefundene Goldstücke unmissverständlich klarstellten, dass diese Zeiten zusammenhingen. 

Diese Portale verbanden keine zwei Welten, sondern Vergangenheit und Zukunft derselben Zeitlinie, die sich aus irgendeinem Grund entzweit hatte. Zwei Zeiten, die sich parallel entwickelten, und von denen die eine dann doch die andere werden würde. Ein vorgeschriebenes Schicksal, scheinbar in Stein gemeißelt.

Für viel Verwirrung sorgte dies. Einzig die Kräuterhexe Reka hatte – dank einer lange zurückliegenden Beratung mit dem tulgorischen Hüter der Zeit – eine Ahnung, was es hiermit auf sich hatte. Die Helden erreichten den winterlichen glühenden Greis gerade rechtzeitig, um zu verhindern, dass er mit dem gestohlenen Winterstein von Cavern außer Landes fliehen konnte, und um zu verhindern, dass er die Nachricht über das Verwandeln des Wintersteins im Land der drei Brüder an sein früheres Selbst weitergeben konnte.\bigskip

Der Greis stürzte zu Boden. „Ich wusste, dass ihr mich aufhalten werdet. Eigenartig, diese Zeitgefüge. Wie kann es sein, dass ich mein früheres Selbst nicht warnen kann, obwohl ich weiß, was euch gleich erwartet?” Er hob seine Hände, blaue Stränge der Magie formend, bereit, erneut davonzuteleportieren ... und sackte bewusstlos zusammen, als ihn ein heldenhafter Schlag auf den Hinterkopf traf.\bigskip

Die Helden brachten die Gestalt zum Kerker unter der verschneiten Rietburg. Eine dortige Zelle hielt eine waschechte Schattenhexe gefangen, da würde doch hoffentlich auch ein teleportierender Greis nicht auszubrechen vermögen.\bigskip

Die Helden waren frohen Mutes, dass der ältere Tranuk und seine neuerdings schreckliche Steppenechse den Winterstein sicher ins Land der drei Brüder zurückbringen würden. Nun war es an der Zeit zu feiern. \bigskip

Ehe einige von ihnen in ihre eigene Zeit zurückkehren würden, versammelten sich alle beteiligten Helden, der jüngere Tranuk und einige andere Eingeweihte ein letztes Mal in der Vergangenheit, bei der herbstlichen Taverne zum Trunkenen Troll. Sie stießen an auf ihren Sieg über die Ewige Kälte, an einem Tisch ein wenig abseits des Hauses, um kein Misstrauen vergangener Tavernengäste zu wecken. Die beiden Rekas unterhielten sich prächtig und ließen sich zu einer Honigmilch überreden, auch wenn die ältere seltsam besorgt aussah. 

Ein Held bekam feuchte Augen, als er die jüngere Gilda in der Ferne eine ihrer Balladen schmettern sah. Bald würde die ältere aus dem Eisschlaf erwachen und wieder solche Lieder zum Besten geben können. 

Manch einem schmerzte der Kopf darüber, was sie soeben erlebt hatten. Oder war es Gildas Met, der ihnen zu Kopfe stieg? Hmmm ... schmeckte dieser Met nicht irgendwie bitterer als sonst?

Einer nach dem anderen kippten die Trinkenden um, als sie ihre Kräfte verließen. Manche rutschten gar von ihren Stühlen. 

Mit einem Knall erschien die jüngere glühende Gestalt triumphierend auf dem Tisch. „Das habt ihr davon, mich aus meinem Versteck zu verjagen versuchen! Ob ihr wohl ahntet, was ich vorhatte? Doch fürchtet euch nicht, ich werde euch nicht schaden. Ich bringe die Zeitverirrten unter euch sogar in ihre eigene Zeit zurück. Ihr Helden müsst es schließlich noch mit Varkur aufnehmen.“ „Wasss?“, nuschelte einer der Helden. 

Der jüngere Greis schnaubte. „Ich habe es vorausgesehen. Ihr werdet Qurun überwinden, das drohende Ende Hadrias, Ihr werdet die zankenden Orden vereinen. Ihr tut Gutes. Und so sollt ihr weiter leben und fröhlich walten. Dennoch will ich euch natürlich nicht die Gelegenheit geben, mich aufzuhalten. Darum dürft ihr euch nicht daran erinnern, was in den letzten Tagen geschah.“ 

„Wiiie?“, murmelte der Held.

„So!“, grinste der Greis. Er zückte eine Phiole, in der nur noch ein kleiner Rest eines leuchtenden Trunks umherschwappte. „Eine Tinktur gegen Seelenschmerz. In geringen Dosen kann sie schmerzhafte Erinnerungen erträglich machen oder gar ganz löschen. Doch vermag sie vieles mehr. Wenn ich sie richtig dosiert habe, werdet ihr gerade schön die Ereignisse der letzten Tage vergessen. Damit ich beruhigt das Eintreffen der Ewigen Kälte vorbereiten kann. Welches, wie ich weiß, ein gutes Ende für mich nehmen wird. Und ihr ... ihr werdet wohl in die sichere Ferne fliehen? Ich empfehle Hadria.“

Ein diabolisches Gekicher entfloh dem Greis.

Dann umfing die Helden der Mantel der Ungewissheit. \bigskip



\az{Jahr 64}

Auf dem Weg ins Land der drei Brüder empfing Tranuk im Traum eine seltsam klare Vision von einem geheimnisvollen Eiswesen, welches den Winterstein hungrig beobachtete. Am nächsten Tag fühlte Tranuk sich stets aus eisigen Augen beobachtet. Manchmal glaubte er, im Schneegestöber vor sich eine geisterhafte Gestalt mit ausgebreiteten Armen zu erkennen. Seit seiner Verwandlung fürchtete Tranuk sich vor keiner Auseinandersetzung mit dunklen Kreaturen oder uralten Zauberern, doch hatte er gehörig Respekt vor Geistern des Eises. Mit den verirrten Seelen im Schnee Gestorbener war nicht zu spaßen. Darum gebot er seiner gehörnten Steppenechse, rasch zurück ins sichere Cavern zurückzueilen, wo er seine wertvolle Fracht wieder in die Obhut der nun umso wachsameren Schildzwerge gab. So erwartete Tranuk die Rückkehr der restlichen Helden aus dem Land der drei Brüder. Erst, nachdem jene das geheimnisvolle Eiswesen unschädlich gemacht hätten, würde er sich wieder auf den Weg zur Höhle der Verwandlung machen. Dieses Abenteuer würde eines Tages als Legenden vom Herz aus Eis in die Chroniken der Bewahrer vom Baum der Lieder eingehen.\bigskip

Als die Helden wieder erwachten, befanden sich alle wieder zurück in ihrer eigenen Zeit. Es war ihnen, als hätten sie einen langen Schlaf mit wilden Träumen hinter sich gehabt. Und sie teilten nur noch unwirkliche, unzusammenhängende und uneinordbare Erinnerungen an die wilden Ereignisse der letzten Tage. 

Zunächst ziemlich verwirrt, kamen die Helden nach Augenzeugenberichten von aus der Ferne zugeschaut habender Andori zum Schluss, dass sie einen weiteren gefährlichen Gegner Andors erfolgreich besiegt hatten. Ein glühendes Wesen, welches durchs Land gehuscht war und seltsame Portale geöffnet hatte. Doch diese Portale waren nun geschlossen und die Gestalt vertrieben, auch wenn sie im finalen Kampf offenbar die Geister der Helden benebelt und ihre Erinnerungen verschleiert hatte. 

Die Helden in der Zukunft wurden zudem darüber informiert, dass sie die Gestalt gefangen und eingesperrt hatten. Leider hatte diese sich aus der Gefängniszelle unter der Rietburg wegteleportieren können, ehe sie die Helden hätte über ihre Motive informieren können. Doch viel wichtiger waren die Taten der restlichen Helden im Steppenland gewesen. Vögel zwitscherten. Die Eisschlafenden erwachten nach und nach: Gilda, Thoralds Trupp, Chada, Hallgard, Melkart und noch so viele mehr ... selbst der alte Troll neben dem alten Wehrturm! Die ewige Kälte lichtete sich. Der Frühling nahte. Die Legende hatte ein gutes Ende genommen. Alles war gut. 

\az{Jahr 63}

Orweyns glühende Gestalt stand in der dunklen Nacht neben der Taverne zum Trunkenen Troll. Sein langer Mantel flatterte im Herbstwind. Der rote Mond beschien seine finstere Taten. Er rieb sich seine schmutzigen Hände. Die zukünftigen Helden waren in ihre Zeit zurückgebracht und die vergangenen hatten immerhin ihre Erinnerung verloren. Und nun war sein wichtigstes Werk vollbracht. Unter ihm, tief vergraben im Schatten der andorischen Erde, ruhte ein unscheinbarer, bläulich schimmernder Quader, übersät von magischen Runen, der leise vor sich hin summte. 

Ein Winterstein. Orweyn hatte ihn hierher befördert, wie von seinem älteren Selbst aufgetragen.

Die Geister der andorischen Erde würden sich gegen den Eindringling in ihr Reich wehren. Doch selbst der große Skuvar würde Monate brauchen, um das magische Artefakt an die Oberfläche zu befördern, ohne selbst zu Eis zu erstarren. Und bis dahin würde sich die Ewige Kälte längst über Andor ausgebreitet haben. 

Der mächtige Zauberer vergoss eine Träne an all die Unschuldigen, die dieses Vorhaben kosten würde. Doch das war es wert, erinnerte er sich selbst. Der schlimme Tarok, die verwandelten drei Brüder, der wütende Enkel seines einstigen Freunds ... so viele lebensbedrohliche Gefahren lauerten in der Zukunft der bekannten Welt. Und auch wenn er es gerne anders hätte, sahen die Chancen nicht gut aus, dass die Helden sie alle überwinden konnten. Er musste realistisch sein. Drastische Maßnahmen waren gefordert. Sein Werk würde die Gefüge der Welt richten, wie es einst sein eigenes Opfer in Hadria getan hatte. Naja, sein eigenes Leben hatte er damals nicht geopfert, als er den Eisernen Turm hatte erbauen lassen. Das wäre ja auch unnötig gewesen. Nur die Leben der anderen Magier, denen er nicht vertrauen konnte, dass sie auch wirklich die Dunkle Magie hinter sich lassen würden. Und zumindest seine eigenen Erinnerungen an die Dunkle Magie würde er eines Tages opfern, wenn sein Ziel erreicht war und Frieden über dieser Welt lag. Nützliches kleines Tränklein, diese Tinktur gegen Seelenschmerz. 

Doch bis dahin gab es noch viel zu tun.

Orweyn würde nicht ruhen, bis die Dunkle Magie und ihre Bosheit endgültig aus dieser Realität getilgt waren. 

Vielleicht könnte er zurückkehren, wenn alles getan war, wenn Andor und das Land der drei Brüder unter meterdickem Schnee ruhten. Er könnte zur Knochengrube aufbrechen, Taroks dem Eisschlaf anheimgefallenen Körper finden und ihn durch gezielte Schwertstöße vernichten. Vielleicht auch noch die Hexe Reka und ähnlich gefährliche magische Zeitgenossen aus dem Wege befördern. Und danach die Wintersteine dorthin zurückzubringen, woher sie einst gekommen waren. Vielleicht würde der Schnee sich wieder lichten. Vielleicht würden einige wieder erwachen. Vielleicht. 

Er konnte hoffen.

Doch seine Taten waren nötig.

Niemand verstand das.

Niemand außer ihm.

Er war alleine in dieser kalten Welt.

Ein Retter, den keiner als Retter erkannte.

So wollte es das Schicksal von allen großen Rettern. 

Über dem Greis begann es zu schneien.

Die Ewige Kälte hatte begonnen.\bigskip

\az{63 und 64}

Doch waren damit wirklich alle losen Enden geschlossen? Wie war es eigentlich dazu gekommen, dass sich die Zeitportale gerade im passenden Moment wieder verschlossen hatten? Nachdem der jüngere Orweyn die zeitreisenden Helden in ihre eigene Zeit zurückgebracht hatte, aber bevor der ältere Orweyn in seiner Zelle aufgewacht war – und prompt mit einem Knall davongesprungen, um den Helden im Land der drei Brüder eine Standpauke über die Gefahren der Zukunft zu halten, mannigfaltig und grausam, wie er sie vorausgesehen hatte? Ein wenig früher, und manche Helden wären in der falschen Zeit gefangen gewesen. Ein bisschen später, und der ältere Orweyn hätte den Winterstein an sich gerissen. \bigskip

Nun, fürs Schließen der Zeitportale war ich verantwortlich. Nicht lange nach dem Endkampf tauchte ich neben dem Schlachtfeld auf. Dort sicherte ich die hadrische Tasche, die Orweyn in der Hitze des Gefechts verloren hatte und um die sich kein Held geschert hatte. Der Großteil des Inhalts war mir egal, den ließ ich liegen. Alte Bücher, wirre Worte auf brüchigem Papier, seltene Steine und magische Pülverchen aller Sorten. Orweyn würde sie später wieder an sich nehmen. Mir ging es nur um etwas: Die Sphäre aus Cavern, die das ganze Chaos hier ausgelöst hatte. Sie stotterte weiterhin wild vor sich hin und stieß blauen Zeitdampf aus. Mein Werk, mein armes Werk. Was hatten sie ihm nur angetan?! 

Mit geschickten Handbewegungen richtete ich verbogene verborgene Hebel.
 
Die Sphäre beruhigte sich. Die Zeitportale, welche wild über Andor gewirbelt hatten, schlossen sich. Die Zeit war nicht mehr entzweit.

Ich stellte sicher, dass 5 Gold und eine hilfreiche Nachricht beim Krallenfelsen lagen. Danach gab es nur noch etwas zu erledigen. Ich reiste ins herbstliche Cavern, stapfte verborgen durch enge Gänge und fand an der Stelle, die mein zukünftiges Selbst mir verraten hatte, einen zitternden Zwerg. Mart hieß er, und er war als einziger in der falschen Zeit verblieben. Argwöhnisch starrte er mich an. „Alles wird gut“, versprach ich, „Ich bin hier, um alles in Ordnung zu bringen.“ 

Ich zeigte Mart seine Sphäre, die ich repariert hatte und die ich nun in die geheime Schatzkammer zurückbringen würde. Und ich zeigte ihm meine zweite Sphäre, mit der ich danach in meine eigene Zeit zurückkehren würde. Dieselbe Sphäre, nur einige Jahrhunderte jünger als die andere. Mart machte große Augen. Ehrfürchtig fiel er auf ein Knie. Vermutlich hatte er mein Antlitz von den vielen Wandteppichen aus der Urzeit wiedererkannt. Leise flüsterte er: „Ihr kommt aus der Hochblüte des Zwergenreichs! Ihr seid ein Zeitreisender! Habt ihr die Zukunft gesehen? Werden wir die Ewige Kälte überleben?“ 

Ich lachte, zog ihn hoch und legte verschwörerisch einen Finger an meine Lippen. „Wer weiß, was die ferne Zukunft bringt. Schließlich kann ich so weit reisen, wie ich will, und ihr Horizont liegt weiterhin in unerreichbarer Ferne. Ich verirrte mich einst kurz in einer weit, weit entfernten Zeit, die jedem hierzulande unvorstellbar erschiene. Doch zumindest für die nächsten paar Jahrhunderte wird die Welt, so wie du sie kennst, florieren dürfen. Dafür sind wir Hüter der Zeit ja da.“ 

„Ihr Hüter der Zeit? Es gibt noch mehr?“

An einer Hand zählte ich die auf, die mir in den Sinn kamen: „Oh, es gibt so viele Hüter der Zeit! Ein tulgorischer, der sich darauf spezialisiert, den richtigen Leuten die richtigen Informationen zu geben, ehe er sie selbst vergisst. Ein Agren, der weiß, dass seine Höhlenmalereien Jahrhunderte später den Lauf der Dinge zu verändern vermögen, auch wenn seine Worte dann längst vergessen sind. Eine Drächin, die ihre eigene Urururgroßmutter ist. Ein grantiger Seher aus Danwar, der vor lauter Wald die einzelnen Bäume nicht mehr sieht. Ein zweiäugiger Zukunftswisser aus dem Süden, der vom ganzen metaphorischen Wald nur einen einzelnen Baum pflegt. Sie alle haben ihre Methoden. Oder hatten. Keine Ahnung, welche jetzt gerade überhaupt noch leben. Ich persönlich lasse sie alle ihre Spielchen spielen, davon verstehe ich nicht genug. Ich kümmere mich nur darum, dass meine Werke bis zu ihrer Zerstörung nicht zu lange in den falschen Händen bleiben. Das schulde ich der Welt dafür, die Sphäre geschaffen zu haben.“ 

Mart schienen tausend Fragen auf der Zunge zu liegen. Er entschied sich für eine bestimmte: „Wenn nicht Ihr ... gibt es denn irgendjemanden sonst, der den Überblick hat?“ „Wer kann das schon wissen? Mutter Natur, die Schicksalssporne, Mächte jenseits unserer Vorstellung? Falls es sie gibt, haben sie sich noch nicht bei mir gemeldet. Aber falls sie wirklich einen solchen Durchblick hätten, müssten sie sich kaum direkt bei mir melden. Wir können nur hoffen, dass sie unsere Werte teilen.“ Ich setzte mein liebstes bevormundendes Lächeln auf. „Die wichtigsten Hüter der Zeitlinie sind natürlich ihre Bewohner, die sich tagein, tagaus mit Elan für das Gemeinwohl einsetzen: Bauern und Bewahrer. Heiler und Helden. Fackelträger aus den Tiefminen und Fürsten von Cavern gleichermaßen. Du. Die Deinen. Ihr Werk können wir tagein, tagaus spüren und bejubeln. Doch hin und wieder tauchen Gefahren auf, deren Ausmaß diese Hüter kaum einschätzen können. Jahrhunderte lang geplante Pläne von finsteren Sehern und erbarmungslosen Mächten. Perfide Prophezeiungen und stibitzte Sphären. Dann gibt es oft lose Enden zu schließen. Irgendjemand muss schließlich dafür sorgen, dass die Zeitlinie konsistent und gut ist. Ich habe mich entschieden, meinen Teil zu tun.“ 

Mart schien sich nicht sicher zu sein, ob er erleichtert oder angegruselt sein sollte.

Behutsam führte ich ihn zurück zu seinem Heim in seiner eigenen Zeit, in jener der Schnee der Ewigen Kälte langsam zu schmelzen begonnen hatte. Mart lud mich auf eine Tasse heißen Cocoa hinein, doch ich lehnte höflich ab. Ich wollte nicht zu viel Zeit mit Plaudereien verbringen, erst recht nicht so weit abseits meiner eigenen Zeit. So blieb ich zurück, als Mart die Tür zu seiner Höhle öffnete. Ich vernahm noch, wie Marts Lebenspartner Bort voller Freude und Sorge zugleich aufjubelte, als er Mart durch die Pforte treten sah: „Bei Boords Bart, ich hatte mir schon solche Sorgen gemacht! Glaube mir, ich war so kurz davor, mit deiner Schwester Ragga aufzubrechen, diesen glühenden Greis eigenhändig aufzuspüren und in die Eisenketten meines Vetters zu legen! Wenn Drak mich nicht beruhigt hätten ... Oh, welch Horror du erlebt haben musst! Wie geht es dir?“ 

Leise flüsterte Mart eine beruhigende Entgegnung.

Und ich spionierte den beiden nicht länger nach, sondern öffnete ein Portal viele Jahrhunderte in die Zukunft, wo der nächste Auftrag meines zukünftigen Selbsts – und eine Bewahrerin namens Josella – bereits auf mich warteten. 



 


















\begin{chapterbox}
    \chapter{Runen im Schnee (2022)}
    \label{Runen im Schnee (2022)}
    \az{Jahr 65}

    \begin{center}
        Teil lll der Magischen Abenteuer
        
        Fortsetzung von \hypref{Der verschwundene Feuertakuri (2022)}
    \end{center}
    
    Eine Runenmeisterin aus Silberhall reist nach dem Tod ihrer Lehrmeisterin zurück in ihre ehemalige Heimat Cavern. Ein Eis-Dämon verlässt nach der Öffnung des Felsentors das ewige Eis. Eine Takuri-Hüterin und ein Steppennomade begleiten eine tulgorische Reisetruppe bei der Unterquerung des gefährlichen Kuolema-Gebirges. Und ein Herold, der unlängst den Tod seines Meisters zu betrauern hatte, findet in einem uralten Kult Verbündete.
\end{chapterbox}







\section{Epilog}

\az{Jahr 563}

Die Bewahrer vom Baum der Lieder schrieben das Jahr 563 nach andorischer Zeitrechnung.

Die Rietburg und die umliegenden Dörfer waren zur großen Rietstadt zusammengewachsen.

Die Ewige Kälte Hadrias war grünem Gras und sprießenden Blumen gewichen.

Das Graue Gebirge war einige Meter höher als zur Zeit der ersten Helden.

Der große See Ava hatte sich beinahe bis zum Hadrischen Meer ausgebreitet.

Die Rote Steppe Tulgors war größtenteils von der Wilden Wüste des Westens eingenommen worden. Nur ganz nahe am Kuolema-Gebirge wuchs das wilde Steppengras noch ungehindert.

Eine einsame Gestalt lief durch diese tulgorische Steppe, eine Spur aus Eis und Schnee hinter sich herziehend. Ein langer Bart fiel auf eine nackte Brust. Bis auf Stiefel, einen Rock und einen Gürtel mit einigen daran befestigten Artefakten war die Person unbekleidet. Ihre Kleidung war so blauweiß wie ihre Haut und ihr langes Haar. So blauweiß wie das Eis.

Dies war ein Eis-Dämon. Lange schon hatte man in dieser Gegend keinen mehr gesehen. Sein Name war Ijsdur. Schon einige Male war er gerufen worden. „Hilf uns, o Dämon des ewigen Eises. Komm herunter von deinem hohen Berg. Rette das Leben von diesen und jenen. Schenke uns Armen mehr Zeit in dieser unfairen Welt.“

Dabei besaß Ijsdur gar nicht die Fähigkeit, weitere lebensrettende Eiskristallketten zu schaffen. Er wusste nicht, wie er neue Durs und Doras erschaffen sollte, ohne selbst daran zugrunde zu gehen.

Diesmal war er dennoch gekommen. Denn es war an der Zeit. Seine fünf Jahrhunderte waren beinahe um.

Ijsdur besuchte ein Dorf am Fuße des Kuolema-Gebirges. Er fand die gesuchte Hütte, trat durch eine Tür und dann noch eine, und dann erblickte er sie. Eine gebrechliche, todkranke junge Tulgori, die in ihrem Bettchen lag und müde vor sich hin hustete. Ihre Familie saß ums Bett herum und schien nicht wirklich zu wissen, wie sie reagieren sollte. Manche gruselten sich vor dem Eis-Dämon. Ein Vater wich Ijsdurs Blick gekonnt aus. Ein anderer weinte ungehemmt. Ein dritter lächelte. Dieser wirkte gar ... hoffnungsvoll?

Sie waren nicht wichtig. Wichtig war nur die kleine Tulgori, die erneut schwach hustete. Nalle hieß sie. Schwach leuchtende Flecken überzogen ihre dunkle Haut. Die Druiden hatten ihr gesagt, dass sie kaum mehr als ein, zwei Wochen zu leben hätte. Der Blick aus Nalles Augen war wachsam und berechnend, als sie Ijsdur musterte.

„Danke“, sprach sie heiser.

„Danke mir lieber noch nicht“, meinte Ijsdur grimmig, ehe er sich eines Besseren besann und ein gezwungenes Lächeln aufsetze.

Nalle verabschiedete sich von allen Anwesenden. Noch mehr Tränen brachen aus. Ijsdur verzog keine Miene und wandte sich höflich ab. Er streckte seine Arme aus und ließ sich eine Decke darüberlegen, damit die Kleine auf dem Weg nicht allzu sehr frieren würde. Dann ergriff er Nalle mit beiden Armen. Sie war zu schwach, um allein zu laufen.

Langsam trug Ijsdur sie die Hänge des Kuolema-Gebirges hoch. Ihr Pfad schlängelte sich durch ein Meer aus Steinen den Berg hinauf.

In der Ferne sah Ijsdur einen brennenden Takuri über den Himmel fliegen, einen Flammenschweif hinter sich herziehend. Ijsdur drehte sich so, dass Nalle den Feuervogel sehen konnte.

„Hast du schon je einen Feuertakuri von nahe gesehen?“

Nalle schüttelte ihren Kopf und Schneeflocken von ihren Haaren. „Ich habe entfernte Familie unter den Hütern. Aber die trauen sich nur noch ohne Feuervögel in die Nähe des trockenen Steppengrases.“

„Hm. Ich habe schon einige gesehen. Sogar auf dem Weg hierher. Ich besuchte den Nestbaum der Feuertakuri in den westlichen Ausläufern des Kuolema-Gebirges, obwohl die mich dort nicht wirklich mögen. Beruht auch auf Gegenseitigkeit. Die Takuri sind keine Freunde des Eises. Und wir Eis-Dämonen sind keine Freunde von Feuervögeln. Einen gab es mal, den ich mochte. Turr. Nun, eigentlich gibt es ihn immer noch. Er trägt inzwischen einen neuen Namen. Und er erkennt mich nicht mehr. Aber es ist immer noch derselbe Vogel. Irgendwie. Faszinierend. Ich wollte mich von ihm verabschieden.“

Seine Stimme stockte, während er gedankenverloren mit einem schwarzen Pulversack an seinem Gürtel spielte, in welchem etwas knisterte. Warum redete er überhaupt? Und warum so viel? Wollte er seine letzten Stunden des Lebens noch auskosten?

Nachdenklich blickte er in die Ferne. Die rote Sonne ging weit im Westen hinter der endlos scheinenden Steppe unter.

Es wurde dunkel.

„Angst“, flüsterte Nalle leise.

„Oh. Das ist natürlich. Das ist verständlich. Das ist vollkommen in Ordnung“, sagte Ijsdur. „Bald wird sich dein Leben für immer ändern. Aber du wirst nicht mehr leiden. All diese hässlichen Gefühle, die in deinem Kopf herumschwirren, werden sich lindern. Du wirst freier denken können. Du wirst es genießen.“

Nalle grinste schwach: „Du sprichst lustig. Deine Betonung. So altmodisch.“

Dann wurde sie wieder ernst: „Wie ist es? Wie ist es, ein Eis-Dämon zu sein?“

„Willst du es bereits fühlen?“

Ijsdur setzte Nalle sanft ab und zeigte ihr die Kette magischer Kristalle, welche um seinem Hals hing und fest mit seiner Brust verschmolzen war.

„Wenn du willst, kannst du die Eiskristalle anfassen. Sie werden deine Furcht, Sorgen und Schmerzen etwas lindern, wenn das für dich angenehmer ist. Dann weißt du, wie es sich anfühlen wird.“

Nalle überlegte kurz und streckte ihre Hand aus: „Werden die Kristalle auch meine Freude nehmen? Meine Faszination?“

„Sie werden alle gedämpft werden, deine Emotionen“, gab Ijsdur zu, „Doch deine Interessen werden bleiben. Du wirst keine gefühlslose Maschine werden, nur ... kälter.“

„Emotionen sind wichtig“, sprach Nalle fest, „Sie drängen uns zu großen Taten voran und geben uns Kraft.“

Vermutlich hatte ihre Familie ihr dies noch vor Kurzem vorgepredigt. Ijsdur fühlte sich nicht kompetent genug, um sich über den Sinn oder Unsinn von Gefühlen zu unterhalten. Dennoch dachte er laut nach: „Sie mögen uns zu großen Taten antreiben, aber manchmal verschleiern sie auch unsere Sinne und hindern uns daran, unser Potential auszuschöpfen. Wir erkennen sie in anderen Wesen wieder. Und wenn sie gedämpft sind, glauben viele fälschlicherweise, einen nicht mehr zu erkennen. Aber ...“

„Danke für das Angebot, Ijsdur. Doch ich würde lieber noch warten mit der Kette.“

„Das ist auch vollkommen in Ordnung.“

Ijsdur hob Nalle wieder in die Höhe und transportierte sie weiter den Berg hinauf. Sie durchquerten das Felsentor, welches Jahrtausende lang die Eis-Dämonen in ihrem Tal festgehalten hatte. Inzwischen war es schon beinahe ein halbes Jahrtausend lang wieder offen. In dieser Zeit hatte sich das ewige Eis, diese legendäre Eisfläche, aus dem Tal nicht nur bis zum Felsentor hin, sondern auch über das Tor hinweg und ein kleines Stück den Berghang hinunter ausgebreitet.

Ijsdur wollte mehr, als nur das ewige Eis zu berühren. Er wollte nicht riskieren, dass die Weitergabe der Eiskristallkette versagte. Er wollte ins Zentrum des Tals, wo die Kraft des Eises am größten war.

Lange schritten er und Nalle über die Eisfläche zum Mittelpunkt des ewigen Eises. Sie passierten eingeschneite Eissäulen in verschiedensten Formen. Manche sahen gar menschlich aus. Doch gab es kein Leben in ihnen. Sie waren das Werk von vergangenen Eis-Dämonen, welche lange im Ewigen Eis verharrt hatten und verrottet waren. Das Eis hatte sich ihrem stetig schwächer werdenden Willen gebeugt und diese Formen angenommen. Nun zeugten sie nur noch vom wirren Wirken derjenigen, die schon längst vergangen waren. Ijsdur schenkte ihnen keine Aufmerksamkeit. Nalle hingegen drückte sich ängstlich tiefer in Ijsdurs schützende Arme.

Trotz der Decke begann sie zu zittern. Das war eigentlich auch gut so. Schließlich musste ihr todkranker Körper erfrieren, damit sie sich bald als unversehrte Eis-Dämonin wieder erheben konnte. Doch fühlte Ijsdur mit ihr und ihrem Leiden in der klirrenden Kälte. Er bot ihr noch einmal an, dass sie seine Eiskristallkette bereits berühren dürfe, um weniger leiden zu müssen.

Nalle blieb eine Zeit lang stumm. Dann fragte sie stattdessen leise: „Magst du... magst du mir eine Geschichte erzählen?“

Ijsdur blieb überrascht stehen.

„Was für eine Geschichte?“

„Irgendeine. Du bist so alt, du musst doch Unmengen kennen.“

„Dem sollte so sein“, sinnierte Ijsdur, „Ich habe unzählige Reiche bereist. Erlebt, wie Kinder und Kindeskinder von Freunden ihre Plätze einnahmen oder ihre eigenen Wege gingen. Ein ganzer Hort von Geschenken und geschichtsträchtigen Gegenständen lagert auf dem höchsten Gipfel des Kuolema-Gebirges, wo ich mir einst eine letzte Behausung suchte, nachdem die Welt sich zu schnell weiterdrehte für mein uraltes Ich. Doch meine Erinnerungen an all die vergangenen Geschichten sind nicht mehr, was sie einst waren. Lass mich meine Gedanken sammeln und eine Erzählung finden, an der du Freude finden könntest.“

„Was ist mit Choranat? Ich habe gehört ...“

„Nicht Choranat“, sprach Ijsdur bestimmt, „Die Geschichte von Choranat ist nichts für Kinderohren. Dieser Bannkreis hat genug Schaden angerichtet.“

Nalle riss ihre Augen interessiert auf: „Oooh, ist das ein Geheimnis? Ich liebe Geheimnisse.“

Ijsdur stolperte. Einer seiner Mundwinkel zuckte. Die erste wirkliche Gesichtsregung, die Nalle in seiner sonst steinernen Miene erkennen konnte. „Du erinnerst mich an jemanden, der ich vor einer langen Zeit kannte.“

Ijsdurs Hand glitt an seinen Gürtel. Er trug nicht viel bei sich – musste er doch im ewigen Eis kaum Nahrung zu sich nehmen und sich auch nicht mit Kleidung vor der Witterung schützen – doch dieser von Runenmustern übersäte Hammer, der an seiner Hüfte hing, war schon seit mehreren Jahrhunderten sein stummer Begleiter.

„Magst du mir von ihm erzählen, an den ich dich erinnere?“, fragte Nalle neugierig.

„Von ihr. Es war eine sie.“

Ijsdur seufzte und begann, zu erzählen. Seine sonst tonlose Stimme wurde weich. Langsam tröpfelten die Worte aus seinem Mund, dann immer schneller, bis er wie ein Wasserfall sprudelte. Er war ungeübt darin, Geschichten zu erzählen. Doch hatte er es einst gerne getan. Er konnte es wieder probieren.

„Diese Geschichte spielt zu einer Zeit, als ich noch ein ganz frischer Eis-Dämon war. Kaum zwei Jahre lang war ich ziellos übers ewige Eis gewandert und hatte mich an meine Existenz gewöhnt. Dann, eines Tages, wurde das uralte Felsentor geöffnet, das das ewige Eis seit Jahrtausenden in unserem schattigen Tal hoch oben im Kuolema-Gebirge festgehalten hatte. Wir Eis-Dämonen konnten erstmals wieder in die weite Welt hinaus reisen. Siantari, die Herrin des ewigen Eises, die uns geschaffen hatte, erlöste uns und ließ uns frei sein. Vergleichsweise frei.

Doch nur wenige Eis-Dämonen waren noch genug bei Verstand nach ihren hunderten von Jahren im Schnee und Eis. Ich glaube, ich war der erste, der die Berghänge heruntertorkelte. Vielleicht auch der einzige.

Jedenfalls stolperte ich, Schnee und Eis hinter mir herziehend, nach Tulgor. Ins Dorf, das einst Ijs‘ Heimat gewesen war.“

Ijsdur stockte kurz und überlegte, wie viel er Nalle zumuten konnte. Er entschloss sich für eine Beschönigung der Geschehnisse. Er musste sie ja nicht anlügen. Nur nicht die ganze Geschichte erzählen.

„Mein Vater – mein leiblicher Vater Saro, nicht der Eis-Dämon, der mir seine Kette übergab – war der erste und einzige Tulgori, der mich wiedererkannte.

Viele Tulgori, vor allem abenteuerlustige Jungen, wagten sich Jahr für Jahr ins Gebirge. Niemand kehrte je wieder, und wenn die ersten Sonnenstrahlen des Frühlings den Schnee an den felsigen Hängen schmolzen, gab ihnen das Gebirge die Toten zurück. Doch Ijs‘ Körper war nie gefunden worden. Mein Vater und mein Bruder hatten nach mir gesucht und unter großer Trauer die Überreste meiner leichtsinnigen Begleiter gefunden, doch nicht mich. Und so hatten sie stets heimlich gehofft, Ijs könnte irgendwie überlebt haben, ja, einen Weg über das Gebirge gefunden zu haben.

Dem war nicht so.

Und mein Vater Saro war nicht glücklich, als er mich wiedersah. Leblos erschien ich ihm. Er zitterte stets in meiner Nähe. Er kam nicht damit klar, dass überall um mich herum Schneeflocken herumschwirrten, Küche und Schlafzimmer bedeckten. Und als ich ihm sagte, dass ich sein Sohn sei, da spürte ich, dass er mir nicht glaubte. Dass er mich verabscheute.

Da wandte ich mich ab zu gehen. Ich wusste, dass ich eigentlich Trauer, Schmerz oder dergleichen fühlen sollte. Aber in mir war nur eine Kälte. Eine Kälte, die meinen Vater von mir abwandte. Dies war das erste Mal, dass ich mir wünschte, wieder ein Mensch zu sein.

Ich will es nicht beschönigen. Solche Wünsche nach Menschlichkeit wirst du auch von Zeit zu Zeit haben, Nalle. Aber sie werden vorbeigehen. Und vermutlich mit der Zeit immer seltener werden. Im Gegensatz zu meinem Vater wissen die deinen, was dich erwartet. Hoffentlich reagieren sie besser auf dich.

Jedenfalls teilte mein Vater mir mit, dass mein Bruder nicht mehr in Tulgor verweilte. Eforas war gemeinsam mit einigen anderen unter dem Kuolema in Richtung eines fremden Landes namens Andor aufgebrochen. In Richtung des legendären Königreichs, aus dem einst ein gewisser Haamun angereist war, und wohin ein gewisser Barz hatte reisen wollen.

Sie waren erst vor wenigen Tagen aufgebrochen. Da fragte ich mich, ob es vielleicht einen Zusammenhang gab zwischen dem Einsturz des Felsentors und dem Aufbruch der Reisenden. Und ohne besseren Plan für mein zukünftiges Leben, ohne Platz in Tulgor, reiste ich los, ihnen nach, unter dem Berg hindurch. Andor machte mich neugierig. Und vielleicht würde mein Bruder mit mir besser klarkommen als mein Vater. Zu Beginn würde er das zwar nicht tun. Doch mit der Zeit würde er sich lockern.

So unterquerte ich das Kuolema-Gebirge. Ich konnte mich sogar besser durch die labyrinthischen Gänge der Temm orientieren als die Reisengruppe selbst, welche immer wieder in Sackgassen landeten. Ich konnte Abkürzungen finden, ja, mithilfe meiner Eismagie gar eigene Abkürzungen schaffen.

So überholte ich unwissentlich die Reisegruppe von Eforas und Haamun. Und ich erreichte als allererster das fremde Königreich.

Zu einer ähnlichen Zeit – oder vielleicht eher einige Wochen zuvor – war eine gewisse Iril kurz davor gewesen, den Wachsamen Wald zu erreichen.“

„Ist Iril die Person, der einst dein Hammer gehörte?“, unterbrach ihn Nalle.

„Genau so ist es“, lächelte Ijsdur. Er orientiere sich wieder neu und räusperte sich. Dann intonierte er mit klirrender Stimme weiter:

„Das stolze Schiff, die BALENA, glitt über die stürmische See. Sie war von Hadria in See gestochen, hatte Iril und einige Silbergüter der Silberzwerge von Silberland aufgegabelt und hatte inzwischen Werftheim passiert. Die Mannschaft hatte aufgeatmet. Die Windrichtung änderte sich bei den Nebelinseln ständig und nie konnte man vorhersehen, ob das Wetter halten würde. Doch wenn man einmal Werftheim passiert hatte, konnte man so gut wie sicher sein, dass die transportierten Waren ihren Zielhafen bald erreichen konnten. Und was hatten sie nicht alles für Waren geladen! Rindsleder aus Sturmtal, Werkzeuge aus Werftheim, Silberarbeiten aus Silberhall und allerlei ähnliche Produkte, die den Seehandel im und um das Hadrische Meer so lukrativ machten.

Die Wellen des Hadrischen Meeres schlugen gegen einen vollkommen überfüllten Bug. Die Gischt trug Meeressalz und den Geruch nach nassem Holz tief in Irils Nase. Sie prustete.

Am Horizont sah Iril hohe Bäume aufragen. Wie an einen entfernten Schatten erinnerte sie sich an die Silhouetten. Dahinter lag es, das Land ihrer Kindheit. Viele Jahre waren vergangen, seit sie von hier aufgebrochen war. Jahre, in denen sie sich großes Wissen angeeignet hatte. Und nun kehrte sie zurück, um Geheimnisse zu entdecken, derer sie sich jetzt erst gewahr wurde. Doch mit dem Horizont näherten sich auch kalte Zweifel: Würden die anderen Zwerge sie überhaupt willkommen heißen?

Sie hatte damals ihre Familie in Cavern zurückgelassen, ja, sich gar mit ihr zerstritten. Viele der Altvorderen hatten es nicht mit Freude vernommen, dass sich einige Schildzwerge nach der Entdeckung der Silbervorkommen unter dem Silberberg am südlichsten Zipfel Silberlands niedergelassen, die Silbermine Silberhall gegründet und den Silberschild dort oben behalten hatten. Diese Sehnsucht nach Neuem hatte so einige Schildzwerge angesteckt, was von den nicht Sehnsüchtigen nicht mit Wohlwollen quittiert wurde. Schließlich hatte dieselbe Sehnsucht den vorherigen Fürst Hallwort und dessen Gefolge auf Jari Dorrs Handelskogge – möge dieser Dieb in der Unterwelt verrotten – in den Norden gelockt und ihn sein Leben gekostet.

Nichtsdestotrotz hatte Aufbruchsstimmung in der Luft gelegen und in den folgenden Jahren hatten sich viele Schildzwerge aufgemacht, um Hallworts Vorbild zu folgen und sich den Silberzwergen anzuschließen. Manche hatten gar daran geglaubt, dass es sich hierbei um einen göttlichen Fingerzeig gehandelt hätte, der die Zwerge in den Norden riefe.

Iril war aus anderen Gründen nach Silberland gereist. Und nun, Jahrzehnte später, war Iril darauf und daran, ihre alte Heimat wiederzusehen. Nicht nur deswegen ging es ihr gerade speiübel.

Du musst bedenken, Nalle, dass sie eine Zwergin war. Zu meiner Zeit lebten in Tulgor noch keine Zwerge, aber du kennst bestimmt einige, oder?“

Nalle nickte stumm.

„Man möge möglicherweise meinen, dass die Körper von Menschen und Zwergen aufgrund ihres derart ähnlichen Aussehens auch ganz gleich funktionierten, aber dem ist nicht ganz so. Bei Zwergenkörpern geht vieles irgendwie langsamer voran. Sie wachsen und altern in anderem Tempo. Verletzungen heilen gemächlicher. Krankheiten brauchen länger zum Abklingen, aber auch zum Ausbrechen. Die erheblich erhöhte Lebenserwartung im Vergleich zu Menschen ist da keine Überraschung. Manche behaupten, die sturen Zwerge bräuchten auch länger, um aus alten Denkmustern auszubrechen und Neues zu lernen. Und ihre Schwangerschaften dauern signifikant länger und sind seltener.

So ganz nebenbei gesagt, gibt es natürlich auch bei den typischerweise seltenen Schwangerschaften Ausnahmen. Ich kannte da einen gewissen Zwergenfürsten, dessen Eltern über ein Dutzend Kinder in die Welt gesetzt hatten.

Auf jeden Fall dauert bei Zwergen auch der Zyklus signifikant länger und ihre Zeit des Blutes kann besonders heftig sein. Wurde mir gesagt. Erinnerungen daran wurden auch schon direkt mit mir geteilt. Sagen wir einfach, es ging Iril gerade speiübel, während ihr Schiff auf die Küste des Wachsamen Waldes zuhielt.

Geräuschvoll übergab Iril sich über die Reling und warf den sie belustigt begutachtenden Matrosen böse Blicke zu. Zu ihren Krämpfen gesellten sich stechende Kopfschmerzen. In Silberhall hätte Iril frisch gebrauten Mondkrauttee schlucken und ihre Füße hochlegen können. Doch hier, auf diesem Schiff, blieb ihr kaum etwas anderes übrig, als die Zähne zusammenzubeißen und zu hoffen, dass die Fahrt nach Andor möglichst bald und möglichst ereignislos zu Ende gehe.

Leider war sie nicht so glücklich.

Ein markerschütternder Schrei hallte durch den Wachsamen Wald.“

Nalle blickte Ijsdur erwartungsvoll an. Dieser legte theatralisch den Kopf in den Nacken und schrie in die dunkle Nacht hinaus.

„DRACHEE!“



\newpage
\section{Der Zorn des Drachen}

\az{Jahr 65}


„DRACHEE!“

Der verzweifelte Schrei hallte durch den Wachsamen Wald und weit ins Hadrische Meer hinaus. Die Nixen, welche bis eben noch dem Schiff nachgeschwommen waren und der blassen Iril mitleidige Gute-Besserungs-Wünsche zugesprochen hatten, waren auf einmal nirgendwo mehr zu erblicken. Vermutlich hatten sie sich in ihre tief unter der Oberfläche liegenden Behausungen zurückgezogen.

Keine Minute zu früh.

Denn auf einmal waren das Rauschen gewaltiger Schwingen und ein fernes Brüllen zu vernehmen.

„Ein Drache?!“, wiederholte Iril ungläubig, „Einen Drachenangriff gab’s doch nicht mehr seit dem Unterirdischen Krieg! Der ist Jahrhunderte her!“

Damit hatte sie fast recht. Sie konnte das nicht wissen, denn die wenigen, unvollständigen Informationen zum letzten aller Drachen waren verstreut über verbotene Schriftrollen in den Schwarzen Archiven und alte Lieder von Tavernenwirtinnen, die Iril nicht kannte. Doch ein einziger Drache hatte den Unterirdischen Krieg und das nachfolgende Massensterben überlebt.

Tarok. Er hatte sich während Jahrhunderten in seiner Höhle im Grauen Gebirge verschanzt und in unruhigen Träumen seinen Zorn gepflegt. Hin und wieder hatte Tarok nichtsahnende Reisende überfallen, doch seit einem legendären Kampf gegen den zukünftigen König Brandur von Andor hatte er auch das kaum mehr getan. Er hatte sich gefürchtet, geschlafen und gelitten. Während ein, zwei Gelegenheiten, die hier nicht näher behandelt werden sollen, hatte Tarok sich aus unruhigen Träumen reißen lassen und in die Lüfte erhoben, um Feuer und Glut vom Himmel regnen zu lassen und Bewohner der Berge und Steppe an seinem Leid Anteil haben zu lassen. Und um diese widerlichen Krahder in ihre Grenzen zu verwesen.

So kam es, dass hin und wieder in der Bevölkerung das Gerücht umging, dass jemand einen leibhaftigen Drachen gesehen hätte.

Doch davon abgesehen hatte Tarok sich während Jahrzehnten still in seiner Höhle versteckt und Kraft aus Krahal gezogen. Die Schildzwerge und ihre Drachenfallen hatte er gemieden. Brandur hatte er gemieden. Andor hatte er gemieden. Ruhig darauf wartend, dass König Brandur das Zeitliche segnete. Er war ein Drache, theoretisch unsterblich, sofern er nicht dem Fluch des Steins anfiel. Brandur war nur ein Mensch. Mochte die Hexe Reka sein Leben mit Tränkchen und Mitteln unnatürlich in die Länge ziehen: Früher oder später würde die Zeit seinen Erzfeind für ihn erledigen.

Aber das musste nicht heißen, dass Tarok nicht nachhelfen konnte.

Nun war es soweit gewesen. Taroks Kreaturen hatten unter dem Befehl seines Schwarzen Herolds die Rietburg eingenommen und Brandur tödlich verwundet. Während die Helden um die Befreiung der Rietburg gekämpft hatten, war der alte König seinen Verletzungen erlegen. Und Tarok war erwacht. Noch in derselben Stunde hatte er sich aus seiner Knochengrube erhoben und war schnurstracks zum Angriff auf Andor übergegangen.

Auf seinem Weg zur verhassten Rietburg hatte er einen Abstecher zum Baum der Lieder gemacht und die Magie des Landes in sich aufgezogen. Nun, mit erheblich gewachsener Macht, erhob sich der Drache mit mächtigen Flügelschlägen aus dem Wachsamen Wald in die Höhe.

Und wurde unter anderem von einer gefährlich nahe segelnden Handelskogge erspäht.

Iril kniff ihre Augen zusammen. Täuschte sie sich, oder saß tatsächlich ein Reiter auf dem Rücken des Drachen? Zwischen zwei Rückenstacheln eingekeilt, ein langes Schwert hoch in die Luft gereckt, den langen schwarzen Umhang hinter sich flatternd.

Irils Aufmerksamkeit wurde vom mysteriösen Reiter weggerissen, als der Drache sich einmal im Kreis drehte, während er einen gewaltigen Flammenstrahl spie.

Bäume und Büsche entflammten. Beißender Rauch stieg auf vom Wachsamen Wald. Und Iril wurde ihrer prekären Lage gewahr. Sie befand sich auf einem brüchigen Kahn vor einem brennenden Wald, mit einer riesigen Echse hoch über ihr, die sie jeden Augenblick erspähen und für Grillspaß befinden konnte.

„Wir müssen weg von hier, echt schnell!“, rief Kapitän Lunor und kurbelte wild am Steuer herum. Lunor war ein stämmiger, muskulöser Mann, an dessen Körper alle Stellen, deren Anblick man schicklicher Weise Anderen zumuten konnte – und vermutlich auch einige unschickliche Stellen – mit Tätowierungen bedeckt waren. Diese Tradition hatte Kapitän Lunor an seine Mutter angelehnt, der legendären Kapitänin Mondrianne, die der Legende nach einst gar ein Handelsschiff aus Oktohans Schlund gerettet hatte. Im Gegensatz zu Irils Runentattoos hatten Mondriannes und Lunors Tätowierungen eher persönliche als magische Bedeutungen. Und im Gegensatz zu Iril hatten sie keine magische Kraft nötig. Mit purer Muskelkraft zog Lunor am großen Steuer und blinzelte in die Gischt. Die BALENA legte sich gefährlich schief und drehte sich langsam vom Ufer ab, in Richtung des sicheren Nordens. Iril klammerte sich an der Reling fest, um nicht übers Deck zu rutschen. Doch der Wind war nicht mit ihnen. Nur langsam kamen sie voran.

Die meisten Reisenden suchten unter Deck Schutz. Nicht so Iril. Sie öffnete ihre Reisetasche und kramte daraus eine bestimmte kleine Metallscheibe hervor, kaum größer als ihre Handfläche. Es war keine komplexe Runenfolge, die sie jetzt brauchte, aber eine sehr mächtige. Sie griff an ihren Gürtel.

Es fühlte immer noch falsch an, Burmrits Runenhammer selbst zu führen. Sorgfältig strich Iril über die gravierte Oberfläche und über das stumpfe Ende des Hammers. Ein leises Summen ertönte aus seinem metallenen Inneren. Er war bereit, magische Ströme weiterzuleiten. Iril fasste den Runenhammer fest und ließ ihn mit der flachen Seite auf die Runenscheibe niederfahren. Ein Geräusch wie von einem Gong ertönte, gefolgt von einem fernen Donnern. Runenscheibe, Hammer und Irils Augen glühten gelblich auf. Ihr ganzer Körper kribbelte wohlig.

Der Wind nahm zu und bauschte die Segel des kleinen Schiffs. Die BALENA nahm Fahrt auf, so schnell wie möglich weg vom Ufer mit dem wütenden Drachen. Die beiden Seekrieger, die nebst Kapitän Lunor als einzige auf Deck geblieben waren, jubelten und klatschten. Doch dann furchte die eine plötzlich ihre Stirn und schrie an Iril zurück: „Abbremsen! Abbremsen! Da vorne ist der Hirschhuf! Klippe voraus!“

Iril drehte die Runenscheibe in ihren Händen, verzweifelt versuchend, den Wind noch rechtzeitig abzulenken.

Kapitän Lunor drehte ebenso verzweifelt am Steuer und fluchte gegen den aufziehenden Sturm.

Mit voller Fahrt raste das Schiff in die Klippe hinein.

Ein hässliches Knirschen ertönte, als der Rumpf des Schiffes sich öffnete und Wasser hineinströmte.

Mit einem Ruck wurde Iril von der Reling gerissen und taumelte übers Deck.

„Sporndreck!“, fluchte sie. Schuldgefühle übermannten sie. Doch sollte sie sich lieber später damit befassen, wenn sie nicht mehr auf einem sinkenden Schiff stand.

Die Seekriegerin fasste sich als erste und schrie: „Hier sind wir nicht sicher. Wenn ein Drache hier ist, sind seine bösartigen Kreaturen nicht fern! Und das Meer wimmelt nur so von grausamen Meereskreaturen!“

„Wo sollen wir hin?! An Land wütet der Drache!“

Iril warf einen wachsamen Blick ans Land rüber. Der Drache schwebte nicht mehr über dem Wachsamen Wald, sondern hatte sich weiter gen Westen bewegt. Der dunkel gewandte Reiter auf seinem Rücken zeigte mit seinem langen Schwert nach unten. Der lange Hals des Drachen folgte. Und wie ein Seeadler stieß der mächtige Drache auf irgendetwas im Westen herab.

Iril öffnete erneut ihre Reisetasche und zog eine andere Runenscheibe hervor. Diese hier bestand aus einer handgroßen gläsernen Linse, die in einem mit Runen übersäten Steinring eingefasst war. Iril hielt ihren Runenhammer darunter und konzentrierte sich. Licht strömte hervor, schillerte in allen Regenbogenfarben und ließ in der Runenlinse ein verzerrtes, bewegtes Bild erscheinen.

Iril blickte aus der Vogelperspektive auf das Geschehen herab. Ein großer Schatten fiel auf das Rietland. Das goldene Rietgras zerfiel unter einem Flammenstrahl zu schwarzer Asche. Der Drache stieß auf einen alten Wachturm hinab.

Iril kniff ihre Augen zusammen und versuchte, in der Schemen dieser Runenscheibe mehr Einzelheiten zu erkennen.

Das Gemäuer, das sie erblickte, musste der legendäre alte Wehrturm sein. Vor Urzeiten war er einst von den Schildzwergen erbaut worden, doch seitdem er einst in einem Kampf gegen einen riesigen Trollfürsten eingerissen worden war, munkelte man, dass ein Fluch auf ihm läge. Schon so oft hatte jemand versucht, ihn wieder aufzubauen, und so oft war jemand daran gescheitert. Jetzt stand der Turm wieder. Vermutlich hatten mutige Andori ihre Kämpfer mit dem Wiederaufbau des Turms unterstützt, so gut es ging. Iril wäre jede Wette eingegangen, dass er in einigen Monaten schon wieder eine Ruine wäre.

Iril konzentrierte sich stärker. Das magische Bild innerhalb der Runenlinse zoomte hinein und verdeutlichte sich.

Vor dem Wehrturm hatte sich eine Gruppe Krieger zusammengeschart. Kleine Gestalten umringten einen Reiter auf einem schwarzen Pferd. Sie flohen nicht, sondern schienen ihre Waffen bereitzuhalten. Ganz oben auf dem Wehrturm hatte sich gar ein Bogenschütze eingerichtet und verschoss wirkungslos Pfeile auf den auf sie zustürzenden Drachen.

Tapfere Idioten.

Schade um sie.

Der verzerrte Drache in der Runenscheiben-Vision riss seinen Mund zu einem Schrei auf. Kurz darauf schallte ein leises Echo seines animalischen Schlachtrufes an Irils Ohren.

Sie zuckte zusammen, als Kapitän Lunor direkt neben ihr zu erkennen gab, dass er die Scheibe ebenfalls beobachtet hatte. Laut rief er: „Der Drache ist abgelenkt. Das Risiko an Land ist echt geringer, als wenn wir im sinkenden Schiff bleiben. Hier sind wir wie ein Signalfeuer für Nerax. Und für Schlimmeres. Echt, wir müssen an Land rüber. Macht euch bereit“

Mit Blick auf Iril fügte er an: „Kannst du schwimmen?!“ Iril nickte hastig. Technisch gesehen konnte sie schwimmen. Sie hatte es nur schon lange nicht mehr getan. So schwer konnte es ja nicht sein, sich an die Züge zu erinnern.

Während Lunor unter Deck rannte und seine kleine Tochter an Bord rettete, packte Iril ihre Runenscheibe zurück in ihre luftdicht verschließende Tasche und setzte an, ihre Schuhe und sonstige schwere Kleidung zurückzulassen. Da tauchten plötzlich zwei Nixen vor ihr aus dem Wasser auf. Sie blubberten und heulten etwas. Unter Wasser hätte es bestimmt schön geklungen und der Klang wäre weit gedrungen, doch hier, bei der Geräuschkulisse eines Drachenkampfs im fernen Hintergrund, hätte Iril sie wohl nicht einmal verstanden, wenn sie Nixisch sprechen könnte.

Natürlich gäbe es entsprechende Runentattoos, die Iril aktivieren könnte, um die Nixen zu verstehen. Magie war ein wundervolles Allzweckmittel, mithilfe dessen man überraschend schnell Strukturen in gesprochenen Worten erkennen und fremde Sprachen entschlüsseln konnte. Doch soweit musste Iril gar nicht greifen. Die freundlich lächelnden Gesichter der Nixen waren verständlich genug, sodass Iril sich die Bedeutung der fremden Worte vorstellen konnte. „Wir können euch helfen“, sagten sie leise.

Die beiden Seekrieger schlugen das Angebot der Nixen zugunsten weniger schwimmfähiger Passagiere aus und sprangen elegant ins Meer. Es geschah nicht selten, dass ein Schiff in einen Sturm geriet (oder schlimmeres) und seine Passagiere dann irgendwo an Land gespült wurde. Seekrieger mussten das Schwimmen beherrschen, und ihre Schulterplatten waren aus leichtem, auftreibendem Material statt schwerem Metall. Kompliziert war vielmehr, dass der eine Seekrieger während der Überfahrt einen strampelnden Streifenmarder an sich drückte und über Wasser zu halten versuchte.\footnote{Beitrag aus der Taverne, "Portale, Umweltverschmutzung und mehr": \newline [Michael Menzel:] Die Streifenmarder sind sehr gute Schwimmer, besonders wenn Sie vorher ein bisschen Nixenstaub hatten. :D}\textsuperscript{,}\footnote{Beiträge aus der Taverne, "Bärig gebutterte Geschichtchen": \newline [Bennie:] Handelst es sich bei dem Streifenmarder-Schiffsmaskottchen um eine besondere nichtschwimmende Art? \newline [Butterbrotbär:] Danke, Bennie! :P Falls ich das Zitat zu den sehr gut schwimmenden Streifenmardern überhaupt zuvor schon mal gelesen hatte, dann hatte ich es leider nicht mehr auf dem Radar. Und ich habe den Streifenmarder wohl ohnehin unbewusst mit einer gehörnten Katze gleichgesetzt (darf ich da dem gestiefelten Streifenmarder die Schuld in die Stiefel schieben? ), statt zu googlen, wie Marder zu Wasser stehen. Gar nicht schlecht, stellt sich heraus. :oops: Kann ich mich noch herausreden? Vielleicht wollte Stinner einfach verhindern, dass der geliebte Marder ihm unter Wasser davonschlüpfen und in eine ganz andere Richtung davonschwimmen könnte? Vielleicht war Stinner sich einfach nicht bewusst, dass der Marder eigentlich prima schwimmen könnte? Vielleicht hatte dieser bestimmte Marder gelähmte Hinterbeine oder irgendetwas Ähnliches, das ihn ausnahmsweise unverträglich mit Schwimmen machte? Überlassen wir die Antwort der Fantasie der Leser. :D}
Aber das Schiffsmaskottchen zurückzulassen kam nicht in Frage.

Lunor weigerte sich, seiner Tochter einer Nixe zu übergeben. „Ich habe echt schon seit meiner Kindheit der See getrotzt! Ich kümmere mich selbst um die Sicherheit meines Töchterleins.“

Er führte seine Tochter sanft in die wogende See und schwamm mit kräftigen Zügen in Richtung Ufer.

Die restlichen Passagiere ließen sich bereitwillig von den Nixen an Land bringen. Iril beobachtete, wie die zwei Seekrieger ans stürmische Ufer traten und sofort ihre langen Naginata bereithielten, als würden diese Stangenwaffen etwas gegen einen angreifenden Drachen ausrichten können.

Dann war Iril an der Reihe. Das Wasser des Hadrischen Meers war kühl und stürmisch. Salz ließ Irils Augen brennen, während sie sich verzweifelt an den glitschigen Körper ihrer Nixe zu klammern versuchte. Sie hustete und prustete und schluckte Seetang. Dann war es vorbei.

Iril schleppte sich an Land und ließ sich aufs sandige Ufer sinken

So hatte sie sich ihre Rückkehr nach Andor nicht vorgestellt.

„Danke.“

Die Nixe blubberte ihr eine unverständliche Abschiedsnachricht zu und sprang dann zurück in die tosende See, um weitere Reisende aus dem Wrack zu retten. Iril wrang ihre nassen Haare aus und betrachtete das Schiffswrack auf dem Hirschhuf, der Klippe vor dem Wachsamen Wald. Sie versuchte nicht an die vielen Waren zu denken, welche nun dort nun entweder verrotten oder von Meereskreaturen gestohlen werden würden. Immerhin waren alle Schiffsbrüchigen in Sicherheit.

Da ertönte ein Schrei direkt neben ihr. Sie wirbelte herum. Kapitän Lunor saß klitschnass neben seiner Tochter. Selbige lag zusammengesunken im seichten Ufergebiet und rang erfolglos nach Luft. Wasser strömte aus ihren Augen und ihrem Mund. Lunor schüttelte seine Tochter panisch. Das half natürlich nicht.

„Lasst mich!“, rief Iril. Sie kniete sich hastig neben die Kleine. Der Runenhammer glitt zu Boden. Iril öffnete ihre Reisetasche und durchsuchte ihre Vorräte an kleinen metallenen Scheiben. Manche Runenfolge war so vielseitig nützlich, dass sie sie permanent auf einer ihrer Scheiben eingraviert hatte und mit sich führte. Iril fand die gesuchte Scheibe, hob sie in die Höhe und ließ sie mit aller Macht auf ihren Runenhammer donnern. Die Scheibe flackerte bläulich auf. Dünne Linien darauf begannen zu schimmern und verschmolzen rasch zu Symbolen. Die Scheibe wurde warm. Ebenso pulsierten Irils Augen und der Runenhammer in blauen Tönen. Ihre gesamte Haut kitzelte wohlig.

Der Sturmwind ebbte ab.

Überall um Iril herum stiegen Wassertropfen aus dem matschigen Boden in die Höhe. Iril selbst fühlte ebenfalls den vertrauten Zug des Wassers in ihrem Körper, auch wenn sie selbst natürlich zu schwer war, um davon in die Höhe gehoben zu werden. Iril bewegte die leuchtende Runenscheibe zur hustenden Tochter und zog sie langsam von ihrem Magen bis zu ihrem Gesicht. Weitere Wasserschwalle brachen aus der Kleinen hervor und sammelten sich als wabbelige Wassermasse um die Schiebe. Ihr Gesicht rot an, ebenso Irils Hand, die die Scheibe hielt. Dann erbrach sich die Kleine und sank schwach in die Arme ihres Vaters. Angeschlagen, doch wieder angemessen atmend.

„Danke. Echt“, hauchte Lunor, seine Tochter eng umarmend.

„Kein Problem. Wasserrunen sind meine Spezialität. Könnte mit der Hintergrundgeschichte dieses Hammers zusammenhänge ... aber nein, die ist jetzt nicht wichtig“, murmelte Iril. Sie tippte so rasch wie möglich auf die Scheibe und brach den Anziehungseffekt ab, ehe sie aus Versehen die Kotze der Kleinen ansaugte.

Erst jetzt konnte Iril sich wirklich auf die Umgebung konzentrieren. Die Nixen hatten sie zu den Anlegestegen der Bewahrer gebracht. An diesem Hafen wurde sonst oft emsig gehandelt. Nicht so heute.

„Hierher. Da lodert es!“

„Eimerkette bilden! Dalli, dalli, der Wald wartet nicht!“

Waldbewohner eilen umher und versuchen, die Brände im Wachsamen Wald einzudämmen.

Der über die restlichen Bäume hinausragende Wipfel des gewaltigen Baums der Lieder war angekokelt. Löschfässer wurden von seinen obersten Ästen gelöst und verschwanden hinter den tiefer liegenden Baumwipfeln, gefolgt von Platschgeräuschen.

Einige Bogenschützen in der Kleidung der Farbe des Sommerlaubes erreichten die leeren Anlegestege. Manche hielten Bögen im Anschlag, andere trugen große Rucksäcke. Sie begleiteten weitere Bewahrer in grauen Gewändern. Ein Bewahrer, in ein weißes Kleid gekleidet, rollte auf einem hölzernen Stuhl mit breiten Rädern an der Seite über den Waldpfad zum Hafen und balancierte mühevoll einen Stapel breiter Bücher und Schriftrollen-Kisten auf seinem Schoß, so hoch, dass er selbst kaum darüber hinwegblicken konnte. Sein kleiner Kopf guckte auf der Seite des Bücherstapels hervor und sank enttäuscht, als er den leeren Hafen erblickte.

„Was nun, Hoher Priester Tion?“, fragte ihn eine Begleiterin.

Tion kratzte sich am Bart. „Unschön. Wenn in den nächsten Tagen keine Schiffe von hier abfahren, müssen wir umgehend kehrt machen und anderswo Asyl suchen. In die Tiefen Caverns oder in die Lande der wilden Völker des Ostens. Nur weit weg vom wütenden Drachen.“

Iril sollte später herausfinden, dass die Bewahrer hier am Hafen auf Boote gehofft hatten. Der oberste Priester der Bewahrer hatte sie weise aufgefordert, das Land zu verlassen. Sie hatten den Auftrag, so viele Pergamente wie möglich vom Baum der Lieder mitzunehmen und nach Sturmtal aufzubrechen. Von dort sollten sie ein Schiff nehmen und weit, weit fortsegeln. Denn Tarok, der Drache, würde den Baum der Lieder bestimmt nicht verschonen, wenn er erst einmal die Rietburg dem Erdboden gleichgemacht hatte.

Und nun lagen nicht einmal mehr Boote zum Fliehen an den Stegen.

Die Bewahrer waren nicht als einzige über die fehlenden Schiffe im Hafen enttäuscht. Garz, ein im gesamten Norden berühmt-berüchtigter Handelszwerg mit einem komödiantisch dicken Rucksack auf dem Rucksack, blickte fassungslos den leeren Steg entlang und murmelte: „Mein Schiff nach Hadria sollte doch schon längst hier am Hafen vor Anker liegen. Bei allen Kreaturen der Tiefe, wo ist mein Schiff nach Hadria?!“

„Auf der Klippe liegt es. Ein Wrack ist es geworden!“, sprach Kapitän Lunor missmutig, immer noch seine Tochter umarmend, „Und es war nicht mal mein eigenes Schiff. Syenna wird mich umbringen.“

Das holte Iril in den Moment zurück.

„Verzeiht mir. Das ist alles meine Schuld“, sprach sie hastig, „Ich weiß nicht, wie ich dafür aufkommen soll. Ich bin nicht reich, habe aber noch einige Goldmünzen ...“

„Nein“, widersprach Lunor, „Es mag dein magischer Wind gewesen sein, der uns auf den Hirschhuf auflaufen ließ. Aber es war auch mein Segel. Und mein Steuer. Ich werde dafür geradestehen. Du hast meine Tochter gerettet. Ich würde sagen, wir sind echt mehr als quitt.“

Iril nickte, doch das Schuldgefühl in ihrem Inneren ließ sie nicht los. Sie war es gewesen, die Lunors Tochter überhaupt erst in Gefahr gebracht hatte. Niemand hätte ihr verziehen, wenn die Kleine ihr Leben verloren hätte.

„Hört ihr das auch?“, unterbrach sie der eine Seekrieger, der immer noch seinen tropfnassen Streifenmarder an sich drückte und streichelte.

„Was meinst du, Stinner?“, fragte die andere Seekriegerin, „Ich höre nichts.“

„Eben.“

Bislang war die ganze Zeit wie in weiter Ferne ein Echo zu vernehmen gewesen. Ein Brüllen und Fauchen, das den Boden leicht erzittern ließ. Der Lärm eines Drachenkampfs. Doch nun war es plötzlich still geworden.

Einen Augenblick lang standen alle Anwesenden wie erstarrt da.

Urplötzlich rumpelte die Erde. Heftig. Vögel stiegen von den Bäumen auf. Laub, Äste, Eichhörnchen und kletternde Streifenmarder wurden gleichermaßen zu Boden geworfen. Sturmwellen platschten mannshoch ans Ufer. In der Ferne hörte man Felsen bersten und Lawinen krachen.

Dann war das Erdbeben auch schon wieder vorbei. Der Wachsame Wald lag wieder ruhig da. Und dennoch konnte Iril das Gefühl nicht abschütteln, dass etwas grundlegend anders war als noch vor wenigen Minuten.

„Was war das? Seht ihr irgendetwas?“, rief eine Bewahrerin zwei anderen zu, welche mit Ferngläsern in den Himmel starrten.

„Kein Drache in Sicht“, kam die Antwort.

„Ich glaube echt, der Drache ist tot!“, jubelte Lunor ein wenig optimistisch. Die Seekrieger stimmten mit ein. Die Bewahrer vom Baum der Lieder schienen noch skeptisch.

Der Hohe Bewahrer Tion gab den Befehl, man solle ihn und die wertvollen Schriften zurück zum Baum der Lieder befördern. Hier am Hafen waren sie nur ein Ziel für Kreaturen. Und hoffentlich würde bald schon ein Falke mit Neuigkeiten eintreffen.

Iril schloss sich ihnen an.

Sie bot Tion auch an, einen Teil seines gewaltigen Bücherstapels abzunehmen, doch der Hohe Bewahrer winkte entschieden ab. Die Schriften wären sowohl sehr wertvoll als auch potenziell gefährlich und sollten somit lieber nicht von einer dahergelaufenen Kriegerin transportiert werden. Iril versuchte, das nicht persönlich zu nehmen.

Als Tion dann allerdings über eine zu große Astwurzel rollte und sein Schriftenstapel beinahe in sich zusammenfiel, konnte er sich immerhin dazu herablassen, Iril die oberste Schachtel voller Schriftrollen tragen zu lassen.

So reiste die Truppe weiter zum Baum der Lieder.\bigskip







Ein riesiger schwarzer Fleck zeigte an, wo unlängst ein gewaltiger Drache auf dem Baum der Lieder gesessen und den uralten Mammutbaum mit Klauen bearbeitet hatte.

Die Lichtung war übersät von angekokelten Riesenästen und den Leichen von Menschen und Zwergen, Gors und Skralen. Die sonst rötlich-pink schimmernden Schuppen der Kreaturen waren von schwarzen Spuren übersät und die sonst weißen Augen schimmerten tiefschwarz. Hatte der Drache sie mit einem dunkelmagischen Blitz gestärkt?

Auch wenn die aktuelle Gefahr durch das Feuer gebannt sein mochte, herrschte an der Lichtung wie am Hafen emsiges Treiben. Lösch- und Bergungsaktionen wurden durchgeführt. Verletzte sammelten sich an feuchten Stellen und blickten furchtsam in den Himmel.

„Doro, überprüfe bitte, ob bereits Falken eingetroffen sind“, sprach der Hohe Bewahrer Tion zu einer seiner Begleiterinnen, kaum hatten sie den Rand des Dorfes erreicht. „Und magst du auch noch beim Ausguck nach Neuigkeiten vorbeischauen?“

„Bis ganz nach oben?! Oh, Mutter, wie ich mir wünschte, dass es Abkürzungen für so etwas gäbe“, sprach Doro, eilte dann jedoch folgsam davon. Tion drehte sich um, wies einen hammerschwingenden Begleiter auf einen seiner Meinung nach höchst einsturzgefährdeten Ast hin und sprach weiter: „Und Giftknödel, kannst du bitte ... Giftknödel, wo steckst du?!“

Der angesprochene junge Bewahrer eilte zu Tion, um weitere Anweisungen entgegenzunehmen. Allzu dringlich konnten sie jedoch nicht sein. Denn anstatt fortzueilen, nickte Giftknödel nur und kehrte danach wieder an Irils Seite zurück. Er warf weitere interessierte Blicke auf Irils leuchtenden Hammer, wie er es schon auf dem ganzen Weg zum Baum der Lieder getan hatte.

„Das ist ein Runenhammer“, meldete Iril kurz angebunden, „Der Runenhammer von Golja.“

„Runenmagie?“

„Genau! Ihr Bewahrer wisst bestimmt so einiges darüber.“

„Aber nicht über einen solchen Hammer. Die Lücken in unseren Archiven schrumpfen stetig und werden doch nicht weniger. Welch mannigfaltige Fähigkeiten verleiht der Hammer dir?“

„Ich habe lange studiert, um verschiedenste Aspekte der Macht der Runen zu meistern. Der Hammer ist nur ein kleiner Teil davon.“

„Noch bin ich nur ein Novize, doch bald, in weniger als zwei Jahren, werde ich ein vollwertiger Adept des Bewahrerordens werden und mitbestimmen dürfen, welche Berichte ich verfolgen will. Darf ich dich dann aufsuchen und mehr erfahren, eine Lücke unserer Legenden schließen?“

„Wenn du mich dann noch findest, dann sicher.“

„Wo dürfte ich dich denn erwarten?“

„Wenn ich das wüsste. Meine Zukunft ist ungewiss. Cavern. Andor. Irgendwo, wo man mich brauchen kann.“

„Das engt es nicht wirklich ein.“

„Das ist wohl wahr. Vielleicht halte ich einfach nach dir Ausschau, wenn es mich in Zukunft wieder an den Baum der Lieder lenken sollte.“

Die beiden wurden vom emsigen Geschehen auf der Lichtung abgelenkt.

Eine krächzende Stimme tönte über das Stimmengewirr hinaus. „Trinkt das. Der Schleim mag euch grausen, im Nachhinein werdet ihr mir aber danken. Nein, Larissa, jetzt vergeuden wir unsere Zeit sicher nicht damit, übers Geld zu sprechen. Die Verletzten brauchen eine Heilerin.“ Die Urheberin dieser Worte, eine grau gewandte Bucklige, eilte im Dorf umher und verteilte orange leuchtende Tränke, die gegen die Hitze helfen sollen. Auch wenn sie scheußlich schmeckten und ihre Konsistenz an schleimigen Schaum erinnerte, sollte man sie so rasch wie möglich runterschlucken. Eine Bewahrerin in einem langen weißen Gewand – Heilerin Larissa – eilte der Alten hinterher und zischte etwas über zu hohe Preise.

Dann erklang Geschrei aus dem Unterholz. Ein Flötenspieler, dessen verschmutzte Kleidung mit einem hübschen Rautenmuster überdeckt war, unterhielt sich ein wenig abseits im Unterholz aufgebracht mit einer blauhäutigen Kapuzengestalt mit Augenbinde. Gerade intonierte der mutmaßliche Barde: „... lügnerischer Habicht, du, der mir die Sonne vom Himmel zu holen versprach, wenn ich nur deinen Wünschen Folge leiste. Worin auch immer deine weiteren Pläne bestehen mögen, zähle mich absent von ihnen, du kleingeistiger Scharlatan niederster Sorte!“

Die erheblich leiser gesprochene Antwort des Blinden entging Iril, denn in diesem Augenblick kehrte auch schon wieder die von Tion ausgesandte Bewahrerin zurück, ein weiteres Ordensmitglied im Schlepptau. Jenes versuchte vergeblich, eine goldene Mitte zu finden zwischen dem möglichst raschen Laufen zu Tion und der Schonung eines jungen Falken in seinen Armen.

„Keine Neuigkeit von der Rietburg. Gända meldet sich nicht. Tapta meint, dass der einzige kürzlich eingetroffene Falke aus dem Osten kam und keine Nachricht, sondern bloß einen angeknacksten Flügel mit sich mitbrachte.“

„So ist es“, bestätigte Tapta, weiterhin den angeschlagenen Falken stützend, „Keine neue Nachricht über den Drachenangriff oder dieses Erdbeben. Keine Informationen von Gända. Immerhin hat der Ausguck seit letzterem keinen fliegenden Drachen mehr erblickt. Doch muss das nichts heißen. Wir können mit den besten Fernrohren nicht ins Rietland blicken. Der Nebel steht heute hoch über den Wipfeln des Wachsamen Waldes. Es könnte weiterhin weise zu sein, unsere wichtigsten Schriften in Sicherheit zu bringen.“

Dann fiel sein Blick auf Giftknödel, und er führte an: „Aber zumindest steht dein Feigenbaum noch, Phlegon. Kein Funken hat ihn erreicht!“

„Immerhin so viel“, entspannte sich Giftknödel, „Dann darf ich wohl annehmen, dass es um unsere Familie und Freunde auch nicht schlechter steht?“

Tapta nickte. Die zwei jungen Bewahrer schienen weitersprechen zu wollen, doch verstummten sie.

Stille breitete sich allgemein auf der Lichtung vor dem Baum der Lieder aus, als eine herrische Gestalt durch die Portale des Baums trat.

Gekleidet war der Priester in ein edles weißes Gewand mit goldenen Verzierungen, welches einige Rußflecken trug. Die langen braunen Haare trug er offen und waren wild zerzaust. Nichtsdestotrotz strahlte er eine ehrwürdige Aura aus.

Iril hatte schon von ihm gehört.

Der Oberste Priester Melkart.

Der Anführer des Bewahrerordens ließ eine Versammlung einberufen. Als erstes machte er sich daran, zwei Mitglieder seines Ordens zu finden, welche mutig genug waren, ins möglicherweise drachenverseuchte Rietland aufzubrechen.

Eigentlich verließen die Bewahrer so selten wie möglich den grünen Radius, sondern warteten darauf, dass Besucher aus der Umgebung ihre Berichte hierherbrachten. Dies war eine Ausnahme. Ein Drachenangriff war ein unglaublich ungewöhnliches Ereignis. Wieder einmal hatte die Geschichte Andors einen Wendepunkt erreicht. Und es war die Aufgabe der Bewahrer, diese Geschehnisse niederzuschreiben. Damit große Geschichten nicht zu Verschollenen Legenden wurde. Melkart wollte das Risiko nicht eingehen, Berichte aus erster Hand zu verpassen, weil Verletzte bald aus dieser Welt scheiden könnten.

So erlaubte der Oberste Priester zwei jungen Bewahrerinnen, zwei fortgeschrittenen Adeptinnen des Bewahrerordens namens Sanja und Jorna, den grünen Radius zu verlassen. Aufgeregt reisten die beiden bald los. Sie sollten in den Westen aufbrechen und herausfinden, ob der Drache tatsächlich gefallen war. Falls dem so sei, sollen sie Berichte der Überlebenden zu sammeln. Falls nicht ... möge die Mutter ihnen gnädig sein.\bigskip







Sanja und Jornas Ziel, das westliche Rietland, war nicht Irils Ziel. Der Pfad unserer Runenmeisterin führte sie weiter nach Cavern. Dorthin hatte sie eigentlich schon von Beginn an aufbrechen wollen.

Auf dem Weg durch den Wald bis zum nördlichen Mineneingang wurde sie von mehreren Falken überflogen. Die Nachrichtenvögel verbreiteten unzweifelhaft Nachrichten über die aktuelle Lage. Iril wünschte sich, sie zu lesen.

Der nördliche Mineneingang wirkte noch unversehrt. Einzig die große Tanne, deren tief hängende Äste die Pforte verbergen sollte, war ein wenig angekokelt.

Zwei gut gerüstete Zwergenwachen standen davor und schienen ihre Gedanken gerade woanders zu haben. Anstatt ihre Waffen zu heben und Iril nach ihrem Belang fragten, blickten sie wie erstarrt in die Ferne. Der eine, mit einem helleren Hautton, streichelte gedankenverloren eine elegante Eule auf seinem Arm.

„Werte Wachen!“, rief Iril beim Nähertreten, „Mein Name ist Iril von Silberhall, und ich bin hier ...“

„Keine Zeit für lange Vorstellungen. Ich bin Bort“, unterbrach sie der eulenlose Wächter, und zeigte auf seinen eulentragenden Kumpanen, „Und der hier ist Mart. Sprecht, habt Ihr Neuigkeiten vernommen?“

„Ich sah den Drachen aufsteigen und über dem alten Wehrturm niedergehen. Das ist alles.“

Bort ließ seinen Kopf sinken.

„Was ist überhaupt los? Wie ist es möglich, dass nach all dieser Zeit ein Dr...“, setzte Iril an.

„Der König der Andori ist tot“, sprach Mart niedergeschlagen, ohne Iril ausreden zu lassen. „Die Lage ist ernst. Der Prinz von Andor wurde verschleppt. Die Rietburg wurde angegriffen und eingenommen, während die Helden von Andor sich hier in unserer Mine um die Bedürfnisse unseres Fürsten kümmerten. Während die Helden schon wieder den Dunklen Magier aus unserer Mine vertreiben mussten, als wären wir Schildzwerge nicht mehr in der Lage, unsere heiligen Hallen zu Beschützen. Es ist eine Schande. Bei Boords Bart, ich mache mir solche Sorgen.“

„Sei beruhigt, dieser Drache kriegt uns schon nicht klein“, grinste Bort schwach, „Und ganz allein waren die Helden ja auch nicht. Manche unseres Volkes haben bei der Befreiung der Mine und nun auch bei der Befreiung der Rietburg geholfen. Wie schon beim ersten Mal, als die Rietburg erobert wurde. Deine Schwester ist dabei. Brolaf ist ausgerückt. Und natürlich der heroische Kram. Ich hoffe, dass es ihnen gut geht. Nein, ich glaube sogar, dass es ihnen gut geht.“

Mart nickte bedrückt und sah alles andere als überzeugt aus. Bort umarmte ihn beruhigend und streichelte Marts Haarschopf. Iril beachteten die beiden nicht weiter. So viel zu den berüchtigten Sicherheitskontrollen der Schildzwerge. Aber angesichts der aktuellen Umstände war das durchaus zu verstehen, dass sie nicht jeden durchreisenden Zwerg filzten.\bigskip







Iril zog es weiter ins Innere der Mine. Mancherorts herrschte Chaos. Zwerge rannten herum, manche kampfbereit, andere im Nachthemd, als hätte sie das Erdbeben direkt aus ihren Schlafgemächern gerissen. Ein übellauniger Zwerg mit rostbraunem Schnurrbart fluchte beim eingestürzten Tiefen Markt über frische Risse an der Stollendecke. Eine stämmige Kriegerin mit einer unpraktisch großen Kampfaxt und golden glänzenden Zwergenstiefeln raste an Iril vorbei, dicht gefolgt von einigen Zwergenkindern, die sie prompt in eine sichere Nische eines Höhlengangs scheuchte. Zwerge rasten hin und her, begutachteten die Schäden des Bebens und stützten Gänge mit komplizierten Metallstützen ab.

Keiner achtete auf Iril.

Einmal trampelte gar eine Gruppe Gors an ihr vorbei, ohne sich um sie zu kümmern, weitergescheucht durch einen Skral mit einer Augenklappe. Iril langte nach ihrem Hammer, doch der Skral warf ihr kaum mehr als einen raschen Blick zu, ehe er etwas in seiner Sprache fluchte und wieder davonsprang. Nichts außer der Gestank nach Blut und Fäulnis verriet, dass sie hier gewesen waren.

Eine elegante Eule überholte Iril auf ihrem Weg tiefer in die Minen. An ihrem Bein hing eine Nachricht. Neue Neuigkeiten der Wächter Bort und Mart? Ob der Drache wohl wirklich gefallen war?

Iril führte noch einige Handvoll getrockneter Algen aus der Zucht Silberhalls in ihrer zum Glück wassersicheren Reisetasche mit sich. Sie pflanzte sich in eine steinerne Nische, belegte ein Stück würziges Silberbrot mit den Algen und bedachte mampfend ihre nächsten Schritte, während sie ihren krampfenden Bauch zu ignorieren versuchte.

Ihr ursprünglicher Plan hatte schlicht darin bestanden, nach Cavern zurückzukehren, den Kontakt mit ihren Freunden oder ihrer Familie wieder aufzunehmen und dann weiterzugucken, was aus ihrem weiteren Leben werden sollte. Viele Geheimnisse der Runen hatte sie im Norden gelüftet und die erworbenen Fähigkeiten in Silberhall bereits zur Genüge einsetzen können. Doch hatte sie es als moralische Pflicht empfunden, nicht nur ihre außergewöhnlichen Kenntnisse zu mehren und die silberzwergischen Kenntnisse über die Runenmagie weiterzubringen. Nun war die Zeit für einen neuen Lebensabschnitt gekommen. Einen Neuanfang. Nun, wo Irils Runenmeisterin verstorben war und ihr ihren unbezahlbaren Runenhammer vermacht hatte. Nun, wo Iril vernommen hatte, wie in Cavern unzählige Kreaturen ihr Unheil suchten und sich ein Dunkler Magier hier versteckt hatte – gar zweimal in dem letzten Halbdutzend Jahren – hatte sie gedacht, dass ihre Hilfe hier nötiger war als im Norden. Diese Meinung hatten nicht viele geteilt. Und so hatte sie sich schweren Herzens von ihren Freunden bei den Silberzwergen getrennt und war auf Lunors Schiff, der BALENA, in den Süden gestochen.

Jetzt, wo sie hier war, schien es gar möglich, dass die Bewohner des Rietlands ihre Hilfe nötiger hatten als die Schildzwerge. Doch noch bestand ihre erste Priorität darin, Bekannte zu finden. Und dies stellte sich als schwerer hinaus als gedacht.

Iril suchte die Behausungen ihrer Familie nahe der Tiefminen auf. Schon dies kostete sie einige Zeit, denn sie war das Orientieren in diesen teils engen, verwinkelten Gängen nicht mehr gewohnt. In Silberhall waren die Gänge breit, rechtwinklig und blankpoliert. Hier in Cavern, insbesondere vor den Tiefminen, waren die Gänge eng, unförmig und von Ruß überzogen. Auch hier wuselte es nur so von Zwergen, die einander mit Reparaturen aushalfen und die wildesten Gerüchte über die Geschehnisse der Außenwelt herumsprachen. Niemand erkannte Iril, ja, niemand schien überhaupt ihren Namen oder die ihrer Familie zu erkennen. Oder sie wollten sich gerade einfach nicht die Mühe machen, darüber nachzudenken, wo es dringlichere Angelegenheiten hab.

Iril bedauerte wieder einmal ihre Entscheidung, mit ihrer Familie zu brechen. Sie hatten einander nicht mehr viel zu sagen gehabt nach ihren vielen Streiten über Silberhall und die Zukunft des Zwergenreichs. Was hatten sie auch so stur sein müssen! Iril war kein Kind mehr gewesen. Sie hatte ein Recht darauf gehabt, ihren eigenen Pfad einzuschlagen. Die Tiefminen hinter sich zu lassen.

Und nun war sie wieder zurück. Iril hatte die Höhle erreicht, in der sie ihre Kindheit lang gelebt hatte. Der Eingang war von einer schweren Steintür verschlossen, die mit massiven Metallbeschlägen verziert war. Dank eines ausgeklügelten Systems konnte man die schwere Tür dennoch ganz sachte beiseiteschieben. Sofern man den richtigen Schlüssel besaß. Iril tat das nicht. Sanft fuhr sie mit ihrer Hand über die rußige Tür und schluckte einen Kloß in ihrer Kehle herunter.

„Verlaufen?“, fragte eine heisere Stimme hinter ihr. Die Stimme gehörte zu einem Zwerg mit angegrautem blondem Haar, der achtsam einen großen Kessel mit irgendeiner köstlich dampfenden Suppe auf den Boden stellte. Er trug einen Schulterpanzer mit einem schwarzen Zwergenseil darüber. Dies wies ihn als Tiefminen-Arbeiter aus. Er war einer der Entbehrlichen, die für die hohen Fürsten Edelsteine aus den feurig heißen Untiefen der Erde bergen „durften“. Kam ja nicht in Frage, dass ein hochwohlgeborener Schildzwerg mit einer langen Ahnenlinie sein Leben in den brüchigen Lebensadern Caverns aufs Spiel setzte.

Solche Ungerechtigkeiten gab es in Silberhall nicht. Zumindest noch nicht. Die Mine war nicht alt genug, als dass sich Zwerge mit langen Ahnenlinien darauf beruhen könnten und verlangen, dass ihre Untertanen ihnen mehr Reichtum verschafften.

„Hast du dich verlaufen?“, wiederholte der Zwerg etwas lauter.

Iril zuckte zusammen. „Verzeiht, ich wollte nicht stören. Wohnt Ihr hier?“

„Nein, nein, wir wohnen zwei Gänge weiter in Richtung Tiefe“, winkte der Zwerg lächelnd ab, „Aber ich kenne die Familie, die hier haust. Und du siehst ein wenig verloren aus, wenn ich das anmerken darf.“

„Dem ist wohl so. Ich bin Iril. Ich liebte hier einmal, Jahrzehnte ist es her.“

„Drak. Freut mich, deine Bekanntschaft zu machen. Wenn du hier wohntest, dann hast du früher einmal in den Tiefminen gearbeitet, nicht wahr?“

„So ist es. Wie wohl alle aus meiner Ahnenlinie.“

„Na, na, wir Zwerge sind auch erst von irgendwoher nach Cavern gezogen. Sag, bist du vor oder nach der Ankunft der Flüchtigen aus Krahd von hier weggezogen?“

„Nachher. Silberhall gab es bei der Ankunft der Andori ja noch gar nicht.“

„Ah, du bist eine derjenigen, die sich von Silberlands Versprechen verlocken ließen?“ Drak verzog sein runzeliges Gesicht kurz.

„Schuldig im Sinne der Anklage. Doch nahm mein Leben in Silberhall eine andere Richtung, als ich ursprünglich gedacht hatte.“

Irils Hand ruhte auf ihrem Runenhammer. Drak zuckte er mit den Schultern und meinte: „Tja, was geschehen ist, ist geschehen. Wenn du vor der Gründung Silberhalls noch hier lebtest, habe ich dich vermutlich doch schon getroffen, auch wenn du dich nicht mehr daran erinnern magst.“

„Ich bin das Kind von Graiah und Perith.“

„Ah! Stimmt, die hatten doch vor langer Zeit eine Tochter, die ausgewandert ist. Warst du das, die uns immer die Stollenwände vollgekritzelt hat?!“ Drak kicherte. Dann wurde er ernst. „Tut mir zutiefst leid, Iril, das mit deiner Familie.“ Drak senkte seinen Blick und stampfte zweimal kurz auf, eine Geste des Respekts.

„Was ist mit meiner Familie?“, fragte Iril argwöhnisch.

„Weißt du das nicht?“

„Ich habe schon seit Jahren nicht mehr von ihnen gehört.“

„Oh. Auweia. Ich bin wohl nicht die beste Person, um dir das zu verraten. Doch ist es oft besser, die schlechten Nachrichten rasch zu überbringen, statt lange um den heißen Brei herumzureden. Willst du, dass ich es dir erzähle?“

Iril nickte, während ein drückender Kloß der Furcht sich in ihrem Innern ausbreitete.

„Na dann ...“, seufzte Drak, „Die Sache mit Iolith hat deine Eltern hart getroffen. Perith wurde vom Fieber der Traurigkeit erwischt und erlag ihm. Graiah mochte sich durchaus gegen den Kummer zu wehren, doch hielt sie es kaum mehr in dieser Höhle aus. Oft zog sie sich von ihren Bekannten zurück und stapfte tagelang ziellos in den verlassenen Gängen umher. Eine Horde Gors überraschte sie. Einen dieser scheußlichen Drachendiener riss sie noch mit ins Reich der Toten. Die restlichen ... nun ... wir fanden nur noch Blutspuren ... und ...“

Drak schüttelte seinen Kopf und schloss: „Ich labere schon wieder zu viel. Es tut mir leid, Iril. Du hast hier keine Familie mehr. Deine Eltern sind beide dahingeschieden. Die Mutter der Erde hat sie zu sich geholt.“ Erneut stampfte Drak zweimal auf.

Iril brachte nicht einmal eine kohärente Antwort zustande. Ihr Herz pochte und sank zugleich in ihre Magengegend, welche wild zu rumoren begann. Ihr wurde übel und schwindelig. Ihre Eltern sollten ihr nicht viel bedeuten, nachdem sie so lange nicht mehr mit ihnen interagiert hatten. Und doch erfüllte sie die Nachricht ihres Todes mit einem Gruseln.

Iril lehnte sich an die Höhlenmauer und versuchte, das Gehörte zu verarbeiten. Gefühle wallten in ihr auf, die sie nicht richtig zuzuordnen vermochte. Bilder von Graiah und Perith, einander anschreiend. Einander umarmend. Sie ins Bettlein bringend und sanft zudeckend. Auch sie hatte der Tod sich geholt. Zuerst Burmrit, und nun ihre Familie. Iril ließ einige Tränen fließen. Nichtsdestotrotz sprach sie fest:

„Was ... was war diese Sache mit Iolith, die alles auslöste? Wer oder was ist Iolith?“

Drak machte große Augen und schluckte.

„Oh ... öhm ... wie lange warst du denn nicht mehr in Kontakt mit Cavern? Iolith ist ... deine Schwester.“

Das war zu viel für Iril. Zum Glück lehnte sie bereits an der Stollenwand, denn jetzt musste sie sich erst einmal hinsetzen und tief durchatmen.\bigskip







Drak war so lieb, die überwältigte Iril zum Abendessen einzuladen. Er bot ihr sogar an, sie am nächsten Tag zum Grab ihrer Eltern zu führen und ihr mehr von Iolith zu erzählen. Aber zunächst einmal knurrte sein Magen. Und er wollte mehr über die wilden Neuigkeiten aus dem Land erfahren. Auf zu seiner Familie.

Darks Wohnraum war wie diejenigen der meisten Tiefminen-Arbeiter grob in die Wand des eines breiten Querstollens gehauen. Seine Familie hatte daraus ein wirklich heimeliges Reich gemacht. Zahlreiche schmucke Laternen hingen von der Decke und erhellten auch die letzte Ecke des Raumes. Grünlich schimmernde Pflanzen waren in eigens dafür geschaffenen Löchern in den Wänden eingelassen und präsentierten ihre vielfarbigen Blüten. Elegante Wandteppiche schmückten den blanken Stein der Wände. Sie zeigten Menschen und Zwerge im und unter dem Gebirge, beim Fördern von Edelsteinen und Zurücktreiben von Kreaturen. Ein Teppich zeigte gar einen schwarzen Drachen, der von einem hochgewachsenen Helden mit einem blauen Schild konfrontiert wurde. Die sternförmigen Runen auf dem Schild kennzeichnete diesen eindeutig als den Sternenschild, den ersten der vier mächtigen Schilde aus uralter Zeit. Warum Drak wohl einen Menschen auf seinem Wandteppich abbildete?

Iril konnte sich nicht lange auf die Verzierung des Wohnraums konzentrieren, ehe sie auch schon zu Tisch gerufen wurde.

Draks Familie war außergewöhnlich groß, besonders für Zwerge. Zehn Kinder drängelten sich neben Iril, Drak und dessen Gemahlin Bairen an einem überlangen Esstisch, der nahtlos in den Steinboden überging. Wobei der Begriff „Kinder“ irreführend sein könnte. Viele der anwesenden Zwerge waren bereits lange erwachen. Groß, stark und rußverschmiert von der Arbeit in den Tiefminen. Einige sahen auch etwas herausgeputzter aus und legten klappernde Rüstungen ab, ehe sie sich zu Tisch begaben. Nur ein, zwei Anwesende waren wahrlich noch Kinder. Der jüngste musste gar noch mit Lätzchen essen und verschmierte vergnügt sowohl seinen eigenen kleinen Bart als auch das Gesicht einer älteren Schwester, die neben ihm saß, mit dem Essen, das eigentlich in seinen Magen gehörte.

Zu essen gab es den köstlichsten Waldpilz-Eintopf, den Iril je die Ehre hatte zu verspeisen. Doch konnte sie sich irgendwie nicht darauf konzentrieren. Zu sehr kehrten ihre Gedanken immer wieder zu dem zurück, was Drak ihr erzählt hatte. Wie ein Geschwür schlichen sich die Gedanken zu ihrer verlassenen Familie in ihren Geist und hinderten sie daran, das geniale Gericht zur Gänze zu genießen.

Sie hatte eine Schwester. Iolith hieß sie. Geboren war sie offenbar kurz, nachdem Iril nach Silberhall gezogen war. Niemand wusste, wo sie sich im Moment aufhielt. Iolith hatte sich ähnlich wie Iril mit ihren Eltern zerstritten und ihren eigenen Weg gesucht. Scheinbar hatte sie sich im Grauen Gebirge in eine Agren verliebt, welche die Schildzwerge und ihre unterirdischen Grabungen als Schandtaten an der Natur ansah. Ganz ungeachtet dessen, wie sehr die Zwerge bei der Erschaffung ihrer berühmten Bauten stets bemüht waren, im Einklang mit den natürlichen Felsformationen zu sein, wie zu Stein gewordene Echos der Natur.

Graiah und Perith hatten ihre zweite Tochter nicht auch noch verlieren wollen, und leider dadurch umso mehr davongetrieben. Eines Tages war an Ioliths Stelle ein Abschiedsbrief auf ihrem Kopfkissen gelegen. Keiner wusste genau, wo sie nun war. Keiner wollte es mehr so genau wissen.

Ein weiteres Geheimnis, das Iril eines Tages ergründen würde.

Sie liebte schließlich Geheimnisse.\bigskip







Mitten ins Abendessen platzte eine aufgeregte Zwergin, welche ein aufgerissenes Stück Pergament schwang. Sie blieb verheißungsvoll grinsend im Eingang stehen.

„Habt ihr es schon gehört?“, ertönte ihre schallende Stimme. Alle Augen richteten sich auf sie.

„Spann uns nicht auf die Folter, Marun!“, rief einer der vielen Brüder und schlabberte seine Schüssel Waldpilzeintopf mit einem Schluck leer.

„Ein Goldstück darauf, dass der Drache tot ist!“, warf eine Schwester ein.

„Wie geht es Kram?“, fragte Drak angespannt. „Wie steht es um meinen Jungen?“

Iril hatte den Namen Kram schon gehört und konnte ihn nach einiger Überlegung einordnen. Ein Zwerg aus den Tiefminen und Held von Andor. Einer derjenigen, die den Dunklen Magier Varkur in den Tiefen der Mine aufgestöbert hatten, und somit einer der Gründe, weswegen Iril überhaupt erst hierher aufgebrochen war. Doch erst jetzt machte sie die langsam die Verbindung zwischen dieser Familie und Kram. Welch Zufall, dass sie ausgerechnet hier gelandet war. Oder war es etwa mehr als das? Konnte es sein, dass dieselbe Hilfsbereitschaft, mit der Draks Familie Iril zu sich eingeladen hatte, auch dieselbe war, weswegen Kram einer der wenigen Schildzwerge war, die sich um das Schicksal der Menschen kümmerten? Dank der er überhaupt zu einem Helden von Andor geworden war?

Einen Augenblick lang herrschte noch Stille. Dann wurden Irils Gedanken unterbrochen, als Marun freudig herausplatzte: „Der Drache ist tot! Kram hat ihn erlegt! Und gleich noch den verschleppten Prinzen gerettet! Heldenhaft. Es geht ihm gut. Sie feiern draußen im Rietland.“

Einen weiteren Augenblick lang herrschte wieder Stille. Dann brach der gesamte Tisch in Jubel aus. Die Anwesenden klopften einander auf den Rücken, jubelten, und prosteten einander mit Rachenputzer zu.

Schon leicht angetrunken erhob sich Draks Gemahlin Bairen und rief: „Ein Hoch auf unseren Sohn! Auf dass er bald wieder gesund zu uns zurückkehrt und uns von dieser Heldentat berichten kann!“. Marun rannte weiter, wohl, um die nächste Wohnung über die fröhliche Neuigkeit zu informieren. Drak rannte indes in den Lagerraum und holte ein Holzfass heraus, welches mit drei X angeschrieben war.

„Drachenfass Rachenputzer. Enorm starker Zwergenschnaps. Das trinken wir sonst nur zu Geburten und Beerdigungen. Doch welch bessere Gelegenheit gäbe es, das hier anzuzapfen, als einen Sieg über einen verfluchten Drachen zu feiern! Möge er der letzte gewesen sein!“

Erneut ertönte allgemeiner Jubel. Iril stimmte mit ein. Doch fühlte sie auch ein Stechen in ihrem Herzen. So fühlte es sich in einer Familie an. Und eine Familie hatte sie hier nicht mehr. Cavern war nicht mehr ihr Zuhause.






\newpage
\section{Die Spur der Drachenkultisten}

Nach dem feierlichen Abendessen verteilten sich die Anwesenden wieder auf die umliegenden Gänge, um sich weiter umzusehen, wo Hilfe vonnöten war. Glücklicherweise hatte das Erdbeben keinen direkten Stolleneinsturz verursacht. Die wenigen kritischen Ritze wurden von Baumeistern rasch gestützt und geflickt. Verletzte gab es kaum welche. Die ersten Invaliden von der Befreiung der Rietburg und dem Kampf gegen den Drachen würden erst in einigen Tagen hier eintreffen. Und so nutzten die Schildzwerge ihre Zeit zum vollen aus.

Ein unbeschreibliches Fest folgte auf den Sieg über den Drachen. Erwachsene tanzten wild in den Stollen und die Kinder lachten und sangen. Die letzten Tage hatte sie alle viel gekostet. Doch nun wurde gefeiert!

Met wurde ausgeschenkt, guter Kuchen aus den Vorratsstollen geholt, Festbänke aufgestellt und komplexe alchemistische Funkenkörper durch den Schlauchgang Barathrum in die Höhe geschossen, wo sie in verschiedenste farbenfrohe Muster explodierten.

Iril bedanke sich ausgiebig bei Draks Familie für deren Gastfreundschaft und zog weiter. Sie glaubte inzwischen kaum mehr daran, sich wieder mit alten Freunden aus längst vergangenen Zeiten auszutauschen.

So zog es sie an die verlasseneren Orte Caverns. Sie hatte Feste noch nie die famoseste Freizeitbeschäftigung gefunden.

Neugierde war aber ihr Ding. Und als sich eine in einen dunklen Mantel gekleidete Zwergin geheimnisvoll an der Ecke vorbeischlich, in welche Iril sich verkrochen hatte, keimte Irils unbändigbare Neugierde wieder auf. Sie liebte schließlich Geheimnisse.

Die Zwergin mit dem dunklen Mantel blickte sich mehrmals um, als ob sie fürchtete, beobachtet zu werden. In ihrer Hand hielt sie einen rötlichen Edelstein, um den jemand ein kompliziertes metallenes Gerüst geformt hatte. Solche Schätze sah man in den Tiefminen selten, und noch seltener wurden sie über legale Händel erworben. Eine Diebin, die die Gunst der Stunde und das Chaos der Feierlichkeiten ausnutzte? Iril sah ein Glimmen in den dunklen Augen der Fremden und kniff schnell die eigenen zusammen. Doch schien die mysteriöse Zwergin Iril im Schatten nicht zu erkennen und schlich weiter den Gang entlang.

Kurz debattierte Iril die Moral davon, jemandem zu folgen, der offensichtlich nicht verfolgt werden wollte. Die Chance, dass es hier ein Unrecht aufzudecken gab. Die potenziellen Gefahren, die am Ende des Gangs auf sie lauern könnten. Und das unwiderstehliche Drängen ihrer Neugierde, die kein Geheimnis unerforscht lassen wollen.

Rasch war Iril auf den Beinen, zückte ihren – aktuell nicht leuchtenden – Runenhammer und huschte der mysteriösen Zwergin leise hinterher. Hinter ihr donnerten die Festmusik der Zwerge, der Klang dutzender blecherner Blasinstrumente dutzendfach verstärkt durch das Echo der dünnen Gänge. Allzu große Sorgen, von der mysteriösen Kuttenträgerin gehört zu werden, musste sie sich nicht machen.

He, diese Gänge kannte sie ja, die mündeten wieder in die nördlichen Oberhallen Caverns!

Die Zwergin im dunklen Mantel eilte an der prunkvollen Halle der vier Schilde vorbei und zwischen zwei prunkvollen Säulen hindurch, bis sie vor einer protzigen, mit Blattgold verkleideten Tür stehen blieb, durch die selbst ein Troll gepasst hätte. Sie läutete an einem goldenen Glöckchen und wartete, bis die Tür sich, von unsichtbaren Mechanismen gezogen, von selbst öffnete.

Rhythmisches Hämmern erklang dahinter, und rötliche Schimmer gleißten hervor. Rauch lag in der Luft.

Nach einem letzten Umschauen, bei dem der geheimnisvollen Zwergin wieder einmal die sich in die Schatten Caverns drückende Gwundernase Iril entging, verschwand sie hinter der Tür. Hinter ihr verschloss sich das gewaltige Tor wieder wie von selbst. Ein darüber hängender golden verzierter Hammer verriet Iril, was für ein Raum dahinter lag.

Dies war die Schmiede von Hildorf, dem Schmiedemeister, welcher – ganz bescheiden – seine Arbeitsstelle direkt neben der legendären Halle der vier Schilde eingerichtet hatte. Seine Spezialität war die Fertigung magischer Artefakte. Magischer Artefakte wie dem Drachenrelikt, das diese mysteriöse Zwergin in den Händen gehalten hatte. Artefakte aus Roteisen, dem Blutstein-Erz, welches seit Jahrhunderten vom Blut der Drachen durchtränkt war.

Die Macht dieses roten Metalls im Stein war ein Geheimnis. Ein Geheimnis der Tiefminen. Wie alle Geheimnisse, von denen sie gehört hatte, hatte Iril auch dieses einst zu knacken versucht. Erfolglos. Selbst in den Tiefminen beherrschten nur wenige Zwerge dieses Metall. Der Roteisenstein, oder auch Blutstein, wie ihn manche nannten, war von dem roten Erz durchsetzt und wurde von den Schildzwergen oft als Baustoff für das Errichten von Sockeln, Mauern und Türmen verwendet. Doch nicht viele vermochten das Roteisen aus dem Stein herauszulösen und in Form zu gießen. Wenige wussten überhaupt, woher der Roteisenstein seinen Namen hatte. Und so gut wie niemand wusste, wie man damit magische Artefakte schaffen konnte. Hildorf behauptete gerne, dass er der Einzige sei, der diese Drachenmagie handhaben konnte. Es stimmte zumindest, dass kaum jemand seiner Schmiedekunst das Wasser reichen konnte. Nicht, seitdem der legendäre Schmiedemeister Xoll Hammeraxt, dessen Ruf sogar zu Iril nach Silberhall gedrungen war, das Zeitliche gesegnet hatte.

Ein wütender Aufruf erklang aus dem Innern von Hildorfs Schmiede.

Iril konnte sich nicht davon abhalten, näherzutreten und zu lauschen.

„Der Drache ist tot?!“ Eine tiefe Stimme, vermutlich Hildorf selbst. Wut und Unglauben schwang in seinem Tonfall mit.

„Die Nachricht war eindeutig. Tarok der Gewaltige ist gefallen“, antwortete eine zweite Stimme, vermutlich die der mysteriösen Zwergin.

„Wer hat es gewagt?!“

„Die Helden von Andor, so sagt man.“

„Welch Unrecht! Tarok der Gewaltige wollte doch nur ...“

„Egal. Wir müssen rasch handeln! Ehe sie seinen Leichnam schänden können.“

„Natürlich. Was braucht Ihr, Serafimma? Weitere Drachenrelikte, um den Uralten mehr Beseelte zu verschaffen? Mehr Geisterfeuerzeuge für die Dunklen Messen? Zusätzliche Waffen für den kommenden Konflikt? Rüstungen, die Feuer und Pfeil gleichermaßen fernhalten?“

„Nichts von alledem, Hildorf. Wir brauchen dich. Die Drachenkultisten scharen so viele Anhänger wie möglich aus und ziehen in den Westen, zum Leichnam. Mit genügend Anhängern sind wir eine bedrohliche Präsenz. Dort nützt du uns mehr, als wenn du dich wieder tagelang hier einpferchst und weitere Artefakte herstellst, die wir – seien wir ehrlich – nicht mehr wirklich nötig haben.“

Nun klang Hildorf widerspenstig. „Ich bin doch nur ein Schmied. Ein guter zweifelsohne, doch nicht mehr. Ich kann keinen Kriegshammer führen.“

„Mit etwas Glück wirst du das auch nicht müssen. Wir wollen nur genügend viele sein, dass man uns nicht ignorieren kann. Los jetzt, pack‘ deine Sachen.“

„Aber ... meine Gehilfin ...“

„Branna wird einige Tage lang ohne dich auskommen können. Auf, mein Freund, auf mit dir! Schamanin Sagramak vom Barbaren-Stamm der Jpaxo wartet bereits vor dem südlichen Tor. Als von Taroks nächstem Verwandten Beseelte steht es ihr zu, das Ritual zu vollziehen.“

„Ich weiß, ich weiß. Du vergisst, dass ich es war, der überhaupt erst das Drachenrelikt schuf, welches nun von Sagraks Seele erfüllt ist.“

„Na bitte! Die Drachen verdanken dir bereits so viel. Da ist es doch ein Klacks, für sie einige Schritte ins Rietland zu tun.“

Hildorf grummelte etwas, das nach verdächtig wie „Einige würden sagen, mein Dienst an die Flammen wäre bereits erfüllt“ klang.

„Was war das?!“, antwortete die Zwergin – Serafimma – schnippisch, „Wenn du willst, kannst du es ja mit den Drachen aushandeln, wenn sie hinter der Pforte des Todes über dich urteilen. Ich weiß, was ich an deiner Stelle tun würde.“

Hildorf grummelte weiter vor sich hin. Den Geräuschen nach machte er sich aber auch zeitgleich daran, seine Sachen zu packen.

Iril duckte sich in den Schatten eines Erkers im Gang, während Serafimma die Schmiede verließ und erneut an ihr vorbeischlich, diesmal in Begleitung von Hildorf.

Der Meisterschmied trug einen gewaltigen Rucksack, an dem einige grünlich schimmernde Tränke klirrten. Anstelle einer linken Hand befand sich dort eine eiserne Prothese, an der eine eckige Laterne befestigt war. In seinen für zwergische Verhältnisse äußerst kurzen, hellbraunen Bart waren Dutzende kleine und zweifellos unverschämt kostbare Edelsteine eingeflochten.

Irils Gedanken rasten. Drachenkultisten! Anhänger der kriegerischen Bestien aus den Urzeiten. Hier, mitten im Herzen von Cavern! Und Schmiedemeister Hildorf war einer von ihnen!

Iril hatte Gerüchte über die Verbreitung dieses Kults von Drachenanbetern gehört, diese jedoch nicht glauben wollen. Hatten sie denn alle die grausamen Taten der Drachen während des Unterirdischen Kriegs vergessen?!

Ihre Gedanken ratterten. Was sollte sie tun? Gab es Obrigkeiten, die man in solchen Fällen informieren sollte? Nein, leben und glauben sollte ja jeder Zwerg, was er wollte, solange er damit niemandem schadete. Und die Drachenkultisten schadeten niemandem mit ihrer Existenz. Doch schienen sie irgendetwas mit Taroks Leichnam zu planen. Iril musste mehr herausfinden.

Sie schlich Hildorf und Serafimma nach. Inzwischen waren die beiden schon so weit fortgeschritten, dass Iril ihnen nicht mehr strikt folgen konnte. Doch hatte sie belauscht, dass eine Schamanin der Barbaren vor dem südlichen Ausgang wartete. Und selbst so betrunken, wie die meisten restlichen Zwerge waren, konnte Iril hin und wieder jemandem finden, der ihr den schnellsten Weg zum südlichen Ausgang weisen konnte.\bigskip







Bis Iril den südlichen Ausgang Caverns erreichte, hatten die meisten Feierlichkeiten längst ein Ende gefunden.

Am Minenportal stieß Iril auf zwei Zwergenwachen, welche mürrisch aufs Rietland blickten. Es herrschte Nacht. Der Mond war nicht zu erkennen, musste er sich doch gerade auf der anderen Seite des Erdenkörpers aufhalten. Doch nicht nur die Fackeln aus der Tiefe und die Sterne am Himmel spendeten Licht. Weit draußen im Rietland auf der anderen Seite der Narne leuchteten weitere Fackeln und Lagerfeuer. Feierten die Andori?

Ein weiteres Licht fiel Iril auf. Ein bisschen weiter inland, doch noch immer vor der Marktbrücke, beleuchtete ein flackerndes Lagerfeuer eine mächtige einsame Eiche. Die legendäre Zwergeneiche von Fürstin Brala.

Iril hinterließ bei den Wachen am südlichen Ausgang eine knappe Nachricht an Draks Familie, in der sie sie über die Lage aufklärte. Zudem kaufte sie sich einen Falken.

Dann machte Iril sich auf in die Dunkelheit und auf zur Zwergeneiche, wo sie Hildorf und die Drachenkultisten vermutete. In genügender Entfernung blieb sie hinter einem Hügel liegen und spähte zur Zwergeneiche.

Wie erwartet, erkannte sie einige aufgespannte Zelte und einige finstere Gestalten um ein Lagerfeuer sitzen. Zwerge und Menschen und sogar einen Skral. Am Rande der Gruppe, nur noch schwach beleuchtet, saß ein gewaltiger rötlicher Raubvogel zusammengesunken auf einem Stein, den Kopf unter einen riesigen Flügel gesteckt. Hildorf, den Schmiedemeister, erkannte Iril am genau anderen Ende des Lagerfeuers, so weit wie möglich vom Raubvogel entfernt, wie er am Boden saß und mit Serafimma ein Kartenspiel spielte. Die meisten anderen sichtbaren Anwesenden schliefen auf behelfsmäßigen Betten. In den Zelten hielten sich vermutlich noch mehr Drachenkultisten auf.

Das mussten über ein Dutzend Personen sein, vielleicht sogar zwei. Iril drehte sich der Magen um. Was hatten sie vor?! Sie sandte ihren braven Falken mit einer entsprechenden Notiz zurück an die Wachen am südlichen Eingang – wohl wissend, dass diese auch kaum etwas damit anfangen könnten – und machte sich daran, sich ein provisorisches Nachtlager zu bereiten.

Und Iril fiel in einen erschöpften, traumlosen Schlaf.\bigskip







Iril wurde von einem kurzen Schrei geweckt. Hastig richtete sie sich auf, während die Erinnerungen an die vergangenen Tage langsam auf sie einprasselten. Ihr Falke war zurückgekehrt und ruhte neben ihrem dürftigen Kissen, auf einem Bein, den Kopf unnatürlich verdreht und in sein Gefieder gesteckt. Dem Falken schien dies nichts auszumachen.

Der Himmel über ihr erglühte in rotem Licht, als wäre das Blut des gefallenen Drachen in ihn aufgestiegen und hätte dem Licht der aufgehenden Sonne gefärbt.

Mondbeeren glitzerten im vom Tau nassen Rietgras. Der Sonnenaufgang konnte noch nicht lange her sein.

Vorsichtig blinzelte Iril über die Kuppe des Hügels, hinter dem sie geruht hatte. Vor der Zwergeneiche war immer noch eine Horde an Drachenkultisten versammelt. Die meisten Zelte waren bereits abgebaut. Das Lagerfeuer war erloschen. Hastig huschten die Anwesenden umher und machten sich reisefertig. Der Riesenvogel von gestern war nirgends zu sehen.

Dafür erklang ein weiterer kurzer Schrei. Und diesmal fand Iril die Urheberin.

Am Rande des Lagers der Drachenkultisten, im Schatten der Zwergeneiche, saß eine Frau im Schneidersitz. Sie trug eine elegante silberne Rüstung mit vielen Verzierungen. Neben ihr lag eine lange Stangenwaffe mit einer wie ein Schneckenhaus verdrehten Spitze. Wirkte eher zeremoniell als kampftauglich. Und dann erst der prunkvolle silberne Helm, der neben der Frau lag und scheinbar metallene Flügel als Verzierung trug. Nicht zwingend zweckdienlich, doch zweifelsohne imposant.

Iril musterte das ferne Gesicht der Kultistin. Deren Augen waren geschlossen. Ihre Nase sah schief aus, als wäre sie gebrochen. Ihr Mund öffnete und schloss sich, ohne Worte zu bilden.

Da! Erneut schrie sie kurz auf. Diesmal klang es weniger überrascht und mehr verärgert.

Zwei weitere Kultisten saßen neben ihr, ein Mann in einem dunkelblauen Gewand und ein Wesen in einem dunklen Kapuzenmantel, das nicht näher erkennbar war, doch der Form der Kapuze nach lange hohe Hörner trug. Es redete leise auf die Kultistin ein. Iril spitzte ihre Ohren, vernahm jedoch bloß Wortfetzen:

„Welche Wahrnehmung ..., Sagramak? ... sich der Zustand ... Tarok?“

Die Kultistin, Sagramak, schlug flatternd ihre Augen auf und rief:

„Es ist der Prinz! Der elende Prinz beansprucht den Leichnam für sich! Seine Garde umkreist ihn und lässt niemanden hin. Sie wollen ihn zerteilen! Rasch, rasch, macht euch zum Aufbruch bereit!“

Ihren Worten wurde rasch Folge geleistet. Iril packte ebenfalls ihr Schlafzeug zusammen und erledigte ihr morgendliches Geschäft, während sie über die Worte der Drachenkultistin nachsinnierte. Als sie wieder über den Hügel blickte, war der Tross der Drachenanbeter bereits aufgebrochen und daran, die Marktbrücke ins westliche Rietland zu überqueren.

Iril eilte ihnen hinterher. Einmal verharrte sie ängstlich, als sie sah, wie in der Ferne der Riesenvogel von gestern vom Himmel stürzte. Gute Güte, das war ein waschechter Krark! Rotbraune, rasiermesserscharfe Federn. Ein gewaltiger gebogener Schnabel, der selbst in Felsen hineinhacken konnte. Unterarmlange, gebogene Krallen, fast so scharf wie die Schneide eines Schwerts. Iril schauderte es.

Der Krark ließ sich vor Schamanin Sagramak nieder und biss ihr spielerisch in die Nase, während sie seinen Hals streichelte. Hatte sie etwa so ihre Nase vernarbt? Nun, der Krark könnte erklären, wie Sagramak vorhin Geschehnisse im Rietland beobachtet hatte. Konnte sie vielleicht mit diesem Krark eine geistige Verbindung aufbauen und durch seine Augen sehen?

Der Tross der Kultisten reiste tiefer ins Rietland. Nach einer kurzen Wartezeit folgte Iril ihnen verdeckt. Der Krark schien sich nicht groß um seine Umgebung (oder etwaige Verfolger) zu kümmern, sondern stolzierte stolz an Sagramaks Seite. Das sah sehr seltsam aus. Aber Iril hatte es lieber so, als dass er am Himmel seine Runden gedreht und sie möglicherweise erspäht hätte. Mit diesen Bestien war nicht zu spaßen, falls sie der Hunger überkam.

Außerdem erlaubte dies Iril, ihren Falken sorglos erneut mit einer Nachricht an die Schildzwerge zurückzuschicken.

Von der Marktbrücke aus erspähte Iril bereits den gewaltigen schwarzen Drachenkadaver Taroks. Der Drache lag zusammengesunken neben den wieder eingestürzten alten Wehrturm, sein langer Hals verbogen, die durchlöcherten Flügel zerknittert, die Schnauze tief in der matschigen Erde vergraben. Offenbar war der Alte Wehrturm im Todeskampf der Riesenechse schon wieder eingestürzt. Das wievielte Mal war das nun schon, dass der Turm von Grund auf neu hatte erbaut werden müssen? Unglaublich, als läge ein Fluch darauf.

Doch nicht nur der Alte Wehrturm hatte unter den Angriffen von Drachen und Kreaturen der letzten Woche gelitten. Das ganze Rietland wirkte verwüstet.

Brandspuren durchzogen das goldene Rietgras. Letzte Rauchwolken hingen tief über dem Land. Bauernhäuser waren eingestürzt. Der Freie Markt, eine Ansammlung diverser Stände verschiedenster Händler nicht unweit des alten Wehrturms, lag in Trümmern. Die Händler waren geflohen. Dies war nicht nur das Werk des Drachen, sondern auch das seiner dunklen Kreaturen. Sie hatten wirklich alles gegeben für diesen Angriff auf das Rietland.

Iril erkannte einige grün gewandte Soldaten der Andori in der Ferne. Einige trugen Flaggen, auf denen das Wappen Andors wehte. Eine Sternblume auf leuchtend rotem Grund. Dies war die legendäre Rietgarde Andors.

Doch machte sich die Rietgarde nicht daran, die Schäden zu beheben. Nein, sie standen rund um den Leichnam der Riesenechse und richteten lange Speere drohend in Richtung der ankommenden Drachenkultisten.

In ihrer Mitte saß ein hünenhafter Mann auf einem mächtigen Rappen und fuchtelte mit seinem Schwert herum.

Es war, wie Sagramak es gesehen hatte. Prinz Thorald von Andor beanspruchte den Leichnam Taroks für sich und hinderte die Drachenkultisten daran, ihn zu erreichen.

Beim Näherkommen erkannte Iril tiefe Furchen in der Flanke des Leichnams. Offenbar hatte Sagramak auch damit Recht gehabt, dass die Ritter des Königs den Drachen zu zerteilen begonnen hatten. Damit hatten sie aber aufgehört. Nun war ihre Priorität, diesen Zwist zwischen Drachenanbetern und dem Prinzen zu ihren Gunsten aufzulösen.\bigskip







„Zurück! Zurück, befehle ich euch! Ich, euer Prinz!“, rief ein offensichtlich aufgewühlter Prinz Thorald. Die nähergetretene Menge an Drachenkultisten folgte nur widerwillig und breitete sich vor Taroks Leichnam aus, während die Rietgarde sie weiterhin mit Speeren auf Abstand hielten. Es waren augenscheinlich mehr Kultisten als Soldaten anwesend, doch waren die Soldaten besser bewaffnet. Die meisten Kultisten wirkten kaum kampftauglich, eher wie eine zusammengewürfelte Sammlung von Bauern und Jägern, welche nervös und wütend blickten.

Rufe von beiden Seiten wurden laut.

„Mörder!“

„Weichet zurück!“

„Lasst den Körper sein!“

„Lasst doch selbst den Körper sein!“

„Was wollt ihr mit ihm?“

„Zurück, sagte ich!“

Armond, Anführer der Rietgarde, schrie nicht. Er trat einige Schritte vor seine restlichen Krieger und legte demonstrativ seinen Speer nieder. Die Menschen direkt vor ihm blickten nervös zu ihm auf. Armond erhob seine salbungsvolle Stimme über die vor ihm versammelte Menge.

„Gibt es jemanden, der Euch anführt? Jemanden, mit dem man verhandeln könnte?“

„Sehen sie so aus, als wollten sie verhandeln?!“, zischte ihm Prinz Thorald ängstlich zu.

„Mit Verlaub, was für eine andere Möglichkeit gibt es, eine Eskalation zu verhindern?“, murrte ein junger Gardist.

„So einige, Malin, so einige“, sagte Armond, „Doch keine ehrenvollen. Wir sind die Krieger des Königs. Uns bindet ein Eid, für Frieden und Freiheit einzustehen. Wer wären wir, wenn wir ihn so leichtfertig brechen würden?“

„Ich bin eine Anführerin mit Verhandlungsmacht“, meldete eine volltönende Stimme aus der Menge. Die Kultisten wichen zur Seite und machten Platz für ... „Ich bin Sagramak, eine Schamanin der Jpaxo. Ich spreche für die hier Anwesenden. Ebenso wie für die Seelen der Drachen. Ich bin beseelt vom Drachen Sagrak, und als Taroks nächstem Nachkommen steht es ihm zu, über seinen Leichnam zu urteilen. Und damit mir.“

Sagramak trat stolz zu Armond, ihre elegante Rüstung mit jedem Schritt scheppernd. Auf dem Kopf trug sie ihren imposanten Drachenhelm. An ihrer Seite hielt sie ihre lange Stangenwaffe, die sie analog zu Armond demonstrativ zu Boden legte. Als sie sich dem Anführer der Rietgarde näherte, öffnete ihr gewaltiger Krark in der Menge hinter ihr kurz demonstrativ seine Flügel und krächzte in den Himmel. Thorald erbleichte.

Sagramak wandte sich ihm zu: „Mit welchem Recht beansprucht Ihr Mörder den Leichnam des letzten Drachen für euch? Nur, weil ihr der Nachkomme desjenigen seid, der sich vor einigen Jahrzehnten stolz selbst zum König von Andor erklärte?“

„Wagt es nicht, von meinem Vater zu lästern“, zischte Thorald, „Er wurde fair gekrönt und ihr Flussländler freutet euch darüber, dass euch jemand vor den wilden Trollen beschützte!“

„Ich stamme nicht aus den Flusslanden, sondern aus den Gebirgsausläufern“, korrigierte Sagramak, „Und wir hatten keinerlei Probleme mit den Kreaturen des Drachen, bis euer Vater dessen Zorn auf sich zog und sein Wille sie zum Angriff eurer Lager stürmte. Unsere Verbindung zu ihm ist größer als eure. Wir haben ein Anrecht auf seinen Körper.“

Thorald wollte etwas entgegnen, wurde jedoch von Armond zurückgehalten.

Waffenlos traten Armond und Sagramak einander gegenüber, während die Kultisten und Krieger einander weiterhin drohend anblickten. Nicht alle natürlich. Iril sah, wie der junge Mann, der vorhin gesprochen hatte – Malin hieß er – seinen Speer demonstrativ locker an seiner Seite hielt, statt damit in Richtung der Kultisten zu fuchteln. Und in der Menge der Kultisten gab es so einige, darunter den Schmiedemeister Hildorf, die dreinblickten, als würden sie am liebsten im Boden versinken. Keinerlei Kreaturen. Wohin die sich wohl verzogen hatten?

Iril glaubte für einen Augenblick, in der Menge der Drachenkultisten auch eine dunkel gewandte Gestalt mit einer seltsamen Maske zu sehen. Der mysteriöse Drachenreiter mit dem langen Schwert? Hatte er überlebt? Doch als sie wieder blinzelte, war die Gestalt verschwunden.

Sie richtete ihre Aufmerksamkeit wieder aufs Geschehen vorne.

Iril vernahm nicht, was Armond und Sagramak leise miteinander diskutierten, doch verstand sie den grundlegenden Konflikt. Die Drachenkultisten wollten den Leichnam des Drachen, wohl aus religiösen Gründen. Und die Andori wollten ihnen den Drachen nicht überlassen. Nicht, dass der Leichnam ihr Eigentum gewesen wäre oder dass sie grundsätzlich verhandlungsunwillig gegenüber ihnen nicht freundlich gesinnten Völkern wären. Doch waren frisches Drachenblut und Drachenknochen äußerst potente magische Mittel. Und Tarok war ein äußerst mächtiger Drache gewesen, vielleicht der größte und mächtigste von allen. Es war nicht auszudenken, was für Schandtaten ein finsterer Magier damit anrichten könnte. Wer konnte schon garantieren, dass sich unter den Drachenkultisten kein finsterer Magier befand?

Iril konnte den Hass der Andori auf den Drachen gut verstehen. Man sollte seine Schuppen und Zähne zu Staub zermahlen, sein Fleisch verbrennen und seine Knochen ins Meer werfen. Seine Überreste vernichten und verstreuen, damit nie wieder jemand von diesem Dämon heimgesucht werden konnte.

Doch konnte Iril auch verstehen, dass die Drachenkultisten da andere Präferenzen hatten.

Iril sah einen andorischen Ritter heimlichtuerisch einen Nachrichtenfalken lossenden, während Armond und Sagramak weiterverhandelten.

Die Andori schindeten Zeit. Doch wohin wurde der Falke gesandt? könnte kommen, um diesen Konflikt noch zu ihren Gunsten zu drehen?

Eine ganze Menge Personen, wie sich herausstellte.

Als erstes tauchte ein Paar auf, das auf einem großen braunen Pferd mit einem weißen Fleck auf der Schnauze angeritten kam. Die hinten sitzende Person, eine grün gewandte Bogenschützin, sprang bereits elegant vom Reittier, während die vorne sitzende Person, ein blau gewandter Krieger, das Pferd vorsichtig drosselte.

„Was geht hier vor sich?“, verlangte die Bogenschützin zu wissen. Entschlossenheit blitzte aus ihren dunklen grünen Augen auf.

Die beiden Neuankömmlinge traten zu Armond und ließen sich von ihm die Lage erklären. Iril erkannte, dass ihre beiden Umhänge vorne von einer Sternblumen-Brosche zusammengehalten wurden. Das Zeichen des Königshauses Brandur. Dies waren Helden von Andor. Diejenigen, die den Drachen geschlagen hatten.

„Vielleicht habt ihr unsere Namen schon gehört. Ich bin Chada. Das ist Thorn“, rief die junge Bogenschützin nun an die Menge, „Wir stellen uns grundsätzlich gegen alle, die die Freiheit anderer berauben wollen. Doch ist es auch so, dass gewisse Geheimnisse zu groß und zu gefährlich sind, um sie allen zugänglich zu machen. Die Bewahrer wissen das. Und ich weiß, dass Ihr das wisst. Im Namen des Friedens muss ich Euch alle bitten, zurückzutreten. Der Körper dieses Drachen ist ein mächtiges magisches Mittel. Er soll die Narne hinuntergeschwemmt werden und sich im Hadrischen Meer verteilen, wo niemand ihn für finstere Zwecke missbrauchen kann. Selbst er hat verdient, das Ewige Glück zu erreichen.“

„Bockmist!“, rief eine Frau mit einer Zipfelmütze aus der Menge der Kultisten hervor, „Taroks Körper steht uns zu. Ihr, die ihr schon den letzten Avatar der Götter ermordet habt, wollt uns nun nicht einmal die Gnade seiner Bestattung gewährend!“

Chada runzelte ihre Stirn. „Tarok war ein zorniger Mörder, der schon seit Jahrzehnten seine Kreaturen zum Krieg gegen die Unschuldigen dieses Landes aussandte! Seine Kreaturen haben den edlen Brandur getötet! Wie hirnverbrannt ...“

Thorn trat hinter sie und legte ihr besänftigend eine Hand auf die Schulter. Chada verstummte. Dann erhob Thorn selbst seine freundliche Stimme und sprach beruhigend auf die murrende Menge ein. Es wirkte kaum. Doch Thorn ließ sich davon nicht beirren.

Langsam bahnte Iril sich einen Pfad durch die Menge der Kultisten. Wie vermutet, achtete niemand groß auf sie. Keiner der Anwesenden konnte wissen, wer alles Teil dieser Menge war, und wer sich nur hindurchschlich.

Ein Aufschrei ertönte, als Sagramak ihre Stangenwaffe wieder ergriff und in die Menge zurückstolzierte. Offensichtlich hatten ihre Verhandlungen mit Armond wenig gebracht. Wütend watete die Schamanin an Iril durch die Ansammlung der Kultisten. Iril senkte ihren Blick, doch auch Sagramak schien in Iril nichts Unübliches zu sehen.

Mit der Zeit erschienen immer mehr Helden von Andor und stellten sich zwischen die Ritter der Rietgrade und die Kultisten. Manche, wie ein massiger Krieger mit grauem Fellumhang, taten dies eher widerstrebend und mürrisch. Andere, wie eine Zauberin in schlichtem braunem Mantel, taten dies sehr gefasst und ernst. Die Zauberin steckte gar ihren Stab in den Boden und breitete theatralisch ihre Arme aus, während sie einzelne Personen in der Menge anblickte, welche daraufhin oft verschreckt umherblickten.

Einer der andorischen Soldaten schrie auf und stach mit seinem Speer gegen einen Mann in dunkler Kutte, der ihm zu nahe gekommen war. Ein neben ihm stehender Held – ein gehörntes Wesen mit einem noch längeren Speer – legte dem Krieger besänftigend eine Hand auf die Schulter und wies ihn zurück. Dieser zischte eine hastige Antwort, die Iril nicht verstand.

Da sah Iril Hass in den Augen einer Kultistin direkt vor sich aufblitzen. Eine Zwergin in glänzender Rüstung, vielleicht gar Serafimma selbst. Die Kultistin grummelte etwas unter ihrem Atem, bahnte sich einen Weg durch die Menge und schwang auf einmal einen gewaltigen Hammer auf einen jungen Soldaten, der erschrocken zurückwich – und da war Iril plötzlich vorgetreten und knallte ihren eigenen Hammer gegen den der Angreiferin. Der Runenhammer glühte rötlich auf und hinterließ eine nicht unbeachtliche Delle in der Waffe der Zwergin. Diese wich vor Irils entschlossenem Blick zurück. Oder vielleicht auch davor, wie Irils Augen dabei magisch aufleuchteten. Der Soldat – Manus hieß er, wie sich herausstellte – bedankte sich überschwänglich bei Iril.

Nun war Iril auch Teil der Kette zwischen Kultisten und dem toten Drachen. Auweia.

Iril überlegte sich, sich zurück in die Menge zu mischen. Nicht, dass die Helden und Krieger noch dachten, dass sie eine Kultistin wäre, die sich auf den Leichnam zu schleichen versuchte. Dann jedoch blieb sie stehen. Hier konnte sie helfen.

Die ernste Zauberin blickte Iril tief in die Augen. Iril schreckte zurück, als in ihrem Kopf eine leise Stimme ertönte: „Danke. Ich spüre die Tapferkeit in dir. Es ist wichtig, weiteres Blutvergießen zu verhindern.“

Iril hatte noch niemals eine so schöne Stimme gehört, ernst und leise und gleichzeitig so klar und rein. Doch hatte sie nicht die Zeit, sich mehr mit ihrer Urheberin zu befassen. Denn von irgendwoher kam ein Stiefel zu fliegen und traf Iril in die Seite. Sie rieb sich ihre Rippen.

„Wenn noch länger gewartet wird, eskaliert diese Lage! Warum wird der Drache nicht einfach verbrannt?“, fragte Iril.

„Auch ihm steht das ewige Glück zu“, murmelte Chada neben ihr erneut bestimmt.

„Und es ist eine Qual, ein so großes Feuer sicher zu entfachen“, fügte eine Kriegerin Andors blinzelnd hinzu, „Ganz abgesehen davon, dass Drachenfleisch lässt sich nicht so leicht verbrennen lässt. Das bleibt selbst im stärksten Schmiedefeuer roh. Ich habe gehört, jemand wollte es schon einmal für isolierende Rüstungen nutzen. Hat aber nicht viel genutzt.“

Keiner wusste, was mit dem riesigen Leichnam anzufangen sei. Schlussendlich befahl Thorald, zum ursprünglichen Plan zurückzukehren. Taroks Leiche wurde er zerteilt und Stück für Stück die Narne heruntergespült – auf dass selbst diese verbitterte Echse vielleicht das ewige Glück finden könne.

Helden und andorische Krieger wechselten sich ab mit verschiedenen Aufgaben. Und davon gab es wahrlich viele. Aufgebrachte Kultisten friedlich vom Körper zurückzuhalten, den gewaltigen Körper mit langen Sägen zu zerteilen, die blutigen Stücke auf Karren zu laden, die Karren sicher zur Narne zu begleiten, den stinkenden Inhalt dort wieder abzuladen und fortzuspülen ...

Die Wassergeister der Narne erhoben sich immer wieder aus dem Wasser und schleuderten wortlose Flüche gegen die Andori. Prinz Thorald rief ihnen zu, dass, wenn sie nicht zufrieden waren mit Taroks Leichnam in ihrem reißenden Fluss, sie ihn doch lieber schnell ins offene Meer bringen sollten.

Es dauerte lange. Mehrere Tage. Und es benötigte eine Menge an heldenhaften Helfern.

Iril sah den legendären Zwerg Kram aus der Ferne, wie er mit einer eleganten Zwergenaxt messerscharfe Schuppen vom toten Tarok löste und auf Karren zum Abtransport stapelte.

Iril sah einen Menschen mit rotem Haarschopf, breitschultrigen Körperbau und einem Raben auf der Schulter, der einige harsche Worte in einer fremden Sprache mit Sagramak austauschte.

Iril sah eine Hüterin in weißem Gewand, die einen Wassergeist umherdirigierte. Dieser Wassergeist raste in Windeseile umher. Wo immer Spannungen auftraten, war er vor Ort, um Hitzköpfe abzukühlen. Und als der lange Hals des toten Drachen sich auf einmal aufblähte und Lava absonderte, war der Wassergeist zur Stelle, um ein ausbrechendes Feuer zu verhindern.

Gemeinsam konnten die Helden von Andor schier jedes Hindernis überwinden.\bigskip







Nach wenigen Tagen schien die Lage sich größtenteils beruhigt zu haben.

Von Tarok war fast nur noch ein grobes, stinkendes Gerippe übrig, welches nun ebenfalls Stück um Stück abtransportiert wurde.

Die meisten Kultisten waren abgezogen. Einige wenige hatten in der Nähe des alten Wehrturms ein Lager aufgeschlagen, saßen in Kreisen herum und riefen wilde Worte in längst vergessenen Sprachen in den Himmel.

Das Interessanteste, was in den letzten Tagen geschehen war, war eine Gruppe von drei Jugendlichen gewesen, welche sich in die Narne geworfen hatten, um einige Knochen des Drachen herauszufischen. Sie hatten Bekanntschaft mit der reißenden Narne und den willensstarken Wassergeistern gemacht und waren erfolglos wieder herausgefischt worden. Nun saßen sie von Decken umgeben um ein Lagerfeuer und zitterten.

Die Krieger der Rietburg waren erschöpft und freuten sich aufs sich abzeichnende Ende der Knochenarbeit. Prinz Thorald hatte sich schon länger nicht mehr blicken lassen, vermutlich, weil er schon lange wieder in seinem weichen Bett in der Rietburg weilte.

Tarok war von einer imposanten Kreatur zunächst zu einem stinkenden Knochengerippe geworden, dessen letzten Überreste kaum als Teile eines Körpers erkennbar waren.

Iril saß auf einem Pferdekarren, der eine Pranke des Drachen in Richtung Narne transportierte, und unterhielt sich mit Chada und dem schlanken Wachmann Manus. Letzterer ließ sich soeben begeistert darüber aus, wie die Rietgarde jegliche Bedrohung von der Burg abwehren könnte, wenn sie denn nur einen Seher fänden und auf ihre Seite zögen. Chada selbst schien in eigene Gedanken versunken. Iril hatte nur am Rande mitgekriegt, wie sie sich kürzlich mit der Hohen Priesterin Gända gestritten hatte und letztere zurück in den Wachsamen Wald abgedampft war.

Drei Personen Begleitschutz pro Karren. Das war die Regel. Und an der Abladestation an der Narne wurden noch stärkere Sicherheitsvorkehrungen getroffen.

An diesem Nachmittag zeigte sich sogar Prinz Thorald wieder. Stolz galoppierte er auf seinem edlen Rappen an den entfernt voneinander durchs Rietland gezogenen Karren vorbei und sprach den Andori Mut und Tapferkeit zu. Manus unterdrückte sich ein Kichern, als Thorald zum dritten Mal an ihrem Karren vorbeirannte und ihnen Zugang zum besten Tropfen aus seinem Weinkeller versprach.

Die Lage war ruhig.

Zu ruhig.

Die Pferde schnaubten.

Ohne Vorwarnung teilten sich die Wolken am Himmel und ein Schatten stürzte sich auf ihren Karren. Ein gewaltiger Krark mit Flügelspannweite von mehrfacher Mannslänge warf sich auf die Pferde und riss sie beiseite. Zwei in Kutten gekleidete Skrale sprangen aus dem hohen Rietgras auf den Weg und gaben den Reittieren den Rest. Das panische Wiehern erstarb abrupt. Die echsenhaften Kreaturen bleckten ihre Zähne und bedachten die Kutscher mit milchigen Augen.

Der riesige Raubvogel drehte sich zum Kutschbock herum und starrte den drei dort stationierten Personen entgegen. Dann schwang er sich über die beiden Skrale und die drei Kutscher hinweg und grub seine Klauen in die Überreste von Taroks Pranke, die auf dem Karren transportiert wurde. Mit mächtigen Schlägen seiner Schwingen versuchte der Krark, abzuheben.

Manus war als erster auf den Beinen, sprang auf die Drachenpranke und stocherte mit einem Schwert nach dem Krark.

Chada wirbelte so schnell herum, dass ihr langer Zopf beinahe Irils Kopf getroffen hätte, wenn er nur ein klein wenig höher gelegen hätte. Chada legte einen Pfeil in ihren treuen Bogen Audax, während Iril magische Kraft in ihrem Runenhammer sammelte. Iril schleuderte die magische Waffe auf den linken Skral und warf ihn zu Boden. Ein Pfeil surrte an ihr vorbei und bohrte sich tief in die Brust des Skrals. Chada warf Iril ein aufmunterndes Grinsen zu.

Ein menschlicher Schrei ertönte und verklang ebenso abrupt. Iril fuhr herum. Der gekrümmte Schnabel der Krarks hatte sich in Manus‘ Nacken verbissen und dem tapferen Soldaten innert Augenblicken ein blutiges Ende bereitet.

Iril versuchte, sich wieder umzudrehen, sich dem letzten Skral zuzuwenden, doch ihr Körper gehorchte ihr nicht. Wie gebannt starrte sie auf Manus‘ Leichnam, versuchte, zu verarbeiten, was soeben aus diesem freundlichen Gesellen geworden war.

Wie in Zeitlupe nahm sie wahr, dass der Krark seine Flügel ausbreitete, seine Krallen tief in Taroks tote Pranke grub und nun ungestört in den Himmel abhob. Unter ihm schwankte ein fleischiges Stück von Taroks Pranke.

Ein Pfeil drang tief in den Schädel des Krarks ein. Der Krark krächzte und ächzte gequält. Leblos krachte das Vieh wieder auf den Boden zurück und begrub den Karren unter sich.

Pferdegetrappel ertönte. Prinz Thorald und sein edler Rappen waren zur Rettung gekommen. Thoralds Lanze riss den letzten Skral zu Boden. Die Hufe seines Reittiers gaben ihm den Rest.

Chada fand Manus‘ Hand, fühlte nach seinem Puls und ließ enttäuscht ihren Kopf sinken. Bei den Pferden machte sie sich nicht einmal mehr die Mühe. Sie waren offensichtlich aus dieser Welt geschieden.

Chada legte ihre Hand auf Irils Schulter und blickte ihr in die Augen.

„Geht es dir gut? Bist du verletzt?“, fragte ihre helle Stimme.

Gute Güte, sie schien so jung, und doch so unbekümmert ob des soeben Geschehenen. Iril schüttelte ihren Kopf und sortierte ihre Gedanken.

„Alles in Ordnung“, brachte sie hervor. Das genügte Chada fürs Erste. Die Heldin richtete sich auf und überprüfte, dass die beiden Skrale auch wirklich das Zeitliche gesegnet hatten.

Thorald stieg von seinem Pferd ab und warf Chada einen selbstgerechten Blick zu.

„Immer wieder gerne bereit, zu helfen“, sprach er.

Traurig blickte er in die Ferne, sein Mantel im Wind wehend.

„Der arme Manus. Er konnte mir so einiges beibringen in seiner Zeit. Wäre ich nur früher hier gewesen ...“

„... hättest du den Riesenvogel etwa mit deiner Lanze abzustechen versucht?“

Thorald murmelte etwas vor sich hin und fuhr sich durch die Haare. Anschließend sprach er mit Bedacht: „Diese Kultisten hofften, sich dieses Stück von Taroks Leichnam mit Gewalt verschaffen zu können, und dies kostete einer der unseren sein Leben. Das darf nicht sein. Wir lassen uns von ihnen nichts vorschreiben. Uns nicht einschüchtern. Lasst dieses Stück Drachenfleisch in die Rietburg bringen. Wir werden sie in unserer Schatzkammer einschließen. Als Trophäe für den erschlagenen Krark – und den erschlagenen Drachen. Als Andenken an Manus – und alle anderen, die in den letzten Tagen ihr Leben gaben. Und vor allem als Zeichen, dass die Kultisten diese Pranke niemals haben können werden!“

„Was?!“, fragte Chada ungläubig. „Das ist nichts als ein vermoderndes Stück Fleisch, das wir gerade eben loswerden sollten, damit es nicht die falsche Aufmerksamkeit zieht. Hat die Schatzkammer überhaupt Platz dafür?!“

Thorald knurrte: „Ich bin der Regent von Andor. Der zukünftige König. Wenn ich will ...“

„Natürlich, wenn du willst, wird es so sein. Aber was bringt dir das? Diese Tatze ist tot und stinkt. Keiner ...“

„Na gut, na gut“, murmelte Thorald, „Der Geruch könnte tatsächlich ein Problem werden. Lasst uns einfach ...“

Thorald stach mit seinem Dolch in das Fleischstück hinein, rümpfte seine Nase und angelte einige Fußknöchelchen des Drachen hervor. Für Tarok mochten sie winzig gewesen sein, doch war jedes einzelne größer als Thoralds Hand.

„... lasst uns lieber nur einige Drachenknochen mitnehmen. Als Andenken, Trophäe und Zeichen, dass wir uns nicht von den Kultisten einschüchtern lassen.“

„Thorald, bitte, halte ein und überlege dir noch einmal gut, worin du dich hier verrennst ...“

Iril ließ die beiden sein, sprang vom Kutschbock, hielt ihre Nase zu und sammelte ihren Hammer von der Fratze des linken Skrals ein. Sie versuchte, die tote Kreatur nicht allzu genau anzusehen.

Mit Nase und Augen abgewandt, nahm dafür ihr Gehör etwas wahr. Ein Rascheln im Rietgras fiel ihr auf.

Vorsichtig trat Iril nach vorne, hob ihren Hammer, und ...

... blickte tränennassen Augen entgegen. Versteckt im hohen Rietgras lag Schamanin Sagramak. Ihre glänzende Rüstung hatte sie gegen ein Lumpengewand eingetauscht, doch diese gebrochene Nase erkannte Iril problemlos.

Sagramak bewegte ihren Mund, als suche sie nach Worten. Ihre entsetzten Augen schwirrten zwischen der vor ihr stehenden Iril und dem toten Krark hin und her. „Nein ... nein, das hätte alles nicht so ...“

„Hierher!“, rief Iril zu Chada und Thorald.

Hastig rappelte sich Sagramak auf und versuchte, ihre Fassung zu wahren. Eine Waffe zog sie allerdings nicht. Stattdessen jaulte sie: „Dies hätte nicht so kommen sollen. Ihr hättet uns einfach Zugang zu Taroks Körper geben sollen!“

„Und nun, wo ihr es nicht konntet, versuchtet ihr es mit Mord und Totschlag?“, erklang Chadas zornige Stimme.

„Euer Krieger ...“, ächzte Sagramak, „Ich wollte nicht ... die Instinkte des Krarks ... es tut mir so leid.“

„Das kommt davon, wenn man zur Axt statt zum Schild greift“, sprach Thorald geschwollen.

„Wir mussten zu solchen Mitteln greifen!“, rief Sagramak. „Was sollen wir denn sonst tun, um unsere Stimme hörbar zu machen?!“

„Wir hören euch doch schon“, rief Thorald wenig hilfreich, „Aber eure Belange kümmern uns wenig!“

Frustriert fauchte Sagramak auf, wirbelte herum und verschwand überraschend, gar unnatürlich rasch tiefer im hohen Rietgras.

Chada hielt Audax im Anschlag und zielte. Und zögerte.

„Na, mach schon!“, fuhr Thorald sie an, „Schieß!“

Chadas blinzelte unentschlossen, doch ihre Hand blieb ruhig.

Iril dachte zurück an die mit einzelnen Pfeilen getroffenen Skral und Krark. Ihr fuhr es schaurig den Rücken herunter. Sie hatte gehört, dass die Bewahrer ihren Bogenschützen beibrachten, Tiere mit nur einem einzigen Schuss zu erlegen, damit sie keine unnötigen Qualen litten. Man wollte sie lieber nicht zum Gegner haben.

Entschieden ließ Chada ihren Bogen sinken. „Sie ist keine Gefahr. Taroks Körper ist die Gefahr, und was man damit alle für finstere Rituale auslösen könnte.“

Thorald schnaubte auf und blickte zurück zu seinem Pferd, unzweifelhaft kalkulierend, ob die durchs Rietgras davonsprintende Sagramak den Aufwand einer Verfolgung wert war. Schließlich winkte auch er ab. „Die sehen wir so bald nicht wieder. Kümmern wir uns lieber darum, den Karren zu reparieren. Und den Toten einen würdigen Abschied zu bereiten.“

Er blickte Chada und Iril vielsagend an und galoppierte dann einfach davon. Vielleicht kam es ihm nicht einmal in den Sinn, dass er selbst hätte aushelfen können.

Iril schluckte schwer und begann damit, Manus‘ Leichnam vom Karren zu heben.\bigskip







Noch am selben Tag war es vorbei.

Der restliche Abtransport von Taroks Leichnam ging ohne Probleme vonstatten. Eine Priesterin aus der Kapelle der Rietburg trat an die Narne und sprach einige friedliche Worte, während die letzten Überreste der Riesenechse ins Hadrische Meer trieben. Wassergeister plätscherten unverständliche Worte der Warnung gegenüber den Kultisten, die der Narne zu nahe kamen – und gegen die Krieger, die ihren schönen Fluss mit finsteren Knochen, Fleisch und Blut besudelt hatten.

Manus Körper wurde aufgebahrt. Krieger zollten ihm Respekt. Seine Familie bereitete den Totenritus vor.

Soldaten stellten sicher, dass die großen Blutlachen vor dem alten Wehrturm aus dem Rietgras gewaschen waren. Dann traten sie zurück und ließen die trauernden Kultisten nähertreten. Viele von ihnen sanken zu Boden. Manche weinten, manche suchten nach Überresten des Drachen, und sei es nur, um ein Stück einer Schuppe oder eine Phiole verschmutzten Drachenblutes finden.

Die meisten Helden von Andor blickten einander unsicher an. Dies war definitiv nicht so, wie sie sich das Nachspiel eines heroischen Drachenkampfes vorgestellt hatten. Und einige schienen unsicher, ob sie auf der richtigen Seite dieses Konflikts gestanden hatten.

Sei dem, wie es sei, es war vorbei. Taroks Leichnam war fort. Keiner konnte den Körper mehr für etwaige finstere Zwecke nutzen. Die Helden konnten wieder abziehen. Manus‘ Familie konnte um ihn trauern.

Die Hüterin der Flusslande teilte den restlichen Helden mit, dass sie hierbleiben würde. Unter den Drachenkultisten hatten sich auch ein, zwei Flussländer befunden. Mit ihnen würde sie demnächst ein Wörtchen sprechen.

Die restlichen Helden verstreuten sich wieder in alle Himmelsrichtungen.

„Kommst du mit?“, fragte eine tiefe Stimme hinter Iril. Diese erschrak, als sie im Sprechen den berühmten Kram aus den Tiefminen erkannte, fasste sich aber auch wieder rasch.

„Wie meint Ihr?“

„Bitte, du kannst das Du nutzen. Ich bin Kram. Danke für deine Unterstützung. Kommst du nicht auch von Cavern? Ich breche dorthin auf. Du könntest dich mir anschließen.“

Iril fasste sich ein Herz.

„Ich weiß nicht. Ich stamme eigentlich von Silberhall. Ich fühle mich in Cavern nicht mehr wirklich zuhause.“

„Das verstehe ich“, murmelte Kram, „Ich bin mir auch nicht mehr sicher, wo mein Herz mehr liegt. Doch ändert das nichts daran, dass es eine wahre Freude ist, bei meiner Familie in Cavern zu sein.“

„Familie, ja“, schluckte Iril, und dachte an Iolith. „Die habe ich nicht mehr hier. Bitte, richten S ... richte Du einen Gruß an deine Familie aus. Es sind feine Gesellen. Aber in Cavern hält mich nichts mehr.“

Kram kratzte sich am behelmten Kopf, nickte dann aber und wandte sich in Richtung Süden. An seiner Seite stapfte der große Wolfskrieger.

„Na, auch auf zum Trunkenen Troll, Kram?“

„Nee, Papa kommt bestimmt schon um vor Sorge.“

„Dein Pech. Ich gönne mir jetzt mindestens eine Woche Auszeit in Gildas Taverne. Danach können wir uns Gedanken machen über den Wiederaufbau all dessen, das vernichtet wurde.“

Iril sah die beiden Helden abziehen. Sie wusste nicht, wo sie hinsollte. Doch wusste sie, dass es sie nicht nach Cavern zog. Beim Gedanken, dorthin zurückzukehren und vor den Behausungen ihrer toten Familie herumzulungern, drehte sich ihr Magen um. Nein, in Cavern gab es nichts für sie. Zeit, zu schauen, ob sie den Andori helfen konnte.\bigskip







Iril folgte einer Gruppe abziehender andorischer Krieger in Richtung Rietburg. Nachdem sie sich im Debakel mit den Kultisten auf die Seite der Helden gestellt und eine edle Rede über Manus‘ tapferen Einsatz geschwungen hatte, erfuhr sie kaum Misstrauen der Rietgarde. Die Krieger des Königs waren ohnehin größtenteils viel zu erschöpft, um sich groß um sie zu kümmern.

Die Türme der Rietburg ragten hoch in den Himmel. Auch über diesem Gemäuer hing eine dunkle Rauchwolke vom letzten, doch das ewige Feuer vor den hohen Toren flackerte in hellem orange. Ein Zeichen, dass die Gefahr sich gelegt hatte. Dass die Andori in das befreite Gemäuer zurückkehren und sich an die Reparaturen machen konnten.

Iril erlebte am Rande mit, wie Prinz Thorald mit viel Theatralik den Sack mit Taroks letzten Fußknochen in den Thronsaal brachte.

„Der bringt die Knochen jetzt in seine Schatzkammer, wo auch der Bruderschild verstaubt. Einer der vier mächtigen Schilde aus uralter Zeit. Die Helden von Andor fanden ihr vor einigen Jahren wieder. Nun verrottet er jedoch die größte Zeit in Thoralds Prunksaal“, sprach eine tiefe Stimme.

Iril blickte sich um. Neben ihr stand ein Zwerg in voller Plattenrüstung, der sich auf einen langen Hammer stützte und das Geschehen mürrisch beobachtete. Seine kahle Stirn glänze im Sonnenlicht. Sein langer Bart war wohl einst rötlich gewesen, nur jedoch eher braun und grau vor lauter Dreck. Einer der Schildzwerge, die bei der Befreiung der Rietburg mitgeholfen hatte?

Iril, die sehr wohl gewusst hatte, was der Bruderschild war, blieb stumm. Dies nahm der mürrische Zwerg zum Anlass, fortzufahren: „Ich könnte es zumindest respektieren, wenn die Helden den Bruderschild für gute Zwecke einsetzen würden. Wenn er aber ohnehin nur in der Privatsammlung eines Prinzen Staub sammelt, könnte man ihn geradesogut denjenigen zurückgeben, die ihn wahrlich verdient hatten. Den Schildzwergen. Den Nachkommen Kreatoks. Nicht wahr?“

Iril dachte zurück an den Silberschild, denjenigen mächtigen Schild aus Kreatoks und Nehals Sammlung, der schon seit Jahrzehnten in Silberhall Staub sammelte. Sturmschild nannte man ihn auch, da er es seinem Träger erlaubte, sich die Winde untertan zu machen. Einige der darauf zu findenden Runen hatten den Runenmeistern gereicht, um solche Effekte in kleineren Skalen zu replizieren. Doch davor, die Sturmwinde über dem gesamten Hadrischen Meer zu befehligen, wie es Träger des Sturmschilds konnten, träumten die Runenmeister nur. Erst recht, solange Arkteron, der Herr der Stürme, in den wogenden Meereswellen lauerte.

Iril hatte den Silberschild nur einmal aus der Ferne gesehen. Schon seit längerem war er nicht mehr aus der Schatzkammer geholt worden. Die Silberzwerge, die Werftheimer und die Taren hatten Rat gehalten und beschlossen, den Schild lieber nicht zu demonstrativ präsentieren. Nicht, dass die Mächte des Meeres sich durch dieses höchste Gut zwergischer Schmiedekunst provoziert fühlten. Eines Tages würden würdige Träger vielleicht in der Lage sein, den Schild zum Wohle des Nordens einzusetzen. Doch bis dahin hielt man ihn lieber versteckt.

Iril schüttelte ihren Kopf. In die Politik der mächtigen Schilde wollte sie sich nicht einmischen. Die meisten Geheimnisse dieser Schilde waren bereits enthüllt worden. Und ob sich seit Kreatoks Tod je wieder wirklich würdige Träger dafür finden sollten, stand in den Sternen geschrieben.

Der mürrische Zwerg vor Iril schien ihre mangelnde Antwort richtig zu interpretieren und deutete zum Themenwechsel auf ein nahegelegenes Hausdach.

„So hätten wir mit dem Drachen umgehen sollen“, murmelte er.

Es dauerte einen Augenblick, bis Iril erkannte, was ihre Augen wahrnahmen. Ein gewaltiges geschupptes Wesen lag quer über dem Dach des Palas‘, mit einigen Pfeilen quer aus seiner Kehle ragend. Rückenstacheln länger als ein Arm. Hörner länger als ein ausgewachsener Mensch. Ein doppelt so langer Schwanz. Eine dampfende, blau glühende Flüssigkeit tropfte aus einem vielzahnigen Mund. Iril schauderte es bei seinem Anblick. Welch finstere Kreaturen hatte dieser Drache befehligt?!

Leitern waren rund um das Gebäude angebracht worden. Um das gewaltige Wesen standen und saßen viele Andori auf dem Strohdach und entfernten Fleischstücke aus dem Leichnam. Andere reichten die Fleischstücke an den Boden, wo sie auf verschiedenen Feuern landeten.

Iril fiel besonders ein kleiner Wichtel in einem Kapuzenmantel auf, der am Boden stand und mit großen Gesten etwas herumdirigierte. Umso überraschter war Iril, als sie sah, dass auf das Winken und Wedeln des Wichtels ganze Fleischstücke der geschuppten toten Kreatur von grünem Licht erfüllt wurden und sanft zu Boden schwebten. Der Wichtel grinste fröhlich bei der Arbeit. Glitzernder Sand rieselte aus seinen Händen, während er weiter herumzauberte.

„Wunderst du dich über das kleine Männlein?“, sprach der mürrische Zwerg. „Das ist Wrort. Er sagt, er sei aus einem fernen Land angereist. Will aber nicht sagen, woher. Magisch begabte Wesen, diese Wichtel. Mir sind sie nicht ganz geheuer. Und das riesige Viech auf dem Dach nennt man einen Mhourl. Schon lange haben wir keine mehr gesehen. Es ist kein gutes Zeichen, dass sie jetzt wieder auftauchen. Ich bin übrigens Lafgar. Sag, kannst du auch sprechen?“

„Wenn man mich lässt“, grinste Iril und stellte sich vor. Lafgar grinste nicht, begrüßte sie jedoch mit einem Faustschlag.

„Warum wird der Mhourl denn erst am Boden verbrannt? Kann man ihn nicht schon dort oben verbrennen?“, fragte Iril neugierig.

„Nicht, ohne die ganzen Dächer anzuzünden. Mit getrocknetem Rietgras wurden sie gedeckt“, lachte Lafgar kopfschüttelnd, „Als wollte jemand, dass sie in vom erstbesten fliegenden Funken in Brand gesteckt werden.“ Er lachte leise beim Gedanken daran.

„Das wäre ein Feuerfest geworden, wenn der Drache bis zur Rietburg gekommen wäre“, brummelte Lafgar weiter. „Aber das hätte mich nicht gefreut. Diese Rieseneidechse entkam dem Tod schon zu lange. Ein hirnloses Biest war das, dafür müsste niemand das Ewige Glück in der Narne suchen. Wie sehr hätte es mich erfreut, wenn man den gewaltigen Fleischvorrat genutzt hätte und auch noch in zwei Jahren davon hätte zehren können.“

„Drachenfleisch kann Körper und Geist vergiften“, meinte eine fremde Stimme, „Es ist gut, dass wir es so rasch wie möglich beseitigten.“

Iril blickte sich um und erblickte einen weiteren Gesprächigen, einen jungen Andori, der sich zu ihnen gesellt hatte. Er trug bereits die Kleidung eines andorischen Kriegers, doch schien das Schwert an seinem Gürtel eher zu Trainingszwecken.

Lafgar und der Andori tauschten einige kreative Beleidigungen des toten Drachen aus. Iril lachte mit ihnen. Dann erinnerte sie sich und fragte interessiert: „Wer ist eigentlich dieser Reiter des Drachen mit dem langen Schwert? Ich glaube, ihn unter den Drachenkultisten gesehen zu haben.“

„Ein Drachenreiter?! Keine Ahnung“, brummte Lafgar, „Man muss doch verrückt sein, um auf diesen Dingern reiten zu wollen.“

Das Gesicht des jungen Andori hellte hingegen auf. Er sprach zur Begrüßung seinen Namen – Peta – und berichtete dann beflissen: „Ihr habt den Drachenreiter unter den Kultisten gesehen? Das ist bedenklich. Wenn auch kein Wunder. Das ist der Schwarze Herold. Eine Sagengestalt. Aber keine erfundene. Ich habe ihn auch schon gesehen. Er unterstützt alles Böse in Andor. Er treibt die Kreaturen des Drachen an.“

„Heißt das, dass die Drachenkultisten die nächsten Bösen der andorischen Geschichte sind, wenn er nun bei ihnen weilt?“

Peta antwortete: „Fest steht, dass diese Drachenkultisten dem Königreich nichts Gutes wollen. Ehrlich, ich kann nicht verstehen, wie man die Drachen anbeten kann.“

„Vielleicht haben sie Angst statt Ehrfurcht“, überlegte Iril, die an die belauschte Konversation von Hildorf dem Meisterschmied zurückdachte. „Wenn man denkt, dass die Drachen einen nach seinem Tode richten werden, würde man durchaus einiges tun, um ihre Gunst zu gewinnen.“

„Verzeiht ihre Taten noch lange nicht.“

„Der Prinzen ist jedoch auch nicht automatisch im Recht. Der Drachenleichnam ist zu gefährlich, um anderen Zugang dazu zu gewähren, und doch lässt Thorald sich dazu hinreißen, einige Knochen in seine Schatzkammer zu bringen?“

„Unser Prinz wird seine guten Gründe gehabt haben“, sprach Peta.

„Pah“, meldete sich nun der mürrische Lafgar wieder zu Wort, „Als ob der Prinz wüsste, was er tut. Mit der Rietgraskrone auf Thoralds Kopf wird das Leben nirgendwo besser werden. Wir können von Glück reden, wenn er keinen Krieg mit den Drachenkultisten anzettelt. Oder einen Aufstand der Flussländler. Oder die Barbaren wieder vertreibt. Er ist vieles, aber kein Diplomat. Und kein feiner Herrscher.“

„So einen feinen wie Brandur wird es nie wieder geben“, sprach Peta andächtig. Lafgar gluckste ungläubig. Ehe er zu einer potenziellen Tirade über den Landräuber Brandur ausbrechen konnte, mischte sich Iril ein:

„Mit Verlaub. Brandur war nur ein König. Ein guter König vielleicht, aber immer noch nur ein König. Ein Mensch, der Fehler macht. Sein Sohn wird auch Fehler machen. Aber er wird dieselben Berater haben, dieselbe Burg, dieselben tapferen Krieger an seiner Seite. Worüber macht ihr euch solche Sorgen?

Peta antwortete leise: „Oh, ihr kennt Thorald noch nicht. Er ist ein ausgezeichneter Reiter und gut im Umgang mit der Lanze, aber das ist auch schon alles. Und ihr versteht nicht, wie beliebt König Brandur hier war. Brandur hat unsere Großeltern im Alleingang als Jugendlicher aus der Sklaverei ins Freie geführt. Todesmutig verschaffte er allein seinem Gefolge einen Weg am Drachen Tarok vorbei. Später verteidigte er die hier Angekommenen, die Andori, in den Trollkriegen immer und immer wieder aufs Neue. Er ist ein Held. Manche sehen ihn als Geschenk von Mutter Natur höchstpersönlich. Schon als kleiner Junge träumte ich davon, einer seiner Krieger zu werden. Als ich ihm das erste Mal gegenüberstand, dachte ich, mein Herz schlüge aus meiner Brust hinaus vor Aufregung. Manche dachten, Brandur könne gar nicht sterben. Er wurde älter und älter, und natürlich auch gebrechlicher, doch sein Wille schien nie gebrochen ...“

„Unsinn, sein Wille ward mehrmals gebrochen.“, unterbrach ihn Lafgar wieder, „Unlängst brauchte es gar ein seltenes Heilkraut, um seine angeschlagene Stimmung zu retten. Und als sein böser Bruder die Agren terrorisierte, bot Brandur ihm einen Platz an seiner Seite an. Sein eigen Blut kümmerte ihn schon seit jeher mehr als alles andere. Erinnert ihr euch noch an die Zeit, als Thorald im Eisschlaf feststeckte und Brandur nur noch vor sich hin trauerte, statt eine Lösung zu suchen? Brandur war anfällig auf Fehler wie wir alle. Und dann erst die Gerüchte, dass er in vor nicht einmal so langer Zeit einigen un- und verheirateten Damen ...“

„Schweig stille“, zischte Peta, „Das ist unser König, über da sprichst!“ Wut glitzerte in seinen Augen.

„Eben nicht mehr“, gab der übellaunige Zwerg zurück.

Peta nickte traurig und wandte sich ab. „Nicht einmal eine ganze Woche ist er von uns gegangen, und schon spricht man schlecht von ihm. Natürlich tat man das auch schon vorher, aber nun, mit Thorald an seiner Stelle ...“

Peta verstummte und schüttelte seinen Kopf. „Nicht verzagen. Wir können noch hoffen. Und unser Bestes geben. Vielleicht, mit etwas Glück und Verstand, werden auch die Jahre von König Thoralds Regentschaft durch Frieden und Freude gezeichnet. Vielleicht sogar mehr als die Jahre von Brandur selbst. Wir können noch hoffen.“

Iril nickte bloß. Sie verabschiedete sich von den beiden und kehrte in den Flüchtlingslagern an der Rietburg ein. Die nächste Zeit würde sie hier verbringen und aushelfen, wo sie konnte.

Anfangen würde sie beim Zerteilen des Mhourls.







\newpage
\section{Der Eis-Dämon}

Über eine Woche war vergangen seit dem Tode Taroks.

Eigentlich hätten viele Bauern wieder ins Land ziehen wollen, doch gab es weiterhin viele Kreaturen, die nach dem Ableben des Drachen ziellos umherzogen. Bauernhöfe ungeschützt wieder aufzubauen, war lebensgefährlich. Daher blieben viele Bauernfamilien vorerst lieber hier, auf der sicheren Rietburg. Auch da gab es zahlreiche Reparaturen von niedergerissenen Türen und beschädigten Türmen zu tätigen. Zahlreiche Verletzte zu pflegen. Und zahlreiche Tote zu verabschieden, auch wenn die Totenzeremonie des gefallenen Königs immer weiter in die Ferne geschoben wurde.

Durchgehend lernte Iril neue Leute kennen. Den Keltermeister der Rietburg, der sich bei jeder Gelegenheit über unerwünschten Pflanzenwuchs in seinem Weinkeller ausließ. Baumeister Mard, der sich um den Säugling seiner verletzten Schwester kümmerte und ihm die eleganteste Wiege diesseits des Ozeans versprach. Torwächterin Felda, die weder lesen noch schreiben konnte, aber jedes einmal gesehene Schriftstück beinahe perfekt replizierte. Meisterbäcker Karmat, dessen luftige Kekse selbst die Kochkunst von Draks Familie in Schatten stellte.

Mit so einigen Anwohnern machte Iril flüchtige Bekanntschaft, doch klickte sie mit niemandem so besonders, wie sie es sich bei ihrer Ankunft in Silberhall direkt an Burmrit geklammert hatte. Diese innere Leere würde nicht so schnell gefüllt werden.

Auch wenn Iril sich stets amüsierte dabei, mit den Hunden der Rietburg spazieren zu gehen. Oder den Barden Grenolin beim Dichten seiner neusten Epen zuzuhören.

Oder einer alten Gelehrten bei der Suche nach den verschollenen Runensteinchen zu suchen.

Wie es sich herausstellte, hatte Tarok bei seinem Angriff auf Andor die Magie des Landes in sich aufgesogen und dabei zahlreiche Runensteine aus allen Ecken des Landes angesaugt und verschluckt – eben auch die der Gelehrten der Rietburg. Es stand zu befürchten, dass sie in Taroks Magen die Narne heruntergespült worden waren.

Im Anschluss daran, der Hohen Gelehrten eine vollständige Runenstein-Sammlung zu beschaffen, hatte Iril der Gelehrten und ihren aufgeweckten Schülern die Macht der Runen präsentieren können.

So, wie die Kinderaugen beim Aufführen von kleinen Tricks mit Irils Runenscheibe schon aufgeleuchtet hatten, schien es gut möglich, dass einige davon eine Zukunft als Runengelehrte in Betracht zogen. Schöne Entwicklung. Der Erforschung der Magie konnte es schließlich kaum zu viele begeisterte Helfer geben. Im Anschluss erzählte die Hohe Gelehrte Iril, was für Rituale man theoretisch mit den Drachenknochen anfangen konnte, die neuerdings in der Schatzkammer der Rietburg eingeschlossen waren. Iril staunte und trauerte auf einmal über die Bedeutung des Schatzes an Drachenmagie, den der Prinz mit Taroks Leichnam ins Hadrische Meer hatte spülen lassen.

In den Lazaretten der Rietburg konnte Iril mit ihrer Runenmagie besonders gut aushelfen. Beim alten Heiler Readem, der Knochen richten und Gliedmaßen amputieren, aber nichts gegen simple Halsschmerzen unternehmen konnte. Bei dessen Assistent Nabib, der die meiste Zeit lang detaillierte Zeichnungen von Knochen und Körpern anfertigte oder studierte. Sie beide blickten immer wieder mit ungläubigem Staunen auf Iril, wenn sie wieder einmal ihre Runenscheibe drehte, zum Glühen brachte, und damit dringend benötigtes sauberes Wasser in einem der nahe gelegenen Brunnen aufsprühte.

Hilfsbedürftige gab es in den Zeiten nach Taroks Zorn zur Genüge, selbst wenn die Stimmung unter den Rietländern eher ausgelassen war und Feiern gefeiert wurden. Iril half aus, wo sie konnte. Sie hätte es wohl auch getan, wenn es sie genervt hätte. Doch wie sie herausfand, erlangte sie ein gutes Gefühl der Erfüllung dabei, die Andori in ihren Wiederaufbauten zu unterstützen. Im Schweiße ihrer Arbeit vergingen die düsteren Gedanken zurück an ihre tote Lehrmeisterin, an ihre toten Eltern, an ihre verschollene Schwester. Bei Tag war sie zufrieden. Nur des Nachts schlichen sich die finsteren Gedanken wieder wie schleimige Schnecken in ihre Träume und hinterließen ihren klebrigen Schleim.

Und wie Iril stets an verschiedensten Winkeln der Rietburg zu Hilfe war, kam es, dass sie als eine der ersten gerufen wurde, als eines Morgens ein Brunnen hinter steilen, hohen, gar unbezwingbaren Westmauer der Rietburg plötzlich vereiste.

„Ich kann’s mir auch nicht erklären“, rief der Junge, der die Nachricht brachte. Jandro hieß er. „Es geht noch Monate, bis der Winter uns erreichen sollte. Doch ist dieser Brunnen und das umliegende Gras bereits jetzt ist mit einer Eisschicht versehen, dicker als in allen Wintern zuvor. Ich konnt’s aus der Ferne sehen. Mein Vater Najuk ist bereits ausgezogen, um näheres zu berichten.“

Ein vereister Brunnen vor der Rietburg? Iril erster Gedanke war, dass dies Bragor gar nicht gefallen würde. In den letzten Tagen hatte Iril sich mit diesem riesenhaften Tarus aus Sturmtal angefreundet, indem sie den Brunnen vor der Westmauer kraft ihrer Runenscheibe immer wieder aufs Neue hatte sprudeln lassen. Bragor hatte seinen schier unstillbaren Durst bewiesen. Hätte Iril nicht irgendwann eingehalten, hätte er wohl getrunken, bis seine Blase geplatzt wäre. Nicht ohne Grund gab es in seiner Heimat das Sprichwort „Ein Schluck Wasser bringt dir Stärke.“

Als Iril dieses geflügelte Wort das erste Mal vernommen hatte, hatte sie es noch mit dem alten Zwergensprichwort „Wasser ist wichtig! Wein ist gut! Met ist besser!“ gekontert, welches es offenbar selbst in die andorischen Kochbücher geschafft hatte. Seitdem sie miterlebt hatte, wie sehr frisches Quellwasser den Tarus erfrischen konnte, zweifelte sie nicht mehr daran, dass manchen auch Wasser völlig ausreichte.

Bragor war wie Iril von einer ihm vertrauten Nebelinsel in den südlichen Kontinent aufgebrochen und hatte sich hier neu zurechtfinden müssen. In den Helden von Andor hatte er jedoch eine neue Familie gefunden. Und manche andorischen Brunnen konnten gar beinahe die Frische der Quellen seiner stürmischen Heimat erreichen. Jedoch eben nur, wenn sie nicht gerade eingefroren waren.

Iril und der junge Jandro eilten auf den Wehrgang der Rietburg und spähten durch ein Fernrohr in die Ferne. Wie Jandro berichtet hatte, war der Brunnen im Südwesten der Burg von einer weißen Wolke bedeckt. Schnee lag um ihn herum. Zu dieser Jahreszeit. In dieser Tiefe. Unmöglich. Unnatürlich. Hexerei?

Iril überlegte sich bereits, ob sie von hier aus ihre Runen dazu befragen oder lieber direkt zum Brunnen ausreisen sollte, da ertönte ein Schrei von weiter unten am Eingangstor.

„Hilfe! Hilfeee! Ich werde verfolgt!“

„Das ist Papas Stimme!“, rief Jandro. Der Andori rannte die Treppe herunter und den Weg zum Tor entlang, wobei er gehörig Staub aufwirbelte.

Iril musterte den schreiend heranrennenden Andori neugierig. Najuk, hatte der Kleine gesagt. Dem „Na“ in seinem Namen nach stammte Najuk vermutlich ursprünglich aus dem Land der drei Brüder im Osten und war mit den einfallenden Sippen des Yetohe-Stammes nach Andor gelangt. Als er nun verschnaufte und hastig eine Antwort hervorstammelte, war in seiner Sprache allerdings nicht der geringste Akzent zu vernehmen.

„Da hat ein fremdes, eisiges Wesen einfach im Brunnen geschlafen. Vermutlich hat es ihn so vereist! Und es verfolgt mich! Guckt!“

Panisch zeigte Najuk auf eine Gestalt, welche nun durchs offene Tor der Rietburg trat und sich interessiert umsah. Schneeweiß war sie, kaum bekleidet, mit zwei kleinen Geweihen links und rechts aus dem Kopf wachsend. Der Schneemann trug eine auffällige Kette mit spitzen Gliedern um den Hals.

Zahlreiche Wachen richteten Speere und Bögen auf das Wesen, allerdings mehr verwirrt als vorsichtig, denn das Ewige Feuer in der Schale vor dem Tor flackerte weiterhin größtenteils orange. Ein Zeichen, dass zumindest aktuell keine große Gefahr für die Rietburg bestand. Die meisten anderen Anwesenden guckten nur neugierig. Helden von Andor schienen keine anwesend. Prinz Thorald ebenso wenig.

Das Wesen hob seine Stimme und sprach einige Worte in einer fremden Sprache. Bestimmt. Entschlossen. Absolut unverständlich.

„So was hat es schon vorhin gebrabbelt“, meinte Najuk schlotternd.

„Ein Eis-Troll!“, reif eine grau gewandte Gestalt aus der Menge der Schlaulustigen. Iril erkannte darin eine Bewahrerin vom Baum der Lieder, wohl hierher gesandt, um die Geschichte von Taroks Tod in die Archive des Baums der Lieder aufzunehmen. Sie wiederholte: „Das ist ein Eis-Troll. Viele Jahrhunderte schon haben wir keinen mehr gesehen. Doch unsere Aufzeichnungen berichten davon, wie die Vorfahren der ersten Bewahrer von Mutter Natur einige Abstecher ins Fahle Gebirge wagten. Und von einem riesigen Eis-Troll überfallen wurden. Die wenigen Überlebenden berichteten von einem riesenhaften Troll, weiß wie der Schnee und das ewige Eis. Er trug ein mächtiges Geweih auf seinem Kopf –und nichts als einen eisigen Lendenschurz, genau wie unser Neuankömmling hier. Ich glaube, er nannte sich ... Rakor? Rokur?“

„Eis-Troll?! Sieht dieser Schneemann für dich etwa wie ein Troll aus, Sanja?“, fragte der Bewahrer neben ihr unsicher.

„Nun, Jorna, vielleicht ist es auch eher ein Eis-Mensch“, gab die Bewahrerin zurück, „Wenn es Eis-Trolle geben kann, ist die Existenz von Eis-Menschen nicht so unwahrscheinlich, oder? Das macht ihn nicht weniger gefährlich.“

Iril blickte den Fremden erneut an.

Er wirkte wirklich eiskalt. Beim Ausatmen quoll Dampf aus seinem Mund, unter seinen eisfarbenen Stiefeln gefror der Boden und er zog eine Spur aus Schnee und Eis hinter sich her.

Erneut rief der Fremde etwas. Seinem klirrenden Tonfall konnte Iril keine Bedeutung entringen. Und die Sprache sagte ihr ohnehin nichts.

Noch nicht.

Nicht ohne Grund hatte sie sich einst eine Übersetzungsrune eintätowieren lassen. Iril fasste sich an den Nacken und ertastete das passende Tattoo. Sie griff den Runenhammer fest und fühlte, wie er Energie aus der Umgebung einsog. Dann presste sie den Hammer mit der spitzen Seite in ihren Nacken und fühlte, wie eine wohlige Wärme sich ihrer rechten Wange entlang ausbreitete und ihr Ohr erhitzte.

„Wer bist du?“, fragte sie den Schneemann. Leider konnte er sie nicht verstehen – so mächtig waren die Übersetzungsrunen dann auch wieder nicht. Aber sie sorgten dafür, dass sie ihn verstehen konnte. Nun musste sie ihn nur wieder zum Sprechen anregen.

„Ich verstehe kein Wort. Doch guckt mich an, ich bin keine Gefahr. Ich trage ja nicht einmal eine Waffe“, sprach der Schneemann mit klirrender Stimme und erhobenen Händen. Die Runen auf Irils Wange flackerten, dann verstand sie. Sie nickte.

„Du kannst mich nicht verstehen, aber ich dich“, sprach Iril ruhig, in der Hoffnung, ihr Tonfall könne irgendwie kommunizieren, was ihre Worte allein nicht konnten. Das letzte Wort des Schneemanns war „Waffe“ gewesen. Demonstrativ zeigte Iril auf einen Speer.

Der Schneemann hielt kurz inne und fragte dann: „Sonne?“

Iril zeigte, ohne zu zögern, hoch in den Himmel, an dem der brennende Feuerball hing.

„Boden“, versuchte der Schneemann es erneut. Iril zeigte auf den Boden.

„Wo Sonne, Mond und Sterne schweigen, denn das längste Licht wird durch den kleinsten Tropfen verwehrt“, sprach der Schneemann nun. Iril schwieg verwirrt. Der Singsang hatte nach einem Rätsel geklungen, oder nach einem Reim. Reime waren besonders interessant durch eine Übersetzungsrune wahrzunehmen, schließlich hörte Iril quasi mit dem einen Ohr den schönen Singsang, wie er im Originaltext klingen sollte, während die Runen ihr die Bedeutung der Worte in ihren Kopf flüsterten, wo sie sich in Bilder und dissonante Klänge auflösten, die ihr Geist verstehen konnte. Und die doch nie genau die ursprüngliche Intention herbeibringen konnte. Dafür waren Sprachen dann doch zu grundlegend verschieden. Manche Gedichte waren einfach unübersetzbar.

„Wolken. Ich meinte Wolken“, sprach der Schneemann etwas resigniert. Iril grinste, zeigte auf einige vorbeischwebende Wolken, und hatte damit dem Schneemann wohl zur Genüge ihre Übersetzungsgabe demonstriert.

„Du verstehst mich?“, fragte der Schneemann. Als ob er es nicht schon längst kapiert hatte.

Iril nickte. Dann hatte sie eine Erkenntnis. Normalerweise brauchten die Silberzwerge besondere Tinte, um die magischen Runen auf oder unter der Haut der Runenträger zu hinterlassen. Schließlich hatte man die schmerzhafte Tradition, stärkende Runen immer neu bis aufs Blut in die Haut zu schneiden, schon seit den Jahrhunderten seit dem Unterirdischen Krieg hinter sich gelassen.

Aber so ein Schneemann bestand doch nur aus Schnee, oder? Da wäre es doch ein leichtes, eine Rune darin zu hinterlassen.

Iril deutete auf ihre eigenen glühenden Runen und dann auf Ijsdurs Brust.

„Ich kann auch dich verstehen lassen. Wenn du mich lässt. Du. Verstehen. Dank. Runen. Wenn ich dir die Rune einzeichnen darf.“

Der Schneemann blickte verwirrt drein. Iril ergriff die Initiative und trat auf ihn zu. Von ihr mit Gesten geboten, kniete er sich zu ihr herunter. Sein kalter Atem kitzelte sie in der Nase. Dann hob sie ihren Hammer und hielt diesen auf Ijsdurs Brust. Sanft führte sie ihn dorthin und zog einen Strich darüber.

Abwesend murmelte sie: „Keine Nippel? Und auch keinen Bauchnabel. Dafür spitze Ohren ... so, als hätte jemand versucht, einen Menschen aus Schnee und Eis zu schaffen, aber keine perfekte Kopie erstellt. Und weiter oben ... ist das ein Geweih? Was bist du nur?“

Der Schneemann antwortete natürlich nicht, da er sie noch nicht verstehen konnte. Er zuckte jedoch zusammen, sobald Irils glühender Hammer mit seiner Spitze seine Brust berührte. Beim Kontakt schmolz ein kleiner Teil des Schnees seines Körpers und tropfte zu Boden. Iril beobachtete vorsichtig die Wunde. Ein klarer, dünner Strich. Kein Blut tropfte daraus hervor, ja, darunter schien nur mehr Schnee zu liegen. Und auch wenn dem Schneemann der Vorgang unangenehm erschien, zuckte er auch nicht zurück wie jemand, dem man die nackte Brust aufgerissen hätte. Iril würde ihn fragen müssen, wie es um sein Schmerzempfinden stand. Bald, sobald sie ihn verstehen konnte.

„Ruhig, ganz ruhig“, sprach Iril beruhigend. Zu ihrer Freude wehrte sich der Schneemann nicht. In Kürze hatte Iril die Runen der Übersetzung auf die schneeige Brust eingeritzt, den Hammer neu aufgeladen, ihn leicht auf die Brust des Schneemanns getippt und die Rune aktiviert.

Grünlicher Schimmer bereitete sich aus und der Schneemann sprach: „Uh, das kitzelt.“

„Das ist ein gutes Zeichen. Das heißt, es funktioniert“, sprach Iril.

Der Schneemann hielt eine Zeit lang überrascht inne und meinte dann: „Tatsächlich. Ich verstehe dich. Was für ein Wunder ist das?“

„Runenmagie“, sagte Iril fröhlich, und tappte stolz auf ihren Hammer, „Runen, die dir erlauben, Strukturen in gesprochenen Worten zu erkennen und sie besser sortieren zu können. Die dir ermöglichen, mich zu verstehen, und alle anderen Anwesenden, auch wenn du unsere Sprache nicht sprechen magst. Ich bin übrigens Iril.“

„Ich bin Ijsdur, der Eis-Dämon“, sprach Ijsdur.

„Willkommen in der Rietburg, Ijsdur.“

„Die Rietburg? Was ist das hier für ein Ort?“

„Die größte Festung außerhalb des Grauen Gebirges. Erbaut von der Schar des legendären Königs Brandur, des Anführers der Angekommenen.“

Zum ersten Mal zeigte Ijsdur eine Regung. Er machte einen verwirrten, doch beinahe freudigen Gesichtseindruck.

„Bran-dur? Ein Eis-Dämon ist euer König?“

„Nein, warum?“, fragte Iril nun verwirrt.

Ijsdurs freudige Miene erstarb. Kopfschüttelnd antwortete er: „Nicht wichtig.“

Die Tore des Palas krachten auf und ein aufgebrachter Prinz Thorald schritt daraus hervor. Saft tropfte von seinem Bart. Trank er bereits so am Morgen? Nichtsdestotrotz blickte der Prinz herrisch um sich und griff zu seinem Schwert, als er Ijsdur erblickte.

„Was ist das hier für ein Klamauk?! Und was ist das für ein Wesen?!“

„Das ist Ijsdur, ein Eis-Dämon. Najuk fand ihn in einem Brunnen.“, flüsterte eine Wächterin ihm zu.

„Was, Eis-Dämon? Dämonen bringen nichts als Ärger. Er soll sich einen anderen Schlafplatz suchen! Oder ins Gebirge zurückkehren, wo er herkommt. Die Rietburg beherbergt schon so zu viele Leute!“

„Wer ist das denn?“, fragte Ijsdur Iril. Da er weiterhin in seiner Muttersprache sprach, blickten ihn die restlichen Anwesenden nur verwirrt an.

„Das ist Thorald, der Herrscher dieses Landes“, murmelte Iril.

„Ich dachte, der Herrscher wäre dieser König Bran-dur.“

Thorald zuckte zusammen, als er Brandurs Namen inmitten der fremden Worte dieses fremden Wesens vernahm.

„Brandur ist kürzlich gestorben“, erklärte Iril.

Obwohl dies zuvor unmöglich geschienen hatte, blickte Thorald noch verwirrter drein, als er diese Antwort hörte. Und trauriger.

„Ich verstehe“, sprach Ijsdur, „Und ich verstehe, dass ich hier nicht erwünscht bin. Ich gehe.“

„Warte“, meinte Iril, „Ich komme mit dir. Ohne mich und meine Übersetzungskünste kannst du anderen Andori vielleicht verstehen, aber sich zu unterhalten dürfte schwer werden.“

„Ich danke dir, Iril.“

Iril fragte die Anwesenden, wo jemand wie Ijsdur sonst noch unterkommen könne. Sie erhielt mehrmals dieselbe Antwort:

„Na, die Taverne zum Trunkenen Troll natürlich!“

„Die Taverne ist wie durch ein Wunder der Zerstörung durch den Drachen und seine Kreaturen entkommen.“

„In Gildas Taverne ist jeder willkommen.“

„Wenn der Eis-Dämon dort kein Zimmer findet, dann nirgendwo.“\bigskip







Still wanderten Iril und Ijsdur durchs Rietland in den Süden, in Richtung dieser berühmten Taverne.

Iril reflektierte über die Stille und darüber, wie angenehm es doch sein konnte, ruhig neben jemand anderes zu laufen, statt irgendwelche oberflächlichen Floskeln auszutauschen, die keinen von ihnen interessierte. Ihre Gedanken kreisten um den Eis-Dämon. Er war wohl am ehesten ein Naturgeist, doch ein ausgesprochen menschlicher. Hatte jemand ihn geschaffen? Hatte er einen Zweck? Im Geiste erstellte sie eine Liste der zehn dringendsten Fragen, die sie Ijsdur fragen würde, sobald er sich hier eingefunden hatte.

Auch wenn es Iril auf der Zunge brannte, seine Geheimnisse zu lüften, so glaubte sie, dass er lieber nicht als erstes mit Fragen gelöchert werden wollte, während er versuchte, sich in einem fremden Land zurechtzufinden.

So wanderten die beiden in Stille durchs Rietland.

„Schönes Wetter. Scheint in diesen Landen immer so oft die Sonne?“, sprach der Eis-Dämon auf einmal.

Iril warf ihm einen verwirrten Gesichtsausdruck zu und gab knapp Antwort: „Öfter als im bewölkten Fahlen Gebirge, würde ich meinen.“

Was wollte Ijsdur? Warum stellte er eine solch seltsame Frage?

Es dauerte kaum wenige Minuten, bis er sich erneut an Iril wandte.

„Was ist da für ein Tier?“, fragte er, seinen Zeigefinger auf ein wildes Pferd richtend. Sein Tonfall klang absolut uninteressiert.

Erneut gab Iril etwas verwirrt Antwort: „Das ist ein Pferd.“

„Kann man es reiten?“

„Ja. Die Rietgarde richtete sie ab.“

Stille.

„Geht es noch lange bis zur Taverne?“

„Nachdem, was ich gehört habe, ja.“

Stille.

„Hinter jeder Person versteckt sich eine interessante Geschichte. Was ist deine?“, sprach Ijsdur kalt. Fast so, als fühle er sich gezwungen, Konversation zu beginnen.

„Ist das so, wie man dir beigebracht hat, Gespräche zu beginnen?“, fragte Iril, „Ich glaube nicht, dass meine Geschichte besonders interessant ist im Vergleich zu deiner.“

„Verzeih, ich bin etwas aus der Übung Doch weißt du, du, solche Fragen sind typischerweise in Konversationen eine Einladung, mehr zu erzählen. Du gibst mir nicht viel Material, mit dem wir arbeiten könnten.“

„Ich weiß“, meinte Iril, „Aber was soll ich schon erzählen?“

„Was auch immer du willst. Konversationen werden begonnen und entwickeln sich danach irgendwie in Richtungen, die beiden Gesprächspartnern gefallen.“

„Sofern beide Gesprächspartner das wollen.“

„Willst du dich lieber nicht unterhalten?“

Iril blieb einen Moment lang stumm.

„Ich weiß es nicht. Vermutlich schon. Ich weiß nur ohnehin schon kaum, wie man gut konversiert, und erst recht nicht mit Fremden, und erst recht nicht mit einem lebendigen Schneemann.“

„Ich bin kein Schneemann“, sprach Ijsdur, „Mich hat niemand gebaut. Würdest du gerne wissen, was Eis-Dämonen wie ich sind? Und woher ich stamme?“

„Ich wäre durchaus neugierig“, meinte Iril, „Ich liebe Rätsel und Geheimnisse.“

„Na, dann hättest du doch fragen können“, gab Ijsdur zurück.

„Das wäre nicht höflich gewesen.“

„Ah.“

Damit war das Eis gebrochen und die Worte begannen zu sprudeln.\bigskip







„Tulgor?“, fragte Iril verwirrt. Sie hatte gedacht, alle bekannten Länder und Reiche zumindest ansatzweise zu kennen.

„Genau, Tulgor. Das Land hinter den Bergen in den Wolken. Wie nennt ihr das hier?“

„‚Bergen in den Wolken‘? Meinst du das Fahle Gebirge?“

„Ja, dieses hohe Gebirge da im Westen. Dahinter liegt unser Land.“

Iril betrachtete die hohen Berge nachdenklich. Sie waren aus hellem, fahlem Gestein, sodass man kaum erkennen, wo Stein in Schnee und Eis überging, und wo in wabernde Wolken. So lange sie zurückdenken konnte, hatten Wolken die Gipfel der Berge verhüllt, sodass lange Zeit niemand gewusst hatte, wie hoch sie wirklich waren. Bis jemand auf die Idee gekommen war, den Schatten des Gebirges zu vermessen. Nun hatten sich die ewigen Wolken gelichtet. Ob da wohl ein Zusammenhang zum kürzlichen Kampf mit und Tod von Tarok bestand?

„Ich wusste gar nicht, dass hinter diesen Bergen noch ein weiteres Reich liegt. Ich dachte, das sei nichts sei als Ödnis. Sind da alle so wie du? Kann ich mir Tulgor als Eiswüste vorstellen?“

„Im Gegenteil!“, lachte Ijsdur, „Tulgor ist ein warmes Land voller wilder Steppen und satter Felder. Das Reich eines friedlichen Volkes. Ich hingegen bin ein Eis-Dämon. Wir kommen nicht direkt aus Tulgor, sondern aus den Bergen davor, die ihr offenbar Fahles Gebirge nennt. Aus dem ewigen Eis. Einer riesigen Eisfläche in einem schattigen Tal hoch oben.“

„Beeindruckend“, murmelte Iril. Das waren eine Menge Informationen auf einmal. „Muss einsam sein im Gebirge, nicht?“

Ijsdur stockte kurz, ehe er fortfuhr. „Ich denke nicht oft über Einsamkeit nach. So gedämpft die Gefühle von Eis-Dämonen im Vergleich zu Menschen sind, so kann fehlende Gesellschaft uns dennoch schmerzen.“

„Warum schlosst ihr euch denn nicht dem Rest der tulgorischen Gesellschaft an? Schmilzt ihr, wenn ihr in zu warmen Gefilden agiert?“

„Ich hoffe nicht“, sprach Ijsdur ernst, „Doch kann ich mir nicht sicher sein. Wir Eis-Dämonen waren für Jahrtausende eingesperrt hinter einem magisch versiegelten Felsentor in unserem schattigen Tal des ewigen Eises. Erst kürzlich wurde das Felsentor geöffnet und wir waren frei.“

Ijsdur blickte hinter sich auf die schwache Schicht aus Schneeflocken, die er auf dem erdigen Weg hinterließ. „Wenn ich mir das so ansehe, mache ich mir keine großen Sorgen ums Schmelzen. Die Magie aus meiner Eiskristallkette ist mächtig und wird vom ewigen Eis gespeist. Dem geht so bald die Kälte nicht aus.“

Iril fröstelte.

Plötzlich schreckte sie ein seltsames Geräusch auf. Es war eine Art Scharren, das aus dem hohen Rietgras direkt hinter ihr kam. Iril blieb stehen und wies Ijsdur an, gleich zu tun.

„Was ist ...“, setzte der Eis-Dämon an. Er erstarrte.

Als Iril sich umdrehte, tauchte wie aus dem Nichts ein Gor auf. Zischend schwang die Bestie ihre riesigen Hornklauen.

„Futter. Futter. Fein.“, zischte der Gor und leckte sich über das Maul. Wie sprachbegabt die Vertreter seiner Spezies doch sein konnten. Und leider waren sie so gut wie nie allein unterwegs.

„Gors!“, brüllte Iril, „Wenn der erste Gor in Sichtweite ist, steht der zweite schon ...“

Ein Heulen ertönte hinter ihr. Sie wagte kaum mehr, als einen kurzen Blick auf Ijsdur zu werfen. Der kurze Blick genügte schon, um zu erkennen, dass ihr Begleiter von besagtem zweiten Gor mitten in die nackte Brust getroffen worden war.

Es sah nicht gut aus! Der Gor schnappte erneut zu und riss dem überrumpelten Ijsdur ein gewaltiges Loch in die Brust. Iril hatte gerade genug Zeit, zu registrieren, dass Ijsdurs Innenleben aus verschiedenen Schneeschichten zu bestehen schien, da sackte der Eis-Dämon auch schon zu Boden. Schneeflocken wirbelte umher und Wasser plätscherte aus seiner Wunde. Doch ehe Iril ihm zu Hilfe kommen konnte, kündigte ein Knurren hinter ihr die nächste Attacke ihres eigenen Gors an.

Einen Herzschlag lang stockte Iril der Atem und das Feuer der Furcht schoss durch ihren Körper. War dies so, wie sich ihre Mutter gleich vor ihrem Tod gefühlt hatte? Stand Iril nun dasselbe Schicksal bevor?

Ehe sie sich groß in ihren eigenen Gedanken verheddern konnte, bewegte sich ihr Körper wie von allein. Jahrelanges Training unter den Mauerbergen hatten sie auf solche Situationen vorbereitet.

Iril duckte sich unter einer unförmigen Hornklaue ihres Gegners hinweg und verbarg sich im hohen Rietgras. Der Gor stak mit seinen langen Klauen haarscharf neben ihr in den Boden. Jetzt hatte sie genug! Iril schwang ihren Runenhammer. Die mit einer Vielzahl von Runen überzogene Waffe folgte Irils Schwung und schmetterte mit der flachen Seite gegen den massigen Bauch des Gors, ehe Iril den Hammer überhaupt magisch aufladen konnte. Der Gor winselte und taumelte einige Schritte zurück. Iril warf sich auf und war bereit, gnadenlos nachzusetzen. Mit gemeinen Gors hatte sie kein Mitleid. Doch der Gor wich dem zweiten Schlag – diesmal hätte ihn die spitze Seite des Runenhammers am Kopf getroffen – haarscharf aus.

Und noch schlimmer: Der Gor packte Iril Hammer mit einer Hornklaue und riss ihn weg von ihr. Mit seiner schieren Kraft konnte sie nicht mithalten. Iril ließ den Hammer los und tappte auf ein Tattoo an ihrem Arm. Eine wohlige Wärme kitzelte ihre Hand, welche rötlich aufleuchtete. Ebenso leuchtete der fortgeschleuderte Hammer, auf welcher nun mitten in der Luft eine Drehung vollzog und zu Iril zurückzufliegen kam. Auf dem Weg zurück in ihre ausgestreckte Hand haute er den Gor nach vorne. So taumelte der Gor zu Boden, sein ungeschützter Nacken offen zugänglich. Ein weiterer Schlag gab der unsäglichen Kreatur den Rest.

Iril wirbelte herum. „Ijsdur, halte dich von seinen Klauen fern und versuche, ihn von hinten anzugreifen! Sein Nacken ...“

Doch, ehe sie fertig sprechen oder gegenüber Ijsdurs Gor gewalttätig werden konnte, hatte Ijsdur sich schon um diesen gekümmert. Stolz und unversehrt ragte der Eis-Dämon vor seinem Gegner auf. Keine Spur mehr der tiefen Wunde, die der Gor in ihn gerissen hatte. Ein helles Licht blitzte aus Ijsdurs ausgestreckter Hand auf und ein längliches, gezacktes Objekt – war das ein riesiger Eiszapfen? – raste auf des Gors garstiges Gesicht zu. Getroffen wurde er zurückgeschleudert, sein ganzer Körper mit Eiskristallen bedeckt. Auch dieser Gor rührte sich nicht mehr.

„Interessante Wesen“, bemerkte Ijsdur, „Sehen sie nicht, dass es in ihrem Interesse wäre, zu fliehen?“

„Elend, diese Kreaturen“, spuckte Iril aus, „Sie leben beinahe überall, wo es Zivilisationen gibt. Sie dienen den Drachen. Ich hatte gehofft, mit dem Tode Taroks wären sie nicht mehr auf die Unschuldigen aus. Aber alte Angewohnheiten sterben schwer und der Hunger auf Fleisch scheint sie immer noch anzutreiben.“

„Von Zeit zu Zeit verirrte sich die eine oder andere Kreatur aufs ewige Eis“, berichtete Ijsdur, „Vielleicht sind gar ein paar von ihnen zu Eis-Dämonen geworden. Aber zumindest diese Gors stehen uns in Geistesgaben nichts nach.“

Iril starrte weiterhin verwundert auf Ijsdurs so unversehrte Brust. „War da nicht zuvor noch eine tödliche Verletzung?“

„Oh, was tödlich ist, hängt immer vom Körper ab. Mein Leib aus Schnee wird wohl nie so kräftig oder stabil sein wie dein Gebilde aus Knochen, Fleisch, und all dem Zeug, doch vermute ich, Verletzungen erheblich rascher und einfacher reparieren zu können. Weg ist natürlich nichts. Bis die Schmerzen nachlassen, könnte es noch eine Weile dauern. Aber Schmerzen kann man gut ignorieren.“

Iril nickte stumm. Ehe sie weiter nachhaken konnte, rasten ihre Gedanken weiter zu Ijsdurs Hand, aus welcher zuvor ein eisiger Blitz den Gor mit einem Schlag überwunden hatte. „Faszinierend, dieser Eisblitz. Was war das? Wie kreierst du die?“

„Ich weiß nicht“, druckste Ijsdur herum.

„Oooh, ist das ein Geheimnis? Ich liebe Geheimnisse.“

„Leider kein Spannendes. Diese Eisblitze lösen sich einfach irgendwie von mir. Wenn ich es will. Und die Kraft dazu habe. Die Eiskristallkette um meinen Hals ist der Pfad zu mannigfaltigen Fähigkeiten.“

Iril fröstelte es erneut, und nicht nur wegen der Nähe zu Ijsdurs eiskaltem Körper. Während die Aufregung des überraschenden Gorkampfs versiegte, haderte sie auf einmal mit dem Gedanken, ob es klug gewesen war, allein an der Seite dieses Fremden ins Land zu ziehen. Nicht, dass sie nicht auf ihre kämpferischen Fähigkeiten oder auf das Urteil des Ewigen Feuers vertraute. Doch ward ihr nun bewusst, dass die Magie des undurchsichtigen Eis-Dämons ihre eigene in Kampfkraft durchaus übersteigen konnte. Dass das Ewige Feuer eigentlich nur Gefahren für die Rietburg sondierte, nicht für kleine Runenmeisterinnen in fremden Landen. Und dass sie noch keine Ahnung von den Motiven des Eis-Dämons hatte.\bigskip







Die Runenmeisterin und der Eis-Dämon schritten weiterhin Seite an Seite in Richtung Taverne, auf ausgetretenen Trampelpfaden zwischen goldenen Rietgras-Feldern voller rosa Rietgras-Blüten. Iril schritt leise voran, mit schnellen Schritten, während ihr Kopf ratterte. Ijsdur glitt langsam voran, wie üblich eine Schicht aus Schnee und Eis hinter sich herziehend – auch wenn Iril glaubte zu sehen, dass die Schicht dünner war als auch schon.

Irgendwann konnte sie sich eine vorsichtige Frage nach Ijsdurs Motiven nicht mehr verkneifen: „Verzeih, dass ich so direkt frage: Was suchst du hier in Andor eigentlich?“

Ijsdur stutzte einen Augenblick und sprach dann so tonlos wie immer: „Wenn ich das wüsste. Endlich bin ich frei von diesem Schattental hinter dem Felsentor. Doch meine Zukunft ist ungewiss. Ich weiß, dass ich in Tulgor keine Zukunft habe. Und dass die vielen einsamen Nächte hoch oben im Kuolema-Gebirge taten mir nicht gut. Ich hoffte, hier auf eine gewisse tulgorische Reisegruppe zu stoßen, doch hörte ich von dir, dass die Tulgori noch gar nicht hier eingetroffen sind.“

„Zumindest ich hätte noch nichts von ihnen gehört“, entschuldigte sich Iril.

„Äußerst eigenartig“, murmelte Ijsdur, „Vielleicht überholte ich sie auf dem Weg unter dem Gebirge hindurch. Die Stollen unter dem Kuolema-Gebirge hindurch sind ein Labyrinth und als Eis-Dämon kann ich einzigartige Abkürzungen schaffen. Wie dem auch sei, so werde ich nun hier auf ihre Ankunft warten. Früher oder später werden sie doch aufkreuzen müssen.“

„Hoffentlich vereisen sie uns nicht ein zweites Mal den Brunnen“, lachte Iril.

„Keine Gefahr“, meinte Ijsdur, „Ganz abgesehen davon, dass keine Eis-Dämonen mit ihnen reisen. Ich wanderte nach dem Austritt aus dem Gebirge einfach ziellos durch die Lande, sah die Rietburg in der Ferne und beschloss, ihr näher zu kommen und in dieser stärkenden Wasserquelle zu nächtigen, um niemanden zu wecken. Ich weiß, dass Menschen ungern aus dem Schlaf gerissen werden. So, wie ich diese Reisegruppe kenne, werden sie sich eher getrauen, einfach an das Tor zu klopfen.“

„Siehst du dich nicht mehr als Menschen?“

„Ich war nie einer. Ich hoffe, dass ich die richtige Entscheidung traf, indem ich hierherkam. Hier könnte ich neue Kontakte knüpfen. Eine neue Existenz aufbauen. Eine Bestimmung finden.“

„Klingt ganz ähnlich wie ich. Auch ich fühle mich zuhause nicht mehr zuhause und suche nun hier nach einer neuen Bestimmung.“

„Dann sitzen wir ja im selben Boot.“ So tonlos wie oft zuvor hängte Ijsdur an: „Es freut mich außerordentlich, dich getroffen zu haben, Iril.“

Iril bedachte die gruselige Macht von Ijsdurs Eisblitzen nach dem Kampf gegen den Gor. Und wie unüberwindbar ein schnell heilender Körper aus Schnee doch sein konnte. Nicht, dass ihre eigene Macht über die Runen nicht auch gruselig sein konnte. Oder dass sie das grundlegende Misstrauen gegenüber Fremden, das so vielen Schildzwergen eigen war, für gut hielt. Doch schüchterte sie Ijsdurs eisige Art immer mehr ein.

Sie erinnerte sich an den bösartigen Eis-Troll, von der diese Bewahrerin in der Rietburg erzählt hatte. Und daran, dass Ijsdur erzählt habe, die seinen während seit Jahrtausenden in diesem Tal des ewigen Eises im Fahlen Gebirge eingesperrt gewesen. Und dann nannte er sich selbst auch noch ‚Eis-Dämon‘. Natürlich war das nur eine ungefähre Übersetzung des originalen Wortes in seiner Sprache. Vermutlich trug es in seiner Heimat keine negative Konnotation.

Erneut bedachte sie den pupillenlosen Blick aus seinen milchigen Augen. Ein Blick, den er sich mit den bösartigen Kreaturen teilte, die dank diesen Augen auch in der tiefsten Finsternis der Nacht unschuldige Opfer erspähen konnten.

Iril packte ihren Hammer ungewollt fester und fragte betont unschuldig: „Darf ich fragen, warum ihr Eis-Dämonen hinter diesem Felsentor eingesperrt wart?“

Ijsdur hielt kurz inne, antwortete dann jedoch wie aus der Balliste geschossen: „Das war ein ungerechtfertigter Fluch. Die willensstarke Kaiserin von Tulgor – damals war Tulgor nämlich noch ein Kaiserreich – fürchtete uns Eis-Dämonen. Ihr Adoptivsohn war in einer bitterkalten Nacht im Gebirge erfroren. Sie hatte uns die Schuld gegeben. Und wie königliche Flüche halt so sind, löste sich unserer erst Jahrtausende später auf.“

Iril nickte. Als Silberzwergin wusste nur allzu gut darüber Bescheid, was königliche Flüche ausrichten konnten. Narkon war auch Jahrzehnte nach Varatans Tod noch immer eine verdammte Falle, voller wahnsinniger Piraten, wilder Hautwandler und wütender Kreaturen. Und niemand wusste, wann sich der Fluch des verstorbenen Seekönigs lichten würde.

Ijsdur betrachtete Iril mit schiefgelegtem Kopf und fügte an: „Keine Sorge. Wir Eis-Dämonen wurden nicht für vergangene Verbrechen eingesperrt oder so. Wir haben den Sohn der Kaiserin nicht retten können, aber ihn auch nicht umgebracht. Wir sind unschuldig. Wir sind keine Gefahr für die Öffentlichkeit. Wir wollen Andor nichts Böses.“

Iril schluckte fest und löste ihren Runenhammer von ihrem Gürtel. Diesen Trick hatte sie schon einmal von einem der hohen Runenmeister eingesetzt gesehen. Sie musste nur überzeugend genug auftreten.

Vorsichtig wählte sie ihre Worte und log: „Ijsdur? Weißt du, dass meine Übersetzungsrunen nicht nur die Bedeutung gesprochener Worte übertragen können? Sie geben mir manchmal auch ein Gefühl der Intention des Sprechers dahinter. Oft nur klein, fein, beinahe vergesslich. Aber Lügen kann ich mit meinen Runen von Meilen her riechen. Und ich rieche Lügen in deinen Worten. Was willst du wirklich hier?“

Iril blieb stehen und beobachtete Ijsdurs Mimik, suchte nach irgendeiner Reaktion, die ihr verraten könnte, ob ihr Misstrauen gerechtfertigt war.

Ijsdur blieb ebenfalls stehen und blickte Iril einige Augenblick lag vollkommen ausdruckslos an.

„Schade“, murmelte er zu sich selbst, „Ich begann gerade erst, dich zu mögen.“

Dann stürzte er sich auf Iril, ohne auch nur mit der Wimper zu zucken.\bigskip







Ijsdur griff nach Irils Armen und ließ nicht los. Sein fester Griff war eiskalt und kribbelte auf Irils tätowierter Haut. Er öffnete seinen Mund und hauchte aus. Eisiger Dampf schwallte auf Iril Kleidung herunter und überzog sie mit einem Reif aus Frost.

„So halte schon still, dann überlebst du das vielleicht“, sprach Ijsdur.

Iril drehte ihre Arme in Richtung seiner Daumen und löste sich mit Mühe von ihm. Erschreckt nahm sie wahr, dass ihre Haut unter seinem Griff schon leicht bläulich angelaufen war. Sie schlug einen Purzelbaum an Ijsdurs strammen Beinen vorbei, während sie seinen Tritten auswich. Definitiv feindlich eingestellt, dieser Dämon. Sie machte ihren Hammer bereit. Ihr ganzer Körper schmerzte und protestierte. Das Herz schlug ihr bis zum Halse. Und es knirschte unschön in ihrer Reisetasche, über die sie soeben gerollt war.

Zwei überraschende Kämpfe gleich hintereinander, das hatte sie schon lange nicht mehr erlebt! Hoffentlich war sie nicht allzu eingerostet.

Ijsdur drehte sich zu ihr um und hob seine eigentlich leere Hand. Etwas knisterte und glomm darin. Ehe er etwa erneut einen Eisblitz schleudern konnte, versenkte Iril ihren Hammer mit der Spitze voran in Ijsdurs Kniekehle. Beim Kontakt entluden sich aufgesparte magische Ströme in farbigen Lichtblitzen. Der Hammer glitt beinahe mühelos durch beide Beine des Eis-Dämons hindurch und pulverisierte sie zu Schneehäufchen. Ijsdur fiel flach auf seinen Bauch. Doch Iril wusste vom Gorkampf noch, dass Ijsdur eine solche Verletzung problemlos davonsteckte. Also setzte sie nach. Ein weiterer Schlag des Hammers hinterließ eine beachtliche Delle in Ijsdurs Rücken. Schnee und Eis spritzten umher.

Iril durfte ihn keinen Eisblitz generieren lassen. In der starken Magie seiner Eiskristallkette schien seine wahre Stärke zu liegen. Also musste sie rasch handeln, so rasch, dass er sie nicht gegen sie einsetzen konnte. Schnell, jetzt!

Zwei weitere Schläge, und Ijsdurs Hände waren Vergangenheit. Temporär. Hoffentlich konnte er nur von dort aus Eisblitze schleudern. Sie ließ von ihm ab, öffnete ihre Reisetasche und sondierte verzweifelt ihre verschiedenen metallenen Runenscheiben.

Ijsdur wand sich neben ihr am Boden. Während sein Körper sich wieder zusammenschichtete, machte sein Kopf eine 180-Grad-Drehung um seinen Hals und starrte Iril durchdringend an. Die Delle, die ihr Hammer in seinem Rücken hinterlassen hatte, begann, sich magisch zu schließen. Ächzend richtete der Eis-Dämon sich wieder auf.

Hektisch durchsuchte Iril ihrer Tasche nach einer bestimmten Runenscheibe.

Gefunden!

Sie sammelte Magie im Hammer und tippte die Scheibe an – silbern glitzernde Lichter flackerten umher – und versenkte sie tief in der Delle in Ijsdurs Rücken. Schnee wuchs darüber, doch darum kümmerte Iril sich nicht groß. Einmal aufgeladen, brauchte die Scheibe kein Sonnen- oder Mondlicht, um ihre Wirkung zu entfalten.

Unter den von den Runenmeistern der Silberzwerge am häufigsten genutzten Runenfolgen gab es eine bestimmte, welche verspukte Höhlengänge von verlorenen Seelen zu reinigen vermochte – denn als erstmals Minen in den Silberberg gegraben worden waren, hatten sich überraschend viele ungewollte Geister aus vergangenen Jahrhunderten gezeigt und die Zwerge zu vertreiben versucht. Seitdem gehörte eine solche Runenscheibe zur Vertreibung fremder geistiger Einflüsse zum Standardrepertoire eines jedes Runenmeisters, auch wenn sie in letzter Zeit nur noch selten mit verirrten Seelen zu kämpfen hatten.

Ijsdur stürzte in sich zusammen, kaum hatte Iril die Scheibe in ihn gesteckt. Also hatte die Runenscheibe ihre gehoffte Wirkung gehabt und Ijsdur den Geist ausgetrieben. Was für ein Glück.

Traurig betrachtete Iril die leblosen Klumpen Schnee vor ihr, der immer noch vage die Form eines menschlichen Leibes hatte. Es wäre wirklich faszinierend gewesen, dieses seltene Phänomen der Natur zu studieren, ihm alle seine Geheimnisse zu entlocken. Doch hatte Ijsdur sie angegriffen und damit seine üblen Absichten für dieses Land kundgetan. Vermutlich war es besser so.

Iril war kurz davor, sich abzuwenden und zurück zur Rietburg aufzubrechen, da bewegte sich der Schneehaufen vor ihr.

Langsam setzte sich Ijsdur auf und hielt seinen Kopf. Seine leicht zerfallene Form festigte sich wieder. Die Eiskristalle an der Kette in seiner Brust glitzerten in allen Regenbogenfarben. Fahle Hände tasteten den Bauch ab, in welchem Iril ihre Geister abwehrende Runenscheibe versenkt hatte. Eisige Augenlider öffnete sich und zeigten pupillenlose Augen.

Starr blickten Iril und Ijsdur einander an. Iril war verwirrt. Falls Ijsdur, wie sie gedacht hatte, quasi eine einen Schneehaufen besessende Seele aus einer anderen Sphäre der Realität wäre, sollte die Scheibe ihn eigentlich in diese andere Sphäre zurückgeschickt haben. Und falls er eine Puppe einer fremden Entität des ewigen Eises gewesen wäre, hätte die Scheibe ihn aus der Kontrolle der fremden Entität befreit. Dass er noch hier war, dass er sich noch bewegte, bewies, dass es sich bei ihm um ein lebendiges, fühlendes Wesen handelte, welches in diese Ebene der Realität gehörte. Wie ein Naturgeist? Doch konnte Iril sich nicht an einen einzigen Naturgeist erinnern, der so eloquent wie Ijsdur gewesen wäre. Welch Wunder barg dieser Eis-Dämon denn noch?

Zu schade, dass sie sie nicht erforschen konnte, wenn ihr ihr eigener Leib und Leben wichtig waren. Iril ließ erneut Magie in ihrem Hammer ansammeln und bereitete sich auf einen mächtigen Schlag vor.

„Nein!“, rief Ijsdur, und hob seine Hand, „Ich danke dir, liebe Iril, von ganzem Herzen, für was auch immer du tatst! Die Stimmen der Vergangenheit sind verklungen. Ich spüre den Willen der Herrin des ewigen Eises nicht mehr. Ich bin frei! Wirklich frei!“

Iril ließ ihren Hammer nicht sinken.

„Ich will dir nichts Böses. Auch diesem Land nicht! Fühle die Wahrheit meiner Worte!“

„Böse und Gut sind sehr relative Begriffe“, knurrte Iril, „Magst du noch spezifizieren? Und erinnere dich daran ...“

„... dass du Lügen spüren kannst, natürlich.“

Ijsdur holte tief Luft – wohl eher theatralisch, denn groß atmen zu müssen schien er nicht – und ratterte herunter:

„Ich will, dass du weder leidest noch stirbst, Iril. Du und alle anderen Lebewesen in Tulgor und diesen Landen. Ich will Siantari nicht mehr bis in alle Ewigkeit dienen. Ich will nicht mehr heimlich diese östlichen Reiche erforschen, um die Schwachstellen dieses Reichs zu erkennen. Ich will nicht mehr die tulgorische Reisegruppe abfangen, ehe sie diese Reiche vor Siantari warnen könnten. Ich will nicht mehr eine Eiswüste über diese Welt verbreiten!“

„Das klingt ja so, als hättest du bis vor wenigen Augenblicken einen relativ ausgeklügelten finsteren Plan gehabt. Oder lese ich da zu viel hinein?“

„Keineswegs“, gab Ijsdur zu, „Wie alle Eis-Dämonen des Kuolema-Gebirges habe ich oft die Stimme Siantaris in meinem Hinterkopf vernommen. Die Stimme der Herrin des ewigen Eises. Und die wirren Rufe ihrer Vorgänger. Dies war ihr Reich. Sie und alle Tari vor ihr waren nicht ohne Grund hinter dem Felsentor eingesperrt worden. Sie wollten die ganze Welt in eine Eiswüste verwandeln. Und ich hätte bis soeben nicht einmal ihre Stimme hören müssen, um mir sicher zu sein, dass es mein Schicksal sei, das ewige Eis in ihrem Namen weiter auszubreiten.“

„Und das ist es nun nicht mehr?“

Ijsdur fiel auf seine Knie. Leise flüsterte er: „In all dem Lärm, in all den Stimmen der Vorgänger, in alle dem konnte ich die Erinnerungen nie richtig lesen. Die Erinnerungen von Ijs, dem im Eis gestorbenen Junge, dessen Körper und Namen ich trage. Ich konnte seine Erinnerungen sondieren und doch deutete ich sie falsch. Unterdrückte, was seine Gefühle mir längst hätten sagen sollen. Deine Runen haben die Stimmer der Tari verstummen lassen. Nun, wo sie stumm sind ...“

Ijsdur schluckte, ehe er fortfuhr.

„Die Jahre im ewigen Eis waren keine schöne Zeit. Ich fühlte mich nicht gut. Verwirrt, rastlos, ziellos. Ijs‘ Jahre in Tulgor hingegen, so verblasst sie in meinen Erinnerungen auch sind, sind so viel schöner. Voller. Angenehmer. Freudiger. Es ist eigenartig. Die Tulgori bedeuten mir nichts. Du bedeutest mir nichts. Ihr alle werdet sterben, lange bevor ich vergehen werde. Und doch ... ich ... ich will Tulgor nicht zu einer Eiswüste werden lassen. Ich will nicht sehen, wie du im Eis erfrierst.“

„Wie beruhigend.“

„Es ist wahr. Eines Tages werde ich sterben. Und ich fühle mich nicht wohl beim Gedanken, das allein zu tun, in einer eisigen Welt. Lebende haben so viel mehr zu bieten.“

Stumm blickten die beiden Kämpfenden einander an.

„Was tun wir nun?“, fragte Ijsdur.

„Was willst du tun?“

„Wenn ich das wüsste. Ich muss zunächst einmal meinen Kopf ins Klare kriegen. Meine Gedanken in eine passende Reihenfolge bringen.“

Stille.

Iril sprach als erste wieder: „Wie wäre es, wenn wir wie vorhin gewollt zu diesem Wirtshaus weiterreisten? Dort können wir weiterschauen, wie er mit dir weitergehen soll.“

Ijsdur nickte.\bigskip







Im Vorbeigehen öffnete Iril ihre vom Kampf lädierte Reisetasche und nahm Inventar auf. Ihre Sammlung an Runenscheiben war größtenteils ganz. Manche der Metallscheiben waren etwas verbogen, aber nichts Schlimmeres. Mit einer Ausnahme.

Der steinerne Runenring mit eingelassenem Glas zur Fernsicht war unrettbar zerbrochen. Zuletzt hatte Iril ihn genutzt, um Taroks Angriff auf die Helden aus der Ferne zu beobachten.

„Einfach nichts schlimmeres“, murmelte sie, um sich selbst zu beruhigen, „Wir können nur hoffen, dass wir in den nächsten Tagen nicht dringend einen Fernblick bräuchten.“

„Wir können hoffen“, bestätigte Ijsdur.

Leise fügte er an: „Es tut mir leid, Iril. Schon zum zweiten Mal heute hast du mir enorm geholfen. Ich verstehe das Konzept von Schuld nicht ganz, doch glaube ich, mich tief in der deinen zu befinden.“

Iril winkte ab.

Ijsdur beugte sich nochmal vor, guckte ihr tief in die Augen und bekräftigte: „Danke, Iril. Einfach nur Danke.“

Er erstarrte für einen kurzen Augenblick. Die um ihn herumwirbelnden Schneeflocken wurden etwas schneller, als er klirrend weitersprach. „Siantari! Ich vergaß sie völlig. Die Herrin des ewigen Eises ist auf dem Weg hierher. Sie verließ als erste von uns das ewige Eis. Und sie hat immer noch vor, die gesamte Welt unter einer dicken Schicht aus Eis verschwinden zu lassen. Wenn wir die hier lebenden Wesen wahren wollen, müssen wir sie vor Siantari warnen.“

„Das ist schon in Ordnung“, meinte Iril, „Wir warnen die Andori einfach vor. Die Helden von Andor sind mächtig. Sie haben einen Drachen niedergerungen. Da werden sie hoffentlich mit einem schnöden Schneeklotz klarkommen können. Nicht persönlich gemeint.“

„Keine Beleidigung angekommen“, versicherte Ijsdur, „Doch unterschätzt Siantari nicht. Sie ist eine formidable Gegnerin. Wenn du nicht ihren Bann über mich gelöst hättest, würde ich aktuell immer noch, ohne mit der Wimper zu zucken, die Ausbreitung des ewigen Eises über das Leben aller Menschen stellen, die ich je kannte.“

„Wie kann es sein, dass Siantari hier noch nicht eingetroffen ist?“, fragte Iril, „Ist sie vielleicht schon hier? Heimlich?“

„Heimlichkeit ist nicht Siantaris Stärke. Sie wollte zunächst, dass ich die Lage sondiere und ihr vielleicht die Möglichkeit einer Täuschung verschaffe, indem ich die von den Eis-Dämonen Wissenden Tulgori abfange. Bevor du meinen Geist abschirmtest, muss sie zumindest im Hinterkopf meine Erfahrungen wahrgenommen haben. Sie weiß, dass diese Bewahrer von Eis-Dämonen gehört haben. Und sie wird annehmen, dass du mich vernichtet hast. Vielleicht versteckt sie sich irgendwo und plant etwas.“

„Was könnte sie planen?“

„Keine Ahnung. Ihre Fähigkeiten sind mannigfaltig, doch haben sie allesamt mit Eis und Schnee zu tun.“

„Also können wir schließen, dass uns keine Gefahr droht, bis der erste Schnee fällt?“, fragte Iril.

„Oder der erste Schnee fällt erheblich früher als üblich. Aber Siantaris bisherige Abwesenheit muss nichts bedeuten. Tulgor liegt erheblich näher Felsentor zum ewigen Eis als Andor. Und der Weg durch die Temm-Pfade unter dem Kuolema-Gebirge hindurch ist erheblich schneller als die mühsame Gratwanderung über die Berge. Insbesondere wenn man wie ich die Abkürzungen zwischen den labyrinthischen Temm-Pfaden nehmen kann, statt wie sie über hohe Gipfel und tiefe Täler zu spazieren, wäre es kein Wunder, wenn ich schneller hier landete, als sie es tat.“

Stille.

Iril brach sie: „Die Tulgori abzufangen, das war ein direkter Befehl von Siantari? Der einzige, dem du folgtest?“

„So ist es.“

„Warum dann so friedlich auf uns zugekommen? Warum versuchen, in der Rietburg Anschluss zu finden, und gleich wieder zu gehen, nachdem man dich nicht herzlich willkommen heißt?“

„Ich wollte die Lage nicht eskalieren lassen. Von euren Verteidigungsmöglichkeiten herausfinden, aber nicht, indem ihr sie auf mich anwendetet.“

„Und du bist dir sicher, dass du dir nicht einfach eine Ausrede suchtest, um nicht sofort mit dem Töten und Morden anzufangen?“

„Ziemlich. Nur mein Ziel bedeutete mir etwas. Die Ausbreitung des ewigen Eises. Der Kampf gegen die ewige Hitze.“

„Warum tötetest du mich dann nicht einfach, nachdem ich misstrauisch geworden war? Ein Eisblitz hätte mich komplett erledigt. Aber nein, du versuchtest stattdessen bloß, mich festzuhalten.“

Ijsdur schwieg eine Zeit lang. „Ich hatte noch nicht genug Zeit zum Überlegen. Ich musste vorsichtig sein. Wenn du verschwändest, direkt nachdem du an meiner Seite ins Rietland zögest, wären Leute misstrauischer geworden. Vielleicht hätte es einen Weg gegeben, dich auf meine Seite zu ziehen. Du hättest eine ausgezeichnete Eis-Dämonin abgegeben. Runenmagie an unserer Seite hätte Siantari stark geholfen.“

Iril kicherte. „Du sagtest, du seist kein Mensch mehr.“

„Ich war nie einer.“

Iril grinste. „Und doch ist es ausgesprochen menschlich, sich für das eigene Verhalten nachträglich halbgare Begründungen auszudenken.“

„Pfff“, machte Ijsdur, „Wenn ich dich aus anderen Gründen verschont hätte, wäre mir das nicht bewusst?“

„Es ist zumindest möglich, dass du dir etwas einredest, damit dein Verhalten nicht deinem Selbstverständnis widerspricht. Vielleicht konntest du doch nicht so rasch zum Morden überwinden, wie du gerne gehabt hättest.“

„Vielleicht“, murmelte Ijsdur, „Du warst die erste Person seit langem, mit der ich interagierte, ohne direkt verachtet zu werden. Und selbst du hattest kein Problem damit, mir zu misstrauen. Du magst im Recht gewesen und ich freue mich, dass du mich von Siantaris Einflüssen befreit hast. Und doch schmerzt es.“

Ijsdur hielt an und blickte Iril mit starrem, pupillenlosem Weiß tief in die Augen.

„Ich bin kein starrer Eisklotz. Ich habe Gefühle. Vertraue mir. Es bringt ja ohnehin nichts, dich anzulügen.“

Iril konnte sich nicht dazu überwinden, Ijsdur hier und jetzt zu beichten, dass sie ihn da angelogen hatte.\bigskip







In stille Gedanken versunken, streiften Iril und Ijsdur weiter in Richtung Taverne. Links und rechts von ihrem Trampelpfad zogen Felder andorischer Bauern vorbei. Die wenigsten waren der Zerstörung durch den Drachen vollständig entkommen. In der nächsten würden sich Andori vor allem aufs Jagen und Sammeln von Nahrung, auf Fischfang und Apfelnusspflücken verlassen müssen.

Dann, endlich, tauchte die Taverne hinter einer Hügelkuppte auf. Rauch stieg von der gemütlichen Stube auf. Aus dem Kamin. Dort, wo Rauch üblicherweise aufsteigen sollte. An der Seite des Gasthauses war ein Anhängeschild zu erkennen, auf das jemand das Antlitz eines Trolls geritzt hatte.

„Die ‚Taverne zum Trunkenen Troll‘“, murmelte Iril leise, „Schöne Alliteration. Was da wohl für eine Geschichte dahinter steckt?“

„Verwirrend“, murmelte Ijsdur vor sich hin, „In meiner Sprache ist dies auch Alliteration. Und in Thelot, einer Stadt in Tulgor, gibt es eine bekannte Taverne, die ebenfalls eine solche Alliteration zum Titel trägt. Die ‚Taverne zum Tauchenden Takuri‘. Kann dies ein Zufall sein?“

„Was?!“ Iril machte große Augen. „Das ist auch in der andorischen Sprache eine ... Wie kann das sein? Und in Werftheim ... ah, das sagt dir natürlich nichts ... das ist eine Nebelinsel ... also, eine Insel im Hadrischen Meer ... also, der Ozean im Norden ... da studierte ich ... verzeih, lass mich von Anfang beginnen:

Der Drang, möglichst alles über die Macht der Runen zu erforschen, führte mich vor einiger Zeit in den Norden. Als Runenmeisterin verschrieb ich mich in Silberhall der Lehre der Runenmagie. Ich studierte nicht nur unter Burmrit, der größten Runenlehrmeisterin der Silberzwerge, sondern auch einige Jahre lang unter einigen menschlichen Lehrmeistern auf einer nahe gelegenen Insel namens Werftheim. Die einst so geschäftige Hafenstadt der Insel leidet unter den stetigen Angriffen der Meereskreaturen, doch noch heute beherbergt sie neben der besten Schiffswerft des Nordens auch exzellente Akademie, an der Physikusse ausgebildet werden, wo man auch einige vergangene Schriften der ersten Runendruiden einsehen kann. Während ich dort lebte, kehrte ich gelegentlich in einer Taverne ein und genoss die berühmten Werftheimer Spezialitäten.

Diese Taverne in Werftheim hatte ebenfalls so einen alliterativen Namen. Die ‚Taverne zur Tanzenden Tare‘. Falls du dort mal vorbeikommst, lass die Gelegenheit nicht aus, dort einzukehren. Nebst einer ausgezeichneten Gerüchteküche zu allem Zwielichtigen, das in den stürmischen Gefilden des Nordens vorgeht, bieten die einem dort auch unglaublich erfrischende Massagen mit erhitzten Runensteinen an. Auch wenn einem Eis-Dämon wohl eher ein eiskaltes Eisbad gefallen würde? Egal. Es wunderte mich schon, mit diesem Gasthaus zum Trunkenen Troll einen weiteren solchen alliterativen Tavernennamen zu hören. Nun, wo du noch von Tulgor erzählt hast ...“

„Das kann kein Zufall sein“, gab Ijsdur zurück, „Zumindest wäre es sehr unwahrscheinlich.“

„Doch was für eine Verbindung könnte dahinterstecken? Gut möglich, dass die Tavernen in Sturmtal und in Andor zum Beispiel denselben Rumhändler haben, und dass die eine ihren Namen von der anderen abguckte. Doch Tulgor?! Die beiden Länder wussten doch vor wenigen Minuten noch nichts voneinander, geschweige denn von den anderen Sprachen. Das macht eine kausale Beziehung unmöglich.“

„Das würde ich nicht zwingend sagen“, sagte Ijsdur, „Wir Tulgori wussten zumindest vage, dass auf der anderen Seite des Gebirges Leute lebten. Aber auch gehörnte Bestien und allerlei andere Gefahren. Wir blieben lieber für uns. Doch der eine oder andere Temm wird bestimmt schon hierhergereist sein.“

„Temm? Dieser Begriff sagt mir nichts.“

„Es sind quasi kleine buckelige Menschen. Sie kleiden sich oft in braune Umhänge. Sehr langlebig. Tendieren zu magischen Begabungen. Aber sie werden wohl kaum gesehen, wenn sie das nicht wollen.“

„Ich glaube, ich könnte schon so einen gesehen haben! Wrort hieß er. Er lungerte nach dem Kampf um die Rietburg herum und half einigen Verletzten. Du sagst, der wäre aus Tulgor gekommen?“

„Nun, er hätte das sicher nicht an die große Glocke gehängt. Doch soweit ich weiß, stammen alle Temm ursprünglich aus Tulgor.“

„Und wie genau weißt du das?“

„Nicht schlechter als du.“

„Vertrauenserweckend.“

„Ironie.“

„Angebrachte. Ah, da wären wir schon bei der Taverne. So schnell kann die Zeit vergehen, wenn man schön konversiert, statt sich gegenseitig abzuschlagen“, meinte Ijsdur, „Solche Unterhaltungen habe ich im ewigen Eis vermisst.“

Iril sagte nichts. Sie lächelte aber.

Dann traten die beiden zur grölenden Menschenansammlung vor der Taverne zum Trunkenen Troll.









\newpage
\section{Alte und Neue Helden}

Vor der Taverne zum Trunkenen Troll saß eine alte Frau mit einigen anderen Bauern an einem Tisch. Als Iril an ihr vorbeilief, zückte die Alte einen Becher mit einigen Münzen darin und hielt ihn Iril klappernd vor die Nase.

„Eine Spende für den Wiederaufbau unserer Kate, die Dame? Wir können uns weder die Werkzeuge noch die Hilfskräfte leisten ... oh, was seid Ihr denn für ein Eisklotz?“

„Das ist Ijsdur. Lasst Euch nicht verschrecken, er ist ganz lieb. Neuerdings.“

Iril ließ eine Münze in den Becher der Bäuerin fallen.

Da ertönte ein Klopfen aus dem Inneren der Taverne. Iril stellte sich auf ihre Zehenspitzen und erblickte durch milchiges Glas hindurch ... Orfen, den Wolfskrieger! Der grauhaarige Held blickte zur alten Bäuerin und sprach dumpf durch das Glas:

„Ich hab‘s dir doch schon oft gesagt, Runga: Du musst kein Geld sammeln. Wir Helden werden Euch mit dem Wiederaufbau helfen. Bald, wenn sich die Dunklen Kreaturen etwas beruhigt haben. Und nachdem wir bei den Schildzwergen ordentliche Unterstützung eingeholt haben. Ich kenne einige der Zwerge gut, eine der Wachen am nördlichen Eingang ist ein alter Gefährte von mir! Es sind raue Zeiten, doch werden sie uns helfen.“

Dann blickte Orfen zu Iril herunter und sprach: „He, dich habe ich doch schon mal gesehen. Warst du nicht beim Gestürm um die Drachenleiche dabei? Na, wie gut ist deine Konstitution? Wenn du mich unter den Tisch trinkst, zahle ich dir die Runde!“

„Du schuldest mir doch bereits Gold für drei Runden“, ertönte eine glockenhelle Stimme. Eine rothaarige Dame mit einem mit Metgläsern überfüllten Tablar in der einen Hand und einem breiten Lächeln auf dem Gesicht tauchte hinter Orfen auf und schnippte dem Wolfskrieger spielerisch ans Ohr. Orfen blickte sie an und seine üblicherweise von mürrischen Falten durchzogene Miene glättete sich. Er flüsterte der Wirtin etwas zu, was Iril im allgemeinen Lärm der Tavernengäste nicht genau wahrnehmen könnte. Die Wirtin lachte und schlug sich mit der freien Hand auf den Schenkel.

Iril und der mit den vielen Leuten etwas überfordert wirkende Ijsdur schritten weiter zur Eingangstür und betraten die warme Taverne. Der leckere Geruch heißer Drachenbohnensuppe schlug ihnen entgegen, akustisch unterlegt mit einem Sammelsurium an kaum verständlichen Gesprächsfetzen, manche davon gesungen. 

% [v23.1:] Gleich neben der Eingangstür saß eine mysteriöse Person in einem grünen Gewand alleine an einem Tisch. Sie trug eine Sternkrautblüte im langen goldenen Haar und mampfte nachdenklich an einer blauen Staude mit kleinen runden Beeren. Blaubachbeeren? Als sie Irils Blick bemerkte, lächelte sie fröhlich auf und sprach mit vollem Mund: „Beachte mich einfach nicht. Ich war nie hier.“ Iril beachtete sie nicht weiter, und als ihr Blick das nächste Mal zu diesem Tisch schweifte, war jener verlassen.

Orfen saß immer noch am Fenster, winkte die beiden zu sich und starrte Ijsdur unverhohlen neugierig an. Da war er nicht der einzige. Ijsdur ließ sich auf dem ihm zugewiesenen Sitz nieder und wirbelte Schneeflocken umher. Die Schneespur hinter ihm schmolz zu kleinen Pfützen.

„Na, was bist du denn für einer?“, fragte Orfen.

„Ich bin ein Eis-Dämon. Ich suche die Helden von Andor. Wir müssen sie warnen.“

„Er ist ein Eis-Dämon. Er sucht die Helden von Andor. Er will sie warnen.“

„Wie passend! Manch einer zählt mich zu den Helden von Andor, auch wenn ich nie Brandurs Brosche trug. Setz dich! Wovor willst du uns denn warnen?“\bigskip







Es war Abend geworden. Iril und Ijsdur saßen weiterhin an Orfens Tisch und genossen Gildas Küche. Wie zu erwarten, hatte die Erwähnung Siantaris Orfen kaum eingeschüchtert. Er versprach, bald den anderen Helden die Warnung zu übermitteln. Sie würden Ausschau halten nach der Eis-Dämonin und ihr eine gehörige Lektion erteilen, wenn sie sich in diese Lande wagen sollte.

Zunächst jedoch gab es Speis und Trank zu genießen.

„Wir hätten ihn verspeisen sollen“, murmelte der Wolfskrieger trunken, „Den Drachen. Es hätte uns gestärkt. Und der Körper wäre nicht nutzlos die Narne heruntergespült worden.“

„Drachenfleisch kann Körper und Geist vergiften“, meinte Iril, „Und Drachenfleisch lässt sich nicht so leicht verbrennen. Das bleibt im stärksten Schmiedefeuer roh. Und rohes Fleisch ist gefährlich.“

„Nicht, wenn man es schnell genug verarbeitet.“

„Da werden wir wohl nicht auf einen grünen Zweig kommen.“

„Willst du lieber über einen anderen grünen Zweig reden?“

„Eigentlich ja. Was könnt Ihr mir über diesen Schwarzen Herold sagen?“, fragte Iril neugierig, „An der Rietburg weigert sich irgendwie jeder, genaueres zu ihm zu verraten. Viele wiesen auf dich, Orfen. Du hast die Andori schließlich erst gerade vor ihm beschützt und ihm einen Denkzettel verpasst. Weißt du mehr über ihn?“

Orfens Gesicht verdüsterte sich. „Lieber nicht. Es tut nicht gut, vom Bösen zu schwafeln.“

„Warum nennt man ihn denn den Schwarzen Herold?“, fragte Ijsdur ungeachtet Orfens Aussage. Iril übersetzte und Orfen fuhr sichtlich irritiert fort: „Nun, öhm, er ist in einen schwarzen Umhang gekleidet, und er ist ein Herold des Unheils. Ein Vorbote von Unwettern und feindlichen Angriffen. Ein Diener des Drachen, der seine Ankunft verkündigte. Der Name ‚Schwarzer Herold‘ hat sich halt so eingebürgert.“

„Und was ist er?“

Orfen seufzte tief. Er blickte kurz an die Decke, murmelte ein Stoßgebet an Mutter Natur und brummelte: „Der Schwarze Herold ist eine Sagengestalt. Vieles erzählt man sich über ihn. Angeblich soll er ein uralter Naturgeist sein, der Mutter Natur verraten hat. Oder ein allzu menschlicher Anhänger des Drachen, der in seinem Namen gegen das Königreich vorgeht. Oder auch nur ein verstorbener Kutscher, dessen Stiefel nie blankgeputzt werden können und der deswegen nie das ewige Glück finden wird. Selbst über die Geschichte seiner hohen eisernen Maske gibt es blutrünstige Geschichten.“

Iril hakte nach. „Wie realistisch sind all diese Thesen? Seit wann tritt der Herold bereits als Feind Andors auf?“

„Mindestens seit den Trollkriegen“, murmelte Orfen, „Vielleicht auch schon früher. Aber wir wissen ja noch nicht einmal, ob diese Schreckgestalt immer dieselbe Person war. Es wäre ja auch möglich, dass zur Zeit der Trollkriege ein Vater diese grausige Maske aufhatte und sie nun sein Sohn trägt. Fest steht nur, dass dieser Herold, was auch immer er ist, nicht auf unserer Seite steht. Und dass man ihn so einfach nicht vertreiben kann. Ich weiß das. Ich selbst habe ihn schließlich besiegt, ihm mein Schwert durch die Rüstung gestoßen und gedacht: So, den sind wir los! Aber im Gegenteil, stattdessen kam er gleich wieder zurück und dirigierte den Drachenangriff von dessen Rücken aus. Und nun gesellte er sich zu den Speichelleckern der Drachen.“

„Habt ihr schon versucht, den Herold auf andere Arten zu beseitigen als ein Schwert durch die Brust?“

Orfen druckste herum: „Nun, vielleicht solltet ihr euch dafür an Korbert und Wilselm wenden. Das waren die beiden anderen Wolfskrieger, die das Königsrudel der Wölfe beschützten. Allesamt andorische Hauptmänner. Korbert und Wilselm haben tapfer in beiden Befreiungen der Rietburg gekämpft und dem Schwarzen Herold auch sonst schon getrotzt. Wilselm hat gar seinen linken Arm an ihn verloren. Ich habe dem Herold hingegen nur einige wenige Male gegenübergestanden. Ich setzte mich längere Zeit nicht mehr aktiv für die Rietlande ein. Ich verliebte mich im Gebirge in eine Agren.“

„Wie meine Schwester“, verknüpfte Iril ihre Hintergrundgeschichten, „Ihr Name ist Iolith. Du hast sie nicht zufälligerweise getroffen?“

„Eine Schildzwergin unter den Agren? Nein, daran hätte ich mich erinnert.“

„Schade. Doch was machst du denn nun hier, so weit weg des Gebirges?“

„Nun, das Königsrudel benötigt schon lange nicht mehr den Schutz von Wolfskriegern, eher im Gegenteil. Und nach dem Tod von ...“

Orfens Faust schloss sich um den Griff seines Schwerts. Iril erkannte erst jetzt, dass es sich dabei um ein gerilltes Trollhorn handelte.

„... nun, mich hält jedenfalls nichts mehr dort oben.“

Iril fühlte mit ihm mit.

Ijsdurs klirrende Stimme meldete sich: „Mein Beileid.“

Iril übersetzte schwer schluckend.

„Jetzt lass mich aber nicht ein alter Trauersack sein“, brummelte Orfen, „Was geschehen ist, ist geschehen, meine große Liebe ist gerächt, und nun können wir alle aufs Neue unser Glück finden. Hier etwa, in dieser tollen Taverne. Dieser Ort ist mir der liebste in ganz Andor, und gute Gesellschaft findet man hier allemal. Lass uns von etwas anderem sprechen. Oder singen. Nur zum Schwarzen Herold vermag ich nicht mehr viel zu sagen.“

Iril überhörte das geflissentlich und überlegte: „Nun, wenn der Schwarze Herold eine verirrte Seele ist, die diese Sphäre noch nicht verlassen kann, könnte ich vielleicht aushelfen. Es gibt da eine bestimmte Runenfolge, die wir Silberzwerge nutzen, um verspukte Höhlengänge zu reinigen. Wenn ihr wüsstet, wie viele Geister in Silberhall umgingen, als wir erstmals Minen unter den Silberberg und die restlichen Mauerberge gruben ... die Insel war verflucht, müsst ihr wissen. Ist es immer noch. Wir dürfen die Mauerberge nicht überqueren, oder wir kehren nie wieder. Und die Berge zu unterqueren, birgt manchmal ähnliche Probleme. Etwas sehr Ähnliches nutzte ich, um Ijsdur von Siantaris Einfluss zu befreien. Was ich sagen will: Eine gewisse Runenfolge kann verlorene Seelen aus dieser Welt verscheuchen. Falls es uns gelingt, den Schwarzen Herold an Ort und Stelle zu halten, kann ich versuchen, ihn aus dieser physischen Welt zu vertreiben.“

Ijsdur widersprach: „Der Schwarze Herold klingt wie eine Naturgewalt. Eine solche kann man nicht vernichten.“

„Alles kann man vernichten mit genügend Macht“, widersprach Iril. Kurzzeitig flackerte ein giftgrüner Schein um den Runenhammer in ihrer Hand.

„Darauf trinke ich!“, meinte Orfen, und hob seinen Metkrug.\bigskip







Ijsdur konnte sich tatsächlich problemlos in der Taverne einquartieren. Iril tat es ihm nach. Die nächsten Tage verbrachten sie in emsigen Gesprächen über Tulgor und Andor, Eis-Dämonen und Zwergen, Siantari und Tarok, und wie man in dieser Gegend am besten aushelfen konnte. Und bald schon war mehr als ein halber Mond vergangen seit der Befreiung der Rietburg und dem Tode Taroks.

Eigentlich sollte nach dem Tod König Brandurs Prinz Thorald der neue König werden. Doch das Volk stand vor dringlicheren Aufgaben. Und so trat die Krönung in den Hintergrund. Thorald hatte die Rietgraskrone noch nicht angenommen und nutzte bislang weiterhin den Titel Prinz. Immer seltener sah man ihn. Gerüchte mehrten sich, dass er wie schon in seiner Prinzenzeit lieber eine gewisse Bäuerin am Fuße des Gebirges aufsuchte, statt sich in der weit entfernten Rietburg um die Regierung sein Königsreich zu kümmern.

Doch nun hatte Thorald zu einer großen Veranstaltung beim Sommerfels am Narnenufer aufgerufen. Fast jeder Bewohner der Rietburg, der laufen konnte, brach dorthin auf – und auch manche, die das nicht konnten.

Orfen hatte Iril und Ijsdur ebenfalls zur großen Versammlung eingeladen. Die beiden waren aus Interesse gekommen. Und auch, weil Iril ahnte, dass die Drachenkultisten der Jpaxo sich zu einem solche Zeitpunkt am ehesten wieder zeigen würden. Achtsam sondierte sie die Menschenmenge.

Schon von weitem war der Trubel an Menschen zu sehen und zu hören, der sich um den Sommerfels herum eingefunden hatte. Manche hatten den gewaltigen Findling gar erklommen. Nahe der Narne war ein behelfsmäßiges Podium aufgebaut worden, auf dem Prinz Thorald stand und unruhig umherlief. Hinter ihm flatterte eine große Flagge im Wind. Sie trug das andorische Wappen, die Sternblume auf rotem Grund.

Rund um das Podium herum sondierten wachsame Wachen die Menge.

Die Menschenmasse blickte indes interessiert hinter Prinz Thorald. Dort lag eine hölzerne Kiste, mit zahlreichen schönen Mustern versehen und einem weißen Totentuch darüber drapiert. Mannsgroß war die Kiste in der Länge. Iril hatte eine Vermutung, welcher Mann darin lag.

„Erinnert ihr euch an das Lied vom Blutstrom?“, begann Prinz Thorald seine schallende Ansprache.

Iril und Ijsdur blickten sich an und schüttelten ihre Köpfe. Die meisten anderen Anwesenden um sie herum nickten hingegen. Gilda die Wirtsfrau und Grenolin der Barde traten fröhlich zwei Schritte näher ans Podest.

„Wir haben nicht die Zeit und Muße, das Lied hier vorzutragen“, sprach Thorald weiter. Grenolin und Gilda ließen enttäuscht ihre Köpfe senken.

„Doch ist die Ballade vom Blutstrom eines der schönsten Werke, welches die Heldentaten desjenigen Menschen ehrt, der schon seit seiner Kindheit für unsere Freiheit kämpfte. Ungeheuer tapfer und mutig war er. Brandur! Ein Anführer! Ein Held! Ein König! Ein Va ...“

Thoralds Stimme stockte. Unruhig blickte er über die Menge. Iril folgte seinem Blick. Unter anderem fiel ihr die Heldin Chada auf, welche ein silbernes Amulett fest umklammerte und starr vor sich hinblickte. Thorn hatte seinen Arm um sie gelegt und kuschelte sich beruhigend an sie.

Thorald fing sich wieder und sprach weiter: „Eine lange Zeit hat mein Vater diesem Königreich gedient. Länger, als wir alle es ihm zugetraut hätten. Er flüchtete mit einer Horde Sklaven aus dem finsteren Reich der Krahder. Er hätte sich Tarok tapfer geopfert, um seiner verlotterten Schar mehr Zeit zu verschaffen, doch stattdessen verpasste er dieser Riesenechse einen Denkzettel, den sie ihr Leben lang nicht mehr vergaß! Er ließ für unser Volk eigenhändig ein sicheres Lager bauen und später eine ganze Burg. Er sorgte für ein sicheres Reich. In den Trollkriegen rettete er das ganze Land mehrfach. Er suchte den Frieden, selbst mit denen, die ihn verachteten. Und nie gab er sein eigen Fleisch und Blut auf, selbst, nachdem es ihn hintergangen hatte.“

Thorald rieb sich gedankenverloren die Rippe. Iril hatte davon gehört, wie Brandurs lange tot geglaubter Bruder Hademar vor nicht einmal einem ganzen Jahr wieder aufgetaucht war und seinen Neffen hatte von einem Schwarzen Ritter abstechen lassen. Thorald schien ihm im Gegensatz zu Brandur definitiv nicht vergeben zu haben.

„In seinen ersten Jahren hier in Andor sprang Brandur bereits in die Narne, um diese unvorsichtigen hadrischen Zauberer vor dem Ertrinken zu retten. Sie schenkten ihm zum Dank das Ewige Feuer. Dessen heilige Schale wurde damals vor einem kleinen Lager errichtet, beschützt nur durch eine schwache Holzpalisade. Heute steht das Ewige Feuer vor der Rietburg, dem prächtigsten und sichersten Bauwerk des bekannten Kontinents!“ – Iril hörte einige Schildzwerge lautstark murren – „Und das Ewige Feuer zeugt noch heute von Brandurs Tapferkeit! Unsere Lieder zeugen davon! Unsere Leben zeugen davon! Wir verdanken ihm mehr, als wir ihm je zurückzahlen könnten.“

In der Menge johlte jemand auf. Thorald nutzte die Gelegenheit, um sich eine Träne aus dem Auge zu wischen.

„Und nun ist er von uns gegangen. Nicht friedlich im Schlafe, sondern im Fieberwahn, hinterhältig verwundet durch Handlanger eines feigen Feuerdrachen! Das ist ungerecht. Doch das Leben ist nun einmal ungerecht. Es gibt nichts mehr zu rächen. Danken wir Mutter Natur dafür, dass sie Brandur so lange unter uns weilen ließ. Dass er dem Tod so oft ein Schnippchen schlagen durfte, wie er konnte. Verfluchen wir seine Mörder. Und lassen wir uns von ihm als Beispiel leiten. Nun ist die Zeit gekommen, dass Vater die Last der Krone absetzen kann. Nun ist die Zeit gekommen, wo Vater meine Mutter wiedersehen kann. Und seinen treuen Harthalt. Und Terba von den Flusslanden. Ja, selbst den viel zu früh von uns gegangenen Janner. All jene, die er in den letzten Jahren von sich gegen lassen musste. Nun ist die Zeit gekommen, wo König Brandur von Andor sich zurücklehnen und das ewige Glück genießen darf.“

Theatralisch schritt Thorald näher ans Ufer der Narne und die kleine Treppe herunter, die an dieser Seite in die Böschung gebaut worden war. Die Gischt des wirbelnden Narnenwassers benetzte ihn. Die unteren Spitzen seines purpurnen Umhangs wurden von den Wassergeistern getränkt, doch kümmerte ihn dies nicht.

Zwei Wachen trugen Brandurs Sarg näher. Die Traditionen Andors verlangten eigentlich, dass Tote am zweiten Sonnenaufgang nach ihrem Tod an ein Floß gebunden und die Narne heruntergeschickt wurden. Doch am zweiten Sonnenaufgang nach Brandurs Tod hatten die Helden noch immer gegen Horden anstürmender Kreaturen und einen wütenden Drachen gekämpft. An ein Totenritual war nicht zu denken gewesen.

Und für Könige galten vielleicht ohnehin andere Regeln. Erst recht für erste Könige eines neuen Reiches, dessen Regeln noch nirgendwo festgeschrieben standen. Statt dass Brandurs Leichnam in ein weißes Totentuch eingeschlagen wäre, war das weiße Tuch über seinen Sarg drapiert worden. Statt auf ein Holzbrett geschnallt zu sein, umgaben sechs Holzflächen die Leiche von allen Seiten. Vielleicht auch, um den Geruch nach Krankheit und Tod zurückzuhalten. Und statt zwei Tagen, hatte Thorald über zwei Wochen mit Brandurs Totenrital auf sich warten lassen. Doch die Zeit des Wartens schien vergessen, als alle anwesenden Andori andächtig ihre Häupter senkten und den Thronfolger anblickten.

Thorald beugte sich nieder und flüsterte eine leise letzte Botschaft an den gefallenen König. Dann gab er dem Holzsarg einen Stoß und starrte melancholisch hinterher, wie er von der Strömung geleitet dem offenen Meer entgegenschaukelte. Diesmal protestieren die Wassergeister der Narne – im Gegensatz zu Tarok – nicht gegen den Toten, sondern schmiegten sich an den Sarg und gaben heulende Geräusche von sich. Trauerten selbst sie um den hohen König?

Iril sondierte ihre Umgebung. Die Erwähnung einer Krone in Thoralds Rede hatte sie aufgeregt nach der Rietgraskrone umgucken lassen. War es möglich, dass sie die Krönung des zweiten Königs von Andor miterleben durfte? Doch die Rietgraskrone war nirgends zu sehen, und eine Krönung war auch nicht die Zeremonie, zu der Thorald als nächstes kommen sollte. Nein, sobald Brandurs Sarg nicht mehr zu sehen war, erklomm Thorald wieder sein behelfsmäßiges Podium und wandte sich der Menge zu. Seine wässrigen Augen straften sein breites Lächeln Lügen. Dennoch gab er sich die größte Mühe, seine tiefe Stimme ruhig zu halten, während er weitersprach.

„Und nun kommen wir zu etwas Fröhlicherem! Beinahe sechs Jahre ist es her, da ernannte mein Vater vier Helden von nah und fern zu Helden von Andor. Chada, Thorn, Eara und Kram. Ihnen wurden die letzten vier von Wulfrons Heldenbroschen anvertraut. Ich sehe die metallenen Sternblumen auch heute noch an ihren Umhängen hängen!“

Erneut brandete Jubel auf. Die vier genannten Helden wurden aus der Menge nach vorne geleitet.

„Doch nicht nur sie haben in den letzten Tagen, ja, in den letzten Jahren Tapferkeit bewiesen. Es mögen keine weiteren Heldenbroschen mehr übrig sein, doch Titel sind unbegrenzt. Es gibt neue Helden von Andor zu ernennen! Die unzertrennlichen Fenn und Hogo, kommt nach vorne! Die tapferen Jarid und Trieest aus dem fernen Dandar, ohne die die Befreiung der Rietburg niemals geglückt wäre! Die ehrenvolle ...“

Während Thorald weitere Namen aufrief, blieb Irils Blick bei dem ungleichen danwarischen Heldenpaar hängen. Eine stolze Wassermagierin in einem eleganten blauen Gewand mit goldenen Mustern. Ein an eine Kreatur erinnernder hünenhafter Krieger mit einem leuchtenden Stein in der Brust. Ihre Geschichte wäre bestimmt interessant zu erfahren.

„... und das waren alle!“, beendete Thorald seine Liste.

Ein Klatschen ging durch die Menge. Iril sah viele lächelnde Gesichter, aber auch einige unglückliche. Ein rundlicher Zwerg mit umgeschnalltem Signalhorn ließ enttäuscht seinen Kopf sinken, während seine Nachbarin ihm tröstend auf die Schulter klopfte. Nicht jeder tapfere Kämpfer hatte die Ehre, zum Helden ernannt zu werden.

Orfen war auch nirgends mehr zu sehen. Nun, so wie Iril ihn kannte, machte es ihm wohl nichts aus, übergangen zu werden.

„Und denkt natürlich nicht, dass die alten Helden leer ausgehen würden!“, rief Thorald freudig,

„Ihr alle werdet nicht nur zu Helden, sondern auch zu Fürsten von Andor ernannt! Möge der Titel euch etwas bedeuten. Ihr seid nun erst recht verantwortlich für unser Volk. Setzt euch für unsere Freiheit ein und für die gleichen Rechte aller. Helft, wo ihr könnt, und man wird euch helfen, wo man kann. Ich lasse euch allen ein Haus in der Rietburg bauen. Mard, der Baumeister, steht schon bereit, um mit euch die Details auszuhandeln. Und fünf...“

„Danke, lieber Thorald, für diese großzügige Geste“, unterbrach ihn Thorn, ehe Thorald den Helden noch mehr Dinge versprechen konnte, die sie gar nicht wollten. Iril vermutete, dass den meisten Helden dieser Titel, „Fürst“, wenig bedeutete, und dass sie den Häuserdeal später unter zwei Augen ein wenig allgemeiner modifizieren würden.

Gesonderte Heldenhäuser in der Rietburg würden wohl den Großteil der Zeit nur leer stehen, und Mutter Natur wusste, wie viele Andori dort schon jetzt in diesen Zeiten der Not kein Dach über dem Kopf fanden. Helden streiften doch lieber frei durchs Land, wohin auch immer sie der Ruf der Hilfsbedürftigen zog. Manche würde es an die Küste im Norden mit Blick auf das Hadrische Meer ziehen, andere ins Rietland im Süden, und wieder andere in den Wachsamen Westerwald. So war es halt mit den Helden.

Iril konnte sich durchaus vorstellen, ein solches Leben zu führen. Ob sie selbst wohl eines Tages diesen Titel erhalten würde? Sie stellte sich vor, wie es wäre, vor dieser gewaltigen Menge auf einem Podest zu stehen, während der Prinz sein Schwert auf ihre Schulter legte und die zeremoniellen Worte sprach ...

„Pst. Sollten wir das melden?“, klirrte es hinter Iril. Irils Gedanken waren von Ijsdur unterbrochen worden, der sich zu ihr heruntergebeugt hatte und ihr mit einem eiskalten Zeigefinger in die Schulter stupste. Verstohlen zeigte er auf die Rietburg. Iril folgte seinem ausgestreckten Arm und sah, wie das leuchtende Ewige Feuer auf dem hohen Hügel in der Ferne aufflackerte. Violette Funken mischten sich in orangenes Feuer.

Von einer vom Freien Markt geflohenen Händlerin hatte Iril letzthin deren letztes Fernrohr erstanden. Nun studierte sie durch dessen grobe Linsen nachdenklich das flackernde Ewige Feuer. Thoralds feierliche Zeremonie zu unterbrechen, um die Helden auf etwaige Gefahren aufmerksam zu machen, war schon keine schöne Manier. Doch wäre es nicht auch fahrlässig, das drohende Feuer nicht zu melden?

Durch die dicken Gläser des Fernrohrs hindurch erhaschte Iril eine Bewegung. Da! Gerangel auf der Burgmauer der Rietburg! Zwei Gestalten im Gefecht! Dann schleuderte die eine Person den anderen Menschen über die Brüstung in die Tiefe. Der Boden war viele Schritte entfernt, einen Sturz würde er niemals überleben.

Oder etwa schon? Die zweite Gestalt sprang waghalsig von der Burgmauer und landete scheinbar unversehrt auf dem harten Felsen, auf dem die Rietburg errichtet worden war. Selbst durch das Fernrohr hindurch war die Gestalt so klein, dass sie kaum mehr als ein Flackern im Gras darstellte, doch bewegte sie sich unzweifelhaft weiter, als hätte sie den gewaltigen Sprung ohne den geringsten Kratzer überstanden.

„Siehst du das auch?“, fragte Iril, „Magie?“

„Ich sehe es auch“, bestätigte Ijsdur, „Vielleicht ist es Magie. Aber ein Mensch ist das sicherlich nicht. Sieh dir die Hörner an.“

Ja, inzwischen glaubte auch Iril, an der von der Burg fliehenden Gestalt zwei lange Hörner zu erkennen. Und sie war schneeweiß.

„Ist das etwa Siantari?“, fragte Iril.

Ijsdur grinste. „Kaum. Wenn schon, hat sie sich in den letzten beiden Wochen sehr stark verändert.“

Iril blickte hektisch zurück zu Thorald, welcher sich soeben räusperte und mit feierlicher Stimme die Heldenernennung zu intonieren begann: „Andori! Wir wurden vom Bösen heimgesucht und haben schreckliche Verluste erlitten. Wir werden das Opfer aller Verstorbenen und Verletzten in Erinnerung behalten. Ihr alle habt euer Leben riskiert. Ihr alle seid Helden. Aber manche haben ganz besonders heldenhaft gehandelt. Ihr Helden, die ihr vor mir steht, habt die Rietburg eigenhändig aus der Hand der Dunkelheit befreit. Ihr habt den letzten Drachen niedergerungen und alle Lande von dieser Pein befreit. Ihr seid diejenigen, die einen Sieg über das Böse ermöglicht haben. Ihr habt immer das Wohl aller Andori im Sinn gehabt. Und von heute an und für immer mögt ihr Helden von Andor sein.“

Er zückte sein zeremonielles Schwert, machte sich auf zum linksten Helden – zu Fenn dem Fährtenleser – und sprach „Bitte knie nieder.“

Fenn tat wie gebeten und senkte seinen Kopf ehrfürchtig vor dem Prinzen.

Ehe Iril ein Weg einfiel, wie sie den Prinzen oder einen Rietgardisten möglichst störungslos auf die unschönen Geschehnisse auf der Burg hinweisen konnte, explodierte auch schon über Ijsdur ein Schauer an Eis- und Schneekristallen. Sobald Ijsdur die Aufmerksamkeit des Publikums und der Helden – und eines verdutzt dreinblickenden Thoralds – hatte, rief er schallend: „Seht! Rietburg! Dieb!“

Wie toll, dass er in der letzten Woche doch schon einige andorische Wörter aufgeschnappt hatte!

Der Dieb mit den langen Hörnern war gerade noch lange genug sichtbar, damit Ijsdur nicht als Lügner durchging. Dann war er nicht mehr zu sehen.

Thorald schickte sofort Boten zur Rietburg, doch bestätigten diese nur den schon gehegten Verdacht einiger Anwesenden.

Die Drachenkultisten hatten zugeschlagen.

Soeben waren Taroks letzte Knochen aus der Schatzkammer der Rietburg geklaut worden.\bigskip







Die Heldenzeremonie wurde rasch zu Ende gebracht, mit einem Versprechen auf eine gewaltige baldige Feier, sobald die aktuelle Situation geklärt wäre.

Thorald ließ die Brücken ins östliche Rietland sperren, da er stark vermutete, dass die Kultisten dorthin fliehen wollten. Sie kamen schließlich bekanntlich aus den nördlichen Ausläufern des Grauen Gebirges, weit im Osten.

Die Helden von Andor, sowohl neu ernannte als auch alte, schwärmten aus und versuchten, die Schuldigen zu finden. Alle hatten ihre eigenen Ansätze. Fenn der Fährtenleser saß in den Schneidersitz und ließ seinen sprechenden Raben Morar ausfliegen. Tenaya, die Wächterin des Feuers, zündete einen tiefschwarzen trockenen Ast an und blickte in die Flammen, als könnten sie ihr etwas verraten. Hinter ihr rollte Flaps der Flederfuchs über den Boden und gab quiekende Laute von sich. Chada, die Bogenschützin, ritt an der Seite von Thorn in den Süden weiter und rief etwas darüber aus, wie toll es jetzt wäre, wenn sich ihr Gezähmter nicht ins Gebirge zurückgezogen hätte. Wobei sie für ‚Gezähmter‘ nicht das übliche Wort nutzte, sondern ein sehr altes. ‚Lonas‘. Faszinierend. Doch das war ein Rätsel für eine spätere Zeit.

Auch Ijsdur und Iril blieben etwas ratlos zurück. Nun gut, eigentlich war nur Ijsdur planlos. Denn Iril hatte einen Plan.

Die beiden zogen sich vom restlichen Klamauk am Sommerfels etwas zurück, bis aus dem Stimmenwirrwarr der tratschenden Andori nur ein leises Rauschen im Hintergrund geworden war.

Iril suchte sich einen Flecken Boden, an dem das goldene Rietgras bereits niedergetrampelt war. Sie langte in ihre Reisetasche und zog eine etwa handgroße kreisrunde Steinscheibe hervor, welche in verschiedene Speichen unterteilt war. Diese Runenscheibe war erheblich dicker und massiver als die zahlreichen dünnen metallenen Runenscheiben, die sie mit sich führte. Ihre Oberfläche war durch zahlreiche Kritzeleien verkratzt, welche einst Runen gewesen waren. Doch die Zeichen waren alle verblasst, kaum erkennbar, und überlagerten sich völlig. Nur am Rand der Scheibe waren deutliche, tiefere Runen zu erkennen, welche in ein Muster aus eckigen Schneckenhäusern und kantigen Wurzeln eingeschlossen waren.

Noch glühten die Runen nicht, weder die in der Scheibe noch die auf dem Hammer noch die unter Irils Armhaut. Einzig die Übersetzungsrunen an Irils Kopf und auf Ijsdurs Brust schillerten aktiv.

Iril setzte sich in den Schneidersitz auf den Boden und holte einen kleinen spitzen Metallstab hervor. Damit begann sie, frische Zeichen in eine Speiche der Runenscheibe zu ritzen. Währenddessen murmelte sie vor sich hin. Als Ijsdur ihr nach fünf Minuten immer noch regungslos zusah, hielt sie inne und erklärte:

„Ich werde die Runen befragen, wo die Geflüchteten gelangt sein könnten. Die Runen erlauben mir manchmal einen Blick auf ferne Geschehnisse. Träger gewisser Blutstropfen zu finden ist besonders leicht. Aber auch wenn wir kein Blut des Flüchtenden haben: Sofern starke Magiequellen involviert sind, wie Drachenknochen, sollte es mir auch möglich sein, sie zu orten. Doch mag dieses Vorgehen eine Zeit lang dauern. Du musst mir nicht die ganze Zeit dabei zusehen.“

„Stehen macht mir nichts aus. ‚Runen befragen‘?“, fragte Ijsdur, „Das klingt ja beinahe so, als könntest du mit ihnen kommunizieren. Können die Runen sprechen?“

Iril gluckste auf: „Das dachte ich zu Beginn meines Studiums auch. Zu lebendig schienen mir die Wunder, die die Runenmeister der Silberzwerge mit ihrer Hilfe schufen.“

Ihr Blick verklärte sich.

„Aber ebenso wenig, wie die Magie selbst lebendig ist, sind es auch die Runen. Es sind nur einfache Zeichen in komplexen Zeichenfolgen, die wir in Objekte ritzen, um mit ihnen magische Ströme zu lenken. Und wahrlich wundervolle Dinge zu erschaffen, die bei genügend großer Komplexität ununterscheidbar von Leben sind, aber eben doch nicht leben.“

Ijsdur sah Iril einen Moment lang stumm an. Dann meinte er: „Nein, das ist nicht wahr. Magie ist doch lebendig. Das ewige Eis ... Siantari ... wann immer wir unsere Kräfte nutzen, dann spüren wir doch, wie ein fremder Geist den unseren streift.“

„Falls hinter diesen Formen der Magie ein Geist steckt, dann ist er zu weit von unserem entfernt, um von uns als solcher verstanden zu werden. Guck dir zum Beispiel diese drei Runen hier an. Die erste saugt ein wenig Magie aus der Umgebung an und leitet sie in den Osten. Die zweite da drüben teilt einen Strom der Magie in drei gleiche Teile. Die beiden linken Teile können wir dann auf diese Art wieder vereinen, um die nächste starke Quelle der Magie zu orten, wie mit einem Kompass. Und den rechten Teil nutzen wir, um die Scheibe selbst schw....“

Iril bemerkte, dass Ijsdur ihren Gedanken schon lange nicht mehr folgte.

„Verzeih mir, ich gerate wieder ins Schwurbeln. Ich meine nur: Ich ritze mit diesem Hammer bestimmte Runen in die Oberfläche, die die allgegenwärtigen magischen Ströme der Energie nutzen, bündeln, allerlei solche Dinge. Und die Runen tun genau das, was wir ihnen so auftragen, verlässlich, folgsam, manchmal für viele Jahrzehnte, ehe sie wieder verblassen. Sehr komplexe Dinge, aber keine intelligenten. Wenn ich nur einen kleinen Fehler in meinen Überlegungen mache, funktioniert nichts mehr so, wie von mir gewollt. Das kann doch kein Leben sein.“

„Kann es sehr wohl“, sprach Ijsdur ruhig, „Die magischen Ströme, die du in diese Muster zwingst, sprudeln nur so von Leben.“

„Ich vermute, dass du eine sehr andere Vorstellung von diesem Begriff hast als ich.“

Iril beendete die Arbeit an der Runenscheibe. Dann zückte sie den grünlich schimmernden Runenhammer und ließ ihn mit Gedöns auf die Steinscheibe niederfahren. Statt dass die Scheibe zersplitterte, wie Ijsdur erwartet hätte, glühten die neu in die Scheibe geritzten Runen grünlich auf.

Ganz kurz.

Dann erloschen die Runen wieder.

Iril stöhnte auf. „Es ist immer ein bisschen ein Glücksspiel, ob es beim ersten Mal funktioniert. Eine einzige falsch geritzte Rune kann das Ergebnis völlig verändern, und wenn man müde ist, häufen sich Flüchtigkeitsfehler nun mal. Wer weiß, vielleicht ist soeben irgendwo im Lande ein klitzekleiner Regenschauer niedergegangen.“

„Du kannst mit diesen Runen Regen rufen?“, fragte Ijsdur. Sein Tonfall blieb kalt, doch Iril kannte ihn inzwischen gut genug, um seine Überraschung zu deuten.

„Ein paar Tropfen vom Himmel holen trifft es eher, aber ja, das können die Runen. Und nicht nur das. Trockene Brunnen auffrischen, Nebelschleier lüften, versiegte Quellen aufs Neue sprudeln lassen ... die Runen vermögen so einiges. Wasser zu lenken ist ein bisschen meine Spezialität. Vielleicht hat es etwas mit meinem Runenhammer zu tun. Der Legende nach wurde dieser als allererstes von einer Nixe geschwungen.“

„Was ist eine Nixe? Dieses Wort erschließt sich mir trotz deiner Übersetzungsrune nicht.“

„Hm. Vielleicht gibt es für dieses Wort gar kein tulgorisches Äquivalent. Denk an einen Fisch-Menschen. Den Oberkörper eines Menschen und der Körper eines Fisches. Oder vielleicht eher einer Seeschlange?“

„Faszinierend. Wie in aller Welt sind entstehen solche Wesen?“

„Nun, wenn eine Nixe und eine andere Nixe einander ganz fest lieb haben ...“

Kopfschüttelnd korrigierte Ijsdur: „Nein, ich meine, die ganze Spezies. Solche Wesen können ja kaum von Fischen oder von Menschen allein abstammen, dafür sind sie einander zu ähnlich. Doch Fische und Menschen können keine gemeinsamen Nachkommen zeugen, wie es beispielsweise Temm und Feen können. War Magie im Spiel? Sagt, sind diese Nixen kompatibel mit Fischen oder mit Menschen?“

„Nun, es könne sich auf jeden Fall Menschen und Nixen ineinander verlieben.“

„Das sagt doch noch nichts aus. Ich könnte mich doch auch in einen Kieselstein verlieben.“

„Was?! Selbst wenn, ein Kieselstein kann dich nicht in irgendeiner Art und Weise zurücklieben, die diesen Namen verdient hätte.“

„Dann lassen wir den Kieselstein doch Kieselstein sein und fragen, ob Menschen wie du ein Kind mit einer Nixe haben könnten.“

Iril kannte die Antwort auf diese Frage nicht, doch hatte sie hier ohnehin etwas richtigzustellen.

„Warte mal. Ich bin gar kein Mensch. Ich bin ein Zwerg.“

Ijsdur hob überrascht eine durchscheinende Augenbraue, sagte aber nichts.

„Was, dachtest du etwa, ich sei ein kleinwüchsiger Mensch mit breiterem Gesicht, größerer Nase, Dunkelsicht und einem ungewöhnlich langen Leben? Ich bin 72 Jahre alt!“

„Du zählst das? Ich habe keine Ahnung, wie alt ich genau bin. Vermutlich weniger als 72 Jahre.“

„Vermutlich. Ich habe schon Runen gelehrt, da wussten die meisten Menschen hier noch nicht mal den Namen ihrer Eltern. Und ich werde vermutlich noch hier sein, nachdem alle heute lebenden Bewohner der Rietburg die Narne hinunter gingen. Außer vielleicht einige der jungen Zwerge, die sich den Andori angeschlossen haben.“

Ijsdur lachte: „Nicht nur die Zwerge leben so lange. Auch ich. Ich bin ein ganz frischer Eis-Dämon. Der größte Teil meines 500 Jahre langen Lebens liegt noch vor mir.“

„Ich nehme alles zurück, was ich über dein kurzes Leben sagte. Ändert aber nichts daran, dass du mich als Nicht-Menschen hättest erkennen können.“

„Ich kenne weder deine Lebensspanne noch deine Dunkelsicht. Und so ungewöhnlich ist dein Aussehen nicht. In Tulgor, meiner Heimat, gibt es auch sehr kleine Menschen mit breiten Nasen.“

„Vielleicht sind das auch einfach Zwerge“, bedachte Iril.

„Faszinierender Gedanke. Was unterscheidet denn einen Zwerg von einem kleinen Menschen?“

„Nun, ein Zwerg wird von Zwergen geboren, ein kleiner Mensch halt von Menschen – die eventuell auch klein sind.“

„Das scheint mir ein wenig ein Takuri-und-Ei-Problem zu sein.“

„Ein was?“

„Egal. Sagt, sind Menschen und Zwerge kompatibel im Sinne davon, wie Braunbär und Schwarzbär es sind? Können sie gemeinsame Kinder kriegen?“

„Ich glaube schon. Aber solche Liaisonen waren lange Zeit verpönt in diesen Landen. Zwerge und Menschen mochten einander eine lange Zeit lang nicht sonderlich.“

„Wirklich, ALLE Menschen und Zwerge mochten einander nicht?“

„Naja, offensichtlich mochten ein paar schon einander, sonst wäre diese Beziehungen nicht verpönt gewesen.“

„Und sind die Kinder, die aus diesen Verbindungen entspringen, Menschen oder Zwerge?“

„Je nachdem. Sie sind halt Teil des Volkes, in dem sie aufwachsen.“

„Also sind diese Begriffe mehr eine Volksbezeichnung?“

„So könnte man es sehen. Interessant, so hatte ich noch nie darüber nachgedacht. Aber ein Mensch, der unter Zwergen aufwächst, ist immer noch kein Zwerg. Er wäre immer noch viel größer als die anderen. Und würde schneller altern.“

„Bist du sicher? Altert JEDER einzelne Mensch schneller als JEDER Zwerg?“

„Nein, natürlich nicht. Der Oberste Bewahrer Melkart ist beispielsweise bekanntlichermaßen schon viel älter, als man einem Menschen seines Aussehens zutrauen würde. Und dann erst Brandur und Reka, die beiden geflohenen ...“

„Soll das heißen, dass sie vielleicht Zwergenblut in sich tragen?“

„Oder dass sie einfach diejenigen Ausnahmen sind, die die Regel bestätigen.“

Iril und Ijsdur blickten einander in Stille an. Ijsdur brach sie wieder.

„Verzeiht, ich habe dich unterbrochen. Du wolltest noch etwas von einer Nixe mit dem Runenhammer erzählen?“

Iril lachte auf: „Tatsache! Wir sind völlig abgeschweift.“

Sie setzte erneut an, die Hintergrundgeschichte ihres Runenhammers zu erzählen. Diesmal war Ijsdur keine Einwürfe ein.

„Die Insel Hadria hoch im Norden ist das Land der Zauberei und der Magie. Dort ist die Magie allgegenwärtig. Sie steigt wie unsichtbarer Dampf aus der Hadrischen Unterwelt auf. Und so werden dort überproportional viele Kinder mit magischen Talenten geboren. Darunter auch manche Nixen. Und diese eine Nixe, deren Name inzwischen verschollen ist, fand einen verborgenen Zugang zur Hadrischen Unterwelt. Sie schwamm mit einem gewöhnlichen Hammer in die Unterwelt hinein und mit einem machtvollen, mit einigen Runen bedeckten magischen Hammer wieder nach ans Tageslicht. Seither birgt dieser Hammer einen Teil Dunkler Magie. Dunkle Magie aus dem weit entfernten Hadria, dem ich nur dank des Zwergenbluts in meinen Adern nicht erliege. Wir Zwerge scheinen irgendwie gehörlos zu sein, was die verführerische Stimme der Dunklen Magie angeht. Noch ein Unterschied mehr zwischen Menschen und Zwergen.

Damals war das Wissen um die Dunkle Magie bei den Zauberern noch nicht vertreten. Oder sie wurde zumindest noch nicht als separates Konzept von der Zauberei erforscht. Sie hatte noch nicht einmal einen Namen. Das kam erst zu Orweyns Zeiten. Ja, lange vor Orweyn war Hadria noch ein blühendes Land ohne ewigen Schnee und ohne nie weichendes Eis. Und die hadrischen Nixen mussten sich noch nicht wie heute mit Blubber gegen das Erfrieren schützen.

Nach dem Tod der Erschafferin des Hammers erhoben immer wieder Zauberer Ansprüche darauf. Irgendwann gaben ihre Nachkommen nach. Der Hammer wurde an Land als besonderes Artefakt studiert. Und dann, eines Tages, fand er seinen Weg in die Minen Caverns. Zu einer bestimmen Runenmeisterin der Schildzwerge namens Golja. Manch einer will glauben, dass es ein gerechter Handel war, doch höchstwahrscheinlich ward der Runenhammer gestohlen aus den Hallen der Zauberer. Vielleicht ist das ein Glück. Denn nach der Entstehung der zwei Zaubererorden wäre er bestimmt in den Eisernen Turm gesperrt worden. So aber konnten die Runenmeister der Schildzwerge von ihm lernen, ihn hegen und pflegen, und für allerlei nützliche Gerätschaften nutzen. Und mit immer mehr Runen versehen. Unser Zwergenblut mag uns vor der anreizenden Stimme der Dunklen Magie schützen, doch es sind die Runen, mithilfe derer wir des Hammers unglaubliche Macht bündeln und steuern können. Wer die Geheimnisse der Runen enthüllt hat, muss kein magisches Talent besitzen, um Magie nutzen zu können. Es ist Tradition, dass der letzte Träger des Runenhammers ihn in seinem Testament einem seiner Schüler vermacht. Und meine Runenmeisterin, Burmrit ...“

Irils Stimme versagte. Doch sie musste nicht mehr sagen, Ijsdur hatte auch so verstanden.

In Stille arbeitete sie weiter an ihrer Runenscheibe.

Schließlich runzelte sie ihre Stirn und musterte die neu gekritzelte Runenfolge. Nachdenklich öffnete sie ihren Reisesack und zog einen breiten silbernen Gürtel hervor, entlang dessen Länge sechzehn Runen abgebildet waren. Einige waren gestickt, andere nur schwach darauf gezeichnet. Manchmal standen noch weitere kleine Kribbel nebendran. Iril überprüfte etwas darauf und glich es mit ihrer Runenscheibe ab. Dann nickte sie und korrigierte zwei Striche auf der Runenscheibe

Ein zweites Mal haute Iril mit dem Hammer auf die bearbeitete Speiche der Runenscheibe. Erneut sprang ein grünlich schimmernder Funke auf die Scheibe über. Diesmal blieben die Runen grünlich leuchtend. Ein leises Summen ertönte.

„Jawohl!“, rief Iril fröhlich auf und klatschte in ihre Hände.

„Die Runen allein reichen meistens nicht, um Magie dauerhaft in einem Objekt zu halten“, erklärte sie, „Es braucht einen Startschuss, der die ersten Ströme durch sie leitet. Mondlicht eignet sich besonders gut für längere Werke. Sogar besser als Sonnenlicht. Ich vermute, dass der Mond ein magische Ströme amplifizierender Reflektor ist. Aber um kurzweilige Werke wie dieses hier mit Strom zu erfüllen, reicht ein solcher Hammer völlig aus.“

Ijsdur nickte nur staunend.

Die leuchtende Runenscheibe erhob sich aus Irils ausgestreckter Hand einige Fingerbreiten in die Höhe und drehte sich zitternd im Kreis. Dann verharrte sie in einer bestimmten Orientierung. Iril nickte.

Sie fasste die Scheibe wieder und schlug sie erneut mit dem Runenhammer, diesmal jedoch an der Seite. Die am Scheibenrand angebrachten Runen begannen, hellblau zu glühen, während diejenigen innerhalb der Speiche erloschen und verblassten. Iril zückte einen rauen Stein aus ihrer Tasche und schmirgelte damit an der bearbeiteten Speiche herum. Der Stein wich dem Schmirgelstein, als wäre sie weich wie Käse. Als Iril den Stein wieder entfernte, war ihr vorheriges Runenwerk ausgebleicht und kein einziger glühender Funken mehr zu erkennen.

„Wiederverwendbare Runenscheibe, mehrspeichig“, grinste Iril, „Geheimtrick meiner Runenmeisterin Burmrit. Sehr praktisch, um neuartige Kombinationen auszuprobieren, ohne eine ganze Metallscheibe zu verbrauchen. Lass uns aufbrechen.“

„Was hast du gerade genau getan?“

„Ich habe die Scheibe auf die unsichtbaren Ströme der Magie sensitiv werden lassen. Insbesondere auf Knochenmagie. Mit etwas Glück zeigte sie soeben in die Richtung der entwendeten letzten Knochenfragmente Taroks. Die Drachen verfügen schließlich über eine unglaubliche Nähe zur Magie, die keinem Menschen oder Zwerg nahekommt. Und die restlichen Überreste von Taroks Körper wurden inzwischen schon weit ins nördliche Meer hinaus geschwemmt, die lenken uns nicht ab.“

„Dann los! Lass uns Drachenknochen jagen!“\bigskip







Der Hüter der Zeit rollte sich von seinem Schlafkissen und räkelte seinen schrumpeligen kleinen Körper. Die andere Temm der Reisetruppe blinzelte ihm fröhliche Aufwach-Grüße zu, doch der Hüter sah sie nicht wirklich. Während er gedankenverloren sein Frühstücksbrot verzehrte und zur Zahnpflege auf einem Kauast herumknabberte, starrten seine Augen ins Leere. Sein Geist forstete durch Jahre von Eindrücken, Bildern und Gesprächsfetzen, die er noch nicht erlebt hatte.

Der Hüter der Zeit erinnerte sich daran, wie einige weiter vorne in diesem unterirdischen Stollen, zwischen ihrem Schlafplatz und Andor, eine riesige Bruthöhle lag. Der Hüter sah das verschwommene Antlitz von Krgur vor sich. Wie der Anführer seine Würmer zur Jagd auf die elenden Eindringlinge in sein Heiligtum anreizte. Üble Sache. Der Hüter befürchtete, dass die Reisegruppe an diesen Kreaturen vorbeimusste, um Andor zeitig zu erreichen. Zumindest fiel ihm nichts anderes ein.

Heimlich richtete der Hüter sich auf und entnahm einen Edelstein aus einem mächtigen Sack, der auf dem Rücken der schlafenden Steppenechse Sabri befestigt war. Dann schlich er sich den Stollen entlang in die nächste Höhle. Stockdunkel war sie, doch der Hüter musste nicht sehen, um die richtige Stelle am Boden zu finden. Er musste nur leise genug sein, damit die schlafenden Kreideskrale schlafend blieben.

Zielgenau langte der Hüter auf den Boden. Uralte Rillen überzogen ihn. An einer bestimmten Stelle kreuzten sich gleich drei Rillen in einer kleinen Kuhle. Sorgfältig platzierte der Hüter den grünen Edelstein dort darin. Er würde sich nicht mehr daran erinnern, den Edelstein hierhin gebracht zu haben. Doch erinnerte er sich daran, in seinen Tagebüchern davon gelesen zu haben. In denjenigen wenigen seiner vergilbten Schriftrollen, die er gar auf diese Reise mitgenommen hatte, um sich die baldigen Geschehnisse kurz nachher noch ins Gedächtnis zu rufen, und sie somit kurz vorher ins Gedächtnis gerufen zu haben.

Fröhlich umhertrippelnd kehrte der Hüter zurück zu den anderen Reisenden. Die meisten schliefen noch. Hexer Haamun murmelte leise im Schlaf, einen großen Sack voller magischer Kräuter und Salben als Kopfkissen nutzend. Steppennomade Barz lag mit dem Rücken an seiner gewaltigen Echse, die mit mehreren großen Säcken und einem wunderschönen, mannshohen Spiegel beladen war. Takuri-Hüterin Aćh schnarchte neben dem goldenen Feuervogel Turr, der seinen Kopf unter sein flammendes Gefieder gesteckt hatte.

Hier, in der Dunkelheit der Stollen unter dem Kuolema-Gebirge, war der Tagesrhythmus der Reisegruppe war völlig durcheinandergeraten. Der Hüter der Zeit freute sich schon darauf, sich endlich wieder unter freiem Himmel zu befinden. Er machte sich daran, sie zu wecken.











\newpage
\section{Der Trupp aus Tulgor}




Iril und Ijsdur folgten der von den Runen angezeigten Richtung, in der Hoffnung, die von den Drachenkultisten gestohlenen Drachenknochen aufzuspüren. Allzu weit konnten sie ja nicht gebracht worden sein. Der Prinz hatte die beiden Brücken ins östliche Rietland und in den Wachsamen Wald absperren lassen. Im Norden lag das Hadrische Meer, im Westen das hohe Kuolema-Gebirge, in allen anderen Richtungen die tückische Narne, und einen Riesenvogel hatten die Kultisten seit dem Vorfall mit Taroks letzten Knochen auch nicht mehr. Die Diebe saßen in der Falle.

Die Runen lenkten Iril und Ijsdur in den Westen, in Richtung des Kuolema-Gebirges. Kurz vor dem Südlichen Wald befragte Iril die Runen erneut. Diese hatten auf einmal ein völlig anderes Ziel im Sinn. Nun zeigten sie weiter in den Süden. Hatten die Drachenkultisten sich in so kurzer Zeit so stark bewegt? Auf welche Knochenmagie außer Taroks Fußknochen könnten die Runen anspringen?

Iril und Ijsdur schlenderten am Rande des südlichen Walds entlang und am Krallenfelsen vorbei tief in den Süden Andors.

Zu ihrer Linken lag der kleine See, aus dem die Narne entsprang. Flugforellen sprangen aus dem See hervor und flatterten kurz mit ihren funkelnden Flügeln, ehe sie wieder im kühlen Nass verschwanden. Er wirkte so friedlich, ganz im Gegensatz zu den scharfen Klippen im Osten, wo die Narne dem See entsprang.

Iril füllte einen Trinkschlauch auf. Ijsdur trank gierig aus dem See. Sein Schneekörper schien an Masse zuzunehmen.

Hinter dem See waren die tiefen Schluchten des Grauen Gebirges zu erkennen, aus deren Quellen wiederum der kleine See gespeist wurde. Doch nicht dorthin zeigte Irils Runenscheibe.

Sondern zu ihrer rechten. Ins Kuolema-Gebirge hinein.

An einen ganz bestimmten Ort.

Am Hang hinauf, nur einige hundert Meter von Iril und Ijsdur entfernt, lag ein kleiner Höhleneingang, kaum groß genug für einen Zwerg.

Solche Gänge gab es zu Dutzenden in den Bergen, sowohl hier im Fahlen Gebirge als auch im Grauen. Uralte, brüchige Stollen, die tief unters Fahle Gebirge führten. Die Wege wandelten sich stetig. Manch ein Eingang schloss sich unter Beben oder tektonischen Verschiebungen, manch ein anderer öffnete sich erst durch eben solche Prozesse. Iril hätte sich unter gewöhnlichen Umständen kaum dort hinein getraut. Zu oft hatten ihre Eltern ihr eingebläut, ja nie in einen unbekannten Berggang zu stapfen. Egal ob Gruftasseln, Hornbären, Arpachen, Sporne,oder auch nur eine verrückte alte Agren sich darin versteckten. Freunde fand man dort kaum.

Ein eiskalter Hauch streifte Irils Schultern, als Ijsdurs Hand – gefolgt von Dutzenden Schneeflocken – an ihr vorbeischwang. Er legte seinen Finger vor die Lippen und deutete auf eine Stelle leicht unterhalb des Höhleneingangs.

Im Schatten eines Baumes, aber immer noch gut erkennbar im Tageslicht, lag ein einsamer Gor. Offenbar waren Gors hin und wieder doch allein unterwegs. Er schlief auf einem behelfsmäßigen Bett aus zusammengescharrten Blättern, leicht zitternd. Auch er spürte die Winde des kommenden Winters.

„Meinst du, die Runen locken uns zu diesem Gor?“, flüsterte Ijsdur.

„Ein einzelner Gor, dessen Knochen als magische Quellen mit Tarok vergleichbar wären? Kaum. Etwas geht in dieser Höhle vor“, murmelte Iril.

Passend dazu flackerte grünes Licht im Höhlengang auf.

„Sind das die Kultisten?“

„Gut möglich. Grüne Flammen deutet auf magisches Feuer hin. Drachenknochen brennen hingegen schwarz-silbern, nicht grün.“

Erneut dachte Iril an alle Schauergeschichten, mit denen sie vom Erkunden solcher Gänge abgehalten worden war. Manchmal lauerten finstere Wesen darin auf ihre Beute. Skrale, so weiß wie die Kreide, mit denen die andorischen Lehrer ihren Schülern Rechnen und Schreiben beizubringen versuchten. Große Trolle, die ihre Höhlen gerne mit den Gerippen ihrer Opfer schmückten. Höhlenwichte, die sich größte Mühe gäben, eines jeden neugierigen Eindringlings die Nase abzubeißen. Es gab Gründe, warum Eltern ihren Kindern verbaten, diese Höhlen auch nur näher zu kommen. Iril ergriff ihren Runenhammer fester. Als erstes würde sie den Gor ausschalten.

Doch, ehe sie dem Gor ein rasches Ende bereiten konnte, trat hinter einer Linde unten am Berghang eine grau gewandte Gestalt hervor.

Es war eine kleine, ältere Frau, auf deren krummen Rücken ein geschnürtes Bündel und ein kleiner Kessel befestigt waren. Sie hielt einen giftgrünen Pilz zwischen langen Fingern, musterte ihn argwöhnisch und ließ ihn in den Kessel fallen. Dann machte sie sich leise murmelnd auf den Weg, den Berghang hinunter.

Als sie Iril und Ijsdur erblickte, winkte sie und hielt einen Finger vor ihre Lippen, während sie mit der anderen Hand auf den schlafenden Gor deutete.

Ijsdur blickte verwirrt zu Iril für Handlungsangaben. Iril zuckte mit den Schultern. Sie wusste nicht so recht, was sie von dieser Frau halten sollte, aber sie hatte merkwürdigerweise das deutliche Gefühl, dass sie ihr vertrauen konnte. Sie hatte sie schon einmal gesehen, während Taroks Angriff. Die Alte hatte Bewahrern Tränke verkauft. Eine Trankbrauerin?

Iril und Ijsdur warteten einige Minuten, bis die Frau den Hang heruntergekraxelt war und sie erreichte. Bei jedem ihrer Schritte klirrte und klimperte ihr Umhang sanft, als trüge sie ein allerlei Flaschen oder Phiolen mit sich. Eine silbern glänzende Schlange schlich ihr nach und wand sich am Bein der Fremden hoch, sobald sie vor Iril und Ijsdur zu stehen gekommen war.

Mit salbungsvoller Stimme flüsterte die Fremde: „Lasst den Gor sein, weckt ihn noch nicht. Ich vermute, dass ihr aus demselben Grund hier seid wie ich?“

Iril setzte zu einer Antwort an, da unterbrach sie die Buckelige schon wieder: „Tatsächlich, seid Ihr! Wie passend. Mehr als ich wisst ihr aber leider auch nicht. Ich bin Reka.“

„Die Kräuterhexe!“, rief Iril. Rekas Künste waren weit über die Grenzen von Andor hinaus bekannt. Ebenso wie ihre eigenwillige Art.

„Genau die bin ich. Wir drei werden etwas Historisches erleben, hier und jetzt. So wurde es mir vor einigen Jahrzehnten prophezeit. Ich bin ja mal gespannt.“

Ijsdur meldete sich zu Wort: „Verzeiht, aber wie könnt Ihr wissen, dass etwas passieren wird, aber nicht, was?“

Reka grinste: „Wahrlich, wie kann ich das wohl wissen? Ich weiß es weniger, als dass ich schlicht jemandes Aussage vertraue, der es wirklich weiß. Seine einzigartige Sicht hat sich noch nicht einmal getäuscht.“

„Das ist eher eine Ausrede als eine Antwort“, merkte Ijsdur an.

„Manch einer würde sagen, dass eine Antwort umso wertvoller ist, wenn sie deine Gedanken an der Hand durch den sich so lichtenden Nebel führt, statt dich direkt an ein Ziel zu teleportieren, welches du dann im Nebel nicht erkennen kannst.“

Ehe Ijsdur zu einer Entgegnung ansetzen konnte, richtete sie ihren Blick auf seine Brust und murmelte: „Faszinierend. Dein Geist verrät nicht leichtfertig seine Geheimnisse. Darf ich fragen, wie ...“

Da erklang ein klirrendes Geräusch. Irils Blick wurde wieder zum schlafenden Gor am Berghang gezogen. Über ihm wellte sich auf einmal die Luft, als könne das Sonnenlicht nicht mehr gerade scheinen.

„Obacht! Ich glaube, es geht los!“, rief Reka freudig.

Auch der schlafende Gor musste das unangenehm laute Klirren und Knirschen über ihm wahrgenommen haben. Erschreckt jaulend sprang er auf. Die drei Zuschauer weiter hangabwärts ignorierte er. Denn in diesem Moment verformte sich der schwirrende Schemen neben ihm und nahm konkrete Gestalt an.

Es war ein gewaltiger, mannshoher Spiegel mit einem wunderschönen Rahmen, der über und über mit glitzernden Steinen versehen war. Grüne, blaue und rote, symmetrisch geschliffen und quadratische Rahmen eingelassen.

Auch nachdem der mysteriöse Spiegel sich komplett manifestiert hatte, schwebte er einige Handbreiten über dem Boden, wie von unsichtbaren Fäden gehalten.

Und wie von unsichtbaren magischen Fäden geführt, wurde der strampelnde Gor in die Höhe gerissen und vor den Spiegel bugsiert.

Von weiter unten hatten die drei Zuschauer nur einen schrägen Blick auf die Spiegeloberfläche. Es wirkte so, als welle sich das Spiegelbild des verschreckten Gors und verforme sich zu einem Menschen. Nicht nur das, auch das Spiegelbild des Berghangs um den Gor verformte und verdunkelte sich. Im Spiegelbild hinter dem gespiegelten Menschen ging irgendein Gefecht ab. Blitze erhellten einen dunklen Stollen. War der Begriff „Spiegel“ überhaupt noch passend? Inzwischen erinnerte das mysteriöse Objekt, das neben dem Gor erschienen war, eher an ein magisches Portal.

Die Person im Spiegel hob prüfend ihre rechte Hand. Daraufhin hob sich die linke Hand des Gors, wie von unsichtbaren Fäden gezogen.

Der Spiegelmensch trat nach vorne und durchbrach die Spiegeloberfläche. Der strampelnde Gor ahmte – wohl unfreiwillig – diese Bewegung exakt nach und glitt in den Spiegel hinein. Ein heller Lichtblitz zuckte auf. Gliedmaßen durchfuhren einander, als wären sie nichts als Illusionen. Dann war der Gor im Spiegel verschwunden. An seiner Stelle stand ein dunkelhäutiger, gehörnter Mann.

Der Spiegel hörte auf zu schweben, fiel zu Boden und zerbarst in hunderte kleiner Scherben.

Der aus dem Spiegel getretene Mann machte einige wackelige Schritte und stürzte ebenfalls zu Boden. Blut tropfte aus einer Wunde in seinem Bauch. Eine giftgrün leuchtende Substanz dampfte auf seinem einen Bein. Ein gehörter Helm mit eingelassenen roten Edelsteinen löste sich von seinem Kopf und kullerte den Hang hinunter. Er war gar nicht gehörnt gewesen!

„Sollen wir ihm ...“, setzte Iril an.

„Ja, sollen wir!“, meinte Reka. Sie wühlte in ihrem Reisesack und zog eine vielblättrige Pflanze heraus. „Geh du, ich bin nicht mehr so rasch auf den Beinen. Gib ihm von diesem Heilkraut. Zwei Blatt unter seine Zunge sollten ihn stabilisieren.“

Iril raste los, den Hang hinauf, zum Verletzten hin. Im Nebenbei bemerkte sie überrascht, dass Ijsdur ihr nicht folgte. Stattdessen blickte der Eis-Dämon starr den Neuankömmling an. Er schien mit sich zu hadern. Faszinierend. Aber ein Geheimnis, das sie später zu knacken hatte. Jetzt musste sie sich um den Verletzten kümmern.

Im Hochlaufen beäugte Iril den gehörnten Helm, der vom Kopf des Verletzten gerollt war. Zeremonieller Natur? Ihr Hammer summte leise auf, als sie den Helm an sich nahm. Definitiv ein magisches Artefakt. Wie der magische Spiegel besetzt mit leuchtenden Edelsteinen und die Hörner schienen aus abgebrochenen Knochen zu bestehen? Faszinierend. In Iril keimte der Verdacht auf, dass ihre Runen sich auf der Suche nach der nächsten Knochemagiequelle leider nicht auf Taroks letzte Knochen fokussiert hatten.

Die beiden Heilkrautblätter waren rasch dem Verletzten übergeben. Iril blickte zurück zu Ijsdur und Reka am Berghang hinunter und zuckte mit den Schultern. Was sollten sie nun tun?

Bevor sie eine Antwort erhalten konnte, lenkte Flackern aus demselben Höhleneingang wie zuvor ihren Blick ab.

Da! Erneut flackerte grünliches Feuer im Höhleneingang des Berghangs auf. Dann explodierte der Stolleneingang in einem Schauer aus grünen Funken und Flammen.

Erde rutschte zur Seite und enthüllte einen größeren Gang, als sie erwartet hatte.

Doch nicht nur das: Die wegrutschende Erde enthüllte auch eine Gesellschaft von vielleicht einem knappen Dutzend verschiedenster Reisender, welche mit zusammengekniffenen Augen ins Sonnenlicht blinzelten. Besonders auffällig unter der Menge war ein Mensch in langem Mantel, um dessen erhobene Hände sich magische grüne Flammen wanden.

Ein Hexer.

„Das war der richtige Weg. Du hast uns gerettet, Haamun!“, rief einer der Reisenden und klopfte dem Hexer anerkennend auf den Rücken. Der Hexer zuckte zusammen und rieb sich ächzend die Stelle.

Die Truppe rannte – oder besser: stolperte – ins Freie. Manch einer blickte ängstlich in den dunklen Stollen zurück. Iril hielt ihrem Hammer bereit. Doch wirkte es nicht so, als würden die Neuankömmlinge eine große Gefahr darstellen. Bis auf vielleicht diesen Hexer, Haamun. Dessen Begleiter mochten vereinzelt Dolche und Schwerter bei sich tragen, doch hatten sie kaum mehr die Koordination, diese geschickt zu führen. Die meisten ließen sich achtlos zu Boden fallen und keuchten nach Luft, betasteten belastete Gliedmaßen und blutige Schnitte. Was war in dieser Höhle vorgefallen? Wie Drachenkultisten wirkte diese Gesellschaft kaum, zumindest konnte Iril keine Drachenrelikte, -figuren oder -knochen erkennen.

Haamun ließ sich nicht sofort zu Boden fallen, sondern setzte zuerst einen kleinen Wichtel von seinen Schultern zu Boden. Die Gestalt erinnerte Iril äußerst stark an Wrort, den Temm von der Rietburg.

„Du steinalter Gneis!“, fluchte Haamun die Temm an, „Warum bist du stehen geblieben?!“

„Ich wollte den Echsenführer nicht zurücklassen! Will ich immer noch nicht! Du bist einfach davongerannt“, krächzte die Temm zurück. Erst jetzt bemerkte Iril die Fremdheit dieser Sprache, die sie dank ihrer in letzter Zeit dauerhaft aktivierten Übersetzungsrune verstand.

Haamun schüttelte bloß seinen Kopf. Dann erstarrte er, als sein Blick auf die grau gewandte Reka fiel, welche langsam den Berghang hochkraxelte.

Ijsdur tauchte wie aus dem Nichts neben Iril auf.

„Die auf des Hexers Schulter ist eine Temm“, flüsterte er ihr hilfreich zu, „Und sie spricht Tulgorisch“.

„Deine tulgorische Reisegruppe, nehme ich an?“, fragte Iril. Ijsdur nickte.

„Dann befindet sich wohl auch dein ...“

Ein Stöhnen hinter ihnen ließ Iril und Ijsdur herumschnellen. Der Verletzte, welcher durch seinen magischen Spiegel hier gelandet war, war erwacht. Er ließ gleich eine ganze Salve kreativer Flüche ab, während er seine Rippen betastete. Manche davon konnte Iril gar mit ihrer Übersetzungsrune nicht verstehen.

Dann fokussierte der verletzte Fremde sich auf Iril und stieß zwischen zusammengebissenen Zähnen hervor: „Da sind noch welche unten in der Höhle. Ich wurde verletzt und musste ...“ Der Fremde betrachtete Ijsdur genauer. Und erbleichte. „Ijs?! Bruderherz? Bist du das?“, rief er aus.

„Ich bin Ijsdur. Es freut mich, dich zu sehen, Eforas“, korrigierte Ijsdur tonlos.

Ungeachtet seiner Verletzungen stürzte Eforas auf den Eis-Dämon zu. Zaghaft streckte er seine Hand aus, hielt jedoch inne, ehe er Ijsdurs schneeweißes Gesicht berühren konnte.

„Was ... was ist mit dir ...“, stammelte er.

„Was mit mir geschehen ist? Mir wurde eine Eiskristallkette verliehen“, sprach Ijsdur, „Ein neues Leben. Ungeahnte Kräfte. Und ein neuer Körper.“

„Was ist mit den anderen?“

„Keiner sonst wurde zu einem Eis-Dämon, soweit ich weiß. Sie sind allesamt nur umgekommen. Ihr habt doch ihre Leichen gefunden. Es war Ijs‘ Schuld. Und ihre eigene. Ich würde sagen, dass es mir leidtut, doch sind diese Schuldgefühle größtenteils mit Ijs gestorben.“

Eforas‘ Gesicht verzerrte sich. Er schien nach Worten zu japsen.

Dies bemerkend fuhr Ijsdur fort: „Ich sehe, dass dies nicht die optimale Antwort war. Ich wünschte mir, es gäbe die perfekten Worte, um deinen Schmerz zu lindern.“

„Ijs, lass diese Maske des Eises fallen. Zeige, dass du mich erkennst. Ich bin’s doch, dein Bruder!“

„Natürlich. Das ist keine neue Information für mich. Ich trage Jahre von Erinnerungen an dich.“

„Den Ijs, den ich kannte, hätte sich gefreut, mich zu sehen.“

„Ich bin nicht den Ijs, den du kanntest. Doch liegt mir immer noch an dir und deinem Wohlbefinden.“

„Warum klingt das so unglaubwürdig? Ijs, ich erkenne dich nicht wieder.“

„Das sagte Saro auch.“

„Du warst bei Papa?!“

„Kurz nachdem ihr ins Gebirge aufbracht. Er hat mich verstoßen. Er konnte meinen Anblick nicht ertragen. Es ist mir ein Rätsel, wie ich vor euch hierher gelangen konnte.“

„Oh ein ... Saro mal wieder ... oh ... Ijs, das tut mir so leid.“

„Muss es nicht. Mir liegt nichts mehr an ihm.“

„ ... “

„Ich sehe, dass dies erneut nicht die optimale Antwort war. Bitte, empfinde meinetwegen keine negativen Gefühle.“

„Das funktioniert so nicht!“

„Das weiß ich auch. Es tut mir leid.“\bigskip







Iril hatte sich auf den Weg zum Hexer Haamun gemacht, der am ehesten nach einer führenden Persönlichkeit wirkte.

Inzwischen hatte auch die Hexe Reka die Tulgori erreicht. Sie verteilte irgendwelche hellbläuliche schimmernde Tränke, welche sie im Innern ihres Mantels umhergetragen hatte.

Und auf einmal war Haamun nirgendwo mehr zu sehen gewesen.

Iril blickte sich um und versuchte, Haamun aufzuspüren. Sie hörte ein Rascheln in einem nahe liegenden Unterholz einer kleinen Baumgruppe. Ein leiser Aufschrei ertönte, gefolgt von einem Fluchen. Dann flackerte eine grüne Stichflamme im Gehölz auf. Versuchte der Hexer etwa, sich durchs Unterholz zum südlichen Walde zu schleichen? Nicht optimal. Iril hatte genug Gerüchte von launischen Waldgeistern gehört, als dass sie dies für ein gutes Verhalten halten würde. Doch der Hexer ließ sich nicht beirren und verschwand noch tiefer im Wald. Warum floh er wohl so rasch?

Dann sah Iril, wie Reka betont in die andere Richtung sah, während einige andere Tulgori Haamun interessiert nachguckten.

Hatten die beiden etwa gemeinsame Geschichte?

Das war nun nicht wirklich relevant.

Ohne sich an jemand bestimmtes zu richten, rief Iril: „Was geht da unten noch ab? Eforas meinte, da wären noch Leute von euch im Berg?“

Natürlich verstand sie keiner. Immerhin zeigte einer der Reisenden mit dem Finger auf Eforas, der ein bisschen weiter oben am Hang stand und von Ijsdur zurückwich. Seinen Namen hatten sie verstanden.

Da trat ein weiterer buckliger Temm aus der Mitte der Gruppe hervor und stakte ungelenk auf Iril zu.

„Ich bin der Hüter der Zeit“, stellte er sich krächzend vor, „Wir sind uns noch nie begegnet, und doch kenne ich dich bereits, o Iril, Runenmeisterin aus Silberhall.“

Iril wusste besser, als in einem dringlichen Moment weiter nachzufragen, auch wenn ihre Neugierde durchaus geweckt wurde.

Da stolperte auch schon Eforas ihnen entgegen. Ijsdur hielt sorgfältig Abstand.

„Du verstehst mich, oder?“, stellte Eforas klar, ehe er fortfuhr: „Zwei von uns stecken immer noch in den Höhlengängen! Wir wurden verfolgt von seltsamen leuchtenden Riesenwürmen. Der Nomade wollte seine faule Echse nicht im Stich lassen. Die anderen sind geflohen. Die Zurückgebliebenen brauchen Hilfe.“

„Lumiwürmer“, warf die kleine Temm ein, „Das waren Lumiwürmer. Und sie wurden von mir bislang unbekannten Vertretern der Creatura geritten. Solche Humanoiden habe ich noch nie gesehen. Riesig wie ausgewachsener Mensch. Ausgemergelt, doch mit kräftiger Muskulatur. Kreidebleich. Mit mächtigen Hauern am Unterkiefer, die sie wohl kaum nur zum Brezelstapeln nutzen.“

„Ich danke euch dafür, dass ihr unsere Zurückgebliebenen verteidigen geht“, rief der Hüter der Zeit entschieden, „Nehmt den Knochenhelm mit. Und die Takuri-Spiegelscherben. Ack ist eine Takuri-Hüterin, sie wird wissen, was damit zu tun ist.“

Iril sah ihn verwirrt. Dann nahm sie den Helm und die zwei Scherben an, die Eforas ihr in die Hand drückte. Wohl Überreste des zerbrochenen magischen Spiegels, durch den sich Eforas ins Freie teleportiert hatte.

„Und von euch will niemand zurückgehen und die Zurückgebliebenen verteidigen?“, fragte die kalte Stimme Ijsdurs, der immer noch gehörig Abstand von Eforas hielt.

Die Mitglieder des tulgorischen Trupps lagen mehr da, als dass sie standen. Wer nicht verletzt oder erschöpft auf die Seite gekippt war, kümmerte sich um die Verletzten und Erschöpften. Keiner wagte es, Ijsdur zu antworten, auch wenn viele ihm rasche, unsichere Blicke zuwarfen. Iril sah gar jemanden ein Schutzzeichen in den Händen zu formen.

Von ihnen war keine Hilfe zu erwarten.

Iril fasste sich ein Herz: „Was meinst du, Ijsdur? Kommen wir mit ein paar seltsamen leuchtenden Wurmdingern klar?“

„Keine Ahnung“, sprach Ijsdur wahrheitsgemäß, „Aber sofern wir schneller sind als sie, spricht nichts dagegen, in ihre Nähe zu gehen und herauszufinden, ob wir helfen können.“

„Dann sind wir ja einer Meinung“, grinste Iril.

Sie setzte den gehörnten Knochenhelm mit den magischen Edelsteinen auf. Eigentlich sagten Helme ihr nicht wirklich zu, die beengten ihren Kopf zu sehr. Aber dieser magische hier interessierte sie durchaus. Außerdem hätte Ijsdur mit seinem breiten Geweih ihn ohnehin nicht tragen können.

Dann verschwanden die beiden im Innern des Höhlengangs.\bigskip







Reka zeigte den Tulgori einen hellbläulich schimmernden Trank. Nachdem der Hüter der Zeit diesen kurz adressiert hatte, nahm Eforas bereitwillig einen Schluck vom Gebräu. Es schmeckte stark süßlich, doch mit einer leicht bitteren Komponente darunter.

„Na, verstehst du mich nun?“, fragte Reka.

Eforas zuckte zurück. Er nahm die fremden Worte seines Gegenübers immer noch als fremd war, doch verstand er nun auch ihre Bedeutung. Es schmerzte, und Bilder und Eindrücke schossen in seinen Kopf, die er nicht näher einordnen konnte. Erst jetzt fiel Eforas auf, wie sich eine silbrig geschuppte Schlange um den Arm der Hexe vor ihm gewunden hatte und ihn anzischte. Er wankte zurück.

„Keine Angst vor Maro, die treue Seele beißt nicht“, grinste Reka.

Seine Stimme wollte ihm nicht gehorchen. Seine Zunge versuchte, seltsame Bewegungen auszuführen, die sie noch nie zuvor vollzogen hatte.

„Das ist Hexerei!“, stotterte er.

Freundlich erzählte Reka: „Kein große Hexerei. Nur eine dieser kleinen Mixturen, die alten Frauen in den Sinn kommen, wenn sie zu viel Zeit allein in der Wildnis verbringen. Wobei, wartet, nein! Diese Rezeptur stammt nicht von mir, ich habe nur die Essenz von Zucker für den Geschmack hinzugefügt. Das Rezept hat mir vor Jahrzehnten einmal ein gehörnter Geselle anvertraut. Leider weiß ich nicht, wohin er sich verzogen hat. Ich nicht einmal sagen, ob er noch lebt. Er mag vorher umgekommen sein, oder er hält sich an einem Ort auf, von dem ich keine Kunde habe. Sowohl sein älteres als auch sein jüngeres ... ah, egal. Jedenfalls erlaubt dieser Sprachtrank es seinem Trinkenden, Strukturen in gesprochenen Worten zu erkennen und sie besser sortieren zu können, ja, die Sprache von fremden Menschen zu erkennen und sich mit ihnen zu verständigen ...“

Ihre Stimme verlor sich. Vor sich sahen die Reisenden einander unsicher an.

„Brecht auf in den Nordosten, am Waldrand entlang“, schlug Reka Eforas vor, „Ihr werdet hoffentlich einen Schlafplatz finden. Morgen solltet ihr dann an einer Taverne vorbeikommen, wo ihr euch mit Speis und Trank stärken könnt. Und dann gelangt ihr zum Freien Markt. Der Freie Markt liegt in Trümmern, die Händler sind vor einem – inzwischen toten – Drachen geflohen. Dort, wo einst die Marktstände standen, könntet ihr ein Lager aufschlagen, eure Waren verkaufen und den Andori Hilfe beim Wiederaufbau anbieten. Vielleicht wird euch gar unser Prinz Thorald großzügig dafür entlohnen. Und sei es nur, weil er nicht in schlechtem Licht erscheinen will, wenn Fremde den Andori bei ihrer Arbeit helfen.“

„Diese Reise wird noch länger dauern“ rief ein Tulgori, „Wir tragen viel Ausrüstung mit uns. Nicht zu vergessen die schweren Mera-Steine. Kein Lasttier mehr, kein Fahrzeug, dafür sind voller Blasen an den Füßen und voller Erschöpfung.“ Er blinzelte Reka an, als könne die Hexe ihm die Last mit einem Wisch ihres Fingers wegzaubern.

Reka sah ihn mit einem mitleidigen Lächeln an.

„Wir finden schon eine Lösung, Seram“, wandte sich Eforas an die Tulgori. „Wenn wir die Steine nicht mehr selbst zu schleppen vermögen, dann wird dies der Fluss für uns tun. Ein Floß kriegen wir ja hoffentlich noch gebaut.“ Er wandte sich an Reka: „Besten Dank, werte Hexenmeisterin, für all Eure Hilfe und Euren Trunk. Was sind wir Euch schuldig?“

„Nichts“, winkte Reka ab, „Der erste Trank geht immer aufs Haus. Ich hätte allerdings noch allseits beliebte Mücken- und Zeckenschutztränke zu verkaufen. Diese Plagegeister gibt es im Sommer hier in Hülle und Fülle.“

Eforas versuchte, zu antworten, lallte jedoch etwas Unverständliches.

„Zeit für den nächsten Sprachtrank-Schluck“, lächelte Reka, „Wenn ihr nur einen Übersetzer hier hättet, würde es durchaus einfacher.“

„Ich bin doch ein Übersetzer“, rief der Hüter der Zeit fröhlich ein.

„Hallo, alter Freund“, begrüßte Reka den Temm, „Aber du wirst dich gerade nicht als Übersetzer um sie kümmern können. Weil ich dich jetzt klauen werde. Wir haben viel zu besprechen. Die Tipps, die du mir damals in den Trollkriegen verrietst, sind beinahe alle durch.“

„Weiß ich doch schon.“ Der Hüter der Zeit grinste und kletterte Reka bereitwillig auf den Rücken. Gemeinsam schlenderten sie davon und waren schon bald in den ersten Nebelschwaden an der Narne verschwunden.

Rekas heisere Stimme verklang: „Nun denn, erinnerst du dich noch daran, wie du mir von diesem Schattenskral berichtetest? Wie es sich herausstellt ...“\bigskip







Iril und Ijsdur stapften durch den dunklen Höhlengang.

Der Runenhammer in Irils Hand warf einen schwachen grünlichen Schein auf erdige Wände. Hin und wieder standen kleine Holzkonstruktionen in Nischen, aus denen eine gelblich schimmernde Flüssigkeit ein Quäntchen Licht spendete.

Im Lichte dieser Laternen waren in von Boden bis Decke noch immer die Spuren von Schaufeln und Spitzhacken zu erkennen. In regelmäßigen Abständen waren Runenfolgen in den Boden eingelassen.

Iril fand gut Platz, doch Ijsdur schlug immer wieder mit seinem Geweih gegen die erdige Decke. Diese Gänge waren unregelmäßig hoch und nicht auf großwüchsige Menschen angepasst.

„Haben die Tulgori diese Gänge gegraben?“, fragte Iril.

„Nein“, meinte Ijsdur, „Das waren die Temm. Lange, lange Zeit ist es her. Ihre verwinkelten Gänge durchziehen einen Großteil des Fahlen Gebirges.“

„Wie die Schildzwerge das Graue Gebirge“, überlegte Iril.

„Nein“, widersprach Ijsdur, „Ich habe in der letzten Woche ein wenig von den Schildzwergen mitbekommen. Sie sehen die Erde als ihre Heimat an und erringen ihr sorgsam ihre Schätze, während sie die Natur achten. Die Temm hingegen stammen aus Tulgor. Diese Gänge sind für sie nichts weiter als Passagen von A nach B. Und A nach C. Und Y nach Z. Es gibt viel mehr Verzweigungen hier, als eigentlich nötig wäre. Ich hatte auf der Hinreise das Glück, als Dämon des ewigen Eises meine eigenen Abkürzungen schaffen zu können. Aber Menschen aufspüren kann ich nicht. Hoffentlich finden wir die zurückgebliebenen Tulgori, ehe eine Abzwei... oh.“

Iril und Ijsdur hatten eine Abzweigung erreicht. Einen kreisrunden Hohlraum, von dem in regelmäßigen Abständen sechs Wege in andere Richtungen abführten. Durch den einen waren Iril und Ijsdur soeben getreten – Iril ritzte rasch ein Symbol in die Tür, auf dass sie sich daran erinnern konnten, woher sie kamen – doch die anderen fünf Gänge schienen allesamt gleich unscheinbar zu sein. In der Mitte des Raums stand eine Art Podest oder Altar, auf dem Iril im schummrigen Licht ihres Runenhammers Kohlereste erkannte.

„Nun, Iril, in welchen Gang wollen wir ziehen?“

„Es gibt nur eine natürliche Wahl: Die gegenüberliegende.“

„Lass dich nicht zu sicher werden. Temm-Gänge sind bekannt dafür, plötzlich in unerwartete Richtungen abzubiegen oder übereinander zu laufen. Vielleicht könntest du wieder deine Runenscheibe zücken und die nächstmögliche Magiequelle anpeilen?“

„Wenn wir wüssten, über was für eine Art von Magie die Zurückgebliebenen verfügen“, murmelte Iril, „Und selbst wenn, wenn diese Gänge so unerwartet krumm verlaufen können, hilft uns die Richtung allein auch nicht. Und es kostet immer Zeit, die Runenscheibe richtig zu gestalten. Ich wäre eher dafür, einem zufälligen Gang zu folgen, und danach ...“

Iril kam nicht dazu, fertigzusprechen. Denn in diesem Augenblick erschütterte ein Poltern die Höhlengänge, und ein roter Schein leuchtete aus dem zweiten Höhleneingang von links.

Iril packte ihren Runenhammer fester. Ijsdur flatterte mit seinen Fingern. Wie aus dem Nichts sammelte sich Wasser davor und formte sich zunächst länglich, dann zu einem Schwert, welches prompt vereiste.

„Hübscher Trick“, merkte Iril an, während sie losrannten, der Quelle des roten Feuers entgegen.

Mit seinen langen Beinen hängte Ijsdur Iril nur beinahe ab.\bigskip







Der Gang mündete schon nach wenigen hundert Schritten in eine gewaltige Höhle. Auch diese war kreisrund, mit mehreren möglichen Ausgängen in regelmäßigen Abständen auf allen Seiten. Allerdings war sie viel größer. Die Grundfläche hätte der Akademie von Werftheim Konkurrenz gemacht, und in der Höhe konnte sie bestimmt den Alten Wehrturm in seiner höchsten Phase übertrumpfen. Wie war es möglich, dass eine derart riesige Luftblase in einem sich wandelnden Gebirge stabil bleiben konnte?

Iril konnte nicht ihren Finger darauflegen, aber irgendetwas war eigenartig an diesem Raum. Gerillte Linien waren in die Wände, Decke und Boden eingelassen, welche irgendwie gleichzeitig gerade und schief wirkten. Die Linien liefen auf der anderen Seite des Raumes nicht so zusammen, wie sie sollten, sondern irgendwie ... enger? Gekrümmter? Irgendetwas war falsch hier, und wenn es sich nur um eine optische Täuschung handelte. Das sollte sie jedoch nicht aufhalten.

Wie in der letzten Höhle stand auch in dieser ein großer Altar in der Mitte. Anders als in der letzten Höhle brannte etwas darauf lichterloh und erhellte den gewaltigen Leerraum. Doch war dies keine Kohle, sondern ... ein Vogel?!

Tatsächlich, da brannte ein Vogel mit goldenem und feuerrotem Gefieder auf dem Altar. Die Flammen schienen ihm nichts auszuhaben, im Gegenteil, er krächzte fröhlich vor sich hin.

Ijsdur stockte und blickte das Tier ängstlich an. Als er weiterrannte, hielt er sich so, dass Iril sich stets zwischen ihm und dem Vogel befand, auch wenn er dadurch an Geschwindigkeit einbüßte.

Da purzelten auch schon die beiden vermissten Tulgori durch eine der vielen Abzweigungen in die riesige Höhle. Ein Mann in einem langen braunen Mantel, der einen Bogen an seinen Rücken geschnallt und ein silbern glitzerndes Seil um seine eine Hand geschlungen hatte, sowie eine Frau mit einem eleganten roten Umhang und einem golden glänzenden Schwert, deren lange Dreadlocks von flackernden Lichtscheinen überzogen wurde, als stünden sie in Flammen.

Zu guter Letzt erschien der mutmaßliche Grund für das Zurückfallen dieser letzten beiden Mitglieder von Haamuns Reisetruppe: Eine gewaltige graue Echse trampelte in die Höhle, verbunden mit dem Mann durch ebenjenes silbern glitzernde Seil an seinem Handgelenk. Der Mann zog und zerrte gelegentlich an dem Seil, doch die Echse – sein Lasttier? – wirkte alles andere als vor Tatendrang sprühend. Unbeirrbar träge setzte sie einen Fuß vor den anderen, während der Mann sie zur Eile antrieb und fluchte.

Die beiden Neuankömmlinge bestaunten ebenso wie Iril und Ijsdur die riesige Höhle und das magische Feuer in deren Mitte. Iril fiel auf, dass Ijsdur aber auch insbesondere die beiden Fliehenden beäugte. Irrte sie sich, oder zeigte sein üblicherweise regloses Gesicht tatsächlich Überraschung?

Der Feuervogel drehte eine Runde um die Höhle, stürzte von der Höhlendecke hinab und landete auf einem Arm der Frau. Erst jetzt, wo er nicht mehr in einer hohen Feuerschale lag, konnte Iril ihn richtig erkennen.

Er war so groß, dass sein Haupt ihren Kopf überragte, und wirkte fast wie ein Adler. Doch seine Augen glänzten wie gelbe Edelsteine und sein Gefieder schillerte golden und orange, so als lodere ein Feuer darin. Das Tier saß stolz und selbstbewusst auf dem Arm seiner Hüterin, als habe es sich bewusst dafür entschieden, sich genau dort niederzulassen. Iril starrte mit offenem Mund auf den Vogel und konnte ihren Blick kaum von ihm abwenden. Es war das schönste Wesen, das sie je gesehen hatte.

Inzwischen waren Iril und Ijsdur nahe genug an die beiden Neuankömmlinge getreten, um Wortfetzen ihres Gesprächs mitzukriegen.

„Bei den verfaulten Spornwalen der Tiefe, jetzt gib dir doch wenigsten ein bisschen Mühe, Sabri!“, fluchte der Mann. Interessanterweise nicht in der Tulgorischen Sprache. Aber auch nicht in der andorischen.

Seine Begleiterin drängte ebenfalls zur Eile, sie allerdings schon in der Sprache der Tulgori: „Jetzt komm schon, Barz! Ich bin mir nicht sicher, wie kräftig Turr noch ist. Wir können ihn nicht auf ewig ausnutzen, um die Würmer zurückzudrängen.“

„Ich bin nicht das Problem! Es ist Sabri, die ...“

„Meinst du nicht ...“

„Ich lasse sie nicht zurück, Ack! Nicht sie!“

Beim Namen ‚Ack‘ horchte Iril auf. Von ihr hatte der Hüter der Zeit erzählt. Die Takuri-Hüterin, die mit den Spiegelscherben etwas anfangen können sollte! Noch hatten die beiden Fremde ihre Anwesenheit nicht wahrgenommen. Wie nahe sollten sie sein, um sie anzusprechen? Wann passte es? Iril fasst sich ein Herz, und rief noch im Näherrennen laut:

„He, ihr da! Ack und Barz! Ich habe hier ... ach verflixt, die Sprachbarriere! Ijsdur, kannst du ihnen sagen, dass ...“

Weiter kam sie nicht. Ack und Barz wirbelten überrascht zu ihnen herum, erstarrten bei Ijsdurs Anblick, machten ihre Waffen kampfbereit ... und in diesem Augenblick brachen ihre leuchtenden Verfolger durch gleich mehrere Seitengänge in die Höhle hinein.

Drei riesige Würmer mit fahler faltiger Haut, länger als fünf ausgewachsene Menschen und im höher als Iril, glitten überraschend geschmeidig über den Boden, eine Spur aus grünlich leuchtendem Schleim hinter sich herziehend. Giftig glitzernder Dampf dampfte von den Schleimspuren auf.

Lumiwürmer! Das war NICHT GUT!

Die Lumiwürmer besaßen keine Augen oder ähnlich sichtbare Sinnesorgane. Doch die Spitze ihrer grausigen Körper öffneten sich synchron und entblößten mächtige Schlunde, gefüllt mit mehreren unordentlichen Reihen nadelspitzer Zähne. Es wirkte so, als hätte jemand zufällig Nadeln im Innern dieser Rachen verteilt. Und als könnten sie eine ausgewachsene Zwergin mit einem einzigen Happs verschlingen.

Kaum waren die Lumiwürmer aus den Seitengängen – in welche sie teils kaum gepasst hatten – in die riesige Höhle eingebrochen, erhoben sich von ihren Rücken humanoide Kreaturen, die sich dort hineingepresst hatten. Kreidebleich waren die Reiter der Lumiwürmer, spindeldürr und mit langen Hauern, aber auch rudimentären metallenen Klingen in den knochigen Händen. Iril hatte bereits das Vergnügen gehabt. Kreideskrale! Wilde Kreaturen, ungeschützter als handelsübliche Skrale und umso wilder. Mit Tageslicht kamen sie nicht gut klar, doch solch unterirdische Gewölbe waren ganz ihr Element.

Drei Lumiwürmer und vier Kreideskrale. Und dies mochte nur die Vorhut der Verfolger sein. Verwirrender war, dass sich auch ein einzelner Gor zu den Angreifern gesellt hatte – bis Iril sich daran erinnerte, dass dies wohl derjenige Gor sein musste, mit dem Eforas mit seinem magischen Spiegel seinen Platz vertauscht hatte. Nun brüllte auch er unverständliche Laute des Zorns und des Blutrauschs.

Noch hielten sich Iril und Ijsdur, Ack und Barz sowie diese Echse – Sabri? – und der brennende Feuervogel – Turr? – rund um den Altar in der Mitte der Höhle auf. Doch so geschwind, wie die Lumiwürmer und ihre Reiter sich fortbewegten, hatten sie kaum mehr als einige Augenblicke, um sich auf die kommende Konfrontation vorzubereiten.

„Ijsdur, frag sie, ob sie die letzten Reisenden sind, oder ob da noch mehr Hilfebedürftige zu finden sind“, rief Iril. Elende Sprachbarriere.

Doch bevor Ijsdur etwas antworten konnte, meldete sich Barz zu Wort und sprach: „Ja, wir sind die letzten! Kann ich dann davon ausgehen, dass die anderen in Sicherheit sind, wir uns ganz nah an Andor befinden und du eine tapfere Andori bist, die uns mithilfe eines Tricks von diesen Verfolgern rettet?“

„Ja. Ja. Jein.“, sagte Iril rasch, „Wir sind hier, um euch nach unseren Möglichkeiten zu unterstützen.“

„Der Eis-Dämon auch? Wie konnte er überhaupt ...“

„Was haben alle immer nur gegen Eis-Dämonen? Ja, Ijsdur ist auf unserer Seite.“

„Das können wir ja später klären.“

„Genau“, brummte Ijsdur, „Wir müssen nur wenige Minuten in diese Richtung laufen, dann sind wir alle schon wieder im Freien.“

„Den Göttern sei Dank, ein Ende in Sicht!“, rief Barz mit Blick auf die Höhlendecke, „Dann wird es an der Zeit, meine letzten Pfeile zu verschießen und die mächtigen Pulver zu verbrauchen.“

Ehe Iril aus seinen Worten schlau werden konnte, gab Barz es auf, an den Zügeln seiner Riesenechse zu zerren. Stattdessen zückte er einen Bogen, griff in seinen Köcher und lief um Sabri herum, um ein freies Schussfeld auf die heranwälzenden Riesenwürmer zu haben.

Prompt drehte sich Sabri um und setzte an, Barz wieder tiefer in die Höhle zu folgen.

„Bei allen Kreaturen der Tiefe!“, fluchte Barz, und rannte zurück auf die andere Seite der Echse. „Nie folgt das sture Vieh meinem Willen!“

Die Echse grunzte tief auf. Barz‘ Stimme wurde weich. „Ich lasse dich doch nicht im Stich, liebste Sabri, und wenn du mir noch ein Jahrzehnt lang hinterherwatschelst. Aber das ändert nichts daran, dass ich es wirklich begrüßen würde, wenn du jetzt einen Zahn zulegen könntest!“

„Wenn du hinten bleiben musst, um die Echse weiter zu locken, dann sind es wir anderen drei, die sie beschützen werden“, sprach Ijsdur ernsthaft und richtete sein Eisschwert auf die anrollenden drei Lumiwürmer. Iril schwang ihren Hammer in der Luft herum. Mit einem tiefen Brummen erwachten grünliche Schwaden reiner Magie um ihn herum. Ack hielt ein goldenes Schwert bereit. Ihre Füße nahmen eine bestimmte Position ein, mit der Selbstverständlichkeit, mit der ein geübter Tänzer seine Position einnahm.

Dann rannten die drei los, den anstürmenden Lumiwürmern entgegen.

Iril bedachte, dass Ijsdur sein Eisschwert in der linken Hand führte. War er ein Linkshänder? Auf jeden Fall war damit klar, dass er den linken Wurm übernehmen sollte. Iril rannte auf den rechten Wurm zu. Leider hatten sie sich nicht abgesprochen, und so tat Ijsdur es ihr gleich, und wie der Zufall es wollte, Ack ebenso. Alle drei Helden stürzten auf denselben Wurm zu.

Ein gutturales Grollen drang aus dem leuchtenden Schlund des Monsters. Es wurde langsamer und richtete seinen Vorderteil auf, bereit, mit gewaltigen Happsen die drei törichten Winzlinge zu verschlingen, die es wagten, sich ihm entgegenzustellen.

Die restlichen beiden Lumiwürmer schlugen einen leichten Bogen um die drei desorganisierten Kämpfer und schlidderten ungestört weiter auf die für sie so viel schmackhafter scheinende Steppenechse zu.

Überraschend behände grub Ijsdur seine Fersen in den harten Höhlenboden, drehte sich um die eigene Achse und sprang den größeren dieser beiden Lumiwürmer an, auf dessen Rücken gar zwei Kreideskrale saßen. Iril vernahm das Klirren von Klingen und das Fauchen eines Kreideskrals, doch konnte sie sich nicht groß darauf konzentrieren, denn in diesem Augenblick ließ der bedrohlich vor ihr und Ack aufragende Lumiwurm sein unförmiges Maul auf sie niederstürzen.

Iril sprang im letzten Augenblick zur Seite. Die Erde erzitterte, als nadelspitze Zähne nur Fingerbreiten von ihrem Bein entfernt in den kalten Felsen bissen und brachen. Ein Heulen entfloh dem Lumiwurm. Iril flehte den Himmel an, Ack möge ihm ebenfalls ausgewichen sein.

Die kurzen, aber soliden Eisenstäbchen, die Iril in ihrer Reisetasche mit sich führte, waren eigentlich fürs Einritzen magischer Runenfolgen in Stein und Metall gedacht. Doch in solchen Situationen stellten sie sich auch als anderweitig nützlich heraus. Iril packte gleich mehrere der Stäbchen in ihre Faust und trieb sie dem Lumiwurm in die Seite. Eitrige, leuchtende Flüssigkeit triefte hervor und tropfte heiß Irils Arm herunter. Die ätzende Flüssigkeit brannte, doch Iril ließ nicht los. Als der Lumiwurm erneut seine Vorderhälfte in die Höhe hievte, wurde Iril mit ihm hochgeschleudert. Ächzend kletterte sie sich und kniete, nein, thronte majestätisch auf dem Kopf des Lumiwurms.

Von weiter hinten am Wurm blickte ihr der verärgerte Reiter des Lumiwurms entgegen.

Erneut stieß der Lumiwurm blind seinen Kopf auf den Erdboden. Offenbar war Ack ihm das erste Mal entgangen. Iril wurde beim Aufprall heftig durchschüttelt, schaffte es jedoch, auf dem gewaltigen Wurmkopf zu bleiben. Wankend richtete sie sich auf. Das war ja schlimmer, als bei Sturm auf hoher See zu sein!

Iril fixierte den einzelnen Kreideskral, der auf dem Rücken des Lumiwurms saß. Er trug keinen Sattel oder ähnliche Reithilfen, führte jedoch zwei Seile, welche links und rechts am Kopf des Lumiwurms befestigt worden waren. Mit diesen steuerte er das Biest wohl. Aktuell hielt er beide Zügel in derselben Hand und löste mit der anderen eine schartige Klinge von seinem Gürtel.

„Ergib dich! Lass mit deinen Würmern von der Jagd ab und du wirst uns nie wieder sehen müssen!“

Der Kreideskral verstand sie und lachte nur. „Ich werde dich auch nie wieder sehen müssen, wenn du Futter wirst. Die heiligen Würmer dürsten nach Blut.“

Irils Runenhammer dürstete danach, die in seinem Innern gespeicherte Magie loszulassen. Iril stolperte ein, zwei, drei wackelige Schritte vom Kopf des Lumiwurms seinen Hals herunter – falls der Begriff Kopf für ein so unförmiges Wesen überhaupt Sinn ergab – und holte mit dem Hammer aus.

„Angriff, Turr! Erledige den Reiter!“, erklang von irgendwoher Acks helle Stimme. Der glühende Feuervogel erschien wie aus dem Nichts, flatterte vor dem Kreideskral herum und hackte mit brennenden Klauen nach dem Kreideskral. Iril bremste den Schwung ihres Hammers aus und verfehlte den Vogel nur knapp. Glühende Funken rieselten auf sie herunter.

„Den hätte ich gehabt!“, rief Iril Ack empört entgegen, bevor ihr wieder einfiel, dass diese sie kaum verstehen konnte. Der Kreideskral hackte mit seinem verrosteten Schwert nach Turr und traf ihn am Flügel. Turr klatschte kreischend auf den Rücken des Lumiwurms und rollte von dort ungelenk auf den Boden. Ehe der Kreideskral sich neu orientieren konnte, schlug Iril die Kreatur mit dem Runenhammer grob vom Rücken des Lumiwurms. Ein Krachen und Knacken verriet ihr, dass er mindestens mit einigen gebrochenen Knochen zu kämpfen hatte. Damit war er wohl außer Gefecht.

Der Knochenhelm auf ihrem Kopf erwachte zum Leben. Mit einem warmen Summen brummte er auf ihrem Schädel. Eine angenehme Ruhe überkam Iril. Sie fühlte sich erfrischt und war bereit, den nächsten Gegnern entgegenzustehen! Welch faszinierender Kopfschmuck. Sie nahm sich vor, die Tulgori bei der nächsten Gelegenheit nach dem Geheimnis seiner Herstellung zu fragen.

Wackelig richtete sie sich auf dem Rücken des Lumiwurms auf und sondierte die Lage. Ack hatte es offenbar bislang tapfer geschafft, dem stacheligem Wurmmund zu entkommen, denn nun befand sie sich in einem Gefecht mit dem einzelnen Gor.

Ack schwang ihre goldene Klinge kunstvoll herum, stocherte aus der Entfernung nach den ungeschützten Hautstellen der Kreatur und fügte elegant kleine, doch sicherlich schmerzvolle Schnitte zu. Der Gor fauchte frustriert auf und schnappte immer wieder nach ihr, bekam die flinke Tulgori allerdings nicht zu fassen.

Nicht weit davon entfernt hatte Ijsdur es wie Iril auch auf den Rücken eines Lumiwurms geschafft. Gleich mit zwei Kreideskralen legte er sich an, welche er mit seinem kurzen Schwert von sich weghielt. Seine Bewegungen wirkten ungeübt und etwas ungelenk – der schwankende Wurm, auf dem er sich befand, half wohl kaum in der Stabilität – doch schaffte er es soeben, seine Klinge im ungeschützten Nacken der einen Kreatur zu versenken. Noch während der Kreideskral umkippte, heulte der andere auf und verbiss sich mit seinen mächtigen Hauern in Ijsdurs Schulter. Die beiden stürzten vom Lumiwurm und verschwanden aus Irils Blickfeld.

Doch was war mit dem dritten Lumiwurm? Was mit Barz und seiner sturen Echse?

Iril erkannte den grauen Körper Sabris am anderen Ende der Höhle, schon beinahe bei einem der Ausgänge. Das war gut. Doch der dritte Lumiwurm wurde von seinem Reiter stetig näher auf die Echse zugetrieben. Und Barz hatte immer noch nicht seinen Bogen gezückt. Das war weniger gut.

Was tat der Dussel denn nun?! Statt eine Waffe zu ziehen, löste Barz ein türkises Pulver von seinem Gürtel und streute sich eine Prise davon auf die ausgestreckte Zunge. Er schmatzte ein, zweimal und schrie kurz auf, als seine Augen blendend hell aufleuchteten. Dann trat er gewandt einen einzelnen Schritt zur Seite. Der Lumiwurm, der auf ihn zugehalten hatte, verfehlte ihn und Sabri um Haaresbreite, konnte nicht mehr ausweichen und knallte in die Höhlenwand hinein. Der Kreideskral auf seinem Rücken holte mit einem gezackten Stück Metall nach Barz aus, doch dieser drehte sich nicht einmal zu seinem Gegner um, sondern wich mit dem Rücken zu ihm mühelos aus, als hätte er den Schlag schon längst kommen sehen. Als Barz sich endlich umdrehte, sprang er gleich hoch und stach er seinerseits mit einem Pfeil aus seinem Köcher in die ungeschützte Brust des Kreideskrals. Die Kreatur sackte zusammen und hing locker vom Rücken des Lumiwurms, während dieser sich heulend und ziellos die Wand entlangwälzte und selbige mit leuchtender giftgrüner Spucke verätzte.

Barz blickte dem zweiten Lumiwurm entgegen, welcher neuerdings führungslos war und sich nun ebenfalls auf Sabri zuschlich. Leuchtendes, dampfendes Blut tropfte aus tiefen Wunden in seiner Seite, und er bewegte sich langsamer, vorsichtiger, doch nicht weniger unaufhaltsam.

Barz löste ein grünes Säcklein vom Rücken der Steppenechse und setzte sich daran, den anstürmenden Lumiwurm mit seinem Inhalt zu beschenken.

Da tauchte Ijsdur wie auf dem Nichts auf, seine Schulter von schwarzem Kreaturenblut getränkt.

„Das ist doch dein Bannpulver, oder?“, fragte er Barz, „Setze es noch nicht jetzt sein. Es kommt noch eine weitere Verzweigung. Banne einen Wurm lieber dort, dann versperrt er den Verfolgern den Weg nach draußen.“

Barz blickte ihn überrascht an, nickte dann jedoch.

Ein lautes Rumms riss Iril aus den Beobachtungen. Der Lumiwurm, den sie immer noch mehr oder minder erfolgreich ritt, hatte erneut nach Ack gebissen und seinen geifernden Kopf in den Boden gerammt. Ack stürzte zu Boden und der gemeine Gor nutzte die Gelegenheit, um mit seinen Hornklauen ihr Schwert wegzuschleudern. Waffenlos lag Ack da. Sie wäre noch nicht einmal schnell genug, wieder aufzustehen, ehe der Gor ihr nicht ein unschönes Ende bereitete.

Iril ließ sich in dieselbe Stelle plumpsen, wo zuvor der Kreideskral auf dem Lumiwurm gesessen hatte. Verzweifelt zog und zerrte sie an den Zügeln des gewaltigen Wurms. Dessen Kopf wirbelte umher, fegte knapp über Ack hinweg, schleuderte den Gor beiseite und verschluckte die Kreatur mit Haut und Schuppen. Iril zerrte erneut an den Zügeln, doch der Lumiwurm brummte nur wohlig und drehte sich mit einem gefüllten Magen zur Seite.

Gerade noch rechtzeitig sprang Iril von seinem Rücken, ehe sie zu Mus gequetscht worden wäre.

Iril stürzte vom Lumiwurm herunter und schlug mit dem Kopf hart auf dem Boden auf. Das wohlige Brummen des Knochenhelms verschwand, als der Helm sich von ihrem Kopf löste und lautstark über den Höhlenboden kullerte.

Schmerz durchzuckte sie und ihre Sicht verschwamm. Wie aus weiter Ferne nahm sie die Gestalt Acks wahr, die am verdauenden Lumiwurm vorbeihumpelte, nur ihre Silhouette vor dem leuchtend pulsierenden Leib des Wurms sichtbar

Die Tulgori hatte sich wieder aufgerichtet. Nun entdeckte sie den verkrüppelten Turr am Boden und eilte zu ihm.

„Bald wird es wieder gut sein. Bald. Halte nur noch kurz durch“, sprach Ack Turr gut zu. Tränen rannen ihr Gesicht herunter, tropften auf den Vogel und verdampften.

Sie goss eine ölige Flüssigkeit aus einem Lederschlauch über ihre Hände, verrieb sie rasch und hob dann sanft den wimmernden Feuervogel in die Höhe, der vergeblich mit seinem gebrochenen Flügel strampelte. Leuchtende Flammen flackerten über die feurigen Federn und leckten an Acks Armen, doch die Hitze schien ihr nichts auszumachen.

„O Turr, du Armer. Ich spüre, dass du am Ende deiner Kräfte bist. Du wirst sterben, bald, sodass du den Zyklus von vorne beginnen kannst. Doch noch ist es nicht so weit, wir brauchen dich noch beim Kampf gegen die Lumiwürmer, die immer näher rücken. Mit all meiner Willenskraft unterstütze ich dich, auf dass du noch ein bisschen länger durchhalten kannst.“

Ein feuriger Schein umspielte Acks Körper. Die Tulgori sank auf ein Knie und begann zu zittern, während das Leuchten über Turrs Federn abklang.

„Sei beruhigt. Nicht mehr lange, Turr. Nicht mehr lange.“

In diesem Augenblick richtete sich der verbleibende Kreideskral auf. Er hatte den verlorenen Knochenhelm aufgesetzt und Acks goldenes Schwert vom Boden aufgenommen. Die in den Knochenhelm eingelassenen Edelsteine glänzten rötlich und die Spitze des goldenen Schwerts war direkt auf Ack gerichtet.

Die waffenlose Ack streckte den verletzten Turr dem Kreideskral entgegen, als wolle sie ihn ihm übergeben.

„Turr! Feuerball!“, rief sie, und schloss schützend ihre Augen.

Der Feuertakuri gab einen weinerlichen Laut vor sich, doch gehorchte er. Ein gebrochener Flügel klatschte auf einen ganzen, und ein gewaltiger Feuerball hüllte das angreifende Echsenwesen ein. Dann ließ der Feuervogel schwach seinen Kopf hängen und sank in Acks Arme.

Der Feuerball schleuderte den Kreideskral mehrere Meter nach hinten. Der Knochenhelm löste sich von seinem kahlen Schädel. Aschespuren übersäten des Kreideskrals nackte Haut. Doch noch war er nicht geschlagen. Zitternd richtete der Skral sich ein letztes Mal auf und hielt Ack erneut das goldene Schwert entgegen.

Ack streckte den Feuervogel in die Höhe und rief erneut: „Feuerball, Turr!“ Doch diesmal folgte Turr nicht, sondern strampelte nur schwach quietschend in ihren Armen.

Der Kreideskral kicherte humorlos auf. Wut und Schmerz verwandelten sein von schwarzem Blut übersätes Gesicht in eine wütende Fratze.

„So sterbt doch endlich, elende Eindringlinge!“, brüllte er und sprang die schutzlose Ack an.\bigskip







Krgur, Häuptling der Kreideskrale, richtete sich keuchend vom kalten Höhlenbogen auf und bleckte seine Hauer. Die Eindringlinge in sein Heiligtum rannten um ihr Leben. Sein neuerdings unberittener Lumiwurm war der Steppenechse dicht auf den Fersen.

Krgur schmatzte am Stück Schnee, das er diesem Schneegeist aus der Schulter gebissen hatte. Er spuckte aus. Kein Geschmack, kein Nährwert. Und der Rest des Schneegeists war auf einmal nirgendwo mehr zu sehen.

Plötzlich spürte er ein Rieseln auf seiner Haut, als ob es regnen würde. Doch nein: Der seltsame Führer der Steppenechse, mit einem Bogen in der einen und einem seltsamen Beutel in der anderen Hand, bestreute ihn mit bläulichem Staub? Ein Schauer überfiel ihn und sein Körper zuckte unkontrolliert. Wie entwürdigend! Diesen Menschen würde er als erstes fressen! Während er auf ihn zu rannte, traf ihn plötzlich etwas Kaltes im Rücken und warf ihn wieder zu Boden. Hinter ihm stand die eisige Kreatur, weiß wie der hadrische Schnee, die Wunde auf der Schulter schon wieder zugewachsen. Aus eiskalten Augen starrte sie Krgur an. Mit einem Schwert aus Eis drang sie auf ihn zu! Schnell sprang Krgur hinter den uralten Altar der Höhle, um etwas Deckung zu haben. Den Uralten sei Dank: Die beiden Gegner wandten sich der dringenderen Bedrohung durch die heiligen Lumiwürmer zu und verfolgten ihn nicht.

Da, am Boden! Ein gehörnter Helm, und ein goldenes Schwert mit einem Griff in der Form der Mondsichel! Es war keine Schande, die Waffen seiner Gegner gegen sie zu verwenden. Krgur griff nach dem Helm und setzte ihn auf. Sanftes Brummen erfüllte seinen Kopf und gab ihm wieder Klarheit.

Krgur hob das goldene Schwert und richtete es auf die nur wenige Schritte davon entfernt sitzende Frau. Doch da schleuderte der brennende Vogel in ihren Händen einen Feuerball! Hart schlug Krgur auf dem Boden auf. Die Klarheit des Knochenhelms verschwand so rasch, wie sie über ihn gekommen war. Zitternd richtete Krgur sich wieder auf. Verbrannt war er, und entsetzt, doch an Flucht war nicht zu denken. Diese Fremden mussten gestraft werden dafür, in sein Reich eingedrungen zu sein, die Stille der Berge gestört zu haben, seine Sippenmitglieder ermordet zu haben. Sie würden gutes Futter abgeben.

Da stand die Frau, geschmückt mit Federn des Feuervogels, Furcht in den Augen! Allein war sie nicht stark. Wenigstens sie würde er kleinkriegen.

„So sterbt doch endlich, elende Eindringlinge!“, brüllte Krgur, und sprang die schutzlose Frau an.

Doch seine Füße landeten nicht sicher auf dem festen Boden.

Er rutschte aus. Die Welt kippte zur Seite. Die Luft wurde aus Krgurs Lungen gepresst, als er erneut hart auf dem Höhlenboden aufschlug. Wie viel Pech konnte man auch haben. Seine Hand öffnete sich und das goldene Schwert flog daraus hervor.

Als letztes sah Krgur die Klinge der Frau auf sich niederfallen.

Schmerz durchzuckte ihn.

Dann sah er nichts mehr. Und nahm auch nichts mehr war.\bigskip







Der Kreideskral sprang die schutzlose Ack an, rutschte auf einem kleinen Steinchen aus, stürzte zu Boden und versenkte Achs Schwert im eigenen Schädel. Ein Leben wurde ausgehaucht, doch die Gefahr war gebannt.

Ack langte nach vorne und hob das Steinchen vom Boden auf. Grünlich schimmerte es.

„Kein Kieselstein. Ein grüner Edelstein“, murmelte sie zu sich selbst. Ack tastete den Boden ab. „Eine exakte Kreuzung dreier Rillen. Eine markante Kuhle. Woher ... ?“

Es war nicht wichtig.

Die Kreideskrale und der Gor waren erledigt, doch die Lumiwürmer waren noch immer eine große Gefahr. Sabri war noch nicht aus der Höhle draußen.

Es war Zeit zu handeln, rasch, jetzt!

Iril stieß sich an der Höhlenwand in die Höhe. Die ganze Welt schwamm wie auf einem Schiff, aber sonst schien alles in Ordnung. Mehr oder weniger.

Und Iril erinnerte sich endlich an die Takuri-Spiegelscherben, die die Tulgori ihr für Ack überlassen hatten.

Sie zog die Scherben aus ihrer Tasche – schnitt sich an einem Finger und fluchte auf – stolperte nach vorne und hielt sie Ack entgegen.

„Hier, nimm diese Spiegelscherben! Jemand meinte, du könntest etwas damit anfangen.“

Mit geweiteten Augen antwortete Ack: „Woher hast du die? Takuri-Spiegel sind von großem Wert. Selbst ihre Scherben bergen noch besondere Kräfte.“

„Darum gebe ich sie dir ja! Du sollst wissen, wie man sie nutzt.“

„Klar, du kannst natürlich mich ebenso wenig verstehen, wie ich dich“, murmelte Ack niedergeschlagen, „Dennoch danke.“

Dem war dank Irils Übersetzungsrunen zwar nicht so, doch nun war auch nicht der Moment, um diesen Fehlschluss aufzuklären.

Ack richtete sich auf und nestelte an ihrer Halskette herum. Iril konnte nicht genau sehen, was sie damit tat. Auf jeden Fall streute sie irgendetwas aus die Spiegelscherben, woraufhin diese zu glühen begannen.

Ack schloss ihre Faust. Die Scherben zerbröselten und leuchteten noch heller auf, bis die dunklen Schatten von Acks Knochen unter ihrer Haut sichtbar wurden, als ihr Fleisch vom Glühen erfüllt wurde.

Ack rannte nach vorne, wo zwei mächtige Lumiwürmer kurz davor waren, die massige Transportechse zu verschlingen.

„Zurück!“, rief Ack und pustete einen Teil des Scherbenpulvers auf die Würmer. Sanft glitzernd breitete sich der Staub über die Lumiwürmer auf und nestete sich auf ihrer leuchtenden Haut ein. Die Würmer wurden langsam durchscheinend. Iril schüttelte sich, als die Innereien dieser Wesenheiten durch ihre Haut hindurch sichtbar wurden.

Ack schüttelte indes ihre Faust mit dem restlichen Spiegelpulver darin und pustete dieses in die gegenüberliegende Seite der Höhle. Wie von unsichtbaren Fäden dirigiert wirbelten die Ströme des glühenden Pulvers durch die Luft und kamen am anderen Ende des Raums zu stehen.

Ein Lichtblitz erhellte die Höhle und ein lauter Knall war zu hören. Und auf einmal befanden sich die Lumiwürmer nicht mehr direkt vor dem Ausgang und der Steppenechse, sondern wieder weit von den Helden entfernt, am anderen Ende des Altarraums.

Der Weg nach draußen war frei!

Iril und Ack – letztere noch immer mit dem verletzten Turr an sich gepresst – eilten nach vorne, wo Ijsdur und Barz etwas ratlos vor zwei sehr nahe beieinander liegenden Stolleneingängen standen. Sabri röhrte hinter ihnen.

„Durch welchen kamt ihr in diese Höhle hinein?“, fragte Barz Ijsdur. Dieser antwortete nicht. Und auch Iril wusste die Antwort nicht. Sie hatte vergessen, beim Betreten dieses Raums ein Zeichen an der Wand zu hinterlassen.

„Ich weiß es nicht mehr“, gab Iril zu.

Ijsdur überlegte kurz und sprach dann: „Es hat keinen Zweck, länger zu überlegen. Lasst uns einfach einen wählen.“

„Welchen?“, fragte Barz.

„Entschiede doch du.“

Das ist eine hohe Verantwortung.“

„Soll der Zufall lieber entscheiden lassen?“

„Können wir einfach möglichst schnell eine Entscheidung treffen?!“

„Fühle dich frei.“

„Links!“, rief Iril.\bigskip







Nach einigen Minuten der Flucht bog der linke Stollen urplötzlich um eine Ecke und über eine Leiter steil nach unten.

„Da kriege ich Sabri nicht runter“, fluchte Barz.

„Ist egal, es ist ohnehin nicht der richtige Gang“, sprach Ijsdur ernüchternd.

„Können wir noch umdrehen?“

Das Knarren und Knirschen, sowie ein immer heller werdender leuchtender Lumiwurm-Schein aus dem Stollen hinter ihnen, verriet ihnen die Antwort.

Ack sprach dem Feuervogel in ihren Armen gut zu, aber Iril glaubte nicht, dass dieser sie noch retten konnte.

„Ich glaube, das ist das Ende für Sabri“, murmelte Iril betrübt.

„Nein!“, rief Barz, „Nicht nach alledem! Es muss noch irgendeine Möglichkeit geben.“

„Gibt es tatsächlich“, sprach Ijsdur leise. „Ich war nicht nur dank meines Bergsinns so geschwind bei der Unterquerung des Kuolema-Gebirges. Barz, halte bitte Ausschau, wie nah die Lumiwürmer schon sind.“

Barz lugte um die Ecke und meinte ängstlich: „Schwer einzuschätzen, aber wir haben keine volle Minute mehr.“

„Wir brauchen keine volle Minute.“

Ijsdur legte seine Hand auf die rechte Höhlenwand und versteifte sich. Die Schneeflocken und Eiskristalle, die ihn stets umgaben, wirbelten noch schneller umher und wurden ... mehr? Iril zitterte, als die Temperatur merklich sank. Von der Stelle, wo Ijsdur die Stollenwand berührt hatte, breitete sich eine bläulich-weiße Schicht aus Schnee und Eis aus. Innert Augenblicken wuchs ein mannsgroßes, ja, gar steppenechsengroßes Tor aus purem Eis am Rand des Stollens. Dann schleuderte Ijsdur einen Eisblitz auf das Tor. Es knirschte und bröckelte beiseite, bis nur noch der Rand der Eisfläche bestand und in seinem Innern einen langen, breiten Gang aus purem Eis enthüllte. Ein Weg durch den Berg.

„Wie ... was?“, stammelte Iril.

„Jetzt ist doch wahrlich nicht die Zeit dafür“, merkte Ijsdur an, und betrat den frisch generierten magischen Eis-Tunnel. „Folgt mir! Mit etwas Glück stoßen wir am anderen Ende des Tunnels auf den richtigen Höhlengang, von dem aus wir hierherkamen.“

Iril wischte sich ein wenig lockeren Schnee von den Schultern. Sie hatte nicht einmal gemerkt, dass Schnee gefallen war.

Dann betrat auch sie den magischen Gang. Die Kälte ließ sie rasch erzittern.

Beim Zurückblicken sah sie, wie der Gang sich vor den leuchtenden Lumiwürmern wieder verschloss. Doch dunkel war es nicht. Irils Runenhammer und Acks Feuervogel ließen einen flackernden Schein über den kalten Stollen schimmern.

Sie waren in Sicherheit.






















\newpage
\section{Die Spur der Drachenknochen}


Schnaufend brach die Kämpfergruppe aus dem Höhlenstollen hervor. Barz hörte auf, an seiner Echse Sabri zu ziehen. Ack blickte erleichtert in den Sternenhimmel und atmete tief durch.

Es war Nacht geworden. Die Tulgori und Reka waren nirgendwo mehr zu sehen. Andor lag still da.

Die vier ließen sich unzeremoniell zu Boden fallen und begutachteten ihre Verletzungen. Barz rutschte zu Ack, zog irgendein glänzendes Mittel aus einer seiner vielen Manteltaschen hervor und machte sich daran, ihr angeschlagenes Gesicht zu beträufeln.

Ijsdurs betrachtete einen Schnitt in seiner linken Hand, der sich unter seinem Blick wie von selbst wieder schloss.

„Danke für den Tipp“, sprach er zu Iril, „Diese Kreaturen sind wirklich im Nacken verwundbar.“

Iril vermutete, dass die Kreideskrale so ziemlich überall an ihrem Körper verwundbar waren, beließ es aber dabei.

„Bäh!“, spuckte sie aus, während sie giftgrün leuchtende Spucke von ihren Armen rieb. Sie sog zischend Luft ein, als die Berührung viele kleine Wunden an ihrem Unterarm aufbrennen ließ. Sie fluchte. „Was ich in meiner Zeit in Silberland definitiv nicht vermisste, waren die riesigen Insekten und Spinnentiere der südlichen Gebirge. Im hohen Norden sind die wenigsten Viecher größer als eine Fingerbreit. Eine zwergische Fingerbreit. Aber nein, hier wimmelt es nur so von gewaltigen Arpachen und Spornen und Lumiwürmern und dergleichen.“

„Heißt das etwa, dass jemand alle nördlichen Insekten geschrumpft hat?“, fragte Ijsdur interessiert. Während alle anderen sich weiter verarzteten, saß er still am Boden und beobachtete aufmerksam das Verhalten der anderen.

„Nee, hier gibt es ja auch kleine Insekten“, meinte Iril, „Bienen und so. Eher hat jemand im Grauen Gebirge einige bestimmte Insekten vergrößert. Falls es überhaupt ein jemand war.“

Sie blickte zurück in den dunklen Stollen. Im schwachen Schein einer Laterne war in der Wand noch der Ausgang von Ijsdurs Eistunnel erkennbar.

„Wie hast du das gemacht? Mir wird schwindlig beim Gedanken an die nötige Kraft, so viel Felsmasse rasch zur Seite zu quetschen, um einen derart breiten Tunnel zu schaffen.“

„Es ist eher eine magische Passage“, meinte Ijsdur, „In einigen Tagen, wenn Ein- und Ausgang geschmolzen sind, wird einfach wieder blanker Fels dort stehen, wo jetzt der Tunnel liegt.“

„Bewegten wird uns überhaupt durch den Berg hindurch? Was wäre geschehen, wenn wir die Eiswand aufgebrochen hätten?“

„Ich nehme an, wir wären auf Felsen gestoßen. Vielleicht aber auch auf die Feenwelt. Oder das blanke Nichts. Wer kann das schon wissen?“

„Du, hätte ich vermutet?“

„Ich bin auch neu im Leben als Eis-Dämon. Der Durchgang wird auf jeden Fall nicht ewig halten und hat nicht so viel Felsen verschoben, wie du meintest.“

Iril nickte. Sie und Ijsdur schauten nun beide interessiert hinüber zu den beiden Neulingen.

Ack hatte sich hingekniet und streichelte Turr. Der Feuervogel protestierte fiepend, als Ack seinen verletzten Flügel untersuchte. Ack hob ihr goldenes Schwert, sprach den Kleinen gut zu ... und rammte es ihm ohne Umschweife in die Brust.

Iril machte große Augen: „Bist du verrückt! Diese Verletzung hätte noch heilen können!“

Barz hob die Hand und murmelte in der Sprache der Bewahrer: „Nicht verrückt. Er wird heilen. Warte.“

Iril erholte sich wieder vom Schock. Turr der Takuri sank in sich zusammen und begann, zu verglühen. Dann zerfiel er zu Asche. Und aus der Asche erhob sich ein klitzekleines Vogelküken, welches auf Ack zuwatschelte und sie mit großen Kugelaugen anhimmelte.

Barz entspannte sich sichtlich und erklärte: „Ack hier ist nicht nur eine Hüterin der Takuri – sie wird auch von einem dieser mächtigen Wesen begleitet. Turr heißt er. Im Kampf steht er uns zur Seite, und er wird stärker, je älter er wird. Doch auch Takuri sind nicht unsterblich ... also Vorsicht! Brennt die Flamme des Takuri zu heiß, endet sein Lebenszyklus. Ein Glück, dass die tapfere Ack ihn immer wieder beruhigen kann.“ Mit einem Blick auf Ack, die ihren Feuervogel streichelte, fügte er an: „Aber warum sollte man ihn beruhigen, wenn man damit nur sein Leiden verlängert? Ihn umzubringen und regenerieren zu lassen, ist manchmal humaner.“

Iril hatte gerade einiges zu verarbeiten. Sie murmelte: „Ach so. Wiedergebärende Vögel. Das ist ja toll. Takuri nennt ihr sie? Und dieser hier heißt Turr? Und die Dame heißt Ack?“

Barz schwankte seinen Kopf herum in einer Mischung aus einem Nicken und Kopfschütteln: „Ja, es wird ‚Ack‘ ausgesprochen, aber es wird anders geschrieben, als du jetzt wohl denkst: Erst ein A, dann ein C mit einem Akzent, und dann ein H. Also: Aćh.“

„Toll“, wiederholte Iril immer noch etwas baff. „Ich bin ... „

Ijsdur unterbrach sie, indem er sich vorbeugte und auf Tulgorisch meinte: „Ich will auch jemanden vorstellen. Das hier ist die Silberzwergin Iril. Sie hat die Geheimnisse der Runen gemeistert. Mithilfe ihrer Runenscheibe ist sie in der Lage, die Geschicke der Helden im Kampf und auch außerhalb zu beeinflussen. Eine der ungewöhnlichsten Personen, die ich je getroffen habe. Ihre besondere Art der Magie macht sie nicht nur zu einer starken Kämpferin. Sie versteht es auch, mächtige Sonderaktionen einzusetzen, die wiederum allen nützen können. Das macht Iril zu einem starken Teamplayer! Das habt ihr ja schon gerade am eigenen Leibe erlebt.“

„Ich kam mir jetzt weniger wie ein Teamplayer vor, wie ich den Großteil des Kampfes benommen am Boden lag“, murmelte Iril.

„Keine Sorge, das kommt noch“, meinte Ijsdur tröstend.

„Wenn du es meinst. Na dann: Aćh und Barz, das hier ist Ijsdur. Er ist quasi ein lebendiger Schneemann“, stellte Iril Ijsdur kurz angebunden vor. „Und dann bleibt ja nur noch unser Echsenfreund übrig.“

„Ich bin Barz. Und auch wenn mir ‚Silberzwerg‘ nichts Genaues sagt, so erkenne ich doch eine Runenmeisterin, wenn ich eine sehe. Auf der anderen Seite des Fahlen Gebirges kannte kaum jemand die Macht der Runen.“

„Soll das heißen, du stammst ursprünglich von dieser Seite des Fahlen Gebirges?“

„Weit aus dem Osten, ja. Ich komme vom großen See Ava in der Barbarensteppe.“

Iril horchte auf, als ihr müder Geist endlich zwei und zwei zusammenzählte: „Ja, davon habe ich zumindest schon mal im Ansatz gehö ... warte, du bist kein Tulgori?“

„Wie sonst könnte ich deine Sprache verstehen?“

„Ich dachte an irgendein Übersetzungspulver. Magische Pulver scheinen voll dein Ding zu sein.“

Barz kicherte auf. „Über solch ein magisches Mittel verfüge ich leider nicht. Zumindest noch nicht. Vielleicht werden wir es finden. Ich war oft in der Steppe unterwegs auf der Suche nach derartigen Kräutern. Ich und mein Freund Nabib. Wir sind Steppennomaden aus den östlichen Landen, obwohl wir vom eigentlich sesshaften Stamm der Iquar stammen.“

„Nabib“, murmelte Iril. Diesen Namen hatte sie auch schon gehört. Doch wo? An einen anderen Steppennomaden konnte sie sich nicht erinnern. Viele mit solchen Steppenechsen siedelten seit über zwei Jahren draußen im östlichen Rietland, auf Erde, die der König von Andor den anstürmenden Barbarenhorden „großzügig“ überlassen hatte – obwohl sie eigentlich den Schildzwergen gehört hatte. Pah!

Sie verzog ihr Gesicht beim Gedanken daran und begrüßte Barz freundlich: „Freut mich, dich kennenzulernen, Barz. Schöner Name übrigens. Ich kenne einen Handelszwerg, der so heißt. Oder zumindest ganz ähnlich.“

Barz grinste. „Garz, nicht wahr? Wie ich schon sagte, Nabib und ich sind wieder zu weiten Reisen durch die Barbarensteppe aufgebrochen, haben Sippen der Yetohe begleitet, und auch immer wieder fahrende Händler angetroffen. Einer davon, ein gewisser Nader, hat uns von Garz und seinem Riesenrucksack erzählt. Ich kenne den Handelszwerg von nah und fern zumindest vom Hörensagen.“

Irils Kopf schwamm, womöglich immer noch vom harten Aufprall auf den Höhlenboden.

„Und ... und wie gelangtest du nach Tulgor? Wussten die Barbaren schon lange, dass da noch eine andere Welt ist? Oder ... nein, sag nicht, dass ihr weit genug in den Osten wandertet, um wieder im Westen aufzutauchen.“

„Wie klein stellst du dir die Weltenkugel vor?!“, lachte Barz, „Nein, nein, ich habe es nur mit einem Experimentierpulver übertrieben. Zu fahrlässig war ich. Zu rasch wollte ich Nabib wiedersehen. Doch das Schicksal hatte andere Pläne. Auf der Suche nach Nabib führte es mich für eine lange Zeit nach Tulgor, und erst jetzt nach Andor und zu euch. Mein Teleport verfehlte sein Ziel. Oder mein Ziel befand sich nicht dort, wo ich wollte. Und danach verfehlte mein Ziel sein Ziel. Dank der Zukunftssicht des Hüters der Zeit weiß ich schon, dass ich Nabib wiedersehen wurde. Dass er sich irgendwo hier befindet. Doch hieß das nicht, dass dieses Treffen leicht werden wird.“

Barz‘ Blick verklärte sich und er wippte nervös mit seinem Knie herum. Irils einhunderttausend Fragen wurden artig heruntergeschluckt.

Aćh rutschte näher zu Barz, vorsichtig das kleine Takuri-Küken an sich pressend. Sie flüsterte ihm etwas ins Ohr. Auch wenn Iril Wortfetzen hören konnte, reichte dies offenbar ihren Übersetzungsrunen nicht aus, um ihr die Bedeutung zu übermitteln.

Barz‘ Blick zuckte herüber zu Ijsdur, seine Augen verengten sich und Iril dachte zurück an das Misstrauen, das die beiden Tulgori ... nein, die Tulgori und der Barbar dem Eis-Dämon bei seinem Auftauchen entgegengebracht hatten.

„Tu nichts Unüberlegtes, was du bald bereuen kön ...“, brachte Ijsdur noch heraus. Dann hatte Barz bereits hastig an seinen Gürtel gelangt und eine Faustvoll eines grünlichen Pulvers auf Ijsdur geschleudert.

Magisches Knattern ertönte und ein hellgrünes Glühen baute sich um Ijsdur auf. Glitzernder Dampf stieg auf. Barz zog seine Hände in seinen Mantel zurück und wedelte die Dampfreste davon, ehe sie noch mehr befallen konnten.

Ijsdur war bereits zur Gänze betroffen. Er wirkte wie eingefroren, mitten in der Bewegung des Aufstehens erstarrt, sein Mund halb zu einer Erwiderung geöffnet. Sein ganzer Körper war von diesem grünlichen Glühen umgeben. Und nicht nur er, sondern auch die ihn stetig umgebenden Schneeflocken hingen starr in der Luft, als stünde die Zeit still.

„Halt, halt, halt, was wird das denn?!“, rief Iril protestierend, „Ijsdur rettete gerade euer aller Leben!“

„Vielleicht nur, damit wir die nächsten Diener Siantaris werden können!“, rief Barz nun. Emotion schwang in seiner Stimme mit. Ob Wut oder Furcht, konnte Iril nicht sagen. Er brummte weiter: „Die Eis-Dämonen dienen der unergründlichen Herrin des ewigen Eises. Bedingungslos! Wir hatten schon mit einer zu kämpfen. Du weißt nicht, wie es sich anfühlt, ein blinder Diener eines Eis-Dämons zu sein. Das ist ein schauderhaftes Gefühl. Man ist in Kontrolle seiner selbst, doch sein Wille ist nicht der, der er einst war, und es ist einem egal. Ich gehe kein Risiko mit Eis-Dämonen ein. Nie wieder.“

„Ihr handeltet etwas zu hastig. Ijsdur steht nicht mehr auf Siantaris Seite. Ich habe ihn von ihrem Joch befreit. Meine Runen beschützen ihn vor ihrem Willen! Und Ijsdur hätte mir doch nicht von Siantari erzählt, wenn er noch unter ihrem Einfluss stünde.“

„Und woher wollen wir wissen, dass du nicht auch auf ihrer Seite stehst?“

„Abgesehen davon, dass ich kein Schneemann bin und keine Hörner trage? Soll ich nochmal unterstreichen, dass wir gerade euer aller Leben gerettet haben? Unter Dankbarkeit stelle ich mir eigentlich etwas anderes vor!“

Die beiden vor ihr blieben still. Barz war sprachlos und Aćh verstand sie wohl nicht einmal. Iril betrachtete argwöhnisch den magisch gebannten Ijsdur, hielt sich aber betont davon zurück, zu ihrem Hammer zu greifen. Eine Eskalation war das letzte, das sie nun gebrauchen könnten. „Ist das umkehrbar, Barz? Was hast du überhaupt mit ihm gemacht?“

„Das ist ein Bannpulver. Gefertigt aus getrockneten Mondbeeren. Gespickt mit Mitteln des Hexers Haamun. Es schadet Ijsdur nicht. Für ihn wird es sein, als wäre gar keine Zeit vergangen, sobald das Mittel gelöst wird. Dafür bürge ich. Ich war auch schon selbst davon betroffen.“

„Wie beruhigend“, gab Iril zurück. „Könnte dieses Mittel denn nun rückgängig gemacht werden?“

Barz blickte fragend zu Aćh, welche ebenso fragend zurückblickte. Barz erklärte ihr in knappen Worten, was Iril ihm erzählt hatte.

„Selbst wenn er ein freier Eis-Dämon außerhalb Siantaris Einfluss ist, macht ihn das noch nicht zu einem freundlichen Gesellen“, gab Aćh zu bedenken.

„Jedem anderen, der euch aus dieser Höhle gerettet hätte, hättet ihr wohl bedingungslos vertraut!“

Barz kraulte seinen kurzen buschigen Bart: „Und das Bannpulver?! Ijsdur riet mir in der Höhle, dass ich das Bannpulver für später aufbewahren sollte. Wie konnte er vom Pulver wissen? Beobachtete er uns schon seit langem? Las er meine Gedanken? Und wie kam er überhaupt aus dem ewigen Eis hinaus?!“

„Das könnte er dir wohl alles sagen, wenn du ihn nicht gebannt hättest“, gab Iril zu bedenken.

Barz knirschte mit den Zähnen und beriet sich mit Aćh. Dann zog einen violetten Stein aus einer Manteltasche hervor. Er schob ihn in seinen Mund und biss darauf. Es knackte, und als Barz seinen Mund wieder öffnete und den Stein in seine Hand spuckte, war dieser in mehrere spitze Splitter zerfallen. Von ihnen allen ging ein immer heller werdendes Licht aus, welches Iril in den Augen schmerzte. Und wo diese Lichtstrahlen auf das hellgrüne Leuchten des Bannpulvers trafen, verschwand das grünliche Glimmen in der Luft. Das Bannpulver löste sich. Schon war Ijsdurs Kopf wieder frei, während sein restlicher Körper weiterhin mitten im Aufstehen eingefroren schien. Da schloss Barz seine Hand wieder um den Stein und verhinderte, dass Ijsdurs restlicher Körper freikam.

„... ntest. Oh! Das ist ein äußerst unangenehmes Gefühl“, beendete Ijsdurs Kopf seinen Satz.

„Das tut mir leid, Ijsdur“, murmelte Barz.

„Das musst du nicht sagen, wenn es dir nicht wirklich leidtut“, sprach Ijsdur kühl. Er reckte probehalber seinen Hals und gab auf, seinen restlichen Körper bewegen zu wollen.

Barz nickte. „Lassen wir die Höflichkeiten. Falls du vertrauenswürdig bist, hast du soeben unser Leben gerettet und wir verhalten uns äußerst undankbar. Das tut mir durchaus leid. Aber falls du geheime Pläne mit unschönem Ausgang für die wärmeren Teile der Welt hegst, würde ich diesen liebend gerne ein Ende setzen. Wie können wir dir vertrauen?“

Ijsdur blieb einen Moment lang still. Dann sprach er: „Bedenkt meine vergangenen Aktionen. Und hört auf Iril. Ihre Runen ermöglichen es ihr, gemeine Lügen von ehrlichen Aussagen zu unterscheiden. Auch wenn ich selbst noch nicht herausgefunden habe, wie genau es funktioniert, könnt ihr sie als Lügendetektor nutzen. Ich vermute, dass ihr ihr ja vertraut.“

Eine leise Bitterkeit schwang in der sonst so tonlosen klirrenden Stimme mit.

Ijsdur blickte herüber zu Iril, welche betroffen zu Boden blickte. Sie seufzte tief und murmelte: „Ich wünschte, ich könnte da aushelfen. Doch im Namen der Offenheit muss ich zugeben, dass dies eine Täuschung war. Als ich dich frisch getroffen hatte, hoffte ich darauf, dich so auszuhorchen. Doch Lügen sollen nicht weitergesponnen werden, bis sie unniederreißbar groß sind. Alles, was ich sagen kann, ist, dass Ijsdur im Glauben, mir nur die Wahrheit sagen zu können, dasselbe sagte, was ich euch sagte. Er steht nicht mehr in Siantaris Diensten.“

Ijsdurs Miene blieb unergründlich. Nicht einmal eine Antwort kam von ihm. Dafür von Barz: „Hättest du das nicht erst in einigen Minuten auflösen können?! Nun könnte er sich alles Menschenmögliche ausdenken, wenn wir ihn nach dem Bannpulver fragen.“

„Vielleicht hätte ich sollen. Aber vielleicht ist es auch besser ...“

„Was wolltest du mich zum Bannpulver fragen? Warum ich darüber Bescheid wusste? Sagt, erkennt ihr beide mich nicht?“

„Sollten wir?“, fragte Barz argwöhnisch.

Ijsdur verzerrte sein Gesicht zu einer betont fröhlichen Miene und rief in einer melodischeren Stimme: „Willkommen auf dem Hängeschiff ARCTOR, meine sehr verehrten Herrschaften, Damenschaften und allerlei anderen Mitglieder der Gesellschaft. Mein Name ist ....“

„... Ijs?!“, rief Aćh ungläubig, „Der Kapitän des Hängeschiffs, das vom Sturmgeist angegriffen wurde?! Was ist mit dir geschehen?“

„Ich vermute, dass ihr das bereits richtig vermutet. Ijs verirrte sich im tödlichen Kuolema-Gebirge.. Ich bin Ijsdur, sein eisiges Ebenbild, geschaffen aus seinem Leichnam durch die Eiskristallkette meines ‚Vaters‘, des Eis-Dämons Beriandur.“

Barz runzelte seine Stirn und blickte Ijsdurs muskulösen Körper entlang: „Hast du an Muskelmasse zugelegt? Wie läuft das mit diesen Schneekörpern, kannst du bis zu einem gewissen Grad bestimmen, wie er aussieht?!“

„Ich sehe, dass mein Körper nicht identisch zu Ijs altem ist. Und doch glaube ich nicht, direkte Kontrolle über seine Form zu haben. Ein bislang unerforschtes Phänomen.“

„Und Siantari hat dich wirklich nicht in ihrem Griff?“

„Einst tat sie es, doch ich bin kein blinder Diener Siantaris mehr“, sagte Ijsdur, „Doch bin ich auch nicht Ijs. Zu wenig Menschliches ist in mir übrig, um diesen Namen weiterzuführen. Ijs‘ Vater Saro sagte, er erkenne seinen Sohn nicht mehr in mir, und ich stimme ihm teils zu. Tulgor bedeutet mir nichts mehr. Meine Heimat ist das ewige Eis. Die Schluchten des Kuolema-Gebirges. Vielleicht wäre es besser für alle, wenn ich einfach dorthin zurückkehrte. Wenn ihr das wollt ...“

„Machst du Witze?“, meldete sich nun Iril zu Wort, „Bei allen Kreaturen der Tiefe, deine Anwesenheit hier ist ein Geschenk! Wenn ihr beide nicht bald aufhört damit, Ijsdur schlecht fühlen zu lassen, mögt ihr beide euch von uns trennen!“

„Wir waren damals nur vom Wunsch erfüllt, Siantari zu dienen“, erinnerte sich Aćh erschaudernd, „Und wir beiden hatten unsere Ketten nur kurz berührt. Wie kann das bei dir anders sein, erst recht nach all dieser Zeit?“

„Magie“, meinte Ijsdur schlicht. „Runenmagie. Scheint genug, um Siantaris Einfluss fernzuhalten.“

„Für den Moment“, murmelte Barz.

„Für den Moment“, bestätigte Ijsdur, „Siantari hat die Eiskristallketten geschaffen und in den Umlauf gebracht. Niemand außer ihr kann wissen, wie sie aus ihrer Hand befreit werden können. Und das heißt ...“

„... dass Siantari dich immer noch in der Hand haben könnte, wenn du ihr zu nahe kommst“, murmelte Barz bedrückt, „Vielleicht reicht es auch nur, wenn sie genügend stark an dich denkt.“

„Nein“, sprach Iril nun bestimmt, „Diese Runenscheiben funktionieren. Dutzende Geister haben sie bereits ausgetrieben. Wenn sie verfluchte Seelen vom Silberberg fernhalten können, dann auch den Willen einer Eis-Dämonin.“

„Wir können hoffen.“

Stille.

„Kann Siantari wahrnehmen, was du wahrnimmst?“, fragte Barz nun besorgt.

„Vielleicht?“, murmelte Ijsdur bedrückt, „Wenn, dann höchstens auf Wunsch. Es gibt so einige halb verrückte Durs und Doras im Eisgebirge. Sie kann sich kaum all deren Geister gleichzeitig bewusst sein. Vielleicht weiß sie ja nicht einmal, dass ich hier bin.“

„Weiß sie nicht“, bekräftigte Iril noch einmal. „Die Gefahr durch sie ist noch nicht gebannt. Aber Ijsdur ist nicht Teil davon.“

„Großartig, dann kennen wir uns ja alle und sind neue beste Freunde“, meinte Barz. Er öffnete seine Faust wieder. Das letzte Licht der violetten Steinsplitter befreite Ijsdur vollends vom Bann des Bannpulvers. Er stürzte zu Boden und richtete sich betont langsam auf.

Unsicher blickten die vier einander an.

„Was nun?“, fragte Barz.

„Zuallererst erkläre ich wohl meine Übersetzungsrune“, meinte Iril. „Ihr habt alle schon Tulgorisch gesprochen, also nehme ich an, dass ihr das schon kapiert habt, aber noch mal explizit: Dank dieser Rune an meinem Kopf können ihr drei auf Tulgorisch schwafeln und wir verstehen einander alle. Außer Aćh mich. Aber damit sollten wir klarkommen.“

Barz übersetzte und Aćh lächelte erleichtert, beim Gedanken, nicht in Zukunft von den meisten Gesprächen ausgeschlossen zu sein.

„Ansonsten könnten wir demnächst noch die Hexe Reka für einige Übersetzungstränke anhauen“, bemerkte Ijsdur, „Ich bekam mit, wie sie Eforas welche anbot.“

„Au ja, solche Tränke kenne ich schon von meiner Schamanin Asbark!“, rief Barz, „Asbark hat das Rezept dazu vom Fleisch gewordenen großen Büffel höchstpersönlich erlernt. Also, eher von Mitgliedern der Yetohe-Sippe, die es vom Yjotege grate hatten, aber das sollte kein Problem sein, orale Tradierung ist durchaus verlässlich. Auf jeden Fall ein wahres Wunderwerk der Kommunikation.“

Er drehte sich zu Aćh herum und stupste grinsend sie in die Seite: „Sieht so aus, als wären unsere Rollen nun vertauscht. Bist du bereit, eine neue Sprache zu lernen?“

Aćh nickte tapfer.\bigskip







Die vier beschlossen, fürs Erste zusammenzubleiben, ein Lager einzurichten und die Nacht hier zu verbringen. In einer sicheren Senke ein bisschen oberhalb des Stollenausgans.

Barz sammelte Holz. Aćh brach eine feurige Feder – welche sie aus einer Tasche zog und nicht etwa dem kleinen Küken abrupfte – und entzündete das Lagerfeuer. Iril kritzelte auf ihrer Runenscheibe herum. Ijsdur saß gehörig vom Feuer entfernt und schien zutiefst in Gedanken versunken. Hin und wieder ertappte Iril ihn dabei, wie er die beiden Neuzugänge verstohlen anguckte. Und sie ihn ebenfalls.

Aćh traute sich als Erste zu Wort: „Wie hast du es eigentlich aus dem Tal des ewigen Eises geschafft? Wir selbst bekamen mit, wie Eis-Dämonen vom Felsentor aufgehalten wurden. Hat Siantari es endlich geschafft, jemanden außerhalb des Felsentors zu wandeln?“

„Nein, auch Ijs starb mitten im ewigen Eis. Eines Tages sprach Santari einfach, dass wir frei wären. Und wir waren es.“

„Das magische Felsentor war offen? Ihr konntet einfach durchmarschieren?“

„Offenbar. Keine Ahnung, warum.“

Die Helden bereiteten sich auf Schlaf vor. Die beiden Mitglieder der tulgorischen Reisetruppe hatten Decken dabei. Auch Iril trug stets eine dünne Plache in ihrem Rucksack. Das hatte sie schon auf Silberland getan, für die Fälle, wo die Mondscheinspaziergänge mit Burmrit an der Bronze- oder Kupferküste zu lange andauerten, um in die Mine zurückzukehren.

Kaum waren drei der vier in Schlafgewänder gewechselt und hatten sich rund ums Feuer verteilt, blickten sie erwartungsvoll Ijsdur an.

Nun war es Barz, der das Wort ergriff: „Sag, Ijsdur, musst du auch schlafen?“

„Es wäre überaus praktisch, wenn ich es nicht müsste. Aber doch, mein Körper muss ruhen. Selbst Nahrung zu sich nehmen muss er, solange er sich nicht auf dem ewigen Eis befindet und von dort aus Energie aufnehmen kann.“

„Wie kann das überhaupt sein, dass du Nahrung zu dir nehmen musst, obwohl du gar kein Blut hast? Du ... du hast doch gar kein Blut, oder?“

„Woher willst du das wissen? Nur weil du schon mal einen anderen Eis-Dämon abgestochen hast?“, gab Ijsdur ein bisschen schnippisch zurück, „Was soll Essen denn mit Blut zu tun haben?“

Barz erklärte: „Wohin soll die Kraft des Proviants ohne Blut schon hin? Der rote Saft transportiert sie durch den Körper dorthin, wo sie hinsoll. Meine Schamanin Asbark erzählte immer, dass man im Falle einer Überessung ...“

„Quatsch, Essen geht doch nicht ins Blut! Das geht durch den Magen und dann hinten wieder heraus.“

Nun war auch Iril neugierig geworden: „Ijsdur, musst du dich manchmal auch erleichtern? Sieht es auch aus wie Schnee?“

Da fragte Aćh auch schon: „Können Eis-Dämonen immer noch schwanger werden?“

Ijsdur blickte sie alle ausdruckslos an. Zu guter Letzt murmelte er nur: „Das kann ich noch nicht wissen. Und das werde ich vielleicht nie wissen.“

Aćh fragte: „Vermitteln diese Eiskristallketten nicht irgendwie intuitives Wissen zum Dämonen-Dasein?“

„Ja, aber solche Dinge weiß doch nicht mal Siantari. Im ewigen Eis gelten für uns andere Regeln als hier in der warmen Welt.“

„Du hast du dich also noch nie erleichtert?“

„Noch nicht, aber ich bin ja erst kurz vom ewigen Eis weg und habe nur wenig gespeist.“

„Da war ich dabei!“, rief Iril, „Er hat einen ganzen Teller Bohnensuppe in einem Happs in sich hineingeleert und seither kein Quäntchen mehr verzehrt.“

„Ich verstehe nicht, weshalb dich das so mit Freude erfüllt.“

„Ist nur interessant, zu lernen, wie dein Schneekörper funktioniert.“

„Ah.“

Stille.

„Was ist der Plan für morgen?“, fragte Barz.

„Wir sollten die anderen Tulgori aufspüren“, meinte Aćh.

Ehe sie weitersprechen konnte, berichtete Iril: „Nun, Ijsdur und ich waren eigentlich auf den Spuren von Drachenknochendieben, ehe ihr aufgekreuzt seid. Wollt ihr uns dabei unterstützen, sie zu finden? Oder vertraut ihr Ijsdur zu wenig dafür?“

Barz‘ Stimme wurde müder. „Drachenknochen? Faszinierend. Meine Lehrmeisterin meinte, dass Drachenknochen vor Jahrhunderten sehr beliebte Mittel für stärkende Pulver waren“, murmelte Barz müde, „Doch diese Zeiten sind vorüber. Es gibt so gut wie keine Drachen mehr.“

„Genau“, sprach Iril, „Tarok, der hoffentlich letzte Drache, ist gerade erst gestorben. Es geht hier um seine Knochen.“

„Tarok?!“, rief Barz plötzlich wieder hellwach. Seine Stimme hatte einen zitternden Unterton angenommen.

„Ja, Tarok, der Gewaltige. Der Mächtige. Der Rächer. Er soll viele Titel haben.“

Barz murmelte leise, wie zu sich selbst: „Haamun sagte noch, dass das Öffnen des Stollens irgendetwas mit einem Drachen zu tun hätte, aber dass es ausgerechnet Tarok sein musste!“

„Kennst du Tarok?!“, fragte Iril.

„Er hat Sabris Mutter ermordet“, kniff Barz zwischen zusammengebissenen Zähnen hervor.

„Was?!“

Barz atmete schwer und zückte eine kleine Drachenfigur aus seiner Manteltasche. Mit betont kontrollierter Stimme sprach er weiter: „Dies ist eine Figur von Tarok. Unser Kind schnitzte diese Figur, nachdem Tarok unsere Pfahlbausiedlung überfallen und unsere Steppe in Brand gesetzt hatte. Karyz hoffte ... nun, eigentlich ging Tarok einfach allen nicht aus dem Kopf.“

„Das wusste ich überhaupt nicht“, sprach Aćh baff. „Du hast nie erzählt, dass du so nahe mit einem Drachen zu tun hattest.“

„Du hast nie gefragt“, sagte Barz.

Aćh murmelte etwas davon, dass sie ja nichts wusste von diesem Land, in der sie sich hier befanden. „Ich kenne seine Geschichten kaum. Weder seine Orte noch seine Leute. Und nicht einmal von dir, Barz, weiß ich so viel, wie ich dachte. Welcher Geist hat mich schon wieder geritten, hierher zu kommen?“

Sie kuschelte sich an Barz. Dieser legte einen Arm um sie und sprach ihr beruhigend zu. „Die Vergangenheit ist vergangen. Da muss man nicht alles voneinander wissen, um auf gute zukünftige Pfade zu kommen. Ich verstehe, wie man sich in einem fremden Land fühlen kann. Mir ging es nicht anders, als es mich nach Tulgor verschlug. Gar schlechter sogar. Doch du und die deinen waren auf jedem Schritt an meiner Seite. Und im Gegensatz zu mir kannst du jederzeit nach Hause aufbrechen, wenn dich das Heimweh übermannt.“

„Was, an den aufgebrachten Lumiwürmern vorbei?!“

„Zugegeben, so einfach wird das nicht werden. Aber auch nicht unmöglich. Erst recht nicht jetzt, wo der Weg über die Berge wieder offensteht. Mit den richtigen Leuten und Mitteln an unserer Seite ist alles möglich.“

Iril ertappte sich beim Starren und wandte ihren Blick Ijsdur zu. Dieser sah ihr ausdruckslos entgegen. Auf seiner Brust glitzerten wie üblich die Stacheln seiner magischen Kette.

Iril dachte zurück daran, was die anderen erzählt hatten. „Sieht so aus, als wäre ich die Einzige hier, die noch nie einer solchen Eiskristallkette unterlag.“

„Es ist gar nicht so seltsam, sie zu tragen. Willst du ausprobieren, wie es sich anfühlt? So eigenartig ist es gar nicht. Außer ihr Zwerge reagiertet anders darauf mit eurem vor Dunklen Magie gefeiten Blut und so.“

Ijsdur streckte seinen Hals vor. Iril betrachtete die glitzernden Kettenglieder, die mit seinem Hals verschmolzen waren. Iril fasste sich ein Herz, streckte ihre Finger hoch und streifte einen eiskaltes Eiskristall. Er war so kalt, dass die Berührung in ihre Haut stach.

Und auf einmal war alles dumpfer. Ihr Ärger gegenüber den undankbaren Neuankömmlingen. Die Sorge um die diebischen Drachenkultisten. Die sich stetig vordrängelnden Gedanken um ihre tote Runenmeisterin Burmrit und um ihre Familie. Sorge um Iolith. Selbst die vor Kälte bibbernden Finger auf Ijsdurs Eiskristallkette. All diese Wahrnehmungen waren noch hier, doch abgeschwächt, weniger mitreißend.

Irils und Ijsdurs Geister streiften einander. Iril spürte, wie eine fremde Präsenz durch ihre Erinnerungen huschte. Bilder und Emotionen ebbten ebenso schnell auf, wie sie aufgewallt waren. Dann war es schon wieder vorbei. Iril ließ die Eiskristallkette los und trat einige Schritte zurück.

„Beeindruckend“, sprach Iril, „Das muss ich mir merken, für wenn mein Gemüt mir mal übel mitspielen sollte.“

„Du bist gerne eingeladen, die Kühle der Kette zu genießen, wenn du sie brauchen könntest.“

Iril blickte sich um und machte Blickkontakt mit Aćh und Barz, welche offenbar ihr Gespräch unterbrochen hatten und ängstlich Irils Reaktion beobachten.

„Keine Dämonin hier“, grinste Iril. „Wie schon gesagt: Ijsdur ist geschützt vor den fremden Einflüssen aus dem ewigen Eis.“

Die beiden entspannten sich.

Im Norden schrie eine Eule.

Iril blickte in die Ferne. Von ihrem Lagerplatz am Hang des südlichen Gebirges waren in weiter Ferne gerade noch so die Turmspitzen die Rietburg zu erkennen. Feuerschein aus vereinzelten Fenstern erhellte das Gemäuer.

Aćh wog die tulgorische Steinflöte in ihren Händen und betrachtete nachdenklich die Silhouette der Rietburg in weiter Ferne, bevor sie die Flöte an ihre Lippen setzte und behutsam eine Melodie zu spielen begann. Ein ruhiges, doch fröhliches Lied, das die Helden aufmunterte und ihnen half, sich auf die Planungen ihres Vorgehens gegen die bedrohlichen Drachenkultisten zu konzentrieren. Während die Töne der Flöte in die dunkle Nacht schallten, tanzte Turr der Takuri fröhlich über ihren Köpfen zu den kunstvollen Klängen.

Die letzten Klänge der fremdartigen Melodie verklangen. Barz klatschte Applaus. Die anderen stimmten mit ein.

Sabri schnarchte auf.

Es war Zeit, sich schlafen zu legen.\bigskip







Die ganze Küche war voller Schneewirbel. Kristalle schmolzen auf dem heißen Metall des Ofens in der Ecke. Saros Kleidung war eiskalt und feucht, sein langer Bart übersät von Reif, als er zitternd sprach:

„Was willst du noch hier? Du wirst mich bald verlassen, wie Eforas uns verließ. Wie Ijs‘ Mutter uns alle verließ. Es scheint mein Schicksal zu sein, allein zu bleiben. Ziehen wir es nicht unnötig lange heraus. Alles ist gesagt. Du bist hier nicht mehr willkommen, Ijsdur.“

„Ijs ... dur? Bin ich das? Bin ich nicht mehr Ijs? Bin ich kein Mensch? Bin ich ein Eis-Dämon?“

„Du bist kein Mensch“, brummelte Saro, „Du bist nicht als eine Erinnerung, die sich im ewigen Eis festsaß. Und schmerzhafte Erinnerungen gibt es schon so zu viele in diesem Haus.“

„Vater, bitte, deine Schmerzen werden sich durch diese Worte nicht lindern. Und die meinen auch nicht.“

„Vater?!“, lachte Saro bitter, „Du bist nicht mein Sohn. Mein Sohn starb auf dem Gebirge. Ich erkenne dich nicht wieder.“

Er griff nach einer Mistgabel und hielt sie in die Flammen des Ofens, bis ihre Zinken rötlich glühten.

„Du weißt nicht, was du tust“, sprach Ijsdur kalt, „Lass die Waffe sein.“

Saro ließ die Mistgabel los und stolperte einige Schritte zurück. Mit Tränen in den Augen blinzelte er zu Ijsdur hoch. Er öffnete seine Arme, als wolle er ihn umarmen. Dann ließ er sie abrupt sinken und rannte aus dem Raum.



Der Traum wandelte sich, änderte sich, spulte vor.



Die ganze Küche war voller Schneewirbel. Kristalle schmolzen auf dem heißen Metall des Ofens in der Ecke. Saros Kleidung war eiskalt und feucht, sein langer Bart übersät von Reif, als er zitternd sprach: „Nein. Ich lasse dich nicht gehen, Ijsdur.“

„Ich habe dich nicht gefragt, Vater. Ich habe dir mitgeteilt, was ich tun werde. Ich werde nach Andor aufbrechen und Eforas suchen, ganz egal, was du willst.“

„Nein“, wiederholte Saro, „Ich sehe, was du vorhast. Warum du dich plötzlich so sehr für die Länder im Osten interessierst. Der Herr des ewigen Eises hat von Tulgor abgelassen und ein anderes Ziel gerochen. Ein Land, in dem niemand weiß, wie man Eis-Dämonen einsperrt. Du willst Eforas abfangen, ehe er die Andori vor euch warnen kann. Deine Herrin hofft, dass du ihn als seinen Bruder täuschen könntest.“

„Saro, wie kannst du so etwas denken?!“, entrüstete Ijsdur sich, „Die Herrin Siantari entließ uns aus ihren Diensten. Wir Eis-Dämonen sind frei. Wir können handeln, wie wir wollen.“

Das war eine Lüge. Ijsdur war frei gewesen, das ewige Eis zu verlassen und nach Tulgor zurückzukehren. Siantari hatte ihr Interesse daran fürs Erste verloren. Doch sobald Ijsdur von Eforas‘ Aufbruch erfahren hatte und sobald Siantari von Ijsdurs Verbindung zu Eforas erfahren hatte, hatte sie sofort seinen Geist wieder in ihre Klauen gepackt und ihm gedanklich befohlen, die Reisegruppe zu verfolgen. Befohlen, nach Andor zu gehen und mögliche Verteidigungen auszuspionieren. Tari würde nicht ein zweites Mal in eine Falle gehen. Die Andori durften nicht vorgewarnt werden.

Und so hatte Ijsdur sich von Saro verabschiedet. Vielleicht hätte er lügen sollen. Denn nur die Erwähnung Eforas‘ hatte Saros Alarmglocken schellen lassen. Und nun schien Saro ihm nichts mehr zu glauben.

Saro griff nach einer Mistgabel und hielt sie in die Flammen des Ofens, bis ihre Zinken rötlich glühten. Wenn er sich gegen ihn stellte, müsste Ijsdur ihn umbringen. Das wollte er nicht.

„Deine Meinung ist bedeutungslos, Saro“, sprach Ijsdur kalt, „Dein Leben ist bedeutungslos. Darum verschone ich es. Verschwende es nicht, indem du mich aufzuhalten versuchst.“

Saro fuhr herum und hielt die feurige Mistgabel in Ijsdurs Richtung.

Ijsdur lachte kalt auf. „Guck mir in die Augen. Du traust dich nicht. Ich bin immer noch dein Sohn.“

Saro zögerte. Dann schnellte er vor und zog die feurige Waffe über Ijsdurs Brust. Tiefe Furchen hinterließ sie. Die Eiskristall-Kette zerbarst in kleinste Teile, welche wie Rauch verdampften. Ijsdurs Bewusstsein wurde ebenso in kleinste Teile zerteilt, welche alle klagend aufschrien. Ein Zittern durchlief seinen ganzen Körper.

Er schmolz dahin.



„Nein“, dachte er, „So war das nicht. Saro hätte nicht so gehandelt. Er hätte es nur sollen.“\bigskip







Ijsdur schreckte hoch.

Offenbar vermochte er nicht nur zu träumen, sondern auch albzuträumen.

Er schwitzte nicht mehr, doch keuchte er angestrengt. Seine Emotionen mochten gedämpft sein im Vergleich zu denen eines Menschen, doch ausgelöscht waren sie deswegen nicht.

Er musste sich beruhigen.

Es war tiefste Nacht.

Die Sterne funkelten, der Boden brummte und sein Magen grummelte. Ihm wurde bewusst, dass er außerhalb des ewigen Eises wohl tatsächlich hin und wieder eine Notdurft verrichten musste. Ijsdur hatte in seiner kurzen Zeit in Andor schon von den andorischen Scheißhäusern erfahren. Was Komfort und Geruch anging, waren diese leider den öffentlichen Toiletten Tulgors unterlegen, doch waren sie dem Gang in die Natur durchaus vorzuziehen. Wobei Ijsdur hier, am Rande des südlichen Walds wohl kaum einfach so ein Scheißhaus antreffen würde. Leise schlich Ijsdur sich ins nächstbeste Gebüsch genügend abseits des gelöschten Lagerfeuers und tat, wie es ihm die Natur gebot.

Dann hielt er inne.

Ein Windhauch wehte durchs Gebüsch und um Ijsdur herum. Er trug einen Geruch nach Metall und Verwesung mit sich.

Etwas war falsch hier.

Jemand war hier. Jemand beobachtete ihn.

In weiter Ferne glitzerten zwei schneeweiße Augen hinter einem Gebüsch hervor. Pupillenlos, wie die von Kreaturen, Drachen oder Riesen – oder Eis-Dämonen. Ijsdur kniff seine Fenster zur Seele zusammen und spürte, wie sich etwas in seinem Kopf verschob. Auf einmal konnte er die fern liegende Gestalt scharf erkennen.

Es war zweifelsohne eine Person. Sie schwebte über dem Boden, der wallende Umhang ihre tatsächliche Form verhüllend. In der einen Hand führte sie ein Schwert. Ihre zwei glühenden Augen stachen hinter einer gezackten eisernen Maske hervor.

„Der Schwarze Herold, nehme ich an?“, brach Ijsdur die Stille, „Solltest du nicht über deine Kultisten wachen? Was hast du hier bei uns vor?“

Laut klirrte seine kalte Stimme durch die stille Nacht. Die Gestalt rührte sich nicht. Was überlegte sie?

„Wir wollen den Kultisten nichts Böses, das kannst du ihnen ausrichten“, fuhr Ijsdur fort, „Iril will nur die Drachenknochen zurück. Wir finden euch mit ihrer Hilfe bald, da könnten wir uns doch genauso gut gleich auf ein Treffen einigen.“

Der Herold schwebte wortlos in der Dunkelheit, als wäre er nichts mehr als ein Schatten, den ein seltsam geformter Baum im Mondlicht warf. Doch Ijsdur wusste es besser.

„Wenn du dachtest, dass ich auf deiner Seite wäre, muss ich dich enttäuschen“, murmelte Ijsdur, „Ich unterstütze das Böse nicht. Da können dir andere Eis-Dämonen besser weiterhelfen. Bitte, so sage doch etwas.“

Er blinzelte. Auf einmal war der Schwarze Herold verschwunden. Hatte Ijsdur dies nur geträumt? Bis vor wenigen Tagen hatte er nicht einmal schlafen müssen. Nein, noch konnte Ijsdur zwischen Traum und Realität unterschieden. Der Herold war wirklich hier gewesen. Das konnte kein gutes Zeichen sein.\bigskip







Als Ijsdur ins Lager zurückkehrte, drehte sich Aćh rasch auf die andere Seite. Aber nicht so rasch, dass es ihm nicht aufgefallen wäre.

„Belauschtest du meinen nächtlichen Toilettengang?“, fragte Ijsdur leise, während er es sich wieder auf seinem behelfsmäßigen Nachtlager gemütlich machte. „Keine Anschuldigung, nur eine ernsthafte Frage.“

Eine Zeit lang kam keine Antwort. Er seufzte und stellte sich darauf ein, ins Reich der Träume überzugleiten. Doch dann ...

„Ich hörte dich sprechen. Mit wem hast du gesprochen?“

Aćh klang misstrauisch. Das war auch nur zu verständlich, wenn auch ärgerlich. Ijsdur schien es am geschicktesten, offen mit ihr und Barz zu sein, bis sie ihr Misstrauen gegenüber den Seinen revidiert hätten.

„Das war der Schwarze Herold. Ich habe mich mit ihm unterhalten. Einseitig.“

„Soll das ein Witz sein?“, meldete sich eine weitere verschlafene Stimme. Iril war nun auch wach.

„Kein Witz“, bestätigte Ijsdur. „Der Herold hat wirklich einfach hier geschwebt. Und uns beobachtet.“

„Wer ist der Schwarze Herold?“, verlangte Aćh zu wissen.

„Der Vorbote allen Unbills. Der Heerführer der Kreaturen. Der höchste Diener des Drachen. Eine Ausgeburt des Bösen“, gab Iril theatralisch zurück.

„Eine finstere Person, die das Böse unterstützt“, übersetzte Ijsdur knapp für Aćh. Iril schien sich damit zufriedenzugeben, denn sie schnarchte wieder lautstark los.

„Was wollte dieser Herold von dir?“

„Von mir? Keine Ahnung. Vielleicht hoffte er, dass ich ihn unterstütze? Doch warum ausgerechnet mich?“

„Könnt ihr euch alle mal ruhig verhalten?“, erklang Barz‘ Stimme, gefolgt von einem lauten Gähnen. „Manche hier würden gerne schlafen.“

Aćh schwieg. Nachdem keine Antwort oder weitere Frage von ihr folgte, ließ sich Ijsdur ebenfalls endgültig wieder ins Reich von Gevatter Schlaf mitziehen.\bigskip







Barz wurde beim Sonnenaufgang als Erster wach und blinzelte in den sich rötenden Himmel. Er vermisste die muffige Stollenluft wirklich nicht. Er reckte und streckte sich und zog aus in die Umgebung, auf der Suche nach Mondbeeren, welche man besonders gut im Lichte der ersten Sonnenstrahlen finden konnte. Leider wurde er hier nicht fündig. Nur mit einigen gesammelten Apfelnüssen als Ausbeute kehrte er zum Lager zurück, streichelte die im Schlaf grunzende Sabri hinter den Ohren und knuddelte Turr, welcher bereits aktiv umherhüpfte.

Iril schälte sich aus ihrem Nachtlager und beobachtete den Takuri aufmerksam. Turr sah schon wieder älter als ein Küken aus. Seine flaumigen Federn strahlten und seine Schwanzspitze sprühte gar Funken.

„Er wächst viel zu schnell, selbst für einen Takuri“, murmelte die aufgewachte Aćh sorgenvoll, „Schon seit seinem Verschwinden damals mit Barz. Ich frage mich, ob er die Welt in Zeitlupe wahrnimmt. Aber er scheint nicht zu leiden. Also ist es wohl in Ordnung.“

Emsig machten die vier sich ans Aufräumen ihres Lagers und an die Vorbereitungen für die Abreise. Aćh erzählte Barz nebenher von ihrem nächtlichen Traum, in welchem sie einen störrischen Takuri davon überzeugen musste, einen leuchtenden Mera-Stein vom Himmel zu holen, damit der Nistbaum der Takuri mit ihm tanzen konnte.

Barz lachte und präsentierte der Gruppe seine gefundenen Apfelnüsse sowie einige Streifen Dörrfleisch aus einem Sack von Sabris Rücken.

Iril fand in ihrer Reisetasche einige Tarenkugeln – gefüllte Teigtaschen, die der kultivierte und feingeistige Bragor gerne zubereitete und anschließend anderen verschenkte, sofern er sie nicht sofort ohne Essforken und Löffeln verzehrte – sowie einen letzten Brocken würzigen Silberbrots. Barz wurde auf die goldbraunen Teigbälle und das glitzernde Gebäck aus Silberhall aufmerksam und bat um Kostproben – welche er sofort mit einem rötlichen Pulver aus seiner Gewürztasche schmackhaft würzte.

Proviantrationen wurden verteilt, verzehrt und beurteilt.

„Köstlich! Diese Teigkugeln munden auch kalt. Sie würden sich bestimmt großer Beliebtheit erfreuen, wenn sich ihr Rezept weiterverbreitete.“

„Warum ist euer Brot so fade? Da gehört eine gesalzene Ladung Drachenfrüchte darauf. Die Drachenfrüchte am Ava wachsen nach oben, wie es der große Seeadler will. Vielleicht wachsen eure hier nach unten und verlieren deswegen ihren Geschmack.“

„Was, ihr tut gar keine Drachenfrüchte da rein? Ihr verpasst etwas. Drachenbohnen sind eine meiner Leibspeisen.“

„Sollen Apfelnüsse so bitter schmecken?“

„Nicht alle. Aber es ist gut, wenn sie es tun. Dann haben sie meisten Nährstoffe.“

Nur Ijsdur meinte, er sei nicht hungrig, und schaute stumm zu, wie die anderen Helden speisten.

Nach dem Morgenmahl flocht Iril sich einen neuen Haarkranz. Barz stülpte seinen schweren Mantel über und pflegte seinen Bart. Aćh legte ihre zeremonielle Rüstung an und salbte ihre Filzlocken mit feuerfestem Sufar ein. Ijsdur behielt seinen Rock an und blickte gedankenversunken in die Ferne. Vielleicht war seine Kleidung Teil seines Schneekörpers und musste darum nicht gewechselt werden.

Anspannung lag in der Luft. Iril rief sich ihre Hauptaufgabe wieder in Erinnerung: Das Aufspüren der gestohlenen Drachenknochen.

Ehe Iril erneut zu ihrer Runenscheibe greifen musste, um die aktuelle Position der Knochen anzupeilen, stürzte ein kleiner Krark aus dem Himmel herab, kaum größer als Irils Unterarm. Der Raubvogel landete ungelenk auf dem Boden und erhob sich flatternd wieder in die Lüfte. Barz griff hastig nach seinem Bogen, doch Ijsdur bat ihn, ruhig zu sein. „Ich sprach doch davon, dass uns letzte Nacht der schwarze Herold beobachtete. Die Drachenkultisten wissen wohl, dass wir hier sind und sie aufspüren können. Es ist auch in ihrem Interesse, diese Angelegenheit zu klären, ehe das Ganze zu einem Konflikt mit der gesamten Rietgarde führt. Folgen wir doch einfach dem Krark und sehen wir, wohin er uns führt.“

„Ja natürlich, folgen wir einfach dem feindlichen Flatterviech mit scharfen Federn und noch schärferen Klauen“ grummelte Iril, „Es wäre im Interesse der Kultisten, uns in eine Falle zu führen.“

„Dann geben wir halt Acht, dass wir in keine Falle tappen“, lächelte Ijsdur.

Aćh blickte ihn misstrauisch an, nickte dann aber. „Ein wenig Vertrauen kann einen weit bringen. Ich würde lieber nicht schon wieder kämpfen. Folgen wir dem Krark.“

Damit schien das letzte Wort gesprochen zu sein. Nachdem die vier ihr Lager zusammengepackt hatten, brachen sie auf. Barz‘ Steppenechse Sabri ließen sie zurück. Sie konnte in Konflikten kaum aushelfen und hätte sie mit ihrer Trägheit nur aufgehalten.

„Keine Sorge“, sagte Barz, „Sabri kann für sich selbst sorgen, wenn nicht gerade eine Horde Lumiwürmer auf sie aus ist. Und bislang hat sie immer wieder zu mir zurückgefunden, als hätte sie einen siebten Sinn dafür.“

„Es gibt doch mehr als sieben Sinne“, warf Iril reflexartig ein.

„Behauptet ja keiner, dass es der siebte der letzte Sinn sei“, grinste Barz zurück.

Der kleine Krark flatterte immer wieder weiter weg, sobald die vier Reisenden einige hundert Meter an ihn herankamen. Doch unmissverständlich wartete er immer wieder auf die Truppe, ehe er weiterflog. Es bestand kein Zweifel, dass er sie irgendwohin führen wollte.

Iril warf hin und wieder einen Blick auf ihre Runenscheiben, hatte aber nicht die Zeit, einen magischen Kompass in Gang zu setzen.

Das war auch nicht nötig.

Geführt vom jungen Krark, drangen die Helden tief in den südlichen Wald ein, marschierten an Bäumen vorbei und Trampelpfaden entlang, bis sie schlussendlich eine bewohnte Lichtung im südlichen Walde am Hange des Fahlen Gebirges erreichten.

Das Lager der Drachenkultisten.\bigskip







Verdiente diese Lichtung überhaupt die Bezeichnung „Lager“? Es handelte es sich vielmehr um eine Gruppe verwahrloster Zelte, die ein glimmendes Feuer umringten. Ein großer Kochtopf hing über den Flammen. Der leckere, einzigartige Duft nach Krallenflechten ging davon aus.

Die Zelte waren geschmückt mit verschiedenen Zeichen und Zeichnungen in dunkler Farbe auf hellem Stoff. Sie alle schienen Feuer, Drachen und ähnlich angenehme Sujets darzustellen, wobei in manchen der Illustrationen größeres Talent als in anderen zu erkennen war. Immer wieder erkannte Iril unter dem Sammelsurium an Symbolen ein bestimmtes, eine Kralle in einem Kreis.

Ein Logo der Drachen-Sippe?

Um Frühaufsteher handelte es sich bei den Kultisten allerdings nicht. Der kleine Krark, der unsere Helden hierhin geführt hatte, schlüpfte durch ein Loch in eines der größeren Zelte am anderen Ende der Lichtung und ward nicht mehr gesehen. Hier und da zeugten Bewegungen in den Zeltwänden und leises Gemurmel von der Anwesenheit von Lebewesen, doch hier draußen war kaum jemand zu sehen. Ein kleines Kind in einem schmuddeligen Kleid versuchte, einen Schluck der Suppe im großen Topf zu erlangen, während eine großgewachsene Frau es lachend zurücktrieb.

„Nein, Reanna, die ist noch nicht für dich! Zunächst dürfen die Beseelten speisen.“

„Immer die Extrawürste für die Beseelten“, knurrte die kleine Reanna mit knurrendem Magen. Dann zuckte sie die Schultern, breitete ihre Arme aus und rannte fröhlich fauchend im Kreis um das Lagerfeuer herum, während die alte Frau im Topf herumrührte.

Nach einem Hinterhalt sah dies kaum aus, dachte Iril erleichtert. Dennoch hielt sie ihren Hammer bereit, während sie nähertrat.

In diesem Augenblick öffnete sich eine Zeltplache nahe des herumrennenden Mädchens. Eine Gestalt trat heraus und räkelte sich gemütlich. Iril brauchte einige Augenblicke, um zu realisieren, dass die Gestalt ein Skral war, so ungewohnt war der Anblick eines müden Skrals ohne Plattenrüstung, auf dessen Kinn eine schaumige weiße Masse blubberte. Nichtsdestotrotz gellten bei diesem Anblick Warnsignale in ihrem Hinterkopf auf. Insbesondere, als der Skral ein langes Messer von seinem Lendenschurz löste und in die Höhe hob. Sein stachelbewehrter Schwanz peitschte über den Waldboden, während er das kleine Mädchen beäugte.

„Weg von ihm!“, schrie Iril dem Mädchen zu und rannte beschützend nach vorne. Ijsdur und Aćh, welche möglicherweise noch nie zuvor einen echten Skral gesehen hatten, guckten einander verdattert an. Verdattert guckte auch der Skral, als Iril zwischen ihn und das kleine Kind trat und drohend ihren Runenhammer hob. Verächtlich wischte er den weißen Schaum von seinem Kinn und ließ darunter einen Dreitagebart zum Vorschein kommen.

Iril erkannte zu spät am fleckigen Spiegel neben dem Zelt, dass der Skral wohl eher seine Gesichtsbehaarung hatte stutzen wollen, statt das kleine Mädchen abzustechen. Beschämt schlich sie zurück zu den anderen Helden, während der Skral sein Messer wegsteckte und argwöhnisch die Helden betrachtete.

Einen wütenden Blick in ihre Richtung werfend, beugte der bärtige Skral sich zum Kind herunter und sprach verblüffend sanft: „Geh, Rrreanna, rrrenn zu deinem Vaterrr. Hierrr wirrrds vielleicht gleichchch gefährrrlichchch.“

Irils Übersetzungsrunen leuchteten nicht auf. Das war die andorische Sprache. Ungewöhnlich für Skrale, sie zu sprechen. Andererseits hatte sich noch nicht mit sonderlich vielen Skralen konversiert.

Die kleine Reanna tat wie gebeten und verschwand in einer Jurte zu ihrer Linken. Kurz öffnete sich die Plache erneut. Ein Mann warf den vier Ankömmlingen einen finsteren Blick zu. Er trug einen langen, schwarzen Mantel und hatte eine Schnittwunde an der Wange, die leicht blutete. Iril war er nicht geheuer. Irgendwie umgab ihn eine dunkle Aura. Dann zog er sich auch schon wieder in die Jurte zurück.

Iril ließ ihre Hand betont ruhig auf dem Runenhammer von Golja ruhen, während der bärtige Skral auf sie zuschritt. Aćh und Barz hatten ihre jeweiligen Waffen zum Schnellziehen bereit. Ijsdur hielt seine eigenen Arme locker auf der Seite.

Der Skral betrachtete die einschüchternd dreinzusehen versuchenden Vier und legte seinen Kopf schief: „Sprrrechchcht schon, was haben wirrr fürrr ein Prrroblem hierrr?“

Iril berief sich auf die diplomatischen Künste, die ihr in Silberland eingetrichtert worden waren.

„Seid gegrüßt. Ich bin Iril von Silberland und werde begleitet von Ijsdur und Aćh aus dem fremden Tulgor sowie von Barz vom Stamme der Iquar. Seid Ihr die Drachenkultisten, die im vergangenen Mond mit dem Prinzen um Taroks Leichnam stritten?“

Der Skral ließ sich einige Zeit, ehe er gemütlich antwortete: „Grrra. Die sind wirrr. Die wenigen, die nochchch übrrrig sind. Viele zogen wiederrr nachchchhause. Habt ihrrr unserrren Zustand gesehen? Ihrrr müsst euch keine Sorrrgen machen, dass wirrr euchchch angrrreifen würrrden. Koh.“

„Wir fürchten uns nicht vor einem Angriff. Wir sind schon hier auf Geheiß des neuen Regenten Thorald. Weil dessen Drachenknochenfragmente gestohlen wurden.“

Der Skral schnaubte und langte nach einem Messer an seinem Gürtel. Hastig redete Iril weiter: „Wir glauben aber fest daran, dass es eine diplomatische Lösung geben kann. Könnten wir mit eurem Anführer sprechen?“

Der Skral ließ das Schwert in seiner schartigen Scheide stecken und murmelte: „Grrra. Ich hole Sagrrramak. Rrrührrrt euchchch nichchcht.“

Ohne weitere Worte schlurfte er davon.

Iril tausche unsichere Blicke mit Ijsdur aus. Aćh redete dem aufgeregt flackernden Turr beruhigende Worte zu. Nur Barz hatte sich aus irgendeinem Grund entspannt und sogar eine erleichterte Miene aufgesetzt.

Im Schatten zwischen zwei Zelten erkannte Iril zwei weiße Augen, die sie wachsam beobachteten. Sie kniff ihre eigenen Augen zusammen und versuchte, die Gestalt besser auszumachen. Sie glaubte, das flackernde Lagerfeuer auf einer metallischen Fläche rund um die Augen reflektiert zu sehen. Eine eiserne, gezackte Maske. Der Schwarze Herold. Stumm lugte er hinter einem Zelt hervor und beobachtete das Geschehen, ohne sich nur im Geringsten zu rühren.

„Das da ist der Schwarze Herold“, flüsterte Ijsdur Aćh hilfreich zu. Aćh nickte.

Weitere Zelte öffneten sich und interessierte Augen starrten die vier morgendlichen Störenfriede an. Auf die Lichtung traute sich jedoch so gut wie niemand. Sogar die alte Frau hatte sich vom Kochtopf zurückgezogen.

„Ich glaube, ich habe unseren Dieb von der Rietburg gefunden“, flüsterte Ijsdur Iril zu. Seine Augen hatten keine Pupillen, aber anhand seiner Kopfbewegung konnte Iril dennoch erkennen, wen er meinte, ohne dass er direkt auf den Verdächtigen zeigen musste.

Aus einer Jurte, an deren Spitze eine beige Fahne mit dem üblichen Symbol der Kultisten flatterte – der umkreisten Kralle – war ein Wesen getreten, welches Iril so noch nie von nahe gesehen hatte. Es schien, als hätte jemand einen vierfach gehörnten Ziegenkopf mit Fangzähnen auf einen kleinen Menschenkörper gesetzt. Weißes Fell bedeckte den Körper des Ziegenwesens. Milchige blaue Augen sondierten die Umgebung. In der Hand führte es eine kunstvolle Hellebarde.

„Einen Dieb?!“, schnarrte das Ziegenwesen, „Einen Dieb mit guten Ohren hat er gefunden! Und dieses ‚Diebes‘ spitze Hörner wird er bald kennenlernen, wenn er weiterhin so freche Anschuldigungen von sich gibt.“

„Verzeiht, o edler Tarus“, entschuldigte sich Ijsdur, „Ich dachte ...“

„Und dann schimpft er dich auch noch Tarus“, lachte das Ziegenwesen. „Ein blinder Träumer ist er! Oh, Fir, er erkennt dich nicht. Du bist doch ein Zwerg. Nichts weiter als ein vermaledeiter Zwerg aus Cavern.“

„Entschuldigt. Ich meinte bloß, Zwerge aus Cavern sähen ...“

„Und jetzt hat er noch die Unverschämtheit, zu meinen, er wüsste, was einen Zwerg ausmacht und was nicht!“, verwarf das Ziegenwesen, Fir, seine Hände, „Verirrst dich einmal zu tief in der Feenwelt, verlierst deinen alten Namen und deine Axt, und schon meint er, du wärst Teil dieser grasfressenden Stinker vom Norden. Was wird nur aus dieser Welt?“

Ijsdur schien beinahe amüsiert, als er sich ein drittes Mal entschuldigte. Fir schnaubte bloß und fuhr sich durch die Frisur.

Iril schaltete sich ein: „Es gibt ja durchaus dieses eine Märchen, laut welchem der erste Tarus von einer Fee ...“

„Und sie, naive Schildzwergin, die sie wohl ist, erzählt dir von Märchen!“, gluckste Fir. „Obwohl es Märchen gibt, die sagen, dass die Sonne jeden Tag im Meer versinkt, um sich abzukühlen. Wenn das die Quellen der Wahrheit der heutigen Jugend sind, so mögen uns die Drachen gnädig sein.“

Iril setzte eine freundliche Miene auf und überließ das diebische Ziegenwesen sich selbst. Es war nicht die Führungsperson dieses Kults.

Stampfende Schritte kündigten die Rückkehr des bärtigen Skrals an. Wobei nicht der Skral selbst laut stampfte – dessen nackten Füße lösten kaum ein Rascheln auf dem belaubten Boden aus – sondern seine Begleitung.

Sagramak die Schamanin. Iril hatte ihre geschwungene Rüstung schon aus der Ferne glitzern gesehen, als sie mit Thorald um Taroks Leichnam gestritten hatte, und ihr Gesicht von nahe, als sie durch ihren Krark einen Mord begangen hatte.

Von nahe sah man nun die vielen Scharten und Flecken, die die Rüstung aufwies. Der aus der Ferne so imposant aussehende Drachenhelm waren bloß zwei unförmige, entfernt an Flügel erinnerte Metallteile, die jemand auf einen Helm geschmiedet hatte.

Auffällig waren zwei Halsketten aus jeweils einigen wenigen unförmigen Knochenfragmenten, die an Sagramaks Hals baumelten. Die einen Knochensplitter waren grau und rissig, wohl Jahrhunderte alt, weitergegeben von Kultist zu Kultist. Die anderen strahlten hingegen weiß und waren zweifelsohne frisch. Taroks letzte Knochen.

Sagramak ließ ihren Blick über die vier schweifen und blieb insbesondere an Turr und am wie üblich von einem Wirbel aus Schneeflocken umgebenen Ijsdur hängen.

„Ich bin die Schamanin und Führerin dieser Diener der Drachen. Ich bin beseelt vom Drachen Sagrak. Es steht mir zu, seine letzten Überreste zu tragen“, sprach Sagramak stolz und hob die uralte Knochenkette hoch. „Ebenso wie die Knochenfragmente des Gewaltigen.“

„Ihr leugnet es also nicht?“, rief Iril, „Dies sind Taroks letzte Knochen! Ihr habt Unschuldige angegriffen, ja, gar umgebracht, um sie zu stehlen!“

„Das hätte nie so kommen sollen“, knurrte Sagramak, „Doch die Schuld dafür liegt nicht bei mir.“

„Bist du für diese Sippe verantwortlich oder nicht? Unabhängig davon, wie ihr sie erlangt habt, fordern wir die Knochen zurück.“

Sagramak blickte finster auf einen Dolch an ihrem Gürtel herunter und wie von Zauberhand geführt erhob sich der Dolch in die Luft. Drohend schwebte er auf Iril zu.

„Wollt ihr nicht lieber fliehen, solange ihr noch könnt?“, zischte sie finster. „Ihr habt keine Ahnung, mit wem ihr euch eingelassen habt und welche Mächte uns zur Verfügung stehen.“ Das Flackern des Lagerfeuers ließ ihre Züge drächisch und düster wirken. Ihre Zunge züngelte kurz zwischen trockenen Lippen hervor.

„Wollt ihr nicht lieber eine friedliche Lösung finden, ehe wir mit der gesamten Rietgarde angerückt kommen?“, warf Iril zurück.

„Hat die Rietgarde nichts Besseres zu tun? Dass sie die beiden Brücken besetzen, ist der einzige Grund, dass wir überhaupt noch auf dieser Seite der Narne kampieren. Ein einzelner Krark könnte die Knochen außerhalb unser beider Reichweite bringen, wenn ich es nur so wünsche. Wir halten die Astze in der Hand.“

Iril, die noch nie zuvor ein Astz-Kartenspiel gespielt hatte, verstand die Anspielung nicht. Barz hingegen blickte wissend.

Der telekinetisch geführte Dolch schwebte weiterhin langsam auf Iril zu. Iril, Ijsdur und Aćh starrten den fliegenden Dolch alle sorgenvoll an und traten einige Schritte zurück. Nicht so Barz. Dieser trat stolz nach vorne und proklamierte: „Dieser Trick ist durchaus einschüchternd. Bis man bemerkt, dass selbst die besten Telekinesen kaum mehr Kraft auf ein Objekt ausüben können als ein Kleinkind. Jemanden ernsthaft verletzen kann man damit kaum. Lass die Spielereien sein, Sagramak. Ich fürchtete schon, dass es hier zu einem unlösbaren Zwist kommen könnte. Doch nun, wo ich dich hier sehe, vergeht diese Furcht wie Schnee und Eis im Drachenfeuer.“

Sagramak stutzte und suchte nach Worten. Genauer: Nach einem Namen. „Öhm ... öhm ... Ja, dich habe ich schon mal gesehen. Dieser reisende Nomade der Iquar? Bart? Baz?“

„Barz! Mein Name ist Barz.“, rief Barz etwas weniger fröhlich.

Ungeachtet dessen fiel Sagramaks zuvor feindliche Haltung völlig von ihr ab. „Verzeih mir, als Schamanin trifft man so viele Leute. Ja, ja, jetzt erinnere ich mich an dich. Wo ist denn dein fescher Begleiter abgeblieben? Und deine große Echse? Was tust du denn hier?“

„Dasselbe könnte ich dich fragen.“

„Dann ist es wohl eine lange Geschichte“, lachte Sagramak, „Sag, haben wir dir nicht einige Krarks geschickt in den letzten Jahren? Warum kam keine Antwort? Wo verstecktest du dich?“

„Ich steckte in einem fremden Land fest. Das ist auch der Grund, warum es sich gerade so seltsam anfühlt, in unserer Sprache zu sprechen. Ich bin völlig außer Übung. Die Geschichte meiner weiten Reisen erzähle ich bei Gelegenheit gerne. Ein andermal. Ich befürchte, es gibt eine dringlichere Sache. Iril hier will mit dir über die Drachenknochen sprechen, die ihr gestohlen haben sollt?“

Barz wies auf Iril, welche sich angespannt räusperte. Wenn Barz die Schamanin kannte, und Aćh Barz kannte, und Ijsdur sich ohnehin nicht um solche Angelegenheiten kümmerte ... war sie auf einmal die Einzige hier, die sich noch darum kümmerte, dass die Drachenkultisten aufgehalten wurden?

Was wusste sie eigentlich über Barz? Konnte er ein Drachenkultist sein? Aus Osten kam er schon mal. Ehe ihre Befürchtungen sich festigen konnten, verdrehte Sagramak ihre Augen.

„Ich mag solche Verhandlungen überhaupt nicht. Sie sind dröge und führen oft nirgendwo hin. Aber dieser Feuervogel sieht aus wie so ein lieber Bursche, da bin ich doch fast gewillt, Euch noch länger zuzuhören ... sofern ich ihn streicheln darf.“

Barz richtete einige Worte an Aćh. Diese bedachte den kleinen Vogel, den eine kaum spürbare Hitzeaura umgab. Er war ein treuer Begleiter. Aber er war noch sehr schwach und brauchte Feuer, um wieder ins Leben zurückzufinden.

Dann fasste sich sie ein Herz und ließ den kleinen Turr zu Sagramak flattern. Er passierte das Lagerfeuer auf seinem Weg und schien einen Teil der Flammen in sich aufzunehmen. Als er auf Sagramaks Schultern landete, bröckelten Aschereste auf ihre gepanzerte Schulter. Sagramak streichelte Turrs Kinn. Die Flammen schienen ihrer nackten Haut nichts auszumachen.

Turr gurrte fröhlich.

Aćhs Hand bewegte sich langsam in Richtung ihres Schwerts, während sie Sagramak scharf im Auge behielt.

Doch in Sagramaks Augen blitzte statt Wut eine nur schlecht maskierte Trauer auf. Iril erinnerte sich zurück an den gewaltigen Krark, durch dessen Augen die Schamanin geguckt hatte, und den Chada vom Himmel geschossen hatte. Sie hatte noch nie Haustiere besessen und konnte sich nicht ganz in Sagramaks Gemüt einfühlen. Aber sie hatte schon einige Geliebte verloren und verspürte Mitleid mit der Kultistin.

„Wollen wir uns vielleicht an einem etwas privateren Ort beraten?“, meinte Iril. „Nicht, dass das ganze Lager involviert wird.“ Sie nickte zu den verschiedensten Augenpaaren, die aus den Jurten zu ihnen spienzelten.

„Na schön“, antwortete Sagramak, „Kommt in mein Zelt. Diskutieren wir.“











\newpage
\section{Kurze Verhandlungen}



Sagramak wies die Helden ins Zeltinnere. Von innen wurde deutlicher, dass der Stoff schon bessere Tage gesehen hatte. Zahlreiche Flicken übersäten die mit Symbolen bemalten Wände.

Der Blickfänger war ein uraltes, aus einem Baumstamm geschnitztes Drachentotem, das in der Zeltmitte thronte. Figürliche Darstellungen hatten offenbar zu den Talenten des längst vergessenen Schnitzers gehört. Stolz blickte die Drachenstatue auf die hereinkommenden Helden herunter. Sagramak neigte andächtig ihren Kopf.

Ehe sie die Zeltplache wieder schließen konnte, brüllte sie nach draußen: „Nehamal, Fir! Kommt her! Ihr seid mein Begleitschutz. Ich will nicht allein mit diesen Störenfrieden sein.“

Die beiden Angesprochenen folgten Sagramak ins Zelt und nahmen links und rechts der Drachenstatue Platz, ihre jeweiligen Waffen bereithaltend.

Nehamal war der Mann mit dem langen Mantel und der düsteren Aura, der mutmaßliche Vater der kleinen Reanna. Fir war das auffällige Ziegenwesen. Während es sich neben der Drachenstatue niederkniete, konnte es nicht unterlassen, einen wehleidigen Kommentar abzugeben: „O weh dir, Fir. Sie ruft an ihre Seite, wie man ein Hündchen bei Schoß ruft. Hat sie völlig vergessen, dass sie ohne dich noch immer auf Steinen kauen und dem Tode entgegenbibbern würde? Es gibt einfach keine Dankbarkeit mehr in ...“

„Klappe, Fir!“, fauchte Sagramak. Fir verstummte. Iril hätte schwören können, dass für einen Augenblick ein breites Grinsen auf dem Ziegenmund aufgeblitzt war.

„Wollt ihr etwas trinken?“, fragte Sagramak betont ungerührt. Sie gestikulierte in Richtung eines großen Fasses, welches am anderen Ende des Raums lag. Barz leckte sich die Lippen. Iril schüttelte ihren Kopf.

„Na dann“, sprach Sagramak langsam, „Sag, Iril, was hast du gegen uns? Dass du dir die Mühe machst, uns hinterherzutraben statt irgendwelchen dringlicheren Bedrohungen? In meiner Erfahrung lassen sich die meisten Handlungen auf zurückliegende Erfahrungen der Handelnden schließen, und die deinen lassen dich nicht positiv erscheinen.“

Entrüstet empörte sich Iril: „Soll das heißen, dass ich meine schlechten Eindrücke zu Drachen ungerechtfertigt an euch ausleben würde? Was sagen deine Handlungen über dich aus? Der Schwarze Herold ist auf eurer Seite, und er setzt sich bekanntlich nur für das Böse ein! Gibt euch das nicht zu denken?“

„Auf der ‚Seite des Bösen‘. Was für ein lächerlicher Gedanke, den die Andori euch da eingeredet haben“, lachte Sagramak, „Der Herold ist nur auf der Seite des Drachen. Wärt ihr alle zuerst auf unsere Gemeinschaft statt auf die Andori getroffen, hätten sie euren Geist nicht korrumpieren können und ihr würdet euch für uns einsetzen.“

„Die anderen drei vielleicht. Ich nicht“, sprach Iril nun, „Ich habe die Geschichten gehört und die Mahnmale gesehen. Die Drachen waren Bestien. Sie sind es nicht wert, abgebetet zu werden.“

„Ah, natürlich, eine Zwergin, welch unabhängige Beurteilerin der Drachenkriege“, verwarf Sagramak ihre Hände, „Ob die Drachen angebetet werden sollen, hängt nicht von ihren Taten ab. Sondern von ihrer Kraft, über uns zu urteilen.“

„Ach ja, stimmt, ihr glaubt, die Drachen würden nach dem Tode auf euch warten und eurer Nachleben bestimmen, oder? Sagt, wie viele Beweise besitzt ihr dafür? Hat auch nur ein verstorbener Geist sich je wieder bei euch gemeldet?“

„Wie viele Beweise besitzt du dafür, dass Mutter Natur oder welche Urmacht auch immer ihr anbetet, nach dem Tode auf euch wartet?“

„Wir lesen die Schriften der Anhänger ...“

„Und wir hören die Stimmen der Drachen! Meinst du, wir bilden uns das alles ein?!“, rief Sagramak. Sie schritt zu einem Tisch und zog ein Objekt daraus hervor. Einen kleinen Stein, glattgeschliffen, um den sich eine elegante metallene Drachenfigur wandte. Das Artefakt glühte rötlich und summte leise.

Eines von Schmiedemeister Hildorfs Drachenrelikten.

Sagramak hielt es Iril demonstrativ ans Ohr: „So höre doch! Lausche und vernehme die Stimme Sagraks, wie ich es tagtäglich tue.“

Iril hielt sich das Relikt widerstrebend ans Ohr. Sie vernahm nur das leise Rauschen des Bluts in ihren Ohren.

„Ich höre nichts.“

„Die Drachen bleiben doch meistens stumm in der Präsenz Ungläubiger“, mischte sich Barz nun beschwichtigend ein, „Die Stämme der Barbaren sind sich oft uneinig, was die Geschehnisse nach dem Tode angeht. Was die Götter für uns geplant haben. Ich erinnere daran, dass die Jpaxo die einzigen sind, die glauben, dass die Drachen ihr Nachleben bestimmen werden. Doch die Drachenseelen scheinen auch als einzige zu den Jpaxo zu sprechen. Lassen wir doch alle Diskussionen zum Nachleben sein und wenden uns der relevanten Frage zu: Was mit Taroks letzten Knochen geschehen soll.“

„Wir haben unseren Standpunkt klar gemacht“, sprach Sagramak bestimmt. „Die Kette der Knochenfragmente bleibt hier. Wenn ihr für uns ein gutes Wort beim Prinzen einlegt, damit er die Brücken in den Osten wieder freigibt, versprechen wir, nicht nachtragend zu sein und die Sache so beruhen zu lassen. Das ist mehr als gerecht.“

„Es wäre mehr als gerecht, wenn diese Knochen nicht so gefährlich wären“, bedachte Iril. „Vor nicht einmal einem Jahr verschaffte sich ein bösartiger Nekromant Zugriff zu den Drachenknochen im Grunde des Grauen Gebirges. Er bündelte ihre magische Macht, verwandelte die alten Wachtürme der Schildzwerge in leuchtende Fackeln der Magie und labte sich daran. Dies waren jahrhunderte alte, zerfallene Drachenknochen, und doch konnte der Nekromant sich derart stärken, dass er den Urtroll mit einem einzigen magischen Blitz zu verscheuchen vermochte. Den Urtroll! Der, der der Legende nach Mutter Natur mit einem einzigen Schlag in den Totenschlaf schlug.“

Sagramak verdrehte ihre Augen. „Drei Knochensplitter aus Taroks Fuß sind kaum vergleichbar mit der schieren Masse an Skeletten der im Unterirdischen Krieg gefallenen Drachen, die diesem Nekromanten zur Verfügung standen.“

„Taroks Knochen sind frischer und potenter. Drachenknochen sehr seltene und auserlesene Zutaten für mächtige Rituale. Und Tarok war der letzte, der mächtigste, der stärkste aller Drachen, als sei die Kraft aller Drachen der Vorzeit in ihn übergegangen. Sie dürfen nicht in die falschen Hände geraten.“

„Und was, wenn wir dir versicherten, dass es unter uns überhaupt keine Nekromanten gibt?“

Iril schüttelte ihren Kopf: „Da kann ich mich nicht auf dein Wort verlassen, insbesondere, weil du es nicht in die Zukunft geben kannst. Dies ist keine fantastische Märchenwelt, in der alle stets friedlich und fröhlich zusammenleben. Dies ist echt. Manche Leute wollen anderen schaden. Wir wollen verhindern, dass sie die Möglichkeiten haben.“

„Und bei deinem Trunkenbold von einem Prinzen wären die Knochen sicher?“

„Sicherer als hier.“

„Doppelmoralische Heuchlerin! Engstirniger Sturkopf!“

„Und du, Sagramak, bist eine ...“

„Nur keinen Ärger“, sprach Barz beschwörend, seine Hände beruhigend hebend, „Sind wir nicht alle Kinder der drei Brüder? Wir müssen uns nicht streiten. Wir können bestimmt eine Lösung finden. Iril, es soll euer Schaden nicht sein, wenn Sagramak die Knochen behält. Dafür könnte Volk der Schildzwerge, das unter Tarok leiden musste, eine ... gerechtfertigte Belohnung in Form von Gold und anderem Schmuck erhalten?“

Barz setzte sein breitestes Grinsen auf, absolut selbstsicher in seinem Glauben, die Lösung für dieses Problem gefunden zu haben. Sowohl die völlig verarmte Sagramak als auch Iril, welche weder eine Schildzwergin noch goldgierig war, schüttelten den Kopf. Iril flüsterte frustriert: „Ich werde einfach so tun, als hätte ich das nicht gehört. Ich bin vielleicht neugierig, oder wissbegierig, Barz. Nicht goldgierig! Nicht alle Zwerge sind gleich.“

Ijsdur nickte besserwisserisch hinter ihr.

„Genau. Neuerdings weiß ich, dass Menschen und Zwerge praktisch gleich sind.“

„Das meinte ich auch nicht.“

Barz lächelte entschuldigend und murmelte: „Naja, es lässt sich zumindest nicht leugnen, dass wir bislang keinerlei Probleme mit den Drachenkultisten hatten, trotz der alten Knochenketten in ihrem Besitz. Wie macht Tarok da einen Unterschied?“

„Mit den uralten versteinerten Knochen von Sagrak oder Nehal oder wie auch immer die alle heißen, könnte ein Magier kaum etwas anfangen. Falls die Knochen überhaupt echt sind. Doch Taroks Knochen sind frische, potente Mittel und gefährlich. Was, wenn sich plötzlich ein Skelettdrache aus dem Hort dieser Schamanen erhöbe? Was würdest du sagen, wenn plötzlich ein Dunkler Magier Taroks Knochen wiederbelebte, kaum hätten sich die Bewohner Andors vom Wüten des letzten Drachen erholt?“

Barz blickte plötzlich wieder besorgter drein. Er hatte noch nicht vergessen, was für eine Verwüstung Tarok im Land der drei Brüder angerichtet hatte.

Iril wandte sich wieder Sagramak zu: „Ich bitte Euch, lasst die Sache sein. Überlasst uns Taroks Knochen und wir besorgen euch Begleitschutz in die nun freie Knochengrube im Grauen Gebirge, wo ihr euch mit einer Vielfalt an Knochenresten für eure religiösen Gebräuche eindecken könnt.“

„Du weißt genau so gut wie ich, dass die Knochen aus der Knochengrube alt und größtenteils versteinert sind, und dass keiner davon von Tarok stammt. Die Knochengrube haben wir bereits zu Taroks Lebzeiten ausgeschöpft. Wir bedürfen seiner Überreste. Seiner Knochensplitter.“

„Und was tut ihr damit?“

„Nicht mein Spezialgebiet. Nehamal kennt sich besser damit aus. Ist schließlich auch vom erfindungsreichsten Drache beseelt. Nehamal?“

Der angesprochene Nehamal erhob sich von seinem knienden Platz neben der Drachenstatue und zog etwas unter seinem langen Umhang hervor. Eine Kette mit einigen uralten Knochensplittern, die um seinen Hals hing, an deren Spitze ein edles Drachenrelikt rötlich schimmerte.

Er schnarrte: „Ich bin Nehamal. Ich trage diesen Namen seit meiner Erleuchtung, und seit dann führe ich die Seele des Drachen Nehal in einem Blutstein-Relikt mit mir und seine Knochen um meinen Hals.“

„Nehal?“, lachte Iril auf, „Der legendäre Drache, der den Zwergen die Drachenfrucht schenkte? Jung, temperamentvoll und stark, und vor allen Dingen gewillt, etwas Neues zu erschaffen? Der Drache mit dessen Assistenz Kreatok seine vier mächtigen Schilde schmiedete?“

Nehamal rollte mit den Augen. „Nehal war ein bisschen mehr als der Assistent Kreatoks. Manch böse Zunge mag behaupten, Kreatok sei vielmehr der Assistent des Drachen gewesen, weil dieser nur genügend kleine Hände benötigte, um seine filigranen Ideen in Tat umzusetzen.“

„Zeige mir einen Drachen, der auch nur ein Schmiedestück schuf, ganz unabhängig der Größe, und ich zeige dir einen Lügner, der sich mit fremden Federn schmückt.“

„Jetzt kriegen wir uns alle mal wieder ein, in Ordnung?“, rief Barz, „Warum kamst du auf diesen Nehal zu sprechen, Iril?“

„Weil ich die alte Geschichte des letzten der vier Schilde kenne, auch man nur hinter vorgehaltener Hand davon munkelt. Nehal wurde in schwarz-silbrigen Flammen des Dunkelschilds verzehrt. Von ihm sind keine Knochen übrig. Was auch immer Ihr hier um den Hals tragt, von Nehal stammt es nicht.“

Nehamal lachte nur kopfschüttelnd. „Wie ihr die Geschehnisse verdrehen müsst, um unseren Glauben leugnen zu können, um uns als böse auszumachen, ist unter aller Würde. Bedenkt unsere Gaben. Übermenschliche Stärke, Schutz vor Feuer, Sagramak ist gar Telekinesin. Die kommen nicht von nirgendwo.“

„Ich leugne nicht eure Gaben, oder deinen Glauben, und ich halte dich nicht für böse. Doch für getäuscht. Ich bezweifle, dass die toten Drachen noch Gedanken oder Gaben von sich geben. Bedenke, dass Tarok Euren eigenen Worten zufolge der letzte Drache war. Wie kann es sein, dass diese Drachenseelen noch mit euch in Kontakt sind, wenn es gar keine lebenden Drachen mehr gibt, um den Drachenseelenhort Krahal am Leben zu erhalten?“

„Die Seelen unserer Beseelten ruhen nicht mehr in Krahal, sondern hier, in unseren Relikten. Und Krahal erhält sich selbst! Es braucht keine lebenden Drachen, um stabil zu bleiben. Im Gegenteil, die Drachen nutzen die Kraft Krahals, um sich selbst zu stärken! Wie Sagramak mit den Naturgeistern spricht, tu ich es mit den Geistern der Drachen, und sie sind seit Taroks Fall nicht leiser geworden.“

Nehamal schloss seine Augen und lehnte seinen Kopf zur Seite, als lausche er andächtig.

Iril nutzte die Gelegenheit, um zu fragen: „Was wollt ihr denn nun mit Taroks Knochen tun?“

„Das geht euch nichts an.“

„Tut es wohl, wenn wir darüber urteilen sollen, wer diese Knochen behalten soll.“

„Wir wollen nicht, dass ihr über uns urteilt.“

„Das macht unser Urteil nicht besser.“

Nehamal stockte einen Augenblick. Dann sagte er: „Wir werden die Knochen ehren und wahren. Ihnen ein würdigeres Begräbnis verschaffen, als sie die Narne herunterzuspülen. In Krahd mag man Verstorbene in einen Lavafluss werfen, um sie von einem Dasein als Skelettkrieger zu bewahren, doch in zivilisierteren Kreisen gehört sich das nicht. In den Bergen bestatten wir unsere Toten in der Erde, auf dass aus ihnen die Erde genährt werden kann. Und dies wollen wir auch Tarok tun. So wenig von Tarok, wie wir kriegen können.“

Iril überhörte die Spitze in seiner Aussage geflissentlich.

Überraschenderweise meldete sich Ijsdurs kalte Stimme zu Wort: „Die Übersetzungsrunen in meiner Brust können nicht nur die Bedeutung gesprochener Worte übertragen. Sie geben mir manchmal auch ein Gefühl der Intention des Sprechers dahinter. Oft nur klein, fein, beinahe vergesslich. Aber Lügen kann ich mit diesen Runen von Meilen her riechen. Und ich rieche Lügen in deinen Worten. Was wollt ihr wirklich mit den Knochenfragmenten?“

Iril blickte Ijsdur überrascht an.

Nehamal starrte ihn ebenso überrascht an. Dann knurrte er: „Na gut. Seien wir offen, auch wenn es euch nicht gefallen wird. Wir werden die Knochensplitter nutzen, um Taroks Seele an ein Relikt zu binden. Wir werden einen Träger ernennen, jemanden, der vielleicht den Namen Taromak annimmt. Was danach geschieht, hängt von Taroks Willen ab. Er wird uns mit seinen Gaben leiten und wir werden seinem Willen folgen, wie wir alle dem Willen der Götter folgen sollten.“

„Und was, wenn Tarok euch den Angriff auf unschuldige Andori befiehlt? Was, wenn er euch einen Nekromanten aufsuchen lässt, um mithilfe dessen Macht den Tod auf Erden zu entfesseln?“

„Ängstigt dich das? Glaubt ihr nicht daran, dass Taroks Geist nun im ewigen Glück weilt? Jemand Glückliches wird keine bösartigen Gedanken mehr hegen.“

„Das glauben die Andori, nicht ich. Wird dieser Taromak mit außergewöhnlichen Kräften gesegnet sein?“

Nehamal hauchte andächtig aus: „Der größte und stärkste von uns allen.“

Iril nickte: „Damit steht meine Entscheidung fest. Rückt die Drachenknochen heraus! Im Namen des Königs von Andor!“

Sagramak mischte sich wieder ein. „Andor hat aktuell keinen König. Und selbst wenn du für ihn sprächest, wissen wir beide, dass dies nicht gerecht ist. Ein feiner Schnösel in seiner hohen Festung soll nicht bestimmen können, was wir, nicht einmal seine Untergebenen, mit Artefakten anfangen dürfen.“

„Ich bin nicht seinetwegen hier. Rückt die Drachenknochen heraus. Meinetwegen im Namen von Iril von Silberhall.“

„Und im Namen von Barz vom Stamme der Iquar“, ergänzte Barz mit entschuldigender Miene..

Sagramak zog die Kette mit Taroks Knochenfragmenten ab und hielt sie grimmig vor sich. Dann hielt sie inne. Ein Grinsen schlich sich auf Sagramaks Gesicht, als wäre ihr eben erst etwas eingefallen.

„Barz, was tust du so, als stündest du hinter dieser Zwergin? Als wärst du nicht auch ein Anhänger der Drachen? Warst es nicht du, der unsere Sippe über Krarks – erbarmungslos vom Himmel geschleuderte Krarks, übrigens – um Sternkraut anbettelte? Ich weiß, was für Rituale man mit Sternkraut vollziehen kann. Wir sahen alle, wie Tarok der Gewaltige sich nur Tage später von seinem Schlafplatz erhob und die Steppe in Brand setzte. Du bist einer von uns, Barz. Stehe zu den deinen.“

Barz blickte auf einmal ganz unwohl in seiner Haut drein. Sagramak grinste.

„Ist das wahr?“, fragte Iril anschuldigend, „Habt ihr Tarok geweckt? Du sagtest uns, Tarok habe Sabris Mutter ermordet!“

„Tat er auch“, versicherte Barz, „Wir wollten ihn nicht rufen. Es war ein Versehen!“

„Leugne es nicht, Barz“, knurrte Sagramak.

„Was bringt es dir, mich von etwas überzeugen zu wollen, an das ich nicht glaube?!“

„Zwiespalt“, sprach Ijsdurs klirrend klare Stimme. „Sie sät Zwiespalt, weil ihr nichts anderes übrigbleibt.“

Barz ächzte: „Komm, Sagramak, so eklig musst du nicht tun.“

„Deine Echse sollte sich glücklich schätzen, im Feuerstrahl des Gewaltigen umgekommen zu sein“, spuckte Sagramak aus, „Ihre Seele ruht nun sicher in der Sphäre der Drachen. Ob du dorthin kommst, steht noch in Frage. Schließe dich uns an und verteidige unser Recht, Barz, oder die Drachen werden dich bis in alle Ewigen im Drachenfeuer schmoren lassen – dich, und deine Familie, und deine Echse, und ...“

„Advaria meza“, hauchte Barz. Hätten seine Augen vor Wut Funken sprühen können, hätten sie es in diesem Augenblick getan.

Aćh, die sich bislang mehr auf die Wandzeichnungen als auf die ihr unverständlichen Gespräche konzentriert hatte, fuhr herum, als sie diese Worte vernahm.

Der auf Sagramaks Schultern sitzende junge Turr legte seinen Kopf schief, als wäre er nachdenklich. Dann klatschte das kleine Vögelein seine Flügel zusammen und schleuderte einen faustgroßen, glühenden Ball aus Feuer in Sagramaks Wange. Diese brüllte auf und klappte zur Seite.

Ijsdur wich vor der urplötzlich aufgeflammten Hitze zurück und stolperte zu Boden. Barz half ihm auf.

Iril stürzte nach vorne und schnappte sich die Kette mit Taroks Knochenfragmenten aus Sagramaks Hand.

Sagramak schoss wieder auf. Ihre Wange wirkte angeschlagen, doch nicht verbrannt.

„Verräterischer Vogel“, zischte sie, „Vielleicht behalte ich dich, nachdem wir deine Besitzerin erledigt haben.“

Turr gurrte bösartig. Fast schien es, als sei er ein wenig größer und kräftiger als vor einer Minute.

Aćh zog ihr Mondschwert und richtete es auf Sagramak, welche ihre waffenlosen Hände hob und einige Schritte zurückwich.

Nehamal zückte einen unter seinem Mantel verborgenen Degen und kreuzte die Klingen mit Aćh.

Fir stand ächzend auf und lamentierte Schmerzen in seinen Knien. Überraschend geschmeidig schlug das Ziegenwesen mit seiner Hellebarde nach Iril, doch diese duckte sich darunter hinweg.

Es klapperte, das Aćh Nehamals Degen zu Boden dirigierte. Aćh kickte die Waffe weg und nahm die Beine in die Hand. Er rief ihr irgendetwas hinterher darüber, wie ihre Zeit abgelaufen sei und er ihrem lächerlichen Dasein noch ein Ende bereiten würde.

Ijsdur riss ein Loch in die Zeltwand. Die vier stürzten ins Freie, nur weg von hier, solange die restlichen Kultisten noch nicht mitgekriegt hatten, was hier abging.

Iril hielt Taroks Knochenkette fest umklammert.

Der Wald lag ruhig da.

Zu ruhig.

Augen blitzten im Gehölz auf.

Der Schwarze Herold stellte sich ihnen in den Weg. Drohend hob er sein Schwert. Sein Umhang flatterte wild. Iril wurde nicht langsamer, sondern schleuderte ihren Hammer. Der Herold wurde mit einem eisernen Klang zu Boden geschlagen. Im Vorbeigehen tippte Iril auf ein Runentattoo auf ihrer Schulter und streckte ihre Hand aus. Der Hammer glühte auf ebenso wie die Runen auf Irils Arm, machte eine Kurve in der Luft und flog in ihre ausgestreckte Hand zurück.

„Schöner Trick!“, komplimentierte Ijsdur.

Die vier rannten einen Waldhang hinauf, auf in Richtung Norden, tiefer in den südlichen Wald hinein.

Im Rennen stampfte Ijsdur immer stärker auf. Die Eisspur, die er hinter sich herzog, wurde breiter und dicker. Dann zuckte ein Blitz vom Himmel herab und traf den Boden vor ihnen. Eine weite Eisschicht breitete sich in Windeseile über den Hang aus.

Faszinierenderweise war das Eis rutschig, doch rutschten Iril und ihre Begleiter nicht den Berghang hinab, sondern hinauf. Iril musste Acht geben, nicht zu stolpern.

Im Zurückblicken erkannte sie, wie die Drachenkultisten unter ihr die Eisfläche zu betreten versuchten. Für sie schien das glatte Eis sich normal zu verhalten. Sie kamen nicht weit und rutschten danach aus.

Aus der Ferne krächzte Fir: „O weh, welch finstere Tricks der Natur sich nun gegen dich wenden. Nicht einmal der Grund vermag uns zu halten. Sieh, da rutscht die edle Sagramak auch schon an dir vorbei. Wäre sie vorhin netter zu dir gewesen, hättest du sie vielleicht vor dem Fallen bewahrt.“

Seine Stimme wurde leiser und verklang, als Ijsdur und seine drei Begleiter weiter den Hang hinaufschlidderten.

Barz lächelte: „Schön, dass wir offenbar schon zu deinen Freunden zählen.“

„Ich mache das nicht“, antwortete Ijsdur, „Nicht absichtlich. Faszinierend.“

„Wir sind sicher“, schnaufte Iril, „Die Drachenkultisten sind keine großen Kämpfer. Sie wagten nicht den offenen Kampf gegen Thoralds Rietgarde, sondern wählten den heimlichen Weg. Und wir haben die Knochen. Alles ist gut.“\bigskip







Am Horizont ragten die Türme der Rietburg stolz in den Himmel hinauf.

„Nicht mehr lange, dann könnt auch ihr von der Gastfreundschaft der Andori profitieren“, versprach Iril den beiden neu Hinzugestoßenen.

Aćh gähnte. „Ich vermisse ein gutes Bett.“

„Der Prinz wird dir bestimmt mit Freuden eines überlassen, wenn wir ihm die Knochensplitter zurückbringen.“

„Sollten wir sie nicht vernichten?“, warf Barz nun ein. Er zückte ein kleines gelbes Pulversäcklein. „Nichts einfacher als das. Ich kann die Knochen im Nu verschwinden lassen. Oder zumindest verwandeln“

„Lieber nicht“, sprach Iril, „Denkt an alles Gute, was mit diesen Drachenknochen angestellt werden kann.“

Da mischte sich nun auch Ijsdur ein: „Während sie in der Schatzkammer eines Prinzen verrotten? Wenn es dem um das Gute gegangen wäre, das man damit anstellen kann, hätte er kaum den gesamten Rest des Drachen die Narne hinunterspülten lassen.“

„Ich kann mir nicht sicher sein, dass Thoralds Ziele hehrer sind als die der Drachenkultisten“, sprach Iril, „Nur, dass er bestimmt nichts mit solchen Knochen anzufangen weiß. Und seine Berater sind hoffentlich feine Gesellen, die sich entweder gar nicht mit finsterer Knochenmagie auseinandersetzten oder aber die Weisheit besitzen, davon abzulassen. Und dann, eines Tages, wenn die Knochen benötigt werden, werden wir sie aus der Schatzkammer holen und verwenden.“

„Wofür könnten wir solche Knochen je nutzen wollen?“

„Ich habe in den Tagen seit Taroks Tod von der Hohen Gelehrten der Rietburg so einiges zu ihnen erfahren. Auch ich trauere nun der die Narne heruntergespülten Magie hinterher. Solche Knochen können als Komponenten gewisser uralter Rituale verwendet werden, welche selbst gewaltige Flüche zu brechen vermögen.“

„Denkst du etwa an Narkon?“, meinte Ijsdur mit hochgezogener Augenbraue.

„Nein“, sprach Iril bestimmt, „Wir mögen uns nicht ganz sicher sein, welches Übel Seekönig Varatan einst auf Narkon bannte, doch halte ich es definitiv für unklug, Varatans Fluch leichtfertig zu brechen. Doch wer weiß, welche finsteren Gestalten in der Zukunft Flüche auf Unschuldige schleudern wollen könnten? Und dann können wir die Drachenknochen aus der Schatzkammer des Königs zurückholen und diese Flüche brechen.“\bigskip







Am Tor der Rietburg – neben einem fröhlich orange vor sich hin flackernden Ewigen Feuer – wurden die vier vom jungen Peta empfangen, der sie über den neusten Klatsch und Tratsch informierte. Auf dem Burghof herrschte geschäftiges Treiben. Mägde liefen mit großen Körben unter dem Arm an ihm vorbei und einige Kinder rannten hinter einem zotteligen Hund her.

Prinz Thorald eilte hastig durch das Rietdorf. Obwohl die Sonne schon hoch am Himmel stand, war er noch in ein Nachtgewand gekleidet. Er trug die Rietgraskrone nicht, hatte sich aber zumindest einen königlichen Umhang übergeworfen.

„Ihr habt die Drachenkultisten überwunden? Ihnen die ungerechterweise angeeigneten ...“

Thorald verstummte, als er Turr erblickte, der stolz auf Aćhs Schulter saß und die Drachenknochenkette in seinen Krallen hielt.

„Welch außergewöhnlich wunderschönes Wesen“, flüsterte Thorald. Er streckte eine Hand nach dem Takuri aus, doch das Vogelwesen sträubte seine orangegoldenen Federn und zischte den Prinzen bedrohlich an. Thorald wich erschrocken einen Schritt zurück.

„Was für eine Bestie!“, wetterte er. „Hochgefährlich! Sag ihm, dass er das lassen soll!“

Aćh schüttelte den Kopf. „Ein Feuertakuri ist es komplexes Wesen mit einem schwer zu bändigenden Gemüt. Turr lässt sich nicht so leicht vorschreiben, wen er mögen soll und wen nicht.“

Ehe der Prinz rot anlaufen konnte, überreichte Iril dem Prinzen mit einer angedeuteten Verbeugung die errungene Knochenfragment-Kette.

Thorald kniete sich zu ihr nieder und nahm sie an sich.

Mit Blick auf Ijsdur fügte er an: „Ich habe mich getäuscht in euch, Eis-Dämon. Ihr alle habt eine große Gefahr abgewendet. Morgen sollen wir die Feiern nachholen, die von dem dreisten Diebstahl unterbrochen wurden. Morgen wollen wir die neu ernannten Helden von Andor feiern!“

Er kratzte sich am Bart. Dann, als wäre ihm auf einmal eine Eingebung gekommen, sprach Thorald: „Und ihr tapferen Vier sollt zu ihnen gehören!“

Er nickte einem Fanfarenspieler zu, der in sein Instrument blies und in Windeseile eine Menge Rietburgbewohner zusammentrommelte.

Für die Zeremonie ließ Thorald sich dann doch noch einen eleganteren Umhang bringen und legte einen Gurt mit einem zeremoniellen Schwert aus der Schmiede des alten Wulfron bringen.

„Kniet nieder“, sprach der Regent.

Die Helden taten wie gebeten.

„Ohne Furcht und ohne Zögern habt ihr die letzten Knochensplitter Taroks zurückgebracht und damit einen Schlussstrich unter das dunkelste Kapitel unserer Geschichte gesetzt. Von heute an und für immer seid ihr Helden von Andor, im Kampf für ein Leben in Freiheit und gleiche Rechte aller, die in diesem Land leben.“

Er schwenkte sein Schwert über den Köpfen der Knienden. Das hatte er irgendwie auch schon motivierter gekonnt. Und bei Ijsdur hielt er tatsächlich größtmöglichen Abstand, während er die zeremonielle Schwertbewegung durchführte. Nichtsdestotrotz war es danach vorbei.

„Das wär’s. Ihr seid nun Helden von Andor. Willkommen im Team. Meldet euch bei ... hmmm ... ich glaube, Eara kümmert sich um die Organisation der Aufgaben? Ah, ihr werdet es schon herausfinden. Viel Glück. Das Königreich braucht mehr wie euch in den Wilden Jahren, die da kommen.“

Schon stolzierte der Prinz wieder davon, etwas davon murmelnd, wo zum Himmel er nun ein sichereres Versteck für die Drachenknochen finden konnte.

Iril schluckte schwer. Das war eine äußerst überrumpelnde Erfahrung gewesen. Keine Helden von Andor waren je so rasch ernannt worden. Vermutlich war der Prinz noch immer etwas neben der Spur.

Grundsätzlich hatte der Titel keine genau definierte Bedeutung. Und dennoch ... Irils Magen verkrampfte sich, als sie sich vorstellte, was nun auf einmal für Verantwortungen auf ihr lasten konnten. Dem Prinzen schien es selbstverständlich zu sein, dass die vier neuernannten Helden sich nun bei Eara melden würden und ihre Hilfe für das Königreich zur Verfügung stellten. Und Iril war auch gerne bereit dazu, hatte sie das ja gerade schon die letzten Tage getan. Doch für wie lange?

Nun, sie konnte sich in nächster Zeit Gedanken darüber machen. Als Helden von Andor würden sie von der Krone mit Nahrung und Logis unterstützt. Sie müsste sich keine Gedanken mehr und ihre schwindenden Goldreserven machen.

Still standen die vier neuernannten Helden im Kreis und starrten dem abziehenden Thorald hinterher. Inzwischen hatte sich ein kleiner Trubel aus Burgbewohnern zusammengefunden, welche sie neugierig begutachteten und flüsterten. Neue Helden, schon wieder?





Iril flüsterte zu den drei anderen: „Und, wollen wir das tun? Helden von Andor sein, zumindest für die nächste Zeit?“

„Aćh und ich sind ein gutes Team“, meinte Barz.

„Ijsdur und ich stellen uns nicht schlecht an“, meinte Iril.

„Wir alle konnten den Lumiwürmern ziemlich einheizen, ohne irgendeine Koordination zu haben“, bemerkte Ijsdur, „Die Frage ist weniger, ob wir ein gutes Team wären. Sondern ob wir eines sein wollen.“

„Ich wäre gerne dabei, es auszuprobieren“, meinte Aćh. „Ich hatte mir unter meinem Besuch hier ohnehin vorgestellt, auszuhelfen und diplomatische Beziehungen aufzubauen. Welch bessere Gelegenheit?“

„Das klingt durchaus gut“, murmelte Barz, „Da ist nur die Sache mit ...“

„BARZ!“, ertönte ein lauter Ruf aus der Menge, „Barz! Habe ich dich endlich gefunden! Wie in aller Welt kommst du hierher?!“

Barz‘ Gesicht hellte sich auf. „Nabib? Bist du es?“, entfuhr es ihm. „Wenn man an die Götter denkt ...“ Ohne weitere Erklärung drehte er sich um und raste in die Menge, zur Quelle der Stimme.

„Diesen Namen kenne ich. Das ist ein Freund von ihm“, erklärte Aćh, „Auf der Suche nach ihm wollte Barz damals nach Andor reisen. Wie es scheint, haben sie sich nun endlich wiedergefunden.“

Die drei folgten Barz in die Menge. Sie fanden den Steppennomaden eng umschlungen in einer Umarmung mit einem hochgewachsenen Krieger, in der die beiden einander leise Dinge zuflüsterten. Einen Augenblick trennten sie sich voneinander und blickten einander ins Gesicht, nur um gleich wieder aufeinanderzustürzen und in einen langen Kuss zu versinken.

„Das sieht doch nach ein bisschen mehr als Freundschaft aus“, grinste Iril.

Aćh blickte überrascht drein und murmelte dann: „Sprache ist kompliziert. Vielleicht kennen die Barbaren nur ein Wort für ... ich meine, der Unterschied zwischen einer tiefen Freundschaft und einer Romanze ist ohnehin minim ...“

„Es wäre angebracht, sie nicht mehr länger anzustarren“, sagte Ijsdur, so laut, dass es auch Barz und Nabib hören mussten.

Die beiden Steppennomaden trennten sich aus ihrer Umarmung und blickten zurück zu den drei Neuankömmlingen. Tränen glitzerten in ihrer beider Augen. Barz schluckte und murmelte dann:

„Natürlich, natürlich. Vorstellungen. Nabib, das hier ist Aćh, die beste Astzspielerin, die ich je getroffen habe. Ich verdanke ihr mein Leben. Und die beiden anderen Helden sind Iril und Ijsdur, die wir gerade erst getroffen haben. Noch könnten sie theoretisch bessere Astzspieler sein.“

„Ijs hat dich schon vor drei Jahren einmal angetroffen“, meinte Ijsdur, „Als Kapitän eines Hängeschiffs.“

Barz machte große Augen und nickte. „Schon wieder vergessen. Es ist herausfordernd, dich gedanklich mit deinem früheren Selbst zu verbinden.“

Aćh nutzte die Gelegenheit, sich an Nabib zu wenden und zu sagen: „Nabib, das hier ist Barz, der beste Astzspieler, den ich in meinem Leben je getroffen habe. Ich verdanke ihm mein Leben. Ich habe tatsächlich schon von dir gehört.“

Barz übersetzte wichtigtuerisch.

Iril nickte Nabib zu. Sie erkannte gleichzeitig seinen Namen und sein Antlitz. Sie hatte ihn bereits im Lazarett an der Rietburg getroffen, wo er aktuell unter Heiler Readem arbeitete.

Barz murmelte: „Das Schicksal war es, das mich auf der Suche nach dir hat von Weg abkommen lassen, Nabib. Und nun erreiche ich nicht einmal mein Ziel, da du es bist, der mich gefunden hat.“

„Da würde ich dem Schicksal nicht böse sein“, lächelte Nabib.

Auf einmal schien Barz fahrig. Er nestelte an seinem Gürtel herum, während er Nabibs Blick auswich und leise flüsterte: „Ich habe unsere Ringkette hoch oben im Gebirge verloren.“

Ebenso betroffen murmelte Nabib: „Und ich habe das reich verzierte Amulett deiner Großmutter verkauft.“

„Weißt du, wie egal mir das gerade ist?“

Während sie einander erneut in die Arme fielen, flüsterte Ijsdur: „Ist es möglich?“

Ijsdur hob seine schneeige Hand ... und steckte sie einfach in seinen schneeigen Bauch hinein! Als er sie wieder hervorzog, war sie zu einer Faust geformt. Er präsentierte Barz ihren Inhalt: Eine elegante Kette. Zwei Ringe hingen daran: Ein geschnitzter aus dunklem Holz und ein gehauener aus hellem Stein.

„Diese Kette habe ich am Felsentor zum ewigen Eis gefunden, als ich es verließ. Gehört sie zufälligerweise ...“

„Ja!“, hauchte Barz. „Bei den Göttern, wie kann dies sein?“ Misstrauisch blickte er die Kette an. Das letzte Mal, als er eine Kette von einem Eis-Dämon angenommen hatte, war es nicht gut gekommen. Dann jedoch fasste er sich ein Herz, schnappte sich die Kette und dankte Ijsdur von ganzem Herzen.

Während er sich wieder Nabib zuwandte, fragte Iril Ijsdur: „Wie viele andere Dinge bewahrst du so in deinem Körper auf?“,

„Nicht viele. So viel findet man da oben im Gebirge nicht.“

„Schon an einen Gürtel gedacht?“

„Ist doch viel unauffälliger, es im Körper mitzuschleppen.“

Iril musste zustimmen.

Da schob sich eine große Echse durch die Menschenmenge. Schnaufend blieb Sabri vor Barz stehen.

Nabib blickte sie verwirrt an.

„Sie ist ... kleiner geworden?“

„Sie ist gestorben“, meinte Barz trocken, „Das ist Sabri, ihre einzige Tochter.“

„Oh, Barz, das tut mir leid.“

„Muss es nicht, es ist schon länger her.“

„Wer ist das?“, fragte Aćh.

Sie blickte zur Schmiede, hinter welcher eine grau gekleidete Gestalt hervorgetreten war und mit energischen Schritten näherkam. Sie trug einen langen Umhang, der bis beinahe an den Boden reichte, und einen dicken braunen Rucksack auf ihren Rücken.

Iril hatte sie schon früher einmal gesehen, wie sie nach Taroks Tod und bei Brandurs Totenfest umhergewuselt war und mit wichtig aussehenden Leuten gesprochen hatte. Nun trat sie näher und räusperte sich wichtigtuerisch:

„Iril von Silberhall?“

„Die bin ich! Und wer seid Ihr?“

„Mein Name ist Sanja. Ich bin meines Zeichens Bewahrerin der Lieder, Sagen und Schriften in den heiligen Archiven des Baums der Lieder.“

Iril lächelte: „Ah, ihr befragt die Anwesenden nach den wichtigen Geschehnissen um Taroks Tode und notiert sie für die Nachwelt? Und nun würdet ihr gerne uns nach unseren Erlebnissen ausfragen?“

Sanja die Bewahrerin nickte wichtigtuerisch: „Eine gewaltige Aufgabe. Wir fühlen uns sehr geehrt, dass der Oberste Priester Melkart ausgerechnet uns beide wählte, um für diesen Zweck den Wachsamen Wald zu verlassen. Bekanntlich tun das nur sehr wichtige Bewahrer zu wichtigen Zeiten.“

„‚Wir‘?“, fragte Iril. Sie brauchte nicht lange, um den Grund zu erkennen: Hinter der Ecke der Schmiede lugte eine weitere in einen langen grauen Umhang gewandter Gestalt nervös hervor.

„Wird dein Begleiter ... oder Begleiterin? ... will deine Begleitung noch zu uns stoßen?“

„Ich glaube, heute fühlt sie sich eher nach einer Sie an. Und ja, sie wird noch zu uns stoßen. Sie muss schließlich alle wichtigen Berichte notieren. Ich stelle nur die Fragen. Meine Handschrift die reinste Krakelei und noch dazu bin ich langsamer. Sie ist nur etwas schüchtern vor Fremden. Jorna, getrau dich doch mal her!“

Jorna, die Bewahrerin, glitt nervös auf die Gruppe zu. Sanja deutete im Nebenbei auf einen grünen Wimpel, der auf Brusthöhe an Jornas grauem Bewahrergewand befestigt war. Durch ihre Übersetzungsrunen konnte Iril den Zeichen darauf ihre Bedeutung zuordnen. „Sieh da. Eine Sie.“

Sanja fuhr fort: „Nun denn, dann könnten wir uns der Befragung widmen, sofern das für Sie in Ordnung ist. Was hattet Ihr soeben mit Prinz Thorald zu tun?“

Während Jorna hastig eine Schreibfeder und eine mit einem Pergament bespannte Steintafel aus ihrer Reisetasche zog, fragte Iril verwirrt: „Wart Ihr nicht anwesend? Habt Ihr das nicht mitgekriegt?“

„Wir sind die Bewahrer vom Baum der Lieder. Wir bewahren die Berichte anderer. Wir versuchen, so wenig wie möglich einzugreifen, und wir stellen möglichst offene Fragen“, gab Sanja zurück.

„Oh, na dann“, murmelte Iril, „Nun, wenn ich es richtig verstanden habe, wurden wir vier soeben zu Helden von Andor benannt für das Zurückbringen der von Drachenkultisten entwendeten Flügelknochenfragmenten des Drachen Tarok.“

Jornas Schreibfeder glitt in Windeseile über die Schriftrolle. Iril vermutete, dass sie irgendein Zeichenkürzelsystem nutzte. Niemand, den sie kannte, konnte beim Niederschreiben mit einer rasch sprechenden Person mithalten.

„Fußknochen“, widersprach Sanja, „Es waren Fußknochen. Flügel hätten eine gelblichere Färbung.“

Iril wusste nicht, was sie darauf antworten sollte. Da stellte Sanja schon die nächste Frage: „Was bedeutet es, ein Held von Andor zu sein?“

Ein bisschen hilflos guckte sich Iril zu den übrigen Helden um. Barz unterhielten sich weiter mit Nabib, Ijsdur starrte planlos in den Himmel und Aćh hätte die Frage nicht einmal verstanden.

„Das weiß ich nicht so wirklich. Ein guter Ruf im Königreich, nehme ich an. Vielleicht auch andere Vorteile? Eine gewisse Verpflichtung darüber, diesem Ruf und Titel treu zu bleiben. Doch bin ich mir nicht sicher, ob wir ihn überhaupt verdient haben. Das geschah alles etwas kurzfristig.“

„Und ihr habt etwa eine eigene Echse?!“, rief Sanja. Iril folgte ihrem ausgestreckten Finger bis zur Steppenechse Sabri, welche soeben von Barz und Nabib hinter den Ohren gekrault wurde und ein wohliges Gurgeln von sich gab.

„Das ist Sabri“, erklärte Iril. „Sie gehört zu Barz. Dem Steppennomaden im langen Mantel mit den vielen Säckchen daran. Er verfügt über uraltes Wissen zu magischen Pulvern. Dieses versetzt ihn in die Lage den Kampf mit anderen Augen zu sehen. Darüber hinaus hält er so manche Überraschung für uns alle bereit. Der mysteriöse Barz versteht es ...“

„Schreibfehler, Jorna. Er heißt Barz, nicht Braz“, zischte Sanja zu Jorna, woraufhin letztere hastig ihre Notizen korrigierte.

Iril fuhr indes fort. „Barz versteht es, verschiedenste Pulver herzustellen, was ihn zu einem der vielseitigsten Helden Andors macht. Dabei gilt es immer abzuwägen, wie viel Pulver er einsetzt, denn das kostet ihn immer auch Kraft. Nur gut, dass das Echsenwesen Sabri weitere Pulver für ihn trägt und ihm stets folgt.“

„Beeindruckend“, rief Jorna, „Hat Eure Heldengruppe denn schon einen Namen gefunden?“

Wieder blickte Iril etwas hilflos zu den restlichen Helden zurück.

„Brauchen wir denn einen? So, ohne Gruppenbesprechung werde ich mir wohl kaum einen solchen aus den Fingern saugen können. Können wir vielleicht ohnehin etwas später mit dieser Befragung fortfahren? Aufregung liegt hinter uns

„Natürlich, natürlich“, sagte Sanja, „Wir wollen Sie nicht weiter aufhalten. Wir werden bei der nächstbesten Gelegenheit wieder auf Sie zurückkommen. Es gibt noch so einiges Faszinierendes über Sie zu erfahren. Das verrät mir mein Bauchgefühl.“

Sanja tippte Jorna auf der Schulter. Die beiden Bewahrerinnen verabschiedeten sich mit einer Verbeugung.\bigskip







Iril zog sich in einen der provisorischen Massenverschläge zurück und versenkte sich in ihre Runen. Ijsdur hatte sich inzwischen in den Augen des Prinzen so sehr verdient gemacht, dass er ebenfalls dort ruhen durfte. Aćh schloss sich ihnen an, auch wenn ihr hin und wieder brennender Vogel allerlei Blicke auf sich zog, sowohl bewundernde als auch besorgte.

Später am Tag bekam Iril mit, wie die Takuri-Hüterin auf ihrer Steinflöte eine wilde Melodie spielte und ihr Takuri in einem Flammenwirbel verschwand.

„Kann er etwa auch unsichtbar werden?“, fragte Iril. Ijsdur übersetzte.

„Ne, nur nach Tulgor zurückspringen. Meiner Familie und Bekannten am Nistbaum signalisieren, dass es uns gut geht. Briefe kann er schlecht mit sich führen, doch sein Auftauchen sollte erste Sorgen beheben. Es gibt so einige, die sich Sorgen um mich machen könnten, sobald Haamuns erste Nachrichten vom Lumiwurmüberfall zu ihnen gelangen. Nelímar. Òkôkó. Nugal. Efroćhin. Yrbstschly.“

Barz trennte sich von der restlichen Gruppe. Er wollte raus ins Rietland. Dorthin, wo laut Nabib einige andere Barbaren ihre Zelte aufgeschlagen hatten.

Auch wenn es sich dabei hauptsächlich um Mitglieder verschiedener Yetohe-Stämme handelte, erkannte Barz den einen oder anderen wieder. Nabib stellte ihm die besten Köche, Schnitzer, und gar die legendäre Jägerin Naldia vor, die der Legende nach schon mal einen Felltroll mit einem einzigen Schuss erledigt hatte.

Vor dem Lagerfeuer erzählten die Barbaren einander Geschichten von vergangenen Zeiten. Und als die Sonne untergangen war und das Sternenmeer am Himmel funkelte, zogen sich Barz und Nabib in eine leerstehende Jurte zurück.

Zum ersten Mal seit langem waren sie wieder allein miteinander.

Und so redeten sie über all jenes, was ihnen widerfahren war. Und was sie in Zukunft sein wollten.\bigskip







Barz‘ Hand schoss an seine wiedererlangte Ringkette. Seine Stimme zitterte: „Und Yafka geht es wirklich gut? Karyz und Zan auch? Und Asbark? Und ...“

„Du darfst noch zehnmal danach fragen und ich werde dich jedes Mal beruhigen. Einiges hat sich geändert in Thakkum in den letzten Jahren. Doch deinen Lieben geht es allen prima, den Umständen entsprechend.“

Ein Kloß der Furcht machte sich in Barz breit „Was soll das heißen?“

Nabib lehnte sich zurück. „Hast du schon von der Ewigen Kälte gehört? Die hat den großen See Ava ganz besonders hart getroffen. Die Nahrungsbeschaffung war hart. Zan wurde vom Eisschlaf erwischt.“

„Meinst du diesen unnatürlich langen Winter? Ja, natürlich habe ich von dem gehört. Tulgor war von der unnatürlichen Kälte ebenfalls betroffen. Wir dachten schon, die Eis-Dämonen des ewigen Eises hätten einen Weg durchs Felsentor gefunden. Erst der Hüter der Zeit konnte uns darüber aufklären, dass der Ursprung dieser Kälte jenseits des Fahlen Gebirges lag und dass die Angelegenheit um den Winterstein hier gelöst werden würde. Die Feuertakuri mochten das Ganze dennoch überhaupt nicht. Einige sind in den Eisschlaf gefallen.“

Barz hatte einige Worte verwendet, die Nabib nicht kannte. Statt nachzufragen, berichtete er leise: „Ich reiste einmal zurück an den Ava. Begleitete einen zwischen dem Osten und dem Rietland umherpendelnden Händler. Es waren beruhigende Monate. Aber ohne dich irgendwie nicht dasselbe. Und auch nicht so aufregend. Hier in Andor ist hingegen der Fluch der Götter los. Gute Güte, stets greifen irgendwelche finsteren Gesellen und Kreaturenarmeen an.“

„Es sind wilde Zeiten.“

Nabib wurde ernst: „Yafka erzählte mir, was mit dir geschah. Wie du dich hierher teleportieren wolltest. Nach Andor. Zu mir. Und wie du in einem Flammenbausch verschwunden seist, und sie seither kein Wort mehr von dir vernommen hätten.“

Er blieb einen Augenblick lang stumm, dann rief er: „Du Dummkopf! Was haben wir immer gesagt? Vorsichtig mit neuen Mitteln umgehen! Testen, dann handeln! Du hättest wer weiß wohin versetzt werden können! Ins eisige Meer des Nordens oder tief in die Lavameere des Südens!“

„Ich wusste, was ich tat! Ich landete in der Heimat des Phoenixes. Nur war diese halt nicht Andor.“

„Leichtsinnig! So leichtsinnig! Warum nur?“

„Ich wollte bei dir sein. Rasch. Du hattest dich so lange nicht gemeldet ...“

„Weil ich im Fiebertraum im Lazarett lag!“

„Da hätte ich erst recht an deiner Seite sein sollen! Ich wünschte mir so oft, du wärst bei mir. Du hättest einen Blick auf diesen fremden Sternenhimmel geworfen und gewusst, wo wir uns befinden. Ich hingegen konnte kaum mit dem Sonnenstand etwas anfangen!“

„Du überschätzt mich, Barz“, grinste Nabib. Seine Mundwinkel zuckten, doch seine Augen blieben wehmütig. Darin erkannte Barz den Blick, den er nun seit über zwei Jahren vermisst hatte.

Barz stand ruckhaft auf. „Nachrichten! Jetzt, wo ich wieder hier bin, jetzt, wo wir wieder die Möglichkeit haben, muss ich eine Nachricht nach Thakkum schreiben. Allen sagen, dass es mir gut geht. Sie fragen, wie es ihnen geht.“

„Klar, Barz. Wir werden das tun, erste Priorität morgen. Ich kenne den königlichen Falkner gut, er wird uns eines seiner besten Tiere überlassen.“

„Auf dass die Krarks es nicht erwischen mögen.“

„Jetzt tu mal nicht so düster, die Lage ist erheblich besser geworden. Apropos Flugviecher: Du wirst ja wohl kaum schon gehört haben, dass Tarok gefallen ist? Dieser elende ...“

„Oh, das habe ich sehr wohl gehört“, sagte Barz stolz, „Seine letzten Knochenfragmente haben wir soeben dem Prinzen ausgehändigt.“

Barz‘ Miene fiel, als er anhängte: „Wir haben diesbezüglich Schamanin Sagramak von den Jpaxo wiedergetroffen. Und uns gleich mit ihr zerstritten. Ich hoffe sehr, dass das kein übles Nachspiel haben wird.“

„Was wollen sie schon tun, sich mit den hiesigen Schamanen um ihren Glauben streiten? Wusstest du, dass die dich hier nur eine Person heiraten lassen? Die Bewahrer meinen, ihre heilige Mutter Natur wolle es so.“

„Mutter Natur?“

„Der Kult am Baum der Lieder hat sich ihr verschrieben. Ich glaube aber, dass dies nur ein anderer Name für den Großen Specht ist. Die Bewahrer meinen, sie segne jeden solchen Bund hier in Andor. Und dass es zu jeder Zeit nur einen solchen Bund pro Person geben könne, auf dass er Bund nicht an Bedeutung verliere.“

„Na, das ist ja limitierend.“

„Lass ihnen Zeit, Traditionen wandeln sich langsam. Vor einigen Generationen sagten die Schamanen der Yetohe auch noch, dass die Götter ausschließlich mit Männern sprächen.“

Barz lächelte. „Der Barbarenkönig und seine Sippen mussten sich bestimmt daran gewöhnen, dass die Leute hier unabhängig ihres Geschlechts gleich viel gelten. Iquar verlangte dies schon vor Urzeiten von uns.“

„Na, die Andori lassen auch erst seit kurzem Frauen für ihr Land kämpfen. Rein männliche Berufe waren eigentlich ein Ding der Sklaventreiber aus dem Süden. Es braucht Zeit, solche Joche der Krahder zu überwinden.“

„Es braucht Zeit und Einsatz. Und es wird uns Zeit und Einsatz kosten, wieder zueinander zu finden. Oh, Barz, ich spürte, dass du am Leben warst, und deine restliche Familie auch, und wir hofften, dass du einfach an einem netten Ort feststecktest. Aber es war schon gruselig, wie kein Laut von dir zu hören war.“

„Ich versuchte, euch mithilfe der Verbindung durchs Meditationspulver Nachrichten zu schicken, aber das ging fehl.“

„Asbark, deine Schamanin, hat ebenfalls versucht, Kontakt aufzunehmen. Irgendwann haben wir aufgegeben. Ich habe hin und wieder deine Stimme gehört, Barz. Es hat mir Kraft gegeben, auch wenn ich deine Worte nicht verstehen konnte.“

„Ja, gezielt Nachrichten zu übermitteln, scheint noch nicht Teil der Pulverkräfte zu sein. Auch wenn es irgendwie möglich sein muss. Ich werde mich dem intensiven Studium des Meditationspulvers widmen. Jetzt, wo ich wieder diesseits des Kuolema-Gebirges ruhe, kann ich mir wieder frische Krarkfedern liefern lassen und ...“

Mit Blick auf Nabibs Blick unterbrach sich Barz und meinte: „... doch nun ist vor allem wichtig, dass unsere Zeit der Trennung hinter uns liegt.“

„Natürlich. Auch wenn ich mir nicht sicher sein konnte, dich je wiederzusehen, betete ich die Götter dafür an. Immerhin wusste ich, dass du noch lebtest und an mich dachtest. Aber die Eindrücke kamen immer seltener mit der Zeit.“

Die leise Anschuldigung blieb nicht unbemerkt.

„Verzeih mir, Nabib. Ich hätte mich häufiger melden sollen. Ich wusste, dass wir uns wiedersehen würden. Es wurde mir prophezeit. Und doch wusste ich nicht, wie es dir ging. Ob ich dir auf die Nerven ging mit meinen Meditationen. Ich gebe zu, ich fürchtete mich gar etwas vor unserem Wiedersehen.“

Nabib gluckste auf. „Alles ist gut. Wir haben einander wieder. Ich hoffe, diese Befürchtungen haben sich nicht bewahrheitet. Hast du nun etwa Geheimnisse vor mir?“

„Niemals. Ich werde dir alles erzählen, was du hören willst über meine Zeit in den fremden Landen. Natürlich ist es anders. Wir sind nicht mehr dieselben, die wir einst waren. Und doch erkenne ich in dir dieselbe einzigartige, liebenswürdige Person, die ich damals ... die mich damals ... du ...“

Angespannte Stille. Das letzte Mal, als sie einander gesehen hatten, waren sie sich uneins gewesen. Barz hatte Nabib ziehen lassen. Nabib hatte Barz zurückgelassen. Solche Spannungen konnten Beziehungen zerstören, das wussten sie beide. Keiner von ihnen wagte, diese letzte Nacht anzusprechen, als der Barbarenkönig mit seinen Sippen in der Pfahlbausiedlung vorstellig geworden war. Keiner wollte das Vergangene Revue passieren lassen.

Nabib kuschelte sich an Barz Schulter. Seine kräftigen Hände massierten, nein, streichelten Barz‘ Rücken. Barz tastete nach der Wange seines Freundes. Ein Glimmen leuchtete in Nabibs Blick auf.

Barz lächelte schief. „Ich habe mich seit Tagen nicht gereinigt. Der Gestank ...“

„Könnte mich kaum weniger kümmern. Irgendetwas Neues, was ich wissen sollte?“

„Ich gab gut Acht, mir nichts einzufangen.“

„Ebenso.“

Barz legte fragend seinen Kopf schief. Nabib ebenso.

Barz nickte enthusiastisch. Nabib ebenso.

Barz schlüpfte aus seinem Mantel. Nabib ebenso.\bigskip







Urplötzlich wich Nabib von Barz zurück und wandte sich ab.

„Was ist? Nabib, was ist?“

Nabib wedelte wegwerfend mit seiner Hand und sprach keuchend: „Nein, es ist nichts. Lass uns ...“

Ein Schluchzer schüttelte ihn, ehe er weitersprechen konnte. Barz hielt angebracht Abstand, während er fragte: „Wie kann ich helfen?“

Nabib schüttelte seinen Kopf und murmelte etwas unverständliches vor sich. „Nein, das ist ... es ist lange her. Ich musste nur daran denken ...“

Barz setzte sich auf den Boden und gestikulierte zu Nabib, es gleich zu tun. Nabib hielt weitere Schluchzer zurück, doch sein starrer Blick zeugte davon, dass nicht alles in Ordnung war.

„Meinst du, es könnte helfen, darüber zu sprechen?“

„Ist bei den meisten Dingen so“, druckste Nabib herum.

„Dann willst du davon erzählen?“

Nabib grummelte etwas. „Ich schäme mich. Ich will nicht ...“

„Du brauchst dich nicht zu schämen. Nicht vor mir.“

Nabib mahlte mit seinen Zähnen und zog sich etwas zurück. Leise grummelte er: „Oh, wie verdammt stolz ich damals war, an der Seite des Yetohe-Stammes nach Andor zu ziehen. Unsere Familien zu verteidigen, indem wir anderen ihr Land raubten. Ich nehme nie wieder mein Schwert in die Hand. Ich habe Unrecht getan hier. Ich habe Bauern und Kinder aus ihren Häusern vertrieben. Ich ... Barz ...“

Nabib schluckte schwer.

„Ich kenne noch nicht einmal ihre Namen. Sie standen plötzlich vor mir, Mistgabeln in den Händen. Sie erwischte mich in der Brust. Ich führte meine Axt und ... ich war bereit, mein Leben für unseren Stamm zu riskieren, und für die Stämme der Brüder, und ich habe Unschuldige umgebracht. Ich werde dieses Unrecht niemals bereinigen können. Und das verlangt niemand von mir. Sie alle behandeln mich wie einen der ihren, diese Andori. Nun, die meisten wissen es auch nicht. Aber ich weiß es. Und die Götter wissen es.“

Barz blieb stumm. Er wusste nicht, was darauf zu sagen war. Sorgsam hob er seine Hand.

„Soll ich dich drücken?“

Nabib hustete trocken auf. Dann nickte er stumm und Barz umschlang ihn.

Leise schluchzte Nabib in seine Schulter hinein: „Lieber gebe ich mein Leben hin, als dass ich wieder in solche Zwiste verwickelt werde. Kämpfen werde ich nie mehr. Ich widme mich der Heilkunde. Ich lag eine Zeit lang flach nach dem Angriff. Monate, gar. Meine Rippen sahen schon bessere Tage. Eine Wunde hatte sich entzündet und warf mich in fiebrige Albträume. Doch Heiler Readem rettete mich. Mich, einen Angreifer in sein Reich! Noch während meiner Genesung begann ich, bei ihm auszuhelfen. Stellt sich heraus, dass die Zeichenkünste, die ich mir als Kartograph angeeignet habe, durchaus sehr geeignet sind für das Skizzieren von Karten des Körpers. Faszinierende Angelegenheit, das.“

Nabib blinzelte seine Tränen beiseite.

„Hast du schon mal gesehen, wie ein Körper von innen aussieht? Unsere Schamanen vollzogen solche Riten ja immer im Geheimen. Hier kann jeder Interessierte die Körperkunde lernen. Was Heiler Readem mir alles gezeigt hat von Innereien ...“

Barz verzog sein Gesicht und drückte ein „Schön, dass dir das gefällt“ heraus.

Nabib grinste schwach, als er Barz‘ angeekelte Miene sah.

„Und, Nabib? Wir das deine Zukunft sein? Du lebst nun hier, in diesem Land?“

„Ich mag es hier.“

„Und Yafka lebt weiter im Lande der drei Brüder?“

„So ist es.“

„Oh, da wird es Entscheidungen zu treffen geben. Wohin ziehen mich die Götter? Wohin wollt ihr mich ziehen sehen? Und wohin will ich ziehen? Es fühlt sich alles noch so unwirklich an. Bist du wirklich hier? Geschieht das hier wirklich? Es geht alles so schnell. Dieser Prinz ernannte mich einfach zu einem Helden von Andor. “

„Gratuliere, mein Held. Diese Ehre erfahren nicht viele. Was hast du getan?“

„Ich? Nur einige Knochensplitter zurückgebracht.“

„Dann stellt sich vielmehr die Frage, was du noch tun wirst.“

„Ist das Zurückkehren zu unserem vorherigen Leben als Steppennomaden noch eine Option?“

„Ach, Barz ... ich bin mir nicht sicher, ob ich das überhaupt will. Ich habe mich hier lange niedergelassen. Vielleicht gefällt mir dieses Leben mehr.“

Barz nickte. „Ich weiß nicht, was ich mir vorgestellt habe. Dass ich dich einfach hier aufgable und wir wieder ins Land der drei Brüder zurückkehren, unsere großen Reisen durch die Steppe weiterführen? Was sind wir? Sind wir immer noch eine Familie?“

„Schwer zu sagen, Barz. Wollen wir noch eine Familie sein? Kannst du mir verzeihen?“

„Oh, Nabib. Ich war nie wütend auf dich, dass du gegangen bist.“

„Du solltest es sein.“

„Bist du wütend auf dich selbst?“

Nabib antwortete nicht. Mehr zu sich selbst als zu Barz murmelte er: „Es wird Aufwand kosten, die lange Zeit voneinander zu überbrücken. Wir werden aneinanderkrachen und zerbrechen und, so die Götter es wollen, uns wieder zusammensetzen und unsere gemeinsame Zeit aufs Neue genießen. Ich bin zuversichtlich, dass wir das hinkriegen. Selbst wenn es wieder so eine Nacht wie im Goldtal geben sollte.“

„Daran denke ich nicht gerne zurück. Ich gebe mein Bestes, dass ich meine Gefühle besser unter Kontrolle behalte. Schlimmstenfalls schnappe ich mir Ijsdurs Kette ...“

„Wäre das wirklich gesund? Gefühle müssen auch gelebt werden. Barz, lass gut sein. Verzeih, ich hätte das Goldtal nicht aufbringen sollen.“

„Ich nannte dich damals einen gehörnten ...“

„Lass es gut sein, wirklich. Ich hatte mich damals auch wirklich gehörnt verhalten. Und wenn wir diese Nacht überstehen konnten, werden wir auch das hier überstehen können.“

„Das gefällt mir.“

„Mir auch.“\bigskip







Einige Zeit später, einige Distanz entfernt.\bigskip



Najuk sah die weiße Gestalt, die sich aus den nebelverhangenen Bergen des Fahlen Gebirges löste. Eine Wolke aus Schnee und Eis wirbelte um sie herum, und es schien, als würde sie darauf schweben. Die Erscheinung trug das Antlitz einer Frau. Sie hatte etwas Faszinierendes an sich.

„Wie viele wollen von euch kommen denn noch?!“, lachte er und machte sich auf, Ijsdurs mutmaßliche Kumpanin freundlich zu begrüßen.

Dann hielt er inne. Seine Beine wollten ihm nicht mehr gehorchen. Sein Blick war auf die fremde Frau fixiert. Najuk konnte sich ihrem Bann kaum entziehen. Sie strahlte eine solche Eiseskälte aus, dass er spürte, wie seine Glieder langsam erstarrten. Bewegungslos sah er die durchscheinend anmutende Gestalt auf sich zukommen. Kalt sprach sie: „Ich bin Siantari. In den tiefen Schluchten des Kuolema scheint niemals die Sonne auf das ewige Eis, und dort ist mein Reich.“

Najuk verstand ihre fremden Worte nicht. Doch sie waren das letzte, das er hörte, bevor seine Sinne ihn verließen.\bigskip







„Wie weit ist es noch?“, fragte Sagramak protestierend. Der Schwarze Herold antwortete nicht, sondern schwebte weiterhin schweigend den Hügel hinauf. Oben an der Hügelkuppe hielt er inne. Die glühenden Augen hinter der gezackten Maske verrieten keine Regung. Mit seinem langen Schwert zeigte er über die verschneite Ebene. Auf die einsame weiße Gestalt, die mit ausgebreiteten Armen an Ort und Stelle schwebte, während ein Wirbelsturm aus Eis und Schnee um sie herumtobte. Vor ihr lag der zusammengesunkene Körper eines Andori.

Die Gestalt zog aus ihrem schneeweißen Gewand eine Kette aus spitzen Eiskristallen hervor. Sie beugte sich herunter, als wollte sie dem starren Andori die Kette anlegen.

„Siantari!“, rief Sagramak laut. „O holde Dämonin des ewigen Eises! Möchtest du deine Macht nicht lieber mit jemandem teilen, der tatsächlich davon profitieren kann? Der vielwissende Herold der Drachen berichtet mir, dass du diese Lande unter einer dicken Schicht ewigen Eises verschwinden lassen willst. Doch sind diese Gefilde gefährlich für Feinde der Krone. Wir könnten uns zusammentun, zumindest eine Zeit lang. Das wäre für uns beide von Vorteil.“

Siantaris eisblaue Augen fixierten die Schamanin. Sagramak spürte, wie sich eine Kälte in ihr ausbreitete. Dann warf Siantari – weiterhin stumm – Sagramak ihre Eiskristallkette zu. Die Schamanin unterdrückte den Instinkt, die Kette mit bloßen Händen zu fangen. Der Herold hatte sie davor gewarnt. Stattdessen tappte Sagramak in sich hinein und nutzte das magische Geschenk, das die Drachen ihr vermacht hatten.

Mit ausgestreckter Hand gebot Sagramak der Eiskristallkette telekinetisch Halt. Siantaris Miene blieb ausdruckslos.

Rasch schluckte Sagramak den Kloß in ihrer Kehle herunter und sprach: „Versteh mich nicht falsch. Ich möchte keine Eis-Dämonin werden. Ich bin vom Feuerdrachen Sagrak beseelt. Die Drachenseele in mir hätte gar keine Freude daran, wenn ich diese eiskalte Kette anfasste. Doch vielleicht kann ich auch so etwas Schönes mit ihr schaffen. Ich spüre die ungezähmte Kraft, die darin eingesperrt ist. Sagraks Seele fühlt es auch. Sie musste sich schon so lange mit meinem schwachen Körper begnügen. Sie drängt darauf, wieder in einen größeren Leib zu schlüpfen.“

Siantari blickte ausdruckslos drein. Sie verstand kein einziges Wort, das die fremde Frau vor ihr sprach. Doch das musste sie auch nicht. Die Eiskristallkette war nun nahe genug an der Haut der Fremden, damit Siantari flüchtige Eindrücke ihres Geistes erhaschen konnte. Und was sie da sah, gefiel ihr.

Siantari nickte und breitete ihre Arme aus. Eine Wolke aus Schnee und Eis erhob sich um sie und trug die Dämonin hoch in den Himmel. Dann wirbelte sie davon.

Zurück blieben der Schwarze Herold und Sagramak, vor deren Hand eine kleine, unscheinbare Kette schwebte.

„Wow!“, rief Sagramak, „Den Drachen sei Dank, du hattest recht! Wie knapp bin ich soeben mit meinem Leben davongekommen?“ Nachdenklich blickte sie der davonschwebenden Siantari hinterher.

„Tu es!“, drang eine zischende Stimme unter der eisernen Maske des Schwarzen Herolds hervor.

„Hetz mich nicht“, protestierte Sagramak. Allerdings schloss sie fügsam ihre Augen und konzentrierte sich. Blauweiße Schleier der Magie umschwirren sie.

Als Sagramak ihre Augen wieder öffnete, flackerte ein roter Schein daraus hervor. Laut rief sie: „Sagrak, o edler Drache, dessen Seele schon zu lange in einem zu kleinen Relikt um meinen Hals steckte. Ich beschwöre dich. Möge der Geist, der in mir ruhte, nun ein neues Gefäß erhalten. Ein besseres. Ein magisches.“

Aus Sagramaks Rüstung erhob sich, wie von unsichtbaren Händen geführt, eine lange eiserne Kette, an der ein rotes Drachenrelikt schimmerte. Die metallene Drachenfigur darauf zeigte Sagrak, wie er zu Lebzeiten ausgesehen haben mochte.

Sagraks Knochenfragmente, die ebenfalls an einer Kette um Sagramaks Hals gehangen hatten, glommen bläulich auf. Sie brachen die eisernen Kettenglieder zwischeneinander und schwebten nach vorne. Knapp über dem Boden blieben die Drachenfußknochen zitternd schwebend stehen.

Die Eiskristallkette summte und brummte. Wasser tropfte aus dem Boden in den Himmel, kondensierte aus der Luft, erstarrte zu Eis und Schneeflocken. Wie aus dem Nichts formten sich weitere Drachenknochen aus purem Eis und ordneten sich neben den echten an. Dann wurden sie von einer Schicht aus Eis und Schnee überdeckt, die zunächst Muskeln, dann Haut und schließlich messerscharfen Schuppen glich.

Die Eiskristallkette schwebte aus Sagramaks ausgestreckter Hand nach vorne und fügte sich nahtlos in die Brust des riesigen Eis-Drachen ein, der vor Sagramak und Siantari schwebte und zitternd ein- und ausatmete.

Schneeflocken wirbelten wild umher, als der schneeweiße Drache seinen vereisten Kopf schüttelte und aus Versehen mit einem ungewohnten vielzackigen Dämonengeweih gegen den Boden stieß. Gewaltige eisblaue Schwingen wurden gespreizt, während eisige Blitze über den Himmel zuckten. Klirrende Dampfwolken bildeten sich vor der langen, durchscheinenden Zunge, die aus einem vielzahnigen Mund herabzüngelte. Dann wurde der Mund aufgerissen und die dahinter liegenden Stimmbänder entließen ein ohrenbetäubendes Gebrüll.

Sagrakdur war erwacht.




















\newpage
\section{Die Spur der Eis-Dämonin}




Die tapferen Helden Chada, Thorn, Eara, Kram, Fenn, Hogo, Bragor, Kheela, Iril, Ijsdur, Aćh und Barz fanden sich beim ersten Sonnenlicht im Rietland zusammen. Leise Worte wurden gewechselt.

„..., zehn, elf, zwölf“, zählte Chada, „Wo ist Tenaya?“

Eara kannte die Antwort: „Tenaya wollte unbedingt mit ihrem ehemaligen Meister Lifornus und diesen beiden Danwaren an der Stätte der heiligen Flammen in der Barbarensteppe diesen Lavastein in des Feuerkriegers Brust untersuchen. Sie hoffen wohl, mit ihren Feuerzaubern etwas darüber herauszufinden, was den beiden Orden Danwars entgangen wäre. Und ausgerechnet so weit im Osten! Mit ihnen – und mit Flaps – können wir in der nächsten Zeit nicht rechnen. Meister Lifornus meinte gar, er wolle bis im nächsten April an der Stätte der heiligen Flammen verbleiben. Irgendeine mächtige Sternenkonstellation soll dort am 1. Tag des 4. Mondes eine besondere Macht enthüllen.“

„Lifornus?“, horchte Barz auf. „Etwa derselbe Lifornus ...“

„Derselbe Lifornus, der so leichtsinnig einen Drachen und einen Takuri beschwor?“, fragte Chada mit einem schiefen Grinsen, „Ich kenne die Geschichten der großen Geschehnisse in Thakkum. Nein, Tenayas Lifornus ist erheblich älter und etwas weiser als der eure.“

„Der Name ist in Hadria weit verbreitet“, steuerte Eara bei.

Barz wollte schon zur Verteidigung der Weisheit seines Lifornus antreten, überlegte es sich dann jedoch wieder und schloss seinen Mund.

„Folglich sind Tenaya, Jarid und Trieest heute allesamt nicht zu erwarten“, rekapitulierte Chada.

Kram fügte an: „Und Orfen ist mal wieder mit Merrik im Grauen Gebirge unterwegs. Auf ihn müssen wir auch nicht warten.“

„Orfen und Merrik?“, meinte Barz, „Dieser brummige Wolfsfreund und dieser Kartenzeichner, der keine rohen Fische essen will, ja, der sogar Stahlfischöl ausschlägt? Ich kenne sie beide, einst verschlug es sie mitten in unsere schöne Steppe. Wie geht es ihnen?“

„Du kennst sie?“, meldete sich Iril, „Orfen haben wir erst kürzlich getroffen. Er hatte vor Taroks Tod einen beeindruckenden Sieg über den Schwarzen Herold errungen.“

Thorn sagte: „Zu schade, dass Merrik weg ist. Er hätte den Tulgori bestimmt bei ihrer Kartographie aushelfen können.“

Kram meinte: „Ich bin mir da nicht so sicher. Kartographen sind so ein Völkchen, die wollen manchmal lieber ihre eigenen Karten zeichnen, als die anderer zu kopieren.“

„Zumindest abgleichen hätte man sie anschließend können.“

„Können sie ja später immer noch.“

Chadas Stimme unterbrach die Gespräche: „Verzeiht, werte Anwesende, doch wollen wir den Tratsch nicht in die Taverne verschieben und uns zunächst auf die dringlicheren Probleme berufen?“

Es wurde still in der Versammlung der Helden. Barz‘ Fokus richtete auf Chadas Bogen, den die Bewahrerin an einen Stein gelehnt hatte und der nun direkt vor ihm lag. Barz griff danach, staunte über seine Länge und zupfte probehalber an der Sehne.

Er nickte Chada anerkennend zu. Diese bat ihn freundlich lächelnd, Audax doch bitte in Ruhe zu lassen. Barz‘ Mund formte lautlos die Silben „Audax“. Dass die Bewahrer ihren Waffen eigene Namen gaben, half ihrem mythischen Status.

Chada eröffnete die wöchentliche Versammlung der Helden von Andor. Es gab Berichte über die Heldentaten der letzten Woche auszutauschen, verschiedene Hilfsbedürfnisse abzuwägen und Pläne zu ihrem Handeln zu treffen. Auch wenn die Helden von Andor keine strikte Struktur wie der Bewahrerorden besaßen und ihre Mitglieder eigentlich frei handeln konnten, hatten sich in der wilden Zeit nach Taroks regelmäßige Absprachen als äußerst hilfreich erwiesen.

Insbesondere jetzt, wo sie die neugierigen Tulgori bei ihrer Kartographie des Landes zu beschützen hatten.

Die Helden waren zum Aufbruch bereit, als sie in der Ferne einen Mann sahen, der schnell in ihre Richtung lief.

„Wir brauchen eure Hilfe! Andor ist in großer Gefahr! Dreifacher Gefahr!“, rief der Andori atemlos, als er endlich die Heldengruppe erreicht hatte. Vierundzwanzig Heldenaugen und einige Tulgori blicken ihn erwartungsvoll an.

In Irils Magen machte sich ein Unwohlsein breit, als sie die Verzweiflung im Gesicht des Mannes deutete. In letzter Zeit hatte sie wie die meisten restlichen Helden auch den Andori geholfen, Taroks Spur der Zerstörung ungeschehen zu machen. Bauernhäuser waren wieder aufgebaut worden. In die Rietburg geflüchtete Bauern konnten endlich wieder zu ihren Katen und Feldern zurückkehren.

Doch die Lage sah nicht gut aus. Kaum eines der Felder hatte den Zorn des Drachen überstanden und so versuchten die Andori durch Fischfang und Jagd zu überleben. Doch mussten sie stets fürchten, selbst zur Beute zu werden. Und wie sollten die tapferen Andori das wenige übrige Korn ernten, wenn sie jederzeit einen Angriff fürchten mussten? Der kommende Winter würde bald seine eiskalte Hand ausstrecken. Die Zeit drängte.

Und so hatten sich die Helden von Andor, darunter auch die neu dazu gestoßenen Iril, Ijsdur, Aćh und Barz, nach getaner Arbeit nicht etwa ausruhen können, sondern mit doppeltem Elan an die Verteidigung der tapferen Andori gemacht.

Tapfer waren die Andori wirklich. Schon von Kind auf bekamen sie von ihren Eltern eingebläut, die Kreaturen zu fürchten, ja, es gab sogar Kinderreime dazu. Abgehärtet, wie sie waren, lösten einzelne Gors und Skrale in ihnen kein Schlottern aus, sondern nur einen zügigen temporären Rückzug in sichere Verstecke. Manchmal gar einen Griff zur Mistgabel. Was konnte also vorgefallen sein, dass dieser eine Andori vor ihnen derart ängstlich angerannt kam? Von welch dreifacher Gefahr sprach er wohl?

„Es ist Shron, der Sohn des Hark, der uns bedroht“, sagte der Mann, nachdem er sich etwas beruhigt hatte. „Er zieht durch das Land und rüstet sich für einen Angriff auf die Rietburg. Mehr und mehr Kreaturen schließen sich ihm an.“

„Den übernehmen wir!“, rief Thorn aus, „Mit ein bisschen Hilfe von Reka konnte ich vor einigen Jahren schon seinen Vater ausschalten. Ich fühle, Harthalts Tod ist noch nicht genug gerächt! Ich dürste nach Skralblut!“

Kram wich etwas zurück, als er die Mordlust in Thorns Blick erkannte. Chada trat hinter Thorn und legte ihm besänftigend eine Hand auf die Schulter. Er verstummte. Chada erhob selbst ihre Stimme und sprach etwas gefasster: „Wenn wir Shron einen Schrecken einjagen können, der für den Rest seiner Tage auf seinem Gesicht geschrieben steht, genügt das schon.“

„Viel kann man von seinem Gesicht nicht sehen“, meinte Eara, „Wir haben Shron gerade erst bei der Befreiung der Rietburg getroffen. Er war der Skral mit der eisernen Maske.“

„Umso besser!“, lachte Thorn, „Da kann man tatsächlich einen Schrecken draufschreiben.“

Chada, Thorn, Eara und Kram wichen etwas zurück und berieten sich darüber, wie man am besten die Bildung eines Kreaturenheers verhindern könnte. Der Andori, der die Botschaft überbracht hatte, atmete rasch ein und aus, als er nach weiteren Worten rang.

„Du sagtest, da sei noch mehr?“, hakte Kheela nach, die Hüterin der Flusslande, die oft mit einem danwarischen Stab einen Wassergeist umherlenkte. Gerade tanzte Vara über ein nahegelegenes Feld und löste die Schneespuren auf, die ein gewisser unachtsamer Eis-Dämon beim Darüberspazieren hinterlassen hatte.

„Ja, da ist noch mehr. Es ist Skuvar“, nickte der Mann, nachdem er sich etwas beruhigt hatte, „und mit ihm sind die Maasavi erwacht. Skuvar ist ein uralter Erdgeist. Die Maasavi sind seine Schatten, die Seele der andorischen Erde. Ich weiß nicht, was sie hervorgelockt hat, doch Skuvar wurde unweit der Rietburg gesehen.“

Der Mann schnappte kurz nach Luft und sprach dann: „Ihr könnte ihn nicht verfehlen. Ein massiger Körper mit einem dehnbaren Mund, der blau leuchtet. Rücken- und Schwanzstacheln, länger als mein Unterarm. Und gewaltige Vorderbeine, auf denen er sich manchmal im Handstand fortbewegt. Seid wachsam. Die Maasavi folgen seinem Ruf und werden angetrieben von Eurem Willen, Andor zu verteidigen.“

„Ich kenne mich mit Erdgeistern aus!“, rief Kheela, „Vara wird uns helfen, sie zu besänftigen. Wir hatten in den Flusslanden schon mal mit einigen zu ringen.“

„Gehen wir vorhin zum Thronsaal und holen uns den Bruderschild, auf dass Bragor uns an der Kraft seiner prallen Muskeln Anteil haben lassen kann?“, schlug Fenn vor.

„Ich verstehe immer noch nicht, warum wir den Schild diesem König geben mussten“, flüsterte Bragor laut genug, um von allen Anwesenden gehört zu werden. Taren hatten bekanntlich unglaublich gute Ohren, flüsterten aber offenbar dennoch ziemlich laut. „Jetzt vergammelt der Bruderschild die größte Zeit in der Burg, während Brandurs Sohn, der Schürzenjäger, oft nicht mal in dort ist, und den Schild noch seltener nutzt.“

Hogo stupste Bragor zwar in die Rippen, um ihn vom unbedachten Sprechen abzuhalten, doch handelte er sowohl zu spät als auch zu sanft, damit Bragor es überhaupt registrierte.

Barz blickte Bragor fasziniert an, wie jedes Mal in den letzten Tagen, wenn Bragor sprach. Barz hatte Iril zwar erklärt, dass er im Tarus nicht den fleischgewordenen großen Büffel sah, den die Yetohe in ihm gesehen hatten. Dennoch schien ihn etwas an diesem Fremden aus Sturmtal zu faszinieren.

Hogo, Bragor, Fenn und Kheela koppelten sich ebenfalls von der restlichen Heldengruppe ab.

Damit blieben noch die vier neusten Helden von Andor übrig.

Die Augen des Mannes wurden leicht glasig, als er zu schlottern begann. Als er sich wieder etwas beruhigt hatte, berichtete er den Helden von einer weißen Gestalt, umgeben von Wirbeln aus Schnee und einem solch kalten Blick, dass man meinte, sofort zu Eis zu erstarren. Nur Augenblicke später sei die Gestalt wieder verschwunden.

„Siantari ...“, flüsterte ein Tulgori blass. „... eine Dämonin des Kuolema. Ihr müsst sie finden und aufhalten, sonst wird sich das ganze Land in eine Eiswüste verwandeln. Es hat sicher schon begonnen ...“

Ijsdur, der die Tulgori überhaupt erst vor Siantari gewarnt hatte, nickte stumm. Der Mann, dem Ijsdurs Anwesenheit überhaupt erst jetzt aktiv aufzufallen schien, blickte ihn erwartungsvoll an. Dieser erwiderte den Blick still, offensichtlich nicht kapierend, was man von ihm erwartete. Da meldeten sich Aćh und Barz beinahe gleichzeitig zu Wort:

„Das klingt nach einem Fall für uns! Wir hatten schon einmal mit einer Eis-Dämonin zu tun und waren siegreich, und da waren wir nur zu zweit! Eigentlich war es sogar Aćh allein.“

„Und du hast meinen Arm gebrochen“, sagte Aćh in Tulgorisch.

„Technisch gesehen hast du ihn dir selbst gebrochen“, gab Barz zurück.

„Da schließe ich mich doch euch an“, hob Iril ihre Hand. Alle blickten gespannt zu Ijsdur. Dieser nickte ebenfalls.

Die vier waren in den Tagen nach dem Erringen der Drachenknochen nur enger zusammengewachsen. Sie hatten einander zu schätzen gelernt. Ijsdur hatte sein strategisches Geschick im Astz-Kartenspiel enthüllt und damit die Herzen von Barz und Aćh errungen. Iril und Barz hatten sich in tiefgründigen Gesprächen über ihre Spezialgebiete der Magie mitreißen lassen.

Und nun machten sie sich auf den Weg, Ijsdurs Dämonin zu überwinden.\bigskip







Barz verteilte die zerriebenen Rietgrasblüten auf dem gewaltigen Eisblock, der aus der Narne bis hin ins Fischerdorf ragte. Langsam schmolz das Gebilde und gab die Boote der ängstlich danebenstehenden Andori frei. Jubel brandete auf. Unsere Helden konnten allerdings nicht einstimmen.

„Das bringt so nichts“, murmelte Barz, „Wir sind schon im ganzen Land Siantaris Spuren gefolgt und haben unter großem Aufwand die Überreste ihrer niederregnenden Eisblitze beseitigt. Doch noch immer sind wir ihr kein bisschen näher als zuvor!“

Als wollte er seine Worte unterstützen, zuckte ein weiterer Eisblitz über den Himmel. Ein erst einige Zeit später ertönendes dumpfes Donnern verriet, dass er in weiter Ferne niedergegangen war. Ein weiteres Bauernhaus getroffen? Ein weiterer einsamer Wanderer im Eis erstarrt? Was es auch war, es würde sie wieder Zeit kosten, ohne mehr über die Position der Urheberin des Übels zu verraten.

„Danke für die exzellente Zusammenfassung, Barz“, sprach Iril, „.Wir müssen unsere Taktik ändern.“ Sie setzte sich entschlossen auf einen Stein am Wegesrand, packte ihre steinerne Runenscheibe aus und begann, daran zu werkeln.

„Ein Fernrohr zum Spähen wäre nicht schlecht“, meinte Aćh, „Nach dem, was Ijsdur erzählt hat, ist Siantari alles andere als unauffällig.“.

„Die Tulgori haben den zerstörten freien Markt zu ihrem Lager erklärt“, warf Ijsdur ein, „So geschickte Spiegel- und Brillenbauer sie auch sind, tragbare Fernrohre scheinen nicht Teil ihres Repertoires zu sein. Ich wüsste nicht, wo wir sonst in der Nähe welche kaufen könnten.“

Barz schlug sich an den Kopf: „Wie konnte ich das vergessen! Wir müssen nur mein ...“ Er griff an seinen Pulvergürtel und seufzte. „Natürlich. Ich habe meinen Vorrat an Umwandlungspulver bei Sabri gelassen.“

Er blickte den Weg durch das Rietgras zurück, der die Heldengruppe zum vereisten Fischerdorf geführt hatte.

„Vermutlich stapft sie brav unseren Fußstapfen nach. Oder stampft gerade einem andorischen Bauern quer durchs Feld und ich werde wieder dafür geradestehen müssen. Wie dem auch sei, ich mache mich besser daran, Sabri aufzuspüren. Ich werde uns mit dem Umwandlungspulver ein Fernrohr herbeizaubern. Und dann kehren wir auf den höchsten Turm der Rietburg zurück und sondieren die Umgebung. Ist jemand mit mir?“

Iril schüttelte ihren Kopf: „Ich bleibe lieber hier sitzen und kümmere mich um meine Runenscheibe. Wenn ich es schaffe, sie auf Eismagie anspringen zu lassen ...“

„Dann sollte ich lieber nicht zu nahe sein und das Ergebnis verfälschen“, sprach Ijsdur. „Ich werde dich begleiten, Freund Barz.“

„Danke dir, Freund Ijsdur!“, grinste Barz. Welch Animosität er einst gegenüber dem Eis-Dämon gehegt haben mochte, sie war geschmolzen wie Schnee in der Wüste, nachdem Ijsdur Barz‘ verlorene Ringkette zurückgegeben hatte.

„Turr und ich bleiben sonst bei Iril und sorgen dafür, dass sie vor lauter Konzentration nicht plötzlich einen anstürmenden Gor übersieht“, meinte Aćh.

Damit schienen alle einverstanden. Während die Fischer in ihr vom Schnee befreites Dorf zurückkehrten und die Helden allesamt zum Essen einluden, brachen Barz und Ijsdur Seite an Seite auf, zurück in Richtung Sabri.

„Passt gut aufeinander auf!“, riefen Iril und Aćh ihnen hinterher.

„Nicht wir sind es, die an einem Ort rasten, der bis soeben noch von einem Eisblitz bedeckt war.“

„Man sagt doch, Blitze schlügen nie zweimal am selben Ort ein“, gab Iril zurück.

„Eisblitze vielleicht schon. Und das Sprichwort stimmt ohnehin nicht. Genau dafür gibt es auf Thakkum ja Blitzfänger-Stangen“, protestierte Barz.

„Ihr fangt Blitze?“, fragte Ijsdur überrascht.

„Keine große Kunst. Du musst nur genug hoch in den Himmel vordringen und die Blitze springen dir von selbst entgegen.“

Still wanderten Steppennomade und Eis-Dämon Seite an Seite am Narnenufer entlang durch das weite Rietland. Dunkle Wolken warfen ihren Schatten auf die wenigen Bäume, die hier und da umherstanden.

„Was wäre denn größere Kunst? Dein Umwandlungspulver?“, fragte Ijsdur nach einer Weile.

„Ja, magische Pulver sind durchaus einiges komplexer als Blitzfänger-Stangen. Die richtige Mischung von Materialien zu finden ist aufwendig und fehleranfällig. Fehler, die überaus gefährlich sein können. Aber die Pulverschamanen der Iquar fanden schon einige stabilere Verbindungen, deren magischen Effekte man mit der entsprechenden Expertise ziemlich konsistent hervorrufen kann. Einige solche Pulver führe ich nun mit mir.“

„Könntest du mir eigentlich beibringen, sie zu nutzen?“

Barz schluckte schwer. „Könnte vielleicht schon, will aber nicht zwingend. Jedenfalls nicht so husch-husch. Es ist besser, wenn nur ich meine Pulver handhabe.“

Ijsdur blickte ihn schief an: „Wie kommst du denn darauf?“

„Ach, weißt du“, verwarf Barz seine Hände, „Ich schenkte Aćh einst eines meiner Pulver, einen Nixenstaub. Leider wusste ...“

„Nixenstaub? So was sah ich dich noch nie einsetzen.“

„Mein Vorrat ist ja auch alle. Scheinbar gibt es weder in Tulgor noch in Andor nennenswerte Nixendörfer.“

„Dafür aber in Silberland. Glaube ich. Wenn du willst, könntest du Iril mal danach ansprechen, sie erzählte mir jedenfalls einmal von der Nixe, die einst ihren Runenhammer schuf.“

„Naja, vielleicht ist es auch besser, wenn dieses Pulver nicht allzu verbreitet ist. Denn wie schon gesagt: Leider wusste Aćh natürlich nichts über die Feinheiten des Nixenstaub-Einsatzes. Insbesondere nicht, dass man es lieber nicht auf verletzte Gegner werfen sollte. Beinahe wären wir deswegen beide zu Eis-Dämonen geworden.“

„So schlimm ist das gar nicht“, meinte Ijsdur mit einem gezwungenen Lächeln.

„Für dich vielleicht nicht mehr. Siantari hätte uns genutzt, um die ganze Welt zu vereisen versuchen. Es war fahrlässig von mir. Ich werde meine Pulver nicht mehr so unvorsichtig mit anderen teilen. Wenn ich damit Fehler mache, liegt die Schuld wenigstens vollständig bei mir und ich ziehe nicht noch andere Leute rein.“

„Das ist die Lektion, die du daraus ziehst? Was wäre damit, Leuten, denen du Pulver schenkst, angemessen zu erklären, wie sie zu benutzen sind?“

„Ich konnte zu diesem Zeitpunkt vielleicht zehn, zwanzig tulgorische Wörter! Und das Pulver war auch mehr eine symbolische Geste. Weißt du, es war mein Fehler, der Turr im ewigen Eis festsitzen ließ, und die arme Aćh schluckte ihre Wut darüber und ihre Sorgen um Turr herunter und half mir, kümmerte sich um mich! Da musste ich doch irgendwie ausdrücken, dass ... ich meine, Worte allein haben nicht dieselbe Wirkung wie eine solche Geste, das verstehst du, oder?“

Ijsdur hob beschwichtigend seine Hände. „Durchaus. Dennoch finde ich, dass die Beschränkung deiner Pulver auf dich selbst nicht der optimalste Schluss ist, den du aus diesem Vorfall ziehen könntest. Du hast ja auch nicht aufgehört, mit neuen Pulvermischungen zu experimentieren, nur weil das Experiment mit Turr schieflief, oder?“

„Nein. Aber es lässt mich all meine künftigen experimentellen Thesen zweimal überdenken, insbesondere wenn andere Wesen involviert sind. Wir alle machen Fehler, aber wenn wir daraus lernen, machen wir sie in Zukunft seltener. Einfach keine Experimente mehr durchzuführen, um keine Fehler zu riskieren, ist auch nicht die glücklichste Lösung.“

„Genau! Siehst du, was ich meine?“

Barz blickte Ijsdur überrascht an, nickte dann aber.

„Also dann, magst du mir demnächst mal den Umgang mit einem deiner Pulver beibringen?“

Barz‘ Grinsen erstarb und er druckste weiter herum. „Es ist ein langwieriger Prozess, und kompliziert, und es darf auch wirklich nichts schiefgehen, also ...“

„Und es ist ein langwieriger Prozess, einen Menschen von etwas zu überzeugen, gegen das sich seine Schuldgefühle sträuben“, beendete Ijsdur die Konversation, „Belassen wir es fürs erste dabei? Ich freue mich schon darauf, in Zukunft weiter darüber zu diskutieren.“

Barz nickte und blieb eine Zeit lang mit seinen Gedanken allein.

„Schau, da vorne, ist das sie?“, fragte Ijsdur.

„Beim großen Seeadler, was hast du denn für Augen? Ich sehe nur Nebel!“

„Da vorne bei der Marktbrücke ruht sie, deine Echse, ich sehe sie immer deutlicher. Schläft tief und fest.“

Barz‘ Schritte wurden beschwingter.

„Die Marktbrücke ist einer meiner liebsten Orte in ganz Andor!“, sprach er.

Als er ansetzte, auf ihre wundervollen Besonderheiten einzugehen, protestierte Ijsdur: „In Tulgor gibt es so viel schönere und stabilere Brücken. Die Marktbrücke wirkt, als habe ein eifriger Bauherr so rasch wie möglich die Narne überwinden wollen und danach einfach zwei Zwergenstatuen auf jede Seite geklatscht. Du willst mir doch nicht etwas sagen, dass diese Brücke stabil sei?! Jeder halbgroße Krake sollte diese Brücke im Nu auseinandernehmen können.“

„Ich kenne mich nicht mit Stabilität aus, vertraue den Werken der Schildzwerge aber diesbezüglich durchaus. Und elegant ist die Brücke doch auf jeden Fall.“

Die beiden erreichten die Markbrücke. Unter ihnen rauschte die Narne. Zwei Zwergenstatuen bewachten die Marktbrücke. Daneben schnarchte Sabri gut vernehmlich.\bigskip







„Arrogscheiße!“, fluchte Iril auf, als der Eisenstift von der Runenscheibe abrutschte und ihre Hand aufschürfte.

„Was ist ein Arrog?“, fragte Aćh grinsend, „So langsam gewöhne ich mich an den Sprachtrank dieser Hexe, doch ein Arrog sagt mir noch nichts.“

„Stell dir eine gewaltige Klippe vor, aber lebendig. Mit muskulösen Armen, gepanzertem Rücken, einem riesigen Maul voller spitzer Zähne und ... ich bin mir nicht mal sicher, wie sie unter der Wasseroberfläche aussehen. Sag, gibt es in deiner Heimat Tulgor auch solch bösartige Kreaturen?“

„Gibt es faszinierende Kreaturen nicht überall? Oh, von welch fantastischen Wesen ich dir berichten könnte. Doch bösartig sind die wenigsten. Viele folgen nur ihren Instinkten. Auch der netteste Takuri könnte mit einem einzelnen Funken der Begeisterung die Rote Steppe niederbrennen.“

Iril schnaubte. „Takuri werden ja auch nicht von einer bösartigen Macht angetrieben.“

„Redest du vom Drachen, der die Kreaturen antrieb?“

„Ja. Oder Dunkle Magier und finstere Nekromanten, die anderen gerne ihren Willen aufzwingen. Oder Mächte des Meeres, die es nicht mögen, wenn man zu tief in den Ozean vordringt. Mir scheint, nur wenige Kreaturen sind wirklich frei.“

„Nun, in Tulgor gibt es weder Mächte des Meeres noch Nekromanten. Und die meisten unserer Magier leben zurückgezogen in hohen Türmen und kümmern sich wenig um die Wesen der Wildnis.“

Aćh streichelte Turrs Kopf sanft und blickte in die Ferne, wo die wolkenverhangenen Hänge des Fahlen Gebirges hoch aufragten. Wehmut lag in ihrem Blick.

Iril hörte auf, das Blut von ihrer verletzten Hand zu putzen, und fragte: „Vermisst du Tulgor sehr? Wie gut lebst du dich hier ein? Wann geht es für dich wieder zurück?“

Ein Moment der Stille herrschte, nur unterbrochen von Turrs fröhlichem Gurren. Dann antwortete Aćh leise: „Ich weiß nicht so recht. Ich bin eine Fremde in einem fremden Land hier. Ich wusste, worauf ich mich einließ, als ich die Reise hierhin antrat. Aber ich war nicht darauf vorbereitet, dass auch Barz seine eigene Geschichte hier hatte. Er und Nabib verbringen viel Zeit miteinander. Und Barz ist nicht nur meine einzige Verbindung zu den Andori, er war auch meine direkte Verbindung zu den tulgorischen Minenarbeitern und anderen Reisenden. Er arbeitete lange Zeit in den Mera-Stollen, nicht ich. Ich ... ich bin ziemlich allein hier.“

„Da sprichst du mir aus der Seele“, meinte Iril, „Auch ich fühlte mich als Fremde hier. Meine Verwandtschaft ... meine Schwester, von der ich nicht einmal wusste, dass es sie gab ... die Schildzwerge wollen mich nicht und die Andori natürlich auch nicht. Und auch in Silberhall würde ich nur stetig daran erinnert, dass Burmrit, meine Lehrmeisterin, nicht mehr unter uns weilt. Doch es wurde besser.“

„Wir können zusammenhalten. Wir beide“, meinte Aćh nun.

„Können wir. Wenn du hier in Andor bleiben willst. Du hast, wenn ich es richtig verstanden habe, im Gegensatz zu mir noch eine Familie auf der anderen Seite der Berge. Du kannst noch dorthin zurückkehren, wenn dir das Rietgras zu den Ohren raushängt.“

„Das ist ja mal ein eigenartiges Sprachbild“, schmunzelte Aćh, „Ja, ich habe in Tulgor noch eine Familie. Verwandte, Bekannte, Freunde. Vielleicht kehre ich eines Tages dorthin zurück. Falls du dich bis dahin nirgendwo eingelebt hast, könntest du da vielleicht auch mitkommen. Wenn du willst. Ich könnte dir den Nistbaum und die weite Steppe zeigen, statt nur darüber zu reden.“

„Vielleicht. Wo immer man unsere Hilfe brauchen kann“, nickte Iril träumerisch. „Und sofern wir uns bis dahin noch ausstehen können.“

„So, wie ich dich in den letzten Wochen kennengelernt habe, mache ich mir da keine Sorgen. Aber für den Moment bleibe ich ohnehin gerne hier. Andor ist ein konfliktreiches Land. Die Leute hier brauchen nach dem Wüten des Drachen mehr Hilfe als die Tulgori. Die Sprache ist ein Hindernis, aber abgesehen davon gefällt es mir ungemein hier. Und Leuten in Not zu helfen, wirklich zu helfen, fühlt sich einfach gut an.“

„Es wäre sogar gut, Leuten zu helfen, auch wenn es sich nicht gut anfühlte. Das gute Gefühl ist nur ein Bonus.“

„Genau. Na los, Iril, lass uns Eis-Dämonen jagen.“ Mit Blick in den Süden, wo Ijsdur davonstapfte, fügte Aćh an: „Nur gefährliche natürlich.“

Iril runzelte ihre Stirn und kehrte zurück zur Arbeit auf ihrer Runenscheibe.

Eine Viertelstunde später war sie soweit. Iril hob ihren Runenhammer in die Höhe und murmelte etwas vor sich. Die in die uralte Waffe eingeritzten Runen glommen grünlich auf.

Interessiert blickte Aćh zu. Noch immer hatte sie keine Ahnung, woher dieses Artefakt genau seine Kraft bezog.

Iril ließ den Hammer auf die Scheibe niederfahren. Ein blechernes Klingen ertönte. Doch das magische Glühen sprang nicht vom Hammer auf die Scheibe über. Vielmehr verstärkte sich das Leuchten des Hammers abrupt. Die Scheibe glitt zu Boden und rollte nutzlos ins Rietgras.

Grünliche Schwaden steigen von Irils leuchtenden Tattoos auf. Aćh fiel es auf einmal schwer zu glauben, dass die Runentattoos nur aus unter der Haut bugsierten Farbteilchen bestanden. Wie kleine Lebewesen, die sich verselbstständigten, wanden die leuchtenden Runen sich. Iril schrie auf und klappte zusammen. Ihr ganzer Körper zitterte.

Aćh fiel neben Iril aufs Knie und streckte ihre Hand aus. Zwischen klappernden Zähnen stieß Iril hervor: „Nein! Fass .... fass mich nicht ... du bist ... Mensch, du bist ... Dunkle Magie ... nicht gewachsen ... Stimme ... verlockend ...“

„Was soll ich tun?“

Irils Körper lag flach wie ein Brett am Boden und wurde von Krämpfen geschüttelt. Eine erschreckend lange Zeit antwortete sie nicht und atmete flach, dann würgte sie hervor: „Geht ... vorbei. Keine Sorge, das ... nicht tragisch.“

Aćh war nicht überzeugt. Besorgte Flammenbäusche überzogen auch Turrs Gefieder. Dann flüsterte Iril: „Bleibe ... bei mir ... bitte.“

Das Schütteln ließ langsam nach. Irils verkrampfte Hand öffnete sich und ließ den schweren Runenhammer zu Boden plumpsen. Das grüne Glühen verlosch. Iril blieb noch eine Weile liegen und blinzelte schwach.

„Guck bitte ... nicht ... einschlafe“, hauchte sie erschöpft. Aćh war sofort wieder auf ihren Beinen und achtete darauf, dass Iril ihre Augen offenhielt. Nach einer Weile hatte sich der Atem der Zwergin wieder normalisiert. Iril streckte ihre Hand aus und die sitzende Aćh half ihr auf die Beine.

„Alles wieder klar, Runenmeisterin?“

„Alles klar, Takuri-Hüterin“, sprach Iril schwach.

„Bei den sieben Feuern des Himmels, was war das?“

„Kompliziert“, murmelte Iril, „Die Runen können Kraft leiten, in Dinge hinein oder hinaus. So was wie vorhin sollte nicht passieren, wenn man genügend vorsichtig ist beim Runenzeichnen. Dieser Hammer ... Die meisten Runen werden vom Licht des Mondes oder der Sonne gespeist. Dieser Runenhammer ist sehr nützlich, um Runen unabhängig von der Tageszeit oder dem Mondzyklus zu aktivieren. Aber er ist immer noch ein von Dunkler Magie erfülltes Artefakt. Und diese fordert meistens ihren Tribut. Die Runen auf dem Hammer erlauben uns, den größten Teil dieser Last nicht selbst tragen zu müssen. Ja, ohne diese Runen könnte auch ich die Magie in seinem Innern nicht nutzen. Aber so mächtig die Runen auch sind, sind sie selbst für die größten Meister schwer zu bändigen. Manchmal scheint es mir, als verselbstständigten sie sich langsam. Und dann zehren sie an meiner Willenskraft. Und selten, ganz selten, bricht die Dunkle Magie völlig aus dem Hammer aus und überzieht die Runen auf meinem Körper. Die saugen mir dann wie kleine Egel die Energie aus. Diese Vorfälle sind unregelmäßig, unberechenbar, meistens ganz unscheinbar, aber hin und wieder einfach unausstehlich. Ich wünschte, ich hätte einige Runensteine bei mir. Darin lässt sich die bei solchen Anfällen durchströmende unvorstellbare Energie speichern und produktiv nutzen. Du hättest nicht zufälligerweise welche bei dir gehabt?“

„Runensteine?“, fragte Aćh, „Ich bin mir nicht sicher, ob ich das Wort verstehe. Ich kenne Edelsteine und Mera-Steine, doch ein Runenstein ist mir unbekannt. Ist das einfach ein Fels, in den Runen gemeißelt worden waren? Die Temm waren Meister darin, solche Gänge zu bauen.“

„Na, ein bisschen mehr als bekritzelte Steine sind Runensteine schon“, meinte Iril, „Aber im Großen und Ganzen hast du recht. Bevor das Geheimnis ihrer Erschaffung verloren ging, schufen die Runenmeister der Schildzwerge vor Urzeiten Unmassen von Runensteinen. Auch heute noch kann man in den Minen und selbst hier draußen im Land welche finden. Leider trifft man sie kaum auf den Inseln des Nordens an, sonst hätte ich schon längst eine ganze Sammlung davon angelegt. Die Kraft der Runensteine kann sehr vielseitig nützlich sein und auch ohne magische Kräfte genutzt werden. Auch wenn sie leider primär zu Kriegszwecken eingesetzt wurden. Gegen die Trolle, gegen die Drachen, sogar in Kämpfen von Zwerg gegen Zwerg. Wie viel potenzielles Wissen wohl verloren gegangen ist, weil wir uns gegenseitig abschlachteten, statt zusammenzuarbeiten? Ohnehin waren die Runenmeister der Urzeiten geschickter als wir es heute sind. Sie schufen ein weit verbreitetes Netzwerk aus Gängen, welche durch magische Zwergentüren verbunden waren, die die Gegensätze von Feuerrunen und Wasser vereinten. Und dann erst die legendären unterirdischen Runengänge in unbekannte, fremdartige Reiche und Gefilde, die angeblich nur dann, wenn der Rote Mond hoch oben am Himmel steht, sichtbar werden und etwaige Durchquerer eine Zeit lang mit einer raffinierteren Übersetzungsrunen versehen sollen – so wie die, die ich Ijsdur anhängte – ach Aćh, so viele Kenntnisse der vergangenen Runenmeister sind verloren gegangen. Geheimnisse aus der Vergangenheit, die wir in harter Arbeit aufs Neue ergründen. Ich komme glatt ins Schwärmen.“

„Es freut mich vor allem, dass es dir wieder besser geht“, schmunzelte Aćh, „Eklige Egel, diese Runen. Doch ich werde daran denken, falls ich einen Runenstein sehen sollte. Wobei, solche massiven Felsen sind bestimmt schwer zu transportieren.“

„Massive Felsen? Wo denkst du hin, Runensteine sind teils kaum größer als Kiesel! Die passen bequem in deine Reisetasche. Glaube mir, von Größe auf magische Macht zu schließen, kann trügerisch sein.“

„Dem ist wohl so!“, erklang Barz‘ fröhliche Stimme hinter ihnen, „Sabri hat etwa noch keinen Funken magisches Talent gezeigt.“

„Ihr seid zurück!“, rief Aćh auf, denn Ijsdur und Barz waren zurückgekehrt.

„Unser Vorhaben war von Misserfolg gekrönt“, berichtete Iril bedrückt.

„Na, dann ist doch umso besser, dass wir etwas herausgefunden haben“, sagte Barz. Stolz präsentierte er ein Fernrohr, welches golden glitzerte. Wohl frisch umgewandelt.

„Ich blinder Büffel hätte sie glatt übersehen, doch Ijsdur hat eine Spur von Santari gefunden. Die wird euch jedoch nicht gefallen.“

Barz übergab das Fernrohr an Aćh und sprach: „Sieh, dort drüben. Am Alten Wehrturm. Dort, wo Tarok fiel.“

„Was in aller Welt ist das?!“, rief Aćh aus.

„Was? Was siehst du?“, fragte Iril ungeduldig.

„Ein Eis-Drache“, sprach Ijsdur tonlos. „Bei der Ruine des alten Wehrturms wütet ein Wirbelsturm der Kälte, wie an so vielen Orten im Reich. Und darin ruht ein gewaltiges Wesen aus Eis und Schnee. Ich hoffte, es möge nur eine Formation sein. Doch sah ich, wie es sich bewegte. Vier Beine, zwei Flügel und ein stacheliger Kopf auf einem Schlangenhals.“

„Ist es Siantari gelungen, einen Eis-Drachen-Dämon zu beschwören?“, fragte Iril.

„Das sollte sie nicht können!“, sagte Ijsdur, „Siantari ist mächtig, aber sie kann nicht aus dem Nichts Leben formen.“

„Dann bleibt uns nur etwas übrig. Auf, zum alten Wehrturm!“\bigskip







„Ich bin mir nicht sicher, ob ich das kann“, murmelte Aćh. „Gegen diese Kultisten vorgehen. Ich habe bislang nur ein Leben genommen. Diese Eis-Dämonin oben im Ewigen Eis, die mich und Barz zu einer der ihren machen wollte.“

„Das kommt schon“, sprach Barz, „Ich habe schon in der Steppe Banditen bekämpft. Der Tod ist Teil des Lebens.“

„Macht es nicht leichter“, sprach Ijsdur, „Ijs ist für den Tod seiner Freunde verantwortlich.“

Niemand wusste, was darauf zu sagen war.

„Ich habe noch nie getötet“, meinte Iril, „Nur Kreaturen erledigt. Die zählen ja kaum als wertvolle Lebewesen.“

Barz widersprach: „Denken und fühlen können sie doch auch.“

„Aber bitte, ein Gor ist noch instinktgesteuerter als ein hungriger Wolf. Und es besteht doch ein himmelweiter Unterschied darin, ob du einen anderen Barbaren oder eine Echse röstetest.“

„Beim Großen Affen, warum würde ich Sabri je rösten?!“

„Verzeih, ich nahm an, dass sie am Ende ihres Lebens ...“

„Ist das, was ihr in Silberhall tut? Am Ende eures Lebens von Gefährten verspeist werden?“

„In Silberhall geben wir uns größtenteils mit den Erträgen des Meeres als Speise zufrieden. Algen und Seetang lassen sich überraschend gut züchten. Wir hatten keine Ahnung, als wir von Cavern dorthin reisten. Doch die Nixen lehrten uns.“

„Die Nixen lehrten auch uns so einiges, als unser Stamm sich im großen See Ava niederließ.“

Das Gespräch zog sich ähnlich weiter, während die Helden weiter durchs Rietland schritten.

Sie hatten kaum einen Drittel der Strecke zum Alten Wehrturm überwunden, da hieß Ijsdur sie an, innezuhalten und das Fernrohr wieder hervorzuholen. Etwas bewege sich.

Tatsächlich! Der gewaltige Eis-Drache am Wehrturm hatte seine Schwingen ausgebreitet und war darauf und daran, sich in die Höhe zu schwingen!

„Ich erkenne eine dunkle Gestalt, die neben ihm schwebt. Der Schwarze Herold?“, berichtete Iril, durch die verschwommenen Linsen blinzelnd.

„Ich vermute es stark. Doch wird er kaum allein sein.“, gab Ijsdur zurück. „Darf ich kurz das Fernrohr ... danke sehr! Ja, da reiten gleich eine ganze Gruppe grimmiger Gestalten auf dem Eis-Drachen. Allesamt eingewickelt in schwere Decken und Tücher. Über ein kompliziertes Geflecht aus Gurten an den Drachen geschnallt. Das wird bestimmt einige Zeit gekostet haben, diese herzustellen.“

Iril staunte wieder einmal über den scharfen Blick des Eis-Dämons und fragte sich, wie seine eisigen Augen funktionierten.

„Erkennst du die Personen?“, fragte Barz, der sich nicht von solcherlei Überlegungen ablenken ließ.

Ijsdur bejahte. „Die eine Rüstung würde ich im Schlaf wiedererkennen.“

„Lass mich raten: Sagramak, die Schamanin der Drachenkultisten?“

„Genau, da ist Sagramak. Aber da sind noch mehr. Dieser Nehamal, und das Ziegenwesen Fir, das die Knochen einst stahl. Und einige anderer ihrer Sippe, allesamt bewaffnet. Sie alle haben sich auf den Rücken des Eis-Drachen geschnallt. Sie wollen irgendwo hin.“

„Das ist kein großes Rätsel. Die wollen zur Rietburg, die Knochenkette zurückfordern“, sprach Aćh, „Warum haben wir Taroks Knochen schon wieder dem Prinzen überlassen?“

Iril brummelte: „Weil wir dachten, dass die Rietburg irgendwie sicher wäre. Hätte ja keiner geglaubt, dass die einen Eis-Drachen herbeirufen können.“

„Das sollten sie auch nicht können, sonst hätten sie das auch viel früher getan.“

„Siantari war das aber auch nicht“, sagte Ijsdur, „Wenn sie von selbst ein derart mächtiges Leben schaffen könnte, wäre das Felsentor zum Tal des ewigen Eises schon viel früher gebrochen worden.“

„Aber sie kann Eiskristallketten schaffen!“, erinnerte sich Barz, „Sagramak trug doch Knochensplitter um den Hals! Knochen von Sagrak dem Drachen. Was würde geschehen, wenn man diese mit einer Eiskristallkette verbinden würde?“

Ijsdur blieb stocksteif stehen: „Ich weiß, was geschehen würde, wenn man einen gut erhaltenen Drachenleichnam mit einer Eiskristallkette verbände. Aber wenige Knochen, uralt und verrottet?“

„Es sind nun mal magisch potente Mittel“, knurrte Iril, „Seht ihr nun, warum ich ihr nicht Taroks Knochen überlassen wollte? Irgendwie haben sich die Drachenkultisten mit Siantari verbündet. Und irgendwie haben sie einen Eis-Sagrak herbeigerufen.“

„Sagrakdur“, flüsterte Ijsdur.

Lautes Donnern übertönte seine Stimme und erfüllte die kalte Luft. Unvermittelt begann die Erde zu beben. Die Helden wandten sich um und sahen in der Ferne, am alten Wehrturm, riesige Wolken aus Schnee und Staub aufwirbeln. Dann erhob sich ein dunkler Schatten am Horizont.

Sagrakdur war gewaltig. Er breitete seine Flügel aus, stieß in den grauen Himmel empor, und in einem weiten Bogen überflog er langsam das schneebedeckte Land.

Unzweifelhaft näherte sich das Biest der Rietburg. Doch der Wirbelsturm der Kälte über dem alten Wehrturm blieb bestehen.

„Siantari ist nicht mit den Drachenkultisten mitgezogen. Sie lauert noch am alten Wehrturm“, zischte Ijsdur.

„Falls wir sie erledigten, würde das auch den Eis-Drachen vom Himmel holen?“, fragte Aćh.

„Keine Ahnung. Aber wenn die Ermordung Siantaris den Eis-Drachen vernichtete, dann auch mich“, antwortete Ijsdur. „Ich präferierte eine längere Existenz.“

„Wir könnten zur Rietburg gehen und uns den Eis-Drachen selbst vorknöpfen“, meinte Iril.

„Oder wir reden mit Sagramak“, sprach Barz, „Sie ist eine vernünftige Person. Wenn sie will.“

Ijsdur meldete sich wieder zu Wort: „Es könnte von Vorteil sein, Siantari nur zu überwältigen und die Drachenkultisten vor ein Ultimatum zu stellen. Ihr Eis-Drache sollte sich wie alle Eis-Dämonen Siantaris Willen beugen müssen. Schließlich trägt er anders als ich keine beschützende Runenscheibe im Hals. Und wenn wir erst einmal Siantari in der Hand hätten ...“

Da sprach sich Iril dagegen aus: „Der alte Wehrturm und die Rietburg sind etwa gleich weit entfernt. Was, wenn wir uns entschließen, Siantari aufzuspüren, sie überwältigten und dann klar würde, dass sie gar keine Macht mehr über den Eis-Drachen hat? Dann hätten wir wertvolle Zeit vergeudet, in der dieser Eis-Drache was weiß ich für bösartige Bosheiten vollbringen könnte.“

Aćh konterte: „Und was, wenn wir zur Rietburg gehen, den Eis-Drachen schmelzen und uns erst danach um Siantari kümmern können? Jetzt wissen wir, wo sie sich befindet. Bis dahin könnte sie schon wieder in alle Welt geflohen sein.“

„Und einen finsteren Plan in Tag umgesetzt haben“, warf Ijsdur ein, „Ich befürchte, ihr seid zu optimistisch, was unsere Gewinnchancen angeht. Siantari ist eine formidable Gegnerin. Unterschätzt sie nicht. Und die Stärke eines Eis-Dämons im Körper eines Drachen jagt selbst mir einen Schauer über den Rücken.“

In der Ferne erklang weiteres raues Gebrüll. Sagrakdur hatte seinen Mund geöffnet und einen bläulich-weißen Strahl der Kälte und des Schnees aufs tief unter ihm liegende Rietland gespuckt.

„Gütige Mutter!“, rief Ijsdur aus.

„Seit wann glaubst du denn an die Mutter?“, fragte Iril.

„Tu ich nicht, diese Redewendung habe ich mir von dir abgeguckt.“

„So schlimm ist die Lage gar nicht“, meinte Iril, „Im Vergleich mit Tarok ist dieser Eis-Drache hier geradezu winzig.“

„Aber die Helden, die Tarok besiegten, sind im gesamten Lande verstreut und haben ihre eigenen potenziell königreich-endenden Gefahren zu bekämpfen.“

„Dann bleibt diese ganze Sache wohl an uns hängen. Wir müssen uns aufteilen. Ich verstehe nichts von Eis-Dämonen, aber so einiges von Drachen. Ich biete mich an, zur Rietburg zu ziehen und mich dem Eis-Drachen zu stellen. Siantari würde ohnehin gar nicht verstehen, was ich ihr für Beleidigungen an den Kopf werfen würde.“

„Ich komme mit dir!“, rief Barz, „Ich bin froh, wenn ich meiner Lebtag keiner Eiskristallkette mehr nahekommen muss – Anwesende ausgenommen. Und ich will eine friedliche Lösung mit den Drachenkultisten nicht aufgeben. Auf mich wird Sagramak am ehesten hören. Ohne dir nahe treten zu wollen, Iril, könnte es dafür sogar besser sein, wenn ich allein dorthin ziehen würde.“

„Willst du von Sagrakdur in einem einzigen Angriff zu einem Eisblock reduziert werden?“, fragte Iril schnippisch. „Meine Runenmagie könnte das einzige effiziente Mittel gegen ihn sein.“

Aćh räusperte sich und gestikulierte zum brennenden Feuervogel auf ihrer Schulter. „Das einzige Mittel?! Mein feuriger Turr könnte aus Sagrakdur bestimmt eine große Pfütze machen.“

„Wenn er nicht zuerst in einen Eiswürfel verwandelt wird“, murmelte Ijsdur, „Ich sähe ihn lieber auf unserer Seite.“

Aćh blickte ihn fragend an. „Unsere Seite?“

„Nun, es scheint passend, dass ich mich Siantari stellte“, erklärte Ijsdur, „Und ich wäre lieber nicht allein dabei. Irils Runen können bestimmt etwas ausrichten gegen einen Dur, aber vielleicht nicht gegen die Tari selbst. Und Hitze schmilzt kleinere Mengen Schnee erheblich schneller als große. Dass Siantari signifikant kleiner ist selbst der kleinste Drache, muss ich dir wohl nicht sagen. Alles in allem ...“

„Klingt nach einem Plan!“ Barz klatschte in seine Hände. „Ist das gut so? Aćh, kümmerst du dich mit Ijsdur und Turr um Siantari?“

Aćh nickte stumm, sah aber alles andere als glücklich aus.

Iril setzte sich prompt wieder an ihre Runenschreibe und kritzelte darauf herum. Ijsdur beobachtete sie stumm. Barz hingegen starrte Aćh an, welche an ihrer Unterlippe kaute. Rasch flüsterte er ihr etwas zu. Iril vernahm Floskeln der Entschuldigung und der Nachfrage. Wohl fragte er, ob sie wirklich einverstanden damit war, sich an Ijsdurs Seite einer Eis-Dämonin entgegenzustellen. Er hatte das Erlebnis mit Nesdora nicht vergessen.

„Fećht!“, fluchte Aćh. Sie stieß einige Sätze auf Tulgorisch aus, ganz leise und schnell. Iril hätte sie nicht einmal gut verstehen können, wenn sie ihre Übersetzungsrune aktiviert hätte. Barz hingegen verstand und machte große Augen. Seine Mundwinkel zuckten, als müsse er sich ein Grinsen verkneifen. Dann nickte er aber ernst und hörte weiter zu. Schließlich erklärte er etwas davon, dass sein Bannpulver ohnehin nicht gegen Siantari wirken könnte. Aćh unterbrach ihn erneut. Doch schließlich nickten die beiden, umarmten einander und gingen dann zu ihren jeweiligen Kameraden für die kommenden Aufgaben.

Aćh hatte nichts mehr einzupacken und brach in Richtung des alten Wehrturms auf. An ihrer Seite folgte Ijsdur, der ohnehin nie etwas einzupacken hatte. Über den beiden drehte ein fröhlicher Turr seine Runden.

Iril studierte in ihren Notizen die genaue Runenfolge, welche in Silberhall so effektiv Geister vertreiben konnte. Eine Variation davon hatte Ijsdur von Siantari befreit. Eine andere Variation hätte ihn in der Eiskristallkette einschließen können. Das hätte sie für den Eis-Drachen gerne.

Iril verschloss ihre Reisetasche und erblickte, wie Barz seinen Pulvergürtel – an dem soeben ein türkises, ein braunes und ein blaues Säcklein hingen – lange betrachtete und zu überlegen schien, ob er lieber noch einen weiteren Abstecher zu Sabri machen und seine Pulver umverteilen sollte. Dann aber zuckte er mit den Schultern und wandte sich Iril zu.

„So, dann machen wir zwei Hübschen uns auf zur Rietburg, oder?“, rief Barz. „Was ist unser Plan?“

„Was meinst du, für was unsere Zeit reicht?! Rechtzeitig ankommen und draufhauen, das ist der Plan.“, rief Iril.

„Reden!“, murmelte Barz, während er seine Schuhe enger band. „Sagramak hat meines Wissens noch niemanden umgebracht, der mit ihr verhandeln wollte. Außer ... außer diesem einen Händler, der ihr die ganze Zeit ... na, das ist eine Geschichte für ein anderes Mal.“

Das konnte ja heiter werden.\bigskip







Vor der Rietburg schwebte der Schwarze Herold bedrohlich auf einer kleinen rauchigen Sturmwolke. Das Ewige Feuer hatte sich violett verfärbt. Der lila Feuerschein flackerte um seine dunkle Silhouette und spiegelte sich in der gezackten Maske, die er auf dem Gesicht trug. Die Zeit war nah. Die meisten Helden von Andor waren in Scharmützel mit allen möglichen Gegnern verstrickt. Jetzt war der Zeitpunkt, seinen Meister zurückzubringen. Er, der Schwarze Herold, war der Antreiber der Kreaturen und der Vorbote des Feuers, und nun auch der Vorbote des Eises. Wo er auftauchte, verloren die guten Menschen von Andor den letzten Mut.

Der Herold begann zu sprechen. Er flüsterte, und doch hörte jeder in Andor seine Stimme:

„Ich verkündige die Ankunft des mächtigen, des gewaltigen Sagrakdur. Vernichtet wurde sein Körper vor Jahrhunderten. Nun kehrt er zurück und holt sich, was der Letzte seiner Art nicht konnte. Die Archive vom Baum der Lieder sollen zerbröseln, wenn mein neuer Meister seinen eisigen Atem über den Wachsamen Wald streifen lässt. Die Knochen der Zwergenbrut sollen zersplittern unter der Kälte meines Gebieters. Und auch Brandurs Sippe soll nicht verschont werden.“

Alarmglocken wurden geläutet. Rufe erklangen. Menschen eilten von Turm zu Turm, von Tür zu Tür. Die Nachricht verbreitete sich in Windeseile. Ein Drache schoss auf die Rietburg zu, schon zum zweiten Mal in diesem Jahr. Und diesmal standen keine Helden von Andor bereit, um sich ihm entgegenzustellen.

Am Ausguck wurde berichtet, dass es sich bei diesem anfliegenden Drachen augenscheinlich um ein magisches Wesen aus purem Eis und Schnee handelte. Einige Bewohner der Burg verkrochen sich in ihren Kellern. Viele flohen ins offene Rietland. Die Rietgarde rüstete sich für den kommenden Kampf.

Und über all dem schwebte der Schwarze Herold. Er genoss den Aufruhr.

Hinter ihm kündigte das Rauschen großer Schwingen die Ankunft des Eis-Drachen an. Der Herold blickte ihm stolz entgegen. Ein Meisterstück der Magie. Sein Meisterstück. Er hatte Siantari und Sagramak auf einen gemeinsamen Pfad geführt. Nun war es an seiner Zeit, zu glänzen.

Sagrakdurs Kopf schwenkte auf seinem langen Schlangenhals nach unten. Seine durchscheinende Zunge züngelte. Sah er einige fliehende Andori, welche sich in Ermangelung angreifender Kreaturen in alle Himmelsrichtungen zerstreuten? Kurz schien der Eis-Drache zu bedenken, die Fliehenden angreifen zu wollen, dann aber schüttelte er seinen Kopf und hielt direkt auf die Burg zu. Pfeile der verteidigenden Rietgarde gruben sich in seinen Bauch, doch kümmerten diese ihn nicht einmal. Bolzen von der großen Balliste der Rietburg hätten ihm wohl geschadet, doch diesen wich Sagrakdur gekonnt aus.

Ein bläulich-weißer Strahl der Kälte und des Schnees ging aus Sagrakdurs Mund auf die Balliste nieder. Sie und die Kriegerin, die sie betätigt hatte, wurden unter einer dicken Schicht aus Eis und Schnee bedeckt. Jubel ertönte von den zahlreichen Gestalten auf dem Rücken des Eis-Drachen.

„Jeder, der sich ergibt, wird verschont!“, verkündete Sagramaks schallende Stimme. „Verlasst das Gemäuer und gebt uns den Weg zum Prinzen frei!“

Ein, zwei Menschen desertierten und rannten ins Rietland hinweg. Und die Drachenkultisten schienen sehr stolz darauf, was für gute Menschen sie doch waren.

Die Kultisten. Der Schwarze Herold lachte innerlich beim Gedanken an diese selbstgerechten Knochendiebe. Sie waren nützliche Spielsteine für seine nächsten Züge. Und manche von ihnen teilten seine Überzeugungen und Ziele gar. Aber nicht alle. Viele ersuchten die Macht der Drachen, um ihre Sippen zu stärken und sich ein gutes Leben zu verschaffen. Sie waren keine wahren Diener der Drachen. Die wahren Diener, das waren der Herold und die von ihm angetriebenen Kreaturen. Und mit seiner Hilfe würden sie bald wieder einen Meister haben.

Sagramak spie weitere Eisstrahlen, bis alle Krieger des Königs sich in ihre steinernen Türme zurückgezogen oder ihr Leben mit einem eiskalten Atemzug ausgehaucht hatten.

Schwer landete der Eis-Drache inmitten der Rietburg, auf einem mit Rietgras besetzten Dach, das unter ihm nachgab. Stab und Dreck stob auf. Stolz stolziert er zu einem freien Fleck, drehte sich zur Seite und entließ die Drachenkultisten von seinem Rücken. Als allererste landete die selbstbewusste Sagramak auf dem festen Boden der Rietburg.

Sie breitete ihre Arme aus – welche einen Schild und einen zeremoniellen Speer führten – und zog tief Luft ein. „So riecht es also in der Rietburg“, murmelte sie, „Stinkt mehr, als ich erwartet hätte. Danke sehr für die feurige Ansprache, werter Herold. Nun denn, wenden wir uns der Sippe Brandurs zu. Wo ist euer Prinz? Wo hält er die Drachenknochen versteckt? Bringt uns Thorald!“\bigskip







Aćh und Ijsdur zogen gemeinsam durchs Rietland. In der Ferne war bereits der alte Wehrturm erkennbar. Weiterhin wirbelte ein Sturm von Schneeflocken und Eisschwaden um ihn herum. Er war stärker geworden. Inzwischen war es beinahe unmöglich, die Ruine des Turms zu erkennen.

Schnell stapfte Aćh voran. Ijsdur, der wie üblich eine Linie aus Schnee und Eis hinter sich herzog, blieb immer mehr zurück. Als sie sich auf einer Hügelkuppe umdrehte und erblickte, wie Ijsdur ein halbes Feld hinter ihr zurückgeblieben war, blieb sie stehen. Davonlaufen konnte und sollte sie ihm nicht.

Stumm holte Ijsdur zu Aćh auf und lief einige Schritte neben ihr her. Sie hoffte, dass er einfach stumm bliebe. Aber natürlich tat er das nicht.

„Das wäre wir uns damals im Hängeschiff nicht im Traum eingefallen, dass wir ein paar Jahre später in einem fremden Land jenseits des Kuolema-Gebirges Seite an Seite kämpfen würden, oder?“

Aćh antwortete nicht.

„Tut mir leid, dass du dich mit mir abfinden musst, Aćh. Ich sehe, dass dir meine Gegenwart nicht behagt. Ich befürchte, dass es für das Wohl des Landes sein muss.“

Stille.

„Ich bin nicht wie diese andere Eis-Dämonin, die du getroffen hast. Ich bin meine eigene Person. Du hast keinen rationalen Grund, mich derart nicht zu mögen.“

War das nun eine Anschuldigung? Schuldgefühle machten sich in Aćh breit. Ihre Familie hatte sie erzogen, möglichst höflich und doch offen zu sein.

„Es ist nicht so, dass ich dich nicht mag. Es ist nur so, dass ich für Feuer stehe und du für Eis. Das harmoniert einfach nicht.“

„Feuer und Wasser werden oft als Gegensätze beschrieben. Mit dieser Hüterin der Flusslande und ihrem Wassergeist schienst du dich dennoch bestens zu verstehen.“

„Tja, bei ihr fühle ich mich sicherer. Ich habe schon schlechte Erfahrungen mit Eiskristallketten gemacht. Und wenn ich allein mit dir unterwegs bin, müsstest du nicht einmal eine passende Gelegenheit abwarten, um mich zu überwältigen und dem Gefolge Siantaris anzuschließen.“

„Meine Eiskristallkette funktioniert nicht mehr so. Und du hast ja Turr. Wenn ich dir gefährlich werden sollte, kann er mich in Grund und Boden schmelzen“, grinste Ijsdur.

Ein gezwungenes Lächeln spannte sich über Aćhs Gesicht.

„Und wenn ich etwas Übles tun sollte, nähmen Iril und Barz bestimmt meine Verfolgung auf und würde mir mit ihrem mächtigen Hammer den Garaus machen.“

„Da fühle ich mich doch schon ganz sicher.“

„Bin mir nicht sicher, ob du das ironisch meinst, aber das sollte es wirklich. Vielleicht vertraust du eher auf meinen Selbsterhaltungstrieb als auf meine Freundlichkeit. Was auch immer dich des Nachts ruhiger schlafen lässt.“

„Ich werde ruhig schlafen können, wenn Siantari dieses Reich verlassen hat.“

Stumm bewegten die beiden Helden sich weiter in Richtung des alten Wehrturms.

Diesmal war es Aćh, die die Stille brach.

„Sag mal, Ijsdur: Du hast deinen Körper schon quasi aus dem Nichts wieder zusammengesetzt. Kannst du dich nicht auch einfach in einen Eis-Drachen verwandeln? Auch ein sehr kleiner Drache würde schon immens helfen.“

„Verzeih mir, Aćh“, ließ Ijsdur seinen Kopf hängen, „Aber das ist nicht so, wie es funktioniert. Ich habe schon versucht, Anpassungen an meinem Körper vorzunehmen. Mental und physisch. Einmal habe ich sogar buchstäblich Wasser aus einem Brunnen genommen und an mich gepatscht, auf dass es in der gewünschten Form dort gefrieren könne. Nicht einmal Stacheln konnte ich mir geben.“

„Zu schade.“

„Vermutlich hat es mit meinem Selbstverständnis zu tun. Von irgendwoher muss der Schnee und der Eis in mir ja wissen, welche Form er zu haben hat – und nicht von meinem Leibe, denn dieser war erheblich weniger muskulös, hatte keine spitzen Ohren und kein Geweih. Das scheint sich auch auf Waffen zu beziehen. Ich kann mir ein magisches Eisschwert rufen, aber keine ausgefeilte Arcuballiste. Am Ende fühle ich mich wohl einfach wie ein Schwertkämpfer und kein Fernkämpfer. Leider scheint es, dass ich kein Eis-Drache sein werde, solange ich mich nicht als Eis-Drache sehe.“

„Zu schade“, wiederholte Aćh.

Warum hatte sie nicht mit Barz und Iril zur Rietburg mitreisen können? Warum hatte man sie hier mit dem elenden Eis-Dämon auf der Jagd nach Siantari zurückgelassen? Eigentlich wusste sie die Antworten auf diese Fragen. Aber dies bedeutete nicht, dass es ihr gefallen musste.

Ijsdurs Anwesenheit ließ sie weiterhin frösteln, und nicht nur wegen der Kälte, die stets von ihm ausging.

„Siantari ist eine mächtige Dämonin. Wir werden sie nicht mit schierer Stärke überwältigen können“, gab Ijsdur zu bedenken.

„Was ist mit Feuerbällen eines Feuertakuri?“

„Ich weiß es nicht. Ich würde lieber nicht unsere Zukunft darauf wetten.“

Aćh kaute an ihrer Unterlippe. „Was wäre mit einem feurigen Seil? Ich habe noch einige Phiolen Takuri-Asche bei mir. Eine Dose Sufarsaft, die einen vor Hitze und Kälte gleichermaßen beschützt. Und ein reißfestes Seil der Meraminenarbeiter. Wir könnten das Seil mit Saft und Asche behandeln. Es würde sich bei Kontakt entzünden, selbst aber unversehrt bleiben. Wenn man damit keine Eis-Dämonin fesseln kann, esse ich einen Besen.“

„Was bin ich froh, dich an meiner Seite zu haben!“, rief Ijsdur. Seine Stimme blieb dabei wie meistens monoton. „Doch werden wir nicht einfach mit einem Seil zu Siantari aufmarschieren und sie gefangen nehmen können. Und pure Gewalt dürfte gegen ihren erstarrenden Blick kaum helfen. Es scheint mir klüger, eine List anzuwenden.“

„Eine List?“

„Noch denkt Siantari, ich wäre auf ihrer Seite. Lass mich dich gefangen nehmen. Nur zum Schein. Ich bringe dich zu ihr und erzähle ihr etwas davon, wie du der Schlüssel dazu bist, das ewige Eis über diese Lande zu verbreiten.“

„Siantari dürfte sich wundern, dass sie dich nicht mehr kontrollieren kann.“

„Auch das können wir vielleicht dir und deiner Feuermagie in die Schuhe schieben. Davon abgesehen wissen wir nicht einmal, wie es sich anfühlt, Herrin aller Eis-Dämonen zu sein. Vielleicht merkt sie nicht einmal aktiv, dass ich ihrer Kontrolle entschlüpft bin.“

„Und vielleicht muss sie sich nur stark genug auf dich konzentrieren, um Irils Runen zu überwinden und dich wieder unter ihren Willen zu kriegen. Dann hätten wir ihr prompt uns beide ausgeliefert. Das kann doch kaum das Ziel sein.“

„Natürlich nicht. Aber es hilft unserem Ziel auch kaum, blind mit zwei Schwertern auf sie zuzustürmen und sie zu erledigen hoffen.“

Stille.

„Wenn du meinst“, murmelte Aćh. Sie kramte ihre Dose Sufarsaft hervor, träufelte eine gehörige Prise Takuri-Asche hinein und rieb ihr Seil in der Flüssigkeit ein. Anschließend schwang sie es an ihre Hüfte, wo es hoffentlich möglichst natürlich zu ihrer zeremonielle Rüstung passte.

Ijsdur nahm Aćhs Schwert an sich und hängte es an seine Hüfte, an der sich auf magische Art ein passender Gurt mit Scheide formte. Dann hielt er Aćhs Arme fest, als wäre sie seine entwaffnete Gefangene, und schob sie vor sich hin.

So bewegten die beiden sich weiter.

„Da vorne!“, rief Ijsdur. Aćh konnte noch nicht gut erkennen, was genau er meinte. Sie hatte die riesige Säule aus umherwirbelndem Schnee und Nebel aber ebenfalls schon erspäht. Der Alte Wehrturm war kaum sichtbar. Falls das Gemäuer nicht schon beim Kampf gegen die Kälte eingestürzt wäre, hatte Siantari ihm nun definitiv den Rest gegeben. Worauf wartete sie nur?

„Da vorne ist Siantari“, wiederholte Ijsdur. „Ich sehe ihre Silhouette. Sie thront oben auf der Ruine des alten Wehrturms. Nun, inzwischen ist es eher ein Schneehügel. Ausgestreckte Arme. Sie befiehlt den Sturm. Und sie ist nicht allein.“

„Drachenkultisten? Etwa gar ein weiterer Drache?“

„Ich glaube nicht. Das sind Gefangene! Mindestens zehn Personen. Klein. Kinder? Sie sitzen in einer Reihe weiter unten am Hügel. Da!“

Ijsdur zeigte in den Schneesturm hinein, doch Aćh konnte bloß dunkle Schemen erkennen.

„Jetzt kusch, Turr! Warte auf deinen Moment!“, rief Ijsdur, und wedelte mit den Armen. Turr gurrte protestierend, flatterte dann aber davon, hoch in den Himmel, in die dunklen Wolken, in denen er hoffentlich nicht einmal von den scharfen Augen eines Dämons erspäht werden konnte.

Ein mulmiges Gefühl machte sich in Aćh breit. Ohne Turr kam sie sich so alleingelassen vor.

Hier lief sie, waffenlos, ungeschützt. Kurz davor, einer bösartigen Eis-Dämonin entgegenzustehen, die sie mit einer raschen Berührung zu einem ihrer Diener machen könnte. Schaudernd dachte Aćh daran, wie sich Aćhdora angefühlt hatte.

Der neben ihr stehende Eis-Dämon hatte versprochen, nicht auf Siantaris Seite zu stehen. Konnte sie ihm wirklich vertrauen? Turr hatte Ijsdur noch nie gemocht. Was, wenn dies die ganze Zeit sein Plan gewesen war? Aćh und Turr an Siantari auszuliefern und die Gefahr des Feuers auszuschalten? Tenaya die Feuerwächterin und Lifornus der Feuerzauberer, ja selbst Trieest der Feuerkrieger befanden sich weit weg von hier, das hatte Eara bei der letzten Heldenversammlung mitgeteilt.

Kalter Wind heulte und klatschte Schneeflocken gegen Aćhs Gesicht. Ihre Schritte wurden schwerer. Ihre Stiefel sackten immer tiefer in den lockeren Schnee ein. Ijsdur hingegen wurde beschwingter. Er lief voran, ohne in den Schnee einzusinken.

Sie hatten den Schneesturm erreicht. Inzwischen konnte Aćh die Silhouette des alten Wehrturms wieder besser erkennen. Neben dem beständigen Rauschen des Sturmwinds in dieser klirrenden Kälte glaubte sie auch, Schmerzensschreie zu hören.

Tappte Aćh soeben in eine Falle? Ihr Atem beschleunigte sich. Nein, redete sie sich ein, das waren nur ihre Sorgen, die sich verselbstständigen. Wenn schon, hätte Ijsdur doch lieber Iril ausgeschaltet, ihre Runenmagie schein die größte Gefahr für die Eis-Dämonen. Naja, außer die Runen funktionierten gar nicht und Ijsdur täuschte ... stopp, sie durfte diesen Gedanken nicht länger nachhängen. Ijsdur hatte ihr Leben gerettet, schon mehrmals. Und er hatte nie ein Anzeichen bösartigen Willens gezeigt, während Nesdora damals schon innert Kürze ihr wahres Gesicht gezeigt hatte. Doch was war mit diesen gefangenen Gestalten vor Siantari? Und was war mit diesem Wirbelsturm?

„Ijsdur, können wir kurz anhalten?“, fragte Aćh, „Ich muss meine Gedanken sortieren.“

„Jetzt hat uns Siantari vielleicht schon erspäht. Wenn wir anhalten, gefährdet dies unsere Tarnung.“ Ijsdur zog an Aćhs Arm. Sie stolperte weiter.

„Ijsdur, halt an. Ich habe kein gutes Gefühl bei der Sache. Wirklich nicht.“

„Umso besser, das wirkt viel authentischer.“

„Ijsdur, lass mich los!“

Aćh riss sich los von Ijsdur. In einer geschmeidigen Bewegung riss sie ihr Schwert von seinem Gurt und taumelte einige Schritte zurück. Blut pochte in ihren Ohren. Das Feuer der Furcht raste durch ihre Adern. Ijsdur blickte sie aus kalten, schneeweißen Augen an. Genau wie Nesdora es getan hatte. Ijsdurs Eiskristallkette glitzerte und glomm schwach umgeben von all diesem Schnee. Furcht überkam Aćh. Sie wirbelte herum und nahm ihre Beine in die Hand.

Im Heulen des Windes hörte sie nicht, ob Ijsdur sie verfolgte. Eine Schneewehe gab unter Aćhs Stiefeln nach. Sie kippte zur Seite, verwickelte sich in ihrem langen Umhang und kullerte unfreiwillig in eine Kuhle. Es war nass und kalt und nass. Bäh! Sie verschnaufte und verfluchte zu gleichen Teilen sich selbst, Ijsdur und Siantari.

Dann schob sie ihren Kopf leicht über die Anhöhe und blickte dorthin, woher sie gekommen war. Der Sturmwind machte es schwer, irgendetwas zu erkennen. Immerhin würde er auch ihre Spuren rasch verwischen.

Da, eine Gestalt! Gehörnt und zu Fuß unterwegs. Ijsdur trat näher, bis er nur noch einige Mannslängen von Aćh entfernt war. Er sah sich ratlos um, blickte allerdings weit über Aćhs aus der Senke hervorlugenden Kopf hinweg.

Doch er war nicht allein. Der Sturm nahm an Stärke zu, als zweite Gestalt in den Fokus kam. Auch sie war gehörnt, doch lief sie nicht, sondern schwebte mit ausgestreckten Armen auf einem Kissen aus Luft. Ihr langes Kleid flatterte im Wind.

Eisblaue Augen richteten sich auf Ijsdur und eine Stimme erklang, so klar und so kalt wie Eis, und noch viel klirrender als Ijsdurs.

„Sieh an. Hat es doch noch ein anderer Eis-Dämon bei Verstand durchs Felsentor geschafft. Der verlorene Sohn kehrt zurück.“

Ijsdur drehte sich um und fiel vor Siantari auf das Knie. „Herrin! Endlich habe ich Euch wiedergefunden!“














\newpage
\section{Die eisigen Drei}


Sagramak klopfte gegen das schwere Holztor, das den Zugang zum Thronsaal verwehrte. Sie war königlicher Laune.

„Hallo? Jemand zuhause?“, rief sie munter. „Werter Prinz? Ein Vögelein zwitscherte mir, dass Ihr allein wüsstet, wo Taroks letzte Knochen versteckt sind. Die Teilung dieses Wissens könnte Eurem Volk allerlei Leid ersparen.“

Als keine Antwort erklang, schlug sie mit ihrem Speer tiefe Furchen ins Holz. Die massive Tür gab nicht nach.

„Ein wenig Hilfe hier?“

Sie bekam keine sofortige Antwort, nur Waffengeklirr. Ein Blick verriet, dass hinter ihr erneut Gefechte ausgebrochen waren. Einige Krieger des Königs hatten sich aus ihren Türmen getraut und den wenigen anwesenden Drachenkultisten entgegengestellt. Warum ergaben sich diese Idioten nicht einfach? Sie mussten doch sehen, dass sie gegen einen Eis-Drachen nichts ausrichten konnten.

Wobei der Drache nicht mal kämpfte, sondern nur zuschaute, was seine Anhänger taten. Nehamal hatte seinen Degen ausgepackt und demonstrierte sein Geschick im Umgang mit der edlen Klingenwaffe. Fir sprang hektisch umher und verteilte hirnerschütternde Huftritte, gleichzeitig laut über die ihm entgegengebrachte Animosität protestierend. Der Schwarze Herold schwebte über den Kämpfenden und stach nach unbehelmten Kriegerköpfen. Doch waren sie zahlenmäßig weit unterlegen. Schon sah Sagramak einen Skral niederfallen und nicht mehr aufstehen.

Was zum Himmel dachte sich Sagrakdur? Nutzlos stand er über den Scharmützeln und blickte umher, als könnte er sich nicht entscheiden, wohin er seinen Eisstrahl richten sollte. Der unerfahrene Eis-Drache hielt nicht, was der Herold versprochen hatte.

„Links von dir!“, brüllte Sagramak. Sagrakdur wirbelte herum. Eine Kriegerin mit einer eisernen Lanze preschte auf einem edlen Schimmel auf Sagrakdur zu. Die Waffe grub sich tief in seinen Brustkorb. Sagrakdur kippte zur Seite und riss die Kriegerin von ihrem Pferd. Er sammelte seinen Atem und sortierte seinen Schneekörper. Dann erhob er sich auch schon wieder und richtete seinen stacheligen Schädel auf die Angreifer.

Der Mensch blieb wie erstarrt stehen. Das Pferd suchte sein Heil in der Flucht. Er würde sie beide erwischen. Der glorreiche eisblaue Drache richtete seinen Schlangenhals auf und spürte verzückt das vertraute kalte Brodeln, wie es von seinem Magen langsam seinen Hals hinaufkletterte und seinen Schlund erreichte. Sagrakdur blähte seine Nüstern, öffnete seinen Rachen und grinste in Antizipation dessen, was er diesem mickrigen Spornfraß vor sich gleich antun würde. Dann spie er einen Strahl puren magischen Eises, bläulich glänzende Wirbel, denen nichts und niemand standhalten konnte.

Nichts außer einem Schild.

Ein weiterer Krieger – ein ganz junger diesmal – warf sich zwischen den Eisstrahl und die vor Furcht erstarrte Lanzenträgerin. Er wuchtete einen Buckelschild zwischen sich und Sagrakdur. Der Eisstrahl traf den Schild und überzog dessen Oberfläche mit Reif, ließ den sich dahinter kauernden Jungen jedoch unversehrt. Der blaue Schneewirbel verpuffte. Verdutzt klappte Sagrakdur seinen Mund wieder zusammen.

„Für Prinz Thorald!“, schrie der Junge mit einer viel zu hohen Stimme. Der Schwarze Herold wirbelte zu ihm und stach ihn nieder, mit solch entsetzlicher Wucht, dass er mehrere Meter weit fortgeschleudert wurde.

„Hierher, Sagrakdur!“, schrie Sagramak und gestikulierte zum Tor des Thronsaals. Sagrakdur schlängelte seinen eisigen Körper zu ihr. Er drückte das Tor mit einer Pranke ein, aus bestünde es aus dünnem Pergament.

Schreie ertönten. Andori, die sich im Palas versteckt hatten, drückten sich noch tiefer in die Ecken des Raumes. Sagramak starrte ins Dunkel. Da, an der Wand, zwischen zwei Säulen hing ein Gemälde, das drei Personen zeigte. König und Königin, mit einem jungen schwarzhaarigen Burschen, welcher zwischen seinen beiden Eltern stand und stolz in die Welt hinausschaute. Dies war Thorald.

Doch der erwachsene Prinz war nirgendwo zu sehen. Und auch keine Drachenknochenkette. Sagramak fluchte.\bigskip







Schon aus der Ferne war der flackernde lila Lichtschein des ewigen Feuers erkennbar.

Einige flüchtende Andori kreuzten den Weg von Iril und Barz. In alle Richtungen rannten sie von der Rietburg weg, ihr wichtigstes Hab und Gut – oder schlicht Proviantrationen – in Bündeln mit sich schleppend.

Sie achteten kaum auf die beiden Helden, die die großartige Idee hatten, der besetzten Burg näher zu kommen.

Barz löste sein türkises Vorhersehungspulver aus Silberblumenblütenpulver von seinem Gürtel und streute sich eine Prise davon auf die ausgestreckte Zunge. Er schmatzte ein, zweimal und schrie kurz auf, als seine Augen blendend hell aufleuchteten. Dann ächzte er auf und legte seine offene Hand auf seine Brust, Handfläche nach außen. Ein Stoßgebet an die Götter.

„Sie sind echt stark. Dieser Eis-Drache ist ein gewaltiger Gegner. Und auch die Kultisten selbst verfügen über verblüffende Kräfte. Sie zu überwinden wird beinahe unmöglich.“

„Keine andere Wahl“, knurrte Iril. Die beiden rasten den Hügel hinauf und unter dem Torbogen in die Rietburg hinein. Kampfeslärm erwartete sie. Schreie und Waffengeklirr. Tote und Verwundete, Krieger unter purpurroter Fahne und Kämpfer mit auf ihren Gesichtern gemalten Zeichen der Drachen.

Ein Krachen ertönte. Der blauweiße Eis-Drache mit dem gehörnten Kopf trat vom Palas zurück. Er hatte ein tiefes Loch im Tor hinterlassen. Immerhin war es kein Feuerdrache, unter dessen Atem die Rietburg mit ihren Rietgrasdächern sofort in Flammen aufgegangen wäre.

Die Drachenkultisten waren den Kriegern der Rietburg in Anzahl ziemlich unterlegen. Ihre Vertreter wurden soeben zusammengetrieben. Doch wie lange konnten die tapferen Andori diesen Vorteil in Anwesenheit eines Eis-Drachen halten? Manche hatten aus ihren Speeren Fackeln gebastelt und fuchtelten damit in Sagrakdurs Richtung, als ob er sie nicht mit einem einzigen Eisstrahl löschen könnte.

Iril erkannte das Ziegenwesen Fir, das soeben vor dem jungen Krieger Malin zurückwich.

Sie sah Nehamal, der eine in einen langen Mantel gekleidete Frau zu Boden warf, ehe sie von einem antrampelnden Pferd überrannt werden konnte.

Doch wo war ... da war sie! Sagramak die Schamanin, Anführerin der Angreifer, stand oben am Palas, stolz über das Geschehen hinwegblickend, und rief: „Dort drüben, Sagrakdur! Vernichte die Törichten!“

Sagrakdur richtete seinen langen Schlangenhals auf und ließ einen Schwall aus Eis und Schnee auf die Krieger von Andor niederregnen. Hatten sie vorhin gerade noch das Dutzend anwesender Drachenkultisten zurückgetrieben, waren sie nun selbst gezwungen, aus dem Weg zu springen und zurückzuweichen. Nur diejenigen, die direkt in einen Zweikampf mit einem Kultisten verwickelt waren, blieben vor dem tödlichen Odem gefeit.

Ein Pfeil schoss aus einem Hausfenster heraus und traf Fir in der Brust. „Das ist das Ende für dich, o Fir!“, heulte das Ziegenwesen, „Ein so kurzes Leben, und so unerfüllt! Was für eine wildere Welt du verlässt als die, in die du hineingeboren wurdest.“ Dann sackte Fir zusammen.

Sagrakdur fuhr herum und spie einen weiteren gewaltigen Eisstrahl. Eisklötze, groß wie Zwerge, trafen das Tor über Irils Kopf, zerbarsten und regneten als grobkörniges Pulver auf sie herunter.

„SAGRAMAK!“, brüllte sie, „Gebiete deinen Drachen, einzuhalten. Lass uns reden!“

„Mit denen lässt sich nicht reden!“, rief eine grimmige Stimme von der Mauer herunter, wo Iril den mürrischen Zwerg Lafgar erkannte.

Iril starrte ihn böse an und gebot ihm, zu schweigen. Doch Sagramak reagierte tatsächlich nicht auf ihren Ruf. Still und stur schaute sie zu, wie Sagrakdur ein weiteres Haus dem Erdboden gleichmachte. Staub wallte hervor.

Da rannten Menschen aus dem zusammenbrechenden Haus hervor. Da war Nabib! Er zog den alten Heiler Readem hinter sich her.

Barz erstarrte, als er Nabib erkannte.

Nabib erstarrte, als er Barz erkannte.

„Was ... nein ... oh, du Idi ... bring dich in Sicherheit!“, stammelte Nabib.

„Niemals!“, rief Barz, seinen Bogen ziehend. Der Eis-Drache drehte sich zu ihm um. Bläulich schimmerte die Kehle des Drachen auf, als er Kraft für seinen Eisatem sammelte.

„NEIN!“, rief Nabib, „Kommt nicht in Frage! Guck hierher, du dummes Schneeding! Renn schon weg, Barz!“

Der Drache schwenkte seinen geöffneten Kiefer wieder in Richtung Nabib. Hinter ihm rannte Heiler Readem hastig davon und verkroch sich hinter einer umgekippten Steintafel.

„NEIN!“, rief Barz nun und rannte nach vorne, um sich zwischen den gewaltigen Kiefer der eisigen Echse und Nabib zu werfen. Doch anstatt den Drachen anzublicken, drehte er sich in Richtung Sagramak, die das Geschehen amüsiert beobachtete.

Schwer atmend rief er: „Sagramak, lass gut sein, bitte. Ihr habt eure Macht demonstriert. Ihr müsst keinen weiteren Schaden anrichten.“, rief Barz. Über ihm knurrte der Eis-Drache weiterhin bedrohlich. Doch als Sagramak stolz ihre Hand hob, ließ er fügsam von den beiden ab.

„Oh, das wird amüsant“, lachte die Schamanin schallend. „Ich dachte nicht, dass du auch hier wärst. Und jetzt willst du reden? Wolltet ihr beim letzten Mal nicht auch nur mit Gewalt bestimmen, wer welches Schicksal erhält?“

Während Barz und Nabib darum rangelten, wer sich schützend vor den anderen stellen durfte, stapfte Iril den Weg zum Palas hoch. Auch wenn man kaum mit Sagramak verhandeln konnte, war es geschickt, den Weg zu ihr zurückzulegen, ehe der gewaltige Eis-Drache ihr da einen Strich durch die Rechnung machen konnte.

Blutgetränkter Schnee knirschte unter ihren Stiefeln. Um sie herum kamen die Zweikämpfe zwischen Kultisten und Kriegern langsam zu einem Halt. Manche, weil ein Kontrahent einen anderen so weit verletzte, dass er nicht weiterkämpfen konnte. Andere, weil die Kämpfer eine Gelegenheit sahen, nicht weiter nacheinander zu stechen, und ihre Aufmerksamkeit auf das Geschehen vor ihnen zu lenken.

Iril hielt Sagramaks Aufmerksamkeit und wählte ihre Worte mit Bedacht. „Ich wollte dir damals nicht schaden, und will es immer noch nicht. Doch muss man kleine Übel annehmen, um größere zu verhindern.“

„Nun, Iril, um ganz offen zu sprechen“, meinte Barz von weiter hinten versöhnerisch, „Vielleicht wäre gar nichts geschehen, wenn wir ihnen die Knochen einfach überlassen hätten. Es war doch nur ihr neu entfachter Hass, der sie dazu trieb ...“

„Der Schwarze Herold hätte sie bestimmt dazu anstacheln können! Bitte, stell dich nicht auf ihre Seite.“

„Es geht nicht um Seiten!“, rief Barz, „Barbaren oder Andori. Drachenkultisten oder Bauern. Wir alle sitzen im selben Boot. Wir alle wollen Frieden.“

„Wenn ich mich einmischen darf: Wir sitzen nicht ganz im selben Boot“, grinste Sagramak vom Hügel herab. „Einige von uns sitzen auf einem mächtigen Eis-Drachen.“

„Es geht euch um Taroks Knochensplitter, oder?“, versuchte Barz mit erhobenen Händen, die Situation weiter zu bereinigen, „Ich bin mir relativ sicher, dass Thorald sie euch nun mit Freuden aushändigen würde, wenn ihr ihn nicht weiter belästigt.“

„Ja, darum geht es. Die Knochen, die du uns bei unserem letzten Treffen so unhöflich entwendet hast. Wo sind sie?“

„Ich weiß es nicht“, gab Barz zu.

Ein Pfeil schoss aus einem Turm der Rietburg hervor und schrammte an Sagramaks Rüstung ab.

Das brach den Bann. Die Zweikämpfe im Burghof brachen wieder aus. Nehamal schrie etwas Unverständliches und stach mit seinem Degen einen Zwerg vor ihm ab. Der Eis-Drache brüllte auf und sprang den Turm an, aus dem der Pfeil geschossen worden war. Sein Schwanz zuckte haarscharf an Barz vorbei und schleuderte Nabib gegen einen Felsen. Barz hechtete ihm nach.

„Siehst du, wie viel das Reden gebracht hat?“, knurrte Iril mehr zu sich selbst. Sie schwang ihren Hammer und legte mit raschen Schritten die letzten Meter zwischen sich und Sagramak zurück. Diese lachte nur schallend.

Der Runenhammer zischte pfeifend durch die Luft. Statt auf harte Rüstung zu treffen, streifte er nur das metallene Blatt am Ende von Sagramaks Speerwaffe. Blitzschnell wirbelte Sagramak umher und trieb Iril zurück. Bei der Mutter des Steins, die Schamanin war stark!

Furcht machte sich in Iril breit. Furcht, sich überschätzt zu haben. Furcht, ihre Chance soeben verspielt zu haben. Lähmende Furcht, die in einem Kampf nichts zu suchen hatte. Iril war klein und hatte nur einen Hammer. Sagramak war überragend groß und führte eine lange Stangenwaffe, mit der sie Iril locker auf Distanz halten konnte. Wann immer Iril ihr näher zu kommen versuchte, war der Speer bereits dort. Es ermüdete sie, den Stichen und Schnitten auszuweichen. Lange würde sie nicht mehr durchhalten können. Dieser Speer musste gehen, und zwar jetzt!

Iril gönnte sich eine Verschnaufpause und trat einige Schritte zurück. Sie hatte schon einen Schnitt in der Stirn, dessen Bluttropfen ihr die Sicht erschwerte. Die weiterhin grinsende Sagramak war hingegen unversehrt. Sie umkreisten einander. Iril atmete tief durch. Fasste ihren Runenhammer fester. Spürte, wie er unglaubliche Energie aus der Umgebung aufsaugte. Ein riesiger magischer Drache war anwesend, das musste doch für etwas gut sein.

Es war Iril, die als erste wieder nach vorne sprang und dem Kampf seinen Rhythmus gab. In rascher Abfolge kamen ihre Schläge, und nicht mehr auf Sagramak selbst gerichtet waren sie, sondern auf ihren Speer. Erschütterung nach Erschütterung traf den Speer und rüttelte am Griff der ihn Führenden. Das Grinsen der Schamanin erstarb. Link, rechts, links, links, schlug Iril darauf los, stets darauf achtend, sich von ihren Schwüngen nicht zu weit treiben zu lassen. Und noch immer hielt sie die Magie im Runenhammer fest.

Sagramak stolperte. Die Speerspitze streifte den Boden. Iril hob ihren Hammer hoch und ließ ihn mit aller Kraft auf den Speer niederfahren. Die Magie der Runen entlud sich. Grünes Feuer blitzte den Speer entlang. Es krachte und splitterte.

Sagramak stürzte zu Boden. Iril fegte ihren glänzenden Helm von ihrem Kopf.

„Ergib dich! Pfeif deinen Eis-Dra ...“, setzte sie an. Sagramak brüllte bloß auf, sprang nach vorne und trieb Iril die abgebrochene Stange ihres Speers in den Magen. Nun war es an Iril, zurückzustolpern. Sagramak stieß die zweite Hälfte ihres Speers mit einem Stiefel in die Höhe und fasste sie elegant. Mit zwei halben Stangenwaffen bewaffnet, jagte sie Iril zornerfüllt nach und ließ ihr keine Ruhe. Hatte sie sich zuvor zurückgehalten?

Der Runenhammer kam nicht nach. Stockschläge trafen Irils Schulter und Arm. Sie verlor ihren Hammer und kam nicht einmal dazu, ihre Rückrufrune zu aktivieren, da gab auch schon der Boden unter ihr nach.

Ein letzter Stoß Sagramaks ließ Iril rückwärts den Abhang hinunterkullern, von der Höhe des Palas‘ bis zu den tief darunter liegenden Steinen des Pferdepfads. Es knirschte unschön in ihrer Reisetasche. Iril hoffte, dass ihre Runenscheibe noch ganz war. Hoch über ihr am Hügel stand Sagramak und guckte selbstverliebt auf sie herunter.

Iril versuchte sich aufzurichten, kitzelte sie eine Klinge am Rücken. „Ergib dich“, klang eine schnarrende Stimme. „Die Drachen wollen deinen Tod nicht. Du hast Platz in ihren Plänen.“

Iril hob ihre Hände und drehte sich langsam herum. Hinter ihr stand Nehamal, der Degenträger, der vor ihr in die Hocke gegangen war und ihr nun seinen Degen an die Kehle hielt.

Ein Blick hinter sich zeigte, dass es um ihre Gefährten nicht besser stand: Von den wenigen im Hofplatz verbliebenen Kriegern des Königs bewegte sich keiner mehr. Der Rest hatte sich wohl verschanzt oder das Weite gesucht. Die Scharten, aus denen die Bogenschützen der Andori hatten Pfeile herabregnen lassen, waren mit einer dicken Schicht aus Eis besetzt. Der Eis-Drache Sagrakdur hatte sich auf den Rücken gedreht und wälzte sich wohlig auf der Ruine eines Hofhauses. Bei allen Kreaturen der Tiefe, wo war Barz?!

Abrupt sprang Iril zurück. Nehamal reagierte zu spät. Sein Degenstoß ging ins Leere. Iril atmete tief durch, sammelte sich und aktivierte ein Runentattoo, das den Runenhammer von Golja zu ihr zurückrufen sollte. Nichts geschah. Sie fluchte, als sie erkannte, dass der gewaltige Schatten Sagrakdurs auf sie fiel und das magische Sonnenlicht von ihr abschirmte. So hatte sie keine Möglichkeit, ihre Tattoos ohne Runenhammer zu aktivieren. Doch noch war sie nicht waffenlos.

Iril zog eine bestimmte dünne metallene Runenscheibe aus ihrer Reisetasche hervor, die für genau solche Zwecke geeignet war. Sie brauchte weder direktes Mond- oder Sonnenlicht noch einen magischen Hammer, sondern konnte sich durch Rotationen mit Energie speisen. Iril drehte mit schnellen Bewegungen ihre Runenscheibe, ihre Augen begannen vielfarbig zu glühen und bunt leuchtende Runen stiegen langsam aus dem Boden. Bunte Schlieren überdeckten ihr Sichtfeld, doch nahm sie noch wahr, wie Nehamal vorsichtig zurückwich. Er hatte wohl noch nie jemanden Runen in die Luft zeichnen sehen und konnte deren Gefahrengrad nicht einschätzen.

Vorsichtig tippte Nehamal mit dem Degen eine Luftrune an und sah interessiert zu, wie sie sich unter dem Stich wellte und verpuffte.

Zu schade, dass dies ihm den Eindruck geben musste, die Runen wären ungefährlich. Wenn er nur lange genug stillhalten würde, könnte Iril ihn problemlos mit solchen Runen fesseln. Aber nun würde er kaum still stehen bleiben. In Ermangelung einer anderen Waffe schleuderte Iril weitere Runenscheiben aus ihrer Reisetasche wie Frisbees Nehamal entgegen.

Erfolglos. Effekte hatten diese natürlich nicht und außer einem Schnitt an seiner Wange kümmerte Nehamal sich nicht um die heranregnenden Scheiben.

Mit drei Schritten war er bei Iril angekommen und griff nach ihr.

Iril sollte eigentlich stärker sein als ein dahergelaufener Mensch. Dennoch überrumpelte er sie mühelos. Ein leises Grollen drang aus Nehamals Kehle und sein Griff um Irils Handgelenke versteifte sich noch mehr. Wie ein Schraubstock hielt er sie fest.

„Du dachtest, du kommst einfach so davon? Du hast nicht die leiseste Ahnung, mit wem du dich eingelassen hast und welche Kräfte mir zur Verfügung stehen.“\bigskip







Barz öffnete die Holztür mit einem Tritt und schleppte den bewusstlosen Nabib über die Schwelle. Dieses Haus stand am Rand der Rietburg, direkt mit der Wehrmauer verbunden. Natürlich war aktuell kein Ort hier im Riethof komplett sicher, aber dieser hier bot zumindest relative Sicherheit.

Barz ließ Nabib sanft auf eine Decke gleiten und atmete durch. Nabib war in Sicherheit. Zeit, zum Kampfgeschehen zurückzukehren

Da bemerkte Barz das Augenpaar, das ihn aus einer Ecke beobachtete. Er traute seinem eigenen Augenpaar kaum.

Ein großer Krieger mit stahlblauen Augen blinzelte Barz wortlos entgegen. Und hinter ihm saß Thorald, Prinz und künftiger König von Andor, zusammengesunken in einer Ecke der Hütte und zitterte. Nicht nur das, er hatte eine Flasche in der Hand, deren Geruch ihren Inhalt verriet. Ein schöner Anführer war das.

„Achtet auf ihn, ja?“, fragte Barz laut und deutete auf den bewusstlosen Nabib. Thorald reagierte nicht, doch Armond nickte. Barz schüttelte seinen Kopf und verließ das Haus wieder. Es war wohl geschickter, Thorald nicht noch weiter in das Geschehen zu verwickeln.

Draußen schlug ihm auch schon wieder die Kälte des allgegenwärtigen Schnees entgegen. Barz versuchte, sich einen Überblick über die aktuelle Lage zu verschaffen.

Der Kampfeslärm war größtenteils verstummt. Von den andorischen Kriegern, die im offenen Riethof herumlagen, rührte sich keiner mehr. Die Scharten, aus denen Verstärkung die sich neu sammelnden Drachenkultisten hätten beschießen können, waren vereist worden.

Der Eis-Drache Sagrakdur balancierte auf einem hohen Steinturm und warf seinen langen Schatten auf ... da vorne! Iril kämpfte gegen Nehamal. Drei andere, teils weniger, teils schwerer verletzte Kultisten näherten sich den beiden aus verschiedenen Richtungen. Unschöne Lage.

Nicht weit von Barz entfernt stand Sagramak auf dem Hügel vor dem Palas und grinste. Ihre Waffe war zerbrochen. Neben ihr lag Irils Runenhammer nutzlos am Boden.

Sie hatte die Macht, diesem Kampf ein Ende zu bereiten. Konnte Barz sie vielleicht überrumpeln und von den anderen Kultisten eine Ergebung verlangen? Das klang durchaus gut.

Barz nahm seinen Bogen und atmete tief durch. Schätzte die Distanz zu Sagramak ab.

Zog.

Zielte.

Schoss.

Der Pfeil vollführte einen wunderschönen Bogen und prallte nutzlos an Sagramaks Beinpanzer ab. Ihr Blick richtete sich auf Barz. Sie lächelte schief und pfiff. Ein Dolch schoss aus ihrem Gürtel auf Barz zu. Die telekinetische Waffe sauste an ihm vorbei und durchtrennte seine Bogensehne, ehe er einen weiteren Pfeil abschießen konnte.

„Immer noch ein nutzloser Trick?“, fragte die Schamanin Barz schnippisch. Barz schnaubte bloß. So schnell wie möglich stapfte er über Schneehaufen und herumliegende Körper auf den Palas zu.

Er warf eine Prise dunkelblauen Schwächungspulvers in die Luft, welches sich magisch glitzernd im umherwirbelnden Wind verteilte und um Sagramak herumwirbelte. Diese fauchte auf, als die ätzende Cantharis-Komponente des Pulvers sich in ihre Haut grub. Barz fühlte frische Energie durch seinen Körper schießen.

Jetzt zählte jeder Augenblick. Barz war kein geschickter Nahkämpfer, doch wenn es ihm gelänge, Sagramak genügend rasch auszuschalten, um den anderen Kultisten zu drohen ... und bevor der Eis-Drache bemerkte, was sich unter ihm abspielte ...

Irils Runenhammer lag nur einige Schritte von ihm entfernt, seine Besitzerin nirgendwo zu sehen. Barz hechtete zum Hammer, nahm ihn auf und haute damit nach Sagramak, welche dem Schwung mühelos auswich.

Interessanterweise leuchtete der Hammer nicht magisch auf, wie er es bei Iril getan hatte. Und das Ding war verdammt schwer. Barz konnte sich drei weitere Schwinger in Sagramaks grobe Richtung erlauben, ehe er diesen Steinklotz von einer Waffe wieder zu Boden sinken ließ.

Wie schaltete Iril die im Hammer gespeicherte Macht frei? Musste er dafür irgendeinen Muskel anspannen? Einen bestimmten Griff? Eine bestimmte Rune? Barz betatschte den Hammer hastig, ohne ein Leuchten auszulösen.

Eine Dolchklinge kitzelte ihn an der Kehle.

„Ergib dich“, befahl Sagramak.

Barz war weise genug, Folge zu leisten. Er hob seine Hände.

„Auf, auf zum Verlies mit dir!“, rief Sagramak. „Du kannst Iril Gesellschaft leisten.“\bigskip







Siantari blickte kalt auf Ijsdur herab. Aćh zitterte in ihrem kalten Versteck, nur wenige Schritte davon entfernt, doch durch den mächtigen Sturm gut verdeckt.

„Herrin!“, wiederholte Ijsdur, „Endlich habe ich euch wiedergefunden.“

„Mein Kind“, sprach Siantari kalt. Sie landete sanft auf dem schneeigen Untergrund und legte eine Hand an Ijsdurs Wange. Ihr Gesicht zeigte keinerlei Regung. „Ich hatte dich schon verloren geglaubt. Was ist mit dir geschehen? Warum sind deine Gedanken vor mir verborgen?“

„Ich wurde überrumpelt. Ich wollte durch eine List aus Freund der Menschen auftreten und mehr über ihre Verteidigungskapazitäten erfahren. Ich versagte. Mächtige Runenmagie hielt unsere Geister lange getrennt, zu lange. Doch bin ich Euch noch immer untertan.“

Siantari schloss ihre Augen. „Ich fühlte deine Präsenz in diesem Land, mein Kind. Ich nahm wahr, wie du untergingst. Ich hätte nicht gedacht, dass du es schaffst, allein weiter zu bestehen. Du warst tapfer. Lass mich dich wieder zu einem Teil meiner selbst machen.“

„Gewiss Herrin, wenn ich nur wüsste, wie.“

Siantari legte ihren Kopf schief. Mit weiterhin geschlossenen Augen bewegte sie ihre Hände Ijsdurs Körper entlang, vielleicht irgendwelche Ströme der Magie fühlend.

Aćh packte ihr Schwert fester. War dies der Moment, die abgelenkte Eis-Dämonin zu fangen, vielleicht gar zu erledigen? Wenn sie den Hang hochgestapft wäre und Siantari erreicht hätte, wäre der Überraschungsmoment schon lange verflogen.

„Wie hast du mich wiedergefunden? Hast auch du die Macht des Wintersteins gespürt?“, ertönte Siantaris klirrende Stimme.

„Der gewaltige Wirbelsturm war schwer zu übersehen.“

Siantari kicherte tatsächlich kurz auf, ehe sie wieder tonlos fortfuhr: „Der Sturm war mir nicht einmal bewusst. Dann werden meine Kräfte tatsächlich schon amplifiziert.“

„Von diesem ... Winterstein, den ihr erwähntet? Befindet er sich hier?“

„Nein, der Winterstein ist eben nicht mehr hier. Ich spüre deutlich die Spuren eines mächtigen Artefakts der Ewigen Kälte. Es befand sich hier, eine ganze Zeit lang, und noch nicht so lange her. Nur ein Bruchteil seiner Kraft sickerte in den Boden, und schon der ist gewaltig. Mit einem solchen Relikt könnte das ewige Eis schon in Jahren die gesamte bekannte Welt überdeckt haben. Und ich kenne nichts außer seinem Namen.“

„Winterstein.“

Siantari nickte. „Die Gefangenen nennen es so. Doch will mir niemand verraten, wo es hingebracht wurde. Oder sie wissen es nicht. Oder sie verstehen mich nicht richtig.“

„Gefangene?“

„Gute Güte, ich vermisste es nicht, mit anderen zu sprechen“, seufzte Siantari. „Ja, Gefangene! Dort hinten, am alten Wehrturm. Und sie werden mir verraten, wo der Winterstein sich nun aufhält. Die einzige Alternative dazu ist ein baldiges Ende ihrer elenden Leben. Sprich, wer war dieser Mensch, den du eben noch bei dir hattest? Weiß sie vielleicht etwas über den Winterstein?“

Aćh erstarrte in ihrem Versteck.

Ijsdur gab langsam und bedacht Antwort: „Eine Takuri-Hüterin aus Tulgor. Eine, die schon einmal fast in unsere Fänge geraten wäre. Sie wird eine würdige Eis-Dämonin abgeben.“

„Nicht sie“, flüsterte Siantari, „Sondern derjenige, den sie hütet. Stell dir das vor. Ein Eis-Takuri, der ewigen Schnee über diese Lande brächte und selbst im Tode nicht von seinem Auftrag abkäme. Das wäre durchaus ein Traum. Schwach von dir, sie einfach so entkommen zu lassen.“

Ijsdur reagierte nicht darauf. Siantari schien auch keine Entschuldigung zu erwarten.

„Ihr Takuri heißt Terebros“, hängte Ijsdur an. Auf einmal sprach er wieder laut, als wolle er den Sturm übertönen. „Ein mächtiges Wesen. Aktuell ist es nichts als ein kleines Küken. Sein Zyklus dauert selbst im Vergleich mit anderen Takuri sehr lange. Doch eines Tages wird es zweifelsohne wieder gewaltig groß sein.“

Aćh hörte nicht mehr zu. In ihrem Kopf hatte es soeben geklickt. Ijsdur hatte Siantari angelogen, mehrfach, demonstrativ. Laut und deutlich, als wolle er gehört werden. Er trug kein Takuri-Seil. Er führte keinen Feuertakuri. Allein konnte er nichts gegen Siantari ausrichten. Er brauchte sie, um dem Plan zu vollziehen. Versuchte er, Aćh von seinem freien Willen zu überzeugen, in der Hoffnung, dass sie ihn hörte? Konnte sie ihm vertrauen?

„Da!“, rief Siantari, und öffnete ihre Augen wieder, „Ja da fühle ich etwas Falsches in dir. Eine eigenartige Scheibe, aus der magische Ströme fließen, die sich an deinen Körper, ja, gar deine Seele klammern. Sie abschirmen und zugleich fesseln. Meine Magie rutscht von ihr ab wie Eisenkufen von einer Eisfläche. Darum bleibt mir dein Geist verborgen. Und sie anzufassen, wage ich nicht. Doch andere könnten sie bestimmt berühren.“

Zielstrebig wirbelte Siantari herum und bewegte sich zurück in Richtung des Alten Wehrturms. Wie üblich breitete sie ihre Arme aus und ließ sich von einer Schneewolke tragen, statt zu laufen.

Ijsdur trottete ihr hinterher. Aćh biss die Zähne zusammen, hielt ihr Schwert und ihr Takuri-Seil bereit und stapfte in angemessenem Abstand hinter, sodass sie nicht entdeckt wurde.

Vor dem alten Wehrturm waren vielleicht ein knappes Dutzend Andori in einer Reihe angeordnet. Beinahe alle schienen Kinder zu sein. Ein Schulausflug, unglücklich geendet? Sie duckten sich hinter die groben Mauerreste des Wehrturms und kuschelten sich aneinander, in der Hoffnung, Wärme zu finden.

Dies war das Auge des Sturms. Der Wind hatte abgeflaut. Weit über sich konnte sie gar einen Flecken Himmel erkennen. Es wurde Nacht. Sterne funkelten. Und ... ein weit entfernter rötlicher Funke verriet ihr, dass Turr sie nicht verlassen hatte. Er drehte weit, weit über ihr seine Kreise.

Aćh duckte sich hinter einen etwas weiter weg liegenden großen Felsen, um nicht gesehen zu werden. Doch die Eis-Dämonen kümmerten sich nicht um sie.

„Du!“, schnauzte Siantari die erstbeste Gefangene an. Die Eis-Dämonin zerteilte ihre Fesseln. „Ich will, dass du etwas aus der Brust des Eis-Dämons vor dir ziehst.“

„Was?!“

„Ich werde dir genau zeigen, wo. Lange hinein und ziehe die metallene Scheibe heraus, die in ihm steckt.“

„Was ... wie?“, stammelte die kleine Andori hilflos weiter. Siantari schlug ihr auf den Hinterkopf, zeigte verlangend auf eine bestimmte Stelle in Ijsdurs Körper und machte eine Geste des Herausziehens. Die Gefangene streckte ihre Hand aus und griff nach Ijsdur. Seine schneeige Haut brach unter der Berührung.

Es war Irils Scheibe, die Ijsdur schützte. Selbst wenn er zuvor Siantaris Ziele nicht geteilt hatte, stünde er gleich wieder unter Siantaris Bann. Jetzt war ihre letzte Chance.

Aćh brüllte aus vollem Leibe: „ADVARIA, TURR! ADVARIA MEZA!“ Sie stürzte aus ihrem Versteck hervor und rannte auf die beiden Eis-Dämonen zu, zum Himmel hoffend, dass Turr sie erhört oder wenigstens erspäht hatte.

Er hatte.

Turr stürzte aus den Wolken hinaus. Siantari wirbelte herum und hob ihre Hand. Ein Eisklotz, größer als ein ausgewachsener Zwerg, fegte den Feuervogel vom Himmel. Es knirschte unangenehm, als der Klotz mit Turr gegen den alten Wehrturm krachte und dabei einige lockere Steine aus der Ruine löste. Turr war nicht mehr zu sehen.

Hoffentlich hatte er als Ablenkung gereicht.

Die kleine Gefangene duckte sich verängstigt zu Boden.

Aćh erreichte Siantari, das goldene Schwert zu einem tödlichen Schlag ausgeholt. Sie stach zu. Siantari fegte die Waffe verächtlich aus Aćhs Hand und blickte sie regungslos an. Eine Kälte überfiel Aćhs Glieder. Kraftlos stürzte sie zu Boden.

Da kam Ijsdur ins Spiel. Er bewegte sich blitzschnell, manifestierte ein Eisschwert und stach damit in Siantari hinein. Oder versuchte es zumindest. Schon beim Aufprall zerstieb das Schwert in einen dünnen Nebelschleier, der keine Schneckenmaus verletzen könnte.

Siantari drehte sich schwebend zu Ijsdur um, die Andeutung eines Lächelns auf ihrem eisigen Gesicht.

„Du wagst es?! Ich bin die Herrin des ewigen Eises. Was meinst du, woraus du gewachsen bist? Du bist MEIN!“

Sie streckte ihre Hand gebieterisch aus. Ijsdurs Körper explodierte in eine Wolke aus Schneeflocken. Als seine Stimmbänder sich in Wassertröpfchen auflösten, brach sein Schrei ebenso schnell ab, wie er begonnen hatte.

Siantari schüttelte ihren Kopf und wandte sich ab, zurück zu Aćh, welche mit klammen Fingern ihr Schwert aufklaubte, als sie schon wieder die eisige Kälte übermannte.

„Heda!“, rief eine schwache Stimme. Siantaris Augenbrauen hoben sich kaum merklich, als sie sich überrascht zu Ijsdur zurückdrehte.

Während der größte Teil von Ijsdurs Schneekörper als lockerer Haufen auf dem Boden verteilt lag, wirbelte ein Nebel aus Eiskristalle um ein Konstrukt, das man mit etwas Fantasie als ein schneeiges Skelett bezeichnen konnte. Stetig rissen Siantaris Sturmwinde Fetzen heraus, doch stetig fügten sich aufs Neue Schneeflocken und Wasserwogen wieder ein, um die Lücken zu schließen. Und in der Mitte dieses unförmigen Schneeskeletts schwebte eine metallene Runenscheibe, aus der blendend weißes Licht schien.

Ein durchscheinender Mund öffnete sich und Ijsdurs klirrende Stimme sprach: „Das ewige Eis ist nicht nur dein. Wir sind freie Eis-Dämonen und du ka ...“

Siantari versetzte dem durchscheinenden Schneewirbel, der Ijsdur war, einen verächtlichen Tritt. Erneut pulverisierte sich sein gesamter Körper und fiel zu Boden. Und erneut flossen Schnee, Wasser und Eis wie von Zauberhand um die Runenscheibe zusammen, um Ijsdurs Körper zu rekonstruieren.

„Genug“, zischte Siantaris eiskalte Stimme. Sie hob ihre Arme. Der gesamte Schnee der Umgebung stieg in die Höhe. Der sich zusammensetzende Ijsdur wurde nicht mehr auseinandergerissen, sondern von Siantaris Magie geleitet zu einem Körper zusammengesetzt. Ein Körper, der strampelnd in der Luft hing.

Da hatte Aćh endlich wieder ihre Kraft gefunden. Sie trieb ihr goldenes Schwert glatt durch die Dämonin hindurch. Kurzzeitig ließ Siantaris Kontrolle über Ijsdur nach, als ihr Torso von ihrer Hüfte glitt. Dann fing sie allerdings sowohl sich selbst als auch Ijsdur. Ihr Körper verschloss sich wieder. Ijsdur wurde erneut von magischen Winden in die Höhe gehoben. Siantari drehte sich erneut zu Aćh herum und strafte sie mit einem zornigen Blick.

Mit bloßen Händen griff die Dämonin nach dem Schwert der Takuri-Hüterin. Raureif überzog es. Dann zerbrach das schockgefrostete Metall wie mürbes Holz unter ihren Fingern. Stechender Schmerz durchzuckte Irils Arm.

Aćh presste die unterkühlte, nutzlose Hand an sich. Sie sank aussichts- und hoffnungslos zu Boden. Ijsdur hing gefangen in der Luft. Was konnten sie beide schon gegen eine solche Gefahr tun?

Ein Blitz zuckte über den Himmel, hinein in Ijsdurs ausgestreckte Hand und sprang von dort auf Siantari über. Der Donner war betäubend laut. Die Eis-Dämonin wurde in den Rücken getroffen und einige Schritte über Aćh hinweggeschleudert, ehe sie sich fing. Mehrere kleinere Blitze zuckten aus ihrem eigenen Körper. Dann leitete sie den Eisblitz auf Aćh weiter, welche noch weiter zurückgeschleudert wurde, gegen den alten Wehrturm knallte und zusammensackte.

Ijsdur wehrte sich tapfer, doch nun, wo die gesamte Aufmerksamkeit Siantaris auf ihm lag, hatte er keine Chance mehr. Seine Gliedmaßen verdrehte sich unnatürlich in alle möglichen Richtungen, während sein Körper mehrfach zu Boden geschleudert wurde.

Da traf Siantari ein Schuh in die Seite, der beinahe ein Loch in ihren Schneekörper riss. Mehr verwirrt als verletzt blickte die Dämonin zu ihren Geiseln um. Welche von ihnen hatte es gewagt, sich gegen sie zur Wehr zu setzen?!

Eine der Gefangenen, eine ältere Dame, wohl eine Aufseherin der vielen andorischen Schulkinder, strampelte gerade damit, ihren zweiten Stiefel auszuziehen und damit nach Siantari auszuholen. Einen Augenblick lang war Siantari verwirrt. Wie konnte eine derart alte Menschenfrau die Stärke aufbringen, sich gegen sie zu stellen?! Beinahe amüsiert schleuderte die Eis-Dämonin einen Eiszapfen auf die Törichte und befreite sie von ihrem jämmerlichen Leben. Das farbige Leuchten in den Augen der Alten erlosch.

Sie ließ eine Schneewelle auf die restlichen Gefangenen niederregnen.

Die Kälte traf auch Aćh und weckte sie aus unruhigen Gedanken. Aćh schüttelte sich und versuchte, das Ringen in ihren Ohren zu unterdrücken. Ihr rechter Arm pochte. Jedes Mal, wenn sie mit einer Eis-Dämonin kämpfte! Beim letzten Mal hatte gelernt, ihr Schwert mit einer beliebigen Hand zu führen. Doch nun hatte sie gar kein Schwert mehr.

Sie orientierte sich: Schnee, Stein, zersplittertes Schwert, Geiseln, ein regloser Takuri. Sie befand sich am Alten Wehrturm, sie musste kämpfen, sie musste Ijsdur helfen, jetzt, jetzt, JETZT!

Aćh kraxelte zum sterbenden Turr. Er durfte sich noch nicht jetzt in ein Küken zurückwandeln! Sie ließ von ihrer Kraft hineinfließen, um ihn zu beruhigen, um ihn zu stärken. Was sie jetzt brauchten, war ein Takuri in der Blüte seines Zyklus.

Kälte durchzuckte Aćhs Glieder. Sie kippte zur Seite, doch Turr stieg in die Höhe. Ein goldener, feuriger Glanz umgab ihn. Stolz breitete er seine Flammenflügel aus. Ein roter Schein stieß daraus hervor, trieb Eis und Schnee zurück und zwang Siantari, ihren Blick abzuwenden.

Ijsdur nutzte die Gelegenheit, um Siantari anzuspringen und zu Boden zu zwingen. Zwei dämonische Geweihe verhakten sich. Die beiden Eis-Dämonen rollten strampelnd über den Boden. Dann schaffte Ijsdur es, Siantari in einen Schwitzkasten zu kriegen.

Aćh stolperte hinzu, zückte ihr Takuri-Seil, rannte hoch und umschlang Siantaris zappelnden Körper damit. Das Takuri-Seil glühte auf und brannte sich leicht in Siantari hinein.

Der Sturm um den alten Wehrturm legte sich schlagartig.

Die Herrin des Kuolema-Gebirges war gefesselt.

Ijsdur stolperte zurück. Denn Turr hatte nicht abgelassen, feurige Wärme auf die Dämonen zu werfen. Während Ijsdur sich die schmelznasse Stirn abwischte und kühlen Abstand hielt, erlitt Siantari die volle Hitze eines Takuri in der Blüte seines Zyklus.

Aćh und Ijsdur guckten einander erleichtert an. Dann konnten sie nicht anders, als aufzulachen. Siantari war gebannt. Der Wirbelsturm abgeflaut. Die Kinder befreit.

„Verzeih mir, Ijsdur, ich hätte dir vertrauen sollen“, flüsterte Aćh.

Ijsdur winkte ab. Ihn sollten solche Sachen schon lange nicht mehr schmerzen. „Lass die Vergangenheit vergangen sein. Ich hätte dich nicht drängen sollen. Belassen wir es dabei und lernen wir daraus.“

Ein Schrei holte sie zurück in den Moment. Jetzt, wo der Schneesturm Siantaris sich gelegt hatte, konnten sich ihre Gefangenen befreien, aus dem Schnee freischaufeln und aufwärmen. Dabei war auch die Leiche derjenigen ausgegraben worden, welche der Eis-Dämonin todesmutig einen Stiefel angeworfen und den Helden so die nötige Zeit verschafft hatte. Es war eine alte Dame, elegant gekleidet, vermutlich eine große Gelehrte. Zumindest einen anderen Lehrer hatten die jungen Andori bei sich gehabt, und dieser hatte nun seine größte Mühe damit, die Kleinen aus dem Schnee zu bringen und von der Leiche ihrer Lehrerin fernzuhalten.

Während die drei Erwachsenen sich um die Schüler kümmerten, bemerkte Ijsdur ruhig: „Wir konnten nicht alle retten.“

„Aber wenn wir nur ein klein wenig schneller gewesen wären ... Wenn ich nicht davon gerannt wäre ...“

„Natürlich können wir uns bessere Verläufe der Vergangenheit vorstellen. Aber auch schlechtere. Mehr als daraus lernen können wir nicht. Und zum Heldendasein, so wenig ich davon verstehe, gehört auch, Verluste einzustecken. Aufzugeben, weil man nicht perfekt handeln kann, scheint mir nicht sinnvoll.“

„Ich redete nicht davon, aufzugeben“, sagte Aćh traurig, „Aber es ist durchaus wichtig, dieses Gefühl der Schuld zuzulassen. Sonst geben wir uns mit allzu leicht mit unseren Handlungen zufrieden, ohne darüber nachzudenken, was wir hätten besser machen können.“

„Wir wissen doch, was wir hätten besser machen können. Und wir können uns gerne Zeit zum Trauern nehmen. Keine Trauer kann schließlich ebenso schädlich sein wie eine ewige Trauer. Doch nun können wir unsere Zeit besser nutzen. Nun gilt es, einen Eis-Drachen von der Rietburg fernzuhalten.“

Beider Blicke wanderten in Richtung Norden, zur Rietburg, auf deren Zinnen sich immer noch ein Eis-Drache räkelte. Die Kultisten konnten nicht ernsthaft vorhaben, das Gemäuer für sich selbst zu besetzen. Oder?

„Iril und Barz sollten schon nun bei der Burg angekommen sein, oder? Wie es ihnen wohl geht?“

„Iril? Die Runenmeisterin?“, fragte eine leise Stimme. Ein von Kälte bläulich angelaufener Junge krabbelte herbei und stellte sich als Jandro vor. „Ich kenne Iril. Sie kam in der Schule vorbei und zeigte uns ihre Runenscheibe. Was ist mit ihr? Geht es ihr etwa auch wie meinem Vater? Wie unserer Hohen Gelehrten? Wird sie auch nie wieder zurückkommen?“

Tränen wallten in den Augen des Jungen auf. Aćh kniete sich neben ihm nieder und sprach ihm die wenigen andorischen Worte des Trostes zu, die sie bislang gelernt hatte.

Da fiel etwas auf. Neben der Leiche der alten Gelehrten lag eine kleine Schatulle, die der Dame aus der Tasche gefallen war. Sie hatte sich geöffnet. Ein kleiner, unscheinbarer, gelb glühender Stein war daraus gerollt.

Aćh hob den Stein in die Höhe. Eine Rune war darin eingeritzt, die an einen Torbogen erinnerte. Und in der Schatulle fand sie zwei weitere solcher Steinchen. Einen grünen und einen blauen. Runensteine, hatte Iril gesagt.

„Die hier sollten bei Iril sein“, murmelte sie zu Ijsdur.

„Prima, Turr soll sie doch bringen!“

„Turr ist schlecht darin, etwas Entzündliches zu transportieren. Und wir brauchen ihn hier, um Siantari zu schwächen.“ Aćh deutete zur Herrin des Kuolema herüber, welche schmelzend vor Turrs ausgebreiteten Flügeln stand und sie hasserfüllt anblickte.

Mit erhobener Stimme wendete sich Aćh an die Befreiten.

„Hat jemand hier einen Falken?“

Ein Bauer mit eleganter Pferdeschwanzfrisur meldete sich zu Wort. „Jawohl! Ich bin zwar kein Falkner, aber mein Bruder ist einer. Ein Lehrer an der Rietburg. Ich sprang heute ein für ihn.“

Ohne weitere Ausführungen zog der Bauer eine kleine eiserne Pfeife hervor – diese hatte an einer Kette um seinen Hals gehangen – und blies hinein. Es dauerte nur eine Minute, da stieß ein mächtiger Falke vom Himmel herab, landete auf seinem nackten Arm und grub seine Krallen hinein. Der Bauer hauchte kurz vor Schmerzen auf.

„Mein treuer Fleo bringt eure Post zielsicher zum Ziel. Egal wie weit, egal zu wem. Sprecht, was wollt ihr wohin senden?“\bigskip







Das Verlies der Rietburg war dunkel, still und verlassen. Nur selten wurden Andori hier in der Kaverne eingesperrt. Öfter wurde als Strafe für Vergehungen eine Verbannung von der Burg verhängt, was in jeder Zeit einem Todesurteil gleichkam.

Doch hin und wieder musste es auch Personen gegeben haben, die eine solche Gefahr für die Öffentlichkeit dargestellt hatten, dass es gerechtfertigt gewesen war, sie hier einzusperren. Düstere Diebe, Schattige Hexen und Dunkle Magier – wobei zumindest letztere ein außergewöhnliches Talent dafür hatten, sich nicht lange von einer Gefängniszelle halten zu lassen.

Oder gerade andersherum – es musste einige Personen gegeben hatten, deren Vergehen keine Verbannung von der Rietburg gerechtfertigt hatte, die aber immer noch kurzfristig einen Sicherheitsabstand zwischen sich und den restlichen Andori brauchen konnten, bis beispielsweise ihr Rausch abgeklungen war.

Sagramak und ihre Kultisten hatten von irgendwoher den Schlüssel zu den Kerkergewölben ergattert und Iril und Barz kurzerhand dort eingesperrt. Sie waren die einzigen. Hatten alle anderen Krieger bis zum Tode gekämpft? Kämpften sie immer noch da draußen, während die Kultisten die Burg nach Thorald und Taroks letzten Drachenknochen durchsuchten?

„Keine Sorge, ich bin dir nicht böse“, hatte Sagramak beim Einsperren fröhlich zu Barz gesagt. „Nicht allzu sehr. Du wirst überleben. Vielleicht verpasse ich dir noch einen Denkzettel, sobald wir die Drachenknochen haben. Aber nicht Gewaltigeres. Wir wollen dich nicht leiden lassen. Das kommt noch früh genug nach deinem Tod, wenn die hohen Drachen dich richten. Die Runenmeisterin hingegen ... sie werden wir mitnehmen und schauen, dass diese mächtigen Runen nicht mehr gegen uns eingesetzt werden, sondern für uns.“

Dann war sie abgezogen, um die Rietburg nach Thorald und Taroks letzten Knochensplittern zu durchsuchen. Das war vor einer halben Stunde gewesen.

Barz brummelte vor sich hin: „Wir hätten Ijsdur mitnehmen sollen. Er hätte uns einen schönen Eisblitz-Tunnel hier raus zaubern können.“

„Nicht das Ausbrechen ist das Problem, sondern die Stärke unserer Gegner“, stellte Iril klar.

„Ich Idiot!“, nervte sich Barz, ohne auf Irils Antwort zu achten. Nicht zum ersten Mal in der letzten halben Stunde. „Ich hätte das Vorhersehungspulver bei Sabri lassen und das Umwandlungspulver mitnehmen sollen. Ich hätte uns starke Schilde schenken können. Oder noch besser, mir einen neuen Bogen. Wir hätten aus deinem Hammer ...“

„Mein Hammer wird doch nicht einfach so umgewandelt“, entrüstete sich Iril, „Und überhaupt, Was-wäre-wenn-Szenarios sind sehr interessant durchzudenken, aber wir beide können unsere Zeit hier sicher sinnvoller nutzen.“

„Was, während wir darauf warten, dass uns jemand retten kommt?“

„Nein, während wir uns befreien! Wir sind zwar in einem uralten Verliesgewölbe eingesperrt, aber wir sind allein. Unbewacht. Wer weiß, was wir alles anstellen können.“

„Jedenfalls nicht kämpfen. Mein Bogen wurde zerschnitten. Und liegt ohnehin noch draußen.“

„Ich schenke dir einen neuen, wenn das alles vorbei ist. Aber nur, wenn du aufhörst zu jammern und mir hier aushilfst. Wo mein Runenhammer? Ich sah ihn dich führen, als Nehamal mich überwältigte.“

„Spornwaldreck! Den habe ich auch draußen verloren.“

„Kein Problem.“

Iril suchte sich einen Spalt im vergitterten Fenster, durch das das Licht der untergehenden Sonne reinstrahlte. Unter den Strahlen glühten einige Runen auf ihrem Arm schwach auf. Iril streckte ihre Hand aus und murmelte etwas zwischen zusammengekniffenen Zähnen hervor.

So verharrte sie eine Weile. Dann drehte sie sich herum. Aus der Ferne vernahm man ein Poltern und Knarren, ein Knirschen und Rumoren. Und dann, unregelmäßig stockend, rutschte der Runenhammer zwischen den Eisenstäben der Gittertür hindurch und in den Raum hinein. Ein Glück, dass niemand der Kultisten auf ihn geachtet hatte.

Iril packte den Hammer, welcher prompt zu glühen begann. Im schwachen Schein des Hammers wirkte Barz‘ in Schatten getauchtes Gesicht ganz gruselig.

„Wie machst du, dass er so leuchtet?“, fragte er neugierig, „Als ich ihn hielt, war er einfach ein ganz gewöhnlicher Hammer.“

„Das ist auch gut so. Dein Menschenblut könnte der Dunklen Magie in seinem Innern erliegen, wenn du seine Macht nutztest.“

„Ich glaube, zu verstehen. Bei meinen magischen Pulvermischungen spüre auch ich die Last der Magie und bin stets darauf bedacht, das Gleichgewicht zu wahren. Doch bedeutet das für mich oft, mich den finsteren Künsten ganz zu entsagen, um gar nicht erst in Versuchung zu kommen. Du meinst, Zwerge wären gegen ihre flüsternden Stimmen immun?“

„Zumindest immuner als die meisten Menschen“, meinte Iril. „Oder hast du schon einmal einen zwergischen Dunklen Magier gesehen?“

„Die gäbe es sehr wohl, wenn es mehr Zwerge in Hadria gäbe.“

„Bist du sicher?“

„Nicht unsicherer, als du es sein kannst.“

Dabei beließen es die beiden.

Neben ein wenig Stroh und einem behelfsmäßigen Scheißeimer hatte ihre Zelle kaum etwas zu bieten. Iril machte sich auf zur vergitterten Tür und spähte in den dunklen Gang dahinter. Es war Nacht geworden und die Beleuchtung hier war selbst bei Tage äußerst dürftig. Wenn sogar ihre Zwergenaugen kaum erkennen konnten, wie es weiter hinten im Gang aussah, musste es für Barz wahrlich stockfinster sein. Wie so oft wünschte sie sich einen Casamatuc, um solche Türen gewaltlos zu öffnen. Doch diese seltenen Zwergenwerkzeuge wurden nicht jedem anvertraut.

Das Schloss der Gittertür zerbarst unter einigen Schlägen von Irils Hammer. Die beiden Insassen horchten, ob ein Drachenkultist den Lärm gehört hatte und nachsehen kam. Nach fünf Minuten der Stille trauten sie sich in den Gang hinaus.

Zu ihrer Linken und Rechten lagen einige weitere vergitterte Zellen. Dahinter führte ein natürlicher Höhlenausgang in den dunklen Burghof der Rietburg, natürlich ebenfalls von einem Gitter verschlossen. Sterne funkelten vom Himmel herab. Ein regelmäßiges dumpfes Stampfen zeugte vom dahinter umherlaufenden Eis-Drachen. Der war wohl auch der Grund, warum Sagramak es nicht für nötig befunden hatte, Wachen aufzustellen. In ihrem jetzigen Zustand waren Iril und Barz ihm alles andere als gewachsen.

Zu ihrer Rechten führte ein verschlungener Gang tiefer ins unterirdische Kellergewölbe der Rietburg. Iril wusste aus ihrer Zeit in den letzten Wochen, dass der Gang in ein Labyrinth von Sackgassen mündete. Hier befanden sich vor allem Vorratskammern und natürlich andere Kerkerzellen. Man konnte der Burg so nicht entkommen. Und eigentlich wollte sie das auch nicht, solange es noch eine Gefahr zu bändigen gab.

„Was starrst du so, Iril? Gucken wir uns lieber noch die anderen Zellen an oder nicht?“

Iril nickte stumm.

Es war überraschend interessant, das Innere der verschiedenen Kerkerzellen zu inspizieren. Zahlreiche Kritzeleien zierten die Wände. Letzte Botschaften von Dieben und Halunken, die nach einem Aufenthalt hier aus der Burg verbannt würden. Man sah einfache Spiele zwischen Eingesperrten, deren Verläufe mit Löffelstielen in die Wand geritzt worden waren. Hier ein Gedicht auf verführerischen Met, da ein Schmähspruch auf die Wachen. Eine Zelle war anders, tiefschwarz gefärbt, die Runen mit Wänden verstehen, die Iril nur vom Hörensagen nach kannte. Hier hatte jemand die Dunkelheit selbst einzusperren versucht. Oder eine Schattenhexe.

Weiter hinten fanden Iril und Barz auch einen großen unterirdischen Raum, in dem zahlreiche Weinfässer der Burg gelagert wurden. Offenbar diente diese in die tief in den Hügel geschlagene Kaverne auch als Lager für den Keltermeister der Rietburg. Doch das schmiedeeiserne Tor war fest verschlossen. Was für edle Tropfen dahinter liegen mussten, nur für den Prinzen und seine engsten Vertrauten bestimmt.

„Kommt nicht in Frage, dass wir uns jetzt betrinken!“, rief Barz kopfschüttelnd, „Ihr Zwerge und euer Durst.“

„Nicht alle Zwerge fahren derart auf betäubenden Trunk ab, wie du meinen magst“, protestierte Iril. „Sieh doch. Da hinten an der Mauer glänzt etwas.“

Barz blickte ins Dunkel und grinste plötzlich auf. „Kannst du die Tür hier aufschlagen? Es wird es definitiv wert sein.“

Irils Hammer brach das Schloss, während Barz Ausschau hielt dass kein Drachenkultist auf die Idee kam, nach dem Ursprung dieser Geräusche zu suchen.

Kaum war die Tür offen, raste Barz an den gestapelten Weinfässern vorbei an die Rückwand der Höhle. Durch eine klitzekleine Öffnung am oberen Rand der Wand warf der Mond sein silbriges Licht in den großen unterirdischen Raum. Und ließ eine unscheinbares, scheinbar aus der Wand sprießendes Gewächs glitzern.

Anerkennend nickte Barz die milchig weißen Beeren an der fremdartigen Pflanze an. „Mondbeeren. Besonders leicht in den ersten Sonnenstrahlen eines neuen Tages zu finden, wenn sie leuchten. Manchem, der die Beeren isst, verleihen sie neue Kraft und manch anderem schien es, als bliebe sie Zeit stehen.“

„Super“, freute sich Iril, und schwang ohne bestimmtes Ziel ihren Hammer herum, „Zeit ist das, was wir am besten gebrauchen können, um uns für den bevorstehenden letzten Kampf zu stärken.“

„Oh, wenn es nur um Zeit geht, kann ich auch aushelfen“, rief Barz. Er löste ein kleines braunes Pulversäcklein von seinem Mantel und schüttelte es.

„Meditationspulver. Kann uns mehr Zeit verschaffen. Ich kann es nicht nutzen, wenn zu viele Personen um mich herum sind – dann verwaschen sich alle ihre Präsenzen irgendwie zu einem neutralen Hintergrundrauschen – aber, wenn nur wir zwei in einem abgeschiedenen Kerker herumlungern, kann ich mich prima auf deine Präsenz konzentrieren und dir mehr Zeit schenken. Nur, dass wir mit mehr Zeit keine größere Chance gegen einen verdammten Riesendrachen haben.“

„Nicht aufgeben, lieber Barz. Es ist ein Anfang.“

Iril blickte in die nächste Zelle und untersuchte im düsteren Schein ihres Runenhammers deren Ecken. Es fühlte sich an, als locke sie ein unsichtbares Seil dorthin. Sie folgte dem Gefühl und gelangte an eine kleine Fensteröffnung, aus der einige Köpfe über ihr ein heller Lichtschein ins Dunkel hineindrang.

Etwas flatterte am Fenster vorbei. Iril zuckte zurück. Ein Schatten ließ sich davor nieder. Vogelaugen starrten auf sie herunter. Dann plumpste es, als etwas durch das Fenster auf den Boden geworfen wurde.

Iril kniete sich nieder und tastete danach. Dann lachte sie auf und krümelte etwas aus dem Dreck hervor. Eine hölzerne Schatulle, kaum eine Zwergenhand breit, deren Oberfläche unverkennbar das Zeichen der Hohen Gelehrten der Rietburg trug.

Iril öffnete sie und stieß einen erleichterten Lacher hervor. Eine fröhliche Nachricht von Aćh! Und als Iril den weiteren Inhalt der Schatulle untersuchte, fand sie drei unscheinbar wirkende Steine ...

Die drei Steinchen lagen auf einem schwarzen Tuch. Ein gelbes, ein grünes und ein blaues. Die uralte, in ihnen gespeicherte Magie ließ sie leise vor sich hin brummen.

Runensteine.\bigskip







Es dauerte bis tief in die Nacht hinein, bis Sagramak und ihre Drachenkultisten endlich Thorald fanden. Er hatte sich in Begleitung Hauptmann Armonds in einem Gebäude an der Burgmauer verschanzt und seine Position erst aufgegeben, nachdem der Eis-Drache Sagrakdur das mit Rietgras bedeckte Dach eingerissen hatte.

Hoch erhobenen Hauptes schritt Thorald aus dem Gebäude hervor und warf sein Schwert vor Sagramaks Füße. Armond tat es ihm nach und erinnerte den Regenten daran, auszuhandeln, dass er die Drachenknochen nur übergeben würde, wenn die restlichen Andori verschont würden. Sagramak stimmte lachend zu.

Die Nacht war bereits lange angebrochen und ihre finsteren Schatten schmiegten sich an die Mauern der Rietburg, als Thorald den Thronsaal erreichte. Der rote Mond schien durch die Fenster hinein, als er an einer Klappe hinter dem hölzernen Thron herumfummelte. Das Versteck war größtenteils leer, denn der Bruderschild war zum Glück gerade bei anderen Helden im Einsatz. Thorald langte hinein und produzierte einen unscheinbaren samtenen Sack, dessen Inhalt leise klapperte.

Hoch erhobenen Hauptes schritt Thorald wieder aus dem Thronsaal hervor und übergab die Drachenknochen Sagramak.

Hinter ihr standen triumphierend die übrigen Drachenkultisten, allen voran Nehamal und Fir. Das Ziegenwesen trug einen provisorischen Verband um seine Brust und schimpfte übel über Blutflecken auf seinem edlen Gewand. „Noch wird diese Welt dich nicht los, Fir, du tapferer Bursche mit deinen exzellenten Heilkräften! Doch welch Schande, dass dieser Pfeil ein Loch in dein edles Gewand reißen musste. Wollen die Drachen etwa, dass du dies selbst flickst?“

Über allen dreien schwebte der Schwarze Herold, sein langes Schwert still in einem eisernen Griff haltend. Die Spitze streifte beinahe Sagramaks Haarschopf. Ihren Helm hatte sie im Zweikampf mit Iril verloren.

Die Schamanin packte aus dem soeben erhaltenen samtenen Säcklein die Kette mit Taroks Knochensplittern, die sie vor einigen Wochen noch selbst getragen hatte.

„Brav, brav, werter Prinz“, lächelte sie und tätschelte Thoralds Wange herablassend. Thorald lief unter seinem Bart rot vor Wut an, beherrschte sich allerdings.

„Und nun tut, was ihr verspracht“, sprach er stolz, „Zieht ab mit eurem Preis. Lasst uns in Frieden unsere Toten bestatten und die Schäden beheben.“

Das mitleidige Lächeln auf Sagramaks Gesicht wurde zu einem finsteren Grinsen. Noch immer starrte sie gierig Taroks letzte Knochen an. „Ihr hättet uns nicht anders behandelt, oder? Warum sollte ich Euch verschonen? Ihr, die Ihr euch einen Dreck um unsere Anliegen schertet! Ihr, die Ihr am liebsten hättet, wenn es uns gar nicht gäbe!“

Nehamal zog sie zur Seite: „Tu das nicht, Sagramak. Gibt ihnen nicht die Genugtuung ...“

„Der Prinz soll fallen!“, zischte Fir, „Durch seine Missachtung unserer Bedürfnisse hat er es verdient. Eine Mahnung an all unsere Feinde soll er sein!“

Sagramak blickte zischen beiden hin und her und wandte sich an Thorald. „Wenn ich euch verschone, werdet ihr uns jagen, sobald sich eine Gelegenheit ergibt, oder uns in Ruhe lassen?“

„Irrelevant. Ihr gabt uns unser Wort, uns in Frieden zu lassen, wenn wir euch die Knochen gäben!“, rief Thorald.

„Eines jeden Wort kann mindestens einmal gebrochen werden. Wenn es sich am besten lohnt“, sprach Sagramak. Ein bösartiges Grinsen überfiel ihr Gesicht. Rotes Feuer glomm in ihren Augen auf. „Und wer seid Ihr, um über gebrochene Worte zu diskutieren? Gab Euer verräterischer Vater nicht sein Wort, dass er die mächtigen Schilde an die Zwerge zurückgäbe?“

„Das zählt nicht! Das war, bevor Rudnar ... nein, darüber will ich gar nicht reden. Erst recht nicht mit Euch. Ich bin nicht mein Vater. Ich erwarte von Leuten, die mit mir verhandeln, dass sie ihr Wort nicht brechen, da ich meine Versprechen auch stets halte.“

„Wir werden nie wieder miteinander handeln“, sprach Sagramak.

„Vielleicht sollten wir Gutwille walten lassen“, meldete sich nun die tiefe Stimme Sagrakdurs. Der Eis-Drache hatte seinen Kopf nachdenklich in seine Tatzen gelegt und brummelte: „Es ist bereits viel Leid über diese Menschen gebracht worden im Namen Taroks. Wir haben, was wir wollen. Wir können gute Gewinner sein.“

Sagramak schüttelte enttäuscht ihren Kopf.

„Dafür, dass ich beinahe mein halbes Leben lang deine Seele in mir getragen habe, hätte ich gedacht, dich besser zu kennen. Sagrak, was ist aus deinem Blutdurst geworden? Aus deiner Rachelust?“

„O Schamanin, du magst von der Seele Sagraks erfüllt gewesen sein, dessen Blut brannte voller Rache auf die Welt, die er viel zu früh verlassen musste. Doch ich bin Sagrakdur. Meine Glut brennt nicht so heiß. Ich dürste nur noch nach dem ewigen Eis.“ Tief schnaufte der Eis-Drache durch. „Diese Welt ist so verwirrend. Nichts ist mehr wie beim Unterirdischen Krieg. Die Drachen sind alle tot. Die Trolle sind an unserer Seite statt unsere Gegner. Die elenden Agren, Zwerge und Krahder leben zurückgezogen in ihre Reiche. Doch überall haben sich diese Menschen ausgebreitet. Sie alle werden ohnehin vergehen, wenn das ewige Eis erst einmal diese Lande überzieht. Bis dahin sollen sie doch ruhig noch weiter bestehen dürfen.“

Der Schwarze Herold schwebte zu Sagramak herunter und flüsterte ihr etwas ins Ohr. Sagramak nickte und ließ ihre Stimme erschallen, während der Herold wieder etwas über ihr schwebte.

„Ich sehe, wie es ist. Mein lieber Sagrak, du warst mir lange Zeit ein treuer Begleiter und ich dir eine treue Dienerin. Doch nun ist es an der Zeit, Abschied zu nehmen.“

„Was ist das für ein Unterton in deiner Stimme?“, fragte der Eis-Drache argwöhnisch.

„Am Ende bist du doch nur ein Abklatsch des Gewaltigen. Dein Vater wird deine Bedenken nicht teilen.“

„Ja!“, zischte der Schwarze Herold.

Der Himmel rötete sich in der Ferne, Anzeichen des kommenden Sonnenaufgangs.

Sagramak drehte Taroks Fußknochen triumphierend in ihren Fingern. Dann streckte sie ihre Hand aus. Telekinetisch geführt schwebten die Fußknochen nach vorne. Beherrschend hielt Sagramak ihre Hand ausgestreckt und gestikulierte zum riesigen Eis-Drachen neben ihr. Ihre telekinetischen Kräfte schlugen an.

„Nein! Tu nichts, was du bald bereuen ...“, jaulte Sagrakdur auf. Die Eiskristallkette in seiner Brust war winzig im Vergleich zu seinem Körper. Und doch schrie er auf, als die Eiskristalle telekinetisch aus seiner Brust gerissen wurden. Knackende Spalte breiteten sich entlang seines Körpers auf. Während Eis-Sagrak zerfiel, gebot Sagramak der schwebenden Eiskristallkette, sich vor Taroks Fußknochen zu versammeln.

Die Eiskristallkette summte und brummte. Wasser tropfte aus dem Boden in den Himmel, kondensierte aus der Luft, erstarrte zu Eis und Schneeflocken. Wie aus dem Nichts formten sich weitere Drachenknochen aus purem Eis und ordneten sich neben den echten an. Dann wurden sie von einer Schicht aus Eis und Schnee überdeckt, die zunächst Muskeln, dann Haut und schließlich messerscharfen Schuppen glich. Eis-Sagraks Überreste zerliefen zu Wasser, während der große Eis-Tarok immer größer und stärker wurde. So, als ginge die Kraft des kleinen Drachen in ihn über.

Die Eiskristallkette schwebte von Sagramaks ausgestreckter Hand geführt nach vorne und fügte sich nahtlos in den Nacken des riesigen Eis-Drachen ein, der vor Sagramak schwebte und zitternd ein- und ausatmete.

Schneeflocken wirbelten wild umher, als der Drache seinen vereisten Kopf schüttelte und aus Versehen mit seinem ungewohnten vielzackigen Dämonengeweih gegen ein Haus stieß. Gewaltige Schwingen wurden gespreizt, während eisige Blitze über den Himmel zuckten. Klirrende Dampfwolken bildeten sich vor der langen Zunge, die aus einem vielzahnigen Mund herabzüngelte. Dann wurde der Mund aufgerissen und die dahinter liegenden Stimmbänder entließen ein ohrenbetäubendes Gebrüll.

Tarokdur war erwacht.


















\newpage
\section{Der Zorn des Eis-Drachen}




Tarokdur war erwacht.

„Ich ... ich ...“, stammelte der gigantische Eis-Drache nach Worten, „Was ... was ist mit mir? Was ist mit ... Krahal? Die Stimmen der Altvorderen? Mein Zorn? Meine Wut? Mein Schmerz?“

„Alles Teil der Vergangenheit“, rief der Schwarze Herold triumphierend, „Willkommen zurück unter den Lebenden, mein Gebieter.“

Der gewaltige Schädel des Eis-Drachen bog sich auf seinem langen Hals, um den Herold ins Auge zu fassen. Tief und klirrend dröhnte die Stimme des Drachen in den Ohren der Anwesenden, ohne dass sein Maul sich geöffnet hätte.

„Ah, mein Herold. Mein treuster Diener. An dich ich erinnere mich gut. Doch irrst du dich. Ich bin nicht derjenige Tarok, den du anbetetest. Ich bin Tarokdur.“

„Pipifax! Trägst du nicht sein Aussehen? Teilst du nicht seine Erinnerungen? Du bist mir Tarok genug. Mit dir an unserer Seite können wir uns endlich an Andor rächen! Alles soll in Schutt und Asche liegen!“.

Tarokdur blinzelte zweimal verwirrt, ehe er seinen Kopf mit den überlangen Dämonengeweihen schüttelte und nickte.

„So sei es. Statt im Feuer zu brennen, sollen die Eindringlinge in mein Drachenland im Eis umkommen. Danach suchen wir die Reiche der verräterischen Zwerge auf und frieren ihre Tiefminen zu. Und dann reisen wir nach Krahd! Es gibt eine alte Rechnung mit den Riesen zu begleichen. Nomions Brut soll ein für alle Mal im Lavasee versinken. Der Enran soll von einer dicken Schicht aus Eis und Schnee bedeckt werden! Die ganze Welt soll leiden für das Aussterben der größten Spezies aus uralter Zeit!“

„Moment einmal“, warf Sagramak ein, „Wir Kultisten sind natürlich gerne dazu bereit, Taroks Wünschen nach unseren Möglichkeiten zu dienen. Doch wollen wir nicht die ganze Welt unter einer Eisschicht untergehen sehen. Das wäre ... exzessiv. So wollen es die anderen Drachenseelen nämlich auch nicht. Sagrak mag dieses Land.“

Demonstrativ hielt die Schamanin das rote Drachenrelikt aus Roteisenstein in die Höhe, in dem bis vor Kurzem Sagraks Seele geruht haben sollte – und es nun vielleicht wieder tat. Da Sagramak soeben selbst noch bedacht hatte, Thorald und die Rietburg dem Erdboden gleichzumachen, ging es der Schamanin wohl vielmehr darum, Macht gegenüber Tarokdur zu demonstrieren.

Nichtsdestotrotz sorgte sie dafür, dass Tarokdur seine Augen zusammenkniff und seinen Kiefer dem kleinen Menschen zuwandte. Welcher Wurm wagte es, seine Autorität zu hinterfragen?! Wer war so töricht, für seinen viel zu früh gestorbenen Sagrak sprechen zu wollen?

Sagramak fuhr fort und verlieh ihrer Stimme einen besänftigenden Klang: „Anstelle einen Völkermord zu begehen, wollen wir doch lieber den Bruderschild und ähnliche uns zustehende Artefakte von den Andori verlangen und uns dann wieder in unsere Bergwohnungen zurückziehen, fortan gut schlafend im Wissen, dass die Andori keine ernst zu nehmende Gefahr mehr sind. Und danach magst du dir immer noch Krahd vornehmen. Bei Nehal, wenn es nicht anders geht, magst du auch Prinz Thorald reißen. Und die Helden von Andor. Aber nicht sein ganzes Volk. Sie können nichts für die Taten ihrer Helden.“

Tarokdur starrte sie eisig und ausdruckslos aus blau glühenden Augen an. Hatte sie gerade ernsthaft ihm die Erlaubnis zum Töten zu verwehren oder zu erteilen versucht? Ihm, dem letzten und mächtigsten aller Drachen?!

„Es geht hier nicht nur um deinen Willen, Tarokdur“, rief Sagramak bestimmend, „Ich gab dir Leben, ich kann es dir auch wieder nehmen.“

Bedrohlich spreizte Sagramak die Hände, bereit, Tarokdur telekinetisch die Eiskristallkette aus der Brust zu reißen.

Ein tiefes, humorloses Lachen schüttelte den gewaltigen Körper des Eis-Drachen. Er blinzelte dem Schwarzen Herold zu. Der Herold, welcher bislang wortlos über Sagramak geschwebt hatte, ließ sich fallen und versenkte sein rostiges Schwert in Sagramaks Schädel. Die Schamanin blinzelte überrascht. Der Herold riss sein Schwert wieder heraus. Tarokdur fackelte nicht lange. Sein langer Schlangenhals beugte sich in die Tiefe und verschluckte die überraschte Schamanin samt Haut, Haar und Rüstung mit einem einzigen Schluck.

„Tu, was du nicht lassen kannst“, gluckste Tarokdur, „Doch tu es nicht mehr in diesem Leben. Man verhandelt nicht mit Drachen. Man folgt ihnen.“

Er warf den restlichen Drachenkultisten einen vielsagenden Blick zu. Diese sahen mit Furcht zum riesigen Eis-Drachen auf, der soeben ihre Anführerin in einem Happs verspeist hatte. Nehamal fasste sich als erster ein Herz und kniete schlotternd nieder.

„Gebietet über mich, o gewaltiger Gebieter.“

Fir tat es ihm gleich. Ausnahmsweise blieb das Ziegenwesen stumm.

Der Schwarze Herold kicherte, schwang sich auf Tarokdurs Rücken und machte es sich zwischen zwei Rückenstacheln bequem.

Sein Plan war geglückt, alle Gefahr beseitigt. Nun konnte er zurücklehnen und das Gemetzel der Andori genießen.\bigskip







Aćh zerrte am feurigen Takuri-Seil. Die gefesselte Siantari stolperte weiter. Turr drehte wie üblich seine Runden über der geschwächten Herrin des Kuolema-Gebirges und hinderte sie durch seine schiere Wärme, Kräfte zu sammeln. Turrs Hitze schmolz auch Teile der Schnee-Spur, den Ijsdur auf dem Weg durch das Rietgras hinterließ.

Turr und das Takuri-Seil waren die größte Beleuchtung in der finsteren Nacht.

Die unterkühlte Schulklasse hatten sie beim nächstgelegenen Gehöft ins Warme bringen können, wo ihnen müde Bauern heißen Cocoa servierten und ihnen Strohbettchen bastelten, in der Hoffnung, die traumatisierten Kinder eines baldigen Tages wieder in eine befreite Rietburg zurückbringen zu können.

Aćh und Ijsdur waren hingegen mit ihrer gefesselten Gefangenen weiter in den Norden aufgebrochen, auf dass die Rietburg auch möglichst bald befreit würde. Siantari hatte seit ihrem Aufbruch kein Wort mehr gesprochen und sich stolz stumm fortbewegt, für einmal nicht schwebend.

Ijsdur stapfte mit ein wenig Vorsprung zu den drei anderen in sicherer Kühle in Richtung Rietburg. Er starrte besorgt auf das nicht mehr allzu weit entfernte Gemäuer, dessen Silhouette kaum erkennbar war vor dem dunklen Sternenhimmel.

„Etwas tut sich!“, rief er zurück zu Aćh. Eine Zeit lang war der Eis-Drache der Drachenkultisten hinter den hohen Türmen der Festung nicht mehr zu sehen gewesen. Nun ragte wieder ein mit einem stattlichen Dämonengeweih besetzter schneeiger Drachenschädel neben dem Palas der Rietburg empor. Doch handelte es sich kaum um denselben Drachen. Dieser hier war um einiges größer, mit ausgeprägteren Rückenstacheln und einem unförmigeren Maul. Und als er seine löcherigen, gewaltigen Flügel ausbreitete und diese wie von einer inneren Magie zu leuchten begannen, da wusste Ijsdur, dass sie einer größeren Gefahr als zuvor bevorstanden.

Eine Hitzewelle kündigte an, dass Turr – und damit auch Siantari und Aćh – Ijsdur erreicht hatten.

Aćh zitterte: „Der ... der ist um einiges riesiger, als er zuvor aussah. Hat er an Stärke gewonnen?“

„Das ist nicht derselbe Drache. Ich befürchte, die Kultisten haben Taroks Knochen gefunden.“

„Vielleicht konnten sie eine friedliche Einigung erzielen und fliegen gleich davon?“

Ijsdur legte seinen Kopf schief. „Ich halte das für unrealistisch. Wir sollten uns lieber auf den schlimmsten Fall vorbereiten. Wir sollten davon ausgehen, dass Iril und Barz höchstwahrscheinlich versagt haben und Tarokdur kurz davor ist, seinen Zorn zu entfesseln.“

„Na dann, schnell, schnell“, spornte Aćh ihn an, rannte los und zerrte Siantari hinter sich her „Wenn wir die Rietburg rechtzeitig erreichen, können wir vielleicht noch etwas ausrichten. Wenn nicht, wird die Burg bald nur noch ein Schneehaufen sein. Und mit ihr das gesamte Rietland. Und danach ...“

Tarokdur öffnete sein Maul und spie einen gewaltigen Eisstrahl auf die Spitze des Königsturms. Steinblöcke wurden aus dem Turm geschlagen. Zwei Gestalten, die dort oben gestanden und Bögen gezogen hatten, wurden von einer Schneewelle ergriffen und vom Turm geschleudert. Aus der Entfernung wirken sie wie kleine Strohpuppen, die von einem Tisch zu Boden plumpsten.

Ijsdur blieb stehen und mahlte mit seinem Unterkiefer.

„Die Zeit reicht nicht. Wir kommen nicht mehr rechtzeitig zur Burg. Und selbst wenn – was wollen wir gegen einen Eis-Tarok anrichten?“

„Wir könnten Siantari dazu zwingen, ihn aufzulösen. War das nicht der Plan, warum wir sie mit uns schleppten?“

„Nicht genau. Siantari hat keinen Grund, unseren Anweisungen zu folgen. Für sie zählt nur das ewige Eis. Wir haben ihr nichts anzubieten außer ihre Freiheit, und die werden wir ihr nicht geben, da sie ist das größere Übel ist. Wir hätten vielleicht über die Drohung, Siantari zu vernichten, die Drachenkultisten zu etwas zwingen können. Aber nicht, wenn wir sie nicht mehr erreichen, ehe das Königreich dem Erdboden gleichgemacht wird. Nicht, wenn die Kultisten nun Taroks Willen folgen.“

Nun hielt auch Aćh an und drehte sich zu ihm um. Angst stand in ihr Gesicht geschrieben. Turr drehte weiterhin große Kreise über ihr. „Aber dann ... was können wir tun in der wenigen Zeit, die uns gegeben ist?“

Es gab nur noch etwas zu tun.

Es stand zu vermuten, dass Siantari die Kontrolle über diesen Eis-Drachen hatte, so wie sie alle Eis-Dämonen des Kuolema-Gebirges kontrollierte. Und wenn sie ihnen Kraft gab, wenn sie die Kraft ihres Willens und ihrer Magie war ...

Ijsdur stieß Siantari unsanft in den Rücken. Die Herrin des ewigen Eises ging vor ihm in die Knie.

„Ijsdur? Ijsdur, was machst du da?“, fragte Aćh argwöhnisch. Ihre Stimme wurde schriller. Ijsdur ignorierte sie und ließ seine rechte Hand von Eis überziehen.

Zum ersten Mal seit Stunden meldete sich Siantari wieder zu Wort. Kalt und emotionslos flüsterte sie: „Guck mir in die Augen. Du traust dich nicht. Wenn du mich endest, dann auch dich.“

„Darauf zähle ich!“, stieß er zwischen zusammengebissenen Zähnen hervor.

Vor seinem inneren Auge wägte er zwei Bilder ab.

Das eine Bild zeigte das Land Andor, von Schnee und Eis überdeckt. Einige wenige Überlebende, die um ein kleines Lagerfeuer gequetscht dasaßen und sich schlotternd um Proviant stritten, immer wieder furchtsam gen Himmel starrend. Hinter ihnen heulten der Wind und die Schneekreaturen. Ein überlebender Ijsdur starrte aus der Ferne auf sie. Sie starrten finster zurück.

Das zweite Bild zeigte hingegen ein unversehrtes, blühendes Land, mit glücklich herumspringenden Kindern. Kein Eis-Drache. Keine Eisdecke. Keine Eis-Dämonen.

„Ijsdur, halte ein. Lass uns noch einmal alle Optionen durchgehen“, meldete sich Aćh zu Wort.

„Keine Zeit“, presste Ijsdur hervor. Er wusste selbst, dass es diese paar Minuten keinen relevanten Unterschied mehr machen sollten. Er wollte sich wohl nur nicht länger als nötig die Gelegenheit geben, sich von Furcht übermannen zu lassen.

Ijsdurs Hand schnellte nach vorne und brach in Siantaris Rücken hinein. Tausend Nadelstiche trafen ihn, doch ließ er sich davon nicht abkriegen. Er spürte etwas Festes, Scharfes, griff zu und zog seine Hand zurück.

Siantari gab keinen Laut von sich, als Ijsdur den faustgroßen blauen Kristall aus ihrem Körper riss. Der Edelstein war ungeschliffen, roh und von einem finsteren Glimmen umgeben.

Siantaris eisiges Herz. Die Quelle ihrer Kraft. Die Quelle von Tari.

Siantari sprach, und doch war es nicht nur sie. Eine tiefe Stimme summte aus der Tiefe des Eiskristalls hervor und drang tief in Ijsdurs Geist hinein. Synchron zu Siantaris Lippen sprach die Stimme des ewigen Dämons in Ijsdurs Schädel:

„Guuut. Es ist an der Zeit, Sian loszulassen. Lass mich dich stärken. Lass mich dich zu Ijstari machen. Dann kannst du den Eis-Drachen befehlen. Dann kannst du das ewige Eis selbst befehlen. Es kann eine Koexistenz geben zwischen dem Eis und den Hitzköpfen. Tu nichts, was du bereuen könntest.“

Glücklicherweise würde Ijsdur, wenn alles gut lief, diese Entscheidung nie bereuen können. Einen Augenblick lang haderte er. Doch er kannte die Geschichten über Tari und die gemeinen Dämonen. Wenn er jetzt einen Pakt schlösse, wäre er in Kürze wieder vom ewigen Eis überzeugt.

Er holte tief Luft und ließ einen gutturalen Schrei seiner Kehle entfliehen. Entschlossen hielt er Taris Kristall gen Himmel. Merkwürdig, dachte er, dass ihm jetzt gerade die Schneeflocken auffielen, die wie gefrorene Tränen um hin herumwirbelten.

Dann wünschte er sich ganz fest, der Kristall möge von einem Blitz getroffen werden. Es knitterte und zischte. Aus Ijsdurs Fingerspitzen sprangen glühend weiße Blitze, sprangen auf den Eiskristall über und ließen dessen raue Oberfläche splittern. Ijsdur drückte zu und der Kristall zersprang in tausende kleinste Splitter, welche sich wie Rauch auflösten.

Siantari hauchte ein letztes Mal aus und kippte zur Seite. Ihr Schneekörper war bereits zerfallen, ehe er auf dem Boden aufschlug.

Ein fernes Brüllen und Fauchen berichtete von den Schmerzen des Eis-Drachen über der Rietburg.

Ijsdur sah Aćh ein letztes Mal an. Er spürte eine Schwäche in sich aufsteigen und ein Kribbeln in seinen Fingerspitzen. Er hätte sich passende letzte Worte überlegen sollen. Aber es war auch in Ordnung so.

Sein Mund verzog sich zu einem letzten Lächeln. Ijsdur schloss seine Augen und ließ sich in die Dunkelheit sinken.

Er hatte so einigen Schmerz in seinem Leben ausgelöst, doch dieses Opfer sollte das wieder gut machen. Ohne Siantari keine Eis-Dämonen. Kein Ijsdur. Kein Eis-Drache.

Frieden.

Er konnte im guten Gewissen sterben, Frieden über diese Welt gebracht zu haben.

Die Schwäche übermannte ihn.

Die Dunkelheit überkam ihn.

Dann schwand sein Bewusstsein.\bigskip







„Ijsdur? Ijsdur, wach auf!“

Ijsdur schlug seine Augen auf und blinzelte verwirrt gegen den dunklen Sternenhimmel, vor dem sich ein brennender Feuervogel abzeichnete, und wiederum davor ein dunkler, besorgter Menschenkopf. Ein wunderschönes Bild, das sich wie in Zeitlupe bewegte. Und eines, das ein toter Eis-Dämon nicht mehr erlebt hätte.

„Ijsdur? Lebst du noch?“, fragte Aćh erneut.

Das Kribbeln in Ijsdurs Fingerspitzen legte sich.

„Ach, verdammt“, fluchte er. Während er sich wieder aufrichtete, erkannte er wie befürchtet immer noch einen über der Rietburg herumflatternden eisigen Tarokdur. Er zog sich an Aćhs Arm in die Höhe. Siantaris Körper lag verschrumpelt und zerfallen vor ihnen. Die Dämonin war tot. Doch als Ijsdur an seinen Hals griff, spürte er noch immer die Kraft, die aus seiner eigenen Eiskristallkette durch ihn floss und ihn mit Entschlossenheit erfüllte.

Siantaris Tod hatte nichts gebracht. Die Eiskristall-Ketten konnten ohne sie existieren. Ijsdur und der Eis-Drache konnten ohne sie existieren.

„Das war sehr tapfer von dir“, flüsterte Aćh.

„Und ineffizient“, gab Ijsdur bissig zurück.

„Verzage nicht. Noch besteht Hoffnung. Auf, weiter zur Rietburg mit uns!“

Ijsdur verdrückte eine einzelne eiskalte Träne, die an seiner Wange festfror. Keine Zeit, sich jetzt mit seinen Gefühlen zu beschäftigen. Es gab weiterhin einen Eis-Drachen zu stoppen.

„Immerhin brauchen wir Turr nicht mehr hier, um Siantari zu schwächen“, meinte Aćh. Sie zog ihre tulgorische Steinflöte aus ihrer Manteltasche und spielte eine wilde Melodie, die abrupt endete.

Turr verschwand in einem Flammenwirbel und sauste in Richtung Rietburg davon.

Aćh und Ijsdur eilten ihm hinterher.

Sie hinterließen einen zerfallenen Schneehaufen, der einst Siantari gewesen war. Langsam schmolz er in den ersten Sonnenstrahlen des neuen Tages dahin.\bigskip







Tarokdur schwebte über der Rietburg. Mit Schlägen seiner gewaltigen Schwingen fegte er Gras, Stein und selbst das eine oder andere Lebewesen im Burghof umher. Was war dies für ein eigenartiges Gefühl in seiner Brust gewesen? Schwäche hatte ihn überkommen, und zugleich war ein gewaltiges Gewicht von seiner Seele genommen werden. Verwirrt blickte er um sich.

„Fokus, mein Gebieter“, zischte die durchdringende Stimme des Schwarzen Herolds.

Tarokdur drehte seinen Kopf auf dem langen Schlangenhals und blickte aus blau glühenden Augen auf Thorald herunter, der von Nehamal an der Schulter festgehalten wurde. Im Gegensatz zu herumrennenden Hunden, Pferden und vereinzelten Burgbewohnern machten die Winde von Taroks Schwingen Nehamal und den restlichen Kultisten kaum etwas aus.

Der Schwarze Herold fläzte sich triumphierend auf dem Rücken des Eis-Drachen und zischte mit seiner blechernen Stimme: „Nun, mein Meister, was wollt Ihr zuerst tun? Die Rietburg in Schutt und Asche legen? Oder den Sohn desjenigen ermorden, der Euch einst bezwang?“

Tarokdur ließ seinen gewaltigen Schädel zwischen dem Königsturm und Thorald hin- und herschwenken, als würde er Ene-Mene-Mu spielen. Er kam vor Prinz Thorald zu stehen, welcher von Nehamal zu Boden gepresst wurde und vergeblich umherwackelte.

Thorald schrie jämmerlich auf, als der Blick des Eis-Drachen auf ihn fiel. Dann jedoch straffte er seinen Körper und blickte Tarokdur stolz in die Augen. „Mein Vater fürchtete dich nicht. Und ich fürchte dich nicht.“

Sein zitternder Körper strafte seine tapferen Worte Lügen. Dennoch nickte der Herold anerkennend. Er reckte seine Faust in den Himmel, als wolle er Tarokdur den Befehl zum Töten erteilen – obwohl inzwischen klar war, dass niemand Tarokdur Befehle erteilen würde. Tarokdur richtete seinen langen Schlangenhals auf. Ein immer lauter werdendes Knistern drang daraus hervor, während er seinen tödlichen Eisatem sammelte.

Da streckte die aufgehende Sonne ihre ersten Strahlen über die Mauer der Rietburg und erhellte des Herolds erhobene Faust ebenso wie des Drachen gewaltigen Kopf. Die Kälte der Nacht wich langsam der Wärme des Morgens.

In weiter Ferne erhob sich der glühende Himmelskörper über die Ausläufer des Grauen Gebirges im Osten und sandte die ersten Sonnenstrahlen des neuen Tages über die Zinnen der Burg, an blinzelnden Kultisten vorbei und bis in den staubigen Kerker der Rietburg.

Die wenigen noch nicht verspeisten Mondbeeren an den Wänden der Kaverne glühten auf. Doch Barz pflückte keine mehr. Der Steppennomade lag in sich zusammengesunken auf einem Strohhaufen an einer Wand des Kelterraums und schnarchte vor sich hin.

Nicht so Iril. Sie war hellwach und starrte aus der Gittertür in den Rietburghof hinaus. Sie sondierte die Lage. Sagrakdur und Sagramak waren beide nicht mehr. Von den Kriegern der Rietburg war nichts zu sehen. Tarokdur war kurz davor, den ungekrönten Regenten Andors die Narne hinunterzuschicken. Jetzt war die Zeit, zu handeln.

Die Tür zum Kellergewölbe zerbarst unter einem einzigen Schlag ihres leuchtenden Runenhammers. Die verbogene Gittertür schlidderte über den vereisten Burghof bis vor Nehamals Füße.

Tarokdur blickte ebenso überrascht wie die restlichen Anwesenden – und wie Thorald, in dem ein Funken Hoffnung entfachte.

„Ist schon ein Wunder, was man mit ein paar Stunden Zeit eingesperrt in einem Kerker alles anstellen kann“, lächelte Iril zu sich selbst, „Sofern man die richtigen Mittelchen hat.“ Sie drehte ihren letzten Runenstein, den grünen, zwischen Zeigefinger und Daumen umher. Dann drückte sie zu und pulverisierte auch dieses Steinchen so mühelos, als wäre es einer dieser staubtrockenen Kekse, die man dem Santa Gor zur Winterzeit vor die Tür stellte.

Magischer Staub stieg von den Pulverresten des Runensteins auf und wehte um Irils Hand herum. Ihre Muskeln, welche schon zuvor unnatürlich angeschwollen gewesen waren, spannten sich noch weiter.

Irils Runentattoos hatten bereits gelblich und bläulich geleuchtet. Nun schossen Schlieren in allen möglichen Farben der Magie aus Irils Tattoos hervor und überzogen ihren Körper. Das Weiß in ihren Augen wurde von einem sich immer veränderndem Wirbel von leuchtenden Farben überdeckt.

„Ergebt euch und händigt die Knochensplitter aus“, verlangte Iril. Ein magisches Echo ihrer Stimme hallte zwischen den Mauern der Rietburg umher.

Stolz trat sie vor die versammelten Kultisten. Täuschte sie sich, oder erzitterte der Erdboden tatsächlich, als sie auf ihn trat?

Ein Drachenkultist in einem langen Kapuzenmantel zog einen Dolch hervor und sprang Iril an. Diese schlug mit dem Hammer tadelnd den Dolch weg, packte den überrumpelten Angreifer mit ihrer freien Hand und schleuderte ihn mehrere Meter weit. Die volle Kraft der Runen pulsierte durch ihren Körper. Wäre sie ein klein wenig größer gewesen, hätte sie den Gegner zuvor theatralisch in die Luft gehoben.

Ein Raunen ging durch die Reihen der restlichen Kultisten. Tarokdur starrte Iril amüsiert an. Nehamal ließ Thorald los. Und als wäre dies ein Signal gewesen, öffnete sich die verbarrikadierte Tür zu einem Turm. Eine letzte Handvoll verzweifelter andorischer Krieger jagte hervor und trieb die überraschten Kultisten zurück.

Tarokdur wirbelte zu den neuen Angreifern herum und sammelte seinen Atem. Iril zweifelte nicht daran, dass er auch seine eigenen Anhänger vereisen würde, um die Andori zu erwischen. Diese durften nicht seine Priorität werden.

„He, Schnee-Tarok!“, rief Iril, und ihre laute Stimme hallte zwischen den Mauern der Rietburg umher, „Flatternder Feigling! Komm auf den Boden der Tatsachen zurück und stell dich doch einer wirklichen Herausforderung!“

Schnee-Tarok wandte seinen massigen Kopf vom zitternden Thorald und dessen tapferer Rietgarde ab und wandte sich Iril zu. Seine eisblauen Augen funkelten böse. Der Schwarze Herold kicherte leise.

„DU WICHT DENKST, DU KÖNNTEST ES MIT TAROKS MACHT AUFNEHMEN?“, knirschte die tiefe Stimme des Drachen in Irils Ohren.

„Mit Taroks Macht ziemlich sicher nicht“, gab Iril zu, „Aber Taroks Macht trägst du auch nicht zur Gänze in dir, oder? Du, Tarokdur, bist nichts als eine Schneefigur, ein schwächlicher Abklatsch des wahren Taroks, gestärkt durch nichts als ein paar mickriger Fußknochen. Dich mache ich im Schlaf fertig.“

Selbstbewusstsein konnte Berge versetzten, das hatte Runenmeisterin Burmrit immer wieder gesagt. Und es konnte unbesonnenere Gegner in Rage versetzen. Auch wenn es nur gespielt war.

Mit einem lauten Krachen plumpste Tarokdur auf die freie Fläche vor Iril. Der Boden erzitterte. Seine Schwingen klappten ein. Immerhin erwischte er kein weiteres Haus bei dieser Landung. Doch sah er nicht mehr amüsiert aus.

Blitzschnell schoss Tarokdurs Kopf herunter, um Iril mit einem einzigen Happs zu verschlucken.

Iril blinzelte.\bigskip







Entfernt vom Geschehen lag Barz weiterhin auf einem Strohhaufen an der Kellerwand und schnarchte tief. Eine bräunliche Pulverpaste verkrustete seine Nase. Meditationspulver. Unter Barz‘ Lidern rollten seine Augen umher, als träume er stark.

Verwaschene Bilder schossen durch seinen halbwachen Geist. Eindrücke der Gesichtsfelder von Menschen und Zwergen, die Klingen ineinander vergruben. Schreiende Kinder in kaputten Häusern. Nabib, der in einem behelfsmäßigen Bett umherrollte. Und Iril, die einem gewaltigen Drachenkopf entgegensah.

Das Feuer der Furcht jagte durch Barz‘ Körper. Diesen Drachenkopf kannte er nur zu gut, auch wenn er nun schneeweiß war. Zuletzt hatte er ihn gesehen, als Tarok die Steppe seiner Heimat in Brand gesetzt hatte. Barz fasste sich, so gut sein müder Geist dazu überhaupt in der Lage war. Tarok war tot. Und was auch immer hier vorging, dies war Barz‘ Gelegenheit, Jirisa zu rächen.

Barz‘ Augen rollten noch schneller unter ihren Lidern umher. Die Pulverpaste um seine Nase begann, magisch zu glitzern.\bigskip







Iril blinzelte.

Sie fühlte sich leicht. Noch immer jagte das Feuer der Runen über ihre Haut, stärkte jeden ihrer Schritte und Schläge. Noch immer stiegen um sie herum vielfarbige magische Ströme aus dem Boden und pumpte sie voller Kampfkraft. Doch war dies anders. Tarokdurs gewaltiger Kiefer, der auf sie zujagte, wirkte auf einmal träger. Seine Bewegung verlangsamte sich.

Oder waren es Irils Bewegungen, die sich beschleunigten?

Farbiges Glitzern überzog ihre Haut, als Iril locker zur Seite trat und Tarokdurs lange Zähne neben ihr ins Leere schnappten.

Tarokdur hob eine Pranke, holte aus und wie im Traum bewegte sich der tödliche Schlag auf Iril zu. Iril tat einen Sprung, der über zwergische Fähigkeiten hinausging, und entkam dem mächtigen Schlag.

Iril federte den übernatürlich hohen Sprung mit einer übernatürlich starken Landung ab. Dieses magische Werk war nicht allein auf ihre Runen zurückzuführen. Ihre zauberhafte Geschwindigkeit kam von Barz. Barz und seinem Meditationspulver. Und gute Güte, fühlte es sich gut als.

Der Drache stellte sich auf die Hinterbeine und richtete, hoch wie ein Berg vor Iril aufragend, seine Flügel auf.

Mit doppelter Geschwindigkeit huschte Iril zur Seite, ehe Tarokdurs Tatzen sich an derjenigen Stelle in den Rietboden grub, wo sie soeben noch gestanden war.

Das magische Glitzern auf Irils Haut ließ noch nicht nach, ebenso wenig ihre übernatürliche Geschwindigkeit. Erneut schnappte Taroks stinkender Schädel an Iril vorbei.

Die Runenmeisterin sprang auf den Kopf des Eis-Drachen und rannte seinen Hals entlang, während Tarokdur sich unter ihr schüttelte und sie abzuwimmeln versuchte. Dass er sich halb so schnell bewegte, vereinfachte den Balanceakt für Iril, machte ihn allerdings nicht zu einem Kinderspiel.

So sehr war Iril damit beschäftigt, nicht von den langen Rückenstacheln Tarokdurs erwischt zu werden, dass sie beinahe den Schwarzen Herold übersehen hätte.

Erst im letzten Moment duckte sich Iril unter der langen Klinge des Herolds hinweg. Der Herold zischte langsame Beleidigungen hinter seiner gezackten Maske hervor und stach erneut nach Iril.

Diese schwang ihren Hammer in seine Richtung, doch natürlich hielt der Herold einen angemessenen Sicherheitsabstand. Und gleichzeitig auf ihn und auf die Bewegungen von Tarokdurs Kopf achten konnte Iril kaum. Sie machte einen Misstritt, verlor ihren Halt und stürzte in Richtung Erde.

Im Fall sammelte sie ihre Gedanken. Herold und Eis-Drache, beide konnte sie potenziell mit ein bisschen Runenmagie aus dieser Sphäre der Realität bannen. Doch der Drache war das aktuere Problem. Er musste ihre Priorität bleiben.

Von magischem Glitzer und leuchtend vielfarbigen Schlieren verfolgt, landete Iril auf dem schneebedeckten Boden, rollte sich gekonnt ab und stieß sich gleich wieder zu einem übernatürlich hohen Sprung in die Höhe.

Eine Drachentatze verfehlte sie nur knapp. Der schwebende Herold hatte nicht mit Irils hohem Sprung gerechnet. Iril klammerte sich an seinen flatternden Umhang und riss den Herold in Richtung Boden. Jetzt, wo sie ihm so nahe war, drang ein unangenehmer Geruch nach Verwesung in ihre Nase. Bei allen Kreaturen der Tiefe, was war dies für ein Wesen?

Herold und Iril knallten wieder auf die schneebedeckte Erde. Iril presste die schwertführende Hand des Herolds mit einem Knie nach unten und versetzte dem Wesen einen hoffentlich betäubenden Hammerschlag auf die Maske. Ihre leuchtenden Runentattoos spiegelten sich dumpf darin. Die eiserne Maske brummte tief wie eine große Glocke. Darunter knackten Knochen. Der Herold ächzte. Man konnte ihm also doch schaden. Doch Iril wusste von Orfen, dass der Herold schon viel Schlimmeres überlebt hatte.

Da erwischte auch schon ein stacheliger eiserner Handschuh Iril und fegte sie zur Seite. Der Herold erhob sich, weiterhin vor Schmerzen ächzend, und flog wie von Zauberhand wieder in die Luft. Und nun, wo sie nicht mehr auf seinem Diener lag, entlud Tarokdur wieder seinen Zorn über sie.

Immerhin waren seine Eisstrahlen viel fokussierter, als Taroks feuriger Drachenatem es gewesen wäre. Geschwind wich Iril dem knisternden Eisstrahl aus und huschte unter den Drachen.

Mit einem gewaltigen Satz schleuderte Iril ihren Hammer auf den ungeschützten Bauch des Drachen. Mit einem blitzschnellen Hieb des stachelbewehrten Schwanzes parierte Tarokdur die Attacke. „Ich hoffe, das kannst du besser“, höhnte er ihr entgegen.

Er richtete sich zu seiner vollen Größe auf und spie einen mächtigen Eisstrahl. Iril versuchte, zur Seite zu springen – zu langsam! Von dutzenden Eiszapfen getroffen, wurde sie zur Seite geschleudert und prallte gegen eine steinerne Hausmauer. Tarokdur lachte höhnisch. Schnee und Eis regneten auf Iril herunter und drückten sie zu Boden, hinderten sie am Aufstehen.

Es wurde kalt um Iril, während sie verzweifelt versuchte, sich freizuschaufeln. Dabei rührten sich ihre Gliedmaßen kaum. Zu schwer war das Gewicht des Schneehaufens auf ihr. Zu stark drang die klirrende Kälte in ihre übermüdeten Glieder. Iril öffnete ihre Augen und erblickte nichts außer einer Schicht blauweißen Schnees vor sich.

War dies das Ende von Irils Geschichte?

Da erklang ein freudiges Kreischen. Wärme durchfuhr Irils Glieder. Der weiße Schnee vor ihrem Gesicht schmolz beiseite und gab den Blick frei auf ein wunderschönes goldenes Wesen, von leuchtenden Flammen übersät, das auf dem Schneehaufen saß, Iril befreite und dabei allerlei gurrende Laute von sich gab.

Turr war gekommen, um sie zu retten, und er war größer und strahlender denn je. Hoffentlich hieß das, dass Aćh und Ijsdur erfolgreich gewesen waren.

Während die letzten Reste von Irils Schneehaufen zu einer unscheinbaren Wasserpfütze zusammenschmolzen, breitete Turr seine leuchtenden Flügel aus und erhob sich in die Lüfte. Er drehte seine Runden über dem gewaltigen Eis-Drachen. Dessen gewaltiger gehörnter Kopf folgte dem pfeilschnellen Feuervogel wie gebannt.

Turr schoss in die vom hellen Morgenlicht erstrahlten Wolken hinauf und wieder daraus herab. Für einen Augenblick verschwand er hinter den Zinnen der Rietburg. Sowohl Iril als auch Tarokdur blickten ihm verdutzt hinterher. Dann schwirrte der Takuri aber auch schon wieder durch das Tor der Rietburg und auf den Eis-Drachen zu.

Sein Gefieder brannte nicht mehr feuerrot, sondern violett. Turr hatte sich am ewigen Feuer in der Schale vor der Rietburg gelabt. Seine golden glänzenden Federn waren von lilafarbenen Flammen übersät. Er sah wunderschön aus.

Das ewige Feuer, Symbol für die Stärke der Andori, konnte auch ein Symbol für die Stärke eines gewissen Feuervogels werden. Violett leuchtend drehte Turr eine Runde um die Rietburg, ehe er seine Flügel zusammenklatschte und... nein, es war keine Explosion und kein Feuerball, sondern ein kontrollierter violetter Flammenwirbel, der auf Tarokdur zujagte, sich sauber durch den linken Flügel des Eis-Drachen hindurchbrannte und ein gewaltiges Loch in seiner Seite hinterließ.

Tarokdur brüllte auf und kippte zur Seite. Turr drehte eine weitere Runde.

Doch ehe Iril sich versah, schlossen sich Tarokdurs Wunden auch schon wieder. Iril wusste zwar von Ijsdur, dass dies mehr eine kosmetische denn eine tatsächliche Heilung war. Nichtsdestotrotz wurde ihr mulmig im Magen. Der Eis-Drache richtete sich zu seiner vollen Größe auf, breitete seine Flügel aus und schwang sich in die Höhe, den Schlangenhals weiterhin auf Turr gerichtet. Dieser flatterte inzwischen eher panisch als graziös. Dutzende Zwergenhöhen über der Rietburg drehte sich Turr um, breitete seine Flügel aus und sammelte Kraft für einen weiteren Feuerwirbel ... da schnappte der Kiefer Tarokdurs zu und verschluckte den Takuri zur Gänze. Kurzzeitig schwappte lilafarbenes Feuer aus Tarokdurs Maul, dann war es auch schon wieder verschwunden. Der Eis-Drache leckte sich die Lippen.

Wenn Sagramak noch am Leben gewesen wäre, hätte sie sich sicher jetzt schmerzerfüllt aufgeschrien.

Doch nicht nur Sagramak hatte Turr zum Sterben süß gefunden. Da gab es auch zwei rasch näher zur Rietburg rennenden Gestalten, eine davon mehr außer Atem als die andere.\bigskip







„Turr!“, schrie Aćh auf, und nahm nochmal an Tempo zu. Ihr Atem ging inzwischen unregelmäßig.

Ijsdur, der neben ihr halb rannte, halb über eine Eisspur am Boden glitt, blickte ebenso entsetzt in den Himmel.

Die beiden waren nur noch einige Minuten von der Rietburg erreicht. Weit über ihnen und weit über der Rietburg schmatzte Tarokdur vergnügt an etwas, das unzweifelhaft ein violett brennender Turr gewesen war. Dann sank er flatternd wieder etwas weiter in die Tiefe. Diesmal landete der Eis-Drache jedoch nicht auf dem Boden, sondern hielt sich mit windigen Schwüngen seiner Schwingen in der Luft. Seinen Schlangenhals beugte er nach unten. Eisstrahlen spie er, einen nach dem anderen, doch keiner davon schien sein von der Burgmauer verdecktes Ziel zu treffen. Er brüllte frustriert auf.

„Was tut er?“, fragte Aćh Ijsdur schnaufend. Dieser zuckte wortlos mit den Schultern.

Da! Eine kleine Gestalt war für kurze Zeit über den Zinnen der Rietburg sichtbar und verschwand gleich wieder dahinter. Dann tauchte die Person erneut auf. Und dann noch einmal. Es sah aus, als hüpfe jemand Tarok entgegen und versuchte, ihn zu erreichen. Aber das war natürlich Unsinn, kein Mensch konnte ein Vielfaches seiner Körpergröße in die Höhe hüpfen. Und die Sprungfähigkeit von Zwergen war noch schlechter. Ganz abgesehen davon, dass die Gestalt in allen Regenbogenfarben strahlte und glitzerte. Ein magisches Wesen war dies, unzweifelhaft.

Und doch kam Aćh der aus der Entfernung nur knapp erkennbare Hammer in der Hand der Gestalt äußerst bekannt vor.

„Ist das etwa Iril?“, fragte Aćh verblüfft.

„Was für andere hammerschwingende Zwerge kennst du noch?“

„Aber seit wann kann sie so hoch springen?“

„Vermutlich stärkt sie sich mit Runenmagie“, zuckte Ijsdur mit den Schultern, „Du hast ihr doch entsprechenden Steinchen geschickt. Dumm ist nur, dass sie immer noch nicht zum Drachen kommt. Fliegen ist schon ein unnatürlicher Vorteil im Kampf.“

Er lachte auf. „Iril lebt noch! Der Kampf ist noch nicht vorbei. Wir haben eine Chance. Komm, rasch, weiter zur Rietburg, vergeuden wir sie nicht.“

„Moment mal“, murmelte Aćh, „Iril könnte deine Magie eher benötigen als dein Schwert. Damals, als Barz und ich beim Felsentor auf Nesdora stießen, konnte diese mit einem Eisblitz eine riesige Eisbrücke wachsen lassen, damit wir über eine tiefe Schlucht steigen konnten. Siantari schaffte es, aus weiter Entfernung Eisblitze im Rietland niedergehen zu lassen. Sagramak ist nicht einmal eine Eis-Dämonin, und sie konnte dennoch mithilfe der Eiskristallketten einen riesigen Schnee-Drachen erschaffen. Und Ijs kannte sich eigenen Angaben nach aus mit den Brücken Tulgors wie kein anderer. Siehst du, worauf ich hinauswill?“

„Ich glaube, schon. Diese Eiskristallketten sind durchaus sehr mächtig. Doch habe ich noch nie versucht, durch Eisblitze ein Konstrukt herbeizurufen. Erst recht kein so riesiges wie eine Brücke. Oder über eine solche Distanz“, gab Ijsdur zu.

„So weit weg sind wir gar nicht“, sprach Aćh ihm gut zu. „Wann probieren, wenn nicht jetzt? Ich glaube an dich!“

Ijsdur blieb entschieden stehen und konzentrierte sich. Er schloss seine Augen, legte seinen Kopf in den Nacken, breitete seine Arme aus und schrie etwas Unverständliches in den heulenden Wind. Schnee, Eis und Kälte breiteten sich von ihm aus wie noch nie zuvor, sodass Aćh zurückweichen musste. Ijsdurs Körper begann, bläulich zu schimmern, gar taghell zu leuchten. Dann zeigte Ijsdur in den Himmel. Funken zuckten zwischen seinen Fingern und glitzernde Eiskristalle stieben in die Höhe. Über der Rietburg donnerte es, als dunkle Wolken sich versammelten und zu drehen begannen.

Erschöpft plumpste Ijsdur zu Boden. Über ihm zuckte ein riesiger Blitz über den Himmel, knapp an Tarokdur vorbei, und schlug mit Karacho in den Boden. Die Erde zitterte. Vom Einschlagort wuchs unglaublich schnell eine Eisbrücke in die Höhe, spiralförmig, vielstufig, wunderschön. Sie jagte an Tarokdur vorbei, der ihr verblüfft hinterherguckte und beinahe vergaß, seine Flügel zu schlagen.\bigskip







Iril war gerade ächzend auf dem Boden gelandet, als Ijsdurs Blitz eingeschlagen hatte. Kurz hatte sie ihn für einen weiteren Eisstrahl Tarokdurs gehalten, doch war dem nicht so gewesen. Unter lautem Getöse wie von berstendem Eis wuchs unter Irils staunendem Blick ein eisiges Konstrukt in die Höhe. Staunend bestarrte sie ein architektonisches Wunderwerk aus purem Eis, eine Dutzende Meter hochragende, dünne, spiralförmige, durchscheinende, magisch glitzernde Treppe.

Ein wie aus dem Nichts kristallisiertes Wunder.

Iril war niemand, der ein Wunder vergeudete.

Das magische Glitzern von Barz‘ Pulver überzog Iril aufs Neue. Unnatürlich rasch sauste Iril die Eisbrücke (oder vielmehr Eiswendeltreppe?) hoch und an Tarokdur vorbei. Hoch oben endete das Gebilde in einer Art offnen Plattform. Iril raste über die Brüstung hinaus und sprang von oben auf den völlig verdutzten Tarokdur herunter.

Sie lächelte.

Dann prallte sie auf den Rücken des Eis-Tarok und wurde gleich wieder meterhoch in die Luft geschleudert, knallte an einen seiner mächtigen Flügel, der sich gerade am Heben war, und rutschte daran herunter, bis sie am unteren Flügelansatz saß, beinahe nicht mehr bei Bewusstsein.

Iril schnappte nach Luft. Sie hatte Mühe, die Situation zu begreifen. Sie lag auf dem Rücken von Tarok. Sie lag auf dem Rücken eines vom abhebenden Drachen. Man könnte argumentieren, dass sie einen Drachen ritt – sie, eine einfache Silberzwergin!

Monströse, messerscharfe Schuppen bedeckten den gigantischen Körper. Iril achtete darauf, nicht auf ihre Kanten zu treten.

Ihre Augen erfassten eine Bewegung. Leise flatterte ihr der schwarze Herold entgegen. Iril hielt ihren Hammer bereit. Sie schlug zwei, dreimal auf den Herold ein. Sein langes Schwert verbeulte. Iril zeigte ihm, dass er sie diesmal nicht aus dem Gleichgewicht bringen konnte. Bevor sie eine passende Runenscheibe ziehen und das garstiges Gespenst aus dieser Sphäre zu vertreiben versuchen konnte, sprang der Herold von Tarokdurs Rücken und schwebte davon.

„Feigling!“, rief Iril ihm hinterher.

Unter ihr brüllte Tarokdur etwas und verdrehte seinen Hals, um Iril von seinem Rücken zu beißen. Sie huschte von seinen langen Zähnen zurück und klammerte sich an seinen Körper, an eine Stelle kurz vor seinem Halsansatz, die Tarok selbst unter größten Verrenkungen nicht erreichen konnte.

Und da fand sie sie. Die Eiskristallkette, die Tarokdur mit ihrer Magie einhüllte. Sie war der Ursprung seiner Kraft, seines starken Willens und seiner Gedanken.

Ein so kleines Ding, in diesem riesigen Körper eingelassen. Die Quelle so großen Unheils.

Iril ließ sich nieder, kuschelte sich in den nasskalten Schnee des Eis-Drachen und ließ ihren Runenhammer mit Magie aufheizen.

Die Welt stellte sich Kopf, als Tarokdur eine Rolle drehte, um Iril von seinem Rücken zu schütteln. Gerade noch rechtzeitig versenkte Iril ihren Hammer tief in Tarokdurs Fleisch, ließ ihn sich verankern, sich von ihm mitführen. Fliehkräfte zerrten Iril in die eine und andere Richtung, die Rietburg tief unter ihr wurde kleiner, Möwen stürzten ihr entgegen. Tarokdur flüchtete panisch in die Wolken.

„INSEKT! WURM! SPORNFRASS!“, schrie er.

Doch Iril, magisch gestärkt durch die Kraft der Runen, ließ ihren Hammer nicht los. Und ihr Hammer war magisch verankert im Eis-Drachen. So leicht konnte er sie nicht loswerden.

Mit ihrer freien Hand öffnete Iril ihre Reisetasche und zog einen kleinen Eisenstab hervor. Dann beugte sie sich vor und ritzte eine bestimmte Runenfolge in Tarokdurs eisige Schuppen, rund um die Eiskristallkette herum. Runen im Schnee. Ein Glück, dass sie diese bestimmte Runenfolge zuvor noch gebüffelt hatte. Bei diesem Geflatter konnte sie ihre Notizen kaum studieren.

Sie wartete, bis Tarokdur sich wieder richtig orientiert hatte. Dann riss sie ihren Runenhammer frei und ließ ihn gleich wieder auf die Eiskristallkette niederfahren.

Magische Ströme in allen Regenbogenfarben sprangen vom Hammer auf die Runenfolge über und ließen sie gleißend hell aufleuchten. Tarokdur brüllte. Ein letztes Ächzen drang aus dem riesigen Maul. Damit schied Tarokdur, der letzte Drache, für immer aus dem Leben.

Iril zerschlug Tarokdur. Oder vielmehr zerschlug sie ihn weniger, als dass er einfach wässerig wurde, seine Form verlor, ja, schmolz, und als riesiger Haufen Schnee-Matsch vom Himmel fiel. Ihre Runen taten wieder einmal, was sie sollten.

Iril wurde ganz leicht ums Herz, als ihre Haare und die Runenscheiben aus ihrer Reisetasche zu schweben begannen. Eiskristalle wirbelten um sie herum. Sie befand sich im freien Fall. Ihr Magen kehrte sich um.

„WEG DA!“, brüllte Iril unten, in der Hoffnung, die wenigen im Rietburghof Anwesenden sähen die Gefahr rechtzeitig.

Ijsdurs wunderschöne Eisbrücke knackste unschön, als Tarokdurs Schneekörper daran abprallte und herunterrollte.

Der Aufprall wurde vom vielen Schnee gedämpft und war dennoch heftig. Iril hatte sich mit unnatürlicher Geschwindigkeit und Stärke darauf vorbereitet und rollte den Fall gut ab, als sie gemeinsam mit Tonnen an Schnee, Eis und Wasser auf neben dem Palas der Rietburg einschlug.

Etwas benommen brach Iril aus dem Schneehaufen hervor, der soeben noch Tarokdur gewesen war. Sie blickte sich um.

Ein glimmendes Turr-Küken kugelte aus dem eisigen Schneehaufen hervor und schüttelte seine Flügelchen. Daneben erkannte Iril eisige Überreste einer Leiche, die einst wohl Sagramak gewesen war.

Doch weiter vorne war das relevante Artefakt.

Da lag sie, die elende Eiskristallkette, die Tarokdur erweckt hatte, und in welcher sein Geist nun womöglich eingesperrt war. Iril schaufelte sich frei und richtete ihren Hammer drohend auf Nehamal. Denn der Drachenkultist hatte sich ungläubig dem Schneehaufen genähert und war kurz davor, die Eiskristallkette hervorzuklauben.

Iril ließ von Sagramaks Leiche und dem Turr-Küken ab, stolperte den Schneehaufen herunter und reckte drohend ihren Hammer in die Höhe.

„Lass die Kette sein“, befahl sie. Nehamals starrte die ausnahmsweise auf Augenhöhe stehende Zwergin finster an und langte nach seinem Degen. In seinen Augen spiegelte sich die Kraft der Runen, welche weiterhin farbenfroh auf Irils Haut herumwirbelte.

„Sei kein Idiot“, murmelte er mehr zu sich selbst als zu Iril. Sein dunkler Umhang bauschte sich auf, als Nehamal herumwirbelte und davonrannte. Einige der restlichen Drachenkultisten, die nicht von andorischen Kriegern überrumpelt worden waren, taten es ihm gleich. Durch das offene Tor der Rietburg sprinteten sie in eine ungewisse Zukunft.

Iril ließ sie ziehen. Für heute hatte sie genug vom Tod und Verderben.

Nun galt es nur noch, die finstere Kette zu vernichten. Iril beugte sich nieder und hob Tarokdurs Eiskristallkette, um sie in ihrer bloßen Faust du zerquetschen. Oder besser gesagt, wollte sie das. Denn in demjenigen Augenblick, in dem die Kette ihre Hand berührte ...\bigskip







... hielt Irildora inne. Die Strapazen der letzten Minuten fühlte sie kaum mehr. Das Glühen der Runen und Glitzern des magischen Pulvers ließen nach.

Warum sollte sie die Eiskristallkette vernichten? Das war nicht zweckdienlich. Siantari mochte tot sein, doch das ewige Eis war es nicht. Noch steckte es unberührt in einem Tal hoch oben im Kuolema-Gebirge. Doch eines Tages sollte es die ganze Welt beherrschen. Und von allein würde das nicht geschehen. Irildora musste einen Plan schmieden. Nein, Irildora musste das Felsentor aufsuchen und das ewige Eis mit ihrer Runenmagie stärken, sie sollte sofort aufbrechen. Aber nein, Irildora sollte ihre Intentionen geheim halten, die Kette verstecken und die nächste Zeit als unscheinbare Heldin auftreten. Nein, Irildora sollte kuzkan rećhiar bolinur.

Unkontrollierte Überreste der Geister uralter Eis-Dämonen schwirren in ihrem Kopf herum und schrien ihr widersprüchliche Anweisungen an, die sie nur zur Hälfte verstand. Irildora musste ihre Gedanken sortieren. Doch die Dämonen ließen sie nicht. Erstarrt blieb Irildora stehen.\bigskip







Nehamal und andere Drachenkultisten kreuzten Aćhs Pfad, als sie durchs offene Tor in die Rietburg rannte. Der Riethof war chaotisch. Zwei Häuser waren eingestürzt, dafür ragte ein gewaltiges Konstrukt aus gefrorenem Wasser neben den Palas in die Höhe. Krieger rannten umher. Blut sickerte aus toten Köpern in unregelmäßig verteilte Schneehaufen. Kein Eis-Drache war mehr zu sehen. Nur noch ein gewaltiger Haufen vor dem Palas. Und davor ... das war Iril!

Es war ein Aćh bereits bekanntes Bild: Irils Runentattoos leuchteten grünlich wie auch ihr Hammer, und grünliche Schwaden stiegen von ihr auf. Doch diesmal zitterte Iril nicht verkrampft am Boden herum, sondern stand nur stocksteif da und starrte auf ihre Hand herunter. Ihr Runenhammer hing locker an ihrer Seite. Ihre Tattoos erloschen nach und nach, doch in ihrer Hand funkelte etwas durchgehend magisch.

Eiskristalle, deren Anblick Aćh nur allzu gut kannte. Sie zwang sich zu einem letzten Effort und rannte zum Palas hoch. Im Rennen löste sie die beiden Takuri-Federn von ihrem Umhang, der im Winde davonflatterte, und knackte die Federstiele entzwei. Bei Iril angekommen, legte Aćh die brennenden Federn um Irils Faust mit der Kristallkette darin. Dann drückte sie zu, bis es knirschte. Und Iril aufschrie.

Irils Faust öffnete sich wieder. Die einstige Eiskristallkette rieselte als harmloses Pulver auf den Schnee nieder.

Iril blinzelte überrascht und blickte ins Aćhs Gesicht hoch, welches wiederum freundlich auf sie heruntergrinste.

„Eiskristallkette angefasst. Anfängerfehler“, zwinkerte Aćh, „Alles wieder klar, Runenmeisterin?“

„Alles klar, Takuri-Hüterin“, sprach Iril schwach.

Sie spürte, wie die Kraft der Runensteine sie verließ. Das vielfarbige Brummen ihrer Tattoos verlöschte. Auf einmal wirkte Irils Hammer tonnenschwer. Ihre Beine gaben nach. Schwach plumpste sie auf den Boden.

„Darf ich dich jetzt berühren?“, fragte Aćh sorgsam.

„Die Runen sind erloschen. Es besteht keine Gefahr. Und stören tut’s mich auch nicht“, sprach Iril leise. Aćh kniete sich nieder, verschnaufte und stützte die zitternde Iril. Zumindest so lange, bis ein zitterndes Turr-Küken auf sie zuzuwatscheln kam und ebenfalls Aufmerksamkeit verlangte.

Dann kam auch Ijsdur durchs Tor hereinspaziert, erschöpft und zitternd, doch mit einem nicht einmal so sehr gezwungen aussehenden Lächeln auf den Lippen.

Er zeigte auf die Eistreppe und sprach tonlos, doch rasch: „Hat es funktioniert? Hast du meine Eisbrücke gesehen? Ich kann Eisbrücken bauen. Ich kann mit meinen Eisblitzen wie aus dem Nichts Konstrukte wachsen lassen, selbst aus der Entfernung. Welch wundervolle Neuigkeit.“

Aćh winkte ihn zu sich, immer noch das kleine Turr-Küken streichelnd. „Das ist noch nicht alles. Ijsdur hätte sich beinahe geopfert, um den Eis-Drachen aufzuhalten. Ohne ihn wäre Siantari noch nicht annähernd gefallen.“

„Das ist doch keine Heldentat im Vergleich zum Erschlagen des Drachen“, protestierte Ijsdur.

„Das wäre ohne euch nicht gelungen. Ohne uns alle.“

Iril guckte ihnen tief in die Augen und bekräftigte: „Danke, ihr beide. Einfach nur Danke.“

„Kriege ich auch ein Danke?“, ertönte eine leise Stimme. Barz war müde aus der offenen Verliestür gewankt und betrachtete staunend Tarokdurs Überreste. „Hübsche Arbeit! Ist es vorbei?“\bigskip







Natürlich war es noch nicht vorbei. Kurz danach grub sich ein gewaltiger Erdgeist aus dem Boden unter der Rietburg nach oben und versuchte, einige vertriebene Maasavi-Erdgeister zu rächen. Und er wurde von den glorreichen Helden Andors zurückgetrieben – wenn auch nicht diesen vieren hier. Doch das ist eine Legende für ein andermal.

Vorbei war es erst später.

Als die geflohenen Burgbewohner sich zurück trauten.

Als die Verletzten versorgt und die Toten aufgebahrt waren.

Als die Zeit fürs Feiern gekommen war.

Die vier neusten Helden, Iril, Ijsdur, Aćh und Barz, wankten durch den Trubel. Hier und da rief man ihnen gute Wünsche zu.

Von irgendwoher rannte Nabib herbei. Er hatte eine üble Beule an der Stirn, doch kümmerte er sich nicht darum, sondern stützte Barz und redete ihm gut zu.

„Wie viel Pulver hast du eingenommen?!“, fragte Nabib besorgt. Barz lallte irgendetwas halb Verständliches.

Kurz darauf kehrten auch noch die letzten Helden in die Rietburg zurück.

Eara berichtete mit voller Stimme, Skral-Häuptling Shron habe durch einen herzhaften Sprung in die reißende Narne dem Urteil der Helden entkommen können, doch seine Horde sei zerschlagen.

Chada streichelte eine silberne Schlange, die Iril seltsam bekannt vorkam. Thorn hielt vorsichtig Abstand und beäugte die Schlange aus dem Augenwinkel, während er sich um die Pferde der Rietburg kümmerte.

Kheela trug einen Beutel mit Mera-Steinen, den sie einigen Tulgori übergab. Mit einem Augenzwinkern fügte sie hinzu, dass sie beim nächsten Mal besser auf ihr Floß aufpassen sollten.

Etwas überrascht stellte Iril fest, dass Hogo und Fenn – die eine große Menge Erde und Dreck von ihren Waffen wuschen – sich nebenbei mit dem Hexer aus Tulgor unterhielten, der sich bei der Ankunft der Tulgori in Andor so rasch von der Reisegruppe getrennt hatte. Offenbar hatte er sich nun doch bei einigen Helden Anschluss gefunden. Des Hexers Augen blieben aber nicht bei Hogo und Fenn, sondern wanderten immer wieder hinüber zu Eara, welche mit blau schimmerndem Feuer von der Spitze ihres Zauberstabs mühelos ein Lagerfeuer entzündete.

Aćh gesellte sich zu dem dreien – Haamun, Hogo und Fenn – und unterhielt sich mit ihnen. Iril erinnerte sich daran, dass Haamun an Aćhs und Barz‘ Seite unter dem Kuolema durchmarschiert war. Vermutlich hatten sie sich einiges zu erzählen.

Doch ehe Iril sich ihnen anschließen konnte, zog sich der Hexer auch schon wieder an Hogos Seite vom Trubel zurück, in den Schatten eines Häuserdachs. Fenn und Aćh blieben allein zurück. Die beiden großen Vogelliebhaber unter den neu ernannten Helden. Wenn Barz Aćh in seiner Zeit in Tulgor die Barbaren-Sprache etwas nähergebracht hatte, konnte sich Aćh vermutlich mit dem ehemaligen Barbaren Fenn besser unterhalten als mit allen anderen Anwesenden – Ijsdur, Barz und Übersetzungs-Iril einmal ausgenommen.

Kram stellte sich stolz zu Iril und gratulierte ihr von ganzem Herzen zum Erschlagen ihres ersten Drachen. Er richtete ihr die besten Glückwünsche seiner Familie aus und informierte sie darüber, dass Schmiedemeister Hildorf sich nach einer unangenehmen Interrogation bei Fürst Hallgard aus Cavern zurückgezogen hatte. Keiner wusste so genau, wo er sich aufhielt. Iril hatte die Vermutung, dass er nicht allzu weit entfernt unter Gleichgesinnten weilte und seiner Schmiede nachtrauerte.

Als letzter der zurückkehrenden Helden tauchte im Burghof ein über und über mit kleinen Stichwunden übersäter gehörnter Geselle auf. In seinem langen Haar steckte eine rosa Rietgrasblüte. Sein Gesicht grinste trotz seiner arbakschen Verletzungen breit, als er Iril erblickte. Bragor rannte mit großen Schritten auf Iril zu und hob sie hoch in die Luft, während er sie im Kreis herumwirbelte. Iril fühlte sich wie eines der Gewichte, die Bragor täglich im Riethof stemmte.

Mit tiefer Stimme rief der Tarus: „Oh, Iril, wie sehr es mich freut, dich unversehrt zu wissen! Wir machten uns solche Sorgen, als wir den Eis-Drachen aus der Ferne wüten sahen. Wenn wir nur früher hier eingetroffen wären ... und diese Bestie hast du eigenhändig niedergerungen? Noch in einem Dutzend Tagen werden wir diese Heldentaten nicht vergessen haben!“

Iril protestierte gegen das übermäßige Lob. Sie hatte Bragor und seinen Zahlenverständnisse gut genug kennen gelernt, dass ein Dutzend für ihn schon fast mit Unendlich gleichzusetzen war.

Ijsdur trat hinzu und tauschte einen komplexen Handgruß mit Bragor aus. Zwei großgewachsene Gehörnte mit einer Abneigung gegenüber sperrigen Schilden und einer Vorliebe für spärliche Kleidung. Sie könnten beide kaum unterschiedlicher sein, und doch hatten sie sich schon bei ihrer ersten Begegnung prächtig verstanden.

Die Hüterin Kheela gesellte sich zu ihnen und zog Bragor nach kurzer Zeit auch schon wieder zur Seite. Offenbar würde der königliche Barde Grenolin bald mit musikalischer Untermalung beginnen und Kheela wollte den Tarus zum Tanzpartner haben.

Vara der Wassergeist waberte am Rande der Festigkeiten herum. Ihr Gesicht war eine Maske der Trauer. Ijsdur bewegte sich zu ihr und legte ihr eine Hand auf die Schulter. Die Hand fuhr durch den Wassergeist hindurch und wurde durchscheinend, während Schneekristalle sich entlang Varas Schulter ausbreiteten. Weder Ijsdur noch Vara sagten etwas.

Einige Schritte entfernt war Barz auf einem Holzhocker zusammengesunken, während Nabib ihn in seine Arme geschlossen hatte und ihm süße Worte zumurmelte. Fenn schien nicht viel um ihre Privatsphäre zu kümmern, denn er trat laut grölend dazu. Den wenigen Wortfetzen nach, die Iril aufschnappte, versprach er den beiden „echt umhauendes Barbaren-Bier“, wenn sie ihm das Geheimnis der Anwendung von Silberblumen verrieten.

Iril blickte sich nach Aćh um und fand sie ein wenig verloren wirkend abseits der Gesellschaft, wie sie Turr und Morar Apfelnüsse fütterte. Iril winkte sie zu sich. In diesem Augenblick setzte Grenolin der Barde eine andorische Flöte an seine Lippen und begann mit dem Spielen einer wilden Melodie.

Sie feierten gemeinsam bis tief in die Nacht hinein. Selbst der sonst eher zurückhaltente Hogo ließ sich von Fenn zu einem wilden Tanzduell locken.

Die Gefahr war gebannt.

Iril war nicht mehr allein.

Alles war gut.\bigskip


Am nächsten Morgen – oder eher schon gegen Mittag – saßen Iril, Ijsdur, Aćh und Barz mit einigen anderen Andori an einem Lagerfeuer und genossen gehörig gesalzene Fladenbrote von Meisterbäcker Karmat, als zwei grau gewandte Bewahrer zu ihnen traten.

Solche waren inzwischen kein seltener Anblick mehr auf der Rietburg. So viel dazu, dass die Bewahrer vom Baum der Lieder angeblich so selten wie möglich den Wachsamen Wald verließen.

Wie es sich herausstellte, hatte der Oberste Priester Melkart seit der Ankunft der Tulgori in Andor ein weiteres Paar Chronisten ins Rietland gesandt, die sich ausführlich mit den Erlebnissen dieser fremden Reisetruppe auseinanderzusetzen hatte. Doro und ... irgendein schwierig auszusprechender Name, der Iril entfallen war. Sie hatte auch nicht viel mit diesen beiden Bewahrern Kontakt gehabt, nein, als Tulgoribeauftragte hatten jene sich viel eher für Ijsdur und Aćh interessiert.

Erleichtert erkannte Iril, dass sie auch gar nicht an den Namen zu erinnern versuchen musste. Denn die beiden sich anschleichenden Mitglieder des Bewahrerordens waren nicht die Tulgoribeauftragten, sondern die schon zu Taroks Tod ausgesandten Sanja und Jorna. Wie üblich war Sanja als erste bei den Helden angekommen, während Jorna schüchtern hinterherhuschte.

„Endlich erreiche ich Euch alle zusammen einmal“, rief Sanja triumphierend, „Stets scheint mir mindestens einer aus dem Weg zu gehen. Hättet Ihr denn gerade etwas Zeit, dem Bewahrerorden einige dringende Fragen für die Nachwelt zu beantworten?“

Hastig packte Jorna wieder ihre Schreibtafel hervor, zückte eine Feder und machte einige hastige Notizen am oberen Rand eines neuen Pergaments.

„Wir haben doch noch nicht mal etwas gesagt, was schreibt sie denn nun schon auf?“, fragte Ijsdur neugierig.

„‚Er‘“, korrigierte Sanja beiläufig, „Aktuell zieht er ‚Er‘ vor.“

Dabei zeigte sie auf einen gelben Wimpel, der auf Brusthöhe an Jornas grauem Bewahrergewand befestigt war. Jornas Wangen röteten sich, doch er lächelte.

Temporeich berichtete Sanja schon weiter: „Und er schreibt nur mal Datum und Titel für die spätere Transkription seiner Notizen in einen vollwertigen Bericht nieder. Nichts Besonderes, irgendetwas im Sinne von ‚Befragung der siegreichen neuen Helden nach dem Vertreiben des großen Eis-Drachen, vernommen von Sanja und Jorna am Xten Tag des Xten Monats des Jahre 65 nach andorischer Zeitrechnung‘. Wobei ‚neue Helden‘ noch ein Platzhalter sein könnte. Hat eure Heldengruppe inzwischen schon einen Namen gefunden?“

Die vier blickten einander an.

„Wollen wir uns denn einen Namen zulegen?“

„Wenn das von uns verlangt wird? Welch bessere Gelegenheit als jetzt?“

„Was vereint uns denn?“

„Nicht viel. Wir stammen ja aus allen möglichen Ecken der Welt. Ferne Helden? Fremde Helden?“

„Ich komme technisch gesehen ursprünglich schon aus Andor“, warf Iril ein, „Oder zumindest aus Cavern.“

„Was ist mit Reisenden Helden?“

„Haben wir nicht vor, uns zumindest eine Zeit lang hier niederzulassen?“

„Das ist wohl wahr.“

„Neuere Helden? Neuste Helden?“

„So ein Titel würde rasch alt.“

„Wir haben schon Turr und Sabri; besorgen wir Ijsdur und Iril noch Tiere und nennen uns Haustier-Helden?“

„Tiere, die mit Schnee und Eis wenig anfangen können, mögen mich üblicherweise nicht.“

„Da hast du vielleicht einfach noch nicht die richtigen Tiere getroffen.“

„Was ist mit Magie? Wir alle haben etwas mit Magie zu tun, oder etwas nicht?“

„Nun, dieser Runenhammer ist definitiv von Magie erfüllt.“

„Meine Pulvermischungen sind auch magisch.“

„Takuri sind ziemlich magische Wesen. Und die Steinflöte kann mithilfe von Magie auch in weiter Entfernung vernommen werden.“

„Und alles an mir, von meiner Eiskristallkette über meine Eisblitze bis hin zu meinem schneeigen Körper, ist magisch.“

„Na, dann ist es klar. Wir sind die Magischen Helden!“

Jorna kritzelte etwas auf seiner Schreibtafel. Dazu, das Interview durchzuführen, kamen die beiden Bewahrer allerdings nicht, denn in diesem Augenblick rannte Prinz Thorald auf die frischgetauften magischen Helden zu. Der Prinz war von den gestrigen Feiern absent geblieben, sondern hatte seinen frisch vor einem Eis-Drachen geretteten Hintern rasch in ein sicheres Bett bugsiert.

„Was ist denn das für ein Klamauk! Und was macht dieser elende Eis-Dämon immer noch hier?! Wir sahen doch soeben alle, dass wir denen nicht trauen können. Oh, wie töricht ich war, dich zu einem Helden zu ernennen. Ein Glück, dass wir solche Titel im Zweifel auch wieder entziehen können. Raus mit der Sprache, was hattest du mit diesem riesigen Eis-Drachen zu tun?!“

„Wir können ihm trauen!“, sprach Aćh. Sie stellte sich zwischen Ijsdur und den zornigen Prinzen. „Ijsdur, der Eis-Dämon, steht auf der Seite der Helden. Mit seinen magiegeladenen Eisblitzen hilft er nicht nur im Kampf. Seine Eismagie macht ihn zu einem der stärksten Helden Andors. Mit dem Erschaffen von Eisbrücken erzeugt er außerdem neue Wege und Abkürzungen, die allen Helden zugutekommen.“ Dann warf sie einen Blick hinter sich. Die gloriose Eisbrücke in den Himmel, dank der Iril den Eis-Drachen hatte erreichen können, war inzwischen kaum mehr als solche zu erkennen, sondern kaum mehr als ein eisiger Zapfen, der aus einer immer größeren Wasserlache hervorstach.

„Leider ist all dies nicht von Dauer, denn nach einer Weile schmelzen seine glitzernden Werke dahin“, hängte Aćh melancholisch an. Ijsdur sollte dieses bestimmte Werk lieber manuell schmelzen, ehe es etwa noch auf dem Palas stürzte.

Thorald suchte nach Worten, fand keine, verwarf seine Hände und stopfte weiterhin wütend murmelnd davon.

„Ich würde das so interpretieren, dass du noch Teil des Teams bist“, meinte Barz grinsend zu Ijsdur.

Aćh fügte an: „Und wenn nicht, werden wir für dich einstehen. So viel schulden wir dir, und noch so einiges mehr.“

Ijsdur blieb stumm.

Iril betrachtete den prinzlichen Abgang nachdenklich. Dann langte sie in ihre Reisetasche und zog zwei Ketten hervor, die nach dem Gefecht gefunden worden waren.

Zwei Ketten, an denen einige unförmige Knochenfragmente befestigt worden waren.

Taroks und Sagraks letzte Überreste.

„Es dauert vermutlich nicht lange, bis Thorald sich an die Knochen erinnert und herumfragt, ob sie schon gefunden wurden. Ich bin mir nicht sicher, ob wir sie ihm wieder überlassen sollten. Diese Angelegenheit ist noch nicht vorbei. Zweimal haben die Drachenkultisten die Knochen nun schon gestohlen. Sie werden es wieder versuchen. Wie können wir verhindern, dass es ein drittes Mal geschieht? Wollen wir sie vielleicht doch vernichten?“

„Ich hätte da eine bessere Option“, meinte Barz. Er nestelte an seinem Pulvergürtel herum und präsentierte einen rosa Beutel. „Ich liebe es, mit neuen Pulvern zu experimentieren. Was dabei herauskommt, weiß keiner so genau. Mit nur einer Prise dieser bestimmten Mischung hier schaffte ich es schon, eine Pfeilspitze, eine Blume und eine Schneckenmaus zu versteinern. Wir könnten vermutlich die Drachenknochen damit in Gestein verwandeln. Sicherlich würde dies sie als magische Quelle untauglich machen, woraufhin wir sie gefahrlos den Drachenkultisten überlassen könnten.“

„Bist du sicher?“, fragte Aćh argwöhnisch, „Das letzte Mal, als du eine solche experimentelle Pulvermischung ausprobiert hast ...“

„Schlimmer als die Knochen zu vernichten kann es ohnehin nicht sein“, meinte Iril bestimmt, „Und wir überlassen sie den Kultisten nur, wenn wir danach keine Magie mehr in ihnen spüren. Ich kann das mit meiner Runenscheibe überprüfen.“

„Es kann immer schlimmer kommen. Prinzipiell“, gab Ijsdur zu bedenken, „Aber wahrscheinlich ist es wohl nicht. Und Thorald können wir immer noch erzählen, dass die Knochen vernichtet worden seien.“

„Ich halte es für gut, ihn nicht anzulügen“, meinte Barz, „Das scheint mir ein würdiges Prinzip zu sein.“

„Ihr mit euren Prinzipien ... dann sagen wir ihm halt, dass wir die Knochen zumindest unschädlich machen konnten. Das sollte auch funktionieren.“

Draufhin ergänzte Aćh: „Davon abgesehen nützen den Kultisten die Knochen allein auch wenig. Tarokdur konnte nur dank einer von Siantaris Eiskristallketten erweckt werden. Und davon sind keine mehr übrig. Oder?“

Ijsdur zuckte mit den Schultern.

„Dein Wort ins Ohr des großen Flederfuchses“, murmelte Barz, „Wir können hoffen.“

Da überlegte Iril: „Aber was, wenn eines Tages ein gewaltiger Fluch über das Land hereinbricht und wir die Drachenknochen bräuchten, um ein magisches Ritual ...“

Barz konterte: „Was, wenn ein zwielichtiger Berater Thorald einflüsterte, dass er mit den Knochen in seinem Thronsaal etwas Böses anstellen sollte? Wer kann schon die Zukunft kennen. Vielleicht ist es besser, einen Schlussstrich unter diese Sache zu ziehen.“

Iril seufzte und stimmte zu.\bigskip







Als Iril und Barz zurückkehrten, stand das Lager der Drachenkultisten am Hang des Kuolema-Gebirges noch. Gerade noch.

Die meisten Zelte waren im Begriff, abgebaut zu werden. Menschen, Kreaturen und ein Ziegenwesen packten ihr Lagermaterial auf große Karren. Das Ziegenwesen murrte über ein Ziehen in seiner verletzten Schulter. Niemand beachtete es.

Nehamal stand vor einem brodelnden Kessel und stocherte mit einem Ast im Feuer herum.

Neben ihm saß die kleine Reanna am Boden und stapelte flache Steine aufeinander.

Nehamal machte sich nicht einmal die Mühe, wütend aufzusehen, als Iril und Barz nähertraten. Knurrend sprach er: „Ihr! Habt ihr nicht schon genug angerichtet?“

Nicht einmal seinen Degen zog er. Seine blutunterlaufenen Augen starrten ins Leere. Die letzten Tage hatte ihm nicht gut zugetan.

Iril sprach: „Es tut mir leid. Es hätte nie so weit kommen können. Ich wollte das alles nicht. Ich wollte nur nicht, dass ihr mit den Knochensplittern ... ihr seht, was geschehen ist.“

„Nur, weil ihr Sagramak angereizt habt. Sie hätte niemandem geschadet, wenn ihr uns einfach zu Beginn an Tarok rangelassen hättet!“, schnüffelte Nehamal.

Iril murmelte: „Es tut mir leid. Thorald und seine Krieger sollten nicht über euch bestimmen. Ich auch nicht. Doch die Gefahr, die von diesen Knochen ausgeht ... ich würde weiterhin behaupten, richtig gehandelt zu haben.“

Nehamal schnaubte.

„Wir hätten die Knochen ungefährlich machen sollen, statt sie zu konfiszieren“, meinte Barz, „Lassen wir die Vergangenheit vorerst ruhen. Nehamal, wir ersuchen, den Konflikt zwischen den Andori und den Drachenkultisten hinter uns zu lassen. Wir bringen euch den Leichnam Sagramaks zurück.“

Barz deutete zu Sabri, welche über den Hügel anzutrotten kam. Über ihren Rücken war ein blütenweißes Totentuch gelegt. Eine Ausbuchtung deutete an, dass sich darunter ein menschengroßes Etwas befand. Und die verbeulte Rüstung an Barz‘ Rucksack verriet, um wen es sich dabei handelte.

Nehamal unterdrückte einen Schluchzer.

Barz hängte an: „Und wir bringen euch die Knochen von Sagrak. Beerdigt Sagramak nach euren Riten, wie die Drachen es von euch verlangen. Und ernennt eine neue Sagramak, die dem Andenken der Schamanin würdig ist.“

„Unter einer Bedingung“, sprach Iril. „Wir haben die Ermordeten unter euren Angriffen nicht vergessen. Wir wollen verhindern, dass so etwas wieder geschieht. Dass ihr und die deinen keine unschuldigen Seelen mehr schaden.“

Nehamal wedelte mit seiner Hand und scheuchte die kleine Reanna zu einer in der Nähe herumstehenden älteren Kultistin. Dann erst blickte er auf und zischte Iril entgegen.

„Ich habe kein Leben genommen. Nicht dieses Mal. Und Eure Krieger nennt ihn unschuldig?!“

„Sagramaks Leichnam steht euch zu, außer Frage. Doch damit wir euch die Drachenknochen Sagraks geben, müsst ihr schwören, dass ihr diese Sache hinter euch lässt. Schwört auf die Drachen der Urzeit, diese Wut auf die Helden und das Königshaus sein zu lassen. Diesen Konflikt zu beenden. Lasst ihr aller Leben in Ruhe und zieht euch zurück ins Gebirge, aus dem ihr stammt.“

„Ich soll schwören? Auf die Drachen, in die ihr nicht einmal glaubt?“

„Das muss ich auch nicht, damit dieser Schwur Bedeutung hat.“

Nehamal betrachtete die ihm entgegengestreckte Knochenkette, schluckte und sprach dann heißer: „Dann schwöre ich. Im Namen der hohen Drachen, die uns alle nach unserem Tod richten werden.“

„Was schwörst du?“

„Ich schwöre, den Konflikt um die Drachenknochen sein zu lassen. Den Andori nicht nachtragend zu sein. Zurück in unser Gebirge zu ziehen.“

Barz blickte Iril hoffnungsvoll an. Iril nickte: „Gut genug für mich.“

Nehamal nahm die Knochenfragmentkette reflexiv an sich. Dann blickte er die näher schlurfende Sabri mit einem unergründlichen Ausdruck an. Sein Kinn zitterte. Leise murmelte er: „Sie ... sie war so rasch weg. Ich dachte, dass wir noch Jahrzehnte miteinander hätten. Es ist ... es ist nicht fair.“

„Der Tod ist nicht fair“, murmelte Barz nickend, „Wie es auch das Leben nicht ist. Wir geben unser Bestes, diese Unfairness zu richten ...“

Nehamal unterbrach Barz: „Nicht der Tod ist unfair. Nicht nur. Seine Personifikation. Der geflügelte Tod. Tarok.“ Nehamal spuckte aus. „Tarok war schuld! Er und der vermaledeite Herold, die keinen Dreck um den Willen der restlichen Drachen geben!“

Iril murmelte: „Das ist jetzt unangenehm. Wir wollten euch auch die versteinerten Knochen von Tarok überlassen. Ihr magisches Potential ist versiegt. Sie von euch fernzuhalten, wäre inzwischen bloß ein Akt des Trotzes..“

„Taroks Knochen? Versteinert?“, horchte Nehamal auf.

Barz zeigte die steinerne Kette und meinte: „Daraus wird nie wieder ein Schnee-Drache erwachen können. Aber für eure Rituale sollte es hoffentlich genügen.“

„Und dem hat Prinz Thorald zugesagt?“

Iril grinste schief. „Sagen wir einfach mal, für ihn sind die Knochen verschollen gegangen.“

Barz legte nach: „Nach dem, was Nabib und Fenn mir von Thoralds Wesen erzählt haben, werden seine Gedanken schon bald wieder woanders sein.“

Nehamal blickte ihnen argwöhnisch entgegen. Dann schnappte er sich die Knochenfragment-Kette und murmelte: „Danke. Taroks Übeltaten hin oder her, wir können die Überreste seiner Seele durchaus nutzen, um uns zu stärken. Wir werden einen Tamarok ernennen. So hätte es Sagramak gewollt.“

„Können wir irgendwie sonst helfen?“

„Ja! Fischt uns Taroks Überreste aus dem Hadrischen Meer!“

„Ich dachte eher etwa an Nahrung für Eure Armen und Schwachen. Diese können schließlich nichts ...“

„Wir können uns noch gut um uns selbst kümmern, ganz herzlichen Dank“, schnaubte Nehamal. „Ich bin schon froh, wenn wir Andor wieder hinter uns gelassen haben. Danke für Sagramaks Körper und für die Knochen, aber auch nicht für mehr. Heilige Hüter im Himmel, was für ein Desaster diese ganze Aktion war. Wenn ich diesen Herold in meine Finger kriege ...“

Iril und Barz verabschiedeten sich wieder und überließen die Kultisten sich selbst. Im Abgehen sahen sie, wie Nehamal in weiter Ferne mit sich selbst zu reden schien.

Dann gab er sich einen Ruck und half den restlichen Kultisten beim Abbau des Lagers. Er packte einen ganzen Karren und atmete tief durch. Seine Augen glühten rot auf. Dann zog Nehamal den Karren in Richtung Osten. Ochsenstärke. Drachenstärke.

Die Drachenkultisten zogen sich wieder in ihr Gebirge im Osten zurück.

Das war wohl für alle das Beste.\bigskip







Aćh ließ den kleinen Turr fliegen und verfolgte seinen feurigen Flug über den andorischen Himmel. Sie mochte es hier in Andor. Nicht, dass sie etwas dagegen hätte, demnächst einmal Barz‘ Heimatland oder die Inseln des Nordens zu besuchen. Doch das andorische Heldendasein war eine wirkliche Freude.

Nur die Sprache war ein Ärger. Warum mussten Wörter auch so viele unterschiedliche Endungen haben, je nachdem, wo sie in einem Satz standen? Warum kam das Verb oft erst mitten in den Satz statt an den Anfang? Warum musste jeder Begriff einem von drei Genera zugeordnet werden, die man sich einfach merken musste? Muttersprachlern mochte dies beim Verständnis von Sätzen helfen, doch weh jedem, der diese Sprache erlernen wollte, um von Übersetzungstränken unabhängig zu werden. Damit hatte Barz damals nicht zu kämpfen gehabt. In Tulgor hatte sich die Sprache schon seit langem aus dem Ur-Tulgorischen mit seinen gefühlt einhunderttausend Geschlechtern, Fällen und Tempi über einige verwirrende Sprachkonzepte hinaus entwickelt.

Aćh verbrachte viele Abende in der Taverne von Andor und lernte, sich langsam an die Sprache zu gewöhnen.

Der gute Geist der Taverne zum Trunkenen Troll hatte sie besonders gerne nach Informationen zu Tulgor gefragt. Und diese Informationen konnte sie inzwischen immer besser kommunizieren.

In der Ferne sah Aćh zwei wohlbekannte Gestalten näherkommen. Eine gedrungene Hammerträgerin neben einem hochgewachsenen Mantelträger mit Bogen. Offenbar hatten sie die Drachenknochen gut abgegeben. Sabri war nirgends zu sehen.

Ja, Iril und Barz waren dabei, das Lager der Tulgori zu besuchen. Unter den gespannten Zelten herrschte emsiges Treiben.

Barz erkannte Aćh und eilte zu ihr. Sie hatte in den letzten Tagen von den unreißbaren goldenen Bogensehnen aus Tulgor geschwärmt, aus einem Material, welches angeblich ganze Schiffe auf Seile hängen können sollten. Nun wollte Aćh am Lager der Tulgori eine solche Sehne für seinen zerschnittenen Bogen finden. Ein neues Mondschwert für sich selbst hatte sie bereits erstanden und in einer schönen Zeremonie mit Goldfarbe überzogen.

Iril blickte den beiden hinterher und ließ das Stimmengewirr auf sich einwirken. Sie mochte die Stimmung hier. Stets war etwas los.

Eine Tulgori mit Augengläsern auf der Nase polierte einige Mera-Steine. Iril blinzelte ihr neugierig entgegen, störte sie allerdings nicht. Wie sehr waren die vom roten Mondlicht gestärkten Mera-Steine wohl mit den Runensteinen der Schildzwerge verwandt?

Einige andere Tulgori beugten sich über detaillierte Karten des Landes, die ihre Truppe in den letzten Tagen hergestellt hatten. Bald wollten die Tulgori weiterziehen. Die Geheimnisse der Zwergenmine Cavern interessierten sie. Bisher hatten sie nur die imposanten Portale der Mine gesehen, und die Aussicht, ins Innere des Berges und in das Reich der Schildzwerge zu gelangen, beflügelte sie.

Überrascht stellte Iril fest, dass sich inzwischen auch Wrort, der reisende Temm, im Lager der Tulgori niedergelassen hatte und sich auf Tulgorisch mit einigen Reisenden unterhielt. Gerade diskutierte er mit Ijsdurs Bruder Eforas darüber, ob irgendeine gewisse uralte dunkle Prophezeiung auf den Königsturm im Grauen Gebirge zutreffen könnte.

Eforas‘ Anwesenheit erinnerte Iril an seinen Bruder. Ijsdur war nicht hier. Vermutlich zog er gerade ziellos durchs Land, eine Spur aus Eis und Schnee hinter sich herziehend. Zwischen Ijsdur und Eforas stand es nicht besser. Ijsdur hatte ihr versichert, dass ihm dies nicht viel ausmachte, und dass ihm dieser Fakt beinahe mehr schmerzte als sein ihm gegenüber kalter Bruder.

Iril hatte ihn dennoch zu trösten versucht.

In ein bisschen Abstand vom Lager der Tulgori setzte Iril sich auf einen Stein, zog ihre Runenscheibe hervor und begann, an einer bestimmten Runenfolge zu werkeln. Es konnte doch nicht so schwer sein, Edelsteine als ähnliche Speicher magischer Kraft wie Runensteine zu nutzen. Eines Tages würde sie dieses Geheimnis knacken, da war sie sich sicher.

Die Zeit rannte vorbei, während Iril unter brütender Sonne arbeitete.

Barz gesellte sich zu Iril und brachte ihr einen Trinkschlauch voller erfrischenden Wassers, auf dass Iril vor lauter Runen nicht das Trinken vergaß. Dann setzte er sich neben sie in den Schatten eines Baumes. Er zückte zwei Phiolen mit glitzerndem Staub darin und begann, sorgfältig Portionen davon zu mischen und auf einem kleinen Holzteller zu platzieren. Es knisterte leise. Zwischendurch sprangen Funken über. Barz fluchte auf und Iril kicherte leise.

Versunken in ihre jeweiligen Interessen, arbeiteten Steppennomade und Runenmeisterin Seite an Seite. In Ruhe und Frieden. Die ultimative Lebensqualität.

Es wurde nur noch besser, als Iril Barz danach fragte, was er eigentlich hier tat, und Barz begeistert von magischen Reagenten mit Ochsenstaub als Katalyst zu erzählen begann, und davon, wie man unter Umständen dank Ijsdurs übernatürlicher Kälte diese bislang zu explosive Reaktion zweier Pulver vielleicht kontrollierter auslösen könne, wenn man zuvor gewisse Vorsichtsmaßnahmen vornahm.

Iril war keine Pulvermeisterin, doch hatte sie an der Akademie von Werftheim genug Intuition aufgeschnappt, um erst nach einigen Minuten von Barz‘ Erläuterungen abgehängt zu werden. Leider hatte ihr Pulvermeister sie damals kaum für diese Lehre begeistern können. So, wie Barz von seiner Schamanin Asbark in Thakkum berichtete, war deren Feuer für diese Kunst kaum einzudämmen.

Schneeflocken landeten auf Barz‘ Holzteller und ein kalter Wind strich um Irils Haar. Ijsdur war zurückgekehrt. Der Eis-Dämon ließ sich neben seinen beiden Mithelden ins Gras sinken und blickte in den Himmel. Sah den Wolken beim Vorbeiziehen zu. Streckte seine Hand in die Höhe und beobachtete, wie die Schneewirbel um ihn herum seinem Willen folgten.

Da schoss ein armlanger Turr vom Himmel herab und ließ sich neben Ijsdur nieder, ja, erlaubte dem Eis-Dämon sogar, seinen Schnabel zu streicheln. Und setzte dabei nur beinahe das trockene Rietgras in Brand.

Zu guter Letzt ließ auch Aćh nicht mehr lange auf sich warten und setzte sich zur Gruppe. Sie hatte sich einen neuen eleganten Umhang gekauft und präsentierte diesen stolz den restlichen Magischen Helden.

„Was nun? Bleiben wir fürs erste hier im Rietland und verteidigen die Hilfsbedürftigen? Ziehen wir woanders hin?“

Barz blickte gedankenversunken in den Osten, wo Familie auf ihn wartete. Dann zur Rietburg, wo Nabib ruhte. Dann zum Kreis der Magischen Helden, in dem er sich nun befand.

„Ich will demnächst noch das Grab meiner Eltern aufsuchen“, meinte Iril.

„Da könntest du dich den Tulgori anschließen“, schlug Aćh vor, „Nach dem, was ich gehört habe, wollen sie demnächst nach Cavern aufbrechen.“

„Vielleicht. Vielleicht bleibe ich auch noch einige Tage hier. Die Gesellschaft ist einfach zu gut.“

„Du willst deine neue Familie nicht schon wieder verlassen?“, lächelte Barz.

„Eine neue Familie? Sind wir das denn?“, fragte Ijsdur.

„Noch nicht“, lachte Aćh. „Aber wir könnten es werden.“

„Das klingt schön“, murmelte Iril.

In der Ferne kündigte ein tiefes Röhren die Ankunft von Sabri an, welche Streicheleinheiten verlangte. Lachend winkte Barz seiner Steppenechse entgegen. „Na komm schon, Sabri! Die schnelle Echse fängt den Fisch!“

Die Sonne hatte sich zwischen den Wolken hervorgeschoben. Hell und beinahe magisch erleuchtete sie das Königreich Andor. Ihr goldener Schein legte sich wie eine wärmende Decke über unsere ruhenden Helden.

Es war ganz so, als wolle sie sagen:

„Ruht wohl und erholt euch. Ihr habt noch eine Menge Abenteuer vor euch.“








\newpage
\section{Epiloge}


„... und so ernannte Prinz Thorald in einer seiner ersten Amtshandlungen als Regent – obwohl er das Königsamt offiziell selbst jetzt noch nicht angenommen hat – ganze zehn neue Helden von Andor. Manche waren etwas gesprächsfreudiger als andere, aber wir sollten zumindest von allen ihre Namen und Professionen aufgelistet haben.“

Jorna händigte die Liste der Heldennamen und Professionen an den Obersten Priester Melkart aus, welcher seine Augen zusammenkniff und die verschiedenen Namen musterte. Dann hob er überrascht eine Augenbraue.

Sanja setzte an, mit ihrer Rekapitulation der Ereignisse der letzten Wochen fortzufahren, doch Melkart hob seine Hand und gebot ihr, innezuhalten. Er fuhr sich durch die perfekt gekämmten Haare und murmelte leise vor sich hin.

Ohne eine Erklärung abzugeben, stand er auf, verließ den kleinen Archivraum und machte sich auf in eine andere Abteilung des Baums der Lieder. Sanja und Jorna blickten einander fragend an und folgten dem Obersten Priester.

Die Wendeltreppe mit den viel zu großen Stufen hoch, durch den Eingang auf die Balustrade hinaus, um die halbe Balustrade herum, die Raumflucht betreten, erste Tür rechts. Melkart suchte ein kleines Pergament hervor, auf dem eine Skizze einer Steintafel abgebildet war. Die Steintafel trug mehrere Zeilen geschwungener Runen. Alte Zwergenrunen, die selbst ein Oberster Priester wie Melkart kaum lesen konnte. Glücklicherweise war unterhalb der Skizze eine Übersetzung angebracht, vermutlich von einem waschechten Schildzwerg.

„Die Tafel berichtet über die mutigen Taten von vier neuen Helden: Fenn, Kheela, Arbon und Bragor“, las Melkart leise vor. „Ich dachte schon, dass mir einige dieser Namen bekannt vorkamen, als damals Hallgards Nachricht mit der Übersetzung eintraf.“

Er verglich die Tafel mit Jornas Notizen.

„Jetzt wissen wir es mit Sicherheit: Sowohl Fenn als auch Bragor schlossen sich den Helden von Andor an, nachdem wir sie ins Rietland schickten. Und nun verlieh Thorald ihnen offiziell den Titel. Nicht länger sollen sie in der Geschichtsschreibung Andors untergehen. Wir werden ihre Abenteuer in unsere Archive aufnehmen. Nur ... Hallgards Tafel erzählt auch von einem gewissen Arbon. Ich dachte, dass unser Arbon diese Lande längst verlassen hätte. Der unschöne Vorfall in den schwarzen Archiven ist schon fast ein Halbdutzend Jahr her. Fenn und Bragor sollten ihn doch eigentlich zu uns bringen ... kann es sein, dass sie ihn aufgespürt, aber unseren Auftrag abgelehnt haben? Die versprochene Belohnung war doch gewaltig! Und Prinz Thoralds Heldenernennungsliste enthält keinen ...“

„Dazu wollten wir uns auch melden“, berichtete Sanja. „Einer dieser neu ernannten Helden erscheint mir verdächtig. Dieser ‚Hogo, Knecht aus dem Rietland‘. Er kleidete sich beim öffentlichen Anlass in einen schlichten Umhang, aber wir haben ihn kämpfen sehen. Er führt eine einzigartige Arcuballiste verdeckt mit sich. Und er kämpft in einem dunklen Gewand, das der Kleidung einer schwarzen Wache ungemein ähnelt. Etwas ist faul an ihm.“

„Arbon“, zischte Melkart, „Wir müssen den Rat der Bewahrer informieren. Schwarze Wachen aussenden. Ihn zurückbringen zum Baum der Lieder. Ihn verurteilen lassen für das Lesen unserer verbotensten Geheimnisse.“ Dann unterbrach Melkart sich und sank in einen Sessel zurück. „Oder aber wir lassen ihn einfach in Ruhe?“

Jorna schluckte schwer und sprach dann: „Mit Verlaub: Bislang kam das ja nie wirklich gut, Arbon einsperren lassen zu wollen. Wenn er die Geheimnisse des Schwarzen Archivs in den letzten sechs Jahren nicht für finstere Zwecke einsetzte, wird er das wohl auch in den nächsten nicht tun.“

Melkart nickte leise: „Ja, ich habe in den letzten Wochen genug Leid und Tod im Rietland gesehen. Wenn Arbon sich als Held von Andor dafür einsetzt, dieses zu vermeiden ... und dennoch sollten seine Vergehen gegen unseren Kodex nicht ungesühnt blieben.“ Er kratzte sich an der Stirn. „Der Rat der Bewahrer wird darüber tagen. Ich danke euch für die Information. Bitte, fahrt fort mit eurem Bericht um Taroks Tod.“\bigskip







Im Norden, einige Zeit später.\bigskip



Ein riesiger Drachenschädel ruhte auf dem Grund des Hadrischen Meeres. Kleine rote Fischchen schwammen zwischen den riesigen Zähnen umher.

Da bebte der Boden. Die kleinen Fischchen suchten hastig das Weite. Ein mächtiger Tentakel platschte auf den Meeresgrund. Staub wirbelte auf. Der Tentakel wand sich um Taroks Schädel, brach brüchige Knochen, quetschte verfaulte Muskeln und begann zu saugen.

Mit einem hässlichen Plopp löste sich eines der Augen des Drachen. Eine kopfgroße, rötlich schimmernde ovale Kugel ohne erkennbare Pupillen. Nervenstränge, die dahinter hervorragten, zeugten jedoch, dass dieses Auge ähnlich wie das eines Menschen funktionierte.

Der Tentakel hielt das rote Augen fest und transportierte es zu einem gewaltigen Mund voller spitzer Zähne, die daran herumzuschmatzen begannen.

Neue Kraft pulsierte durch den Körper des Kraken. Drachenmagie, wie er sie noch nie verspürt hatte. Ein wohliges Grunzen drang aus seinem glubschigen Innern.

Mehrere weitere Tentakel näherten sich dem Drachenschädel, ploppten sich daran fest und verzehrten noch mehr der seltenen Speise.

Und mit jedem Bissen wuchs Oktohans Macht ein Stückchen mehr.\bigskip



\az{Jahr 66}



Im Osten, einige Zeit später.\bigskip



Es war der 1. Tag des 4. Mondes.

Meister Lifornus umklammerte seinen elegant geschwungenen Zauberstab und blickte ohne zu Blinzeln in den Nachthimmel.

Er schlang seinen Mantel enger, um sich vor der Kälte zu schützen.

Dann war es so weit.

Das Licht zweier hintereinanderstehender Planeten, deren schwacher Schein nur durch ein besonderes Fernrohr – ein Geschenk des Zeitzauberers Kirr – erkennbar war, verschwand.

Der rote Mond hatte sich soeben vor sie geschoben.

Mond, Kurip und Xoriol überdeckten einander am Nachthimmel, standen in einer Linie. Auf dieser Linie lag auch die Stätte der heiligen Flammen im Land der drei Brüder. Und in der Mitte der Stätte stand Meister Lifornus, breitbeinig.

Lifornus hob seinen Zauberstab in die Höhe. Ohne sein eigenes Zutun entzündete sich das magisch gestärkte Holz. Rote, Grüne und violette Flammen tanzten über den Stab und kitzelten Lifornus‘ Hände, ohne sie zu verletzen. Dann sprangen die Funken auf die Stätte der heiligen Flammen über und ließen den Steinkreis in allen Regenbogenfarben aufleuchten.

„Beim Horror der Unterwelt! Hombudts Gefasel hatte tatsächlich eine Bedeutung! Hombudts Worte waren nicht wirr!“, hauchte Lifornus.

Eine uralte Zauberformel in der althadrischen Sprache vor sich murmelnd, ließ er den Fuß seines Zauberstabs auf den Ritualkreis donnern. Leuchtendes Glimmen lief über die Runen, löste sich von ihnen schoss in die Höhe. Der Schein formte sich zu einer rotierenden, durchscheinenden Scheibe. Zischend öffnete sich das magische Portal. Rauch, Dampf und eine gehörige Menge Lava schossen daraus hervor. Ein leises hämisches Kichern erklang und verschwand ebenso schnell wieder.

Mühelos wischte Lifornus die gefährlichen Substanzen mit einem Schwung seines Zauberstabs zur Seite und trat näher ans Portal.

Vorsichtig spienzelte er hinein. Dunkelheit waberte ihm entgegen. Er spürte die Nähe unglaublich starker Ströme reiner Magie. Was hatte Hombudt damals entdeckt? Worin bestand das Geheimnis hinter dieser Konstellation?

Schon verlangsamte sich die Rotation des magischen Portals. Seine Ränder schrumpften. Lifornus letzte Chance, in diesem Jahr etwas über dieses Phänomen herauszufinden! Alle Vorsicht fahren lassend, sprach Lifornus einen kurzen Schutzzauber über seine Hand und griff blind in die Dunkelheit hinein. Es war schleimig und glitschig, aber auch ... da! Da war etwas Festes, Greifbares! Lifornus ergriff es.

Jemand oder etwas tippte Lifornus auf die Schulter. Er wirbelte herum, doch da war niemand. Ein leises hämisches Kichern erklang und verschwand ebenso schnell wieder.

Gerade noch rechtzeitig zog Lifornus seine Faust zurück, ehe das Portal sich darum geschlossen hätte. Mit einem letzten Plopp verschwand das wirbelnde Portal.

Lifornus öffnete seine Faust und erblickte eine kleine hölzerne Schatulle, finster wie die Nacht und kaum eine halbe Handbreit lang. Wohin hatte das Portal gereicht? Woher hatte Lifornus diese finstere Schatulle gezogen? In den Deckel eingelassen, blinzelte ihm ein glühendes Auge entgegen. War dies ein geschickter Mechanismus? Das Auge wirkte erschreckend lebendig.

Mit zitternden Fingern machte sich der Meister des Feuers daran, die Kiste zu öffnen.

Doch noch ehe er den Deckel berühren konnte, sprang dieser von selbst auf. Eine stinkende Flüssigkeit spritzte daraus hervor, direkt in Lifornus‘ Augen. Es brannte. Lifornus erschrak, quietschte auf und ließ die Schatulle fallen. Noch im Fall löste sie sich in Rauch auf.

Ein leises hämisches Kichern erklang und verschwand ebenso schnell wieder.

Die vielfarbigen Flammen erloschen.

Verwirrt und allein stand Meister Lifornus in der dunklen Nacht und rieb sich blinzelnd die Augen.\bigskip





\az{Jahr 67}

Im Süden, einige Zeit später.\bigskip



„Griun! Warte auf mich!“

Iolith raste den Hügel hinauf. Oben verschnaufte sie und schnappte nach Atem. Der Ausblick auf das winterliche Graue Gebirge war ohnehin atemberaubend und der rasche Aufstieg hatte nicht weitergeholfen.

Nun, wo hatte sich Griun wieder versteckt?

Ein Rascheln, ein Schnauben, und schon wurde Iolith von hinten angefallen. Sie stürzte auf weiches Gras, rollte einige Schritte weit und blieb in einem Blumenfeld liegen. Es roch gut. Sie erhaschte einen Blick auf graue Wolken weit über ihr, ehe sich ein grüner Kopf in ihr Gesichtsfeld schob. Lange, verfilzte Haare fielen auf sie und kitzelten sie in der knubbeligen Nase.

„Beruhige dich, Griun“, protestierte Iolith, „Meine edlen Kleider. Es wird Stunden dauern, die ganzen Grasflecken ...“

Griun beruhigte sich nicht, sondern übersäte Ioliths Gesicht mit Küssen. Dazwischen erzählte sie ausführlich, wie viel besser Ioliths edle Kleider mit Grasflecken darauf aussehen würden.

Der wilde Blick in Griuns Augen hatte sich noch nicht gelegt, als sie Iolith auf die Beine zog. Iolith befürchtete – oder hoffte – jeden Moment wieder wie von einem wilden Warbock angesprungen zu werden.

Doch Griun führte ihre Geliebte ohne weitere Abschweifungen zum Ziel des heutigen Ausflugs. Griun hatte Iolith von Nehals Stein erzählt, einst ein beliebter Schlafplatz des legendären Nehal, des vielleicht beliebtesten aller Drachen. Eine riesige Steinstaue ragte auf der Seite des Bergs hervor. Griun, welche keine lobenden Worte an die zahlreichen Bauwerke („Verschandlungen der Natur“) der Schildzwerge auf und unter den Bergen verlor, sah in der Statue gar keine Statue. Manch ein Agren munkelte, bei der Statue an Nehals Stein handelte es sich um den versteinerten Nehal selbst, weswegen Vertreter des Agrenvolkes periodisch diesen Ort aufsuchten, um die Statue zu polieren.

Iolith hingegen war sich sicher, dass es sich bei Nehals Stein um ein künstliches Mahnmal der Schildzwerge an die Macht der Drachen und die Fragilität von Bündnissen hielt. Nichtsdestotrotz hatte sie ihn noch nie persönlich gesehen und freute sich darauf.

Schon aus der Ferne merkte sie, dass etwas nicht so war, wie es sein sollte. Melodisches Stimmengewirr erklang. Griun wies Iolith an, leise zu sein. Sorgsam schlichen die beiden den Berg hoch.

Auf dem Gipfel hatte sich eine Ansammlung in Kutten gekleideter Personen in einem Kreis. versammelt. Daher kam also der wirre Gesang. Doch nicht nur Leute in Kutten waren anwesend.

In der Mitte des Kreises knieten zwei Menschen, nackt bis auf Ketten mit ... Knochenfragmenten? – um ihre Hälse. Hinter beiden stand ein gehörntes Ziegenwesen, ebenfalls kleidungslos, auf dessen faszinierende Anatomie wir hier nicht näher eingehen wollen. Es präsentierte zwei echsenartige Statuetten in seinen erhobenen Händen und brummelte leise etwas vor sich hin.

Ein langhaariger Mann hob soeben ein rötliches Relikt in die Höhe und sprach salbungsvoll: „Nun kommen wir zur Hauptattraktion dieser dunklen Messe. Ich, Nehamal, der ich vom Drachen Nehal beseelt bin, ernenne euch beide, die ihr hier vor mir und vor Nehals Stein ruht, zu erleuchteten Drachenkultisten. Zu Beseelten. Möget ihr die Stimmen der Drachen deuten und uns leiten in den Tagen, die da kommen. Von nun an hört ihr nie mehr auf die Namen Niamos und Kataka, denn euch werden würdigere zuteil. Legt diese Ketten an und nehmt eure neuen Namen an: Samagrak und Tamarok!“

Jubel ertönte.

Iolith stolperte zurück, fiel auf ihren Rücken und ließ lautstark Geröll den Hang herunterrollen. Während sie selbst hinter eine Erhebung rollte und verborgen blieb, presste die vor Schreck erstarrte Griun sich für alle Augen sichtbar an den Hang.

Die Kultisten blickten auf. Mache lugten ängstlich unter ihren Kutten hervor. Die beiden nackten Menschen schauten vielmehr verwirrt einander an, als ob sie sich nicht ganz sicher wären, ob dies zur Zeremonie gehöre. Nehamal zog gar einen spitzen Opferdolch hervor.

Als er Griun erblickte, lächelte er allerdings erleichtert. „Eine Agren. Nur eine Agren.“ Die umstehenden Personen in hohen Kutten entspannten sich. Samagrak und Tamarok blickten weiterhin unsicher drein.

Dann erkannte Nehamal die zweite Anwesende. Iolith, die sich erhob, den Hang hocheilte, sich schützend vor Griun stellte und eine Sichel kampfbereit hielt.

Er lachte. „Keine Sorge, wir beißen nicht. Ich bin mir bewusst, dass unsere Riten für Außenstehende ein wenig ... ungewohnt wirken mögen.“

Der nackte Ziegenwesen hinter ihm legte seinen Kopf schief. „Eine Agren und eine Schildzwergin? Hier, bei der Statue eines Feuerdrachen? Was haben die Drachen denn nun geplant für uns?“

„Ich bin keine Schildzwergin“, trotzte Iolith.

Nehamal fragte nach: „Bist du sicher? Dein Gesicht erinnert mich wirklich stark an eine. Nur die Tattoos fehlen. Und die Haare sind ganz ...“

„Ich weiß nicht, wovon du sprichst“, sprach Iolith, „Ich mag vielleicht einst eine gewesen sein. Doch glaube ich kaum, dass mich die Schildzwerge wirklich noch als eine der ihren ansehen.“

„Verzeih mir, da scheine ich einen Nerv getroffen zu haben“, sprach Nehamal mit erhobenen Händen. Betont ruhig steckte er seinen Dolch weg, „Manchmal schickt das Schicksal einem die spannendsten Gestalten über den Weg. Bitte, gesellt euch zu uns, setzt euch, erzählt von euch.“

„Wir wollen nicht stören“, meinte Griun nun, „Wir sind gleich wieder ...“

„Unsinn, ich bestehe darauf! Ich mag es ungemein, Geschichten von ehemaligen Schildzwergen zu hören. Ihr wisst nicht zufälligerweise, wo Kreatoks versiegelte Kultstätte liegt? Eine gewisse alte Seele vermisst ein dort eingeschlossenes Artefakt zutiefst.“

Iolith hob ihre Augenbraue: „Das verborgene Heiligtum der Schildzwerge?“

„Genau dieses“, grinste Nehamal.

„Woher sollte ausgerechnet ich seine Lage kennen?“

Ein rotes Glimmen glitzerte in Nehamals Augen auf. „Früher oder später müssen wir doch auf jemanden treffen, der darüber Bescheid weiß.“\bigskip







\az{Jahr 563}

Und zu guter Letzt im Westen, beinahe fünfhundert Jahre später.\bigskip



Die Bewahrer vom Baum der Lieder schrieben das Jahr 563 nach andorischer Zeitrechnung.

Eine einsame Gestalt lief durch über das ewige Eis. Ein langer Bart fiel auf eine junge Tulgori, die in eine Decke eingewickelt von muskulösen Armen getragen wurde. Arme, so blauweiß wie das Haar und die Kleidung der Gestalt.

„Und in diesem Augenblick richtete sich Tarokdur zu seiner vollen Größe auf und verschluckte Sagramak mit einem einzigen Bissen!“, berichtete Ijsdur.

„Nein, das ist gemein!“, protestierte Nalle schwach. Erneut schüttelte sie ein Hustenanfall.

Ijsdur nickte und passte seine Geschichte an: „Was ich sagen wollte: Tarokdur wollte Sagramak mit einem einzigen Bissen verschlingen. Doch in diesem Augenblick brachen die ersten Sonnenstrahlen des neuen Tages über die Rietburg. Und prompt zum ersten Hahnenschrei zerbarst die Verliestür. Iril trat stolz heraus, ihr Hammer so leuchtend wie ihre ganze Haut, und stellte sich tapfer vor den gewaltigen Drachen. Sie schwang ihren Hammer auf ihre Runenscheibe, ein grüner Blitz zackte vom Himmel herab, und der Drache zerfiel in hunderte kleine Schneeflocken. Das Land war gerettet. Und seither trafen wir uns immer wieder mit den restlichen Helden von Andor, um zu helfen, wo wir konnten. Für mich ist das Ganze schon beinahe 500 Jahre her.“

„Woah“, gab Nalle schläfrig von sich. Ijsdur hatte ihr noch gar nicht erzählt, wie er und Aćh such um Siantari gekümmert hatten, doch Nalle schien das nicht zu stören. Vermutlich hatte sie sich mit seiner Geschichte primär von den düsteren Gedanken zu ihrem baldigen Dasein als Eis-Dämonin ablenken wollen. Und nun war sie so schläfrig, dass die meisten solcher Gedanken von selbst fernblieben.

„Warum trägst du nun Irils Hammer?“, fragte Nalle auf einmal, „Kannst du inzwischen auch mit Runen umgehen?“

„Sie hat mich den einen oder anderen Runentrick gelehrt. Aber so gut wie sie war, werde ich niemals sein“, erzählte Ijsdur, „Ein Glück, dass sie mich fand. Wer weiß, an wen sonst die gewaltige Macht dieses Hammers hätte gehen können. Stell dir mal vor, wie überrascht sie war, als sie herausfand, dass ich als Eis-Dämon ohne jegliches Blut in meinem Körper auch relativ gut gegen die Einflüsse der Dunklen Magie geschützt war.“

„Blut schützt vor Dunkler Magie?“, flüsterte Nalle verwirrt.

„Nun, Irils Zwergenblut schien jedenfalls die verlockende Stimme der Dunklen Magie ...“ Ijsdur hielt inne. Vermutlich hatte er Nalle noch nicht einmal davon erzählt, welche Gefahren in diesem Hammer lagen. Es war wirklich schwer, den Überblick zu behalten zwischen seinen Erinnerungen an die vergangene Zeit und davon, wie viel er Nalle soeben wirklich erzählt hatte. Gedanken waren so viel schneller als ihre Ausformulierung und Kommunikation. Erst recht, wenn man außer Übung war, mit anderen außer sich selbst zu sprechen.

„Ich trage Irils Sprachrune immer noch auf mir“, erzählte Ijsdur als Themenwechsel. Er hob seinen langen Bart und zeigte eine überraschend simple Rune aus wenigen Strichen, die unten an seinem Hals zu sehen war.

„Bei unserer ersten Begegnung zeichnete Iril sie viel größer als nötig auf meine Brust. Eine Zeit lang trug ich sie hinten am Hals, damit man sie nicht sehen konnte. Aber inzwischen trage ich sie wieder vorne. Da kann man sie mit meinem langen Bart auch nicht sehen und es wird einfacher, sie nachzuzeichnen. Ich brauche sie eigentlich nicht mehr, um mit anderen zu sprechen. Ich habe viele Sprachen gemeistert in meinem langen Leben. Ich könnte die Rune verblassen lassen. Aber ich ziehe sie gerne nach. Wie ein Ritual. Eine Erinnerung.“

Ijsdur wurde wieder nachdenklich und stumm. Nalle, die primär still in seinen Armen zitterte, sagte auch nicht.

Stumm bewegten sie sich einige Minuten lang über das ewige Eis. Dann war es so weit.

„Dies ist das Zentrum des ewigen Eises. Wir sind hier“, sagte Ijsdur. Sanft setzte er die kleine Nalle aufs ewige Eis.

„Ich werde mir nun die Eiskristallkette abziehen und dir aufsetzen“, sagte er, „Danach werde ich im Eis versinken und diese Welt verlassen. Doch du wirst dich wieder erheben als Nalledora, meine ‚Tochter‘. Es wird ein ungewohntes Gefühl für dich sein, doch kein schmerzhaftes. Im Moment des Übergangs werden sich unsere Seelen berühren und wir werden Eindrücke vom Leben des anderen sehen. Das braucht dich nicht zu fürchten. Alles wird gut werden. Vielleicht siehst du sogar einige Ausschnitte von Irils Geschichte, die ich dir vorhin erzählt habe. Bist du soweit?“

„Ich ... ich weiß nicht so recht“, flüsterte Nalle schlotternd, „Wirst du ... wirst du meine Erinnerungen wirklich sehen? Da ist nicht alles gut. Als ... als ich noch klein war, habe ich mal ... ich habe eine ganze Keksdose geklaut und es auf mein Geschwister geschoben.“

Ijsdur stockte einen Augenblick. Was für viel schlimmere Dinge die Tulgori aus seinen Erinnerungen sehen mochte, wollte er sich gar nicht ausmalen.

„Vielleicht werde ich das, wenn es für dich eine wichtige Erinnerung ist. Aber das braucht dich nicht zu kümmern. Mich wird es ohnehin sehr bald nicht mehr geben.“

Er haderte mit seiner Entscheidung. Wäre es nicht besser, jemand Älteres, Weiseres als nächsten Träger der Eiskristallkette zu suchen? Konnte er jemand so Unerfahrenem derart viel Macht verleihen? Doch letzten Endes konnte er nur hoffen, dass sein Nachfolger das Geschenk würdig weitertragen würde. Niemand konnte abschätzen, wie der neue Eis-Dämon handeln würde. Die Erinnerungen der Vorgänger würden sie anleiten. Und ein viel zu kurzes Leben wie Nalles zu retten, wäre bestimmt edler, als das einer überalten Fürstin zu verlängern. Oder?

Ijsdur versuchte ein Augenzwinkern. Nalle wirkte nicht wirklich beruhigt. Dann jedoch schluckte sie tief und sprach: „Ich bin soweit.“

Ijsdur nickte und schluckte den allzu menschlichen Kloß in seiner eigenen Kehle runter.

„Weißt du, Nalle, ich habe ein längeres und volleres Leben geführt als die meisten anderen Menschen, ja, selbst als die langlebigeren Zwerge und Taren, Temm und Trolle. Ich bin glücklich mit dem, was ich mit meiner Zeit gemacht habe. Ich kann auf viele schöne Erinnerungen zurückblicken und darauf, was ich alles gelernt habe. Ich konnte mich mit meinem kommenden Tod abfinden. Ich kann friedlich und rasch gehen. So viel, das sich so viele andere wünschten, wurde mir geschenkt. Und doch, jetzt, wo der Moment hier ist, will ich noch nicht gehen. Ich nehme an, nur die wenigsten wollen das je wirklich. Wie dem auch sei ...“

Ijsdur kniete sich neben Nalle hin und hielt seine Hand vor seinen Hals. Warum zitterte seine Faust so sehr? Er griff nach der Eiskristallkette, die tief in seiner Haut verankert war. Seine Eisfinger glitten in seinen Hals, als bestünde dieser aus weicher Butter. Die seltsame Empfindung schüttelte ihn.

Ijsdur setzte an, seine Eiskristallkette abzureißen und sein Dasein zu beenden.

Da hielt er inne.

Ein weiterer Windhauch wehte über die riesige Eisfläche und wirbelte um die beiden liegenden Gestalten herum. Er trug einen Geruch nach Metall und Verwesung mit sich.

Etwas war falsch hier.

Nalle bemerkte, wie er verharrte.

„Was ist?“, fragte sie sorgenvoll.

„Etwas ist falsch hier. Spiel mit“, zischte Ijsdur zurück. Er gebot der obersten Schicht der Schneefläche, sich in den beißenden Wind um ihn herum zu erheben und die Sicht auf ihn und Nalle zu verschlechtern. Dann tat Ijsdur so, als würde er etwas von seinem Hals zu Nalles führen. Er wankte ein wenig theatralisch vor und zurück und ließ sich aufs kalte Eis sinken. Nalle guckte ihn weiterhin verwirrt an. Dann aber folgte sie ihm und ließ sich neben Ijsdur aufs ewige Eis sinken.

Eine Minute lang geschah nichts. Ruhig lagen die beiden auf dem ewigen Eis. Ijsdur totenstill, Nalle zitternd, während sie langsam blau anlief.

„Warte noch. Warte. Nicht mehr lange“, wisperte er ihr beruhigend zu.

Und auf einmal, wie aus dem Nichts, huschte eine völlig schwarz gewandte Gestalt aus dem Schneetreiben hervor. Wie ein flatternder Umhang schien sie zunächst, doch die gezackte Maske aus Metall und die Eisernen Handschuhe verrieten, dass sich unter dem Umhang mehr als nur Stoff befand. Die Gestalt schwebte einige Ellen über dem Schneeboden und glitt rasch auf die beiden liegenden Gestalten zu. Das lange Schwert in ihrer linken Hand zitterte leicht. Es war auf Nalle gerichtet und darauf und daran, sie zu durchstechen.

Doch Ijsdur war vorbereitet. Der Eis-Dämon warf sich in die Höhe, in seiner rechten Hand einen mächtigen Eisblitz. Er schleuderte ihn auf die schwarz gewandte Gestalt, die überrascht wurde und nur haarscharf ausweichen konnte.

Donner schüttelte das ewige Eis.

„Bist du mutiger geworden? rief Ijsdur spöttisch, „Bist du überhaupt noch derselbe Herold wie zu Taroks Zeiten? Früher hättest du dich noch nicht einmal getraut, ein bewusstloses Kind ohne Unterstützung anzugreifen!“

Der Schwarze Herold sagte nichts. Seine gezackte Maske stand starr in der Luft, während sein langer Umhang darunter herumwehte.

Ijsdur setzte nach: „Ich habe die Geschichten deiner Herkunft gehört. Wenn ich dich deiner dreckigen Stiefel entledige, wirst du uns dann endlich in Ruhe lassen?!“

Eine Stimme, flüsternd und doch so durchdringend, klang hinter der eisernen Maske des Herolds hervor: „Viele Sagen und Legenden ranken sich um mich, und nur die wenigsten enthalten ein Körnchen Wahrheit. Die Geschichte mit dem Kutscher ist erstunken und erlogen. Genauso wie die Geschichten deines angeblichen Triumphs über den Urgeist des Avas.“

Ijsdur überhörte das. „Ah, du kannst ja doch sprechen! Hast du irgendetwas zu sagen, was mich dein Leben verschonen lassen würde?“

Die Antwort des Herolds klang abfällig, gelangweilt: „Lass die Einzeiler sein, Eis-Dämon. So was ist unter deiner Würde.“

„Auch gut, dann bringen wir es rasch hinter uns. Ich habe lange auf den Moment gewartet, an dem dich aus dieser Sphäre vertreibe.“

„Und ich habe lange auf den Moment gewartet, an dem ich mir deine Eiskristalle einverleibe und mächtiger werde denn je. Hätte sie lieber der Kleinen abgenommen, aber mit dir allein nehme ich es auch noch auf.“ Der Herold legte seinen Kopf schief und fügte gehässig an: „Wenn du artig bist und dich ergibst, verrate ich dir Irils letzte Worte, ehe ich dich vernichte.“

Eine offensichtliche Provokation mit dem Ziel, Ijsdur unvorsichtig zu machen. Es funktionierte. Ijsdur öffnete seinen Mund zu einem wortlosen Schrei des Hasses, zückte einen weiteren Eisblitz und sprang mit einem gewaltigen Satz auf den Schwarzen Herold zu, den Eisblitz wie einen riesigen Eiszapfen auf den Herold niederstechend.

Der Eiszapfen kollidierte mit dem rostigen Schwert des Herolds und vereiste dessen Spitze, ehe der Herold zur Seite glitt, sein Schwert quer durch Ijsdurs Brust zog und den Eis-Dämonen in zwei Hälften geteilt auf die Eisfläche prallen ließ. Nutzlos rollte der kristallisierte Eisblitz vor die zitternde Nalle.

Die gezackte eiserne Maske des Herolds drehte sich hin, dann wieder her. Ausdruckslos. Tonlos. Der Herold schien zu überlegen, ob er sich lieber ganz Ijsdur widmen oder zuerst Nalle abzustechen versuchen sollte.

Ijsdurs Körper setzte sich zusammen und richtete sich auf. Ijsdur brauchte einige Augenblicke, um sich neu zu orientieren. Diese Zeit nutzte der Herold, um sein Schwert in Ijsdurs Kopf zu versenken. Blind griff Ijsdur nach vorne und bekam Stoff zu fassen. Mit einem Ratschen löste sich ein Teil des Umhangs des Herolds, als der Finsterling erneut zur Seite glitt.

Als Ijsdurs Augen sich endlich wieder geheilt hatten, sammelte sich der Schwarze Herold noch. Sein einer Handschuh betastete seinen anderen Arm, wo Ijsdur einen Ärmel seiner schwarzen Kleidung abgerissen hatte. Darunter war gräuliche, verfaulte Haut sichtbar geworden.

Aus dem Schnee und dem Eis formte sich ein Schwert in der Hand Ijsdurs, dessen Haut nun blau zu schimmern begann. Er holte aus und schlug mit diesem Eisschwert nach dem Schwarzen Herold. Dieser holte mit seinem eigenen Schwert aus und zerschlug Ijsdurs Eisschwert, als bestünde es ... naja ... aus Eis.

Der Herold machte verächtliche Geste. „Gib auf. Ohne Eisblitze bist du so gut wie wehrlos.“

„Zur Not nehme ich dich mit Fäusten auseinander! Aber zum Glück habe ich das nicht nötig.“

Ijsdur löste Irils Runenhammer von seinem Gürtel und hob ihn in die Höhe. Die darauf eingravierten Runen begannen, flackernd zu glühen.

„Runenmagie, du? Wirklich?!“, lachte der Herold, „Du kannst ja noch nicht einmal mit Runensteinen umgehen!“

„Die Meister der Magie wissen, was sie tun, wenn sie die Balance wahren. Mit Runensteinen hätte ich dich schon längst vernichtet!“, knurrte Ijsdur. Nun wusste er mit Sicherheit, dass der Herold sich seiner Sache nicht mehr sicher war. Üblicherweise hielt er es nicht nötig, zu Spott zu greifen, um seine Gegner zu verunsichern.

Erneut sprang Ijsdur den Herold an. Diesmal schwang er den Runenhammer, dem ein magischer Schleier grünlichen Schimmers folgte. Der Schwarze Herold wich zur Seite aus und ließ sein Schwert auf Ijsdurs Hand niederfahren. Selbige löste sich samt Runenhammer und schlidderte nutzlos übers ewige Eis. Da schlug Ijsdur auf eine bestimmte Rune auf seiner Schulter, welche rötlich aufleuchtete. Der Runenhammer hob sich wie von selbst in die Luft und flog zurück in den Stumpf von Ijsdurs Arm, an dem sich in Windeseile eine neue Faust formte. Andernorts wäre es vielleicht effektiv gewesen, seinen Körper zu zerteilen, bis er nicht mehr genug Kraft zum Regenerieren hatte. Doch dies war das ewige Eis, das heilige Reich der Eis-Dämonen. Hier musste man schon seine Eiskristallkette vernichten, um ihm ernsthaft zu schaden. Und wenn der Herold diese zu klauen gehofft hatte, würde er dies nur äußerst ungern tun.

Nun durfte der Herold nur nicht verschwinden. Nicht jetzt, wo endlich eine reale Chance bestand, dieses uralte Übel aus der Welt zu befördern.

Als hätte der Schwarze Herold seine Gedanken gelesen, begann er wortlos, immer höher zu schweben und sich von Ijsdur zu entfernen. Schon flog er viele Mannlängen über ihm. Unerreichbar. Oder doch nicht?

Siantari hatte zu fliegen vermocht. Und nun war Ijsdur der Herr über das ewige Eis. Er breitete seine Arme aus. Ein Wirbelsturm aus Schnee und Eis erhob sich um ihn und riss ihn in die Höhe. Doch noch war er nicht schnell genug, den Herold einzuholen.

Ijsdur griff an seinen Gürtel und zückte ein weiteres Artefakt, das ihm einiges bedeutete. Ein schwarzes Pulversäcklein. Ein gewisser Steppennomade hatte ihm einst auf dem Sterbebett ein geheimes Rezept verraten. Und Ijsdur hatte vor dem Besuch bei Nalle nicht ohne Grund den Nestbaum der Takuri besucht und einen von Aćhs Ur[ur\^{}n]enkeln nach Turrs Asche gefragt.

Ijsdur griff tief in den schwarzen Pulversack hinein und drückte seine Faust mit übermenschlicher Stärke zusammen. Er unterdrückte einen Schmerzenslaut, als die heiße, pulverisierte Takuri-Asche sich in seiner Faust zu einem Klumpen verklebte, sich entzündete und sich in seinen Schneekörper fraß. Dann holte Ijsdur aus und schleuderte den Pulverklumpen hoch in die Luft. Die Asche entflammte beim Aufprall, ein Feuerball umhüllte den Schwarzen Herold, und auf einmal saß der Herold wieder auf dem Boden, auf dem ewigen Eis. Seine gezackte Maske blickte sich kurzzeitig verwirrt um. Dann schoss er erneut in die Höhe und versuchte, davonzufliegen, nur um erneut von einer Prise Takuri-Pulver getroffen und prompt wieder auf die kalte Eisfläche teleportiert zu werden.

Ijsdur landete einige Schritte neben ihm und grinste. So leicht würde der Herold nicht entkommen.

Der Schwarze Herold fauchte frustriert auf und zeigte drohend mit seinem langen Schwert auf Ijsdur. Dieser verschloss das Pulversäcklein wieder und hob drohend den Runenhammer, während der Herold langsam auf ihn zuschwebte.

Doch urplötzlich erstarrte der Herold und fuhr herum.

Die kleine Nalle hatte den zur Seite gerollten letzten Eisblitz gepackt und sich von hinten dem Herold genähert. Nun verharrte sie ängstlich, als der Herold zu ihr herumfuhr. Dann setzte sie eine tapfere Miene auf und stach dennoch nach ihm. Genervt schlug der Herold die kleine Tulgori mit einer eisernen Ohrfeige zur Seite. Doch ihr knisternder Eisblitz machte Kontakt mit seinem Unterarm.

Und die Wut Ijsdurs entfaltete sich.

Dies war das ewige Eis, und er war sein Eis-Dämon. Dies war sein Reich. Sein Eis. Hier herrschte er!

Von der Stelle, wo der Eisblitz den Unterarm des Herolds berührt hatte, breiteten sich in Windeseile wachsende Eiskristalle aus, welche im Nu seinen gesamten Arm mit einer dicken Eisschicht überzogen hatten. Seine Maske blieb ausdruckslos, doch daran, dass der Herold in die Höhe flog und mit seine anderen Faust nach dem sich ausbreitenden Eis auf seinem Arm schlug, erkannte Ijsdur seine Verzweiflung. Das gefiel ihm. Insbesondere auch, weil das Eis so auch auf die andere Faust des Herolds überspringen und sich von dort aus weiter ausbreiten konnte.

„Du kannst mich nicht töten. Keiner kann das!“, rief der Herold, als wollte er sich selbst davon überzeugen, dass er nicht vernichtet werden konnte. Und üblicherweise hätte Ijsdur ihm recht gegeben. Doch er gebot über Mächte, welche das Unmögliche möglich machten. Er umfasste Irils Runenhammer fester.

Bald schon war der gesamte Körper des Herolds mit Eis überdeckt. Das Gewicht zog die dunkle Gestalt in die Tiefe. Mit einem lauten Krachen stürzte er auf die ewige Eisfläche. Seine gefrorene Schale platzte, doch Ijsdur war da, um ihn gleich wieder aufs Neue im ewigen Eis einzufrieren.

Eine behandschuhte Hand des Herolds blieb frei und wackelte nach Ijsdur, doch der Rest seines Körpers war vollends mit einer dicken Schicht aus Eis und Schnee übersehen.

Gut. Sehr gut.

Ijsdur trat näher und hob Irils Runenhammer. Dann begann er, in den Schnee zu zeichnen. Runen im Schnee über dem Schwarzen Herold, im Schnee neben dem Schwarzen Herold, ja, er zeichnete gar einen Runenkreis im Schnee rund um den Herold herum.

Dabei musste er immer wieder innehalten und stark überlegen. Er wünschte sich wie so oft, Iril wäre noch hier. Ihr waren diese Kritzeleien immer so viel leichter gefallen.

Dann war er fertig. Jetzt hieß es, nur, zu hoffen, dass er keinen Fehler gemacht hatte.

Ijsdur langte in seine Brust, versenkte seine Finger in seinem Körper, als wäre er durchlässig, und zog etwas aus seinem Inneren hervor, was seit Jahrhunderten dort drinnen gesteckt hatte.

Eine völlig verrostete Metallscheibe, in die einige Runen eingeritzt worden waren. Die Runenscheibe, die Iril vor all dieser Zeit genutzt hatte, um Ijsdur von Siantaris Willen zu befreien. Voller Runen, die dem Vertreiben von fremden Geistern dienten. Ohne sie fühlte er sich ganz nackt. Aber es gab keinen Dämon des ewigen Eises mehr, der von ihm Besitz ergreifen wollte. Er musste sich nicht mehr mit dieser Scheibe schützen.

Ijsdur verankerte die verrostete Runenscheibe fest im Eisblock, in welchem der Schwarze Herold zitterte. Dann trat er einen Schritt zurück, hob den Runenhammer, ließ ihn Magie ansaugen und ließ ihn auf die Runenscheibe niederdonnern.

Das uralte Metall zersplitterte in hunderte Brösel, doch ein grünlich schimmerndes Abbild der Runen blieb bestehen.

Das ewige Eis rumorte. Es knackte und knirschte unter Ijsdur. Das grünliche Glühen des Hammers sprang auf die ins Eis geritzten Runen und überzog den Eisblock des Schwarzen Herold.

Dann erlosch das Glühen, so schnell, wie es gekommen war.

Der freie eiserne Handschuh des Herolds hörte auf zu zucken.

Ijsdur langte auf das ewige Eis, welches sich seinem Willen folgend verformte und zerbröckelte. Der Umhang des Schwarzen Herolds wehte davon. Seine Maske und sein Schwert, seine Eisernen Handschuhe und seine blanken Stiefel blieben übrig und kullerten sich über die Eisfläche. Leer. Der Schwarze Herold war nicht mehr.

Ächzend eilte Ijsdur zu Nalle zurück. Diese lag weiterhin frierend auf dem ewigen Eis, doch zitterte sie nicht mehr und reagierte auch nicht mehr auf Ijsdurs Rufe. Ihr Puls war unregelmäßig.

War es schon zu spät? Sollte er lieber einen anderen Kandidaten für den nächsten Eis-Dämon auswählen, statt zu riskieren, dass sie schon nicht mehr zu retten war?

Nein, redete sich Ijsdur zu, das nicht sein rationaler Geist, der hier sprach. Noch immer hing er ein klein wenig zu sehr an seinem Leben. Doch war es nun an der Zeit, dieses aufzugeben.

Ijsdur ergriff Eiskristallkette, die um seinen eigenen Hals hing, und riss sie gewaltsam von seinem schneeigen Körper. Er krümmte sich und ächzte, während sich von den zurückgelassenen Einbuchtungen der Eiskristalle in seinem Hals Splitter und Spalte in alle Richtungen seinen Körper entlang ausbreiteten.

Mit rissigen Händen hängte Ijsdur die Kette um Nalles Hals. Kaum hatten die kalten Kristalle ihre Haut berührt, beruhigte sich ihr Atem. Ihre Furcht und ihre Anspannung legten sich. Eine Taubheit breitete sich von ihrem Hals aus über ihren ganzen Körper aus.

„Willkommen, Nalledora, zu Deinem zukünftigen Dasein als Hüterin des ewigen Eises. Achte die Vergangenheit. Nutze die Erfahrungen deiner Vorgänger. Handle weise. Handle mit Herz. Hilf, wo du kannst. Für eine bessere Zukunft.“

Das waren die letzten Worte Ijsdurs. Er trat einige Schritte zurück, stolperte dann und verschmolz mit der Eisfläche, versank buchstäblich darin.

Nalle ersuchte vergeblich, die fremden Eindrücke und Erinnerungen in ihrem Geist zu sortieren und zu vorstehen. Ihr Körper gehorchte ihr nicht mehr. War dies ihr Ende?

Mit diesen Gedanken starb Nalle, und mit ausgebreiteten Armen und eisigem Blick erhob sich Nalledora. Ihr Körper gehorchte ihr wieder. Sie hob ihre nicht mehr zitternden Hände. Diese waren schneeweiß geworden.

Langsam erhob sie sich. Es schien, als ob sie dem ewigen Eis des Kuolema-Gebirges entwuchs. Sie schaute auf die Gestalt, die vor ihr lag. Die Gestalt ihres "Vaters", des Eis-Dämons Ijsdur, die mehr und mehr vor seinen Augen verschwamm und mit der endlosen Eisfläche eins wurde, bis nur noch einige wenige Eiskristalle übrigblieben. Sie rieb sich die nun eisblauen Augen.

Vier Artefakte lagen über Ijsdurs aufgelöstem Körper auf der Eisoberfläche. Artefakte, die vorhin noch an Ijsdurs Gürtel gehangen hatten. Artefakte, die sie dank Ijsdurs Erinnerungen näher einordnen konnte. Eine gespaltene Heldenbrosche. Ein schwarzer Pulversack. Eine feuerrote Schreibfeder. Ein Runenhammer.

Warum hatte Ijsdur sie ihr überlassen, statt sie würdigeren Trägern zu übergeben?

Er musste eine große Hoffnung in sie setzen.

Nalledora spürte gedämpfte Freude darüber, wieder auf kräftigen eigenen Beinen stehen zu können. Sie hopste auf und ab. Dann hielt sie inne und betrachtete das eiskalte Gewand, das das ewige Eis ihr verliehen hatte. Sie strich mit den Fingern darüber und spürte, wie es sich ihren Wünschen gemäß verformte. Tiefe Taschen erschienen an der Seite des Kleids. Nalledora kniete sich hin und nahm die vier magischen Artefakte an sich.

Dann drehte sie sich um, und schlenderte davon. Und bald schon schritt sie durch das Felsentor und verließ das ewige Eis und das Fahle Gebirge.

Zeit, nach Tulgor zurückzukehren. Ihre Väter mussten schon umkommen vor Sorge.














\begin{chapterbox}
    \chapter{Stürmische Rätselnacht in der Taverne (2019 bis 2020)}
    \label{Stürmische Rätselnacht in der Taverne (2019 bis 2020)}

    \az{Jahr 66}

    Während einer Sturmnacht verbleiben viele Andori im "Trunkenen Troll". Ein betrunkener Bauer erzählt Schauergeschichten von Skralreigen. Mysteriöse Steintafeln mit unbekannten Schriftzeichen geben Gilda Rätsel auf. Die Wassermagierin Jarid und der Feuerkrieger Trieest aus dem fernen Danwar treffen einen blinden Seher.
\end{chapterbox}



\section{Die Sage vom Butterbrotbären}

\az{Jahr 66}

Nein, nein, es ist natürlich nicht so, dass ich beim Auswählen meines Nutzernamens einfach einige Wörter alliterativ zusammengewürfelt hätte, auf dass ihre Gesamtheit angenehm klinge. Vielmehr steckt hinter diesem Namen eine Geschichte – und selbige würde ich nun gerne mit euch teilen. So vernehmet die Sage vom Butterbrotbären:

Es war früh am Morgen (oder noch spät in der Nacht?) als Gilda dabei war, ihr berühmtes andorisches Tavernenbrot mit feiner Butter (von ihrer Ziege Molli) zu backen. Ein ganzes Brett Butterbrote stand bereits auf den Fenstersims und verbreitete seinen frischen Brotgeruch in der gesamten Backküche. Gerade drehte Gilda sich um, um ein neues Tablett zu holen, da vernahm sie ein lautes Krachen hinter sich. Als sie sich umblickte, lag das erste Brett am Boden und die Brötchen waren nirgendwo mehr zu sehen. Geschwind eilte sie zum Fenster, um sich nach dem Dieb umzusehen – doch in der Dunkelheit war ohnehin kaum etwas zu erblicken. Als Gilda sich bückte, um wenigstens das gefallene Tablett wieder aufzuheben, fand sie jedoch darunter ein versiegeltes Pergament. Ein Brief? Schnell rollte Gilda das Papier auf – Grossvater Erloth hatte ihr schliesslich nicht umsonst Lesen und Schreiben beigebracht – und versuchte zu entziffern, was darauf stand. Doch die meisten Symbole auf dem Dokument waren ihr völlig unbekannt. Einige Kritzeleien sahen aus, als hätte jemand versucht, ein Tier zu zeichnen. Mysteriös, die ganze Sache.

Eine Woche darauf war Gilda erneut dabei, Butterbrote zu backen (und jawohl, in der Andor-Welt werden Butterbrote als solche gebacken, nicht erst später geschmiert), als sie erneut hinter sich ein Krachen vernahm. Die Brote zu retten vermochte sie nicht mehr, dafür war es schon wieder zu spät. Aber diesmal konnte Gilda einen Blick auf den Übeltäter werfen, wie er aus dem Lichtschein ihres Fensters in die Dunkelheit des Südlichen Waldes verschwand – und es war kein Mensch, vielmehr ein pelziges Ungetüm, welches geschwind, wenn auch ungeschickt, alle Butterbrote in sein Maul gepackt hatte und damit davongerannt war. Wie schon beim ersten Mal fand Gilda auf dem zurückgelassenen Butterbrotbrett ein Schriftstück. Dieses hier war allerdings eine Schiefertafel, in welche jemand – oder etwas – krakelig einige Zeichen hineingekratzt hatte. Mit etwas Fantasie könnte man darin ein Strichmännlein erkennen. Was das wohl zu bedeuten hatte?

Die nächste Woche war Gilda gewappnet. Schon wie zuvor backte sie ihre berühmten Tavernenbutterbrote und stellte ein Brett voller frisch gebackener Backwaren auf ihren Fenstersims zum Auskühlen. Doch als sie dem Fenster den Rücken zuwandte, war sie auf das Krachen gefasst. Blitzschnell fuhr sie herum, sprang behände über den ganzen Haufen von Schiefertafeln, den der Langfinger diesmal zurückgelassen hatte, und aus dem Fenster hinaus. Gilda rannte hinter dem diebischen Riesen hinterher – aha, ein Bär war das! Selbiger raste schwerfällig davon, als wäre der Teufel hinter ihm her, und verschwand im Dickicht des Südlichen Waldes. Doch Gilda gab noch nicht auf. Einst hatte sie sich ein wenig Fährtenkunde von Fenn, dem Fährtenleser, beibringen lassen, und das konnte sie jetzt nutzen. Wobei die Schneise, welche das braunfellige Ungetüm hinter sich gelassen hatte, selbst für absolute Anfänger in der Kunst des Nachstellens unglaublich offensichtlich gewesen wäre.

Der Spur der Verwüstung durch Gebüsch und Gesträuche folgend, erreichte Gilda nach einem Marsch von einigen Minuten eine kleine Hütte in einer vom Mondlicht beschienenen Waldlichtung. Das war seltsam, sie hatte nicht gedacht, dass sich hier eine befände – und sonst kannte sie sich eigentlich gut im Südlichen Wald aus. Vorsichtig näherte sie sich dem Gemäuer. Der Geruch nach frisch gebackenem Butterbrot – ihrem Brot! – lag in der noch kühlen Waldluft. Das erste Vogelgezwitscher kündigte den baldigen Sonnenaufgang an. Und Gilda trat an das Fenster des kleinen Häuschens und guckte hinein.

Das Innere der Hütte war von einem warmen Kaminfeuer erleuchtet und sah unglaublich gemütlich aus. Gilda erhaschte einen Blick auf eine grosse Stube, in deren Holzwände zahlreiche Zeichen hineingeschnitzt worden waren, ebenjene Zeichen, die sich auch auf dem Pergament und den Schiefertafeln befanden, die das butterbrotestehlende Wesen zurückgelassen hatte. Am Boden erkannte sie einen waschechten tulgorischen Teppich – wertvolle Ware war das! In einem grossen Sessel vor dem Feuer sass eine Gestalt, Gilda den Rücken zugewandt, und genoss augenscheinlich eine Tasse dampfenden Gebräus.

Gilda lehnte sich ein ganz vorsichtig ein wenig weiter vor, um zu sehen, wer denn da in dieser Hütte sass. Da knackte es unter ihren Schuhen – ein trockener Ast, das war ja auch zu erwarten gewesen – und die Gestalt im Sessel schreckte auf. Rasch zog Gilda sich zurück und rannte los zur Taverne, da krachte auch schon die Tür zur geheimnisvollen Hütte auf. Im hellen Schein aus der Hütte erblickte Gilda den diebischen Bären, welcher aus der Tür getreten war und nun bedrohlich brummelnd schnell auf sie zuhetzte. Geschwind drehte sie sich um und hetzte zurück zur sicheren Taverne, durch Unterholz und Gestrüpp, in der Hoffnung, es würde ihren massigen Verfolger etwas abbremsen.

Dem war auch so: Als Gilda die sichere Pforte der Taverne erreichte, hatte sich der Bär erst gerade aus einem Dornenstrauch am Ende des Südlichen Walds befreit. Und wie wir alle wissen, dürfen Bären die Taverne zum Trunkenen Troll ja nicht betreten. Somit blieb dem Bären nichts übrig, als sich nach einem letzten Blick auf das Gebäude langsam umzudrehen und zurück in den Wald zu trotten.

Gilda beobachtete vom Fenster aus, wie das Tier das Weite suchte, während ihr Atem sich langsam wieder beruhigte. In den Tagen, die da kommen, versuchte sie ein-, zweimal im Tageslicht, die mysteriöse Hütte wiederzufinden – erfolglos, die Spuren des Bären führten ins Nichts. Und natürlich berichtete sie in der Taverne von dieser Begegnung, aber mehr ausser Orfens spöttische These, dass sie vielleicht etwas über den Durst getrunken hatte, konnte sie nicht hervorbringen. Nicht einmal bei den Bewahrern vom Baum der Lieder hatte man von dieser seltsamen Hütte gehört. Und nicht einmal Eara konnte den vom Bären zurückgelassenen Schiefertafeln Informationen entlocken, und sie hatte es weiss Mutter Natur versucht.

Seitdem stellte Gilda ihre berühmten Butterbrote jedenfalls nicht mehr frisch gebacken auf ihre Fensterbank. Bloss von Zeit zu Zeit liess sie einige wenige am Rande des Südlichen Waldes liegen. Wenn sie später wiederkam, würden die Brote bereits weg sein, das wusste sie, und manchmal fand sie sogar einige weitere unerzifferbaren Schriftstücke an deren Stelle. Aber den Butterbrotbären sah sie nie mehr – und vielleicht waren es auch einfach irgendwelche Waldgeister, die die Brote jeweils wegschnappten und sich einen Streich mit ihr erlaubten.

Hier endet die Sage vom Butterbrotbären. Gilda hofft immer noch darauf, dass eines Tages jemand die Taverne besucht, der die rätselhaften Botschaften auf den Tafeln entziffern kann. Ob das der Fall sein wird, kann noch nicht mal ich sagen…



\newpage
\section{Der Skralreigen}


Ich hab sie gesehen! Ein Skralreigen! Ich schwör’s euch, ich hab den gesehen! Weit draussen im Rietlande war ich, alleine der Narne entlangschlendernd auf dem Nachhauseweg durch das helle Mondlicht – ich muss euch ja gar nicht erzählen, wie wild die Party in der guten alten Taverne diese Nacht war, ihr wart ja selbst alle dabei, ihr elenden Halunken! Also ging ich da diesem Fluss entlang, sturzbetrunken natürlich, und da erblickte ich im Licht des roten Mondes, dass das Graue Gebirge sich bewegte!

Zunächst dacht’ ich natürlich, ich spinne! Ich mein’, das Graue Gebirge steht da schon, seitdem Mutter Natur es als Keil zwischen diese guten Lande und das düstere Krahd getrieben hatte, und einen solchen Keil wird weder wanken noch weichen, egal, was geschieht, so sagten’s diese eitlen Gesandten der Bewahrer jedenfalls, wo sie unser Dörfchen besuchten, als ich noch ein kleiner Hosenscheisser war, und uns allen Wissen und Tugend übermitteln wollten. Diese armen Kerle, ihre Saat hat bei uns Haudegen kaum gefruchtet. Aber ich schweif’ ja mal wieder grandios ab!

Ach ja, das Graue Gebirge! Eben, ich sah mit meinen eig'nen Augen, wie der Gipfel des Skralbergs sich vor mir sich etwas in die Höhe erhob und wieder niedersenkte. Das haute mich glatt von den Socken, und so setzte ich mich auf meinen gepflegten Allerwertesten und staunte die riesigen Gesteinmassen vor mir an, die immer noch herumschwankten.

Nein, es war schon keine grosse Bewegung, aber dennoch deutlich wahrnehmbar.

Halt’s Maul, Mard, willst du mich etwa einen Lügner nennen?! Du bist mehr Trunkenbold, als ich es je sein werd’!

Jedenfalls näherte ich mich dem wogenden Grauen Gebirge, und da, an einer Flussböschung, erkannte ich ein Feuerchen. Und ihr kennt mich, ich bin immer zu haben für ein Feuerchen, am Feuerchen tun Gesellschaften ihr Innerstes kund, nichts Schlechtes hat je mit ’nem Feuerchen begonnen, bla bla bla, da nähere ich mich also diesem Feuerchen und den Personen, die im Kreis drum rumsitzen.

Gilda, Schätzchen! Haste noch ’n bissle vom Rachenputzer der Schildzwerge übrig? Ein halbes Mass gerne!

Also, diese Gestalten, die da im Kreis um das Feuerchen rumsitzen, ich näher’ mich denen, und mir bleibt fast die Spucke im Mund stecken, als ich erkenn’, was die sind: Das waren Skrale! Waschechte Skrale, einer neben dem anderen schön brav um dieses grosse Feuerchen rumsitzend. Nicht nur die Krieger, die unsereins des Nachts terrorisieren kommen, nein, auch ihre Frauchen und Kindlein, und, behüte mich, selbst eine grausige Skralhexe hatte sich an der Stelle breitgemacht.

Nun, zu diesem Zeitpunkt hatt’ ich natürlich schon mit meinem bescheidenen kleinen Leben abgeschlossen. Eine ganze Skralhorde, das überlebten nicht einmal die prächtigen Helden von Andor! Nicht persönlich gemeint, Eara, ich mag dich, aber das waren bestimmt zehn, zwölf, nein, einhundert grausame Skrale! Da hättest auch du nichts mehr ausrichten können.

Die blickten mich alle an, als wär’ ich ein Geist. Und das Wunder war, die haben mich nicht angegriffen, nicht mal angefaucht. Stattdessen klopfte so ein Wichtigtuer mit seinem Stab auf den Boden – was meinst du, Eara? Wen kümmert das schon, wie der Stab ausschaute, wie so 'n Zauberstab jedenfalls, mit so’n Paar Klunkerchen dran, auf die der geizige Garz ganz schön anspringen würde. Also ja, der klopft mit seinem Stab auf den Boden, und die Skralhexe, gebeugt und bucklig überragt’ die mich immer noch mindestens drei Köpf’, also, diese Skralhexe, die gebietet mir, näher zu treten.

Kann natürlich nicht Andorisch sprechen, das ungebildete Pack, aber mit ihren Armen und einigen grossen Gesten konnt’ dieses Ungetüm von ’ner Hexe sich schon verständlich machen. Ich hab’ natürlich schon die Gerüchte gehört über die Skralreigen, die des Vollmonds am Skralberg tagen, um Dunkle Rituale zu vollbringen, und ich wusst’ weit hinten in meinem sturzbesoffenen Hinterkopp sicher auch, dass sie in jenen Nächten ums Verrecken kein Blut vergiessen wollen, weil, wenn man den Geschichten Glauben schenkt, so kriegen die Skrale in diesen Vollmondnächten ihren dreckigen Nachwuchs, und ich mein’, wenn man diese Zeit der Geburt mit Kriegereien verbringt, so wird der Nachwuchs schwach und verkümmert, weil man von seiner Kriegskunst stiehlt, bla bla – aber hey! Ich weiss’ ja offensichtlich doch noch’n bissle was davon, was diese Bewahrer uns beibringen wollten!

Ja, natürlich ist das Quark mit Sosse, diese Skrale glauben doch absoluten Blödsinn, aber jedenfalls würden sie diese Nacht niemanden angreifen, auch nicht mich, ich war also vergleichsweise sicher, solange der Vollmond da stand.

Trotzdem schlotterte ich wie Eschenlaub, als ich mich denen näherte. Und angewidert war ich auch. Aber auch betrunken bis zum Gehtnimmer, also dacht’ ich, was solls, und setz’ mich zu diesen Kerlen ans Feuer. Die sitzen da rum und murmeln ’n bissle was, so’n kehliger Singsang halt, die alte Hexe am lautesten. Einige blinzelten mir entgegen und ich blinzelte zurück und einige starrten mich nur aus dem Augenwinkel an und nein wirklich, was mach’ ich eigentlich hier?! Ich konnt’ di ganze Zeit nur daran denken, was die Skrale dem armen Gerdan und seiner Familie angetan hatten. Da hatt’ ich mich schon anders besonnen und woll’t aufstehen und mich wieder vom Acker machen, Dunkle Hexerei war das, was die da taten, Dunkle Hexerei, das sag’ ich euch, aber als ich Anstalten machte, aufzustehen, packte mich einer dieser sitzenden Grobiane mit einer seiner dreckigen Tatzen an der Schulter und zog mich wieder nach unten! Die wollten nicht, dass ich abhaute! Da fiel mir das Herz schon ’n bissle in die Hose. Ich wusst’ ja nicht, was für’n Ritual so’n Skralreigen abhält und ob die dafür Opfer brauchen oder so. War mir aber sicher, dass sie mich nicht einfach nur in den Kreis eingeladen hatten, um mir ihre Kultur oder so zu präsentieren, die geben ja nicht viel auf Menschen, ausser als Futtterquelle.

Was? Eara, ich konnt’ doch nicht auf ihre Kleidung achten, ich war zu sehr um mein Leben besorgt! Bei meiner Seele, ich werd’ im Moment, wo Gevatter Tod seine Flügel über mich ausbreitet, seinen Schwanz um mich windet, mich zu seinem Gesicht hochhebt und mich mit seinem Feuerstrahl versengt, in dem Moment werd’ ich mich nicht gross darum kümmern, ob’s n grossen Unterschied zwischen den weiblichen und männlichen Skralen gab, Schande über mich! Ja, ’n paar hatten einfach Kindlein auf den Knien sitzen, und sahen weniger kriegerisch aus, da dacht ich nur... egal, weiter jetzt!

Ich sitz’ da also vor mich hin, umgeben von gruseligen Skralen, und mal mir aus, was die mit mir anstellen werden, sobald die Sonne aufgeht und ihre Vollmondnacht des Friedens vorbei ist, da hör’ ich ein Rummeln vom Skralberg, dann kam das näher, und dann flammte das Feuer plötzlich doppelt und dreifach auf, und kein Scherz, ich konnt ’n Gesicht darin erkennen, das grinste bösartig in die Runde.

Die Skrale jubilierten, und einer von ihnen stand auf und rief was, das wie „Darakinai“ klang. Er griff nach einem durchlöcherten, prall gefüllten Sack, welcher neben ihm stand, und zog einen riesigen Säbelfisch hervor, präsentierte ihn stolz den Anwesenden, und warf ihn dann ins Feuer, wo dieser riesige neu erschienene Feuergeist stand. Ich beobachtete, wie der silbrig glänzende Fisch vom Feuer verzehrt wurd', und der Feuergeist noch stärker grinste. Dann – mir fielen die Augen fast aus’m Kopf – krabbelte aus dem Feuer ein klitzekleiner Skral hervor, der in die Runde starrte und seine Zähne bleckte. Sofort sprang der Skral, der den Fisch ins Feuer geworfen hatte, auf, und schnappte sich das kleine Wesen.

Als wär’n Bann gebroch’n worden, standen noch viel mehr Skrale auf, und sie alle griffen in Säcke, die sie mitgebracht hatten, und warfen Fische, Eidechsen, Vyperas, behüte, sogar einen Hund glaube ich geseh’n zu haben! Also eben, sie warfen diese toten Tiere ins Feuer und kaum hatten die das Feuer berührt, krabbelten weitere kleine Skrale aus den Flammen des Feuergeists hervor und stürzten sich in die Menge. Die erwachsenen Skrale schnappten sich die Kinder, wo sie konnten, und brüllten und tanzten ausgelassen – ein Wunder, dass im Dorf auf der anderen Seite des Hügels nicht alle Bewohner senkrecht auf ihren Matratzen standen, mich hätte das Gejohle bestimmt aus dem Federn geholt. Aber dort drüben tat sich nichts, und die Skrale wurden immer wilder, und einige hauten einander und auch mir ihre Pranken auf den Rücken, und die Kindlein wurden in die Luft geworfen, und ich, mir wurde speiübel, weiter zuzusehen, darum schlich ich mich so leise ich konnte rückwärts aus der Masse raus. Die waren inzwischen so ekstatisch, die achteten nicht mal auf mich! Viele hatten sogar ihre Augen geschlossen und schmusten mit den kleinen Skralbalgen, die aus’m Feuer traten.

Dann war ich raus aus der Menge, und ich nahm meine betrunkenen Beine in die Hand, und torkelte, nein, rannte so schnell ich nur irgendwie konnte den bemoosten Hügel hinunter, zur sicheren Narne, weg von den Skralen, weg von diesem höllischen Skralberg.

Ich war kaum ’n Dutzend Schritt vom Lager entfernt, da tönte ein langgezogenes Kreischen hinter mir. Als ich mich umgedreht hatt’, sah ich nur, wie die Skralhex’, diese verfluchte Skralhexe, mir nachstarrte und auf mich zeigte, die langen Arme ausgestreckt, den Kiefer zu diesem schauderhaften Schrei geöffnet, der durch Mark und Bein ging. Und der Feuergeist, der hörte auf, Skralkindlein auf die Lande loszulassen, und hob ab von seinem Lagerfeuer, setzte sich in Bewegung und huschte über das trockene Gras hinweg auf mich zu, riesig, feurig heiss, und Zeter und Mordio spuckend. Die Skrale wollten mich doch noch als Opfergabe nutzen, ich hatt’s ja gewusst!

Doch der Feuergeist war nicht schnell genug. Es gelang dem feurigen Biest zwar, mich einzuholen – guckt ma’, die Brandnarben auf meinen Arm, die wird’ ich meiner Lebtag nimmer loswerden – aber kaum hatte der Geist mich erreicht, so war ich nah genug am rettenden Ufer, und ich warf mich in die Fluten der Narne, betend, dass ich keine Klippe treffen möge.

Ich klatschte hart auf’m kalten Wasser auf, und ging schon in den ersten paar Schwimmzügen beinahe unter – jetzt wünscht’ ich mir, einer der edlen Herren der Rietburg zu sein, die sich’s leisten können, in der Freizeit schwimmen zu gehen. Ich, ich konnt’ nur angestrengt strampeln und ich hatt’ schon zum x-ten Mal in dieser Nacht mit meim armselgen Leben abg’schlossen, da sank ich zum Grunde der Narne wie’n Stein.

Doch die Flussgeister und vielleicht sogar der grosse Arkteron höchstpersönlich müssen mir huldig gewesen sein, denn ich wurd’ wieder hochgehoben, und der Fluss spuckte mich auf der anderen Uferseite aus, weg von dem Skralreigen, weg von dem tobenden Feuergeist, weg von dieser grusligsten aller Begegnungen, die ich je in meim ganzen Leben erlebt hatte.

Nein wirklich, ich sag’s euch, die G’schichte stimmt. Was für’n Grund hätt’ ich, mir so’n verstörenden Quatsch auszudenken?! Ich hab den Skralreigen gesehen! Den Skralreigen! Und so warn’ ich euch jetzt, sodass ihr euch meine Worte für immer merken sollet: Wenn der Vollmond scheint über dem Lande Andor, so meidet den Skralberg. Meidet den Skralberg, wenn euch euer Leben lieb ist!






\newpage
\section{Die Wassermagierin, der Feuerkrieger und der Seher}



Es regnete in Strömen. Schon seit Stunden peitschten Wassermassen gegen die Fenster der Taverne zum Trunkenen Troll, doch die Tavernengemeinschaft liess sich dadurch nicht die Stimmung vermiesen. Gildas Metvorrat würde noch lange reichen, und falls man des schlechten Wetters wegen nicht nach Hause konnte, so wurde halt in der beliebtesten Taverne von Andor gleich doppelt und dreifach so lange gefeiert.

Die Tür zur Taverne öffnete sich langsam, als hätte jemand sachte dagegen gestossen.

Seltsamerweise fielen keinerlei Regentropfen durch die dunkle Öffnung ins Innere des gemütlichen Raums. Eine Hand erschien aus der Dunklen und griff an den hölzernen Rahmen der Tür, danach ein wohlbekanntes Gesicht, welches in den hell erschienenen Raum guckte – es war Jarid, die Wassermagierin aus dem fernen Danwar! Ihr sonst übliches fröhliches Lächeln fehlte, stattdessen zeigten ihre Gesichtszüge Unbehagen, ja sogar Sorge!

Sie schleppte sich in den Schankraum. Ihr selbst ging es ja prima, aber ihrem stetigen Begleiter Trieest eher weniger. Ihn sorgsam unter einer Achsel stützend, hievte sie den schwer berüsteten und vollkommen durchnässten Feuerkrieger vorsichtig über die Türschwelle und hinein in die gute Stube. Kurz flatterte sie mit ihren Fingern, und aus ihrem Haar, ihrem verzierten Umgang, ihren Schuhen, Trieests Rüstung, seinen Ohren – von überallher perlten kleine Wassertropfen hervor und schwebten, geleitet von Jarid, hinaus aus der Taverne in die finstere Sturmnacht. Mit einem dumpfen Knarren fiel die Tür hinter ihnen ins Schloss.

Der nun wieder trockene Trieest konnte sich kaum auf den Beinen halten. Er ächzte und schnaufte, doch war äusserlich keine Verletzung an ihm zu erkennen. Der fest in seiner Brust verankerte Lavastein leuchtete in einem satten Orangeton und strahlte spürbar Hitze ab. Der Stein war es wohl auch, der Trieest so sehr zu schaffen machte – die Hitze selbst schien ihm ja nie etwas anhaben zu können, doch litt er auf eine ganz andere Weise durch das Tragen dieses magischen Edelsteins.\bigskip



„Bitte, Jarid, ich kann nicht mehr lange“, flüsterte Trieest flehend aus brüchigen Lippen hervor.

Jarid blickte ihn mit einem traurigen Blick an und sprach leise: „Tri, du weisst, dass ich den Stein nicht entfernen darf, selbst, wenn ich mir sicher wäre, wie es funktioniert. Ich helfe dir ja schon beim Tragen deiner Bürde. Und ich tue, was ich kann, um die Pein zu lindern.“

„Fünf Jahre! Fünf Jahre trage ich ihn jetzt schon auf mir! Was verlangt Mutter Natur denn noch von mir?!“, stiess Trieest zwischen zusammengebissenen Zähnen hervor, ehe er wieder zum bettelnden Ton wechselte, „Nur schon fünf Minuten... fünf Minuten Ruhe und Stille... ich bitte dich...“

Jarid hielt kurz inne und schien zu überlegen, doch dann setzte sie ein strenges Gesicht auf und sagte bestimmt: „Nein. Wir wissen nicht mal, was mit dir geschehen würde, wenn ich die Verbindung kappte. Ich will den Ältesten nicht berichten gehen müssen, dass dein Prozess des Wandels nochmals von Grund auf zu beginnen hätte. Und du willst das doch auch nicht. Der Stein bleibt, wo er ist. Kümmern wir uns lieber um deine Schmerzen, da kann ich tatsächlich von Hilfe sein.“

Trieest brummelte etwas unter seinem Atem und seine langen Ohren wackelten. Dann liess er sich mühselig auf die nächstgelegene Sitzbank fallen und stöhnte schmerzerfüllt auf. Jarid quetschte sich sorgsam neben ihn.

„Gold...“, ächzte Trieest jetzt, „Hast du noch etwas von der Goldsalbe übrig?“

Jarid fischte ein kleines braunes Döslein aus einer Tasche ihres blauen Gewands und schraubte es auf. Ihr enttäuschter Blick sprach Bände: „Wasser wird es tun müssen. Viel Wasser.“

Mit diesen Worten blickte die Wassermagierin um sich, erwartend, von einem Haufen neugieriger Bauern umringt zu sein, von denen bestimmt einer ein Fass voller Wasser zu holen vermochte. Mit Erstaunen nahmen die beiden Danware jedoch wahr, dass sich noch kein Trubel gebildet hatte, ja, dass sich gar niemand gross um sie scherte. Das war aussergewöhnlich, da die beiden doch noch jedes Mal im Nu die Hauptattraktion in der Taverne zum Trunkenen Troll geworden waren, wenn sie bislang dort vorbeigeschaut hatten. Tatsächlich waren heute die meisten Besucher der Taverne aber in einer ganz anderen Ecke des Raumes um einen grossen Tisch versammelt und diskutierten angeregt.

Schnell fischte Jarid einen Trinkschlauch aus einer Falte ihres Kleids – nur noch zur Hälfte gefüllt mit abgestandenem Brunnenwasser, aber bei weitem besser als nichts – zog den Zapfen heraus und konzentrierte sich auf die klare Flüssigkeit. Mit einigen geübten Gesten, begleitet von einem dumpfen Summen und leuchtend blauem Schein, geleitete sie das Wasser aus dem Schlauch auf Trieest Brust, wo es eine Schicht über dem Lavastein bildete. Trieest seufzte auf und entspannte sich ein wenig. Doch bei der Hitze, welche der Lavastein abstrahlte, war das Wasser bald verdampft. Jarid versuchte erfolglos, die Tropfen wieder aus der Luft zu greifen. Die Tür zum Unwetter draussen wollte sie nicht mehr öffnen, wenn es sich vermeiden liess – es war sehr anstrengend, den Orkanböen Sturmwasser zu entreissen, und sie war bereits stark ausgelaugt. Gilda hatte bestimmt einige Fässer frischen, sauberen und vor allen Dingen nicht widerspenstigen Trinkwassers in ihrem Keller, welche Jarid vielleicht verwenden dürfte.

„Ich werde Gilda nach Wasser fragen, warte hier“, sprach sie zu Trieest. Dieser lachte heiser auf und bereute dies gleich wieder, als das gewohnte Ziehen in seiner Brust zurückkehrte. Wo wollte er schon hingehen in seiner jetzigen Kondition? Momentan würde es ihn nicht mal stören, von neugierigen Andori umringt zu sein. Ihre mitleidigen Blicke würden ihn bestimmt schon etwas besser fühlen lassen.\bigskip



Jarid bewegte sich auf den grossen Tisch in der Ecke der Taverne zu, um den sich die meisten Tavernengäste versammelt hatten. Sie erkannte Eara, eine Zauberin aus dem noch ferneren Hadria und gute Freundin von ihr, wie sie mit geschlossenen Augen, den Körper leicht vor- und zurückwippend, ihre Fingerspitzen über die Tischoberfläche gleiten liess, während die umliegenden Tavernengäste ihr laut Dinge zuriefen. Einige schienen auch Wetten darauf abzuschliessen, ob Earas Vorhaben gelingen würde – wenn der Handelszwerg Garz seit Taroks Niedergang auf den Kontinent zurückgekehrt wäre, so hätte man ihn jetzt bestimmt in dieser Menschenmenge gefunden, glücklich grinsend einigen Bauern ihre letzten Silberstücke aus der Tasche ziehend. Doch der Handelszwerg war nirgends zu sehen, und so blieb es vor allem den wohlhabenden Händlern vom freien Markt vorbehalten, die Wetten abzunehmen.\bigskip




Trieest beobachtete das aufgeregte Treiben von seiner Sitzbank aus angestrengt, die orange glühenden Augen fest zusammengekniffen, um möglichst scharf zu sehen. Viel mehr konnte er so jedoch auch nicht erkennen. Es schien sich um ein Spiel zu handeln, welches offensichtlich die Gemüter hochkochen liess. Doch da! Eine gebückte Gestalt löste sich aus der Menschenmenge und kam langsam auf Trieest zu. Die Person trug einen schweren dunklen Mantel und stützte sich im Gehen auf einen langen Stock – Jarid war das definitiv nicht, und auch sonst erinnerte sie Trieest an niemanden, den er kannte.

Der Feuerkrieger sah weit entfernte Dinge nur verschwommen, und seine Wahrnehmung von Farben war nicht dieselbe wie die der meisten Menschen, denen er bislang begegnet war, doch selbst er konnte erkennen, dass die Hautfarbe des nun vor ihm angekommenen Mannes sich von den hierzulande üblichen unterschied. Als dieser sich mühsam Trieest gegenüber niedergelassen hatte, fiel letzterem auf, dass der Neuankömmling gar nicht so alt war (oder zumindest nicht so alt aussah), wie Trieest zunächst gedacht hatte. Die langsamen, vorsichtigen Bewegungen und der Stock kamen daher, dass die Augen des Mannes von einer Binde bedeckt waren – er war blind!\bigskip



„Blau sind die Weiten des Meers“, sprach der fremdartiges Blinde zur Begrüssung, sehr zur Überraschung Trieests, welcher schmerzerfüllt hustete und mit krächzender Stimme antwortete: „Und Blau sind die Weiten des Himmels.“

Stille.

Trieest ergriff das Wort: „Seid... seid ihr aus Danwar?“

Der Blinde lachte, und sagte dann: „Selbst bin ich noch nie dort gewesen. Gehört habe ich allerdings viel. Von den örtlichen Gepflogenheiten über eure grossen Errungenschaften wie das Fangen der Blitze – bis hin zur Tatsache, dass kein Feuerkrieger seine Heimat verlassen sollte.“

Ein Grinsen huschte über das Gesicht des Mannes, als er seinen Satz beendete. Trieest atmete schwer. Da war viel auf einiges zu verarbeiten. Noch keiner, den er hier auf dem Festland getroffen hatte, hatte die Floskeln gekannt.

Nun gut, ganz richtig war das nicht. Reka hatte die Floskeln verwendet. Die alte Hexe hatte sich in ihrem ganzen Leben noch nie näher als zwei Meter an ein Gewässer gewagt und hasste schon das Überqueren von Brücken – dass sie je die Nebelinseln besucht hatte, hielt Trieest für sehr unwahrscheinlich. Sicherlich hatte sie damals mit Hexerei in seinen Geist geblickt und ihn auf die Art und Weise begrüsst, wie es für ihn am normalsten war.

Jedenfalls hatten die Bewahrer vom Baum der Lieder, deren Pergamentsammlung den Wissensstand der Andori ganz gut repräsentierte, noch nie von den Floskeln gehört gehabt. Ganze drei Tage lang hatten sie ihn in Anspruch genommen, um jedes Detail über Danwars Kultur aus ihm herauszuquetschen – und danach hatte sich Gända Jarid geschnappt und mit ihr dasselbe durchgezogen. Sechs volle Tage hatten sie insgesamt dadurch verloren! Aber das lag hinter ihnen. Jetzt reiss dich zusammen, Trieest, keine Abschweifungen mehr!\bigskip



Konnte es sein, dass der Blinde log und schon in Danwar gewesen war? Warum hätte er dann lügen sollen? Die Floskeln hatte er bewusst verwendet. Woher kannte er sie? Was wollte er von ihm? Und vor allem: Wie hatte ein Blinder überhaupt erst erkannt, dass da ein Danware vor ihm sass? Ein ungutes Gefühl baute sich in Trieest Magengegend auf – eine Mischung aus Verwirrung und Furcht.

Schnell blähte Trieest seine Nasenflügel auf und versuchte möglichst unauffällig, den Duft der Umgebung in sich aufzunehmen. In der schon etwas abgestandenen Tavernenluft lag nebst dem allgegenwärtigen Metgeruch und natürlich den individuellen Gerüchen der lauten Tavernengäste hinten am Tisch deutlich etwas Metallisches. Jawohl, da steckte einen Dolch im Gewand seines schweigenden Gegenübers. Andererseits war sich Trieest sicher, auch in seinem geschwächten Zustand dem Blinden körperlich weit überlegen zu sein. Gefährlich könnte es für ihn also nur werden, wenn der Mann ein Zauberer war. Und sein Stock mit dem seltsam geformten Ende sprach wahrlich dafür. Trieest verspannte sich. Dann fiel ihm auf, dass vorhin ein Hauch Eara an ihm vorbeigeweht war. Eara war die weiseste Zauberin, die der Feuerkrieger kannte (wenn auch die einzige), und wenn sie hier war und sich nicht um den Blinden sorgte, so musste er das auch nicht. Trieest entspannte sich wieder etwas.\bigskip



Den Blick auf sein Gegenüber richtend, fiel Trieest auf, dass schon eine Weile lang Schweigen zwischen ihnen beiden herrschte. Der Blinde machte keine Anstalten, das Schweigen zu brechen. Trieest erinnerte sich, dass er eine angedeutete Frage gestellt hatte. Er wollte wohl wissen, warum ein Feuerkrieger sich abseits von Danwar befand. Und Trieest hatte keine Lust, ihm das mitzuteilen. Eigentlich wollte Trieest nichts weiter, als dass Jarid mit einem Zuber voll kalten Wassers zurückkehrte und das Pochen in seiner Brust linderte. Er war müde. Aber er musste sich jetzt wohl um den geheimnisvollen Mann kümmern.

Jarid hätte in einer solchen Situation sicher mit einer Gegenfrage geantwortet – „Wie habt ihr von Danwar erfahren?“ – und bei jeder passenden Gelegenheit nachgebohrt, um an mehr Informationen zu kommen und zugleich das eigene Herausrücken von Informationen zu verhindern. Jarid war geschickt in dieser Kunst. Trieest auch, aber momentan wollte er sich nicht auf solch ein geistiges Gefecht einlassen. Stattdessen beugte er sich etwas nach vorne (er sog scharf Luft ein, als das Pochen in seiner Brust zunahm) und sprach leise, aber bestimmt: „Verzeiht bitte. Es war ein langer Tag, mir tut alles weh und ich will nichts mehr als einfach einzuschlafen und die Welt zu vergessen. Teilen sie mir doch einfach ihr Anliegen mit und dann sehen wir weiter.“

Trieest lehnte sich vorsichtig wieder zurück und hoffte, dass der Blinde seinem Appell folgen würde. Vielleicht war er ein Tulgori? Die Andori hatten erst kürzlich erfahren, dass hinter den Fahlen Bergen ein weiteres Land lag, und seitdem die Eisdämonin, die diese Gipfel gehütet hatte, erschlagen worden war, florierten die Handelsrouten. Mit den Tulgori waren auch einige Temm, kleine gescheite Wichtel, nach Andor gekommen – vielleicht gab es noch mehr humanoide Spezies von dort? Oder war sein Gegenüber doch ein Mensch? Der lange Mantel machte es schwer, viel mehr von ihm zu erkennen als seine Mundpartie.\bigskip



„Du hast immer noch die Augen des Ungeheuers, das du einst warst“, sprach der Blinde jetzt plötzlich. Das traf Trieest unerwartet. Etwas defensiv warf er zurück: „Woher wisst ihr das?“

„Ich muss es nicht mit meinen eigenen Augen erblicken, um es zu wissen. Ich bin ein Seher.“

Trieest versteifte sich wieder. Ein Seher, das hatte ihm noch gefehlt. Konversationen mit Sehern waren... kompliziert, gelinde gesagt. Oft hatten sie schon im Voraus eine Vorstellung davon, was man sagen wollte, weshalb Gespräche mit ihnen sehr rasch voranschritten und von Thema zu Thema sprangen – anstrengend für alle Beteiligten ausser dem Seher. Oder zumindest war es so beim grantigen Kord gewesen, dem Seher von Danwar.

Die Worte des Sehers hatten ihre Wirkung allerdings nicht verfehlt. Jetzt war Trieests Interesse geweckt. Jetzt wollte auch Trieest Informationen von seinem Gegenüber, nicht nur umgekehrt. Jetzt konnte verhandelt werden.\bigskip



„Wie lange...?“, begann Trieest. Eine reine Farce, falls der Blinde wirklich ein Seher war, wusste er schon im Voraus, was Trieest von ihm erfahren wollen könnte. Die Frage war vielmehr, was er dafür verlangen könnte. Gold oder Wertsachen? Da gab es hier nicht viel zu holen, Trieest besass nichts von Wert ausser seiner Rüstung und seiner Dornenklinge, die er ganz bestimmt nicht herausrücken würde. Wollte der Fremde Informationen? Der Seher schien schon so viel zu wissen, was könnte Trieest ihm da noch Neues erzählen?

„Darf ich?“, sprach der Blinde und streckte seine Hand nach dem Lavastein aus. Trieest fühlte sich erneut überrumpelt, nickte aber aufgeregt. „Vorsichtig, der Stein ist sehr h...“

Überrascht brach Trieest seine Warnung ab, als der Seher seine Hand auf den Lavastein legte. Der Edelstein leuchtete in vollster Pracht und strahlte solche Wärme aus, dass selbst Jarid etwas Abstand hätte nehmen müssen. Doch diesem blinden Seher schien die Hitze nichts auszumachen, er schien sie sogar zu geniessen. Wie das Licht so durch seine Hand schein, waren seine Fingerknochen deutlich als Schatten unter der dunklen Haut erkennbar. Doch waren da keinerlei Anzeichen von Verbrennungen oder Schmerz.

Stattdessen leuchteten die Augen des Sehers auf, in glühendem Weiss, so hell, dass es selbst durch die dunkle Augenbinde, die er trug, ein klein wenig hindurchschimmerte. Trieest versuchte, so still wie möglich zu sitzen. Dies könnte der Moment sein, der zum Abschluss seines Prozesses des Wandels führte. Heute, hier und jetzt, könnte er Informationen darüber erhalten, wie er die Bürde ablegen konnte. Jetzt musste er nur noch herausfinden, was der Seher verlangen würde, damit er seine Vision preisgab. Er lachte innerlich auf. Noch vor einigen Minuten hatte er nichts lieber gewollt, als dass dieser Mann ihn in Ruhe lassen würde, und jetzt konnte er kaum erwarten, was der Seher berichten würde.

Wie schon gesagt – bei Interaktionen mit Sehern geht alles sehr rasch voran.\bigskip



„Mutter Natur meint es gut mit dir“, schmunzelte der Seher, „Noch drei Monde, und dann wird Jarid den Stein entfernen können. Das Zeichen, welches den richtige Moment kennzeichnet, habt ihr ja bereits erraten, aber ich bestätige es gerne: Es sind deine Augen, Trieest, die schon so bald eine wunderschöne bernsteinfarbene Iris tragen werden. Damit es soweit kommt, musst du nur noch eine weitere – eine letzte – Prüfung deiner Entscheidungskraft durchstehen.“

Eine Vielzahl von Gefühlen schoss durch Trieest – Erleichterung, Freude und Anspannung. Vergessen war seine Müdigkeit, vergessen waren für einen kurzen Moment sogar seine Schmerzen. Drei Monde! Das war nichts im Vergleich dazu, was hinter ihm lag. Und er Tor hatte sich gefürchtet, dass es doppelt und dreifach so lange dauern hätte dauern können! Eine letzte Prüfung lag noch vor ihm, hatte der Seher behauptet. Worin die wohl bestand?

„Ich kann dir nicht sagen, worin deine Prüfung besteht, aber ich kann dir sagen, wo sie stattfinden wird. Ich sah es vor meinem inneren Auge, so deutlich, wie ich nur selten eine Vision hatte – die Insel Narkon ist es, der du dich stellen musst. Ziehe von Süden her über die Mauerberge. Dort wird der letzte Test auf dich warten. Dann wirst du wissen, worin diese Aufgabe besteht.“

Jetzt war Trieest euphorisch. Er hatte ein Ziel! Narkon! Eine geheimnisvolle Nebelinsel, über die nicht viel bekannt war. Das Ende seiner Tortur war nahe. Danach... er hielt inne. Ja, es gab endlich Aussicht auf ein „danach“ für ihn! Viel von der Welt hatten Jarid und er inzwischen gesehen, vielleicht war es ja nach dem Ablegen seiner Bürde Zeit, sich zur Ruhe zu setzen. Wollten sie nach Danwar zurück? Ihre Freunde und Familie wiedersehen? Oder sich lieber fernhalten, die alten Wunden nicht wieder aufreissen, und ein neues Leben in einem fernen Land beginnen? Er lachte. Egal, wozu sie sich entscheiden würde, es würde grossartig werden. Bald, bald würde er frei sein!

Als Trieest seine Aufmerksamkeit wieder zum Seher wandte, fragte er verwirrt: „Wartet... warum teilt ihr mir das schon alles mit? Was wollt ihr denn für eine Belohnung?“

Aufrichtig lächelnd, sprach der blinde Seher: „Nicht alle Seher verlangen eine Bezahlung. Der Seher von Danwar – ist es immer noch der grantige Kord? – sollte es wahrlich besser wissen!“\bigskip



Trieest lachte auf. Es war ein wundervoller Tag. Sobald dieser Sturm sich gelegt hatte, konnten Jarid und er in den Wachsamen Wald reisen, um von dort aus ein Händlerschiff nach Sturmtal zu nehmen. Die Taren waren freundlich und ihre Insel lag ganz nahe an Werftheim, was nun wirklich ein Streifenmardersprung von der Silbermine entfernt lag, und vor dort aus würden sie die Mauerberge überqueren, und dann... dann würden sie weitersehen. Trieest war zuversichtlich. Er hatte wieder ein Ziel, einen Sinn. Er war wunschlos glücklich.

Plötzlich flackerte ein Ansatz von Unsicherheit über die Mundpartie des Blinden. Dann hatte er sich wieder unter Kontrolle, drehte sich mit einem breiten Lächeln auf dem Gesicht um und sprach: „Grün sind die Wogen der We...“

„Spar dir das Gefasel“, unterbrach Jarid ihn kalt. Die Wassermagierin war endlich zurückgekehrt, ein grosses Fass in den Armen. Wie lange hatte sie bereits hinter dem Seher gestanden? Als Trieest ihre vor Wut blitzenden Augen sah, wusste er, dass irgendetwas nicht in Ordnung war. Seine Hoffnung schwand.

Jarid ignorierte den Seher und wandte sich direkt an Trieest: „Glaube ihm kein Wort! Das ist ein Lügner der schlimmsten Sorte!“

Trieests Euphorie brach in sich zusammen wie ein Gor, dem man zünftig eins auf den Schädel gehaut hatte. Er kannte Jarid gut genug, um zu wissen, dass sie solche Anschuldigungen nicht leichtfertig machen würde. Und als sein Blick zurück auf den Seher fiel, sah er es in dessen Mimik, unter der Kapuze – ein kurz aufblitzender Anflug von Ärger und Verdruss, aber auch Enttäuschung. Ein Anblick, der Jarids Aussagen Recht zusprach. Der Seher war ein Lügner!

Der Blinde gab sich vergeblich Mühe, seine gute Miene wieder aufzusetzen. Sein Kopf ratterte bestimmt auf Hochtouren, versuchend, eine Möglichkeit zu finden, die Situation wieder zu seinen Gunsten zu wenden. Erfolglos, offenbar, denn der Seher erhob sich zackig, verbeugte sich – fast schon spöttisch – vor der Danware und stürmte überraschend schnell für einen Blinden und ohne zu zögern aus der Taverne hinaus in stürmische, nasse, kalte Nacht. Wenn er es nicht besser gewusst hätte, hätte Trieest geglaubt, eine Träne über das Gesicht des Sehers geflossen gesehen zu haben, ehe die Tür hinter ihm zuschlug.\bigskip



Jarid setzte sich vorsichtig neben Trieest und legte ihm eine Hand auf die Schulter. Sie sagte nichts. Trieest blieb ebenfalls ruhig. Still teilten sie gemeinsam seine Trauer, den Verlust einer Hoffnung, die sich so schnell wieder verzogen hatte, wie sie gekommen war. Dann nahm Jarid all ihre Kräfte zusammen und öffnete das Wasserfass, welches ihr Gilda netterweise überlassen hatte. Vorsichtig, um ihre magische Kapazität nicht zu überstrapazieren, führte sie einen dünnen, blau schimmernden Wassersstrang aus dem Fass zu Trieests Lavastein und flocht sorgsam ein abschirmendes Netz um den Edelstein in Trieests Brust. Dieser seufzte auf.

„Woher wusstest du... “ setzte der Feuerkrieger an, doch ein Hustenanfall schüttelte ihn – beinahe wäre Jarids Wassergeflecht gebrochen! – und er schloss den Mund. Jarid hatte ihn verstanden.

In einer etwas zu hohen Stimmlage antwortete Jarid vorsichtig: „Tri, ich habe dir nicht die ganze Wahrheit gesagt.“

Trieest blickte sie stumm an, die orange glühenden Augen ausdruckslos, doch fühlte Jarid, wie er sich wieder verspannte. Seine Hände ballten sich zu Fäusten.

Ihre nächsten Worte sehr sorgfältig wählend, antwortete Jarid:

„Bevor wir Danwar... verliessen, brachten mich die Ältesten zur roten Grotte.“

Trieest blickte Jarid überrascht an. Diese fuhr stockend fort:

„Dort wurde mir... uns... eine Prophezeiung auf den Weg gegeben. Ich... ich habe dir nichts davon gesagt, weil... weil... ich durfte nicht. Ich darf nicht. Sie sagten, dass du... dass, wenn ich es dir sage... Tri...“

Trieest blickte Jarid noch überraschter an. Es kam nicht oft vor, dass sie die Fassung verlor. Doch jetzt waren Jarids Augen feucht, und auch wenn Trieest wusste, dass sie den Tränen nicht erlauben würde, zu fliessen, so war er doch besorgt. Beim Barte des Warx, was hatte das Orakel Jarid denn gesagt, dass sie solch eine Reaktion zeigte?

Jarid nahm einen tiefen Atemzug, dann hatte sie sich wieder etwas gefasst: „Ich weiss, wo du deinen Prozess des Wandels beenden wirst – und es ist nicht Narkon.“

\begin{center}
    Weiter geht es in \hypref{Der Feuerkrieger und die Wassermagierin (2023)}.
\end{center}




\newpage
\section{Eine mysteriöse Tafel}

Rund um die Taverne zum Trunkenen Troll tobte ein Sturm, wie man in Jahren keinen gesehen hatte. Die Tavernengäste, auf der Suche nach einer Beschäftigung, hatten sich einmal mehr den mysteriösen Tafeln zugewandt, die Gilda bei einer seltsamen Begegnung erhalten hatte.

Nicht einmal der alte Leander, der einst behauptet hatte, alle Sprachen der bekannten Welt zu kennen, hatte es geschafft, die Nachrichten zu entziffern. Der mürrische Blinde hatte zwar behauptet, das läge daran, dass er nicht gut ertasten könne, was für Zeichen auf der Tafel eingeritzt waren, und wenn er das Bild nur vor Augen hätte, würde ihm die Bedeutung sofort klar – er sei schliesslich ein Seher. Die meisten hatte diese Aussagen allerdings als Humbug abgetan, als Ausreden eines Mannes, der es zu einer solchen Meisterschaft in einer Disziplin gebracht hatte, dass er schlicht verlernt hatte, zuzugeben, etwas nicht zu können.

Ebenjener Leander wandte sich soeben ab von Earas erneutem Versuch, die Tafel mithilfe von Heller Magie zu entschlüsseln – was für eine Närrin sie war, so etwas überhaupt mit Zauberei zu probieren. Mit dem trotzigen Auswurf „Ich weiss doch schon, ob es ihr gelingen wird. Bin ja ein Seher!“ entfernte er sich von der immer noch lautstark mit Eara miteifernden Menge. Leander hatte eine interessantere Präsenz in der Ecke des Raumes gespürt. Die Tavernengäste beachteten seinen Abgang nicht – offenbar würden sie ihn nicht vermissen. Innerlich grinste Leander aufgeregt. Er liebte es, den mürrischen Alten zu spielen. Und noch wichtiger: Das Schicksal hatte ihm soeben eine Möglichkeit geschenkt, die Rettung seines Bruders in Gange zu setzen.



Eara liess ihre Fingerspitzen mit geschlossenen Augen über die Oberfläche einer der eigenartigen Schriftstücke gleiten und wippte ihren Körper leicht vor- und zurück, während sie sich einen Zauberspruch der Erkenntnis und der Übersetzung in Erinnerung rief und gleich mehrmals hintereinander flüsterte.

Gilda beobachtete Earas Tun amüsiert – sie glaubte nicht, dass es beim fünften Versuch anders enden würde, aber Eara war ehrgeizig und nur sehr schwer von etwas abzubringen, wenn sie es sich einmal in den Kopf gesetzt hatte. Zudem lenkten ihre Versuche, das Schriftstück auf magische Art und Weise zu übersetzen, die Gäste vom tobenden Sturm ab, und was wäre Gilda für eine Tavernenbesitzerin, wenn sie sich nicht um die Freude ihrer Gäste gekümmert hätte?

Plötzlich wurde Gilda von hinten auf die Schulter getippt. Sie blickte sich um und war überrascht, Jarid dort zu erblicken – sie hatte die Danwarerin gar nicht hereinkommen sehen. Nun gut, wenn irgendjemand in diesem Unwetter umherreisen konnte, dann war es wohl eine Wassermagierin. Ihrer Anwesenheit nach befand sich Trieest wahrscheinlich auch schon in der guten Stube, und das hiess...

Ehe Jarid ihre Bitte überhaupt stellen konnte, hatte Gilda Zwei und Zwei zusammengezählt und fragte: „Ich hab‘ mehrere Fässer sauberen Wassers im Keller, wie viele braucht Trieest?“

Dankbarkeit blickte in Jarids Gesicht auf, als sie antwortete: „Mehr als eines ist wirklich nicht nötig. Und bemüh‘ dich nicht, ich hole es mir selbst gleich.“

Sie zögerte kurz und beugte sich noch näher zu Gilda, um leise zu wispern: „So sage mal, was macht Eara da?“

Gilda grinste und sprach: „Sie verzweifelt am Übersetzen einer unübersetzbaren Schiefertafel. Sobald es Trieest etwas besser geht, könntet ihr euch auch mal daran versuchen. Wer weiss, vielleicht ist das ja gerade die glorreiche Aufgabe, die seine ‚Wandlung‘ vollenden kann. Na?“

Jetzt war es an Jarid, kurz aufzulachen: „Das sind doch keine unübersetzbaren Tafeln, das ist kinderleicht! Solche benutzten wir schon während unserer Schulzeit in Danwar – du verstehst, in einer so feurigen Welt arbeitet man lieber nicht mit Schriftrollen aus Papier. Die Zeichen sind Danwarische Stichrunen, die kann ich besser lesen als eure andorischen Lettern! Schau mal, da im oberen Teil steht zum Beispiel ‚S.134: Jeder wackere Bauer vertilgt bequem zwei Pfund Gorfleisch.‘ Zugegeben, dieser Satz wirft mehr Fragen auf, als er beantwortet. Ich komme später gerne nochmals darauf zurück, für den Moment muss ich mich allerdings leider entschuldigen – Trieest leidet.“

Mit diesen Worten verzog sich Jarid in den Keller der Taverne und liess eine überraschte Gilda zurück. Eara hatte ihr Tun unterbrochen und starrte der Wassermagierin mit grossen Augen nach. Offenbar hatte sie nicht mit einer so einfachen Lösung für ein so lange ungeklärtes Rätsel gerechnet. Hoffentlich nahm sie es nicht persönlich, dass ihre Zauberei daran gescheitert war.

Als Eara sich gleich wieder der Tafel zuwandte und mit einem freudigen Lächeln auf dem Gesicht ein kleines Notizbüchlein und einen Stift hervorzog, wurde Gilda klar, dass die Zauberin das Ganze eindeutig nicht persönlich nahm. Jetzt wollte sie aber erst recht den auf den Tafeln stehenden Text entschlüsseln. Und wenn sie ehrlich mit sich war, war Gilda auch schon ziemlich gespannt darauf, was auf den Tafeln stand.








\newpage
\section{Endlich entschlüsselt}



„... und dieses Zeichen hier sollte ein ‚B’ darstellen.“

„Warte, das sieht eher wie ein ‚D’ aus.“

„Du hast recht! Also, dann lautet dieses Wort F – E – U – E – R – D – ?“

„Zwei Silberstücke auf ‚Ä’!“

„Drei Silberstücke für ‚R’!“

„Der nächste Buchstabe ist ein ‚R’!“

„Feuerdrache! Das Wort ist ‚Feuerdrache’!“\bigskip



Die andorischen Bauern und Händler in der Taverne zum Trunkenen Troll waren Feuer und Flamme für das Entschlüsseln der rätselhaften Tafel mit der danwarischen Schrift. Sie konnten auch nicht viel anderes tun, solange es draussen weiterhin zu heftig stürmte, um sicher nach Hause gelangen zu können. Jetzt, wo Jarid, die weise Wassermagierin aus Danwar, ihnen den entscheidenden Hinweis gegeben hatte, war das Dekodieren des Texts auch gar nicht kompliziert. Unterstützt durch mehr oder weniger hilfreiche Kommentare von Chada und Thorn, notierte Eara Buchstaben um Buchstaben in einem kleinen blau gebundenen Büchlein, welches sie stets bei sich trug. Gilda guckte ihr immer wieder mal über die Schulter und rief einen weiteren Buchstaben aus, die Spannung zwischen den Lettern natürlich so sehr in die Länge ziehend, dass die Tavernenbesucher auch genügend Zeit hatten, einen kleinen Teil ihres hart erarbeiteten Geldes darauf zu verwetten, welcher Buchstabe als nächstes kommen würde. Dabei würden sich die meisten von ihnen nicht gross für den Text interessieren, wenn er in andorischen Buchstaben geschrieben worden wäre. Aber die Stimmung war grossartig, und das war es, was Gilda zu erreichen erhoffte.\bigskip



Triumphierend verkündigte Gilda jetzt: „Die Überschrift lautet \textbf{Die Agren, der Schildzwerg und der Feuerdrache!“}

„Was ist das denn, etwa ein Märchen?“

„Die Form des Titels deutet auf ein Märchen hin. Aber die alleine hat noch nichts über den Text auszusagen.“

„Wer würde ein Märchen in danwarischen Runen in eine Steintafel ritzen und diese danach durch Gildas Fenster werfen?“

„Und vergiss nicht dieses verwaschene Bild nebendran. Soll das einen Hornbären darstellen?“

„Diese Angelegenheit schreit doch förmlich nach einem Spässchen der Waldgeister!“

„Ich muss zugeben, ich hatte mir etwas Besseres als eine Gutenachtgeschichte erhofft.“

„Ach Gord, so schweig stille! Du musst ja nicht zuhören, wenn du nicht magst.“

„Meine Mutter hat mir die Märchen jeweils nicht vorgelesen, sondern vorgesungen. Gilda, willst du das versuchen?“

„Ha! Da ist der gute Grenolin aber gar nicht glücklich, dass du nicht zuerst an ihn gedacht hast.“

„Was habt ihr denn? Laute spielen kann unser Greno doch prima!“

„Eara, brauchst du noch lange? Gilda, wie weit ist Eara schon?“

„Die Arbeit ist schon fast vollendet“, sprach Eara, ohne von ihrem Büchlein aufzublicken.

„Dieses Zeichen hier sollte doch ein ‚T’ sein“, warf Thorn ein. Eara funkelte ihn an, korrigierte den Buchstaben, kritzelte einen letzten entschlüsselten Satz in ihr Büchlein und übergab dieses dann breit lächelnd an Thorn: „Warum tust du uns nicht die Ehre, die Geschichte vorzutragen?“

Thorn zögerte zunächst, ergriff das Büchlein dann aber und verkündete:\bigskip



\textit{So vernehmet diesen Tatsachenbericht aus der Zeit des Unterirdischen Krieges, einer unsicheren, gesetzlosen Zeit. Dies ist die Legende von der Agren, dem Schildzwerg und dem Feuerdrachen.}\bigskip



„Zu schade, dass Kram in der Mine zu tun hat. Der mochte Überlieferungen aus der Vorzeit wie sonst keiner“, warf ein Schildzwerg ein.

„Lauter“, rief ein Bauer aus einer hinteren Reihe, und Thorn wiederholte mit etwas festerer Stimme:\bigskip



\textbf{\textit{Die Agren, der Schildzwerg und der Feuerdrache}}\bigskip



\textit{„DRACHEE!“}

\textit{Der verzweifelte Schrei hallte durch die Täler des Grauen Gebirges und schreckte allerlei Kleingetier auf, welches schleunigst das Weite suchte, selbst wenn es die Warnung nicht verstehen konnte. Die wenigen Agren, die sich ausserhalb ihrer sicheren Höhlen aufhielten, schreckten auf. Sie rannten allerdings nicht blindlings davon, sondern richteten ihre Augen furchtsam gen Himmel, blickten nach oben, suchten den kleinen schwarzen Punkt, welcher Feuer und Tod ankündigen würde.}

\textit{Gouri, eine vergleichsweise junge Agren (sie hatte doch erst kürzlich ihren vierzigsten Jahrestag gefeiert), versuchte ...}

[Siehe \hypref{Die Agren, der Schildzwerg und der Feuerdrache (2020)}.]

\textit{... einen Anflug kindlicher Freude. Sie würde diesen Zwerg zu den Hallen der Zwergenkönige begleiten! Sie würde auf ein Abenteuer gehen!}

\textit{Erst als die beiden die Alte Zwergenstrasse erreichten und eine weitere Höhle passierten, fiel Gouri ein, dass die anderen Agren wohl gerade umkamen vor Sorge um sie.}\bigskip







„Hier ist der Bericht zu Ende“, meinte Thorn erstaunt. „Doch die Geschichte scheint noch nicht so, als wäre sie zu Ende.“

Eara nickte: „Wenn das ein Tatsachenbericht darstellen soll, so wurde er bestimmt einige Zeit nach diesen Vorfällen niedergeschrieben, und da würde es nur sinnig sein, dass auch der Rest davon irgendwo notiert wurde. Aber wo?“

Gilda meldete sich zu Wort: „Vielleicht finde ich ja noch mehr Tafeln, wenn ich mehr Butterbrote backen würde und in den südlichen Wald brächte.“

„Sind wir sicher, dass die Tafeln von da kommen? Das sind Jarids Urteil nach eindeutig danwarische Runen.“

„Aber die Geschichte stammt aus dem Grauen Gebirge! Weiter entfernt von Danwar geht’s doch kaum!“

Die Helden sahen einander nachdenklich an.

Die anderen Tavernengäste tratschten indes munter weiter: „Ein Märchen war das ja wahrlich nicht. Diese Geschichte hat ja nicht mal irgendeine moralische Lektion dahinter!“


\newpage
\section{Reisetagebuch von Wrort, dem reisenden Temm}


\textit{Wenn ihr Lust habt, die Fortsetzung dieser Geschichte zu hören, so bin ich gerne bereit, sie zu erzählen – aber so unordentlich, wie ich bin, scheine ich die nächsten Tafeln schlicht und ergreifend verlegt zu haben! :oops:}

\textit{Das einzige, was ich noch in meiner Schriftensammlung auftreiben konnte, war ein Bericht von Wrort, dem reisenden Temm. Vielleicht kann dieser ja bei der Suche helfen?}\bigskip



Da stand ich nun also endlich in Andor, dem legendären Drachenland! Ich hatte eine Höhle erreicht, in der ein netter Fährtenleser mit seinem sprechenden Raben lebte, und teilte eine einfache Mahlzeit mit ihm (grösstenteils Apfelnüsse, diese wachsen in diesem Reich einfach überall). Er erzählte von allerlei Geschichten aus diesen Landen jenseits des Kuolema und fachte damit meinen Entdeckergeist nur noch mehr an. So brach ich am nächsten Morgen nach einer grossen Mütze tiefen Schlafs freudig in Richtung der von König Brandur und seiner Schar erbauten Burg auf. Dieses architektonische Wunderwerk wollte ich mir genauer ansehen. Ich hielt mich auf meinem Weg \textbf{so nahe wie möglich} am Kuolema, da ich mich nahe der Berge sicherer fühlte – die Kreaturen, von denen es in Andor den Gerüchten nach nur so wimmeln sollte, kamen ja bekanntlich vor allem aus dem Osten.

Schon nach zwei Stunden stand ich vor dem majestätischen Gebäude.

Ich notierte mir meine momentane Position.\bigskip



König Brandur war ein gütiger Mann, der den Worten eines Fremden rasch Glauben schenkte, und so gewährte er mir Einlass in seine Burg und liess mich sogar von einem Wachmann herumführen. Dieser fragte mich, aus welchen Landen ich den stammte, und ich berichtete ihm von Tulgor. Seine Augen leuchteten, als er von unseren grossen Kränen und Baumaschinen hörte. Dann berichtete er, dass am anderen Ufer der Narne, im Wachsamen Walde, der Orden der „Bewahrer vom Baum der Lieder“ herrschte, der meine Berichte über Tulgor liebend gerne in sein Archiv aufnehmen würde. In den Worten des Wachmanns: „Diesen eitlen Pinselnschwingern läuft doch schon doppelt und dreifach das Wasser im Munde zusammen, wenn sie nur hören, dass hinter dem Fahlen Gebirge etwas anderes als eine leere Einöde liegt.“

Ich versprach dem Wachmann, es mir zu überlegen, und brach wieder auf. Die Rietburg verliess ich durch das südliche Tor. So reiste ich fünf Stunden lang, vom ewigen Feuer vor der Burg aus am düsteren Krähenstamm vorbei bis zum Freien Markt der Händler der Andori und schliesslich zu einer kleinen, hell erleuchteten Taverne, wo ich mir mit einigen Silbermünzen eine Übernachtungsgelegenheit ersteigern konnte. Ich plauderte mit einigen Tavernengästen und auch sie rieten mir grösstenteils, zum Wachsamen Walde weiterzureisen. Doch vorher wollte ich noch etwas mehr vom Rietland sehen und das Leben der hiesigen Bauern hautnah miterleben.

Das östliche Rietland hatte ich noch gar nicht besucht, darum heuerte ich zwei Andori an, mich dorthin zu begleiten und mir auf dem Weg die Dunklen Kreaturen vom Leibe zu halten. Es waren beide einfache, aber gute und tapfere Menschen, und ich weiss nicht, wie weit ich ohne sie gekommen wäre. Kaum hatten wir die nächstgelegene Brücke überquert, so wurden wir auch gleich von einem waschechten Troll angegriffen! Ich hatte noch die in meinem Leben so ein furchteinflössendes Wesen gesehen. Die beiden Andori hielten mir das Ungetüm vom Leibe, während ich zum angrenzenden Bauerngehöft floh. Dort wurden ich und die beiden Andori gut versorgt. Wir unterstützen den Bauern als Gegenleistung bei ihrer Arbeit, so gut wir konnten (was in meinem Fall nicht sonderlich gut war, aber was soll’s). Nach dem Abendessen – leckere andorische Butterbrote – trugen mich meine kleinen Beinchen noch gut eine Stunde lang eine kleine Strecke der Narne entlang gen Süden. Dort setzte ich mich an eine Böschung und meditierte in der Stille der Nacht.

Ich notierte mir meine momentane Position.\bigskip



Beim nächsten Sonnenaufgang fanden mich die beiden Andori frierend an diesem Ort, und ich musste den beiden versprechen, nie mehr einfach kommentarlos zu verschwinden. Sie hatten sich wohl Sorgen um mich gemacht. Wie nett.

Wir drei nutzten an jenem Tage die 7 sonnigen Stunden des Tages zu ihrer Völle aus und reisten zum Wachsamen Wald und sogar bis in den Wachsamen Wald hinein, auf den sagenumwobenen Baum der Lieder zu. Auf unserem Weg überquerten wir eine auf Pfählen gebaute Brücke, die sogenannte Bogenbrücke. Davon würde der Hohe Architekt Tulgors bestimmt gerne hören.

Ich war froh, als wir die nebligen Gebieten nahe der Narne und des Likko verliessen, da sich darin jederzeit weitere Dunkle Kreaturen verstecken konnten.

Ganz bis zum legendären Baum der Lieder schafften wir es an diesem Tag nicht mehr, aber ich konnte den mächtigen Stamm durch das Fernrohr, welches einer meiner beiden Begleiter stets bei sich trug, bereits ganz gut erblicken, als wir nach 7 Stunden Wanderung auf die ersten Bewahrer vom Baum der Lieder stiessen.

Diese Bewahrer waren misstrauischer als die Bewohner der Rietburg und konnten sich nur langsam davon überzeugen lassen, ihre Bögen zu senken. Den beiden Andori vertrauten sie natürlich, aber mit mir und meinen magischen Kräften, die ich ihnen demonstrierte, war es etwas anders. Offenbar waren Zaubertricks in diesen Landen nicht mehr so gut angesehen, seitdem ein gewisser Dunkler Magier hier sein Unwesen getrieben hatte. Gut zu wissen.

Schlussendlich zeigten die Bewahrer uns aber trotzdem den kleinen Lagerplatz, den sie an dieser Stelle gebaut hatten.

Ich notierte mir meine momentane Position.\bigskip



Dann legten wir, das heisst ich, die beiden angeheuerten Andori und die beiden Bewahrer, die uns gefunden hatten, uns schlafen.

Am nächsten Tag dann trafen wir endlich beim Baum der Lieder ein. Der Oberste Priester, Melkart war sein Name, war gekleidet in ein edles weissen Gewand und tatsächlich komplett aus dem Häuschen, von einer neuen Welt hinter dem Kuolema zu hören. Er quetschte mich bei allerlei gutem Essen über die Geographie, Kultur und dergleichen meiner Heimat aus. Die beiden Andori kehrten bald darauf ins westliche Rietland zurück und die beiden Bewahrer begaben sich wieder auf ihren Wachposten

Und ich, ich werde nach einigen Tagen ergiebiger Gespräche mit Melkart und den seinen fröhlich weiterreisen. Vielleicht kehre ich später einmal zurück, um von meinen weiteren Erlebnissen zu berichten.




\newpage
\section{Ein danwarischer Epilog}

In einem gemütlichen Zimmer der Taverne hatte Trieest sich auf dem Bett breitgemacht, während Jarid neben ihm sass, immer wieder mal ihren Arm schwenkte und damit eine neue Schicht frischen Wassers über den Lavastein zog. Langsam beruhigte sich Trieests Brust. Jarids Behandlung tat ihm gut. Seine Aufgebrachtheit schwellte mit der aufsteigenden Müdigkeit ab. Das war ein anstrengendes Ende eines anstrengenden Tags gewesen. Jarid hatte vor der Verbannung einen Orakelspruch erhalten, das war die einzige neue Information, die wirklich zählte. Wie lange hätte sie es noch vor ihm verheimlichen wollen? Was war ihr nur gesagt worden, das sie selbst in Gedanken daran so sehr durcheinanderbringen konnte? Natürlich wusste Trieest, dass sie nicht darüber sprechen sollte, ja, nicht darüber sprechen durfte, dass etwas Schreckliches eintreten würde, wenn sie zuviel verriet. Aber ein kleiner Schubs in die richtige Richtung würde sicherlich nicht schaden können. Er wagte sich vor:

„Wie lange wird die Bürde noch mein sein?“

Stille.

„Ist die letzte Prüfung schwer zu meistern?“

Stille.

„Wenn du entscheiden dürftest, wohin wir als nächstes reisen, was würdest du vorschlagen?“

Stille.

„Jarid, so sprich doch!“

Stille.

„Weisst du, wir könnten dennoch nach Narkon aufbrechen.“

„Ich will nichts tun, wozu dich ein lügnerischer Zukunftswisser überreden wollte. Und still jetzt, dein Körper muss genesen.“

„Ich fühle mich schon viel besser, dir zum Dank – nein, bitte nicht aufhören! Ich meine nur: Wenn dieser Blinde wirklich ein Seher war, so werden wir letztendlich ohnehin das tun, zu was er uns bringen wollte. So könnte er etwa deine Reaktion vorhergesehen und den wütenden Abgang gespielt haben, damit wir gerade nicht nach Narkon gehen, obwohl wir das eigentlich sollten.“

„Kein Seher verfügt über derartige Allwissenheit. Das wäre zu viel für einen menschlichen Geist.“

„Nicht mal der grantige Kord?“

„Insbesondere nicht der grantige Kord.“

Stille.

„Wohin willst du als nächstes ziehen, Tri?“

„Könnten wir uns mit Merrik absprechen? Vielleicht gibt es noch ein uns unbekanntes Reich, welches Kartographierung bedürfen könnte.“

„Gerüchten zufolge ist Merrik leider kürzlich in den Norden geschifft.“

„Etwa nach Narkon??“

Seufzer.

„Ich bin ja schon still.“

Stille.

„Du, Jarid?“

„Hm?“

„Nach den Berichten der Temm, die ich vernommen habe, klingt Tulgor ziemlich spannend.“

„Dann wird also Tulgor unser nächstes Ziel. Ich werde morgen zum Baum der Lieder springen, um mich von Melkart bei der genauen Wahl des Reisepfads beraten zu lassen.“

„Wundervoll!“

Stille.

„Das kommt schon, Tri. Bleib tapfer.“

„Danke. Du musst den Orakelspruch nicht für dich behalten, das weisst du. Eara könnte dir sicherlich beim Interpretieren helfen. Oder die alte Reka. Die können so was.“

„Ich weiss. Danke.“

Stille.

„Gelb ist die Sonne in der Senke.“ Gute Nacht.

„Und Gelb sind die Blumen in der Blüte.“ Danke gleichfalls.

\begin{center}
    Weiter geht es in \hypref{Der Feuerkrieger und die Wassermagierin (2023)}.
\end{center}


























\begin{chapterbox}
    \chapter{Der Feuerkrieger und die Wassermagierin (2023)}
    \label{Der Feuerkrieger und die Wassermagierin (2023)}
    \az{Jahr 68}

    \begin{center}
        Fortsetzung von \hypref{Stürmische Rätselnacht in der Taverne (2019 bis 2020)}
    \end{center}
    
    Ein weiteres Abenteuer von Jarid und Trieest aus dem fernen Danwar, vollgepackt mit sinistren Steinen, starken Skralen, Echos der Toten, seltsamen Alchemisten, klangvollen Wasserbecken, weisen Lehrpersonen, und vielleicht gar dem einen oder anderen altbekannten Helden von Andor, der lieber nicht im Rampenlicht steht.
\end{chapterbox}


\section{Der wilde Hraak}

\az{Jahr 61}

\textit{Halle des Ältestenrats, 61 a.Z.}\bigskip



„Nächste Aussagende. Vortreten, bitte. Name?“

„Jetzt tut doch nicht so, Älteste Freiga. Ihr wisst ganz genau, wer ich bin.“

„Protokoll ist Protokoll!“

„Und das gilt insbesondere für dich, Jarid! Sprich deinen Namen.“

„Mutter, du auch?!“

„Jetzt unterlasse bitte schön diese Spielchen. Dass du meine Tochter bist, verleiht dir keine Privilegien. Wenn du angehört werden willst, so halte dich gefälligst an die Gepflogenheiten des Ältestenrats. Mach’s dir nicht schwieriger, als es ohnehin schon ist.“

„Also gut, wie ihr wollt. Mein Name ist Jarid Morgentau, Wassermagierin des dritten Zirkels. Was für eine Überraschung.“

Gekritzel, Pergamentgeraschel.

„Jarid Morgentau. Sind Sie hier, um den Täter zu verteidigen?“

„Ich bin hier, um an die Vernunft des Ältestenrates zu appellieren. Bei den Wogen der Wellen, Trieest ist doch bloß noch ein Kind! Wenn er ...“

„Ruhe! Jetzt ich noch nicht die Zeit für Ihre Rede. Zunächst müssen wir die Formalitäten hinter uns bringen. Halten Sie sich für fähig, ein neutrales Urteil über den Vorfall abzugeben?“

„Natürlich nicht, keiner hier ist das! Aber ich habe Trieest seit klein auf begleitet, ich kann dafür bürgen, dass er sich üblicherweise ganz anders verhält. Ich hatte einige Tage zuvor bereits ein Gespräch mit seiner Lehrerin Ganwa über seine Stimmung gesucht und ich war die erste Wassermagierin, die am Unfallort eingetroffen ...“

„Völliger Schwachsinn! Einen ‚Unfall‘ willst du das nennen?!“

„Die Gepflogenheiten gelten nicht nur für meine Tochter, sondern auch für dich, Älteste Freiga! Rede mir nicht drein. Denk an das Protokoll.“

„Ich steck dir dieses Protokoll gleich sonst wo hin, Rowinda, mein Enkel hat sein Augenlicht verloren wegen dieses Ungeheuers!“\bigskip







\az{Jahr 68}

\textit{Sieben Jahre später.}\bigskip



Der wilde Hraak stieß ein ohrenbetäubendes letztes Brüllen aus, in einer Tonhöhe, die für Menschen knapp unhörbar, für Trieest jedoch nur allzu gut wahrnehmbar war. Noch vor wenigen Minuten hätte das Gebrüll des Hraaks ungemeine Furcht ausgelöst in all jenen, die es zu hören vermochten. Doch dem war nicht mehr so. Stattdessen schien die Furcht nun im Geheul des finsteren Wesens zu liegen. Falls der Hraak überhaupt dazu imstande war, Furcht zu empfinden.

Trieest grunzte zufrieden und stach mit seinem gewundenen Rankenschwert tiefer am Geweih des stinkenden Ungetüms vorbei in den Schädel des Biests. Seine Klinge sog gierig die Dunkle Hexerei auf, die den verwesten Körper in Bewegung hielt. Trieest befahl den Ranken seines Schwerts, den Ursprung des Übels zu finden. Er fühlte, wie sich die Klinge unter seiner geistigen Führung in viele kleine Stränge teilte, durch Fleisch und Knochen stach und schließlich die steinerne Quelle des Bösen fand, welcher hinter der Stirn des Hraaks saß. Triumphierend ließ Trieest die Ranken des Schwerts um den Kristall winden und zog einmal kräftig an seinem Schwert.

Ein melodisches Klingen erklang. Der Hraak bäumte sich ein letztes Mal auf und stürzte dann zu Boden, als Trieest sein Schwert an sich riss. Die Spitzen der Rankenklingen umfassten einen rötlich schimmernden Kristall. Trieest vermochte nicht immer, Farben zu unterscheiden, doch diese strahlende Kristallfarbe war unverwechselbar.

Trieest blieb einige Lidschläge lang angespannt, doch der Körper des Hraaks lag verrenkt im Matsch und regte sich nicht mehr. Die lange Zunge lag im Dreck, die Schlitzaugen waren geschlossen. Das Wesen war besiegt.

Der Kampfrausch verflog so schnell, wie er gekommen war. Trieest sank zusammen und ließ sein Rankenschwert fallen, dessen blutige Spitzen weiterhin den roten Kristall fest umschlossen. Die zwei dünnen Ringe aus orangeroten Feuerschlieren, die Trieest im Kampf umgeben hatten, erloschen. Seine ausgebrannte Brust schmerzte.

Dann erinnerte er sich und sein Kopf schoss erschrocken wieder in die Höhe: „Jarid!“

Hektisch sah er sich um. Die Wassermagierin lag bewegungslos am Rande der Waldlichtung, ihr linker Arm in einem unnatürlichen Winkel abgespreizt. Trieest stemmte sich hoch und humpelte zu ihr hinüber. Erleichtert erkannte er, dass ihre Brust sich hob und senkte, wenn auch unregelmäßig.

Erinnere dich, du tumber Troll!, fluchte er. Was hatte Jarid nur schon wieder über ohnmächtige Personen gesagt? Auf den Rücken drehen? Auf die Seite? Ja, die Seite war es gewesen!

Vorsichtig fasste Trieest die bewusstlose Jarid an den Schultern und drehte sie auf die Seite. Ihr verletzter Arm streifte den Waldboden unsanft. Jarid zog scharf Luft ein, riss ihre Augen auf und murmelte etwas Unverständliches.

„Jarid?“, fragte Trieest ängstlich, doch die Wassermagierin hatte ihre Augen bereits wieder geschlossen und ihre Hand fiel schlaff zu Boden. Erst jetzt roch Trieest aktiv den metallischen Gestank menschlichen Blutes in der Luft und sein Blick richtete sich auf die üblen Kratzspuren in Jarids Bauchgegend.

Sporndreck! Das Biest hatte sie erwischt! Was nun? Denk, tu törichter Arrog, denk!

Die Wunde musste gereinigt und danach geschlossen werden. Aber frisches Wasser besaßen sie keines mehr und Verbandsmaterial hatten sie gar nicht erst dabeigehabt.

Ratlos riss Trieest einen Fetzen seines roten Gewands ab und presste diesen auf Jarids verwundeten Bauch. Jarid stöhnte schwach auf, woraufhin Trieest gleich wieder von ihr abließ. Wenn er auf die Wunde drückte, würde nur noch mehr Blut herausquellen. Das war schlecht ... oder? Irgendwie musste der Blutfluss doch gestoppt werden, aber draufdrücken konnte nicht gut sein? Verflucht noch mal! Jarid wüsste, was zu tun ist. Oder Reka. Irgendein Hexer. Das andorische Dorf hinter der nächsten Hügelkette, welches so lange vom Hraak terrorisiert worden war, hatte einen guten Heiler. Doch würde Trieest die Strecke nicht schnell genug hinter sich bringen können, erst recht nicht in diesem Zustand.

Wütend schlug Trieest mit der Faust auf den feuchten Waldboden. Das brachte natürlich nicht viel, außer dass einige Apfelnüsse von einem nahegelegenen Baum purzelten. Bäume! Pflanzen! Vielleicht wuchsen hier irgendwo Heilkräuter? Wie sah Wolfskraut schon wieder aus? Und diese Pilze dort drüben ... die erkannte Trieest an der lustigen Form, das waren Zauberhutpilze! Waren die nun schon wieder giftig oder wundheilend? Haariger Höhlenwicht, warum hatte er nicht besser aufgepasst, als ihm solche Dinge in der Ausbildung zum Feuerkrieger gezeigt worden waren?!

Er durfte es nicht riskieren, Jarid die falschen Kräuter einzuflößen. Aber irgendetwas musste Trieest doch tun können! Unsicher griff Trieest nach Jarids kalter Hand (derjenigen, deren Arm nicht in einem unpassenden Winkel abstand) und führte sie zu seiner Brust. Dort, tief in seinem Körper versenkt, pochte der orange-rote Lavastein, welcher ihm aufgebürdet worden war. Sieben Jahre lang hatte er diesen Stein nun schon getragen. Dieser Stein hatte ihm bereits unzählige Male das Leben gerettet und ebenso viele Male das Leben zur Hölle gemacht. Doch in diesem Moment wirkte der Stein ... geradezu euphorisch?

Es war Trieest zu Beginn seiner Zeit als Lavastein-Träger falsch vorgekommen, einem aus dem Felsen Danwars geschlagenen Stück Stein Gefühle zuzuschreiben, doch inzwischen konnte er schon lange nicht mehr leugnen, dass er immer wieder Eindrücke und Emotionen verspürte, die nicht die seinen waren – diese Stimmungen mussten vom Lavastein kommen. Das siegreiche Schlachten des Hraaks und das Anzapfen dessen dunkler Energie hatten den Lavastein offenbar glücklich gestimmt. Vielleicht gab er sich gnädig. Trieest konnte nur hoffen.

Er zögerte noch kurz, doch dann ließ er sich darauf ein und drückte Jarids Hand auf den Edelstein. Etwas zischte. Der sonst oft kühle Stein war unnatürlich heiß, ein weiteres Zeichen seiner Erregung. Bitte, flüsterte Trieest innerlich, ich weiß, dass du mich nicht magst, doch Jarid hat dir und den Danwaren nichts getan, Jarid hat mich so oft davon abgehalten, dich aus mir rauszureißen, und sie wird es wieder tun. Beim Rauschen der Wellen und beim Knistern des Feuers. Beim Barte des Warx! Ich schwöre, mich nicht mehr gegen dich zu stellen.

„Rette sie! Bitte!“

Trieest hatte kaum Zeit zu realisieren, dass er die letzten Worte laut geschrien hatte, da setzte bereits das wohlbekannte Ziehen in seiner Brust ein. Feuer füllte seine Innereien, schoss durch seine Adern und bohrte sich in seinen Kopf. Trieest unterdrückte einen Aufschrei und biss sich auf die Zunge, als sein gesamter Körper sich verkrampfte. Doch ließ er Jarids Hand nicht los, drückte sie nur noch fester an sich und an den brennenden Stein, bis er hörte, nein, spürte, wie ihre Knochen unter seinem Griff knacksten.

Trieests Körper stand in Flammen. Nicht buchstäblich, nein, aber so fühlte es sich zumindest an. Sein Sichtfeld verging in der Farbe des Feuers und des Rauchs, seine Kehle wurde staubtrocken und jeder Atemzug fühlte sich an, als stünde er wieder an der Stätte der heiligen Flammen im Land der drei Brüder oder inmitten eines riesigen Sandsturms in der tulgorischen Wüste. Und während Trieest ächzte und haderte, leuchtete der Lavastein auf, heller und heißer als je zuvor, und verbrannte Jarids Hand.

Hätte Trieest noch sehen können, so hätte er bemerkt, wie Jarids Wunde sich in Minutenschnelle schloss, bis nur noch eine feine Narbe unter einem Riss im Stoff ihres Gewands zu sehen war. Doch da Trieest nichts mehr sehen konnte, verblieb er in seiner Agonie, die einzigen Gedanken, zu denen er noch fähig war, waren, dass er nicht loslassen durfte, dass er Jarid halten musste, zerbrechlich, wie sie war, denn wenn sie nicht mehr wäre ... sie durfte nicht nicht mehr sein ...

Wie lange Trieest in diesem Zustand lag, konnte er nicht sagen. Aber plötzlich war sie wieder da, Jarid, sie war hier, und wach, und gesund, und sie löste sich aus seinem Griff. Doch der Schmerz und das Brennen des Lavasteins in seiner Brust blieb, und Trieest griff unverständig nach Jarid, warum wollte sie gehen, sie durfte nicht gehen, er brauchte sie doch.

Kühles, glänzendes Nass ergoss sich über ihn, als Jarid mit einigen Gesten ihrer Finger Tautropfen vom sie umgebenden Gras auf Trieests Stirn leitete. Es waren Tropfen auf einem heißen Stein, und doch war diese Linderung zunächst alles, was Trieest brauchte, um seinen Verstand etwas zu klären.

Es hatte gewirkt! Der Stein hatte Jarid geheilt! In seinem schmerzerfüllten Zustand konnte Trieest sich nicht leicht darüber freuen, aber eine frohe Nachricht war es dennoch allemal. Und nun war das kühle Nass nicht nur auf seiner Stirn. Überall an seinem Körper ergossen sich kühle Blasen kalten Wassers, befeuchteten aufgesprungene Haut und spülten frische Schrammen aus vom Kampf mit dem Hraak aus.

„Tri, sag mal. Wie kommt es eigentlich dazu, dass, selbst wenn mein Leben von dir gerettet wurde, am Ende dennoch du derjenige bist, der dringend Pflege nötig hat?“, fragte Jarid schmunzelnd. Dann musterte sie fasziniert ihre versengte Hand. „Der Lavastein hat ja ganze Arbeit geleistet, selbst meine Schulter scheint er wieder eingerenkt zu haben. Ich wusste gar nicht, dass er das kann.“

„Ich ... ich auch nicht“, hustete Trieest vom Waldboden her. Aufstehen schien gerade nicht drin zu liegen. „Ich konnte es nur hoffen. Und wusste bloß, dass es nötig war.“

Jarid sagte nichts, doch Trieest konnte sehen, wie es in ihrem Kopf ratterte. Dann wechselte sie das Thema und sah Trieest aus zusammengekniffenen Augen an.

„Wie fühlst du dich, wackerer Krieger?“

„Wie ein Feld aus Rietgrasblüten, nachdem eine Horde Trolle darüber getanzt hat. Aber das ist nicht das ...“

„... was ich meine, genau.“

„Nun ja, ich fühle mich ... nicht anders als vorher. Nur mit mehr blauen Flecken.“

Jarid ließ ihre Schultern hängen und in ihren Augen blitzte Enttäuschung auf, woraufhin sie sich etwas abwandte. Einen Augenblick später drehte sich zurück, ein erzwungenes Lächeln auf den Lippen: „Nun denn, dann wollen wir uns mal daran machen, deine blauen Flecken zu heilen. Und dann zurück zum Dorf. Dorfälteste Ronder wird überaus erfreut sein, wenn wir ihr den Kopf des Hraaks präsentieren können.“

„Warte noch einen kurzen Augenblick“, bat Trieest und hob seinen Arm. Jarid legte ihren Kopf schräg und half ihm dann auf.

Trieest stützte sich an einen Baumstamm und atmete tief durch. Dann setzte er an und seine tiefe Stimme stockte einige Male, während er die richtigen Worte suchte:

„Das Überwinden des Hraaks war offensichtlich nicht die Aufgabe, deren Erfüllung meine Verwandlung vollziehen wird. Aber ...“

Erst jetzt löste sich der Knoten der Furcht in Trieests Brust, dass Jarid nicht überleben würde, und die Enttäuschung überkam ihn mit voller Wucht. Eine derartige Herausforderung wie den Hraak hatten sie seit Monaten nicht mehr erlebt. Wenn selbst die nicht genug war ... wie lange würde er ... würden sie beide noch mit dem Stein in seiner Brust hadern müssen?!

Damals, als er einer seltsam klaren Vision nach Andor gefolgt war, um den Andori gegen die Ewige Kälte auszuhelfen und einen Winterstein aus den Klauen einer eisigen Kreatur zu entreißen, war er sich sicher gewesen, dass dies die Aufgabe war, die ihn von seiner Bürde befreien würde. Ein Feuerkrieger im Kampf gegen den Geist einer Eisdämonin, das passte doch zu gut. Doch daraus war nichts geworden.

Inzwischen wusste dank der alten Gerüchteköchin Allanta bestimmt jeder halbwegs informierte Bauer diesseits des Fahlen Gebirges von Trieests Bürde. Dass er irgendeine großartige glorreiche Aufgabe erfüllen musste, um seine Wandlung zu vollziehen. Auch wenn die wenigsten ahnten, was diese Wandlung und sein Lavastein für ihn bedeuteten. Wann immer er und Jarid in der Taverne zum Trunkenen Troll einkehrten, hatte die Wirtin Gilda oder ein fröhlicher Tavernengast wieder ein neues Rätsel oder eine neue Aufgabe für ihn bereit, in der Hoffnung, dass ihm dies helfen würde. Und das schlimmste daran war, dass Trieest nicht einmal mit Sicherheit sagen konnte, dass seine glorreiche Aufgabe am Ende mehr als ein halbwegs herausforderndes Kreuzworträtsel würde. Warum mussten die Stimmen der Roten Grotte auch nur so verschnörkelte Prophezeiungen geben?!

Die Rote Grotte. Trieest grummelte schon alleine beim Gedanken daran. Zu schicksalshaften Zeiten schickte der danwarische Ältestenrat Danware zur Roten Grotte, um den Stimmen der Toten lauschen. Trieest verband keine positiven Gefühle damit. In der Roten Grotte waren die Echos unzähliger Verstorbener gefangen, und wer sich zu lange darin aufhielt, drohte, verrückt zu werden. Trieest hatte selbst im endlosen Gemurmel und Geschrei nur das Rauschen des Meeres vernommen, keine einzige sinnvolle Aussage. Die Ältesten waren zwar sehr enttäuscht gewesen, als er ihnen das mitgeteilt hatte, hatten dann aber mit den Schultern gezuckt und sich nicht weiter darum gekümmert. Vielleicht konnten ja nur „wahre“ Danware die Stimmen ihrer Vorfahren hören? Trieest, versuchte, die Erinnerungen abzuschütteln, doch diese waren hartnäckig.

Jarid musterte Trieest besorgt, als er nicht weitersprach. Ihre Augen wanderten argwöhnisch seinen Körper entlang. Trieest wusste, dass sie nach größeren Wunden suchte, inneren Blutungen und dergleichen. Sie würde nichts finden, nichts außer diesem verfluchten Stück Feuer in seiner Brust!

Trieest spürte, wie der Stein sich über seinen Missmut ärgerte und gefährlich heiß wurde. Er versuchte, das Gefühl zu verdrängen und sammelte sich.

„Ich bin enttäuscht darüber, dass ich weiter diese Bürde tragen muss, und das ist absolut verständlich. Aber wie kommt es, dass du enttäuscht bist, Jarid? Solltest du nicht bereits längst wissen, wann ich meine Bürde ...“

Schnell brach Trieest den Satz wieder ab, als Jarid das Gesicht verzog und sich brüsk abwandte. Seit jener stürmischen Nacht im Trunkenen Troll, in welcher ein blinder Seher ihm eine falsche Prophezeiung hatte andrehen wollen, hatte Trieest es nicht mehr gewagt, Jarids Orakelspruch aus der Roten Grotte anzusprechen. Bis heute, hier und jetzt. Trieest sammelte seine Frustration und sprach weiter:

„Warum bist du enttäuscht? Du sagtest, du wüsstest, wo mein Prozess des Wandels beendet werden wird. Solltest du nicht bereits wissen, wo meine letzte Aufgabe stattfindet? Wusstest du nicht schon seit langem, dass der Hraak uns nicht weiterbringen wird?!“

Das Gesicht immer noch abgewandt, entgegnete Jarid scharf: „Der Hraak mag uns persönlich nicht weitergebracht haben, aber seine Abwesenheit bedeutet das Überleben unzähliger Dorfbewohner! Sind solche Heldentaten denn nichts wert?“

„Natürlich sind sie das wert! Aber nicht in Bezug auf den elenden Lavastein vor meinem Herzen!“, brauste Trieest auf. Dann fing er sich wieder und fuhr fast flehend fort: „Du musst nicht enttäuscht tun, Jarid, du musst doch gewusst haben, des der Hraak uns nicht persönlich weiterbringt, und das ist auch nicht so wichtig im Bezug aufs große Ganze. Kannst du mich in Zukunft bitte einfach nicht mehr einer falschen Hoffnung erliegen lassen? Ich weiß nicht, wie viel ich noch zu verkraften vermag. Du magst eine tapfere Heldin sein, Jarid, aber ich bin keiner. Solche wie mich gibt’s doch wie Sand am Meer. Und ich will nicht wie einer dieser hoffnungslosen Glutträger enden!“

Jarid blieb stumm und trat einige Schritte in Richtung des Hraaks. Dessen Kadaver lag immer noch still in der Mitte der Waldlichtung und rührte sich nicht.

Die ersten Sonnenstrahlen des neuen Tages durchbrachen die östlichen Baumwipfel und erhellten den Waldboden, fingen sich im kleinen roten Kristall, welcher in Klingenranken verschlungen auf der Spitze von Trieests Schwert steckte. Das Schwert lag immer noch achtlos weggeworfen am Boden, als Jarid vor es trat. Sie bückte sich und hob die Waffe hoch, drehte sie leicht im Morgenlicht, als würde sie die Spiegelungen betrachten. Trieest glaubte allerdings zu wissen, dass Jarids Gedanken nicht beim Schwert lagen. Schwer atmend wartete er auf eine Antwort.

Endlich durchbrach Jarid die Stille. Leise bat sie: „Lass deinen Ärger bitte nicht an mir aus.“

„Natürlich nicht“, erwiderte Trieest und trat näher, „Dennoch hast du ...“

„Mir gefällt es auch nicht!“, unterbrach Jarid ihn nun, „Nicht im Geringsten! Wir sitzen hier im selben Boot. Ich hoffe genauso wie du, dass jede Herausforderung die letzte wäre. Ich will dich frei und glücklich sehen, nicht leidend!“

Jetzt war es an Trieest, nach Worten zu stammeln. Natürlich hatte sie recht, natürlich wollte sie ihn nicht leiden sehen, aber das war es nicht. Sie wusste etwas, was er nicht wusste, so viel wusste er, und doch wusste er nicht, welche Sorte von Wissen Jarid in der Roten Grotte erhalten hatte. Jarids Aussagen dazu konnte er nicht richtig deuten, und das machte ihn wahnsinnig. Nein, eigentlich war es sein Schicksal, das ihn wahnsinnig machte. Aber das änderte nicht daran, dass er mehr wissen wollte über seine Zukunft.

„Seit sieben Jahren arbeiten wir uns nun schon durch Quest nach Quest, wann ist es denn endlich vorbei?“

„Sei nicht ungeduldig, Tri. Wir haben vielen Leuten geholfen. Und den Titel ‚Held von Andor‘ erhält man nicht leichtfertig.“

„Ach, jetzt hör schon auf! War das nicht bloß eine politische Entscheidung? Wie kommt es, dass wir Danware beide den Titel verliehen bekommen haben, nicht aber dieser Brolaf, der lautstarke Herold der Schildzwerge, oder dieser famose Bogenschütze, der das Auge des Drachen getroffen hatte? Der, der fast so wie ich hieß? Weest? Feest? Egal. Nein, dieser Titel, ‚Held von Andor‘, der verleiht Anerkennung, und Macht, und so vergibt ihn König Thorald nur an diejenigen, die nicht mit ihm streiten mögen. Mit wahrem Heldentum hat das nicht viel zu tun, sondern nur damit, dass wir Danware von weit her sind statt politische Rivalen des Königsreichs.“

Jarid blickte Trieest nur stumm an, bis dieser schließlich zu reden aufhörte und leise brummelte: „Ist ja gut, ich bin wieder zu zynisch, ich meins ja nicht so. Und zugegebenermaßen wurde auch schon diese Flussländlerin mit dem danwarischen Stab zur Heldin ernannt, obwohl das die potentielle Unabhängigkeit der Flusslande vom Königreich etwas wahrscheinlicher macht. Vielleicht steckt hinter dem Titel doch mehr. Und wenn ich meine Bürde erst einmal los bin, helfe ich gerne weiter als Held aus. Aber weil ich will, nicht, weil ich muss!“

Trieest atmete tief durch und fuhr fort: „Und habe ich nicht ein Recht darauf, meine Bürde abzulegen! Sieben Jahre! Sieben Jahre sind das nun schon! Wie lange will Mutter Natur mich denn noch quälen?! Warum kannst du mir nicht verraten, was die Echos der Toten dir in der Roten Grotte sagten? Wie soll mir das denn helfen, meinen Prozess des Wandels zu beenden?!“

Stille. Natürlich Stille. Inzwischen schrie Trieest auch nicht mehr Jarid an, sondern vielmehr die ganze Welt, die sich seinem Prozess des Wandels in den Weg zu stellen schien.

Das war natürlich nicht wahr. Nicht die ganze Welt hatte sich gegen ihn verschworen. Vermutlich nicht einmal sein Schicksal. Er war bloß enttäuscht, und das war in Ordnung, das war verständlich, aber das gab ihm nicht das Recht, ausfällig zu werden.

Trieest schnaubte ein letztes Mal, faltete seine Hände und atmete tief durch. Er sog alle Gerüche der Umgebung ein. Der Schweiß auf seinem Körper, Jarids Blut, der Duft des spätherbstlichen Waldes, selbst der Gestank des Kadavers des Hraaks. Trieest hörte die Vögel zwitschern. Einige Bäume weiter drüben huschte ein Streifenmarder quietschend durchs Geäst. Der Feuerkrieger nahm alles wahr, war sich allem bewusst, und versuchte, es einfach anzunehmen. Seine Emotionen fließen zu lassen, wie er es schon so oft versucht hatte.

„Weiß sind die Seemöwen nicht nur im Winter“, sprach er die leise Floskel der Entschuldigung.

Er linste zu Jarid, welche sich sorgfältig aufgerichtet hatte und die frisch verheilte Haut unter ihrem zerrissenen Gewand abtastete, gefolgt von ihrem linken Arm. Erst, als sie sich davon überzeugt hatte, dass nichts mehr gebrochen war, blickte sie Trieest wieder an. Ausdruckslos nestelte sie an der Einhorn-Horn-Schnalle ihres Umhangs herum.

„Weiß sind die Seemöwen nicht nur im Winter“, wiederholte Trieest etwas lauter.

„Sobald der Schnee geschmolzen ist, ist er nicht mehr weiß“, antwortete Jarid rasch und ein nur leicht gezwungenes Lächeln huschte über ihr Gesicht.

Vorsichtig humpelte sie auf den Feuerkrieger zu: „Komm, Tri, gucken wir uns an, was wir aus dem Schädel des Biests errungen haben.“

Trieest folgte ihrem ausgestreckten Finger zum Rankenschwert, welches immer noch etwas abseits am Boden lag. Dessen Spitze umschloss weiterhin den großen roten Kristall, der im Kopf des wilden Hraak gesteckt hatte. Dort, wo sich bei einem handelsüblichen Tier der Denkapparat befinden sollte.

Trieest war nicht riesig erpicht darauf, herauszuknobeln, um was für Magie es sich dabei handeln konnte und wie dieses Wesen wohl zustande gekommen war. Aber Jarid liebte solche Untersuchungen, und eine solche könnte Trieests Gedanken von seinem weiterhin unerfüllten Prozess des Wandels nehmen.

So griff Trieest vorsichtig nach seinem Rankenschwert und führte es vor seine Augen, ganz nahe, dass er den Kristall endlich scharf sehen konnte. Komplizierte Runen waren in seine glatte Oberfläche geritzt worden, vielleicht gar welche mit magischer Bedeutung. Die Runenmeister der Silberzwerge würden bestimmt mehr dazu wissen. Aber Iril reiste gerade mit Aćh durch Tulgor, oder? Dann würden Jarid und Trieest wohl einen anderen Experten finden müssen.

Trieests Blick wurde vom Edelstein wie magisch angezogen. Er leuchtete in einem satten Dunkelrot. Unter der steinernen Oberfläche waren hin und wieder dunkle Schwaden zu erkennen, welche willkürlich im Innern des Kristalls umherwaberten. Etwas Böses war das, das fühlte Trieest in seinem Magen.

Angewidert befahl den Ranken, sich zu teilen. Mit einem rostigen Knirschen glitten die Klingenstücke wieder an ihre angestammten Plätze und ließen den roten Kristall fallen – direkt in Trieests ausgestreckte Hand.

Sobald der Kristall seine Handfläche berührte, wurde Trieest schwarz vor den Augen. Das letzte, was er wahrnahm, war eine konzentrierte Bosheit von Etwas, das sich in seinen Geist bohrte, durch seinen Körper strömte und ihn mit Gefühlen von Wut und Hass erfüllte. Dann umfing ihn der Mantel der Ungewissheit und Ohnmacht.\bigskip







Vorsichtig tastete Jarid erneut ihren Arm ab. Konnte es wahrlich sein, dass Trieests Lavastein ihre Wunde geheilt hatte? Sie wusste über die unglaubliche Macht, die der Stein barg, doch derart direkte Heilkräfte hatte er bislang nie gezeigt, und erst recht nicht bei ihr.

Ein tiefes Grollen riss sie aus ihren Gedanken. Erschreckt drehte sich Jarid um und sah sich dem zähnefletschenden Gesicht Trieests gegenüber. Dessen Augen hatten sich tiefschwarz verfärbt und dunkler Geifer lief aus seinem leicht geöffneten Mund. Sein Kopf zuckte unwillkürlich und seine Hände hatten sich zu Fäusten geballt. Das Rankenschwert in seiner Hand zuckte ebenfalls und verformte sich in zwei verschlungene Geweihe. Diese sahen dem Geweih, welches der verstorbene Hraak immer noch auf dem Schädel trug, unangenehm ähnlich. Und der Lavastein ... zum ersten Mal, seitdem Trieest den Lavastein in seine Brust eingesetzt bekommen hatte, leuchtete er gar nicht, sondern steckte fahl in Trieests Brust, als wäre er eine unbelebte Verzierung seiner eleganten Rüstung. Jarid glaubte gar, finstere Schemen unter der Oberfläche des Lavasteins umherhuschen zu sehen.

Dann sah Jarid den rot glühenden Kristall, welcher in Trieests linker Hand steckte. Der Kristall, den Trieest aus dem Schädel des Hraaks gezogen hatte, dort, wo sie ein Gehirn vermutet hätte.

Rasch wich Jarid zurück und stolperte prompt über eine nervige Wurzel am Boden mitten in eine Ansammlung Waldpilzen, welche bei Jarids Aufprall eine Menge Sporen entließen. Jarid unterdrückte den Niesreiz. Pilze, so spät im Jahr?

Trieest, oder vielmehr die böse Macht, die Trieests Körper steuerte, fand sich in seinem Körper offenbar noch nicht vollkommen zurecht. Finster-Trieest tat einige stolpernde Schritte, während sein Mund weiterhin unbekannte Wörter formte und seiner Kehle ein gutturales Dröhnen entsprang. Mit pechschwarzen Augen sah er um sich und fixierte die am Boden liegende Jarid. Er ließ einen Fuß nach dem anderen auf den Waldboden krachen und bewegte sich Stück um Stück auf sie zu.

„Keinen Schritt weiter!“, polterte Jarid um einiges selbstsicherer, als sie tatsächlich war. Sie mochte kein Schwert tragen, war aber definitiv eine ernst zu nehmende Gegnerin.

Finster-Trieest blieb tatsächlich stehen und musterte sie aufmerksam, den Körper hin- und herwiegend. Auch diese Bewegung erinnerte Jarid ungut an den Hraak, der stets auch so auf seinen Beinen umhergestakst war.

„Wer bist du?“, rief Jarid nun aus.

Finster-Trieest verzerrte seine Lippen zu einem Lächeln und ein schmerzhaft klingendes heiseres Flüstern entsprang seinem Mund: „Ahhh, ein menschlicher Körper. Naja, halbmenschlich zumindest. Wie sehr ich das vermisst habe. Der Hraak, dieses Untier, war wahrlich kein würdiges Gefäß für meine Wenigkeit.“

Jarids Blick schweifte panisch über ihre Umgebung. Allerlei Tautropfen waren da zu finden, aber keine Wasserfläche, welche groß genug gewesen wäre, um einen Wassertunnel zu errichten. Nun, eigentlich wollte sie Trieest ohnehin nicht allein lassen. Zeit, sich diesem Monster zu stellen.

Finster-Trieest sprach krächzend weiter und nahm nun wieder Schritt in Richtung Jarid auf. Offenbar konnte er von Augenblick zu Augenblick besser mit diesem fremden Körper umgehen.

„Du fragst mich nach meinem Namen. Habe ich noch einen Namen? Lang ist es her, dass man mich nannte. Ich fühle mich nicht mehr nach jemandem, der einen Namen trägt.“

Jarid ließ sich nicht davon abbringen. „Und wie nannte man dich, als du noch einen Namen trugst?“

Finster-Trieest verzog das Gesicht. „Die Druidin, die mich in diesen verfaulten Körper, diesen Hraak, hineinpresste, nannte mich das pure Böse. Dass ich Tod und Zerstörung über so viele Leben bringen würde, ja, bereits gebracht hätte. Völliger Schwachsinn, ich bin nur ein einsamer Geist, der viel zu lange ohne ein Gesicht und ohne Hände auskommen musste. Meine dunkle Macht ist genau so wenig böse wie es ein Schwert ist; es kommt darauf an, wie man sie nutzt.“

„Und doch hast du deine dunkle Macht für Böses genutzt, nehme ich an?“, versuchte Jarid, das Wesen weiter reden zu lassen,

„Ich habe sie für mich genutzt. Wer an meiner Stelle würde das nicht?!“, sprach Finster-Trieest, Zornesfalten in der gefurchten Stirn. „Ich musste töten, um zu überleben. Und ich bin bereit zu töten, um die offenen Rechnungen meiner Vergangenheit zu schließen. Wer würde das nicht? Ich bin nicht das Böse! Warum nannte sie mich so?!“

„Ich nenne dich nicht so, wenn du nicht willst. Du sagtest, du hättest einen Namen gehabt, bevor diese Druidin dich in den Hraak steckte. Wie lautete dieser Name?“

Finster-Trieest grinste. „Ja, du würdest diesen Namen vielleicht bereits kennen. Aber ich fühle mich nicht mehr damit verbunden. Es bringt mich nicht weiter, mich hier von dir ausfragen zu lassen. Lass mich stattdessen dich nach deinem Namen fragen, o blaugewandte Andori, damit ich weiß, wessen Namen ich nennen soll, wenn ich deine Heimat zugrunde richte.“

Jarid dachte nicht daran, ihren Namen auszusprechen – Namen besaßen Macht, das wusste sie – und so tat sie das, was ihr im Augenblick am sinnvollsten erschien: Sie krümmte ihre Finger und fokussierte all ihre Gedanken auf das Wasser in der Luft und im Boden, in ihrer gesamten Umgebung ... und auf das Wasser in Trieests Blut. Sie wusste, dass ihre Magie niemals ausreichen würde, um seinen inneren Blutstrom völlig zu kontrollieren und den Körper bewegungsunfähig zu machen, aber das war auch nicht nötig. Hauptsache, ihr Gegner war verwirrt. Und abgelenkt.

Jarid hob ihre Arme ruckartig in die Höhe und die Tautropfen in ihrer Umgebung folgten ihrer Bewegung ebenso wie ein kleiner Teil in Trieests Blut. Noch während ein verwirrter Ausdruck über Finster-Trieests Gesicht zog, als seine spitzen Ohren rötlich anliefen und seine Beine zu zittern begannen, stürzte sich Jarid auf ihn, packte seine beiden Handgelenke mit ihren Händen und zog ihn zu Boden. Trieests Rankenschwert fiel aus seinem Griff, doch die Faust mit dem roten Kristall hielt er fest geschlossen, und nun hatte er sich von der Überraschung erholt.

„Nun, wenn du mir deinen Namen nicht sagen willst, werde ich wohl einfach in den Erinnerungen des Feuerkriegers nachforschen müssen“, grinste Finster-Trieest ungetrübt, während er mit Jarid rang. Nach einer kurzen Pause erwiderte er triumphierend: „Aha, du bist gar keine Andori! Jarid! Jarid Morgentau aus dem fernen Danwar. Jarid, die Brunnenspringerin. Was für ein ordinärer Name. Was für eine ordinäre Person du doch bist.“

Mit einem groben Tritt stieß Finster-Trieest Jarid von sich. Kurz danach grub sich seine immer noch den Kristall umklammernde Faust in ihre Magengegend. Jarid japste nach Luft. Ihre flache Handkante fand Finster-Trieests Hals. Nun lagen beide Kontrahenten nebeneinander auf dem Waldboden und schnappten nach Luft.

Jarid rollte sich als erste weg, Finster-Trieest war dafür als erster wieder auf den Beinen. Sorgsam darauf achtend, die Faust mit dem roten Kristall nicht zu öffnen, tastete er seinen Hals ab und war offenbar zufrieden mit dem Ergebnis, denn nun wandte er sich wieder Jarid zu.

Vor zwei mächtigen Schwingern konnte Jarid sich wegducken, der dritte erwischte sie. Doch statt dem erwarteten Faustschlag fühlte sie nur eine kalte Berührung, als Finster-Trieest den roten Edelstein beinahe sanft an ihre Wange drückte.

Da spürte sie, wie \textit{Etwas} durch ihren Körper streifte und ihren Geist berührte. Sie erstarrte. Nicht, weil sie es wollte, sondern weil ihre Muskeln nicht mehr taten, wie gebeten.

Trieests Körper presste auf Jarid, während ihre eigenen Muskeln erstarrten und ihr größere Bewegungen verwehrt blieben. Unbeweglich sank Jarid zu Boden, Trieests Gewicht auf sich, den eiskalten spitzen Kristall weiterhin in ihre Wange stechend.

Seltsame Gedanken schossen durch ihren Kopf, die nicht die ihren waren. Bilder von grausamen Riesen in Gebäuden aus Knochen. Ein fröhlicher kleiner Junge, der älter und älter wurde und auf dessen ergrauenden Haaren sich eine golden geschwungene Krone bildete. Ein finsterer kleiner Junge, der von Rauch und Schatten umgeben brüllte. Tränen und Qualen in Schnee und Eis. Untote vor einer Burgruine mit schneebedeckten Zinnen. Ein roter Edelstein in einer blauhäutigen Hand, die Jarid bekannt vorkam. Schemen dunkler Gestalten hinter einer roten Scheibe. Ein rothaariger Zwerg vor einem bläulich schimmernden Portal. Eine zahnlos grinsende grünhäutige Druidin mit Vogelscheiße im verfilzten Haar. Und dann der Hraak, dieses Untier, und dessen animalische Instinkte. Einen Augenblick lang fühlte Jarid sich so, als sei sie selbst der Hraak, als würde sie selbst mit dessen scharfen Zähnen frischen Blut kosten.

Dann traten neue Erinnerungsfetzen vor Jarids inneres Auge, die sie als ihre eigenen erkannte, die sie jedoch nicht selbst hervorgerufen hatte. Dieses finstere Etwas analysierte ihre Erinnerungen, ihre Vergangenheit!

Alles in allem erfuhr Jarid nur einige Minuten der Bewegungsunfähigkeit, während die fremde Macht aus dem an ihre Wange gepressten Kristall ihren Geist erforschte und Bilder aus ihrer Vergangenheit hervorrief. Ihre Kindheit in Danwar, ihre Eltern, ihre Freunde, ihre erste romantische Beziehung, durch all das jagte die fremde Macht in Windeseile, es bedeutete ihr nichts. Der Orden der Wassermagier, wie Jarid ihre Ausbildung begann, wie sie mit Bravour bis zum dritten Zirkel aufstieg. Die fremde Macht verblieb einige Momente bei ihrer Lehre, schien Jarids Fähigkeiten zu beurteilen, verwarf dann aber auch diese Erinnerungen. Ihr Geist wurde weiter durchforstet, erste Erinnerungen an Trieest kamen auf. Wie Jarid den kleinen Trieest tröstete, als sein erster Milchzahn ausgefallen war. Wie sie den Fangzahn zu Feuermeisterin Nidwal brachte und von ihm untersuchen ließ. Wie Nidwal besorgt den Kopf schüttelte. Die fremde Macht kicherte mit Jarids Mund. Schließlich der Aufbruch von Danwar. Jarid, wie sie umgeben von den Echos aus der Roten Grotte am Boden saß und schluchzte. Wie Jarid und der feuerkriegerische Trieest, nun größer als Jarid, auf einem kleinen Kahn in den Westen aufbrachen. Wie sie Seite an Seite Heldentaten vollbrachten. Die Furcht, als sie an der Ewigen Kälte erstarrte. Trieests lächelndes Gesicht, das sie als Erstes erwartete, als sie aus dem Eisschlaf erwachte. Wie Brandur sie für ihre Taten lobte. Ihr Kiefer verkrampfte sich ohne ihr Einwirken zu einer wütenden Fratze. Der Kampf um den Alten Wehrturm während der Befreiung der Rietburg, an den Seiten von Feuermeister Lifornus und dem reisenden Temm Wrort. Brandurs Beerdigung, wo sie und Trieest vom frischen Landesleiter Thorald höchstoffiziell zu Helden von Andor ernannt wurden. Die Bilder der restlichen Helden von Andor traten einer nach dem anderen vor Jarids inneres Auge. Erneut verzerrten ihre Gesichtsmuskeln sich, als wäre sie wütend. Ihre Zeit im Land der Brüder, ihre vergangenen Abenteuer in Tulgor, die Rückkehr zum Königsfrieden ...

Dann, endlich, zog sich die fremde Macht aus ihrem Geist zurück. Trieests schweres Gewicht löste sich von ihr, als der Feuerkrieger sich unsorgfältig wieder aufrichtete und den roten Kristall wieder von ihrer Wange nahm. Jarid zappelte probehalber mit ihren Beinen. Ihr Körper gehörte wieder ihr!

„Nimm das nicht persönlich, Jarid Morgentau, aber der Körper deines Trieests ist meiner Wenigkeit würdiger als deiner”, sprach die finstere Macht durch Trieests krächzende Stimmbänder, „Dein Geist ist allerdings schon interessant … du hast schon Mumm, deinen Begleiter so lange anzulügen. Keine Sorge, er nimmt gerade nichts war. Er weiß noch von nichts. Fast fühle ich mich dazu verlockt, ihn zu wecken und ihm zu erzählen, was du ihm alles verschweigst.”

Jarid drehte sich keuchend auf den Bauch und stemmte sich langsam hoch. „Du hast keine Ahnung, warum ich das tue! Du hast kein Recht, mich zu …”

„Ich weiß genau, warum du das tust! Du hast Angst, Jarid, Angst um dich und um ihn! Aber keine Sorge, ich werde es ihm nicht erzählen. Sein Körper ist alles, was ich von ihm brauche. Ich muss nur noch dafür sorgen, dass ihn mir keiner wegnehmen kann.”

Finster-Triefst grinste breit und führte seine Faust zu seinem Mund. Jarid begriff gerade noch rechtzeitig, was er vorhaben könnte: Er wollte Trieest den roten Kristall schlucken lassen! Wenn seine Kontrolle über Körper von physischer Berührung mit diesem roten Kristall abhing, wäre es äußerst ungünstig, wenn dieser Kristall sich in Dauerberührung in einem Magen befände. Bald wäre Trieest nicht mehr zu retten.

Obwohl ihre Glieder bereits schmerzten, zögerte Jarid keine Sekunde, sondern schleuderte sich Finster-Trieest entgegen, packte seine Faust und zerrte sie von Trieests Gesicht weg. Wieder fokussierte sie sich auf Trieests Blut und versuchte, dieses weg aus dem Arm, weg von dieser Hand, in seinen Kopf strömen zu lassen. Doch ihre Kapazität war erschöpft, genauso wie der Rest von ihr. Ihr einziger Trost war, dass Trieests Körper ebenso erschöpft sein musste.

Finster-Trieests Miene verzerrte sich zu einem Zähnefletschen und seine Lippen entsprang erneut dieses schmerzhaft kratzige Flüstern: „Du kannst ihn nicht mehr retten. Er ist MEIN, so wie alle Welt bald MEIN sein wird!“

Jarid verhakte ihr Bein mit Trieests. Finster-Trieest ging zu Boden. Diesmal war Jarid darauf gefasst, als weiterer Tritt sie zu verscheuchen versuchte. Sie verlagerte ihr Gewicht und stützte sich auf Trieests Brust. Seine Hand, die den roten Kristall festhielt, hielt sie wiederum immer noch eng umklammert. So presste sie den Finsterling in den weichen Waldboden.

Finster-Trieest wand sich unter ihr und ihr Griff begann sich zu lösen. Lange würde sie ihn nicht mehr auf den Boden pinnen können. Schmerz stach durch ihren Torso. Da erstrahlte der Lavastein in Trieests Brust, zunächst zögerlich, dann immer heller. Orangerote Feuerschlieren schossen daraus hervor und umringten Finster-Trieest. Dieser brüllte auf und spie Jarid schwarze Spucke entgegen, die pechschwarzen Augen hasserfüllt zusammengekniffen. Dampf zischte, als Trieests Körper sich schlagartig erwärmte, Haut und Haare vertrockneten. Der Lavastein wehrte sich und war ihnen endlich zu Hilfe gekommen! Jarid sprang auf und trat einige Schritte vor der Hitze zurück.

„Sei still! Du bist tot!“, schrie Finster-Trieest ins Leere. Er starrte nicht mehr Jarid an, sondern in den Himmel, seine dunklen Augen auf der Suche nach einer Gestalt, die nur er wahrnehmen konnte. Er klang verwirrt, ja, gar verängstigt: „Du bist gestorben, wie du es verdient hast! Ich sah dein Grab! Ich sah dein Reich in der Hand deines Nachfolgers leiden! Das ist nicht deine Stimme! Das kann nicht sein!“

Er griff sich an die Brust und sackte zusammen. So zuckte er am Boden, wie es der echte Trieest schon zu oft hatte erleiden müssen. Er griff nach dem glühenden Stein in seiner Brust und zerrte an ihm, hob dann die Faust mit dem roten Kristall darin und ließ sie auf den Lavastein krachen. Sein grollendes Knurren wurde nur lauter.

„Das ist nur eine Illusion! Du bist gefallen!“

Erneut ließ er den Kristall auf den Lavastein krachen und Jarid war sich sicher, etwas brechen zu hören. War das der Lavastein oder Triests Rippe gewesen?

Als ihr schwindelig wurde und ein Ziehen in ihrer Brust sich bemerkbar machte, sah Jarid zum ersten Mal seit dem Beginn des Kampfes an sich herunter und erschrak ganz gewaltig. Da befand sich Trieests Rankenschwert, immer noch in Form eines finsteren Geweihs. Hatte er es nicht fallen gelassen? Nun steckte es in Jarids Torso, knapp unter ihrem Herzen.

Er hatte sie doch noch erwischt.

Wie lange steckte das Schwert nun schon da? Wie schlimm waren ihre Innereien verletzt? Wie weit war es bis zum nächsten Dorf, wo man es sicher entfernen konnte? Oh nein, da kam sie schon, die Schwäche in den Gliedern und die Dunkelheit im Gesichtsfeld.

Die Stimmen der Echos aus der Roten Grotte begleiteten sie, während Jarids Bewusstsein entschwand. Das letzte, was sie wahrnahm, war ein fernes Wiehern. Pferde?

Hier, mitten im Wald?

Dann umfing sie der Mantel der Ungewissheit und Ohnmacht.\bigskip







\textit{Die Pferde trafen ihn unterwartet und schleuderten ihn zu Boden. Der Kristall wurde aus Finster-Trieests Hand geschleudert. Einen Augenblick lang konnte er seine Kontrolle über Trieests Körper beibehalten, irgendetwas an diesem Lavastein in dessen Brust schien die Verbindung zu verlängern. Dann brach seinen geistigen Fokus zum Feuerkrieger dennoch ab.}

\textit{Das Böse brauchte einige Sekunden, um die urplötzliche Absenz von Trieests Sinneswahrnehmungen zu verarbeiten und sich auf die stumpfen Sinne seines kristallenen Gefängnisses zurückzubesinnen. Etwas war da, etwas war über ihm. Ein großer Schatten von etwas Hölzernem. Schwer. Schwerfällig. Laut. Und dann war da dieses Getrappel.}

\textit{Ein Wagenrad rammte den Kristall. Das Böse spürte, wie dessen Stabilität unter dem schweren Druck nachließ. Risse durchzogen seine Oberfläche.}

\textit{Dem Bösen blieben nur Augenblicke. Es verspürte keine Schmerzen per se, wurde aber beinahe ohnmächtig vor Todesangst. Es war magisch an diesen Kristall geknüpft. Wenn dieser zerstört würde, würde auch das Böse ausgelöscht, ohne Möglichkeit, sich selbst wieder als Geist neu zu formen. Dazu durfte es nicht kommen. Nicht jetzt, wo es endlich so weit gekommen war!}

\textit{Ohne ein nahe gelegenes Lebewesen, das es kontrollieren konnte, blieb dem Bösen nur keine Möglichkeit, seinen Einfluss auszuweiten. Es musste eine andere Taktik wählen. So zog sich das Böse in eine einzelne dünne Kristallfaser seines Gefängnisses zurück und bangte darum, diese möge stabil bleiben. Die schwarzen Wogen seiner Seele verblassten unter der Oberfläche und ließen den Kristall blutrot zurück bis auf den kleinen Bereich, in den sich das Böse quetschte.}

\textit{Das Wagenrad drehte sich weiter und der finstere Kristall zersplitterte. Hunderte von blutroten Bruchsplittern übersäten den blutigen Waldboden und verfärbten sich langsam zu einem strahlenden Weiß, als das einst in ihnen gefangene Licht des roten Mondes hinausquoll.}

\textit{Ein einzelner Splitter stand auf den ersten Blick heraus aus den vielen Überresten. Nicht aufgrund seiner Form oder Größe, sondern wegen seiner Farbe. Während die anderen sich langsam weiß verfärbten, wurde dieser einzelne Splitter tiefschwarz.}





























\newpage
\section{In der Weinkutsche}









\az{Jahr 61}

\textit{Halle des Ältestenrats, 61 a.Z.}\bigskip



Das aufgebrachte Gemurmel im danwarischen Ältestenrat verklang erst, als die Älteste Rowinda – Jarids Mutter – energisch auf das Ratspult schlug und um Ruhe rief.

„Ruhe im Rat! Dann bist du also als Zeugin hier, Jarid?“

„Unter anderem. Ich will hier bloß meine Perspektive teilen und meine Meinung kundtun. Seit wann achtet Ihr denn plötzlich wieder auf all diese oberflächlichen Details in Ratssitzungen?“

„Seitdem wir darüber urteilen müssen, ob der Angeklagte weiterhin ein Dasein auf Danwar verdient hat! Es ist lange her, seit ein solch schwerwiegender Vorfall unsere friedliche Gemeinschaft erschüttert hat.“

„Gute Güte, überlegt ihr euch ernsthaft, Trieest von der Insel zu verbannen? Er ist doch erst dreizehn!“

„Dreizehn Jahre sind für solche wie ihn weitaus genug, um die Adoleszenz zu erreichen. Er hätte seinen Instinkten längst Herr werden sollen!“

„Keiner leugnet, dass dieser Unfall besser nicht geschehen wäre, und keiner weiß, womit wir das Risiko eines weiteren möglichst gering halten ...“

„Oh, ich wüsste da schon was. Und ich weiß auch schon, dass ich dafür stimmen werde!“

„Ruhe im Rat, Freiga! Jarid, kannst du guten Gewissens sagen, dass so eine Tat nicht wieder begangen werden wird, wenn wir Trieest weiterhin frei hier herumflanieren lassen?“

„Natürlich nicht, aber ihr könnt mir auch nicht guten Gewissens sagen, dass Jormudd Trieest nicht anfallen wird, wenn sie sich das nächste Mal sehen ... ähm ... treffen.“

„Jormudd ist ein guter Junge, aus einem guten Elternhaus. Von deinem Trieest lässt sich das nicht sagen.“\bigskip







\az{Jahr 68}

\textit{Sieben Jahre später.}\bigskip




Es war warm, als Trieest langsam wieder zu Bewusstsein kam. Dies war keine Besonderheit, als Feuerkrieger fror Trieest so gut wie nie. Dennoch konnte er feststellen, dass seine Umgebungstemperatur unüblich hoch war. Ein raues Gefühl auf seiner Haut bestätigte: Er lag auf einer Decke!

Das Zweite, was Trieest bemerkte, während seine Wahrnehmung langsam wieder einsetzte und der graue Nebel der Ohnmacht sich lichtete, war, dass er nicht mehr auf dem ruhigen Waldboden lag. Vielmehr holperte und schaukelte der Boden, auf welchem er lag, so stark, dass er kaum glauben konnte, dass er nicht früher erwacht war. Es war auch kein Boden, auf dem er lag, sondern ... Holz? Ein Tisch? Lag er auf einem schaukelnden Tisch? Aber da war auch der unmissverständliche Gestank von Eisenverschlägen.

Trieests Nasenflügel bebten, als er seine Umgebungsluft einsaugte. Jarid war da, zu seiner Rechten, das war schon mal beruhigend. Zu seiner Linken, etwas schwächer wahrzunehmen, befand sich eine weitere Person, die ihm unbekannt war. Eine Person, die nach faulen Zähnen und alten Kleidern duftete. Um sie alle herum roch es nach eigenartigen Früchten ... nein, das waren keine Früchte mehr, das war... Met? Wein? Fässer mit einer alkoholischen Flüssigkeit jedenfalls. Und ... Pferdemist! Das waren Pferde, mehrere sogar!

Er befand sich in einem Karren. Vielleicht gar in einer Kutsche.

Trieest schlug die Augen auf und schloss sie gleich wieder, als selbst die Plane über seinem Kopf das gleißende Sonnenlicht kaum mindern konnte.

Jarid zu seiner Rechten regte sich und trat an ihn heran, legte ihre Hand auf die seine.

„Gut, du bist endlich erwacht. Wie fühlst du dich?“

Sie klang vorsichtig, ängstlich. Warum bloß?

Trieest hustete. Seine Brust schmerzte ungewöhnlich stark. Woran das wohl liegen könnte?

Langsam regte Trieest sich, nur um festzustellen, dass seine Bewegungsfreiheit eingeschränkt war. Starke eiserne Ketten an seinen Handgelenken hinderten ihn daran, sich umzudrehen. Das war äußerst seltsam. Andere Leute waren manchmal ob Trieests Erscheinung verängstigt, aber Jarid sollte doch wissen, dass er keine Gefahr darstellte.

„Wie fühlst du dich, Trieest?“, fragte Jarid erneut, diesmal noch angespannter.

„Als wäre eine Horde Steppenechsen über meine Brust getrampelt. Was ... was ist hier los?“, fragte Trieest.

Jarid antwortete nicht sofort.

Von links – woher auch der Duft der unbekannten dritten Person stammte – ertönte eine knorrige Stimme: „Ist das er? Ist er wach, Liebes?“

„Ja, Lysbett“, rief Jarid als Antwort.

„So sage mir, Meisterin über das Wasser, trachtet dein Begleiter uns immer noch nach dem Leben?“

Trieest schluckte. Immer noch?

„Was ist geschehen, Jarid?“, flüsterte Trieest argwöhnisch.

Langsam öffnete er seine Augen und zuckte nur noch leicht zusammen, als das Sonnenlicht sie traf. Seinen Kopf nach links und rechts drehend erkannte Trieest, dass er in der Tat auf dem Boden einer Kutsche lag, seine Arme mit festen Eisenketten am Kutschenboden befestigt. Eine große Plane verwehrte ihm den Blick auf dem Himmel oder aus der Kutsche hinaus, wo den Geräuschen und Gerüchen zufolge mehrere Pferde emsig vor sich hin eilten.

Abgesehen von den festen Eisenketten und seiner roten Stoffkleidung trug Trieest nichts mehr. Seine Schuhe hatte man ihm ausgezogen, seine Rüstung ebenfalls. Dieses Ungetüm an filigraner Schmiedekunst war sowohl zweckdienlich als auch zeremoniell, und Trieest lag einiges daran. Dennoch konnte er sie nirgends sehen. Hatte man sie vielleicht in eines der Fässer gesteckt? Und wo war sein Schwert? Ein solches Rankenschwert war eine noch größere Rarität als seine Rüstung, und ein Rankenschwert, welches an einen selbst gebunden war, war buchstäblich unbezahlbar.

Trieest verrenkte seinen Kopf und kniff seine Augen zusammen, um auf seine Brust zu spähen. Etwas war hier ganz und gar nicht in Ordnung! Der eiserne Reif, welche den Lavastein in seiner Brust verankert halten sollte, befand sich ebenfalls nicht mehr an seinem angestammten Platz. Stattdessen lag der Lavastein völlig ungeschützt in Trieests Fleisch, als könne er jeden Moment herausfallen. Trieest wusste, dass der Lavastein das nicht tun würde, in den sieben Jahren des Tragens hatte dieser Stein sich so tief in seinen Körper gegraben, dass dieser ihn fast schützend festhielt. Dennoch war der Anblick des Steins ohne den ihm angestammten Rahmen zutiefst erschütternd.

Zudem war da ein Spalt im Lavastein zu sehen! Ein Riss im unzerstörbaren Lavastein?! Bei Kentars Dreizack, was war geschehen?

Trieest musterte nun Jarid, welche neben ihm saß und seinem Blick auswich. Im Gegensatz zu ihm, welcher von Schrammen übersät und mit Blut und Erde verdreckt war, sah man Jarid beinahe keine Spuren des vergangenen Kampfes gegen den Hraak an. Ihre Wunden waren ja vom Lavastein geheilt worden und das zerrissene Kleid musste sie bereits wieder geflickt haben. Doch warum war das Kleid über Jarids Bauch in einem dunklen Rot verfärbt? War das wieder Blut?

Es hatte die Wassermagiergilde aus Danwar Jahrzehnte gekostet, Stoffe und Talismane aus einer Art festen Wassers zu entwickelt. Ihre Träger konnten (sofern sie in der Kunst des Wasserformens bewandert waren) diese Kleider wie Wasser formen und so etwa auch bei Reisen durch Wassertunnel mit sich führen – eine unglaubliche Errungenschaft. Jarids Einhorn-Horn-Schnalle an ihrem Umhang war ein ähnliches Artefakt. Nun fehlte nur noch, dass die Wassermagier größere Waffen entwickelten, die sie über große Distanzen mitnehmen könnten, wenn sie von Wasserspeicher zu Wasserspeicher teleportierten. Doch für den Moment genügte ihnen, dass sie nicht nackt am anderen Ort emergieren mussten.

Die mysteriöse Kutsche holperte und polterte über Stock und Stein, als gäbe es kein Morgen, und so wurde Trieest heftig durchgeschüttelt. Heftig durchgeschüttelt wurden auch die vielen Fässer, welche schön aufeinandergestapelt vor Trieests nackten Füßen standen und mit weiteren Ketten stabil aneinander befestigt waren. Der Weingeruch war eindringlich.

Lysbett, diese Person auf dem Kutschbock, musste eine Weinhändlerin sein! Als hätte sie darauf gewartet, dass er sich an sie erinnerte, öffnete sich die Plane zu Trieests Linken und das Gesicht einer alten Frau blinzelte ins Innere. Brandnarben übersäten ihre linke Gesichtshälfte und eine schnörkellose Augenklappe verdeckte den Blick auf eine vermutlich leere Augenhöhle. Warnend polterte die Weinhändlerin: „Liebes, wenn du mir nicht bald ausrichtest, dass der Kerl wieder putzmunter und dämonenfrei ist, halte ich dieses Gespann an und entledige mich auf meine Art dieses Monsters. Mit derart dunklen Kräften ist nicht zu spaßen.“

„Alles ist gut, Lysbett, er scheint wieder in Ordnung zu sein“, rief Jarid nun hektisch zurück.

„Das ist Lysbett“, stellte Jarid Trieest die Alte vor, „Sie rettete uns und ließ unseretwegen einige Fässer ihres Transports im Wald liegen.“

„Drachenfass Rachenputzer. Hochexplosives Zeug“, murmelte Lysbett, „Kostete mich schon meinen linken Arm. Und dieses Biest von einem Krieger wird mich nicht meinen rechten kosten!“

Sie nickte mehr zu sich selbst als zu Jarid und zog sich wieder auf den Kutschbock zurück, von wo aus einige Male „Hü!“ und „Ho!“ erklang. Die Kutsche holperte noch etwas stärker, als die Pferde einen Zahn zulegten.

„Ich soll wieder ‚dämonenfrei‘ sein?!“, fragte Trieest ängstlich, „Was ist geschehen? Was habe ich getan? Wie lange war ich weg?“

„Was ist das letzte, woran du dich erinnerst?“, antwortete Jarid mit einer Gegenfrage.

Trieest überlegte. Sie hatten den Hraak bekämpft, das hatte er deutlich in Erinnerung. Sie hatten sich unterhalten über die Prophezeiung aus der Roten Grotte. Er war enttäuscht gewesen über die Stille des Lavasteins und das Fortbestehen seiner Bürde und hatte dies ausgedrückt. Danach ... was danach lag, konnte Trieest nicht mehr sagen.

Er erzählte, was er wusste: „Wir haben den Hraak überwunden, aber das hat meinen Prozess des Wandels nicht beendet. Wir wollten gerade den roten Kristall untersuchen. Dann ... dann wachte ich hier auf.“

„Du hast dein Schwert genommen und den roten Kristall berührt. Irgendeine finstere Macht ergriff Besitz von dir. Du hast mich angegriffen“, sprach Jarid gezwungen ruhig.

Trieest ließ ihre Worte besorgt auf sich wirken. Er zweifelte nicht an ihnen, wohl aber an Jarids scheinbarer Unversehrtheit. Das konnte nicht gut gegangen sein. Im waffenlosen Kampf war Jarid zwar geschickt und hatte Trieest schon dutzende Male in Grund und Boden gerungen, doch war Trieest um einiges stärker als sie, ganz zu schweigen davon, dass er ein magisches Schwert besaß und Jarid nichts außer ihre Hände.

„Wie ist es ausgegangen?“

Ein Grinsen huschte über Jarids Züge, als sie salopp erwiderte: „Lysbett kam in ihrer Kutsche angefahren und hat dich über den Haufen geritten.“

Dann verdüsterte sich ihr Ausdruck wieder als sie nachsetzte: „Zu diesem Zeitpunkt lag ich bereits am Boden, dein Rankenschwert in meinem Torso. Hätte der Lavastein sich nicht gegen diese finstere Macht gewehrt, wäre ich jetzt nicht mehr hier.“

Jarid berührte geistesabwesend ihr magisch geflicktes Gewand an der blutverschmierten Stelle und verzog schmerzerfüllt das Gesicht.

„Was ... was geschah dann? Wie steht es jetzt um dich?“

„Zum Glück ist Lysbett eine geschickte Heilerin. Sie brachte mich wieder zu Bewusstsein und schleppte mich zu dir und deinem Lavastein. Dieser erfrischte mich erneut, als ich ihn berührte, wenn auch nicht so ausführlich wie beim ersten Mal. Und nun bringt sie uns beide zum Baum der Lieder. Du warst zum Glück nur einige Stunden weg, dein Lavastein muss auch dich geheilt haben. Ist dir aufgefallen, dass er einen Riss hat? Die Faust des finsteren Dus konnte tatsächlich irgendwie Schaden anrichten an diesem angeblich unzerstörbaren Stein. Faszinierend. Beunruhigend.“

Ja, Trieest war aufgefallen, dass der Lavastein einen Spalt trug. Kein Wunder, dass gerade konstant ein leichtes Anzeichen von Furcht vom Stein auszugehen schien. Das hatte diese finstere Macht mit Trieests eigener Faust vollbracht? Wäre er auch selbst dazu in der Lage gewesen?

Er überlegte weiter. Der Lavastein hatte Jarid also bereits zum zweiten Mal an diesem Tag geheilt. Warum schien ihm plötzlich so viel an ihr zu liegen?

„Glaube mir, es war eine Mühsal, Lysbett davon abzuhalten, dir nicht dann und dort die Kehle durchzuschneiden. Sie bestand zumindest darauf, die ganzen Kristallsplitter zu vergraben. Was wirklich eine Schande war. Sie hatten sich allesamt glühend weiß verfärbt. Das muss doch irgendetwas bedeuten. Wer weiß, was wir alles für Erkenntnisse daraus hätten gewinnen können!“

Die letzten Worte hatte Jarid um einiges lauter gesagt, als nötig gewesen wäre, und prompt ertönte eine heisere Antwort vom Kutschbock her:

„Liebes, wenn du noch lange wegen diesen Steinsplittern in den Ohren liegst, wird euch diese Fahrt so einiges mehr kosten! Ich musste bereits drei Drachenfässer vom allerbesten Rachenputzer der Schildzwerge mitten im Wald stehen lassen, um Platz für euch zu schaffen, da lasse ich mir doch keine Trauerreden auf verfluchte Objekte gefallen!“

Jarid schwieg stille und verschränkte ihre Arme.

„Ich versprach ihr, dass wir sie reich bezahlen, wenn wir am Baum der Lieder ankommen“, flüsterte Jarid, „Gold besitzen wir kaum welches, mehr, wärst du notfalls bereit, einen Teil deiner Rüstung einzutauschen?“

„Meine Rüstung gibt es noch?“, horchte Trieest freudig auf.

„Wir mussten sie nur entfernen, um dich ein bisschen zurechtzurichten. Die Pferde haben dir nicht gut getan, Tri. Sie liegt hinter einigen Fässern. Ebenso dein Schwert“

„Dem Flammenden Gott sei Dank! Ohne sie komme ich mir so nackt vor. Doch war es wirklich klug, den potenziell gefährlichen Kristall einfach so zurückzulassen, auch wenn er zersplittert ist?“,

„Lysbett und ich haben die Splitter in einem Tuch gesammelt, natürlich ohne sie zu berühren, und sie tief im Boden vergraben, mitten im Wald. Die sollte niemand so schnell wiederfinden können. Ganz unabhängig davon waren die Splitter allesamt weiß. Keine Spur mehr von finsteren Schemen unter der Oberfläche. Und es steht doch zu vermuten, dass die finstere Macht zu diesen finsteren Schemen im Kristall gehörte?“

„Na, ebenso könnte ein glühend weißer Schemen sein, der all diese Splitter befiel. Bosheit kennt viele Farben.“

„Weiße Kristalle gibt es so einige viele in den Tiefen der Erde, davon können dir die Schildzwerge ein Liedchen singen, und keiner von denen war verflucht. Einen roten mit schwarzen Schemen wie diesen sah ich hingegen bislang nur diesen. Diese weißen Splitter tragen den Geist des Hraaks nicht mehr mit sich.“

„Dann starb diese finstere Macht bei der Zerstörung des Kristalls?“

„Wir können nur hoffen.“

„Also, auf jeden Fall enthielt dieser rote Kristall etwas Böses, welches vom Hraak unabhängig war, oder? Jemand muss den Stein diesem Monster in den Kopf gesetzt haben, um diese Gegend zu terrorisieren.“

„Das habe ich mir auch schon gedacht. Wahrscheinlich war es diese Druidin, von dem das ganze Dorf behauptete, dass sie finstere Experimente in ihrem einsamen Turm vollbrachte. Und als die finstere Macht dich kontrollierte, erwähnte sie ebenfalls eine Druidin. Zu schade, dass diese Druidin schon vor Jahren gestorben ist, sonst hätte ich ihr gerne ein paar Fragen gestellt. Andererseits ist es vielleicht auch besser so, dass eines ihrer Experimente diese Zauselin erwischte. Wer weiß, was sie sonst noch alles für Chimären in die Welt gesetzt hätte. Der Hraak allein war schon schwerwiegend genug. Wir können nur hoffen, dass sie keine anderen Monstrositäten in sonstigen Kellergewölben versteckt hielt. Oder, falls schon, dass diese nicht ausbrechen, ehe sie an Nahrungsmangel verenden.“

Trieest schwieg. Zu viele Gedanken kreisten in seinem Kopf. Er war von einer bösen Macht besessen gewesen. Eine Kutsche hatte ihn überfahren. Der Lavastein war angebrochen. Wie viele Neuerungen würde dieser Tag noch bringen?

Jarid legte sorgsam ihren Kopf auf Trieests Brust und horchte in den Lavastein hinein. Ihre Haare kitzelten Trieest, doch er unterdrückte den Kicherdrang.

„Kalt wie das tiefe Meer“, sprach Jarid, „Und ich höre wie immer nur das Geräusch der Brandung an die heißen Klippen Danwars. Bis auf den Splitter wirkt er völlig normal.“

Trieest nickte. „Ich fühle leichte Furcht von ihm, höre jedoch auch nichts außer das Rauschen des Meeres. Diese Legenden von Feuerkriegern, die mit ihren Lavasteinen sprechen, sind vermutlich übertrieben. Oder geschieht so was vielleicht erst am Ende meines Prozesses der Wandlung?“

„Als das Böse dich übernommen hatte, rief es wild ins Leere, als hörte es Stimmen. Ich glaube, dein Lavastein hat mit ihm gesprochen. Und dich gerettet.“

„Er hat uns beide gerettet. Und doch bin wieder ich am Ende wieder derjenige, der versorgt werden muss“, grinste Trieest schief.

„He, mein Bauch ist noch lange nicht verheilt. Und wenn das alles durch ist, kannst du dich gehörig revanchieren. All die Massagen und Linderungen, die ich dir verschaffe, zahlst du mir noch mit Zins und Zinseszins zurück“, scherzte Jarid.

Trieest grinste zurück und verdrängte all die Gedanken, die seinen Kopf zum Kreisen brachten. Selbst die Schmerzen in seinem Körper flauten ab. Er drehte sich zur Seite ab und versuchte, sich in den Schlaf zu verkriechen.

„Was nun, Tri?“, fragte Jarid abrupt.

„Wie meinst du das?“, murmelte Trieest müde.

„Wir können das sonst auch später bereden“, meinte Jarid, „aber wir müssen nicht so weitermachen wie bislang. Wir müssen nicht von einer heldenhaften Aufgabe zur nächsten ziehen und deine Hoffnungen und Träume auf ein Ende des Prozesses des Wandels strapazieren. Wenn du willst, können wir das Heldendasein für einen Moment auf Eis legen. Chada lässt sich von Reka ausbilden, Thorn züchtet seine Pferde, Kheela kümmert sich wieder primär um die Flusslande und Barz zog es zurück zu seiner Familie im Osten. Der Königsfrieden könnte auch uns Frieden schenken. Wir könnten uns ebenfalls irgendwo niederlassen. Der Ältestenrat lässt uns nicht nach Danwar zurück, während dein Prozess des Wandels anhält, aber die restliche Welt ist groß, so groß.“

„Ich hätte nicht gedacht, dass du das als Option betrachtest.“

Jarid und Trieest hatten sich selten über die ferne Zukunft unterhalten. Es war ihnen oftmals einfach oder wichtiger erschienen, die nächste Reise, die nächste Herausforderung anzustreben. Beiden von ihnen war schwer bewusst, dass Trieests Lebensspanne sich nicht im Ansatz mit der von Jarid vergleichen ließ. Als sie sich freiwillig dafür gemeldet hatte, ihn als Begleiterin beim Tragen der Bürde zu unterstützen, war er gemäß der Lebensspanne eines Menschen noch ein junges Kind gewesen, auch wenn er nicht mehr danach ausgesehen hatte. Nun zeigten sich bereits die ersten weißen Haare in seinem Schopf. Wer wusste, wie lange es ihn noch geben würde ... Zehn Jahre? Zwanzig? Länger konnte er wohl kaum hoffen. Seinem Leben hatte er in dieser Zeit kaum eine eigene Richtung geben können, alles hatte sich immer um den Lavastein gedreht. Und nun kamen ihm plötzlich Zweifel auf, ob er seinen Prozess in seiner kurzen Lebenszeit überhaupt noch beenden können würde.

Er lachte bitter auf: „Und all dieser Ärger nur wegen J ... wegen Jormudd.“ Es war für ihn immer noch schwer, Jormudds Namen auszusprechen. Nicht zu denken, was geschehen wäre, wenn er seine Kehle erwischt hätte. Trieest hatte manchmal immer noch Albträume davon. In letzter Zeit aber immer seltener. Jormudd und Danwar, das lag inzwischen lange hinter ihm. „Wir wissen, dass ich falsch handelte. Es wird nie wieder geschehen. Es tut mir leid und ich zahlte den Preis. Um ein Vielfaches. Meine Schuld sollte schon längst getilgt sein. Manchmal frage ich mich schon, ob ich irgendetwas grundsätzlich falsch angehe mit meinem Lavastein. Ob die entscheidende Aufgabe, die meinen Prozess des Wandels beenden wird, eine persönliche sein wird. Ob ich nach Danwar zurückkehren sollte, mich bei Jormudd entschuldigen und ihm gegenüber meine Taten gutmachen. Ob es vielleicht schon immer so leicht war.“

Jarid schüttelte ihren Kopf. „So etwas wäre vielleicht die Lösung, falls das hier ein Märchen wäre, aus dem man eine Moral ziehen sollte. Aber es würde nicht viel Sinn geben, dich aus Danwar zu verbannen, wenn deine entscheidende Aufgabe sich dort versteckte.“

„Soll ich aus meiner Bürde nicht eine Moral ziehen?“

„Idealerweise ja, wie bei jeder großen Handlung. Aber das ist nicht der Hauptgrund hinter einer Bürde. Die Hoffnung des Ältestenrats war, dass du abgeschreckt wirst, so etwas nie wieder zu tun, oder zumindest nicht in Danwar.“

„Natürlich tu ich es nie wieder! Ich war nicht ich selbst. Diese Instinkte ...“

„Diese Instinkte hast du inzwischen völlig unter Kontrolle, ich weiß. Und ich bete, dass dein Prozess bald hinter dir liegt und wir einen bürdenfreien Weg einschlagen können. Ich meine ja nur, dass wir in letzter Zeit relativ gut mit dem Stein klar kamen. Und dass ich mir vorstellen könnte, ihn eine Zeit lang zu ignorieren, so gut er uns lässt. Uns eine Auszeit zu gönnen, bis du dich bereit für deine nächste Aufgabe fühlst. Ist natürlich deine Entscheidung, wie immer.“

Trieest knurrte. So sehr er sich ein Gefangener seiner Bürde fühlte, so ärgerlich war Jarids Weigerung, ihm eine Meinung über ihre gemeinsame Zukunft mitzuteilen. Stets hatte sie ihm aufgetragen, auszusuchen, in welche Länder und Reiche sie ziehen sollten, welche Aufgaben sie priorisieren sollten. Jarid war bloß die eifrige Begleiterin gewesen. Und ohne ihre tröstenden Worte und kühlenden Wassertropfen wäre er schon lange ausgebrannt. Trieest stand so tief in ihrer Schuld.

„Ich werde nie zurückzahlen können, was du für mich getan hast, Jarid“, sprach Trieest heiser und ernst.

Jarid drückte seine Schulter. „Und das musst du auch nie. Ich begleite dich nicht aus Zwang, sondern weil ich will. Ich lasse dich deinen Weg und deine Aufgaben wählen, weil ich muss, doch folgen tu ich dir aus freiem Willen, so frei, wie ein Wille jedenfalls sein kann.“

Trieest gluckste und Jarid zischte zurück: „Und ja, das darfst du mir das wieder sagen, wenn ich dich das nächste Mal wegen einer deiner hirnrissigen Entscheidungen anschnauzte. Wir stehen das gemeinsam durch. Ich will noch an deiner Seite sein, wenn dein Leiden ein Ende findet.“

„Danke“, flüsterte Trieest. „Und wohin willst du danach reisen? Was hast du danach vor? Zurück nach Danwar und Frieden mit deiner Mutter schließen?“

Jarid sah ihn verwundert an: „Denkst du etwa, ich würde dich einfach so verlassen?! Nach all dieser Zeit will ich auch nicht nach Danwar zurück!“

Trieest sah sie stumm an. Überrascht. Jarid fuhr fort: „Ich glaube, ich will mich irgendwo niederlassen. In einer Gemeinschaft von Menschen, die sich nicht um unsere Vergangenheit schert. Ein kleines Dorf, hier im Wachsamen Wald, oder an der Küste, oder hinter den Bergen, oder hoch im Norden, wo auch immer. Ein nettes kleines Dorf, und es wird unsere Heimat sein, und wir werden es zum Erblühen bringen.“

„Wir beide?“

„Wenn du willst.“

Und ausnahmsweise entscheidest du, welcher Ort?“

„Wenn du willst.“

Sie lächelte beim Gedanken, und Trieest erwiderte das Lächeln. Dann rasselte er demonstrativ mit seinen Ketten: „Meinst du, diese Lysbett vertraut deinem Urteil genug, dass sie sich darauf einlassen würde, mich aus diesen Ketten zu lösen?“

„Was meinst du, Lysbett?“, rief Jarid zum Kutschbock.

Lysbett gab keine Antwort.

Wo Trieest so darüber nachdachte, fiel ihm auf, dass sich Lysbett schon eine ganze Zeit lang nicht mehr in ihr Gespräch eingemischt hatte. Und die Kutsche ... sie holperte und polterte ja gar nicht mehr, sondern stand still.

Es war ruhig draußen.

Zu ruhig.

Trieest blickte Jarid in die Augen und erkannte, dass ihr dieselben Gedanken aufgingen. Langsam stand Jarid auf und bewegte sich einige Schritte auf das Ende der Zeltplane zu, die den Kutscheninhalt vor dem Tageslicht beschützte, ihnen nun aber den Blick nach draußen versperrte. Trieest kniff die Augen zusammen und blähte seine Nasenflügel, zog den Duft der Umgebung ein, aber bei diesem dominanten Duft von Met und Wein konnte er kaum etwas erkennen. Da! Eine leise Andeutung von... Blut! Er roch Blut! Blut vermischt mit Lysbetts Geruch.

Trieest zischte Jarid zu, und als sie sich fragend zu ihm umdrehte, gebot er ihr mit hochgezogenen Augenbrauen, stehen zu bleiben. Er hatte einen weiteren Geruch vernommen. Ein Geruch, der ihn bereits sein ganzes Leben lang verfolgt hatte.

Der Geruch nach verfaultem Fleisch.

Skrale!\bigskip







Trieest fürchtete den Geruch von Skralen bereits sein ganzes Leben lang. Die Insel Danwar, die Heimat seiner Kindheit, war schon seit eh und je immer wieder von Skral-Sippen geplagt worden, und die Orden der Wassermagier und Feuerkrieger arbeiteten Hand in Hand, um den Kreaturenangriffen stand zu halten. Doch im Gegensatz zu den Kreaturen des Meeres, wie etwa den grauenvollen Arrogs aus den Heeren der Mächte des Meeres, konnte man die Skrale nicht zurück ins Wasser jagen, denn sie entstammten dem Wasser gar nicht. Stattdessen lebten und verkrochen sich die Skrale in den Höhlen und Gängen, die das poröse Gestein Danwars durchzogen. Sie waren eine Pest, welche sich einfach nicht ausrotten ließ. Es gab zu viele von ihnen und sie wurden rasch mehr, und so waren die Danware froh, dass sie sich abgesehen von gelegentlichen Überfällen auf das Leben in unterirdischen Hohlräumen beschränkten.

Diejenigen Skrale, mit denen die Danware zu kämpfen hatten, waren entfernte Verwandte derjenigen, welche Jarid und Trieest in Andor immer wieder angetroffen hatte. Die Skrale aus Andor trugen aufwändige (wenn auch oft von Zwergen gestohlene) Rüstungen und Schwerter, zogen in Sippen durchs Land, lachten und sangen, wenn sie ein Dorf plünderten und trauerten, wenn einer der ihren von ihnen gegangen war. Die danwarischen Skrale hingegen trugen, wenn überhaupt etwas, dann vor allem Lendenschürze und Holzstöcke als Waffen. Manche hatten eine dunklere Haut und rundere Gesichter – die Andori nannten sie „Nord-Skrale“, weil sie sich hier im Süden rar gemacht hatte. Diese waren oft die Anführer über die „Kreideskrale“, kreideweiße Skrale mit schuppenloser Haut, dafür mit mächtigen Hauern am Unterkiefer ausgestattet, die sie wohl kaum nur zum Brezelstapeln nutzten. Diese Kreideskrale waren mehr Tier als kulturschaffende Wesen, schliefen auf kaltem Stein rund um Steinformationen, die ihre Anführer als Heiligtümer bezeichneten, sprachen selten und ließen ihre Toten gedankenlos zurück.

Die meisten der Nord-Skrale Danwars hatten sich jeweils eine Sippe aus folgsamen Kreideskralen zusammengesucht und diese aus dem Schatten in den Kampf gegen die Nebelinseln geschickt, wo sie Mensch und Vieh gleichermaßen entführten und sich dann in die dunklen Höhlen zurückzogen, um nie mehr gesehen zu werden. Doch manche Nord-Skrale traten hin und wieder aus dem Schatten heraus und kämpften höchstpersönlich an der Seite ihrer Untertanen. Sie waren es, die die Feuerkrieger und Wassermagier stets zu finden und auszulöschen versuchten, wenn sie eine Sippe abwehrten oder eine (leider oft erfolglose) Expedition tief in die unterirdischen Höhlen Danwars anführten. Und diese Nord-Skrale waren leider auch diejenigen, welche ein besonderes Vergnügen daran hatten, in die Dörfer einzudringen, waffenlose Bürgerinnen in eine Ecke zu drängen und ...

Eine Feuerkriegerin hatte das Haus von Trieests Mutter Talemma gerade noch rechtzeitig erreicht, um sie vor einem grausamen Tod retten zu können. Aber nicht rechtzeitig genug, um sie vor etwas anderem zu bewahren.

Neun Monate später war Trieest geboren worden.

Viele Danware hatten ihn für seine reine Existenz gehasst, oder für sein Aussehen, oder für die anderen Müttern, die an den Geburten von Halbskralen verstorben waren und diesen Hass nun auf Trieests Familie lenkten. Viele andere Kinder in der Schule hatten Trieest gefürchtet, erst recht, nachdem sie sahen, dass sich selbst die Kinderhüterin vor ihm fürchte. Doch seine Mutter Talemma zog ihn auf und genoss jede Sekunde davon. Am meisten liebte Trieest es, stundenlang in seinem selbst gebauten Versteck zu sitzen und den Seefahrern zuzusehen. Doch wuchs er rasch zu einem kräftigen Jungen heran, wild und ungestüm. Er biss und kratzte, statt seine Worte zu nutzen. Er verbiss sich in Arme, statt einen dummen Spruch einfach wegzustecken. Und Talemma wusste nicht, wie mit seinen Instinkten umzugehen war. Die Orden der Feuerkrieger nahm sich seiner Ausbildung an. Warum das Ungeheuer nicht dorthin stecken, wo so viele andere wie Ungeheuer aussahen? Doch die Lavasteine machten ihre Träger nicht nur ungeheuerlich, sondern auch einander sehr ähnlich, und der lavasteinlose junge Trieest war dennoch stets anders gewesen, sowohl von den großen Kriegern als auch von den anderen menschlichen Anwärtern. Nicht auf ewig würde das so bleiben, hatte er sich seit seiner Verbannung eingeredet. Wenn ein Feuerkrieger Danwars seinen Lavastein nach einem vollendeten Prozess des Wandels wieder absetzte, kehrte sein Körper oftmals in eine wohlgeformtere menschliche Form zurück, als er es sie vor dem Prozess gehabt hatte. Vielleicht würde auch Trieest eines Tages einen menschlichen Körper besitzen.

Den Geruch eines Skrals hatte Trieest zum allerersten Mal am eigenen Leibe gerochen und gehasst, stets versucht, ihn mit blumigen und bäumigen Gerüchen zu überdecken. Skrale waren nichts mehr als eine Landplage, welche sich am sinnlosen Leid anderer ergötzten. Man sollte die Höhlen Danwars allesamt durchfluten und mit Feuer füllen, ausräuchern und die Höhleneingänge mit schweren Steinen verschließen. Die Skrale waren im Herzen verdorben und durften nicht weiter bestehen.

Ein Glück, dass Trieest keiner von ihnen war.

Er war keiner von ihnen!

Egal, was der Ältestenrat sagte!\bigskip







Trieest verdrängte die Gedanken an seine Vergangenheit mit geübter Effizienz und hob seine Hand mit fünf ausgespreizten Fingern. Den Gerüchen nach befanden sich mindestens fünf Skrale außerhalb der Kutsche. Jarid quittierte die Botschaft mit einem unglücklichen Kopfschütteln. Fünf Skralen hätten sie sich in ihrer Höchstform locker gestellt, doch nicht in diesem Zustand. Wahrscheinlich hatte die Sippe Lysbett, die Weinhändlerin, auf dem Kutschbock überfallen und fragte sich nun, ob es sicher war, die Kutsche zu betreten.

Sie hatten wohl Geräusche aus der Kutsche gehört, wussten aber vielleicht nicht, wie viele Reisende sich da noch befanden. Und wie gut bewaffnet diese Reisenden waren. Ob sie gar über magische Künste verfügten. Es war aus der Sicht der Skrale wohl geschickter, zu warten, bis sich die Insassen der Kutsche zu erkennen gaben.

Jarid nickte, huschte von Trieest weg, hob etwas vom Boden auf und trat dann näher. Der Schlüssel zu seinen Ketten!

„Ahoi, ihr elenden Weinratten! Kommt ihr von selbst aus eurem Schiff oder müssen wir euch mit Gewalt rausholen?“, ertönte plötzlich eine laute, tiefe Stimme von außerhalb der Kutsche. Dieser Skral hatte wohl das Kommando inne. Ein seltsamer Widerhall lag in der Stimme, als würde der Skral in einen großen Eimer hineinreden und sein Echo zurückhallen.

Der geringen Lautstärke nach befand er sich weitaus mehr als nur einige Schritte von der Kutsche entfernt. Vielleicht hatten die Skrale das Gefährt aus der Ferne angegriffen? Dann würden sie sich als noch größere Feiglinge herausstellen, als Trieest sie ohnehin schon gehalten hatte.

„Na kommt schon!“, rief der Kommando-Skral nun, „Wir haben euch umzingelt! Wir versprechen, euch unversehrt ziehen zu lassen, wenn ihr euch ergebt und die Kutsche aufgebt!“

Dann veränderte sich der Tonfall des Kommando-Skrals etwas und er brüllte: „Bögen bereithalten. Schießt auf alles, was diese Kutsche verlässt!“

Trieest grinste Jarid schief an.

„Ein wenig zwiegespalten scheint er schon, dieser Skral-Anführer, nicht?“

Jarid zog verwirrt eine Augenbraue hoch.

„Nein, Calrai, geh dort drüber hin, zur linken Flanke!“, brüllte der Kommando-Skral nun, und eine etwas gelangweilt klingende Stimme antwortete: „Aye, aye, Häuptling Shron.“

Dann bellte Häuptling Shron einige weitere blecherne Befehle. Sechs weitere Stimmen antworteten zustimmend.

„Acht Skrale!“, flüsterte Trieest entsetzt, „Was machen wir nur?“

„Was wir tun müssen“, antwortete Jarid mit einem grimmigen Lächeln. Dann senkte sie einen Schlüssel in die verrostete Kette, die Trieest am Kutschenboden festhielt, und machte sich daran, seine Ketten zu öffnen.

Trieest plante bereits. „Wir könnten versuchen, durch die hintere Seite zu fliehen. Dieser Shron hat kein taktisches Geschick, er hat nur die Front und die Seiten mit Schützen gedeckt.“

Jarid hielt beim Lösen der Ketten inne und sah ihn fragend an: „Woher willst du das wissen?“

Perplex antwortete Trieest: „Er hat es ja laut genug gebrüllt.“

Die beiden starrten einander bewegungslos an. Dann sprach Jarid, „Kannst du die Skrale etwa verstehen?“, während Trieest zur selben Zeit sprach: „Kannst du die Skrale etwa nicht verstehen?“

Jarid antwortete als erste: „Ich verstand noch, dass er uns Weinratten nannte und uns anbot, uns ziehen zu lassen, wenn wir uns ergeben. Danach wechselte er zu einer kehligeren Skral-Sprache.“

Die beiden guckten einander erstaunt an. Dann sagte Trieest „Konntest du etwa all die Skrale, die wir bislang bekämpft haben, nicht verstehen?“ gleichzeitig mit Jarids „Konntest du etwa all die Skrale verstehen, die wir bislang bekämpft haben?“

„Es ist ja nicht so, als hätten wir uns lange mit ihnen über die Moralität ihrer Taten unterhalten“, rechtfertigte sich Trieest, „Dass uns das nicht früher aufgefallen ist, ist dennoch eine Schande.“

„Werden die Kenntnisse der Skral-Sprache etwa irgendwie vererbt? Basieren sie auf Instinkten? Ist ein solches Sprachverständnis überhaupt möglich? Stehen Skrale mit gewissen Entitäten wie Drachen oder dem Schwarzen Herold in unbewusster geistiger Verbindung und tauschen so intuitive Wortverständnisse aus?“, fragte Jarid nun mit großen Augen, „Die Möglichkeiten, die das eröffnete ... die Konsequenzen solcher Theorien ...“ Dann fing sie sich wieder. „Denken wir lieber an die Möglichkeiten, die uns hier und jetzt aus dieser Situation bringen können.“

Trieest nickte und fasste zusammen: „Acht Skrale, bewaffnet mit Bögen, auf allen Seiten umzingelt außer hinten, Lysbett wahrscheinlich gefallen. Zeit, meine Rüstung wieder anzuziehen, haben wir wohl kaum. Wir haben mein Rankenschwert, den Lavastein und deine Magie.“ Dass er eine weitere Gemeinsamkeit mit den Skralen teilte, gefiel ihm ganz und gar nicht. Das war ja gruselig, dass man eine andere Sprache verstehen konnte, ohne zu kapieren, dass es eine andere Sprache war!

„Und du verstehst ihre Sprache“, ergänzte Jarid, „Kannst du sie auch sprechen?“

„Weiß ich doch nicht“, zischte Trieest zurück.

„Ahoi, Weinratten, wie lange wollt ihr noch warten da drinnen?“, rief Shron der Skralhäuptling nun, und diesmal glaubte Trieest zu verstehen, dass er die Sprache der Menschen sprach.

„Vielleicht ist ja keiner drinnen?“, sprach eine leise Stimme, ein weiterer Skral. Ja, das waren wiederum eindeutig nichtmenschliche Laute!

„Vielleicht ist ja keiner drinnen?“, versuchte Trieest leise, die Laute nachzuahmen.

„Und wer hat denn vorhin darin gescheppert, Madenhirn?!“, fluchte Häuptling Shron, „Los, Grobek, vorrücken!“

„‚Vielleicht ist ja keiner drinnen?‘“, wiederholte Trieest die unvertrauten und doch so verständlichen Worte. Dann setzte er nach: „Keiner drinnen? Niemand ist hier. Jemand ist hier. Dort. Überall. Unendlichkeit.“ Stück um Stück tastete er sich vorsichtig in diese fremde und doch so bekannt wirkende Sprache der Skrale vor. Er hatte bereits einmal eine fremde Sprache erlernen müssen, als es sich herausgestellt hatte, dass nur die wenigsten Völker des Südens die Sprache der Danware sprachen. Das Lernen hatte sich als äußerst mühsam herausgestellt. Selbst nach einigen Jahren intensiver Auseinandersetzung mit diesen fremden Kulturen hatte sich Trieests Zunge weiterhin oft an der falschen Stelle in seinem Mund befunden, um die richtigen Laute zu produzieren. Ganz zu schweigen davon, dass er sich nie an die passenden Worte zu erinnern schien.

Das hier war quasi das Gegenteil davon. Je mehr er Worte in dieser seltsamen Sprache der Skrale murmelte, desto mehr Worte kamen ihm in den Sinn. Die Laute und Krächzer, die er seiner Kehle entlockten, waren in keiner Weise gezwungen, sie flossen praktisch direkt aus seinem Hirn in die Luft. Jeder Laut erschien einfach so passend, so richtig. Es ergab alles so viel Sinn. Immer schneller sprach er, seine Gedanken sprangen von Assoziation zu Assoziation, und stets wusste er direkt, welchen Laut er von sich geben müsste, um den Gedanken zu kommunizieren. Die Skrale mochten scheußliche Finsterlinge sein, aber ihre Sprache war ein Wunder! Jarid hatte recht, dieses Phänomen zu erforschen, könnte so viele Erkenntnisse liefern.

Das Geräusch von dumpfen Schritten, die sich der Kutsche von außen näherten, holten ihn in die Realität zurück. Fokus, Trieest! Es gab Wichtigeres. Zunächst einmal mussten Jarid und er unversehrt aus dieser brenzligen Situation entkommen.

Jarid hantierte immer hektischer am Schloss von Trieests Kette herum und fluchte, als es nicht nachgab. Trieest bedachte ihre Chancen. Er war verletzt und rüstungslos, Jarid trug keine Waffen. Aber sie war eine Wassermagierin ... und alle sie umliegenden Fässer waren mit Met und Wein gefüllt! Trieest schöpfte wieder etwas Hoffnung.

Jarid hatte sich panisch umgesehen und schien zum selben Schluss wie Trieest gekommen zu sein, denn nun blitzten ihre Augen fröhlich auf und sie ließ von Trieests Ketten ab, hob ihre Hände an das nächstgelegene Fass und versteifte ihren Körper.

Ein leises Rumpeln ertönte aus dem Fass, aber nichts mehr.

Trieest sah er den glasigen Blick in Jarids Augen und seine Hoffnung sank, ebenso wie Jarid, welche zu Boden sank und sich an die Brust griff. Dort, wo Trieests Rankenschwert sie erwischt hatte. Verletzt, wie sie war, reichte ihre Kapazität nicht einmal aus, um Wein aus einem Fass zu leiten ... wie wollte sie es dann mit diesen Skral-Horden aufnehmen?

Sie hatten eine andere Option. Trieests Kenntnisse in der Skral-Sprache, und die Tatsache, dass er in Ketten lag, eröffneten diese andere Möglichkeit.

„Jarid“, zischte Trieest, und war überrascht, wie kompliziert ihm die Laute erschienen, die er nun in der menschlichen Sprache von sich gab.

„Jarid, wir werden hier rauskommen. Aber nicht durch direkte Konfrontation. Versteck dich hinter den Fässern. Ich werde mich als einen der ihren ausgeben. Vertrau mir.“

Jarid blieb einen kurzen Moment still und bedachte ihre Optionen. Dann nickte sie schwach und zog sich hinter die Fässer zurück. Trieest hörte sie bebend ein- und ausatmen.

Gute Güte. Wie sollte das alles nur enden? Er wappnete sich, ging die entsprechenden Laute im Kopf durch, holte tief Luft und brüllte dann:

„FREUNDE, HELFT MIR! ICH LIEGE IN KETTEN!“

Die Schritte vor der Kutsche verebbten und leises kehliges Gemurmel ertönte, gefolgt von Häuptling Shrons tiefer Stimme: „Welch Schande bist du, der du in einer menschlichen Kutsche reist und doch unsere Sprache sprichst?“

Trieest setzte an: „Ich reise nicht in dieser Kutsche ...“

Dann fiel ihm auf, dass das wohl nicht dem gepflogenen Umgangston zwischen diesen Skralen entsprach, und er setzte aggressiv nach: „Bist du taub oder was? Ich sagte doch bereits, dass ich in Ketten liege, du Irktiolt!“

Fast musste er grinsen. In der Skral-Sprache existierten Schimpfworte, für die es nicht einmal ein menschliches Äquivalent gab.

„Dich werd‘ ich lehren, mich einen Narren zu schimpfen, während du dich hast gefangen nehmen lassen“, grölte Häuptling Shron.

„Mich hat nur die Freude mitgerissen, die Stimmen der Unseren zu vernehmen“, rief Trieest nun, „Wenn du wüsstest, was diese Ziegenköpfe mir angetan haben!“

Stampfende Schritte, dann wurde die Zeltplache zu Trieests Linken zur Seite gerissen. Trieest konnte nur einen kurzen Blick auf den Lysbetts Körper werfen, aus welchem mehrere schwarz gefiederte Pfeile steckten, dann traten krumme Klauen in sein Blickfeld, als ein großer Skral mit einer mächtigen Axt in den Händen die Kutsche betrat. Sein auffälligstes Merkmal war die eiserne Maske, welche über seinen Mund gestülpt war und seine Stimme blechern klingen ließ.

„Hast dir ja lange Zeit gelassen, bis du dich gemeldet hast, Abomination!“, spie er Trieest ins Gesicht, „Mann, bist du hässlich! Was ist denn mit dir geschehen?“

Fast aus der Pistole geschossen antwortete Trieest, während seine Gedanken hektisch kreisten: „Musste mich zunächst von einem Knebel befreien. Dieser elende Stein in meiner Brust! Der hat sich in mich gefressen und meine Züge verändert. Diese kranken Menschen“ – er spuckte aus – „wollten mich zu einem der ihren machen. Testen, ob sich die ‚Verdorbenheit‘ in uns heilen lässt.“

Nervös achtete er auf die Reaktion des hochgewachsenen Skrals. Dieser musterte ihn prüfend. Dann hob der Skral seine mächtige Axt und ließ sie zweimal gezielt auf Trieests Ketten niederfahren. Trieest zuckte zweimal zusammen. Mit einem mächtigen Bersten brachen die Scharniere der Ketten ab. Er war frei!

Langsam, um nicht gefährlich zu erschienen, richtete Trieest sich auf, rieb seine Unterarme und streckte seinen Rücken durch. Erst jetzt fiel ihm auf, wie verspannt er war.

„Trikkest“, stellte er sich vor, und schlug dem hochgewachsenen Skral zur Begrüßung auf den Unterarm. Mit Schrecken stellte er fest, dass der Skralhäuptling dort einen eisernen Schoner mit einer rostigen Klinge daran befestigt hatte. Trieest verfehlte die Klinge nur knapp.

„Shron“, knurrte sein Befreier überrascht. „Aber du kannst mich Häuptling Shron nennen. Du bist nicht von hier.“

Keine Frage, eine Feststellung. Trieest dachte bei sich, dass es am besten wäre, so wenig wie möglich zu lügen.

„Ich komm‘ aus dem Hohen Norden, aus den stinkenden Höhlen der Ratteninsel Danwar. Da schaut unsereins ein wenig anders drein.“

„Ich glaub‘ dir kein Wort. Trägt unsereins im Hohen Norden wallendes Haupthaar?! Hat unsereins dort eine menschliche Anzahl Finger?“, fragte Shron grimmig und schlug Trieest mit dem Griff seiner Axt auf die Hand. Trieest verzog keine Miene und antwortete schlicht „So ist es.“ Woher sollte dieser Häuptling auch wissen, wie Nord-Skrale genau aussahen?

Glücklicherweise beließ Shron es dabei und stieß ihn erneut mit dem Axtstiel an.

„Hast du dich etwa allein von diesem alten Weib mit nur einem Arm gefangen nehmen lassen? Oder sind noch mehr dieser Ratten hier?“

Trieest fluchte innerlich. Falls er nichts sagen würde, und die Skrale Jarid dennoch fänden, wären sie beide geliefert. Falls er hingegen Jarid auslieferte, könnte er vielleicht ...

„Eine verfluchte Hexe reiste mit uns. Sie kann uns aber nicht mehr gefährlich werden, ich habe sie einmal gut in der Brust erwischt. Versteckt sich hinter den Fässern dort drüber, dieses feige Ding.“

Aufmerksam blickte Häuptling Shron auf die Stelle, in die Trieest zeigte. Dann, blitzschnell wie eine angreifende Vypera, stieß er nach vorne und riss die Wand aus Fässern zu Boden. Links und rechts platzte Holz. Met und Wein ergoss sich über den Boden zu einem stinkenden Mischmasch und tropfte durch die Kutsche auf den darunterliegenden Pfad.

Da lag sie, die arme Jarid, durchnässt und zusammengekauert, von ihrem Versteck aufsehend. Sie wagte es nicht, Trieest auch nur anzublicken. Schwach hob sie die Hand zur Verteidigung, flatterte mit ihren Fingern und ein klein wenig Wein erhob sich in runden Tropfen vom Boden. Shron war allerdings schneller, packte Jarid mit seiner freien Hand und zerrte sie unsanft aus der Kutsche. Jarid schrie auf. Trieest folgte mit beschleunigendem Herzschlag und erblickte, wie Jarid am Boden lag und ihren verletzten Torso hielt, während sich ihr Kleid wieder roter verfärbte. Die Lage war kritisch.

Nur einmal hatte Trieest hier in Andor einen anderen Halbskral gesehen. Dieser hatte blaue menschliche Augen gehabt, und schon allein deswegen war er von seiner Horde als Aussätziger behandelt und verstoßen worden. Für Trieest galt das nicht, und nun war er überglücklich darüber. Er versuchte, möglichst skralhaft aufzutreten. Während er die Kutsche verließ, fühlte er tief in den Lavastein hinein und bat ihn, das orange Leuchten seiner Augen verklingen zu lassen, auf dass das milchige Weiß seiner Kreaturenaugen darunter zum Vorschein kommen möge.

Trieest versuchte, nicht auf Lysbetts Leichnam zu achten, welcher immer noch achtlos auf dem Kutschbock lag. Er wusste, dass ihre Seele inzwischen bereits bei der Mutter Natur liegen musste und schwor sich innerlich, dass er bei Gelegenheit zurückkehren und ihren Körper begraben würde.

Nun durfte er allerdings keinen Argwohn auf sich ziehen. Neben Shron umringten sieben ihm unbekannte Skrale die Kutsche (oder bildeten besser gesagt ein Hufeisen um die Vorderseite). Die Pferde waren nirgends zu sehen, die Zügel der Kutsche waren leer. Aber auch diesem seltsamen Umstand konnte Trieest nicht zu viel Aufmerksamkeit schenken. Stattdessen konzentrierte er sich darauf, hoch aufgerichtet und mit bedachten Schritten vorwärtszuschreiten und jedem einzelnen der sieben bewaffneten Skrale überheblich in die Augen zu starren.

Du bist der Mächtigste hier, Trieest. Wenn du wolltest, könntest du jeden von denen im Zweikampf besiegen. Außer vielleicht den Obermacker, diesen Häuptling, Shron. Aber dazu würde es hoffentlich nicht kommen.

Die Skrale tuschelten aufgeregt miteinander und wechselten Blicke untereinander, ehe sie wieder zurück zu Trieest starrten. Fast hätte dieser aufgelacht. Diese Skrale zeigten ganz und gar nicht Verhaltensweisen, die er von blutrünstigen Bestien erwartet hätte. Dann erinnerte er sich an Lysbett und seine Laune sank wieder. Ganz zu schweigen davon, dass Jarid sich immer noch im Griff des Shron befand und von ihm in die Mitte der Skralsippe gezogen wurde.

Ein Glück, dass sie ihm seine Rüstung und Stiefel ausgezogen hatte. In nur seinem roten Fetzenmantel erinnerte er mehr an eine Kreatur. Abgesehen von Shron trugen die übrigen Skrale ebenfalls so gut wie keine Rüstungen, höchstens einige zerbrochene Stücke von Schulterplatten und Brustpanzern, die sie auf ihre Kleidung genäht hatten. Wardraks als Reittiere oder Gors als Lakaien waren ebenfalls keine zu sehen. Dies war keine Skralsippe in ihrer Blüte, sondern ein Haufen verlorener Halbstarker, die stärker auszusehen versuchten, als sie es in Wahrheit waren. Da die meisten von ihnen klobige schwarze Bögen trugen, wären sie dennoch in ihrer Gesamtheit keine leicht zu überwindende Bedrohung.

Ich bin einer von euch, auch wenn ich seltsam aussehe. Ich gehöre zu euch. Kein Grund, misstrauisch zu sein, flüsterte Trieest innerlich. Er spannte seine Gesäßmuskeln an und der knubbelige Stummel dessen, was vor der Amputation einst sein Skralschwanz gewesen war, hob sich in die Höhe und teilte sein rotes Fetzenkleid, auf dass die Skrale der Sippe einen Blick darauf erhaschen konnten. Ein Raunen ging durch die Menge und Trieest hätte erneut beinahe aufgelacht. Er war einfach nicht für derart angespannte Situationen geschaffen.

Shron hatte Jarid indes zu Boden geworfen und war einige Schritte zurückgetreten, während sich die restlichen Skrale um ihn herum versammelten. Er keifte: „Rovuk, du hast heute gut geschossen. Dir gebührt die Ehre, sie ihrem gerechten Ende hinzuzufügen und das erste Blut zu kosten.“

Jarid sah verzweifelt und verwirrt um sich. Die Worte mochte sie nicht verstehen, aber die Geste sprach für sich selbst, als Häuptling Shron einem kleingewachsenen Skral mit stämmigen Armen ehrenvoll ein langes schwarzes Messer überreichte. Abscheu stand in Rovuks viehischen Gesicht, als er vor Jarid.

Etwas musste geschehen, und zwar sofort. Noch ehe Trieest einen vollständigen Plan ausgearbeitet hatte, trat er vor den kleingewachsenen Rovuk und bellte: „Nein, wir brauchen sie noch! Sie ist die Einzige, die mich von diesem verfluchten Stein in meiner Brust befreien kann!“

Dann verfluchte er sich selbst, als ihm auffiel, dass er zwei unnötige Fehler begangen hatte: Zum einen konnte es den Skralen egal sein, wenn er weiterhin seinen Lavastein trug. Zum anderen hatte er sich nicht an den Häuptling gewandt. Und wenn er etwas von den hiesigen Skralen gelernt hatte, dann, dass sie es ganz und gar nicht mochten, wenn ihre Autorität in Frage gestellt war.

Rovuk nahm es gelassen und blickte nur fragend zu Shron, deutlich zeigend, dass er jedem Befehl seines Häuptlings Folge leisten würde.

Shron nahm es ganz und gar nicht gelassen und schlug Trieest grob auf die Schultern.

„‘\textit{Wir} brauchen sie noch‘?“, ahmte er Trieests Stimme nach, „Denkst du etwa, du bist einer von uns? Denkst du etwa, irgendeiner von uns schert sich auch nur einen feuchten Drachenkot darum, ob du lebst oder stirbst? Ich mag dich befreit haben, weil du Skralblut in dir trägst, aber wenn du mir noch einmal auf die Nerven gehst, werden ich dich höchstpersönlich der großen Echse im Himmel zuführen, \textit{Hallenlaskrala}!“

Kurz überlegte Trieest, ob es geschickter wäre, seinen Status als Halbskral zuzugeben. Immerhin kamen Halbskrale den Skralen derart unrein vor, dass sie Begegnungen mit solchen krankhaft mieden, ja, gar ihr Fleisch nie konsumieren würden, während sie bei anderen Skralen und bei Menschen weitaus weniger Skrupel davor hatten. Als Halbskral konnte er den Skralen derart eklig erscheinen, dass sie ihn vielleicht ziehen lassen würden. Andererseits war da immer noch Jarid, und als Halbskral hatte er keine Chance, den Respekt der restlichen Truppe zu erlangen und für ihre Freiheit zu verhandeln.

So wehrte er sich: „Bist du vergesslich oder einfach blöd? Ich bin kein Halbskral, ich komm‘ nur aus dem Norden, ignorantes Blechkinn!“.

Als Trieest ausspuckte, musste er seine Abscheu vor dem Wort „Halbskral“ nicht verstellen. Er sprach weiter: „Und ich sagte doch bereits, dass es dieser feurige Edelstein in meiner Brust ist, der mich ...“

„Sei still, du Wurm, und überlasse die Entscheidungen mir! Es gibt bestimmt noch andere auf dieser Welt, die dir mit diesem Stein-Problem aushelfen können. Die weisen Hexen Drunn und Trumm hätten dieses Steinchen bestimmt im Nu aus dir draußen. Man könnte fast meinen, dass diese Menschenfrau dir etwas bedeutet.“

Trieest bewegte sich immer noch nicht zur Seite und versperrte Shron so weiterhin den Weg zu Jarid. „Sie bedeutet mir nur so lange etwas, wie sie mir noch nützlich ist. Doch könnte sie auch euch noch nützlich sein, denn sie vermag, Formen von Flüssigkeiten zu wandeln und den Weg im tiefsten Dunkel zu finden!“

Trieest wusste, dass er zu spät war. Shron konnte seine Meinung nicht mehr ändern, ohne vor seiner Sippe schwach zu erscheinen, und Trieest wusste aus Erfahrung, dass es in den Augen der Skrale kaum eine größere Schande gab. Darum wollte Shron nun nichts mehr, als rasch seine Autorität durchzusetzen. Aber Trieest konnte nicht nachgeben. Jarids Leben stand auf dem Spiel. Und so blieb er tapfer stehen.

Shron erteilte ihm eine Ohrfeige, die ein unangenehmes Klingeln in Trieests Kopf verursachte. Danach blieb er vollkommen still und starrte Trieest provokativ in die weißen Augen, ohne auch nur einmal zu blinzeln. Leise, jedes Wort einzeln kostend, ehe er es ausspuckte, sprach er: „Willst du mich etwa herausfordern, Strohhirn?“

Trieest überlegte kurz.

Er traf eine Entscheidung.

„Ja“, sagte er dann, „Ich fordere dich zu einem Zweikampf um das Leben der Menschenfrau. Ich brauche sie zumindest so lange, bis der Stein aus meiner Brust ist.“

Er hätte nicht gedacht, je so etwas wie Überraschung in den kantigen Zügen des Häuptlings sehen zu können, aber so war es. Shron riss seine Augen auf und verzog sein Gesicht. Seine eiserne Maske rutschte nach unten und enthüllte Teile eines entstellten, echsenhaften Mauls mit einem schiefen Grinsen, aus welchem spitze Zähne in alle Himmelsrichtungen ragten. Unwillkürlich glitt Trieests Zunge über seine eigenen Beißwerkzeuge, welche im Gegensatz zu denen des Skrals glatt und menschlich aussahen.

Shron liess ein gutturales Lachen aus seiner Kehle entschlüpfen, während er seine Maske richtete und japste: „Mumm magst du ja haben, aber das Hirn eines Spatzen! Du kannst nicht einfach so deine eigenen Bedingungen stellen. Ich mag dich nicht. Bei diesem Zweikampf geht es um alles oder nichts! Wenn wir zwei uns kloppen, bleibt nur einer übrig, und dieser wird der Häuptling dieser mickrigen Sippe sein. Und der mag dann bestimmen, was mit der Menschenfrau geschieht. Und ich werde bestimmen, dass Rovuk sie schlachtet, wie es sich gehört!“

Kämpfe auf Leben und Tod hatte Trieest schon oft mit Kreaturen und Finsterlingen geführt, erst gerade kürzlich mit diesem verdorbenen Wesen, dem Hraak. Das hier war im Grunde genommen nichts anderes, redete er sich ein. Er mochte damit nicht recht haben, schließlich war das bislang der erste Kampf, den er und sein Kontrahent unter kontrollierten Bedingungen bestreiten würden. Aber das zählte nicht. Shron war genauso ein Monster wie alle anderen Skrale, denen sich Trieest bislang gestellt hatte, und die Welt würde ohne ihn ein besserer Ort sein. Er fasste sich.

„Wie du willst, Shron. Dann fordere ich dich zu einem Zweikampf um die Führung der Sippe heraus.“\bigskip







\textit{Nicht allzu weit entfernt regte sich ein vermeintlicher Leichnam, den ein Skral zur späteren Konsumierung achtlos beiseite geworfen hatte. Eine alte Hand griff nach einem schwarzen Pfeil und zog ihn aus einer Brust. Das Böse beobachtete fasziniert das rote Blut, das daraus hervorfloß.}

\textit{Lysbetts Körper richtete sich auf und blickte sich vorsichtig um. Keiner der Skrale achtete auf sie, alle starrten nur auf Trieest und diesen Häuptling mit der eisernen Maske.}

\textit{Das Böse warf sich ins nächste Gebüsch und stolperte von dannen, weg von diesen Kannibalen. Lysbetts Körper würde nicht lange hinhalten, es brauchte einen besseren!}

\textit{Eines von Lysbetts Beinen knickte ein und das Böse verschluckte sich beinahe am tiefschwarzen Kristallsplitter, den Lysbett unter ihrer Zunge versteckt hielt, in ständigem Kontakt mit ihrem Körper. Was für ein Glück es doch gewesen war, dass Lysbett diesen Splitter als ersten berührt hatte, ehe Jarid ihn erblickt hatte. Diese wäre nicht so nachlässig gewesen.}

\textit{Das Böse überdachte seine Situation kurz und spuckte den Splitter dann aus, in Lysbetts Hand hinein. Dort würde es ihn fortan festhalten, bis es ein besseres Gefäß für seine Macht fand. Am besten eines mit magischer Kapazität, aber aktuell war es alles andere als wählerisch.}

\textit{Eine Erinnerung an die Konfrontation mit Trieest und Jarid schoss durch den Kopf des Bösen. Als der Lavastein aufgeglüht hatte, hatte das Böse eine Stimme vernommen, eine ihm wohlbekannte Stimme, deren Träger doch schon lange tot sein sollte. Diesem Mysterium hatte es nachgehen wollen, sobald es mit seiner Kutsche zum Geheimversteck der Druidin gelangt wäre und die beiden Danware dort eingesperrt hätte. Nun würde es ihm halt allein nachgehen. Es brauchte Trieest nicht. Es hatte in seinen und Jarids Erinnerungen so viel mehr gesehen.}

\textit{Es hatte wieder ein Ziel, zum ersten Mal in so langer Zeit in seinem langen, nutzlosen Leben. Es hatte ein Ziel, und es wusste, wie es dorthin kommen konnte.}

\textit{So stahl sich das Böse in Richtung Norden davon, auf zur Küste.}










\newpage
\section{Eisenmaske gegen Lavastein}



\az{Jahr 61}

\textit{Halle des Ältestenrats, 61 a.Z.}\bigskip



Rowinda beugte sich vor und senkte ihre Stimme, als sie zu ihrer Tochter sprach: „Ich werde offen mit dir sein, Jarid. Der Orden der Feuerkrieger hat sich bereits gegen Trieest ausgesprochen. Sie wollen ihn nicht weiter ausbilden. Die Häuser des ...“

„Aber Mutter, seine Ausbildung hat doch kaum erst begonnen! Und für eine andere Profession ist ihm ...“

„Schweig, Jarid, bitte schweig und höre mir zu. Die Häuser des Aufenthalts haben sich gegen ihn ausgesprochen. Sie wollen ihn nicht mehr beherbergen. Die halbe Insel will nichts mehr mit ihm zu tun haben. Vielleicht ist es für ihn das Beste, andernorts einen Neuanfang zu wagen. Hier blüht ihm keine große Zukunft. Nicht so, wie er ist. Nicht nach dem, was geschehen ist.“

„Das ist doch die Höhe!“

„Ja, das ist wirklich die Höhe“, meldete sich nun die aufmüpfige Freiga zu Wort, „Jormudds halbes Gesicht hat er zerfleischt, und nun soll der Täter als Belohnung dafür eine Reise in die weite Welt geschenkt kriegen!“

„Freiga, wenn du noch einmal außerordentlich dein Wort erhebst, verweise ich dich des Saals! Ich verstehe deinen Ärger, wir alle verstehen ihn, aber das Protokoll muss gewahrt werden, sonst versinkt dieser Rat im Chaos.“\bigskip







\az{Jahr 68}

\textit{Sieben Jahre später.}\bigskip




Der Reaktionen der Skrale waren äußerst interessant zu beobachten gewesen. Die meisten hatten eine Mischung aus Abscheu und Neugierde gezeigt, als Trieest aus der Kutsche getreten war. Manche hatten gar gegähnt, als wären sie es nicht gewohnt, bei Tageslicht wach zu sein.

Als Trieest sich schützend vor Jarid gestellt hatte, hatte die Abscheu in den Gesichtern der restlichen Skrale Überhand gewonnen und nebst wütendem Schwanzwedeln hatte man den einen oder anderen Skral ausspucken sehen können. Nun, als Trieest sich gegen Shron stellte und diesen zum Zweikampf herausforderte, schienen die Meinungen auseinanderzulaufen. Einige Skrale, darunter dieser kleinere Skral mit den stämmigen Armen, welcher immer noch mit dem Opfermesser dastand ... ja, genau, Rovuk hieß er. Also eben, einige Skrale, darunter Rovuk, hatten ihre Münder zusammengekniffen, ihre Augen zu Schlitzen verengt und blickten nun schwanzwedelnd und grimmig in die Runde. Wie konnte diese dahergelaufene halbe Portion es wagen zu glauben, auch nur eine Chance gegen ihren mächtigen Häuptling zu haben?!

Einige andere Skrale schienen aber entspannter, fast neugierig. Trieest glaubte sogar, aus dem Augenwinkel ein hastiges Lächeln auf einem echsenhaften roten Gesicht aufblitzen zu sehen. Trieest musterte diesen Skral. Er trug grobes wollenes Wams, wie es die Fischer in Andor trugen, doch reichte dies bei weitem nicht, um seinen massigen Körper zu bedecken. Er schien kaum Rüstungsteile gefunden hatte. Dieser verfilzte Bart ... wie es wohl anfühlte, diesen zu kraulen?

Rasch senkte Trieest seinen Blick von diesem unbekannten Skral und musterte die beiden anderen Skrale, die neben ihm standen. Auch sie trugen weniger Rüstungsteile und praktisch leere Köcher, welche noch dazu um einiges ramponierter waren als den Köcher, die Rovuk trug.

Dies waren die Skrale mit der schlechtesten Ausrüstung, selbst in dieser mickrigen Sippe. Die Unterlinge? Vielleicht mochten sie es unter Häuptling Shron nicht? Vielleicht hofften sie ja auf einen Führungswechsel? Auf jeden Fall schienen sie keinesfalls empört darüber, dass sich jemand Shron im Kampf stellte. Aber belustigt wirkten sie auch nicht. Einfach nur ... entspannt.

Jarid war indes ganz und gar nicht entspannt, sondern blickte weiterhin furchterfüllt und still von einem Skral zum nächsten, ohne ein Wort davon zu verstehen, was gesprochen wurde. Trieest konnte nicht erkennen, wie viel dieser Panik gespielt war. Es war klar, dass diese Angelegenheit rasch beendet werden musste.

„Wo findet der Kampf statt?“, fragte er Häuptling Shron.

„Nicht so hastig, Todessüchtiger“, lachte der kleine Rovuk, „Ein Ungeheuer wie du hat gar kein Recht, einen so ehrwürdigen Häuptling wie Shron herauszufordern! Wie viele Menschen willst du schon ermordet haben?! Shron trägt derer neune an seinem Namen! Er entführte einst die höchste Schamanin des Yetohe-Stammes aus ihrer angeblich sicheren Zeltstadt, er war an der Ermordung des Diebeskönigs Brandur und an der Belagerung der Rietburg beteiligt, er sammelte nach dem Tod des großen Drachen die führungslosen Kreaturen unter seinem Heerbanner, ja, es gelang ihm gar in zahlreichen Gelegenheiten, den gefürchteten Helden von Andor zu entkom...“

Shrons Schwanz watschte gegen Rovuk und gebot ihm, das Sprechen sein zu lassen. Offenbar war Shron nicht allzu gut auf seine früheren Fluchten vor den Helden zu sprechen. Dies war also einer der Mörder von Brandur, der die Befreiung der Rietburg überlebt hatte. Vielleicht war Trieest ihm sogar schon einmal über den Weg gelaufen. Und er hatte glorreiche neun Menschen ermordet. Trieest lachte beinahe auf. Falls er heute das Zeitliche segnen würde, würde er ihnen mit seinem letzten Atemzug mitteilen, wie viele Mitglieder ihrer Spezies er bereits auf dem Gewissen hatte, das schwor er insgeheim. Dann würden sie große Augen machen.

Wie es sich herausstellte, sollte er die Gelegenheit dazu kriegen. Rovuk mochte einem Zweikampf zwischen Shron und Trieest gegenüber negativ eingestellt sein, aber Shron selbst war es nicht.

„Sei mal nicht so, Rovuk“, wies er den kleinen Skral zurecht, „Wenn unser Verfluchter hier einen Kampf will, so kann er einen kriegen. Ich bin mir sicher, dass er während des Duells zeigen kann, wie wenig Menschlichkeit tatsächlich in ihm steckt.“ Die letzten Worte hatte er spöttisch ausgespuckt. „Natürlich, da dies ein Kampf zwischen zwei vollwertigen Skralen sein soll, werden wir uns waffenlos gegenübertreten. Es sei denn, dieser verfluchte Stein in deiner Brust hat dich bereits so sehr vermenscht, dass du dich nicht mehr auf deine Fäuste, Fänge und Klauen verlassen kannst!“

Trieest teilte den versammelten Skralen laut mit, dass er sich natürlich sehr wohl auf seine Fäuste, Fänge und Klauen verlassen konnte. Im Innern sank seine Hoffnung. Im waffenlosen Training war er Jarid oft unterlegen. Und Jarid hatte selbst nicht zu den geschicktesten Kämpfern des Wassermagierorrdens gezählt.

Allerdings hatte er sich auch immer Mühe geben müssen, Jarid nicht zu verletzen. Als kleines Kind hatte Trieest zu oft die Kontrolle verloren. Der Vorfall mit Jormudd hatte ihn gar erst in seine jetzige Misere gebracht: Mit einem Lavastein in der Brust und einem nie enden wollen scheinenden Prozess des Wandels in der Zukunft. Vielleicht könnte Trieest es mit Shron aufnehmen, wenn er sich nicht zurückhielt? Konnte Trieest überhaupt wirklich loslassen und sich nicht zurückhalten?

Häuptling Shron bellte einige Befehle. Ein Skral führte zwei Pferde hinter der Kutsche hervor, vermutlich hatten jene die Kutsche gezogen. Dann gab es eine kurzzeitige Aufregung, als Shron den Transport „dieser alten Weinratte“ verlangte, und plötzlich niemand mehr den Leichnam der Weinhändlerin Lysbett auffinden konnte.

Zu guter Letzt befahl Shron Trieest, Jarid zu schultern und ihm zu folgen. Vielleicht wollte er ihn dadurch bereits vor dem Kampf etwas schwächen, aber Trieest war es einerlei. Er hatte es viel lieber, Jarid selbst zu tragen, als dass einer dieser anderen Skrale sich ihrer annahm. Sorgsam darauf achtend, sich nicht anmerken zu lassen, wie viel Jarid ihm bedeutete, folgte Trieest dem Häuptling. Er drückte Jarids Hand sanft und hoffte, dass sie das als Zeichen der Zuversicht interpretieren könnte. Jarid regte sich nur schwach auf seiner Schulter.

Die übrigen Skrale tuschelten wieder miteinander und umkreisten Trieest sorgsam, auf dass er nicht entkommen konnte. Eine Flucht wäre vergeblich, selbst wenn Jarid ihn nicht verlangsamte.

Die Truppe erreichte den Lagerplatz der Skrale, eine moosige Lichtung mitten im Wachsamen Wald. Ein kleines Lagerfeuer stand neben einem behelfsmäßig errichteten Zelt am Rande. Ein ebenso behelfsmäßiges Banner steckte vor dem Zelt, mehr nur ein roter Fetzen an einer Stange als die glorreiche Flagge, die das Banner zu sein versuchte. Widerwillig war Trieest fasziniert. Er hatte noch nie einen Skral-Unterschlupf mit eigenen Augen gesehen. Jarid ruckelte auf Trieests Schulter. Versuchte sie ebenfalls, sich einen Überblick über das Lager zu verschaffen?

Skrale stellten nur wenig Kleidung, Waffen oder andere Materialien selbst her. Stattdessen verließen sie sich größtenteils darauf, von Menschen und Zwergen zu klauen und wiederzuverwerten. So war das Zelt aus nicht zusammenpassenden Stoffplachen und vereinzelten Brettern aufgebaut. Es war zu klein, als dass die gesamte Skralsippe darin Platz finden könnte. Shron wirkte nicht wie einer, den es störte, wenn seine Untergebenen draußen in der Kälte nächtigten, solange er selbst ein warmes Zeltinneres hatte. Einige Blätterhaufen und weitere Stofffetzen lagen rund um eine angeschlagene hölzerne Kiste, aus welcher weitere Pfeile und Schwerter ragten. Wahrscheinlich fanden die Unterlinge in der Rangordnung der Sippe dort Unterkunft für die Nacht.

Zelt und Blätterhaufen lagen inmitten einer Kuhle im sonst erhöhten Gebiet. Das war keine optimale Lage: falls es regnen würde, würde die gesamte Ausrüstung durchnässt werden und der Regen sich in dieser Kuhle sammeln.

Wie um Trieests Gedanken zu bestätigen, ertönte am Himmel ein leises Rumpeln und ein Nieselregen setzte ein. Na großartig, dachte Trieest. Es musste ja so kommen.

Sorgfältig setzte Trieest Jarid auf der feuchten Erde ab. Ein letztes Mal noch blickte er ihr zuversichtlich in die Augen. Ihre starrten stumpf und glasig zurück. Rasch löste er sich von ihr und wandte sich abrupt ab.

Irrte er sich oder hatte Jarid ihm noch ein leises „Tri“ zugeflüstert? Er musste diesen Kampf, seine Herausforderung gegenüber Häuptling Shron, so schnell wie möglich hinter sich bringen und sich danach um die verletzte Jarid kümmern.

Als er sich umdrehte, hatte Häuptling Shron sich bereits seiner Schulterplatten und Armklinge entledigt. Abgesehen von seiner eisernen Maske mit nichts als einem Lendenschurz bekleidet, schritt der Häuptling stolz auf Trieest zu. Dieser wäre beinahe peinlich berührt zurückgewichen, überlegte es sich dann allerdings anders und zog sein eigenes Wams über seinen Kopf, um Shron ebenso ungeschützt und scheinbar ebenso unbekümmert gegenüberzutreten.

Aus dem Augenwinkel bekam er mit, wie einer der Skrale Jarid fesselte und ins rissige Zelt bugsierte. Zu seiner Erleichterung verließ der Skral das Zelt kurz danach allerdings wieder. Offenbar wollte keiner der Skrale das bevorstehende Spektakel verpassen.

Betont ruhig stand Trieest vor Shron und betrachtete sein Gegenüber, ohne ihn wirklich zu erkennen. In seinem Kopf ging er die Meditationsübungen durch, die ihm die Danware während seiner Ausbildung und später Jarid immer und immer wieder vorzeigt hatten. Einatmen. Ausatmen. Einatmen. Ausatmen. Seine Nervosität in den Griff kriegen. Das Feuer der Furcht – Drenlynn nannten die Feuerkrieger es – würde noch früh genug durch seine Adern strömen. Er musste sich nur auf das hier und jetzt konzentrieren, seinen Körper in den Griff kriegen, eins mit sich selbst und dem Lavastein sein. Dann würde alles so kommen, wie es kommen musste. Er hatte dies im Griff. Er hatte schon so oft Zweikämpfe mit Skralen gewonnen. Er wusste, wie sein Gegenüber dachte. Die vielen Male, in denen er Jarid gegenüber in Trainings-Zweikämpfen unterlegen gewesen war, die würden sich nun auszahlen. Und er konnte auf die Stärke des Lavasteins zurückgreifen, auch wenn er später in Schmerz dafür bezahlen würde. Trieest konnte diesen Kampf gewinnen.

Häuptling Shron schien keinen solch beruhigenden Gedanken nachzuhängen. Stattdessen schritt er unruhig auf und ab, schlug in die Luft und peitschte mit seiner Schwanzspitze, während einer seiner Lakaien ein wenig Gerümpel zur Seite rollte.

„Ach, es ist zu lange her, dass ich einen angemessenen Kampf hatte“, lamentierte Shron. „Diese elenden Helden von Andor haben mich meinen Vater gekostet, mein Ansehen, meinen Rang als Häuptling einer ehrenwerten Skral-Sippe! Es vergeht kein Tag, an dem meine Wut auf sie mich nicht anführt. Mit dieser Wut konnte ich bislang jeden einzelnen meiner Gegner bezwingen. Du hast keine Chance!“

Trieest zweifelte an der Effizienz von Shrons Wut, dem schrecklichen Zustand vom kläglichen Rest seiner Sippe nach zu urteilen. Er war aber äußerst froh darüber, dass Jarid und er während der Befreiung der Rietburg nach noch nicht so öffentlich als Helden von Andor aufgetreten waren. Nicht auszudenken, wie Shron reagiert hätte, hätte er sie erkannt.

„Es gibt keinen Ring und keine Regeln“, flüsterte Shron heiser, „Wenn dieser Kampf einmal begonnen hat, endet er nicht, ehe einer von uns mit den Drachen fliegt. Du magst um dein Leben betteln, doch von mir wird es keine Gnade geben.“

„Gut so“, knurrte Trieest. Er hatte keine Lust, Shron am Leben zu lassen. Trieest war kein Mörder, aber er hatte er sich schon vieler Kreaturen entledigt, Maschinen des Todes, die sie waren, und die Welt war dadurch ein besserer Ort geworden. Bei Shron würde das genauso sein.

Trieest konzentrierte sich auf den Lavastein und bat um Unterstützung. Er atmete erleichtert auf, als er die vertrauten Gefühle in sich aufsteigen fühlte. Der Stein erwärmte sich trotz des dicken Spalts auf seiner Oberfläche und begann zu schimmern. Trieests Körpertemperatur stieg zunächst kaum merklich, dann immer stärker an. Der Nieselregen tropfte nicht mehr an ihm herunter, sondern verdampfte zischend von seiner Haut. Trieests Sichtfeld verfärbte sich feuerrot, als das vertraute Glühen in seinen Augen wieder einsetzte. Das waren bislang rein kosmetische Veränderungen, die auf seine Kampffertigkeiten keinen Einfluss hatten, aber vielleicht konnten sie Shron einschüchtern.

Von irgendwoher ertönte ein blecherner Gong. Trieests Blick zuckte zur Quelle und enthüllte, dass es vielmehr das Geräusch eines gezackten Schwerts auf einem blechernen Schild gewesen war.

Das war wohl das Startsignal für den Zweikampf gewesen, denn Shron rannte ohne Verzögerung auf Trieest los und hatte seinen krallenbesetzen Fuß bereits zweimal zwischen Trieests Beine gepflanzt, ehe Trieest überhaupt reagieren konnte. Dann drehte er sich gewandt um und wischte Trieests Beine nonchalant mit seinem Schwanz zur Seite. Trieest ging zu Boden, schmeckte Matsch und konnte ein schmerzerfülltes Stöhnen nicht unterdrücken.

Shron drehte sich zurück und trat unerbittlich auf Trieest ein. Trieest spürte leichte Kratzer und zukünftige blaue Flecken, aber nichts Schlimmeres. Das Feuer der Furcht floss durch ihn und schwächte sämtliche Schmerzempfindungen ab.

Ehe Shron ihm Schlimmeres zufügen konnte, griff Trieest seinerseits an. Shron stand über ihm, folglich musste Trieest ihn zunächst zu Fall bringen. Blitzschnell drehte Trieest sich aus seiner Embryonalstellung auf den Rücken und trat nach Shron, nicht ungezielt, sondern an eine spezifische Stelle über der Ferse, die Jarid ihm einst bei Menschen gezeigt hatte.

Shrons rechtes Bein knickte ein und Trieest schwang sich hoch, traf Shron mit der flachen Hand an der Brust, ergriff Shrons Schulter und schleuderte ihn zu Boden, während er sich selbst in die Höhe stemmte. Er heizte seinen Lavastein an. Bald könnte er auf dessen Fähigkeiten angewiesen sein, doch wollte er sich nicht zu rasch ausbrennen.

Shron mochte keine großartige Taktik haben, doch seine reine Muskelkraft und Robustheit konnten diesen Nachteil definitiv wettmachen. Andere wären nach Trieests Tritt nicht wieder aufgestanden, doch Shron knurrte bloß kurz und warf sich wieder in die Höhe. Dann verharrte er.

Trieest hielt ebenfalls inne.

Er war es gewohnt, dass Skrale sich ihm beinahe willenlos entgegenwarfen, praktisch leere Hüllen für die finstere Macht, die sie antrieb, egal ob die Skrale nun gerade Werkzeuge für Drachen, Dunkle Magier, Finstere Herolde oder Nekromanten waren. Das waren Skrale gewesen, die selbst nach gröbsten Verletzungen nicht aufgaben, sich stattdessen wieder erhoben und bis zum letzten Tropfen schwarzen Bluts in ihren Adern weiterstritten.

Shron war nicht so. Stattdessen blieb er stehen und starrte Trieest unruhig an. Er musterte Trieests Brust mit dem leuchtenden Lavastein. Auch für Shron musste dieser Kampf eine neue Erfahrung sein. Er hatte wohl geplant, Trieest rasch und brutal zu erledigen. Stattdessen war Trieest standhaft geblieben (so standhaft, wie man sich nennen durfte, wenn man am Boden lag) und hatte sich zu wehren gewusst. Und dann erst das Bild, das Trieest abgab: Orange leuchtende Augen, gebleckte Zähne, zischend an seiner Haut verdampfende Regentropfen – und leichte orangerote Feuerschlieren, die seinen den Edelstein in seiner Brust umgaben. Trieest sah zum Fürchten aus und immer noch so fit wie zu Beginn des Kampfes.

Shrons Augen verengten sich, und Trieest konnte für einen Herzschlag durch seine Fassade hindurchsehen.

Der Häuptling hatte Todesangst. Er wusste nicht, was Trieest war, und wozu er fähig war. Er wollte nur weg von hier. Aber er konnte nicht. Nicht, ohne sich eine Blöße zu geben.

Eine Erinnerung flackerte durch Trieests Geist, eine, die er schon viel zu lange unterdrückt hatte. Der Blick des kleinen Jormudd aus Danwar, als der noch viel zu kleine Trieest sich auf ihn gestürzt hatte. Der schiere Schrecken in Jormudds Augen würde Trieest nie vergessen können. Entsetzen, Furcht, Verwirrtheit, als wäre Trieest ein wildes Tier gewesen und kein Mensch, kein denkendes, fühlendes Wesen. Und diesen Blick erkannte Trieest nun in Shron wieder.

In diesem Moment erkannte Trieest, dass sich hinter dieser dunklen Kreatur vor ihm kein tumbes Vieh, sondern eine denkende, fühlende Person verbarg. In diesem Augenblick hatte Trieest Mitleid mit Shron. Mit ihm, einem elenden Skral!

Dann war der Augenblick vorbei. Shron brüllte blechern auf, warf sich Trieest entgegen und bohrte ihm seine Klauen in die Seite. Trieest schrie ebenfalls auf, ergriff Shrons mächtigen Rumpf und packte ihn an einem bestimmten Bereich in seinem Rücken, wo er bei einem Menschen gehörigen Schaden anrichten hätte können. Bei Shron hingegen traf er nur auf steinharte Schuppen.

Trieest konnte immer noch die Todesfurcht in Shrons Augen erkennen, doch dies hielt den Häuptling nicht davon ab, immer und immer wieder auf Trieests Brust einzuschlagen, während Halbskral und Skral gemeinsam im Matsch umherrollten.

Zunächst konnte Trieest keinen Sinn hinter Shrons Verhalten sehen, doch dann erkannte er, was der Häuptling vorhatte. Er hatte den Lavastein als Quelle von Trieests Kraft identifiziert und versuchte, ihn zu zerstören.

Viel Glück dabei, dachte Trieest spöttisch, das habe ich in sieben Jahren nicht geschafft.

In diesem Moment knirschte etwas unter Shrons blutigen Knöcheln und der Spalt auf dem Lavastein vergrößerte sich. Trieest fühlte Panik vom Lavastein hinüberschwappen.

Ausgerechnet jetzt bröckelte der Lavastein?! Hatte dies damit zu tun, dass er bereits vorhin vom bösen Hraak angesplittert worden war? Konnte er diesen Kampf noch überstehen?

Shron knurrte erleichtert und grub seine Krallen in Trieest Brust, versuchte mit aller Kraft, den Lavastein zu fassen zu kriegen. Trieest keuchte auf. Sein Arm wurde glühend heiß, schoss hoch und bohrte sich tief ihn Shrons Brust, ehe Trieest den Befehl dazu gegeben hatte.

Zwei, drei, vier Ringe aus gleißenden Feuerschleiern umringten Trieest und brannten sich in ihn und seinen Gegner zugleich.

Wie Trieest und Shron da standen, die Krallen jeweils in die Brust des Gegenübers gegraben, sang der Lavastein in Trieests Brust auf einmal glockenhell. Trieest blickte an sich herunter. Die Spalten im Stein glühten auf, zuerst langsam, dann heller und heller. Shron brüllte ein letztes Mal auf, der Ruf eines Skrals, der meilenweit zu hören sein musste und Trieests sensible Ohren unangenehm dröhnen ließ.

Shrons Brüllen wurde allerdings abrupt abgerissen, als \textit{die ganze Welt in orange Töne getaucht wurde und der Boden unter Trieests nackten Füssen zur Seite kippte. Ins Nichts.}

\textit{Dann war Trieest allein.}

\textit{Allein in einer feuerroten Welt.}\bigskip







\textit{Trieest schwebte in einem Wirbel aus gelben, orangen und roten Flammen, die miteinander tanzten. Es gab keinen Boden unter seinen Füßen und keinen Himmel über seinem Kopf. Wärme umspielte seinen nackten Körper, doch fühlte er trotz der lodernden Flammen um ihn herum keinen Schmerz.}

\textit{Mehr Feuer als Luft gab es hier, wie Trieest auffiel, als er panisch versuchte, einzuatmen. Erst nach einigen verzweifelten Japsern kam er zum Schluss, dass Atemluft an diesem Ort offenbar nicht benötigt war, um zu überleben. Bei Kenvilars Dreizack, wo befand er sich?}

\textit{Überall um ihn herum flackerten helle Flammen ohne Quelle und ohne Rauch. Manche leckten an seinem blutigen Körper, doch fühlten sie sich kaum heißer an als warme Sonnenstrahlen an einem Sommertag.}

\textit{Ein unförmiges Gesicht schälte sich aus den Flammen vor Trieests Augen, nichts weiter als ein weißes Grinsen mit strahlend weißen Augen, die Skizze einer Miene vor leuchtenden Flammen.}

\textit{Das Gesicht sprach mit einer tiefen, dröhnenden Stimme: „Sieben Jahre habe ich dich nun schon begleitet, Trieest, Feuerkrieger aus Danwar. Doch jetzt ist es an der Zeit für mich, dich zu verlassen.“}

\textit{„Mein Lavastein? Bist du das?“, fragte Trieest verblüfft.}

\textit{„Natürlich! Du hast mich doch oft genug gefühlt, um mich zu erkennen.“}

\textit{„Du kannst sprechen?!“, rief Trieest empört aus.}

\textit{„Ja, ja, ich kann sprechen, o Wunder“, rief das flackernde Gesicht des Lavasteins aus. „Mir bleibt nicht viel Zeit, denn ich muss gehen. Ich ... ich wollte mich bloß noch einmal melden und sichergehen, dass ich mit dir im Reinen bin, ehe ich diese Welt verlasse.“}

\textit{Trieest ließ die Worte auf sich einwirken. Dann beschloss er, sich auf die absurde Situation einzulassen. Zornig sprach er: „Jetzt hältst du es für die passende Gelegenheit, mich um Vergebung zu bitten?“}

\textit{„Ich gebe zu, dass dies unpraktisch ist. Es geht nicht anders. Bist du etwa froh, dass ich dich jetzt verlasse, Trieest?“}

\textit{Trieest blieb kalt: „Du hast mich jahrelang gequält. Meine Schuld sollte schon längst beglichen sein. Mein Prozess des Wandels hätte schon längst abgeschlossen sein sollen. Doch du weigertest dich, mich gehen zu lassen. Und zu einem vollwertigen Menschen machtest du mich auch nicht. Du wirst deine Gründe dafür haben. Dennoch hast du falsch gehandelt. Es hätte mir bereits enorm geholfen, ein Gesicht zu dir zu haben, mit dir zu kommunizieren, statt bloß einen stummen Schmerz in meiner Brust lodern zu spüren!“}

\textit{Der Lavastein – das Gesicht im Wogen der Flammen – blieb still. Falls es überhaupt eine Emotion zeigen konnte, so tat es dies nicht. Doch fühlte Trieest eine Verlegenheit in seiner Brust, die nicht die seine war.}

\textit{Trieest relativierte seine Aussage: „Nichtsdestotrotz will ich natürlich nicht, dass du meinetwegen stirbst. Ich ... ich dachte, dass man den Stein eines Tages entfernen könnte und deine Seele oder so erhalten bliebe.“}

\textit{Das Gesicht lachte auf: „Ich habe schon längst keine Seele mehr, ich bin doch schon lange tot! Ich bin einer der Alten, der Vorfahren, der Stimmen in der Roten Grotte, ein Echo eines Toten, ein Nachhall aus längst vergangenen Zeiten, weiter nichts. Um mich brauchst du mich nicht zu kümmern. Nicht mehr. Hiermit erlöse ich dich nun förmlich von deiner Bürde, Feuerkrieger Trieest. Du hast weitaus mehr Gutes getan, als deine Schuld von dir verlangte. Und nun kann ich auch mit gutem Gewissen sagen, dass keinerlei Gefahr besteht, dass du wieder die Kontrolle über dich verlieren könntest. Du darfst hiermit nach Danwar zurückkehren.“}

\textit{Trieest lachte auf: „Jetzt, wo du gehen musst, willst du es plötzlich noch so drehen, dass du im Recht warst?! Ich habe nichts als Unfairness und Verachtung von dir erfahren!“}

\textit{„Das stimmt nicht. Ich habe dich oft geheilt, dir oft das Leben gerettet. Dir und Jarid!“}

\textit{„Und genauso oft ... nein. Das ist es nicht wert. Lass uns nicht so enden. Wenn du gehen willst oder gehen musst, so gehe doch einfach. Gehe und lasse mich in Frieden.“}

\textit{„Bald, Trieest, bald. Kannst du noch einen kurzen Moment des Wartens verkraften?“}

\textit{Der Lavastein schien ernsthaft besorgt. Widerwillig willigte Trieest ein. Das Gesicht zeigte ein schiefes Grinsen und fuhr dann fort: „Es kommt selten vor, dass ein Lavastein Danwar verlässt. Hier gibt es keine solchen Steine, die Tote simulieren können. Diese fremden Lande des Südens, durch die du reistest ... in all diesen Landen hingen so viele Stimmen der Gefallenen fest, ohne dass es ein Ohr gäbe, welches sie hören könnte. So viele Verstorbene, deren letzte Nachhalle in der Erde verankert blieben, bis einer vorbeikommt und ihnen ihre letzten Sekunden der Bedeutung schenkt. Auf all deinen Reisen durch diese Welt konnte ich so einige dieser konservierten Nachhalle erhaschen und ... und wenn ich jetzt gehen muss, so würde ich sie gerne erklingen lassen. So viele, wie ich in meiner schwindenden Zeit zu vermitteln vermag. Da sind so einige, die sprechen wollen.“}\bigskip



\textit{Mit diesen Worten verschwand das weiße Gesicht im Flammenmeer und die Flammen teilten sich, um den Blick auf eine kleine grünhäutige Frau mit verfilzten Haaren und erdiger Kleidung freizugeben. Die Frau stürzte nach vorn, direkt vor Trieest, welcher immer noch schwerelos im Flammenmeer schwebte. Gemeinsam umkreisten sich diese beiden Gestalten im leeren Raum.}

\textit{„Finde meinen tapferen Wolfskrieger. Richte ihm aus, dass ich auf ihn warte. Dass er mich wiedersehen wird. Er soll sich nicht beeilen, er hat alle Zeit der Welt und soll sie genießen, aber richte ihm aus, dass ich auf ihn warte. Mein Orfen muss seine Fehde nicht zu Ende führen. Er soll sich nicht in den Hass auf eine ganze Kreaturenart hineinsteigern. Es gibt so viel Schönes im Leben, und so viele Schöne, und der Hass verbirgt dies vor ihm.“}

\textit{So sprach die grünhäutige Frau. Sie präsentierte Trieest ihre kleine rechte Hand, an deren Mittelfinger ein schnörkelloser silberner Ring steckte. Dann, von einem Lidschlag zum nächsten, war sie verschwunden.}\bigskip



\textit{Trieest hatte kaum Zeit, sich über diese Begegnung zu wundern, da teilten sich die Flammen erneut und eine in ein elegantes dunkles Kleid gewandte, blauhäutige Frau mit langem, schneeweißem Haar schwebte vor Trieest. Sie griff mit langen Fingern nach Trieests Händen, presste diese fest zusammen und sprach zitternd: „Kannst ... kannst du ihm mitteilen, dass ich ihm verzeihe? Meine blaue Blüte in der Dunkelheit. Ihm und seinem Bruder. Die beiden mussten vieles erdulden und wurden irregeleitet, doch er muss nicht daran festhalten, ihn zu suchen. Er plant, so viel Leid anzurichten, um ihn wiederzusehen, doch das muss er nicht. Sie werden sich auch so wiedersehen, frei und unverflucht. Er kann sich entspannen. muss keine weitere Pein auslösen. Aber das Wichtigste, wie schon gesagt: Ich – wir verzeihen ihm für alles, was er bislang getan hat, und was er noch tun wird. Richte ihm das aus.“}

\textit{Ein weiteres Blinzeln Trieests, und die blauhäutige Frau war verschwunden.}\bigskip



\textit{Erneut teilten sich die lodernden Flammen und ein weißhaariger Mann mit einem langen Bart glitt majestätisch nach vorne. Dieses Gesicht erkannte Trieest sofort, auch wenn er es bislang stets bloß mit einer goldenen gewellten Krone auf dem Schopf gesehen hatte – zum ersten Mal vor fast fünf Jahren, als Trieest auf der Spur einer geheimnisvollen Eiskreatur, die er in einer seltsam klaren Vision gesehen hatte, bis zur Rietburg gereist war.}

\textit{Er fasste sich: „Mein König! Ihr seid...“}

\textit{Brandur unterbrach Trieest, ohne auf seine Worte einzugehen: „Ich konnte es doch nicht ahnen ... warum hat er nichts gesagt? Warum musste Melkart sein Wort geben?! Sie ahnt es noch nicht einmal, doch muss sie es unbedingt erfahren.“}

\textit{Nahm der verstorbene König Trieest überhaupt wahr? Mit tränennassen Augen flüsterte Brandur weiter: „Nein, ich hätte es wissen müssen! Ich hätte bei ihr sein sollen!“}

\textit{Dann starrte er Trieest in die Augen: „Kannst du ihr ausrichten, dass es mir so unendlich leid tut? Kannst du ihr sagen ... sag ihr, dass ich sie liebe, wirst du? Und dass ich unglaublich stolz auf sie bin. Ich kann mich so glücklich schätzen, sie meine Tochter nennen zu dürfen.“}

\textit{„Wen meint Ihr, von wem sprecht Ihr, Herr?“, fragte Trieest.}

\textit{„Oh, meine tapfere Tochter ... ich habe eine Tochter!“, rief Brandur mehr zu sich selbst als zu Trieest. Seine Augen füllten sich erneut mit Tränen. Mit einem Ruck wurde er wieder in das wogende Meer der Flammen gezogen.}\bigskip



\textit{An Brandurs Stelle trat eine schwarzhaarige Frau mit einer spitzen Nase, um deren Hals ein rhombisches Amulett hing, dessen Oberfläche sich kräuselte.}

\textit{„Nicht nur von Brandur! Bitte, teile ihr auch von mir mit, dass nicht ein Tag vergeht, an dem ich nicht wünschte, bei ihr sein zu dürfen. Mir, ihrer Mutter Mhare. Mit ihrem Willen bringt sie Licht in jedes auch noch so tiefe Dunkel und ich bin so, so stolz ...“}

\textit{„Mhare? Von wem sprecht ihr? Wer seid Ihr überhaupt?!“, rief Trieest ihr entgegen, doch die grün gewandete Frau brabbelte weiter, als wäre Trieest gar nicht hier.}

\textit{Hände tauchten aus der Flammenwand vor Trieest auf und zogen die Frau mit dem Amulett zurück ins Feuer.}\bigskip



\textit{Die Flammen teilten sich und gaben den Blick frei auf die Besitzerin dieser Hände: Eine weitere schwarzhaarige Frau, und diese kam Trieest bekannt vor, auch wenn er sie nicht sofort einordnen konnte. Sie trug eine kleine Krone auf dem Kopf, wie ein goldener Blumenkranz, und sprach flehend: „Mein Sohn, mein kleiner Prinz! Er ist der Last nicht gewachsen, die auf seine Schultern gelegt wurde, und ein dunkler Schemen versucht, sich seiner zu bemächtigen. Ken dürstet nach seinem Amt. Das Land ist darauf angewiesen, dass mein Sohn seine Ratgeber weise wählt. Ich bitte euch, gebt ihm guten Rat und helft ihm zu erkennen, dass den Thron nicht behalten muss ... es läge keine Schande darin ... ich wünschte, ihn nicht länger leiden zu sehen unter dem Gewicht seiner falschen Entscheidungen ... ihn und das Reich ...“}

\textit{Sie brach ab, als zwei weitere Händepaare in den Flammen erschienen und die Frau aus Trieests Blickfeld zogen.}\bigskip



\textit{Zwei Personen traten gleichzeitig durch die Feuerwand auf Trieest zu. Hinter den sich teilenden Flammen waren kurzzeitig dutzende weitere Seelen zu erkennen, welche sich um einen Platz im Scheinwerferlicht drängelten. Wie viele Echos der Toten hatten sich hier versammelt?}

\textit{Die erste der beiden Gestalten, die nun von Trieest standen, war ein schwarz gewandeter Bewahrer aus dem Wachsamen Wald, welcher, die Kapuze tief ins Gesicht gezogen, murmelte: „Teile ihm mit, dass er das alles nicht verdient hat. Sein ganzer Ruhm ist unverdient! Ein lügnerischer Verräter ist er, und wenn er nicht gewesen wäre, würde Folla heute noch leben! Wenn ich ihn in die Finger gekriegt hätte, hätte ihm diese Angelegenheit mit der Ziege und dem Sternbild des Hornfalken endlich leid getan! Und dieses elende selbstgerechte Grinsen auf seinem Gesicht hätte ...“}

\textit{Die wütende Schwarze Wache wurde von der zweiten Gestalt unsanft zur Seite geschubst und verschwand in der Flammenwand.}\bigskip



\textit{Bei der zweiten Gestalt handelte es sich um eine ältere Frau mit einem von Flicken übersäten Gewand, in deren Körper immer noch zwei schwarze Pfeile steckten. Eine primitive Augenklappe zierte ihr linkes Auge. Jetzt, wo Trieest ihren ganzen Körper sehen konnte, erkannte er, dass die Brandnarben auf ihrer linken Gesichtshälfte sich bis weit über ihre Schulter zogen, und dass ihr linker Arm fast zur Gänze fehlte.}

\textit{„Trieest, häh?“, fragte die Alte ihn krächzend nach seinem Namen.}

\textit{Trieest atmete auf. Endlich jemand, der ihn wahrzunehmen, ja, gar zu erkennen schien.}

\textit{„Viel Zeit bleibt mir nicht“, flüsterte Lysbett die Weinhändlerin, „Dieser feurige Stein hat keine Ahnung, was er tut. Die Echos dunkler Gestalten sind auf dem Weg hierher, also rasch! Mein Appell ist wichtiger als der ihre. Du hast mich nie wirklich kennen gelernt, doch habe ich vorhin dir und deiner Begleiterin das Leben gerettet. Meines musste ich nun leider lassen, doch du kannst dich auf eine andere Art revanchieren. Der finstere Geist, welcher in diesem roten Kristall des Hraak steckte, ist noch immer Teil deiner Welt. Er zog sich in einen kleinen Splitter zurück, einen tiefschwarzen, den ich törichterweise anfasste, als Jarid sich noch nicht neben mich gestellt hatte. Das Böse war es, das Jarid und dich in meinem Körper auf einer Kutsche in den Norden fuhr. Fürchtete sich wohl davor, euch offen zu konfrontieren. Doch nicht einmal der Skralangriff konnte es erledigen. Trotz der schwarzen Pfeile in meinem Körper weigert das Böse sich, ihn loszulassen. Der böse Geist hat dem Tod bereits einmal ein Schnippchen geschlagen, und er wird es wieder tun. In diesem Augenblick dirigiert er meinen Leichnam weiter in den Norden. Ich weiß nicht, was sein Ziel ist, aber ein gutes wird es nicht sein. Verfolge ihn! Vernichte den schwarzen Kristallsplitter! Oder vernichte denjenigen, der den Splitter trägt. Eine andere Wahl hast du nicht!“}

\textit{„Das Böse kontrollierte deinen Körper die ganze Zeit? Was hat es vor?“, rief Trieest, doch von einem Augenblick zum nächsten war Lysbett verschwunden. Ein dunkles Rumpeln kündigte die Ankunft des nächsten Abbilds an. Doch etwas war anders dieses Mal ... waren das etwa Schreie von außerhalb der Flammenwand? Ein dumpfes Brüllen? Was konnte dies ...}\bigskip



\textit{Das Flammenmeer teilte sich ein letztes Mal und ein riesiger, unmenschlicher, schwarz geschuppter Schädel mit glühend roten Augen ragte hindurch. Drachenstacheln und scharfe Schuppen sägten sich durch Trieests Körper hindurch und bohrten sich in sein Fleisch. Überraschenderweise schmerzte es nicht einmal. Der Drachenkopf schüttelte Trieests sich auffasernde Gestalt in ihrem Mund. Mit einem gewaltigen Grollen brüllte Tarok, der letzte Drache:}

\textit{„RICHTE SIE! VERBRENNE IHRE BURG! VERSENGE IHR LAND! TÖTE DIE KÖNIGSSPROSSE! RÄCHE MICH! RÄCHE MIIIIICH!“}

\textit{Dann wurde alles schwarz.}

Und Trieest erwachte wieder.\bigskip







Trieest saß inmitten der matschigen Waldlichtung und weinte. Die Regentropfen auf seiner Haut verdampften zischend. Vielleicht war dies das letzte Mal, dass dies geschehen würde. Kalte, kühle Luft traf seine Brust und streifte die frische Haut in der Delle, in welcher der Lavastein gesessen hatte. Seine Haut kribbelte und sein Kiefer fühlte sich geschwollen an. In seiner Faust hielt er immer noch ein matschiges Etwas, das er aus seinem Kontrahenten gerissen hatte. Ein Organ? Ein Stück Muskel? Es kümmerte er nicht.

Sein Kontrahent, der ehemalige Skralhäuptling Shron, lag etwas abseits am Boden und rührte sich nicht mehr. Tot. Genaueres konnte Trieest durch sein verschwommenes Sichtfeld nicht erkennen. Aber das war auch nicht wichtig. Da am Boden vor ihm lag das, was einst sein Lavastein gewesen war. Er war zerborsten, nur noch eine unordentliche Ansammlung orangefarbener Splitter im Matsch. Und so saß Trieest da und weinte, weinte um all die, die gefallen waren, deren letzte Echos in seinem Kopf weiterhallten, auch wenn er die meisten ihrer Bitten bereits wieder vergessen hatte. Er riss seinen Kopf in den Nacken, ließ die Tropfen des Himmels auf seine Augen prasseln und stieß ein gutturales Brüllen aus.

Ein Brüllen, das einem Ruf der Skrale verblüffend ähnlich klang.

Trieest konnte nicht sagen, wie lange er sich in diesem aufgewühlten Zustand befand. Von fern her drangen viehische Schreie und das Klirren von Stahl auf Stahl zu ihm hin, doch er beachtete es nicht. Der Ruf eines Skrals ertönte, laut, ganz in seiner Nähe, und brach abrupt ab. Auch das kümmerte Trieest nicht. Er fühlte sich voller Erinnerungen und Emotionen, und doch so leer und allein. Der Lavastein war nicht mehr, und erst jetzt wurde Trieest klar, wie er die ganze Zeit schon passiv der Stimmung des Steins gelauscht hatte. Doch an dessen Stelle war nun ... nichts. Nichts als Leere. Wahrscheinlich war es besser so.

Nach einer scheinbaren Ewigkeit tippte ihm eine Kralle auf die Schulter.

„Häuptling? Ich weiß nicht mal deinen Namen, aber ... ist alles ... was ist geschehen?“

Trieest drehte seinen Kopf zur Seite und sah einen muskulösen Skral mit einem verfilzten weißen Bart vor sich stehen. Er hatte diesen Skral schon einmal gesehen. Unverständlich blinzelte er ihn an.

„Wer ... wer bist ...“

„Calrai, Herr, mein Name ist Calrai“, fuhr der Skral fort und schien angespannt, als könnte ihm Trieest jeden Augenblick an die Gurgel springen, „Wie können wir dir helfen, Häuptling?“

„Spiegel!“, keuchte Trieest, „Habt ihr einen Spiegel? Und nenne mich Trieest. Ich bin dein Häuptling nicht.“

Calrai winkte einen anderen Skral mit einem runden Metallschild herbei. Während dieser Skral vorsichtig auf ihn zutrottete, bemerkte Trieest eine frische Schnatte an dessen Schädel, aus welcher ein dünner Blutstrom floss.

Ein rascher Rundblick über die Lichtung verriet Trieest, dass nur noch vier Skrale anwesend waren: Calrai direkt bei ihm, der nähertretende Skral mit dem Schild sowie zwei weitere ein wenig abseits, von denen der eine dem anderen irgendwelche Kräuter auf den Baum presste.

Ein fünfter Skral lag leblos nahe des Zelts. Drei schwarze Pfeile steckten in seiner Brust. Shron lag ebenso mausetot vor Trieest im Dreck.

Die übrigen Skrale der Sippe – nach Trieests Zählung sollten es zwei sein – waren nirgends zu erkennen. Die Pferde der Kutsche ebensowenig.

Irgendetwas war soeben hier vorgefallen. Aber darum konnte sich Trieest später kümmern.

Endlich erreichte der Skral ihn und reichte ihm seinen metallenen Schild. Dieser war von Beulen und Dellen übersät, dennoch konnte Trieest seine eigene Form schemenhaft darin erkennen.

Aus dem Schild blickte ihm ein Halbskral entgegen.

Trieests Kinn war breiter, viel breiter als zuvor, und als er seinen Mund öffnete, bleckte er scharfe Zahnreihen. Seine Nase war weiter in sein Gesicht zurückgewandert und breiter geworden. Ein Blick auf seine Hände verriet, dass seine kleinen und Ringfinger verkümmert wirkten. Als er an seinen Kopf griff, löste sich sein langes schwarzes Haar gar büschelweise und verteilte sich neben ihm im Dreck.

Instinktiv griff Trieest an seinen Schwanzstummel, aber nein, dieser war nicht nachgewachsen. Die Orden der Wassermagier hatten damals gedacht, dass die Amputation seines Schwanzes alles wäre, was es brauchte, damit er als Mensch durchgehen würde. Seine scharfen Zähne und spitzen Ohren waren erst später vollkommen hervorgetreten.

Nach dem Vorfall mit Jormudd hatte er diesen elenden Lavastein sieben Jahre lang getragen, in der Hoffnung, dass dieser ihm eine menschliche Gestalt geben könnte. Nun, zumindest so menschlich, wie ein Feuerkrieger aussehen konnte. Den anderen Kriegern nahmen die Lavasteine einen Teil ihrer Menschlichkeit im Aussehen, Trieest hingegen hätte sie welche gegeben. Und nun, wo dieser Kristall endlich zerbrochen war, war er blitzschnell zu seiner vorherigen Form zurückgekehrt. Seine Schuld mochte in den Augen der Danware endlich bereinigt sein, aber dorthin zurückzukehren wünschte er ohnehin nicht. Und in seinen eigenen Augen hatte er diese Schuld schon vor Jahren beglichen. Es war alles für nichts gewesen. Diese Bürde hatte ihm für nichts und wieder nichts geschadet. Alles umsonst.

So sank Trieest wieder in sich zusammen, von Weinkrämpfen geschüttelt. Passend zu seiner Stimmung öffneten sich die Schleusen des Himmels erneut, noch heftiger als zuvor, und ergossen ihren Inhalt auf den Halbskral. Nass und kalt prasselte der Regen auf Trieests dicke Haut ein, doch dieser fühlte es nicht einmal.

Dann erinnerte er sich und sein Kopf schoss erschrocken wieder in die Höhe: „Jarid!“

Die heftig verwundete Wassermagierin war zuletzt ins Zelt des Skralhäuptlings gebracht worden. Schwankend erhob Trieest sich und drehte sich um die eigene Achse, versuchte durch den dichten Regenschleier, das Zelt zu erkennen.

Dort drüben war es!

Seine Beine setzten sich in Bewegung, ehe er sie richtig koordinieren konnte, und mehr stolpernd als rennend glitt Trieest durch den immer matschiger werdenden Untergrund auf das Zelt zu.

Die beiden Skrale huschten ihm eilig aus dem Weg, und er hörte hinter sich Schritte, als sich der massige bärtige Skral – Calrai – in Bewegung setzte. Rannte er auf ihn zu oder von ihm weg? Es war nicht relevant.

Trieest riss die Zeltplane zur Seite und erwartete, tropfnasse Erde und eine gefesselte Jarid zu erblicken. Stattdessen war der gesamte Zeltboden von einer dünnen Eisschicht überzogen. Jarid war nirgends zu sehen, nur noch der blutige Verband, der ihren verletzten Bauch gestützt hatte. Ein der Mitte des Zelts stehender dünner Torbogen aus vereistem Regenwasser verriet, wie Jarid geflohen war.

Sie hatte sich davonteleportiert, trotz ihres verletzten Zustands. Wohin, hatte er keine Ahnung. Wie er ihr helfen sollte, wusste er ebensowenig.

Trieest war allein.\bigskip







\textit{Lysbetts Körper stolperte durch den Wachsamen Wald, krachte durchs Unterholz und schnitt sich an Dornen. Das Böse achtete nicht einmal darauf. Wozu den Anschein geben, es wäre auf Blut in seinen Adern angewiesen? Es musste nur schnell einen kräftigeren Körper erreichen, ehe seine magischen Kapazitäten erschöpft waren.}

\textit{Noch immer hallte in seinem Kopf eine Stimme nach, von welcher es geglaubt hatte, dass es sie nie wieder hören würde. Worte, die plötzlich in Trieests Kopf erschollen waren, und die sich doch so persönlich ans Böse gewandt hatte.}

\textit{„Wisse, dass ich dir noch immer verzeihen kann. Wisse, dass ich noch immer an dich glaube. Beseitige deine alten Zwiste. Nur gemeinsam können die Menschen dieser Welt die Widrigkeiten der Zukunft überstehen.“}

\textit{Erneut hallte das Echo der Stimme durch den Geist des Bösen. Es war unmöglich, dass dies wirklich die Stimme dessen gewesen war, für den das Böse ihn hielt. Es konnte einfach nicht sein.}

\textit{Eine hämische Stimme aus der realen Welt riss das Böse aus seinen Gedanken.}

\textit{Rasch kauerte sich Lysbetts Körper an den Boden. Bei Borgs glänzender Glatze, wer war denn zu dieser Zeit noch im Unterholz unterwegs?!}

\textit{Das Böse lauschte.}

\textit{„Ich versichere Euch, niemand außer uns weiß, dass sich dieses Drachenherz nicht mehr an der ihm angestammten Stelle befindet. Nun, niemand außer uns und dem überaus geschickten Dieb ... öhm ... Geschäftspartner, der mir die beiden anbot. Und der wird diese Tatsache bestimmt nicht an die Schildzwerge verraten. Er ist zutiefst verfeindet mit Hallworts Brut. Familiengeschichte. Üble Sache. Ich schwafle schon wieder zu viel, häh?“}

\textit{Eine ruhige Stimme antwortete leise: „Tu dir keinen Zwang an, Handelszwerg. Ich weiß, von wem du die zwei Herzen des Nehal erhalten hast. Ich weiß vermutlich mehr über diesen Dieb als du selbst, halte ich doch stets ein interessiertes Auge auf alle, die die Geschicke dieses Königsreichs zu lenken vermögen und ersuchen. Und dieser Dieb hat viel mehr Macht über dieses Reich, als du ahnen magst. Doch ist dies nebensächlich. Sei dir gewiss: Wenn du mit anderen Kunden ebenso frei über unsere Beziehung schwafelst wie über den Dieb mit mir, so kannst du selbiger Beziehung bald nachtrauern.“}

\textit{Der Handelszwerg grummelte eine Entschuldigung, wurde in jeder allerdings gleich wieder von der kühlen Stimme seines Geschäftspartners unterbrochen.}

\textit{„Lass uns zum Geschäftlichen kommen, Garz. Ich biete dir diese Phiole für das noch nicht vergebene Drachenherz, und lege gar fünf Goldstücke drauf.“}

\textit{Etwas Gläsernes klimperte.}

\textit{Da hämische Stimme des Handelszwergs antwortete jedoch: „Rede dein Gebräu nicht unnötig hoch, das steht dir nicht. Bei meinen Beziehungen zu Reka brauche ich keinen zwielichtigen Trank aus zweiter Hand zu ersteigern. So was könnte ich mir doch gleich selbst zusammenmischen! Bin in der ganzen Zeit im Grauen Gebirge nämlich überaus geschickt geworden im Brauen von ...“}

\textit{„Die Dosis macht das Gift, und die Zutaten für eine derartige Dosis sammelst du in deiner Lebzeit nimmer zusammen. Nicht einmal deine Reka könnte das, und im Gegensatz zu dir vermag sie Krallenflechten von Herbstschwurz zu unterscheiden.“}

\textit{„Fünfzehn Goldstücke.“}

\textit{„Zehn.“}

\textit{„Abgemacht!“}

\textit{Weiteres Geklimper ertönte. Dann sprach die hämische Stimme weiter. Das Böse glaubte, ein leises Seufzen von der ruhigen Stimme zu vernehmen, doch unterbrach sie den Handelszwerg nicht, als dieser weiterschwafelte.}

\textit{„Darf ich meinem liebsten Seher noch ein anderes Sonderangebot anbieten? Beispielsweise hätte ich hier ein äußerst seltenes Artefakt aus Sturmtal, ja, gar ein Unikat: Die Schamanenmaske der Taren. Sie wurde von ihnen eingesetzt, um Furcht in den Herzen ihrer Gegner zu säen. Die Taren mögen geschworen haben, sie nie wieder aufzusetzen, doch nicht Ihr. Nur fünf Goldstücke. Fast geschenkt, bei dem Schutz, den sie gibt. Meine Mutter selig meinte schon immer, ich hätte ein zu weiches Herz.“}

\textit{„Sehe ich so aus, als hätte ich eine solche Maske nötig?“}

\textit{„Naja, zumindest nähme sie dir im Gegensatz zu vielen anderen nicht die Sehkraft.“}

\textit{Die hämische Stimme gluckste auf.}

\textit{Bei diesen Worten festigte sich im Bösen der Keim des Verdachts, um wen es sich beim anderen Handelspartner handeln könnte. Es wagte einen Blick aus dem Gebüsch hinaus und erblickte aus Lysbetts gutem Auge einen Steinwurf von sich entfernt zwei Gestalten, welche geduckt nebeneinander im Unterholz standen.}

\textit{Sie hätten unterschiedlich kaum sein können. Die linke Gestalt war selbst für einen Zwerg gewaltig breit, was aber nicht zuletzt am überdimensionierten klimpernden Rucksack lag, den die Gestalt auf ihrem krummen Rücken trug. Garz, der Handelszwerg von überall und nirgendwo. Er verstaute soeben mit seiner einen Hand eine kleine, in allen Regenbogenfarben schimmernde Phiole in einer Bauchtasche, und hielt mit der anderen eine große bräunliche Holzmaske mit zwei langen Hörnern daran in die Höhe.}

\textit{Die andere Gestalt war groß gewachsen und trug einen langen braunen Umhang. Die Augenbinde, die blaue Haut, der lange knorrige Stab ... das Böse wusste sofort, um wen es sich handelte. Und es erkannte auch den großen weißen Edelstein, den Leander soeben in seiner Tasche verschwinden ließ.}

\textit{Das Schicksal konnte gelegentlich einen unglaublichen Sinn für Humor haben. Sollte das Böse etwa wagen, diese schicksalshafte Transaktion zu unterbinden?}

\textit{Sowohl mit Garz als auch mit Leander hatte das Böse noch eine Rechnung offen, wenn auch aus ganz unterschiedlichen Gründen. Der gierige Garz hatte einst eine eigentlich vertrauliche Nachricht des Bösen für eine beträchtliche Menge profitschlagenden Materials in fremde Hände gegeben, die ohne ihn nicht einmal von der Existenz dieser Nachricht gewusst hätten. Und Leander wusste es zwar noch nicht, doch hatte er das Böse einst in seinem kristallenen Gefängnis eingesperrt. Doch dies alles war Jahre her, und in seinem geschwächten Zustand konnte das Böse wohl nicht einmal gegen Garz allein etwas anrichten. Es konnte einen Lacher nicht verkneifen über die schiere Absurdität seiner Lage und duckte sich rasch wieder in sein Gebüsch.}

\textit{Das Gebüsch raschelte verräterisch.}

\textit{„Was war das?!“, fragte Garz.}

\textit{Leander antwortete nicht, doch sich rasch entfernende Schritte verrieten dem Bösen, dass der Seher von Narkon genug von dieser Szene hatte und sich ohne Verabschiedung zurückzog.}

\textit{Garz rief ihm hinterher: „Solltet Ihr eines Tages weiteres Interesse an reinen Edelsteinen haben, so denkt bitte an mich, den Handelszwerg Eures Vertrauens.“}

\textit{Als das Böse sich sicher war, dass Leander sich nicht mehr in Hörweite befand, traute es sich wieder hinter seinem Gebüsch hervor. Zu diesem Zeitpunkt war auch Garz längst wieder abgezogen.}

\textit{Allein durchquerte das Böse den Wachsamen Wald weiter. Auf seinem Weg wich es wachenden Bewahrern und wütenden Kreaturen gleichermaßen aus.}

\textit{Leanders Anwesenheit zufolge befand sich das Böse bereits in der Nähe von dessen Hütte, und somit ganz nahe am Ufer zum Hadrischen Meer. Bald schon vernahm es die Geräusche der Brandung am Ufer und schmeckte die salzige Meeresluft auf Lysbetts Zunge.}

\textit{Das Böse führte Lysbetts Körper bis hin zum Wasser, weit weg von Leanders Hütte oder vom nächsten Hafen. Es suchte sich eine Stelle aus, wo das Ufer tief abfiel. Lysbetts rechte Hand lupfte den schwarzen Kristallsplitter aus ihrem Mund und fasste ihn sorgfältig in Lysbetts Fingerspitzen. Dann sprang es hinein ins kalte Nass.}

\textit{Hinein ins Hadrische Meer!}

\textit{Luft zum Atmen hatte das Böse natürlich ebensowenig nötig wie das Blut in Lysbetts Adern, doch strampelte es nun theatralisch mit aller Kraft, die ihm blieb, während sein Körper von wilden Wellen weiter in den Ozean gezogen wurde. Dann ließ das Böse Lysbetts Körper erschlaffen und von den Fluten wie einen Spielball umherwirbeln.}

\textit{Es dauerte nicht lange, bis eine naive Nixe Lysbetts leblosen Körper erreichte und ihn verzweifelt in Richtung der Wasseroberfläche zu zerren begann. Das Böse passte den passenden Moment ab, schlug abrupt zu und stach den schwarzen Kristall in den Arm der Nixe. Diese hatte nicht einmal die Zeit, überrascht aufzuschreien, da übernahm das Böse auch schon die Kontrolle über sie.}

\textit{Während Lysbetts Körper losgelassen wurde und seinem nassen Grab am Meeresgrund entgegensank, zog das Böse mit den Fingern der Nixe des schwarzen Kristallsplitter wieder aus ihrem Arm und hielt ihn fest in ihrer Faust.}

\textit{Kurz studierte das Böse die vielen Erinnerungen der Nixe. Dann verwarf es sie als belanglos und schwamm zielstrebig in Richtung Nordosten los.}

\textit{Auf, in Richtung Danwar.}
















\newpage
\section{Der Alchemist im Südlichen Wald}




\az{Jahr 61}

\textit{Halle des Ältestenrats, 61 a.Z.}\bigskip



„Der Ältestenrat versinkt doch ohnehin im Chaos!“, warf der grantige Kord ein, welcher sich einen Hut mit Krempe über die Stirn gezogen hatte und locker auf einem Stuhl in der Ecke fläzte. Nun saß er auf und sprach mürrisch. „Ihr redet und redet, dabei habt ihr schon längst eine Entscheidung getroffen. Trieest ist in den Augen des Ältestenrats schuldig. Als Strafe wird er so lange von Danwar verbannt, bis er einen Prozess des Wandels hinter sich gebracht hat. Durch diesen Prozess wird ihn ein Lavastein der Feuerkrieger leiten, den Trieest stets als Bürde mit sich tragen soll.“

Die aufmüpfige Freiga setzte zum Sprechen an, wurde aber prompt von Kord unterbrochen: „Das ist keine Belohnung, du rachesüchtige Huschel, es wird sich für ihn jedenfalls nicht so anfühlen. Jeder Lavastein ist eine schwere Bürde, selbst wenn er nur halb so groß ist wie das Glühende Herz des Flammenden Gottes. Und nicht alle Feuerkrieger klagen so selten wie die stoischen Glutträger.“

Die Älteste Rowinda versuchte das Wort an sich zu reißen, wurde aber ebenfalls von Kord unterbrochen: „Ich habe es vorausgesehen. Der Orden der Feuerkrieger wird Trieest nur allzu gerne mit Schwert und Rüstung ausstatten. Kein Vergleich damit, was es kosten würde, ihn weiterhin durchzufüttern. Und zumindest die Rüstung braucht es, damit der Lavastein halten kann. Oder ist dem etwa nicht so?“

Kord blickte vielsagend zur Leiterin der Feuerkrieger. Diese machte sich nicht einmal die Mühe, zu nicken. Der grantige Kord war mit dem zweiten Gesicht ausgestattet. Wenn er etwas sagte, so stimmte es immer. Oder zumindest meistens.\bigskip







\az{Jahr 68}

\textit{Sieben Jahre später.}\bigskip




Es war warm, als Jarid langsam wieder zu Bewusstsein kam. Als sich ihr Drachengewand aus dem Brunnenwasser schälte und ihr Körper sich darin rematerialisierte. Wie üblich fühlte Jarid die große Erschöpfung durch die Unmenge an Konzentration, die sie hatte aufwenden müssen, um durch die Wasserfläche zu tunneln – nur noch erschwert durch ihren lädierten Zustand und die Tatsache, dass das Regenwasser, das sich in der Tiefe der Zeltkuhle gesammelt hatte, kaum vom matschigen Untergrund hatte unterscheiden lassen. Sie versuchte, so viel Energie wie möglich aus dem Wasser zu ziehen, ehe sie mit ihm uneins wurde, aber anders als üblich konnte sie ihren Körper dadurch kaum sinnvoll stärken.

Um Jarid herum brodelte und dampfte es, als sie sich aus dem Brunnen hievte und zu Boden fallen ließ. Sie hatte sich bloß darauf konzentriert, möglichst weit weg vom Lager der Skrale zu landen ... verflixt, wo lag sie denn nun?

Langsam erkannte Jarid, dass sie sich auf einem Dorfplatz befand. Eine Reihe von Bauernhütten umgaben den dampfenden Brunnen. Die meisten Bauernkaten waren neu gebaut und in ganz passablem Zustand. Einige andere waren verfallen, bis auf die Grundmauern niedergebrannt. Hier war ein großes Unglück vorgefallen, doch sah es aus, als wäre dies schon ein Jahrzehnt her.

Ein kleiner Junge rannte über den Platz und starrte sie mit großen Augen an, wie Jarid zitternd neben dem immer noch dampfenden und zischenden Brunnen zu Boden sank. Jetzt war es ihre Brust und nicht wie üblich Trieests, die schmerzte, als würde sie in Flammen stehen. Sie griff sich daran und tastete die Wunde ab, die das Rankenschwert verursacht hatte.

„Heiler“, krächzte Jarid zum Jungen, „Ich brauche einen Heiler!“

Der kleine Junge stand immer noch da und starrte sie mit großen Augen an.

„Ich bin verletzt!“, hauchte sie, „Bitte, hole Hilfe. Wo sind deine Eltern?“

Endlich drehte sich der Junge auf und rannte davon, in den nächstgelegenen Schuppen hinein. Jarid konnte nur hoffen, dass er tatsächlich Hilfe holte.

Als der Junge zurückkehrte, hatte er leider keinen Heiler im Schlepptau. Und auch keinen Erwachsenen. Wo befanden sich denn alle? Feldarbeit, Ratsversammlungen, die kommende Gedenkfeier an der Rietburg? Immerhin trug der Junge nun eine kleine Schiefertafel bei sich. In deren Oberfläche eingeritzt erkannte Jarid die Schriftzeichen der andorischen Sprache.

Der Junge deutete kopfschüttelnd auf seine Ohren, dann hielt er Jarid die Tafel hin. Jarid vermochte es, die ersten sieben Buchstaben des Wortes ‚Hilfe‘ zu buchstabieren und auf die blutige Stichwunde in ihrem Unterleib zu deuten, die sich schon wieder geöffnet hatte und den dunklen Fleck auf ihrem Kleid vergrößerte.

Verständig leuchteten die Augen des Jungen kurz auf, ehe sie gleich wieder von Sorge verdunkelt wurden. Er packte Jarid an der Hand und führte sie stolpernd mit sich mit. Weg von dem Brunnen, aus dem sie gekommen war, weg aus dem kleinen Dörfchen. Gemeinsam streiften Junge und Jarid durch goldenes Rietgras.

Diese Gegend kam Jarid bekannt vor. In der Ferne sah sie Rauch am Himmel aufsteigen, und darunter erkannte sie die sichere Taverne zum Trunkenen Troll, wie sie vor dem Südlichen Wald stand. Erleichterung machte sich in Jarid breit. Sie wusste nun, wo sie sich befand. Als sie allerdings strikt auf die Taverne zuzuhalten versuchte, schüttelte der taube Junge entschieden den Kopf und zog sie nach Westen, in den Südlichen Wald hinein. Verwirrt folgte Jarid ihm.

Hin und wieder blieb der kleine Andori stehen und blickte sich angestrengt im Wald um. Dann erkannte er wohl plötzlich das, wonach er gesucht hatte, und zog Jarid weiter mit sich. Diese vermochte nicht zu erkennen, ob er einer Art Wegweisern oder einer willkürlichen Laune folgte, und war somit alles andere als glücklich. Ihre Bauchwunde pochte immer stärker und sich im Südlichen Wald zu verlaufen würde ihr sicher nicht helfen. Einzig für die Kräuterhexe Reka oder jemand mit ähnlich geschickten Kräuterkenntnissen würde sie bereitwillig einen solchen Umweg machen, aber Reka hielt sich eigentlich so gut wie nie in diesem Wald auf. Ein unbekannter Schrecken lauere hier, hatte Reka beide Male erwidert, als Jarid sie danach gefragt hatte.

Schon war Jarid kurz davor, sich einfach weiterzugehen zu weigern und den Jungen zu bitten, sie zur sicheren Taverne zurückzuführen, oder schlichtweg den Jungen stehen zu lassen und selbst diese Richtung einzuschlagen, da erkannte sie zwischen zwei Bäumen etwas, das das Ziel des Jungen sein könnte: Eine kleine Hütte in einer vom Sonnenlicht beschienenen Waldlichtung. Wahrlich wundersam, führte doch kein Weg durchs Unterholz zu ihr. Ein großer Apfelbaum schien gar in die Hütte eingewachsen zu sein – oder die Hütte um ihn herum gebaut. Wer hier wohl lebte?

Der Junge trat an die Tür, nickte Jarid beruhigend zu und hob seine Hand.

Klopf. Klopfklopf. Klopf. Eine ganz bestimmte Reihenfolge. Vielleicht ein Signal?

Stille.

Dann ... Schritte.

Die Tür zur Hütte wurde aufgerissen. Im Türrahmen stand ein Mensch mit einer langen Nase, einer längeren Zipfelmütze und einem noch längeren blauen Mantel. In seiner Hand hielt er ein seltsames gläsernes Gefäß mit einer grünlich dampfenden Flüssigkeit darin. Sie roch süßlich.

Der Mensch sah überraschend lang relativ verdattert drein, winkte dann dem Jungen zu und meinte dann: „Hallo, Soraf. Hallo, mysteriöse Blaugewandte.“

Er verstummte wieder und stellte sein gläsernes Gefäß auf ein kleines Brettchen neben der Tür, auf irgendeine Reaktion wartend.

Der Junge blickte fordernd zu Jarid zurück. Diese verstand den Wink und meinte: „Grün sind die Wogen ... ich meine, guten Tag, werter Herr. Ich wurde von einem Schwert am Bauch verwundet und brauche einen Heiler. Dieser Junge hat mich hierher gebracht. Könnt ihr mir helfen?“

„Grün sind die Wogen der Wellen“, meinte der Mensch, ehe er überrascht und mehr an sich selbst als an Jarid gewandt anhängte: „Eine Danware? Eine danwarische Wassermagierin, so weit weg von der Insel? Ungewöhnlich, äußerst ungewöhnlich.“

Jarids Bauch meldete sich mit einem stechenden Schmerz, und Jarid stöhnte auf. Der Mensch zuckte zusammen und meinte: „Verzeiht. Ich weiß auch nicht, warum Soraf meint, ich könnte Euch behilflich sein. Heilkunde ist wirklich nicht mein Ding, ich bin eigentlich eher im Geschäft von ... explosiveren Tränken.“

Ein leises kicherndes „Hihihihihihihihihihihihihihihihihihihihihihihi“ entschlüpfte ihm, ehe er sich wieder fing und Jarid beruhigender zusprach: „Ich kann auch ganz passable Gegengifte brauen. Ihr wurdet nicht zufälligerweise von einer Vypera gebissen? Nein, nein, ist schon gut, das ist besser so. Nur die Bauchwunde? Ach, ich werde sicherlich sehen, was ich für Euch tun kann. Kommt herein, kommt herein, Ihr solltet Euch hinlegen. Es ist gefährlich, im Stehen das Bewusstsein zu verlieren.“

Der Fremde wandte sich Soraf zu und warf ihm eine Goldmünze entgegen, ehe er rasch einige Handzeichen mit ihm austauschte. Soraf nahm die Münze entgegen und spazierte wieder davon.

Und Jarid betrat wie angeleitet die eigenartige Hütte. Die Wände waren überstellt mit Regalen voller Tränke und Salben, Basteleien und Zeichnungen und Skizzen, so vieles angefangen, so wenig fertig gestellt. Der Trankmeister eilte umher, hantierte indes fahrig an einigen Phiolen in einer Kiste herum, ließ die Kiste dann zufallen und rannte zu Jarid zurück:

„Wartet, lasst mich kurz diesen Tisch frei räumen, und dann könnt Ihr Euch darauf hinlegen. Ja, genau da. Bequem? Nein, natürlich nicht. Ich hole gleich ein Kissen. Ich muss vorher nur kurz in mein Kämmerchen, einige Kräutersäfte holen. Soll ich ein Glas Wasser bringen? Ich weiß ja nie, was ihr Wassermagier damit so anstellen könnt. Nein? Sicher nicht? Wäre auch zum Trinken sehr gut, das kann die Lebensgeister ungemein erfrischen. Oder wollt Ihr etwas Tee? Ich danke tagtäglich meinen TeeSIEBEN, dass sie mich mit wundervollen Getränken beschenken. Ach herrje, ich habe mich ja noch gar nicht vorgestellt. Naraven ist mein Name. Was meint Ihr, wirke ich wie ein Gelehrter oder Alchemist oder aber gehöre ich einem geheimen Orden von Hexern an? Mittleres stimmt, ich bin meines Zeichens danwarischer Alchemist. Mein Name lautet Naraven. Und wie lautet Eurer?“

„Jarid Morgentau, Wassermagierin des ...“

Naraven rauschte so schnell aus dem Raum, dass Jarid sich nicht sicher war, ob er ihre Antwort überhaupt mitgekriegt hatte. Ächzend ließ sie sich wieder auf den Holztisch niedersinken. Der Raum begann bereits damit, sich leicht um sie zu drehen. Sie unterdrückte den Schwindel und konzentrierte sich auf ihre Umgebung.

Das Innere von Naravens Hütte war von einem warmen Kaminfeuer erleuchtet und sah unglaublich gemütlich aus. Am Boden vor dem Kamin erkannte sie einen waschechten tulgorischen Teppich – wertvolle Ware war das!

Zahlreiche Papiere und Pergamente waren über die sie umgebenden Regale und Tische verteilt. Das waren aber nicht alle Schriftstücke in diesem Raum. Besonders auffällig waren verschiedene schwarze Schiefertafeln mit eingeritzten Schriftzeichen, die zum Teil an die Wände gehängt waren und zum Teil in instabil hohen Stapeln in die Regale gequetscht worden waren. Solche Tafeln würde Jarid im Schlaf wiedererkennen, das waren danwarische Schriftzeichen auf feuersicheren danwarischen Steintafeln. Eine davon konnte sie gerade noch entziffern, sie handelte von ... wackeren Bauern, die Gorfleisch vertilgten? Etwas daran kam Jarid bekannt vor, doch konnte sie es in ihren müden Zustand nicht einordnen.

Die danwarischen Schriftzeichen verschwammen vor ihren Augen, doch ihr Inhalt war auch nicht so relevant. Wichtiger war, dass dieser Naraven war also wirklich ein danwarischer Alchemist zu sein schien. Es war natürlich ungewöhnlich, dass ein Danware von Zuhause aufbrach, aber längst nicht so ungewöhnlich für gewöhnliche Bewohner, wie es für Feuerkrieger oder Wassermagier war. Vielleicht hatte er in der weiten Welt nach Abenteuern gesucht und sich dann hier niedergelassen? An Kundschaft würde es ihm nicht mangeln, aber warum hätte er seine Hütte dann so abgelegen im Wald gebaut? Einen großen Ruf konnte er nicht haben, sonst hätte Jarid ihn doch sicherlich bereits vernommen.

„Da bin ich wieder“, erklang die quiekende Stimme des Alchemisten. Er stolperte zur Tür hinein und verteilte eine Vielzahl an Kräutern, Döschen, Säckchen und einem einzelnen großen Kissen vor sich. Das Kissen schob er unter Jarids Kopf – es duftete nach nassem Fell, vielleicht von einem Hund? – und den Rest hievte Naraven unzeremoniell neben die liegende Jarid, ehe er sich die Hände rieb und verkündete: „So! Zeit, sich diese Wunde anzuschauen! Auweia, da ist ein Austrittsloch auf der anderen Seite. Das ist schon einmal ganz schlecht. Das wird keine RoutiNEUNtersuchung. Doch fürchtet Euch nicht, “

Naraven mochte ein quirliger Kerl sein, und er strahlte ganz und gar nicht die Selbstsicherheit aus, die Jarid von Reka oder Larissa gewöhnt war, aber nach kurzer Zeit fühlte sich Jarid auch in seiner Behandlung relativ sicher. Das, oder die Erlebnisse der vergangenen Tage holten sie endlich gemeinsam mit der Müdigkeit ein. Es fiel ihr immer schwerer, ihre Augen offen zu halten.

Naraven versicherte ihr, dass er vermutete, dass das kein schlechtes Zeichen war. Sie war schließlich müde, und die Kräuter und Pulver konnten im Schlafe ohnehin eine stärkere Wirkung entfalten. So schlug Jarid ihre Augen zu und ließ sich völlig erschöpft in die Dunkelheit sinken. Hin und wieder vernahm sie, wie aus weiter Ferne, ein Klirren eines Glasbehälters oder der stechende Geruch eines Krätuersuds. Dann holte sie der Schlaf wieder ein.

Und mit dem Schlaf kamen Träume aus ihrer Vergangenheit.\bigskip





\az{Jahr 65}


\textit{Unruhig tigerte Jarid vor dem Steinkreis auf und ab. Trieest lag auf einem verwitterten Steinquader und atmete flach. Links und rechts von ihn standen Meister Lifornus, ein sehr mächtiger Zauberer des Feuers, sowie dessen ehemalige Schülerin Tenaya, eine Wächterin des Feuers, die sehr deutlich ausgedrückt hatte, dass nicht mehr einem Zauberorden angehörte. Der kleine Flederfuchs Flaps flatterte fröhlich um sie herum. Trieest atmete immer unregelmäßiger. Jarid haderte mit sich selbst, ob sie die Untersuchung unterbrechen sollte.}

\textit{Dies war die Stätte der heiligen Flammen, und Mitglieder aller drei Barbaren-Stämme versammelten sich zu bestimmten Zeiten hier, um in die Bruderfeuer zu vereinen und in den Flammen Visionen der Zukunft zu erhalten. Doch ob sie hier wirklich mehr über Trieests Lavastein erfahren konnten? Lifornus war doch vor allem hier, weil er die Stätte untersuchen wollte, und half ihnen bloß, weil Tenaya ihn darum gebeten hatte.}

\textit{Trieest brüllte kurz auf. Der Lavastein in seiner Brust brüllte flammend mit und sandte einen kreisrunden Reif aus Feuer um Trieest herum. Tenaya lenkte den Feuerstrom sicher ab. Meister Lifornus schnalzte kopfschüttelnd mit seiner Zunge und legte Trieest beruhigend die Hand auf die Schulter.}

\textit{Dies war ein totes Ende. Trieest würde auch hier nicht seinen Prozess des Wandels beenden können, das wusste Jarid. Und diese Feuerzauberer waren ratlos. Oh, wenn sie nur Trieest beruhigen könnte, was seine Zukunft anging. Doch wusste sie nicht, was diese beinhielt. Wenn nicht einmal die Feuerzauberer aushelfen konnten, musste dieser elende unzerstörbare Stein in seiner Brust davon überzeugt werden, Trieest sein zu lassen. Und der Stein ließ nicht mit sich sprechen. Jarid fluchte.}\bigskip



\textit{Der Traum wandelte sich.}\bigskip


\az{Jahr 66}



\textit{Jarid quetschte sich an einer großen Türsteherin mit dunkelgrüner Haut vorbei in den schummrigen Schankraum. Ihre Informantin wiederholte besorgt ihre Anweisungen: „Blicke ihr nicht ins Gesicht. Mache keine hastigen Bewegungen. Trage ruhig und deutlich deine Wünsche vor. Nimm ihren Preis an oder nicht. Dann gehe wieder. Keine unnötigen Konversationen, keine Fragen. So kommst du unversehrt davon.“}

\textit{Jarid versicherte ihr, dass sie wisse, was sie tue. Sie tat es nicht. So oft musste sie Leuten versichern, dass sie wüsste, was abging, obwohl sie es nicht tat, insbesondere Trieest gegenüber. Trieest, der aktuell in irgendeiner Düne in der Roten Wüste Tulgors verdurstete.}

\textit{Sorgsam trat Jarid vor die kleine Temm. Diese hatte ihren kahlen Kopf unter einer Kapuze verborgen und saß in einer dunklen Ecke des Schankraumes, neben ihr eine großgewachsene Temm in einem langen Mantel, die die Arme verschränkt hielt. Eine Wache?}

\textit{Die kleine Temm war nur unter dem Decknamen Trortra bekannt. Wenige Jahrzehnte alt, beinahe noch ein Kind für eine Temm, und doch bereits eine legendäre Schmugglerin. Sie konnte angeblich jede Information beschaffen. Da der tulgorische Hüter der Zeit aus Tulgor fortgereist war, war Trortra laut Jarids Informantin ihre beste Hoffnung. Jarid hoffte, dass sie übernatürliche Wege hatte, ihr Wissen zu erlangen. Ein zweites Gesicht oder so. Nur so konnte sie darauf hoffen, Trieest rechtzeitig wiederzufinden.}

\textit{Jarid trat an ihren Tisch und räusperte sich. Keine Reaktion, weder von Trortra noch von der Wache. Sie senkte ihren Blick und sprach leise, doch betont:}

\textit{„Ich suche meinen Begleiter. Trieest. Ein danwarischer Feuerkrieger mit einem magischen Lavastein in seiner Brust. Wir sind hier, um die tyrannische Goldene Fürstin zu stürzen. Wir wurden in einem Sandsturm in der Roten Wüste getrennt. Man sagte mir, Ihr könntet aushelfen.“}

\textit{„Was ist dein Wunsch, konkret?“, fragte die Wache ungehalten.}

\textit{„Ich wünsche mir, zu erfahren, wie ich Trieest wiederfinden kann.“}

\textit{Stille. Nach einer gefühlten Ewigkeit griff die kleine Trortra nach einem Zettel, kritzelte etwas darauf und reichte es ihrer Wache. Diese las deutlich vor: „Der Preis sind zwei Monde in unseren Diensten, für dich und deinen Feuerkrieger.“}

\textit{Jarid verengte ihre Augen. Mit so etwas hatte sie nicht gerechnet. „Was bedeutet dies? Welche Art von Diensten?“}

\textit{Die Wache antwortete nicht.}

\textit{Jarid starrte unmutig von der Wache zu Trortra. Trieest hatte keine Zeit für solche Spielchen!}

\textit{Die Wache verschränkte ihre Arme und sprach: „Wenn du den Preis nicht zahlen willst, dann gehe. Jetzt.“}

\textit{Jarid schluckte schwer. Sie ging die Warnungen ihrer Informantin im Kopf durch und wandte sich niedergeschlagen zum Weggehen. Dann schüttelte sie ihren Kopf, drahte sich abrupt zurück und griff über den Tisch nach Trortras Handgelenk. Die Wache sprang vor, doch Jarid lenkte mit ihrer freien Hand ein Rinnsal an Bier in ihre Augen. Nur genug, um sie einen Moment abzulenken.}

\textit{„Bitte, hilf mir“, sprach Jarid zur Temm, „Trieest und ich können Euch allen hier helfen. Wenn die die Goldene Fürstin erst einmal gefallen ist, werdet ihr ...“}

\textit{Die Kapuze der kleinen Trortra fiel zurück. Ihre Lippen zitterten, ihre Augen blickten furchterfüllt zu Jarid hoch. Die kleine Trortra hatte Todesangst, und Jarid bedachte, dass sie diese Situation vielleicht falsch eingeschätzt hatte.}

\textit{Lautlos formte der Mund der Kleinen Worte, die Jarid nicht verstand. Dann kritzelte sie mit ihrem Federkiel etwas auf Jarids Handgelenk. Jarid nickte ihr dankbar zu und ließ sie los.}

\textit{„Ich gehe, ich gehe. Verzeiht mir.“}

\textit{Die großgewachsene Temm-Wache scheuchte sie zurück. Kurz schien sie zu bedenken, ob sie Jarid mit ihren Fäusten bekannt machen sollte, dann allerdings eilte sie stattdessen zu Trortra zurück und streichelte diese beruhigend.}

\textit{„Herrjemine, was hat sie gesagt? Hat sie dir etwas angetan? Wie geht es dir?“}\bigskip



\textit{Der Traum wandelte sich.}\bigskip



\az{Jahr 67}



\textit{Verzweifelt schöpfte Jarid mit ihrer Magie Wasser aus den Fluten der Narne und lenkte sie auf den Mann, der zappelnd vor ihr lag. Wilselm, der alte Wolfskrieger, der seinen Arm schon vor Jahren im Kampf gegen den Schwarzen Herold verloren hatte. Ein begabter Schildzwerg hatte ihm eine raffinierte Prothese aus einem seltenen Erz geschaffen, doch selbige schien zu malfunktionieren. Ätzende Flüssigkeiten drangen aus seinem metallenen Arm hervor und gruben sich zischende Bahnen seine Schulter entlang. Erneut brüllte Wilselm auf und wälzte sich auf dem Boden. Sein Metallarm zuckte und streifte Trieest, welcher den Schlag kaum registrierte.}

\textit{„Fixiere ihn!“, sprach Jarid selbstsicherer, als sie sich fühlte, „Wir müssen das Konstrukt abtrennen!“}

\textit{Trieest grunzte bestätigend, zog mit einem wohlklingenden SLING sein Rankenschwert und ließ die Ranken wie tänzelnde Flammen über Wilselms Schulter züngeln, sich an ihr festsetzen. Wilselms metallener Arm schlug aus, traf sein Ziel und schleuderte Trieest rückwärts. Dann, ehe sie reagieren konnte, hatte er auch schon nach Jarids Kehle gegriffen. Und zugedrückt.}

\textit{Jarid japste vergeblich nach Luft, schlug vergeblich auf den Metallarm ein, schrie vergeblich nach Trieest ...}\bigskip





\az{Jahr 68}

Jarid schlug ihre Augen auf und sog gierig Atemluft ein. Ein Geruch nach vielseitigsten Kräutern und Suden lag in der Luft, doch konnte diese den Gestank nach nassem Fell nicht ganz überdecken. Es war dunkel.

Sie brauchte einen Moment, um sich zu orientieren. Jarid befand sich nicht in Tulgor, nicht im Steppenland, und auch nicht an Trieests Seite in Andor. Sie befand sich beim Alchemisten Naraven, in einer versteckten Hütte mitten im Südlichen Wald. Sie war verletzt. Und ... Trieest war mit einer verdammten Skral-Horde allein!

Jarid schreckte in die Höhe.

Neben ihr schreckte Naraven ebenso schnell von seinem Tisch auf, sah sich wild um und brauchte eine ganze Weile, um sich wieder einzukriegen. „Bei der heiligen Mutter, habt Ihr mich aber erschreckt!“

Er atmete durch und ratterte los: „Willkommen zurück in der Welt der Wachen, Jarid. Ihr wart etwas fiebrig. Wisst Ihr noch, wo Ihr seid? Wisst Ihr noch, wer ich bin? Ich bin Naraven, der Alchemist aus dem Südlichen Wald, und Ihr befindet Euch in meiner Hütte.“

Er verharrte kurz. „So allein und verletzt in der Hütte eines Fremden in einem fremden Land zu sein, ist wohl nicht allzu beruhigend. Ich will Euch versichern, dass ihr hier sicher seid und jederzeit gehen könnt, ohne Euch erklären zu müssen. Die Tür steht offen und die Taverne ist nahe. Ich holte bei Gilda sogar Heilmittel, frisches Brot und gut gezapften Met für Euch. Die gute Gilda erzählte mir ja so einiges über Euch. Kennt alle Gerüchte der Gegend so gut wie ein Kind seinen SchNULLer. Eine Heldin von Andor seid Ihr? Eine Danware in den ehrenwerten Diensten dieses Reichs, die den eitlen König Thorald in bessere Bahnen zu lenken vermag? Wie großartig! Doch rede ich schon wieder zu viel. Sagt, Jarid, wie fühlt Ihr Euch?“

Jarid wollte zu einer positiven Antwort anstimmen, aber Naraven fuhr gleich fort: „Ganze drei Tage habt Ihr im Reich der Träume verbracht. Ich habe mein Bestes gegeben, die Bauchwunde zu reinigen und Euch zu stärken. Aber dass Ihr so lange nicht mehr aufgewacht seid, hat mich schon besorgt. Wie viele Finger halte ich hoch?“

Naraven hob eine Hand mit drei ausgestreckten Fingern in die Höhe, was Jarid ihm mitteilte. Zufrieden nickte Naraven und redete dann gleich weiter:

„Ich musste Euch eine Rippenspitze entnehmen. Das Ding war abgebrochen und hätte für schlimmste Entzündungen gesorgt.“

Naraven griff zielsicher auf ein silbernes Tablett mitten im ganzen Sammelsurium und hielt eine kleine, gelblich-weiße Knochenspitze herum, welche Jarid gar nicht so recht anschauen wollte. Knochen gehörten innerhalb eines Körpers, nicht außerhalb.

„Ihr habt im Schlaf gemurmelt. Scheint einige Abenteuer erlebt zu haben“, fuhr Naraven eifrig fort, „Gestattet Ihr mir, mithilfe dieser Knochenspitze in Eure Vergangenheit zu spähen und diese Abenteuer zu dokumentieren, falls ich die Zeit finden sollte? Ihr werdet davon nichts spüren, geht alles über den Knochen. Viel zu wenig Erlebnisse aus dieser und vergangenen Zeiten wurden bislang auf danwarischen Steintafeln für die Nachwelt verwahrt. Ein einziger Feuersturm könnte das Wissen der Bewahrer vom Baum der Lieder auf einen Schlag vernichten. Das wollen wir doch verhindern. Erst recht in Bezug auf die Abenteuer der Helden von Andor. Erst recht, wenn Danware dabei sind.“

Jarids Kopf schwamm. „In meine Vergangenheit blicken? Das könnt Ihr?“

„Oh, nicht ich allein“, grinste Naraven, „Aber sehr wohl ein gewisser magischer Spiegel aus Cavern, ersteigert aus der Krimskramskammer von Fürst Hallwort, deren Inhalt vor einiger Zeit endlich versteigert wurde. Ich erzähle sehr gerne mehr davon, habe jedoch die Erfahrung gemacht, dass nicht alle den Enthusiasmus dafür teilen.“ Naraven deutete auf einen großen Rundspiegel, welcher achtlos in eine Ecke der überfüllten Stube gestellt worden war. Ein schlichtes Ding, dessen silberner Rand mit komplizierten Runen versehen war, welche Jarid nicht verstand. Seltsamerweise zeigte das Spiegelbild nicht eine Reflektion der Stube, in der Jarid und Naraven sich soeben befanden, sondern eine verschwommene gehörnte Gestalt, welche an einen großen Troll erinnerte, mit einem winzig wirkenden Rundschild in ihrer groben Hand.

Sie beschloss, dass dies nicht ihre Priorität war, und murmelte bloß: „Tut mit meinen Knochen, was Ihr wollt, haltet sie bitte bloß aus meinem Blickfeld.“

Naraven bedeckte Jarids Rippenspitze rasch mit einem Stück Stoff. Dann schlug er sich an die Stirn und rief: „Ah, wie konnte ich das vergessen? Da schwebt schon seit Stunden so eine komische blaue Blume um meine Hütte herum. Ich habe sie nur aus der Ferne gesehen, doch sie scheint aus Wasser zu bestehen. Eine fliegende wässrige Blume, in der Form einer Wasserlilie, welch Wunder der Wassermagie! Habt Ihr etwas damit zu tun?“

„Eine Brieflilie?!“, rief Jarid, „Die muss für mich sein. Seltsam, dass sie mich nicht sofort gefunden hat, eigentlich sollten Brieflilien ihre Zielperson direkt ansteuern können.“

Naraven schien nachfragen zu wollen, was eine Brieflilie sei, unterbrach sich dann und meinte: „Es ist schon seltsam genug, dass dieses Wesen überhaupt in diese Nähe fand. Ein Tarnzauber liegt über meiner Hütte. Es ist äußerst schwer, sie zu finden.“

„Eine Brieflilie ist doch kein Wesen, sondern nur eine Botschaft! Eine Botschaft eines anderen Wassermagiers. Sie muss meiner Spur gefolgt sein, bis ich in die Reichweite des Tarnzaubers gelangte. Was ist das überhaupt für ein Tarnzauber? Dieser Junge hatte doch kein Problem damit, die Hütte zu finden.“

„Ah, ja. Der kleine Soraf ist einer der wenigen, die diese Hütte auf natürlichem Wege zu finden vermögen. Könnte etwas mit seinem fehlenden Sinn zu tun haben? Oder er hat einen besonderen Blick auf magische Ströme und dergleichen, das gibt es manchmal auch bei Menschen. Bei manch anderen Tieren kann dies ebenfalls auftreten. Ich selbst halte mir eine Schar verschiedenster Haustierchen, die mich im Notfall hierher zurückbegleiten könnten. Ohne Hilfe finde ich sie selbst nicht.“

Jarid zog eine Augenbraue hoch. „Welche Sorte Magier verhängt einen Tarnzauber über seine Hütte und braucht danach die Hilfe von Haustieren, um sie wiederzufinden?“

Naraven gluckste, doch seine Augen lachten nicht mit. „Für einen Magier haltet Ihr mich? Nein, ich bin nur ein Pechvogel, der sich einmal zu viel dazwischen gestellt hatte, als sich an dieser Stelle ein paar launige Waldgeister mit ein paar Feen angelegt hatten. Ich meine, gut für sie, die Waldgeister und Feen dieser Gegend verstehen sich seit diesem Vorfall wieder prächtig. Aber nur, weil sie ihre Wut an mir ausgelassen hatten. Die ‚Späßchen‘, die die sich erlauben ... ‚In einigen Jahrhunderten können wir wieder darüber reden, ob du dein Haus wieder finden darfst. Bis dahin bedenke, was für ein böser Bube du warst!‘ Feen! Keine Vorstellung davon, wie lange ein Mensch lebt. Vielleicht haben sie einfach gar keine Vorstellung davon, was Zeit ist.“

„Und Ihr habt nie etwas dagegen unternommen?“

Naraven gluckste. „Waldgeist- und Feenzauber zu lösen? Sehe ich aus wie jemand, der Problemen hinterherläuft? Nein, nein, es ist schon in Ordnung so, wie es ist. Für mich allein hier in diesen Gemäuern zu forschen ist alles, was ich brauche. Nebst Nahrung natürlich. Aber auch davon wächst hier im Wald jede Menge. Ich habe meinen eigenen Gemüsegarten. Die Erde hier ist so viel nahrhafter als der trockene Boden Danwars. Dort mussten wir von glitschigem Seetang und glitschigeren Fischen leben, hier gibt es hingegen saftige Möhren und Unmengen an Apfelnüssen. Von denen kann ich gar nicht genug kriegen!“

Naraven plapperte noch eine Weile so fröhlich vor sich hin, und Jarid hörte ihm weiterhin mit einem Ohr zu, doch gleichzeitig probierte sie auch, sich von ihrem Tisch zu erheben.

Sie stolperte, doch blieb schwankend stehen. Naraven unterbrach sich und eilte geschwind mit einer klappernden Schüssel voller Apfelnüssen und einer zweiten mit einer orangenen Suppe – Möhrensuppe? – zu ihr.

„Sachte, sACHTe, Ihr müsst euch stärken, ehe Ihr Euch erhebt. Wollt Ihr gleich wieder kollabieren? Und achtet bitte auf den Verband!“

Jarid blickte auf den fleckigen Verband, der ihren Bauch umwickelte. Darunter saftete irgendeine milchig-grünliche Flüssigkeit hervor. Ihr wurde schwindlig und sie stützte sich wieder auf den Tisch.

Sie schüttelte ihren Kopf: „Schön und gut, dass ich auf mich achten muss, aber mein Begleiter steckt in Lebensgefahr und da ist eine Brieflilie mit meinem Namen drauf, die diese Hütte umschwebt. Gleich nachdem ich mich gestärkt habe, würde ich gerne nach draußen und mir diese ansehen.“\bigskip







Es war gar nicht Nacht, wie Jarid ob der dunklen Stube gedacht hatte. Sobald sie die Türe zu Naravens Hütte öffnete, strahlte gleißendes Sonnenlicht herein.

Naraven erklärte: „Entschuldigt die Dunkelheit im Innern. Manche Tränke werden durch Sonnenlicht verdorben. Und es ist manchmal schwer zu denken, wenn einem zu viel Licht entgegenströmt. Wie manche Pflanzen in der Nacht am besten wachsen, so gedeiht auch manches Wissen in Finsternis am besten.“

Tatsächlich umschwirrte eine wässrige Wasserlilie die Hütte in unregelmäßigen Schleifen. Sie war vielleicht so groß wie Jarids Kopf und fast durchscheinend. Kaum war Jarid einige Schritte vor Naravens Hütte gestolpert, rieselte die durchscheinende Erscheinung vom Himmel herab und blieb knapp über Jarids Kopf stehen. Jarid griff mit ihrer Magie danach und die Lilie entrollte langsam ihre Blütenblätter, bis Jarid den blassen Text darauf erkennen konnte:\bigskip



\textit{Liebe Jarid,}



\textit{Ein Falke der Gastwirtin Gilda erreichte mich und besorgte mich ganz gewaltig. Du seist verletzt im südlichen Rietland aufgetaucht, von Trieest keine Spur. Ich setzte bereits an, in deine Nähe zu tunneln, doch anscheinend ist der Brunnen in Thorns Dorf geleert? Ach, was habt ihr beide nur wieder angestellt? Eara meinte, der Heldenorden hätte von euch zuletzt auf der Jagd nach dem wilden Hraak im östlichen Rietland gehört. Geht es dir gut? Wie kann man dir helfen?}

\textit{Ich hatte ja gehofft, euch bei der neunten Gedenkfeier zum Andenken der heroischen Gefallenen der ersten Befreiung der Rietburg zu treffen. Die Vorbereitungen sind in vollem Gange. Der Stein der Erinnerungen ist bereits geschmückt. Es sieht leider so aus, als würde der Schnee dieses Jahr erst später fallen. Dafür gibt es wundervolle Neuigkeiten. Rate mal, auf wen Kar éVarin und ich während den Vorbereitungen stießen: Pyros! Pyros, den legendären Glutträger, den du so bewunderst! Er wird bei der Gedenkfeier anwesend sein, um sich mit den tulgorischen Diplomatinnen auszutauschen, ob das Land der tausend Flammen weit im Westen liegen können.}

\textit{Wir können kaum erwarten, welche Plattitüden König Thorald dieses Jahr den heroisch Gefallenen auftischen wird. Wir hoffen sehr, dass ihr kommen könnt! Wenn nicht, und wenn ihr Hilfe braucht, melde dich bitte blütenwended! Fast alle anderen Helden werden anwesend sein, wir können helfen!}



\textit{Mögen deine Wasser frisch bleiben!}

\textit{Deine Base Jirid}\bigskip



Jarids Gedanken rasten, während sich die Brieflilie in kleine Tröpfchen auflöste.

Jirid hatte also die Rietburg aufgesucht, gemeinsam mit Kar éVarin, diesem Feuerdämon, dieser „Lebenden Flamme“ aus der Hadrischen Unterwelt, die Jirid einzudämmen und zugleich zu nähren ersuchte, schon seit sie eine Novizin gewesen war und ihn in eine gestohlene Feuerkrieger-Rüstung gepackt hatte. Riesig war dieser Kar, vernarbt und mit einer kratzigen Stimme, aber mit einem weichen Kern. Er war stark, sehr stark, aber seine wahre Stärke lag im Heilen, nicht im Verletzen. Jirid musste ihm helfen, seine Wut zu bezwingen, seine Stärken auszunutzen und ihn von seinem Rachedurst abzulenken. Eine Beziehung, nicht ungleich jener, die Jarid und Trieest hatten. Letztere hatten Kar bei der Kontrolle seiner flammenden Wut zu helfen versucht. Im Gegenzug hatten Jirid und Kar Trieest beim Ablegen seiner Bürde zu helfen versucht. Doch weit waren sie allesamt nicht gekommen.

Jirid war zu wagemutig. Eines Tages würde sie das teuer zu stehen kommen, befürchtete Jarid. Andererseits hatte Jirid dasselbe zu ihr über ihre Reisen ins Land der drei Brüder und nach Tulgor gesagt. Wer konnte schon wissen, was die Zukunft brachte?

Und nun waren die beiden offenbar auf Pyros gestoßen. Pyros, den legendären Glutträger des Glühenden Herzens des Flammenden Gottes, des größten und reinsten Lavasteins Danwars. Pyros hatte wie alle Glutträger dreizehn Jahre Zeit, um das sagenumwobene Land der Tausend Feuer zu finden, ehe sein Körper zu Asche zerfallen und das Glühende Herz zurück ins Weiße Feuer Danwars springen würde. Jarid rechnete kurz durch. Fünf Jahre mussten bereits vergangen sein seit Pyros‘ Ernennung, und er war dem Land der Tausend Feuer wohl noch keinen Schritt näher gekommen. Dennoch hatte er die Hoffnung nie aufgegeben. Solche an Obsession grenzende Aufopferungsgabe hatte Jarid stets bewundert.

Sie fühlte sich auf einmal ein wenig erleichtert. Welche Bürde Trieest auch zu tragen hatte, Pyros ging es umso schlimmer. Dreizehn Jahre hatte er, keinen Moment mehr, ehe das Glühende Herz des Flammenden Gottes ihn auslöschen würde, weil er eine vermutlich unlösbare Aufgabe nicht zu bewältigen vermochte. Doch Pyros haderte nicht damit.

Die Rote Prophezeiung, die Worte des Lichts, die den Uralten vor Jahrhunderten in der Roten Grotte mitgegeben worden waren, sie waren längst nicht mehr eindeutig zu interpretieren, zu verwaschen durch vielfache Tradierung und Debatten über ihre Bedeutung. Jarids Prophezeiung der Roten Grotte war immerhin ein konkreter Appell gewesen. Konnte sie hoffen, dass dies alles bald hinter ihr läge?

Jarid hatte in ihrer Zeit in Danwar so einige Feuerkrieger einen Lavastein als Bürde der Schuld tragen, die Schuld begleichen und den Lavastein wieder ablegen sehen. Trieest Lavastein hingegen war nie glücklich mit ihm, quälte ihn stets weiter. Was war anders in Trieests Fall?

Vielleicht lag es daran, dass Trieest jemanden ... ja, sie musste es eingestehen, er ihn beinahe umgebracht. Jormudd. Dieser arme Junge. Die Ältesten hätten es kommen sehen sollen, Trieest hatte sich schon die ganze Woche seltsam verhalten, und als dieser Junge sich mit ihm verstritten hatte, da war Trieests Instinkt aus ihm herausgebrochen, und er hatte sich auf den Jungen gestürzt, hungrig Zähne und Klauen in ihn gegraben. Erst als man ihn von Jormudds Körper weggezogen hatte, hatte Trieest sich mit einem erschreckten Blick in seinen weißen Augen beruhigt und zu weinen begonnen. Jormudd hatte auch geweint. Sein Ohr hatte man bis zum heutigen Tag nicht mehr gefunden.

Einmal halbe Kreatur, immer halbe Kreatur, so sagten die Ältesten. Diese sturen Böcke!

Trieest war damals doch selbst noch mehr Kind als Erwachsener gewesen, und für das Blut in seinen Adern konnte er wahrlich nichts. Dennoch war das Urteil des Rats eindeutig: Trieest musste einen Lavastein als Bürde tragen, bis seine Schuld bereinigt war (soweit ein ganz gewöhnliches danwarisches Verfahren). Und bis Trieest seine Schuld beglichen hatte, wäre er aus Danwar verbannt. Das war einzigartig. Nebst Glutträgern verließ sonst so gut wie nie ein Feuerkrieger seine Heimatinsel.

Jarid war neben Trieests Mutter Talemma die einzige gewesen, die mit dem kleine Trieest Mitleid gehabt hatte. Bei Mutter Natur, sie waren Wassermagier! Ein Orden, der für die Armen und Schwachen einstehen sollte, sich nicht gegen sie richten!

Oft war sie für Trieest eingestanden, und sie würde es wieder tun!

Jirid hatte recht, die Gedenkfeier an der Rietburg war tatsächlich ein passender Ort, um andere Helden zusammenzutrommeln und zu Trieests Rettung aufzubrechen. Während Jarid Naraven dazu ausfragte, wurde rasch klar, dass die Zeit drängte: Die diesjährige Gedenkfeier fand heute Abend statt!

Die jährliche Gedenkfeier an die bei der ersten Befreiung der Rietburg aus Varkurs Klauen gefallenen Krieger. Dort würden nicht nur Jirid, Kar und Pyros anwesend sein, nein, dort befanden sich vermutlich sogar die meisten anderen Helden, um ihre jährlichen Lobpreisungen einzuheimsen. Wenn Jarid bis zur Rietburg reisen könnte, könnte sie Hilfe für Trieest holen. Bestimmt war dort jemand, der den Lavastein anpeilen konnte. Der Trieest finden und hoffentlich rechtzeitig retten könnte.

Entschlossen sprach Jarid: „Naraven, ich werde Sie verlassen. Mein Gold musste ich leider weiter weg zurücklassen, als ich hierher getunnelt bin. Aber ich werde zurückkommen. Ich werde meine Schuld abbezahlen. Wenn das alles vorbei ist, werde ich wiederkehren und mich für Eure Dienste revanchieren.“

„Abgesehen davon, dass das nicht nötig wäre, habt Ihr Euch doch schon längst revanchiert“, grinste Naraven und deutete auf das Stück Stoff, unter dem wohl immer noch Jarids Rippenspitze lag. „Damit werde ich zahlreiche Aufzeichnungen für die Nachwelt festhalten können.“

Jarid zuckte mit den Schultern. „Wie Ihr wollt. Dann könnt Ihr mir noch ein letztes Mal helfen? Sagt, wo liegt von hier aus die nächste Wasserquelle?“

„Ich habe ja schon einmal einen Becher Wasser angeboten“, meinte Naraven verschmitzt.

Jarid lächelte schief. „Ich benötige leider nicht bloß einen kleinen Becher von Wasser. Meine Kapazität ist zu erschöpft für große Wunder. Ich brauche eine große Ansammlung von Wasser, idealerweise mit Verbindung zum Grundwasser. Ein Brunnen oder etwas Vergleichbares. Der einzige mir bekannte Brunnen hier in der Nähe wurde zu einem großen Teil verdampft, als ich hier aufgetaucht bin. Es wird bestimmt noch einen ganzen Tag gehen, bis er wieder zur Genüge gefüllt ist.“

„Euer Zeitgefühl unterschätzt die Dauer Eurer Ohnmacht. Ihr seid doch schon mehrere Tage hier. VerZWEIfelt nicht, dieser Brunnen ist schon längst wieder gefüllt. Oder ... wartet mal, er wäre es zumindest. Heute ist ja die diesjährige Gedenkfeier an die Toten, die vor neun Jahren ihr Leben für so viele heute Lebende gaben. Ich nehme an, dass einige andere Helden von Andor wie üblich auf dem Weg dorthin beim Brunnen durchgekommen, sind, und die Helden lassen bekanntlich keinen Brunnen ungeleert. Aber verzagt nicht“, grinste Naraven, „Ich hätte da vielleicht etwas, das den Brunnen wieder auffrischen könnte.“

Rasch rannte er in ein Hinterzimmer. Jarid hörte es für einige Minuten rumoren, bis der kleine Alchemist reemergierte und triumphierend einen grünlich schimmernden Meißel präsentierte.

„Den habe ich aus dem Nachlass von Runenmeisterin Burmrit der Silberzwerge ersteigert“, meinte er stolz. „Hat mich ein Vermögen gekostet, doch ich dachte, dass es eines Tages nützlich sein könnte. Los, auf zum Dorf!“\bigskip






„Lasst es mich ein letztes Mal überprüfen.“

Zum dritten Mal in den letzten zehn Minuten raschelte Naraven durch den Stapel loser Pergamente, den er mitgebracht hatte. Sie zeigten verschiedenste Runen unterschiedlicher Komplexität mit kleinen Erklärungen zu ihrer Bedeutung.

„‚Wasser‘“, murmelte Naraven, „Ja, diese Rune für ‚Wasser‘ sollte doch genügen. So nutzten sie bereits die Urahnen der Schildzwerge für manche wässrigen Zwergentüren. Natürlich für ganz andere Zwecke, doch die Runen selbst sollten dieselben sein.“

Naraven kniete sich neben dem leeren Brunnen hin und haute mit dem grünlich schimmernden Meißel drei gerade Striche in den staubigen Bogen: Zwei Striche parallel zueinander, den dritten quer über die anderen.

„So. Nun brauchen wir nur noch eine Kraftquelle. Eine Zauberin könnte der Rune mit einem Spruch Kraft verleihen, oder eine Runenmeisterin natürlich mit einer Runenquelle. Ein Artefakt, in dem einige Runenmagie durch Mondlicht festgehalten wurde, würde auch gehen. Oder etwas, dass die hier immer noch allgegenwärtige uralte Drachenmagie des Landes anzapft. Aber da wir das alles nicht haben ... holde Jarid, habt Ihr schon jemals versucht, eine Wasserrune auszulösen?“

Jarid schüttelte bloß den Kopf.

„Versucht es. Als Wassermagierin solltet Ihr eigentlich leicht dazu in der Lage sein.“

„Wie Ihr wollt. Doch bin ich nicht zuversichtlich.“

Jarid schloss ihre Augen und hörte in sich hinein, versuchte, sich das Rauschen des danwarischen Meeres in den Sinn zu rufen. Sie versenkte sich im Geiste in den Boden, spürte das Wasser, das ihn durchsickerte, ebenso wie der Tau auf dem darüber liegenden Gras.

Überrascht bemerkte sie, dass sie auch die Rune spürte, die Naraven in den Boden gehauen hatte. Unglaublich stark, sogar. Sanft befühlte sie mit ihrem Geiste das fremdartige Objekt, spürte, dass etwas fehlte, nein, etwas nur leicht am falschen Ort war. Sie übte Druck auf die ungleiche Stelle aus, etwas klickte, und ...

„Wasser“, sprach Jarid.

Die Rune glühte blau auf und eine drei Meter hohe Fontäne reinen Wassers schoss aus dem Brunnen hervor.

Naraven quiekte vor Vergnügen.

Jarid leitete das Wasser sanft in die Höhe, bis ein halbwegs ordentlicher Wasserstrudel neben dem Brunnen schwebte und stetig flacher wurde. Naravens Augen leuchteten.

Jarid spreizte ihre Finger und versenkte ihre Hand bis zum Ellbogen in der Fläche. Sie trat nicht wieder auf der anderen Seite aus. Dafür verspürte Jarid eine angenehme Wärme ihren Arm entlangströmen. Die goldenen Linien auf ihrem Gewand leuchteten auf.

Einen letzten Blick auf Naraven werfend, sagte Jarid:

„Ich kann Euch nicht genug für Eure Hilfe danken. Gehabt Euch wohl, Naraven. Lila blühen die Blumen auf der Asche.“

„Lila glühen die Augen der Knochen in der Asche“, entgegnete Naraven geistesabwesend. Er starrte weiterhin mit großen Augen auf das Wasserportal. Jarid wandte sich auch ebenfalls diesem zu. Der Strudel hatte sich inzwischen gelegt und die Fläche lag ruhig wie eine stille Teichoberfläche da, einfach senkrecht statt waagrecht. Jarid dirigierte die Scheibe zu Boden und gebot ihr, ein Tor zu formen Dessen Ränder begannen bereits, einzufrieren. Rasch trat Jarid einen Schritt nach vorne und spürte die vertraute Wärme, die sie durchströmte, als ihr Körper und ihr Gewand sich verwässerten und davontragen ließen, während das Tor hinter ihr vereiste.

Ein leiser Plumps ertönte, als ihr blutiger Verband durchnässt neben den eisigen Torbogen fiel. Diesen konnte sie natürlich nicht mitnehmen durch das Portal, dachte Jarid. Sie musste zugleich schmunzeln als auch sich über sich ärgern, dass sie das vergessen hatte.

Dan(n) war sie eins mit dem Wasser.\bigskip







Wasser dachte anders als Menschen. Wasser dachte auch anders als Zwerge und Riesen, Temm und Taren. Manche behaupteten, dass es ganz und gar nicht dachte. Diejenigen, die dies denken, haben noch nie einen Wassergeist getroffen. Aber das ist Nebensache.

Jarid dachte immer anders, wenn sie eins mit dem Wasser war. Nicht wie ein Mensch, aber auch nicht wie ein Wassergeist. Wie Wasser, halt.

Als Mensch war es manchmal schwierig, komplizierte Entscheidungen zu treffen, überwältigende Emotionen zu spüren und nie zu wissen, wohin alles hinführte. Im Wasser war alles klarer. Alles war einfacher in dieser Wassergestalt, alles war im Gleichgewicht. Alles hatte ein klares Ziel, und alles folgte diesem Ziel, diesem einen Ziel. Alles floss dorthin, wo es sollte, und es war gut so. Und Jarid floss mit allem dahin, zufrieden und glücklich. Sie wusste, wo sie hinwollte, sie wusste, wo sie hinsollte, und dorthin würde sie gehen, fließen, plätschern, und das war gut. Alles war gut.

Der Brunnen vor der Rietburg war so nahe, sie konnte ihn buchstäblich spüren. Sie vernahm, wie das Wasser in seinem Innern zu dampfen begann und erste Blasen sich daraus hochlösten, ja, sie war eins mit dem brodelnden Wasser im Brunnen, als die vertraute Kälte durch das Wasser floss, das sie ja war, und das war nicht unangenehm oder angenehm, es war einfach. Und das war gut. Sie würde dort ankommen, wo sie hatte ankommen wollen.

Doch plötzlich war das nicht mehr alles, was da war. Ihr Fluss, der eigentlich zur Rietburg hätte strömen sollen, änderte seine Richtung. Das war nicht so geplant. Aber Jarid, die eins mit dem Wasser war, brauchte das nicht zu kümmern. Wenn der Fels sich verschob, so ging das Wasser einen neuen Weg. Und so floss Jarid halt sorglos davon, weg von dem Brunnen, weg von der Narne gar. Jarid war Teil des Grundwassers unter der andorischen Erde, Teil der Wassertröpfchen in der Luft und Teil des gigantischen Ozeans, der nördlich von hier lag. Und dorthin floss sie nun, zu diesem gigantischen Ozean zog es sie, denn dorthin wurde sie gezogen, und wie alles Wasser floss sie dorthin, wo der Sog sie hinzog.

Weg von der Küste Sidra und dem Kontinent, weg von den Brunnen des Südens, auf, in den Nordosten.

Jarid wusste von den Gefahren, die darin lagen, durch das offene Meer zu tunneln. Nicht wenige Wassermagier waren dort schon verschollen gegangen. Es war schwer genug, sich einen geistigen Pfad durch das Wasser bis zum Zielort zu schaffen, denn wenn man zu lange durch das Wasser reiste, so wurde man endgültig Teil des Wassers und konnte sich nicht mehr von allein daraus lösen. Doch Jarid kümmerte das nicht. Sie war Teil des Wassers und floss dorthin, wo sie gezogen wurde. Und das war gut.

Langsam spürte sie ihren neuen Zielort näher kommen, ein kleines Felsmassiv inmitten des kalten Ozeans, unter dem es brodelte und sprudelte. Als Mensch hätte sie sich wohl Sorgen gemacht, wer hier an ihr zog, wäre vielleicht gespannt gewesen, wen sie treffen würde, hätte sich dem eventuell gar widersetzt. Doch sie war eins mit dem Wasser und sie floss, wie der Fels ihr gebot.

Und der Fels zog sie nach Danwar.

Danwar, die Insel der widerstreitenden Elemente, ragte wie eine steingewordene Flutwelle aus der stürmischen See. Die aufbrausenden Wellen schlugen selbst in das geschützte Becken unter dem Fels und brachten die kleinen Fischerboote ins Schaukeln. Regen verdampfte auf dem heißen Gestein und peitschte gegen die windschiefen Hütten auf dem hochgelegenen Plateau des Eilands. Die Siedlung auf dem Plateau des Gipfels lag zur Hälfte auf einem Felsüberhang, der aussah, als müsse er jeden Moment unter der Last des Steins zusammenbrechen.

Schon bald konnte Jarid das große Becken von Quodlon am Rückgrat Danwars spüren, zu dem sie gezogen wurde. Jarid fühlte ein vertrautes Kribbeln in ihrem liquidierten Körper, das sie schon lange nicht mehr verspürt hatte. Etwas Fremdes brodelte unter Danwar. Der schwarze Fels, aus dem der Kern Danwars bestand, stammte nicht von dieser Welt. Zumindest sagten das die Ordensmeister des Nachthimmels, die mussten solche Dinge doch wissen. Und als Danwar damals in den Urzeiten vom roten Mond gefallen war, war das Massiv laut ihnen noch ein einziger steinerner Klotz gewesen. Nun saß dieser schwarze Kern Danwars nicht nur tief verankert im Meeresgrund, sondern war auch umgeben von dicken Schichten Lavagesteins, welches der durch den Einschlag ausgelöste Vulkan über Jahrtausende und Abertausende angesammelt hatte. Dieses Gestein und der schwarze Kern Danwars waren beide durchzogen von verschiedenen Höhlen und Gängen. So kamen unter anderem die beliebten Quellbäder Danwars zustande. Und das große Becken von Quodlon war das größte dieser Quellbäder, welches am unteren Ende der ‚Wirbelsäule‘ Danwars lag.

Fünf Wassermagier hatten sich hier und heute in regelmäßigen Abständen um das Becken versammelt und tanzten rhythmisch um es herum, während das Wasser im Becken nach ihren Geboten wogte und Jarid zu sich rief.

Für einen kurzen Moment war Jarid die Schweißtropfen auf den Stirnen der Magier und hätte sich beinahe gefragt, warum sie schwitzten. Es war warm nahe der Quelle, ja, aber konzentrierte Wassermagier hatten andere Wege, sich abzukühlen. Der Schweiß zeugte vielmehr davon, dass sie aufgeregt waren, abgelenkt, ja, gar ängstlich.

Die warme Quelle kochte auf. Jarids Körper und Kleid materialisierten sich, schossen aus dem Becken von Quodlon in die Höhe und wurden von den erleichtert aufatmenden Wassermagiern ans Ufer geleitet.

Zitternd vor Kälte sackte Jarid auf dem warmen Felsen zusammen, während ihr Körper verzweifelt protestierte. Er wollte zurück ins Wasser, wollte weiterfließen. Das hier war ein Widerstand gegen den natürlichen Fluss, das hier war \textit{falsch}. Dann war auch schon die erste Wassermagierin bei Jarid. Sie verjagte alle Nässe aus Jarids Gewand und gebot dünnen Wasserbändern, sich um Jarid Handgelenke und Ohren zu schlingen. Wärmebänder. Es zischte und gluckerte, als die Bänder sich erhitzen und die wohlige Wärme sich in Jarids Körper ausbreitete.

Jarids aufgewühlter Geist konnte sich aber nur für einen kurzen Augenblick beruhigen, denn nun traten gewöhnliche, nicht-wässerige Sorgen an ihr Bewusstsein.

Wer hatte es gewagt, ihre Wasserreise durch Andor aus der Ferne zu unterbrechen? Wer hatte es gewagt, sie hierher umzulenken? Wer hatte es gewagt, sie durch den Ozean zu zerren?! Die Tat war nicht nur verantwortungslos gefährlich gewesen, sondern auch ein unanständiger Bruch ihrer Privatsphäre. Wassermagier sollten das verstehen!

Protestierend erhob sich Jarid, nur um gleich wieder zusammenzusacken, als ein stechender Schmerz in ihrer Seite von der unverheilten Wunde aus dem Kampf gegen Finster-Trieest zeugte.

„Nicht, Jarid!“, rief eine helle Stimme, die Jarid nur allzu gut kannte, auch wenn sie sie schon seit Jahren nicht mehr vernommen hatte. Gemurmel erklang, dann wieder deutlich: „Sie blutet, lass mich sie doch ansehen!“

Jarid drehte ihren Kopf zur Seite und erblickte ihre Mutter Rowinda, stolze Wassermagierin des fünften Zirkels und oberste Streitschlichterin des Ältestenrats, wie sie einige Schritte entfernt stand. Sie hatte einige Falten mehr und ihr schlohweißes Haar war schütterer geworden, doch redete die alte Rowinda soeben ebenso energisch auf eine kleingewachsene Wassermagierin ein, wie Jarid sie in Erinnerung hatte. Die kleine Wassermagierin wiederum hielt Rowinda ebenso energisch davon ab, näher zu Jarid zu treten.

Die drei restlichen Wassermagier, welche Jarid in das Becken von Quodlon gelotst hatten, standen abseits als kleines Grüppchen zusammen und blickten sehr betreten drein. Der hibbeligste von ihnen war ein großgewachsener blonder Bursche, dessen größtenteils helle Haut teils von großen blauen Flecken geziert war. Er trat unruhig von einem Fuß auf den anderen.

Weiter hinten erspähte Jarid drei auf einem Felsen sitzende Feuerkrieger, welche aus dem Schatten reglos das Geschehen vor ihnen betrachten. Es war im ganzen Dampf, der aus dem Becken von Quodlon stieg, schwer zu erkennen, doch schienen sie alle drei ihre Rankenschwerter gezogen zu haben und die drei Wassermagier ganz genau im Auge zu behalten.

Das mulmige Gefühl in Jarids Magen verstärkte sich. Erneut versuchte sie, sich aufzurichten. Diesmal schmerze ihr verletzter Torso nicht zu stark. Sie stemmte sich breitbeinig in die Höhe und unterdrückte einen Anfall von Schwindel, indem sie sich auf das wohlige Gefühl der warmen Wasserbänder an ihren Handgelenken konzentrierte.

Ihre Mutter blickte zurück zu Jarid. Rowindas Augen wurden groß. Mit wütendem Blick setzte sie an: „Jivin, das ist doch unter aller Würde! Du kannst doch nicht ...“

Ehe sie vorfahren konnte, boxte die kleine Wassermagierin – Jivin – Rowinda in den Bauch und schleuderte sie einige Schritte von sich. Rowinda schlidderte mehr oder minder elegant in das Becken von Quodlon hinein und schaffte es knapp, auf der unruhigen Wasseroberfläche zum Stehen zu kommen. Sie starrte ungläubig auf Jivin, als könnte sie es kaum glauben, dass die Kleine sich das getraut hatte. Doch wehrte sie sich nicht.

Die kleine Wassermagierin drehte sich zu Jarid um und starrte ihr hasserfüllt ins Gesicht. Sie verschloss ihre Hände zu Fäusten und führte sie auseinander. Sofort gefroren die Wasserbänder um Jarids Handgelenke und zerrten ihre Arme auseinander, ja, hoben Jarid gar in die Höhe. Jarid strampelte und fokussierte sich geistig auf ihre Fesseln, aber diese bewegten sich kaum von der Stelle. Von einer solchen starken Kontrolle über das Wasser konnte selbst Jarid nur träumen. Jivin musste mindestens im fünften Zirkel des Ordens sein!

Sie trat näher. Erst jetzt erkannte Jarid, wie alt die kleine Jivin schon war.

Ein leiser Schrei ließ ihren Blick rüber zum Becken von Quodlon wandern, wo eine großgewachsene Feuerkriegerin ihre Mutter gerade unsanft an den Boden presste. Weiter hinten war zwischen zwei Dampfschwaden gerade noch knapp zu erkennen, wie die zwei restlichen Feuerkrieger die drei anderen Wassermagier in Schach hielten.

Das hier war offensichtlich kein Auftrag des Ältestenrats gewesen, Jarid hierher zu holen, um über sie zu richten. Das hier war ein Auftrag dieser drei Feuerkrieger und der kleinen Wassermagierin gewesen, Jarid hierher zu holen, um ...

Ehe Jarid sich genaue Gedanken dazu machen konnte, warum rebellierende Feuerkrieger sie hierher holen wollen könnten, schweifte sie ab. Denn soeben war ihr aufgefallen, dass etwas die drei feindlich gesinnten Feuerkrieger von den ihr altbekannten unterschied: Die Lavasteine in ihrer Rüstung leuchtete nicht hell in orangeroten Tönen, nein. Die Lavasteine waren alle fahl, und in ihrem Innern wirbelten schwarze Schwaden herum.

Jarid hatte erst dieses Phänomen erst vor wenigen Tagen erlebt: Als Trieest den bösen Kristall aus dem Schädel des Hraaks in seiner Faust gehalten hatte. Als das Böse Trieests Körper kontrolliert hatte.

Jarids Blick fiel zurück auf die kleine Wassermagierin. Diese legte ihren Kopf schief und verzog die runzelige Miene zu einem grimmigen Grinsen.

„Grün ist der Seetang, der das Boot an der Weiterfahrt hindert, werte Jarid aus dem fernen Danwar“, sprach sie gehässig. Sie präsentierte Jarid theatralisch ihren linken Unterarm, zog den Ärmel ihres zeremoniellen Kleids zurück und enthüllte einen dünnen, tiefschwarzen Kristallsplitter, der knapp zur Hälfte in ihrem Arm versenkt war.

„Willkommen zurück in Danwar, liebe Jarid. Ist ja erst einige Tage her, dass wir uns gesehen haben, aber dein letzter Besuch in Danwar muss Jahre her sein. Hoffentlich bereitete die Reise keine Unannehmlichkeiten. Du wirst mir sicher sehr hilfreich sein.“\bigskip







\textit{Das Böse ließ die kleine Wassermagierin Jivin mehr Wasser aus dem Becken von Quodlon leiten und Jarid mit einer dicken Schicht Eis bedecken. Zeitgleich ließ es Jarids Mutter und die drei restlichen Wassermagier von den Feuerkriegern fesseln. Es war erheblich anstrengender als gedacht, mehrere Körper auf einmal zu steuern, und einmal entschlüpfte die willensstärkste Kriegerin seiner Kontrolle beinahe. Dann aber hatte es sich gefasst und das Bewusstsein der Feuerkriegerin wieder schlafen gelegt.}

\textit{Und dann war es endlich soweit.}

\textit{Es war enttäuscht gewesen, als es den ersten Lavastein einer Feuerkriegerin hier in Danwar kontrolliert hatte. Keine Stimmen aus seiner Vergangenheit hatten es abzulenken versucht. Der Stein war stumm geblieben. Hoch erfreut hatte es jedoch herausgefunden, dass sein Geist in den Lavasteinen der Feuerkrieger nachhallen konnte, und so keinen konstanten Kontakt zu ihnen brauchte, um ihre Körper weiterhin zu kontrollieren. Nachdem es einige weitere Danware übernommen und ihre Erinnerungen durchsiebt hatte, sah es seine These bestätigt, dass es das Echo, dass das Böse vernommen hatte, mit Trieests Lavastein zusammenhängen musste. Schade, dann würde es wohl noch etwas warten müssen, ehe es dieses Kapitel seines Daseins endgültig abschließen konnte. Die flugs hierhergerufene Jarid würde Trieest aber sicherlich bald zu ihm führen können.}

\textit{Und vielleicht war Jarids Anwesenheit ja nicht einmal nötig, außer, um ihm Genugtuung verschaffen zu können. Denn in den Erinnerungen der Danware hatte das Böse nicht nur vom Becken von Quodlon erfahren, sondern auch von einem anderen besonderen Ort.}

\textit{Abseits von diesem ganzen Geschehen am Becken von Quodlon stapfte eine vierte vom Bösen kontrollierte Feuerkriegerin durch karge Felsen auf einen verdeckten Höhleneingang zu. Es fuhr mit der Hand über warmen Stein und erhob die viel zu hohe, fremde Stimme der Feuerkriegerin:}

\textit{„Oh, ihr Echos der Roten Grotte! Sprecht, auf dass ich hören kann. Ich bin hier, um mit einem Toten zu sprechen!“}













\newpage
\section{Zankende Skral-Hexen, zwei an der Zahl}




\az{Jahr 61}

\textit{Halle des Ältestenrats, 61 a.Z.}\bigskip



Jarid setzte zu einer Frage an, ohne zu erwarten, mit ihr fertig zu werden. Tatsächlich griff der grantige Kord sofort durch und herrschte sie an: „Nein, Trieest wird nicht auf dieser Insel bleiben können. Ich weiß zwar nicht, was geschehen würde, wenn er hier bliebe, aber ich vermute nichts Gutes, bei all dem Hass, der auf ihn geschürt wurde.“

„Nicht auf...“

„... ihn, sondern auf das fremde Blut in seinen Adern. Und dieses muss ihm ausgebrannt werden. Der Prozess des Wandels wird lange andauern und schmerzhaft sein. Aber danach ist er frei, wieder nach Danwar zurückzukehren. Nun, Jarid, wer von Danwar aufbricht, hat einen Orakelspruch zugute. Willst du dich nicht langsam zur Roten Grotte aufmachen, um den Stimmen der Toten zu lauschen?“

Kords Mund verzog sich zu einem schiefen Grinsen. Hin und wieder hatte er doch noch Freude an seinen Gaben. „Du hast dich doch schon lange entscheiden, Trieest auf seinem Pfad zu begleiten. Oder etwa nicht?“\bigskip







\az{Jahr 68}

\textit{Sieben Jahre später.}\bigskip



Trieest war elend zumute. Alles an diesem Tag erinnerte ihn an den Tod.

Die Überreste seines Lavasteins warf er etwas abseits vom Lager in ein kleines Loch und bedeckte er mit Sand.

Den gefallenen Skral legten sie in die Narne, nachdem Trieest klarmachte, dass er ihn definitiv nicht den Skralgepflogenheiten entsprechend verspeisen wollte.

Die Pferde hatten die revoltierenden Skrale mitgenommen. Der flammende Gott allein wusste, wo die nun waren.

Ach, diese revoltierenden Skrale. Sie waren die Ursache der Kampfgeräusche gewesen, welche Trieest nach seinem Zweikampf mit Shron vernommen hatte. Drei der acht Skrale aus Shrons Sippe hatten sich aufgemacht, nach Shrons Ermordung dessen Mörder umzubringen. Natürlich wollten nicht alle einem mutmaßlichen Halbskral als Häuptling folgen.

Überrascht – und dankbar – war Trieest darüber, dass sich einige Skrale gegen diese Revolte gewehrt hatten. Dass ihn vier Skrale nun tatsächlich für ihren rechtmäßigen Häuptling hielten.

Skrale, die ihn argwöhnisch betrachteten, und dennoch vor seinem Zelt schliefen, während Trieest sich in Shrons Bett wälzte. Er musste Jarid auffinden. Er musste Lysbetts Warnung bedenken und die Welt vor dem Bösen bewahren. Er musste weg von hier!

Langsam, um die Skrale nicht zu wecken, schlüpfte Trieest in seine Feuerkrieger-Rüstung. Diese hatten die Skrale gemeinsam mit seinem Rankenschwert von Lysbetts Kutsche geholt. Jedes ungewollte Klirren und Klappern ließ ihn zusammenzucken. Er war nicht an diesen größeren, muskulöseren Körper gewöhnt. Und erst recht nicht an die Zähne eines Halbskrals. Ständig schnitt er sich in die Zunge. Und dieses Blut weckte in ihm einen Instinkt, den er lange nicht mehr verspürt hatte

Er hatte erwartet, dass er diesen veränderten Körper noch viel mehr hassen würde, als er seinen Körper zuvor verabscheut hatte. Doch dem war nicht so. Stattdessen ... ähnlich wie bei der Sprache der Skrale ... fühlte es sich einfach richtig an. Gut, sogar. Als hätte er ein großes Gewicht von seinen Schultern gelöst. Immerhin eines. Wenn er nur nicht über so viel anderes nachsinnieren müsste.

Leise, ganz leise, öffnete Trieest die Tür des Häuptlingszelts. Gänsehaut bildete sich im Loch in seiner Brust, als die warme Mittagsluft darauf trat. Er konnte von Glück reden, dass sich unterhalb des Lavasteins zumindest Haut befunden hatte und nicht nackter Muskel. Dennoch kam es ihm ganz ungewohnt vor. Kalt und leer. Vorsichtig berührte Trieest die Kuhle. Frische, dünne, aschfarbene Haut spannte sich über die Innenseite der Kuhle, welche unter seiner sorgfältigen Berührung gleich einen rötlicheren Ton annahm.

Die Skrale hatten wie befohlen keine Wache aufgestellt. Da lagen sie, im hellen Tageslicht, in ihren mickrigen Betten aus Laub und Stroh, zusammengerollt wie in Eiern, ihre Schwänze um sich geringelt. Sie sahen so friedlich aus, manche hatten sogar ihre scharf bezahnten Münder im Traum zu einem Lächeln verzogen. Und dennoch wusste, dass sie in der Nacht wieder zu marodierenden Menschenmördern werden würden. Trieest konnte den Gestank nach verdorbenem Fleisch ja selbst von hier aus riechen.

Moment mal, da fehlte ja einer!

„Wo soll‘s denn hingehen?“, flüsterte Calrai in Trieests Ohr. Trieest zuckte zusammen, packte den Skral kurzerhand an den Schultern und zog ihn vom Lager weg. Calrai war einen ganzen Kopf größer als Trieest, ließ sich aber widerstandslos mitziehen.

Als sie außer Hörweite der schlafenden Skrale waren, ließ Trieest Calrai los. Sofort begann dieser, zu protestieren.

„Du kannst uns jetzt nicht einfach im Stich lassen!“, knurrte er in der kehligen Sprache der Skrale, „Wir haben unser Leben für dich riskiert. Diese Sippe gehört zu dir. Du trägst die Verantwortung über uns!“ Sein Schwanz zuckte über den Laubboden.

„Ich muss gar nichts“, entgegnete Trieest, „Ihr habt unzählige Menschen getötet und hättet beinahe auch meine Begleiterin auf dem Gewissen gehabt. Ihr habt Glück, dass ich mich nicht gegen euch wende!“

Entrüstet wich Calrai zurück: „Ach, plötzlich sind es die Menschen, die dir wichtiger sind als dein eigen Fleisch und Blut? Sie haben dich in Ketten gelegt und gefoltert!“

Trieest lachte bitter auf: „Das ist eine lange Geschichte, die dich nichts angeht. Ich will einfach nur weg von hier und du kannst mich nicht aufhalten.“

Stumm blickte Calrai ihn an, sein stacheliger Schwanz unruhig hin und her peitschend. Schließlich brummelte er: „Ich will mir nicht anmaßen, zu verstehen, woher du kommst oder was du durchgemacht hast. Es ist gut möglich, dass wir in deinen Augen nichts anderes als Monster sind. Doch sind wir ab jetzt deine Monster. Du hast Shron getötet und seine Stelle eingenommen.“

„Ich mag schon nicht, dass hierzulande die Nachfahren von Herrscher als nächstes herrschen. Nun soll sogar der Mörder des vorherigen herrschen?“

„So läuft das halt. Wir folgen dir als Häuptling, und es wäre schön, wenn du dies respektieren würdest. Du bist alles, was wir noch haben.“

„Und was, wenn ich der Meinung bin, dass die Welt ohne euch besser dran wäre? Was, wenn ich euch befehlen würde, in den nächsten See zu springen und unten zu bleiben? Würdet ihr das dann tun?“

„Nein, vermutlich nicht“, meinte Calrai beinahe verlegen, „Und wir werden dir auch sonst nicht blind folgen. Ein Anführer ist kein Anführer, weil jeder seinen Worten blind Folge leistet. Sondern weil seine Sippe weiß, dass sie unter ihm weiter kommen werden als allein.“

Trieest blickte den Skral entgeistert an.

„Ich kann doch nicht einfach ... ich habe unzählige Skrale getötet. Ich bin ein verdammter Halbskral. Du kannst unmöglich wollen, dass ich das Überbleibsel dieser Sippe leite! Und ich will es auch nicht!“

Calrai saß da und starrte Trieest an, seine rot beschuppte Brust hob und senkte sich rasch. Dann knurrte er zwischen zusammengebissenen Zähnen hervor:

„Es kommt nicht darauf an, wie viele Skrale du umgebracht hast. Viele würden sogar behaupten, dass das von Stärke spricht. Und dass du ein halber Skral bist ... nun, ich kann nicht für die anderen sprechen, aber ich hatte angenommen, dass du einer wärst. Und es störte mich nicht.“

Trieest fiel auf, dass des Skrals gestikulierende Hand vier klauenbewehrte Finger hatte anstelle der üblichen drei. Vierfingrige Hände hatte Trieest durchaus schon gesehen, etwa bei Schamanen und Kreideskralen, doch war sicher es nicht die Norm. Und jetzt, wo er sich darauf achtete, erschien ihm auch Calrais Stirn um einiges hoher war als die Flachköpfe generischer Skrale. Trug Calrai ebenfalls eine Spur Menschenblut ins sich, einige Generationen zurückliegend? Fanden doch nicht alle Skrale Halbskrale abscheulich und abschreckend?

Heiser fuhr Calrai fort: „Shron hat deine Herausforderung angenommen und sich dir zum Zweikampf gestellt. Das war sein Fehler, aber das macht dich würdig, ein Häuptling zu sein, Menschenblut hin oder her. Bitte, nimm deine Pflichten wahr. Wir haben bereits zu lange gelitten. Während Shron seine Ehre und viele Sippenmitglieder verlor, ließ er seine Wut an uns aus. Du bist unsere letzte Hoffnung, wieder ein wenig Achtung und ein sicheres Dasein zu erlangen.“

Trieest blickte in Calrais Augen. Wie bereits im Kampf gegen Shron erkannte er, dass es sich bei seinem Gegenüber nicht um eine willenlose Kreatur handelte. Da war ein denkendes, fühlendes Wesen am anderen Ende, das ihn aus weißen Augen anblickte und sich dann abwandte.

„Leite doch du sie, Calrai! Dich kennen sie, dir vertrauen sie. Sag ihnen meinetwegen, du hättest mich herausgefordert und gestürzt! Irgendetwas, was mit Eurer elenden Führungstradition übereinstimmt. Ehrlich, ich kann mir nicht vorstellen, wie es noch so viele Skrale geben kann, wenn ihr euch wegen jeder kleinen Streitigkeit gegenseitig abmurkst.“

Calrai blickte Trieest nachdenklich an. Er schien seine Optionen abzuwägen. Dann schüttelte er seinen Kopf. „Es muss nicht immer ein Kampf auf Leben und Tod sein. Doch ich will dich nicht bekämpfen. Und ich bin auch kein Anführer. Kurgat ebensowenig. Und auch nicht Tran und Bark.“

Trieest wand sich. Er wollte gar nicht die Namen dieser Skrale hören, er wollte sie gar nicht als Personen sehen, die Hilfe brauchten.

Calrai fuhr fort: „Du, Trieest, hast dich um diese Menschenfrau gekümmert. Du hast ein gutes Herz. Wir wollen so jemandem folgen. Wir wollen ... ich will mein Glück mit dir versuchen.“

Stille machte sich breit. Trieest saß da und betrachtete Calrai. Der große Skral hatte ein Stück Holz hervorgezogen und begann mit seinen Klauen, kleine Stücke davon abzuspalten. Mit nichts als seinen Händen formte er das Holzstück, fein und detailliert. Faszinierend. Sein Blick blieb an Calrais Kinn kleben. Gestern hatte der Skral noch einen stoppeligen Bart getragen, dessen war er sich sicher, doch heute war sein Kinn glatt rasiert und ... was machte er sich hier nur für Gedanken?!

Brüsk stand Trieest auf. Dadurch wurde Calrai wieder auf ihn aufmerksam und liess vom Holzstück ab.

„Hjork?“, fragte er, ein kurzer Ausruf, der so viel wie ‚Nun?‘ bedeutete.

Was sollte Trieest bloß antworten? „Verzeih mir. Ich muss euch verlassen. Ich muss meine Begleiterin finden.“

„Wir können dir dabei helfen“, warf Calrai fast flehend ein, „Wir können die übrigen Skrale gleich jetzt wecken und sich mit dir auf die Suche machen. Wir haben gute Nasen.“

„Sie ist weit weg von hier. Nasen werden uns kaum weiterhelfen.“

„Dann finden wir andere Wege. Wir suchen den Rat der weisen Hexen, die können so gut wie jeden finden. Wir sind jetzt deine Sippe, und wir stehen füreinander ein.“

Calrai stand nun erneut auf und Trieest wich unweigerlich einen Schritt zurück, als der Skral ihn überragte und sanft seine Pranke auf das Loch in Trieests Brust legte: „Leugne es nicht, unter dieser Haut fließt derselbe Saft wie in meinen Adern.“

„Fass mich nicht an!“, stieß Trieest hervor. Calrai zog seine Hand abrupt zurück. Ein elendes Schuldgefühl machte sich in Trieest breit, als er Calrais verletzten Gesichtsausdruck sah.

„Es ... ich meine es nicht so“, sagte Trieest nun, „Es ist ... ich bin kein Skral. Ich habe nichts mit euch zu tun. Ich ...“

„Ich kann dich nicht zwingen, Häuptling Trieest“, meinte Calrai nun, seinen Kopf gesenkt, „Aber ich vermute, dass du unter den Menschen nie so sehr zuhause sein wirst, wie du es in dieser Sippe wärst.“

Mit diesen Worten wandte Calrai sich ab. Sein Schwanz streifte ein letztes Mal Trieests Bein, dann zog er von hinnen.

Trieest blieb allein im Wald zurück. In seinem Kopf hallte das Echo von Calrais Worten nach. Er besaß wieder den Körper eines Halbskrals. Eines halben Nord-Skrals zumindest, folglich würden viele Menschen aus diesen Landen ihn nicht direkt als Skral erkennen. Aber dennoch ... würden er sich je in einem Dorf niederlassen können? Würden seine Nachbaren ihn akzeptieren? Würde Jarid ihn jetzt akzeptieren, wo er den Körper einer Kreatur und ihren Blutdurst teilte?

„Beim Barte des Warx!“, knurrte Trieest. Er spürte einen feinen Stich in seinem Herzen. Dann verließ er den Wald und folgte Calrai zurück ins Lager der Skrale.\bigskip







Müde versammelten sich die geweckten Skrale um Trieests Zelt. Calrai schwankte ein wenig auf seinen Beinen, aber in seinen Augen zeigte sich immer noch mehr Entschlossenheit als Müdigkeit. Die übrigen drei Skrale schienen erheblich verunsicherter zu sein.

Alle schienen auf eine Anrede Trieests zu warten. Er räusperte sich. „Calrai erwähnte weise Hexen. Ich werde wohl zu ihnen aufbrechen. Ihr dürft natürlich tun, was ihr wollt.“

Calrai antwortete: „Und wir wollen dir zu ihnen folgen.“

Die drei übrigen Skrale nickten zustimmend. Einer von ihnen fügte feste nickend an: „Ja, so soll es sein. So will es die Tradition. Wenn sich die Führung einer Sippe ändert, gehen wir das bei Drunn und Trumm berichten. Das ist gut.“

Trieest fühlte Mitleid in sich aufsteigen. Diese vier Skrale vor ihm hatten sich verzweifelt an die Tradition geklammert und Trieests ‚Anrecht‘ auf den Häuptlingssitz gegen die restlichen Skrale verteidigt. Sie hatten sich gegen ihre vermutlich langjährigen Begleiter gestellt und viel riskiert, dass sie hier sein konnten. Sie waren vermutlich genauso verunsichert wie er. Und er würde sie enttäuschen müssen.

„Na dann, brechen wir alle auf zu Drunn und Trumm“, sagte Trieest resigniert. „Ich nehme an, du kennst den Weg?“

Calrai reckte seine Faust in die Höhe. „Natürlich. Immer nur in den Süden, bis zum Skralberg. Die Anhöhe hoch in Richtung Trummwald. Drunn und Trumm beraten die meisten Sippen in dieser Gegend. Sie berieten bereits manche freien Sippen, als die meisten von uns noch unter der Kontrolle Taroks standen. Sie wissen viele Dinge.“

Rasch und geübt packten die Skrale Zelt und Banner zusammen und marschierten los. Der eine und andere unterdrückte ein Gähnen. Trieest musste an alle Nächte zurückdenken, in denen sein pochender Lavastein ihn davon abgehalten hatte, eine volle Mütze Schlaf zu ergattern.

Er war immer noch weit davon entfernt, sich mit der Leitung einer Skral-Sippe anzufreunden. Aber um die elende Stille zu durchbrechen, die die Gruppe der vier Skrale umgab, fragte er sie irgendwann mal danach, sich vorzustellen.

Calrai kannte er schon am besten. Er war Häuptling Shron bereits bei der Besetzung der Rietburg gefolgt. Calrai war kein einfacher Krieger wie die meisten anderen Skrale hier, sondern ein angehender Schamane gewesen, bis seine Lehrmeisterin in einem Scharmützel mit Bewahrern gestorben war. Calrai konnte die Stimmen der Bäume hören, in den Spuren der Vögel lesen und mit genügend Konzentration und Kraft Nebelschwaden heraufbeschwören, die ihn und die seinen schützten.

Dann waren da die unzertrennlichen Tran und Bark. Tran war im Kampf verletzt worden und wurde von Bark mit allerlei Kräutern und guten Zubrummungen überschüttet. Reden tat Bark dabei nie, Tran sprach für sie beide. Bark hätte sich während der Belagerung der Rietburg einem direkten Befehl des Schwarzen Herolds widersetzt, um Tran aus einer heiklen Situation zu befreien. Als Strafe waren sie beide dem in Ungnade gefallenen Shron zugeteilt worden, sprudelte es fast förmlich aus Tran heraus. Er hoffe, dass in ihrer Zukunft weniger unnötige Aufgaben und mehr Ruhezeiten lagen.

Und dann war da Kurgat, der beste Bogenschütze in der Gruppe, der sich vor allem Möglichen fürchtete. Wie Trieest bald feststellte, war Kurgat unumstößlich davon überzeugt, dass es das Meer nicht gab. Oder vielmehr, dass das, was wir als Meer sehen würden, in Wahrheit Illusionen der Wasserbewohner waren, die uns von unter der falschen Wasseroberfläche beobachteten und jeden Augenblick mit der Invasion der Südlande beginnen konnten. Trieest versuchte kurz ungläubig, ihm dies auszureden, und zog sich bald zurück, als Kurgat starr an seiner Meeresfurcht festhielt. Hinter sich hörte er Calrai kichern.

So wanderten sie durchs offene Rietland, immer wieder im Schatten verbleibend, nach Menschen oder Zwergen Ausschau haltend. Sie schlichen dem Ufer der Narne entlang, wo es für Skrale besonders sicher war, seitdem sich viele Menschen aufgrund übler Skral-Trupps dem Fluss entlang fernhielten. Kurgat hielt sich so weit wie möglich von den reißenden Fluten entfernt und schielte immer wieder ängstlich dorthin.\bigskip







Eines Nachts erreichte die Gruppe den Skralberg und zog westlich an ihm vorbei, sorgsam einen Bogen um alle Bauernhöfe und Ziegenstallungen auf ihrem Weg schlagend. Skralmägen knurrten. Tran schlug vor, eine Ziege zu klauen. Trieest widersprach. Dies behagte ihm nicht. Skrale konnten tagelang ohne frische Nahrung auskommen, doch irgendwann mussten sie speisen. Trieest würde sich etwas überlegen müssen. Einige Wochen konnte die Sippe sich bestimmt auch mit Beeren und Früchten durchschlagen können, aber irgendwann würde der Durst nach Fleisch und Blut sie völlig überwältigen, wenn er den Geschichten über gefangene Skrale gut genug vertrauen durften. Konnten sie am Freien Markt Fleisch kaufen gehen? Ah, wer weiß, ob er diese Gruppe überhaupt so lange am Hals hätte. Zunächst musste er Jarid finden. Und danach das Böse aufspüren.

Bald darauf erreichten sie eine grün bewachsene Anhöhe, die sie im Licht der aufgehenden Sonne erklommen. Hier und da sprossen Heilkräuter und Trieest steckte einige davon ein. Schließlich deutete Calrai auf eine kaum sichtbare Höhle am Rande des Bergs.

Zwei Skral-Schamanen hielten Wache vor dem Höhleneingang, Felle unbekannter Spezies über ihren breiten Köpfen. Im Gegensatz zu Calrai waren sie weder muskulös noch mit Klingen bewaffnet. Doch ihre langen Totemstäbe, die sie bei seiner Ankunft fester packten, verrieten Trieest, dass sie sehr wohl eine Gefahr für ihn sein konnten.

Ein Geheimversteck der Skrale?

„Was erwartet uns hier?“, flüsterte er Calrai zu.

„Keine Sorge, du bist hier sicher“, flüsterte dieser zurück, „Du gehst hinein, sprichst mit Drunn und Trumm, informierst sie über alles, was geschehen ist, und sie verraten dir mit etwas Glück, wo sich deine Freundin befindet.“

„Werdet ihr nicht dabei sein?“

Calrai grinste und klopfte ihm auf die Schultern: „Alles kommt gut. Komm ja nicht einfach auf die Idee, sie anzulügen. Drunn spürt das.“

Ein mulmiges Gefühl machte sich in Trieests Magen breit. Diese vier Skrale um ihn herum wirkten zwar ganz annehmbar, aber bei Gedanken, hochrangige Skral-Hexen zu treffen, trat das natürliche Misstrauen gegen diese Kreaturen wieder deutlich hervor.

„Warte, warte, nein, ich will da nicht allein rein. Was hindert sie daran, mich einfach umzubringen?“

Calrai legte seinen Kopf schief. „Was hindert die Menschen in deiner Umgebung daran, dich einfach umzubringen?“

„Willst du mich davon überzeugen, dass sie über ein gutes Gewissen verfügen?“

„Natürlich tun sie das! Aber, wenn du schon nicht auf ihre Gutherzigkeit zählen magst, dann darauf, dass Vollmond ist. Diese Nacht wird ein heiliger Skralreigen stattfinden und viele Skrale werden neuen Nachwuchs empfangen. Da darf davor doch kein Blut vergossen werden. Aber wenn du unbedingt willst, begleite ich dich halt hinein.“

„Und ich behalte mein Rankenschwert.“

„Das werden wir sehen müssen.“

Calrai musste sich eine Zeit lang mit den beiden wachenden Skral-Schamanen vor dem Höhleneingang abmühen, bis sie endlich nickten und Trieest mitsamt seines Rankenschwerts und mit Calrai als Begleiter in die Höhle marschieren ließen. Die drei restlichen Skrale saßen in etwas Distanz davon auf den Boden und breiteten ihr Schlaflager für den Tag vor.

„Ich habe Drunn und Trumm bislang nur einmal getroffen“, flüsterte Calrai in Trieest Ohr, „Lass dich von ihren Eigenarten nicht irritieren.“

Trieest schluckte das mulmige Gefühl herunter.\bigskip







Die Höhle der Skral-Hexen war überraschend groß. Selbst Trolle hätten darin Platz gehabt. Ein schwaches Leuchten ging von einem großen Teich in der Mitte der Höhle aus, der schöne Muster auf die Höhlenwände warf. Darin platschte irgendetwas Stacheliges herum. Ein Säbelfisch oder Stahlfisch?

Die Höhlenwände waren über und über mit verschiedenen Fässchen, Kesseln und Pflanzen behängt. Hier und dort flatterte etwas Fledriges herum. Doch die auffälligsten Wesen in dieser Höhle waren natürlich die beiden Skralinnen.

Trieest hatte gedacht, mit Shron den riesigsten Skral seines Lebens gesehen zu haben. Er hatte sich geirrt. Drunn und Trumm waren wahrlich gigantisch. Selbst so bucklig, wie sie auf dem Höhlenboden um den funkelnden Teich lungerten, überragten sie Trieest immer noch um mindestens drei Köpfe. Die beiden Skralinnen waren gebettet in lange, unförmige Mäntel, die aus vielen Einzelteilen zusammengestickt waren. Sie hantierten an einem langen Stoffteppich herum, welches von komplizierten Runenmustern übersät war.

Calrai kniete sofort auf den Boden, als hinter ihnen das Klopfen von Totemstäben auf hartem Felsen ihre Ankunft ankündigten. Trieest tat es ihm hastig nach.

Zwei lange Gesichter, beide so groß wie Trieests Torso, wandten sich ihm zu. Zwei breite Münder öffneten sich zu einem Lächeln.

„Trieest, wie schön, dass du hier bist“, grinste eine der beiden Skralinnen mit tiefer, heiserer Stimme, „Ich bin die Hexe Trumm. Du magst mich nicht kennen, aber ich habe dich schon aus der Ferne beobachtet, als du bei der Befreiung der Rietburg zahlreiche unserer Spezies ins Reich der Drachen sandtest.“

Trumm riss den Runenteppich aus den Händen der anderen Skralin, die wohl Drunn war, und faltete ihn sorgfältig zusammen.

Trieest antwortete nicht, sondern langte vorsichtig nach seinem Rankenschwert.

„Steck das weg, Bursche“, knurrte Drunn unwirsch, „Wir wollen dir nicht schaden. Dies sind heilige Hallen. Und heute ist ein heiliger Tag, an dem erst recht kein Blut vergossen wird.“

„Jetzt verschreck den armen Jungen doch nicht, er hat sicher sehr aufwühlende Tage hinter sich“, scheuchte Trumm die andere Skralin zurück. An Trieest gewandt fuhr sie fort: „Du musst ihr verzeihen, sie hat einfach kein Taktgefühl.“

Jetzt wusste Trieest erst recht nicht, was er sagen sollte. Das war aber auch in Ordnung, denn Drunn fuhr ungestört fort: „Das Schicksal hat manchmal einen miesen Sinn für Humor. Wusstest du, dass Calrai neben dir ebenfalls an der Belagerung der Rietburg beteiligt war? Es war nicht unwahrscheinlich, dass ihr euch dort hättet treffen können. Ihr hättet euch gegenseitig zerfleischt. Na, Calrai, wusstest du, dass Trieest ein Held von Andor ist? Einer eurer erklärten Feinde?“

Calrai blickte Trieest entgeistert an.

„Drunn, Was ist denn heute mit dir?!“, rief Trumm und stieß Drunn unsanft in die Schulter, „Sonst stößt du Besucher doch nur halb so unsanft vor den Kopf.“

„Zu langsam bist du, Trumm, das ist, was ist“, schnaubte Drunn, „Du hättest sie stundenlang von Dingen berichten lassen, die wir doch schon lange wissen. Lass uns zur Sache kommen. Ich will bald das Ritual beginnen. Ich weiß, dass es für dich schwer ist, ein Zeitgefühl zu entwickeln, aber selbst in deinen dicken Schädel sollte es doch irgendwann gehen, dass wir nicht den ganzen Tag Zeit haben.“

Trumm brummelte etwas in ihre Kehle, verwarf ihre riesigen Hände und brummte: „Verzeih, verzeih, es können nicht alle so pflichtbewusst sein wie du.“

Sie zwinkerte Trieest und Calrai zu.

„Na dann, Drunn, stell deine Fragen. Sag, was wissen wir noch nicht?“

Trieest blickte Calrai fragend an, wie er sich zu verhalten hatte. Calrai musterte ihn immer noch angespannt, als erwartete er, eine andorische Heldenbrosche an seiner Rüstung zu finden.

„Wir wissen, wer ihr seid“, lamentierte Drunn, „Wir wissen, dass Calrai und seine Begleiter noch vor kurzem zu Shrons Sippe gehörten. Wir wissen, dass Shron getötet wurde, von diesem Triiest. Wir vermuten, dass Triiest nun Anspruch auf den kläglichen Rest von Shrons Sippe erheben will. Ist dem nicht so, Trumm?“

Trumm kramte in einem Stapel alter Stofffetzen umher und brummelte: „Na, sieh mal einer an, selbst ein löchriges Gehirn wie deines kann ein paar Fakten merken. Den Namen hast du aber wie üblich völlig falsch betont, der Junge heißt Trieest.“

Drunn gluckste kurz auf und fuhr dann fort: „Wir wissen nicht, warum Trieest Shron herausforderte. Warum sich für das Amt des Häuptlings interessiert. Und warum er sich dafür eigenen sollte.“

Trumm hielt in ihrem Kramen inne und blickte Trieest interessiert an. Als keine der beiden Skralinnen weitersprach, räusperte er sich und gab so kurz wie möglich die Geschichte vom Zweikampf mit Shron und dem geborstenen Lavastein zum Besten.

Schon bald war Shron tot, und Calrai musste eingreifen, denn Trieests Stimme begann zu versagen, sobald seine Gedanken zurück an die Echos der Toten schweiften.

„Was als nächstes geschah, war, dass Rovuk sich offen weigerte, den Ausgang des Kampfes als valide gelten zu lassen. ‚Ich lasse mich doch nicht von einem halbherzigen Halbskral anführen!‘, waren seine Worte. Wir wussten zu diesem Zeitpunkt auch gar nicht, ob Trieest das Zerbröckeln dieses Lavasteins überlebt hatte. Ich muss zugeben, damit gerechnet zu haben, dass jeden Augenblick Flammen aus ihm hervorbrächen und ihn einhüllten, während sein Körper zu glühender Asche zerstöbe.“

„Du warst so gut auf dem direkten Pfad, nun schweife doch nicht ab“, grummelte Drunn.

„Nun lass den Jungen doch! Eine gute Erzählung will Weile haben“, protestierte Trumm.

„Nun erzähl mir nicht, dass du etwas von guter Poesie verstündest“, lachte Drunn auf, „Du heulst ja bereits los, wenn ich die Saga von Korn und dem Urtroll zu erzählen beginne.“

„Und du bist doch bloß eifersüchtig darauf, dass Brom der Feurige den Drunn-Wald völlig verzehrte und meiner noch steht!“

Trieest machte große Augen: „Wie alt seid Ihr bereits?“

Trumm grinste: „Na, Broms Zeiten erlebten wir schon nicht selbst mit. Es gibt schon seit vielen Jahrhunderten Hexen mit unseren Namen im Grauen Gebirge. Aber wir beide waren schon am Zanken, als dein ach-so-geliebter Ex-König noch ins Bettchen gemacht hat.“

Trieest überschlug in seinem Kopf Brandurs Alter. Gerüchten zufolge hatte die Hexe Reka ihm sogar allerlei lebensverlängernde Tränke verabreicht, da sein Tod der Legende nach großes Unheil über Andor bringen sollte. Hatte es ja auch tatsächlich. Diese Hexen mussten uralt sein!

Drunn gluckste. „Ich weiß, es ist schwer zu glauben, Jungchen. Also nicht, wenn du dir Trumm anguckst. Aber ich habe mich für mein Alter doch sehr gut gehalten.“

Calrai ignorierte die zankenden Skralinnen und sprach weiter: „Trieest kniete nur so da am Boden und reagierte nicht. Doch atmete er noch, und Shron nicht. Rovuk trat vor und hob sein Messer. Da trat ich vor und schlug ihn zurück. Ein Kampf brach aus.“

„Ein Kampf, der zu euren Gunsten ausging?“

Calrai wackelte unentschlossen mit seinem Schwanz. „Von uns wurde nur Tran ernsthaft verletzt, von den anderen Grobek gar getötet. Doch es war nicht zu unseren Gunsten, vielmehr ein Verlust für die versammelte Skralgemeinschaft.“

„Lüg mich nicht an!“, knurrte Drunn, „Trumm, der Junge wollte mir ernsthaft glauben machen, dass er sich einen feuchten Dreck um die anderen Skrale schere.“

„Ts, ts, ts“, grinste Trumm, „Du solltest es besser wissen.“

„Verzeiht. Es gibt gewisse Gebote der Höflichkeit ...“

„Verziehen und vergessen, Kleiner“, winkte Trumm ab.

„Hey!“, protestierte Drunn, „Das ist meine Entscheidung.“

„Ich kann dich lesen wie eine offene Schriftrolle, du olle Tante. Wenden wir uns lieber wieder dem interessanteren Besucher zu. Nicht böse gemeint, Calrai.“

Calrai lächelte höflich und trat einige Schritte zurück. Trumm und Drunn hingegen schritten nach vorne und beäugten Trieest aus dem Schatten ihrer Mäntel. Trumm betatschte gar seinen Kopf und seine Arme mit ihren spindeldünnen Fingern.

Sie sprach als erste: „Ich habe schon so einiges gesehen. Menschen, Taren, Feuerdämonen ... wenn man nur lange genug einen dieser Lavasteine in ihrer Brust sitzen ließ, sahen sie so gleich aus. Und nun einen halben Nord-Skral. Wie faszinierend.“

Ein Schnauben ertönte aus den Tiefen von Drunns Kapuze. „Weißt du, Trumm, das mag für dich schwer zu verstehen sein, aber ich habe tatsächlich auch schon mit einigen Halbskralen Kontakt gehabt. Die meisten sind wild und verenden rasch, aber es gibt diesen einen, der im Rietland für Unordnung sorgt. Hatten wir uns eigentlich nicht mal vorgenommen, mit dem ein ernstes Wörtchen zu reden?“

„Knöpf du dir den Schattenskral selbst vor. Ich habe andere Prioritäten.“

Trieest zuckte zusammen, wie so oft, wenn er an sein kurzes Leben denken musste. In den Geschichten, die die Menschen Danwars einander erzählten, sowohl wahren als auch erfundenen, sprachen sie so oft von Wesen, die so viel älter werden als sie. Zwerge und Taren, Feen und Naturgeister. Menschen kamen sich so kurzlebig vor, darum fühlte es sich natürlicher an, von Wesen zu berichten, die langlebiger waren. Aber wirklich kurzlebig waren eigentlich nicht die Menschen, sondern die Kreaturen. Und erst recht die Halbkreaturen. So wie die Taren viel länger lebten als Ziegen oder Menschen gemeinsam, so leben Halbskrale viel kürzer als Skrale oder Menschen. Etwas in ihrer Natur mischte sich nicht richtig. Der Körper verrottete rasch. Das Leben war kurz. Was nur umso mehr ein Grund wäre, es zu genießen.

Vielleicht hatte der Lavastein Trieests unnatürlichen Alterungsprozess aufgehalten oder zumindest abgeschwächt. Hoffentlich zumindest nicht beschleunigt. Aber Trieest hatte keine Ahnung, konnte keine Ahnung haben. Es gab ja keine anderen wie ihn, an denen er sich orientieren konnte. Keinen Präzedenzfall.

Hatte er noch einige Jahre zu leben? Zumindest ein Jahrzehnt? Was könnte er alles in dieser Zeit tun? Was wollte er alles in dieser Zeit tun?

Keine Skral-Sippe anführen, das war sicher.

„Trieest, sag mal, was liegt dir an diesen Skralen in dieser Sippe?“

Trieest schluckte schwer. Offenbar war es ungeschickt, Drunn anzulügen, weswegen er wahrheitsgetreu sprach:

„Offen gesagt nicht viel. Mit Verlaub: Ich will einfach nur diese ganze Angelegenheit hinter mich bringen.“

Trumm und Drunn kicherten beide.

Trieest verzog sein Gesicht und führte aus: „Euch droht von mir keine Gefahr. Wirklich keine. Doch wenn die Option besteht, an meiner Stelle einen anderen Häuptling zu wählen, so würde ich diesen wohl unterstützen. Ich will mich nur auf die Suche nach meiner Begleiterin machen.“

„Ja, Jarid suchen willst du wirklich“, sprach Drunn, „Doch machst du dir etwas vor, wenn du denkst, dass dir überhaupt nichts an diesen Skralen liegt. Du kennst sie doch nur seit kurzem, und doch verbindet euch schon jetzt ein stärkeres Band, als es seit deiner Kindheit je zwischen dir und einem Menschen gab.“

„Abgesehen von Jarid natürlich. Drunn, vergiss bloß nicht Jarid.“

„Natürlich, wie könnte ich auch. Der Junge kann sie ja selbst kaum vergessen. Als wäre sie seine leibliche Mutter.“

„Seine leibliche Mutter hat sich auch nur mäßig um ihn gekümmert.“

Stille. Trieest wusste nicht, was sagen, und Calrai traute sich vermutlich nicht, etwas zu sagen.

„Na gut“, meinte Trumm, „Ich habe eine Entscheidung getroffen.“

„Warte, warte, diesmal bin ich an der Reihe, alte Hexe!“

„Wer zuerst kommt, mahlt zuerst.“

„Nicht, wenn ich mit Mahlen dran bin!“

Drunn stieß Trumm zur Seite, dass es dröhnte, und sprach dann feierlich:

„Ich habe eine Entscheidung getroffen. Es ist uns im Grunde genommen egal, was mit euch geschieht. Solange es unserer Gemeinschaft nicht schadet. Wisset, dass vor einigen Stunden die beiden anderen überlebenden Skrale aus Shrons kläglicher Sippe vor uns traten, um über euch zu klagen.“

„Drunn hat ihnen tatsächlich gestattet, sich einem anderen Häuptling anzuschließen, ohne vorher eure Seite der Geschichte anzuhören. Ein schauderhafter Präzedenzfall, den sie hier setzt.“

„Was Trumm sagen will, wenn sie ihre Worte nur ein bisschen besser wählen könnte, wäre, dass die anderen Skrale wieder Teil einer Sippe sind, einer größeren und mächtigeren als zuvor. Sie sind zufrieden. Wenn ihr die Sache nie wieder aufbringt, sollte sie vermutlich gegessen sein. Und Shron besitzt keine anderen Kinder oder mächtige Verbündete, die sich ihm verpflichtet fühlten und den Streit wieder anfechten würden.“

„Drunn, du übersiehst doch mal wieder das Wichtigste. Ist die Sache denn auch für diese Skrale hier gegessen?“

„Keiner hier will unnötiges Blutvergießen. Das solltest doch auch du sehen können.“

Trieest und Calrai blieben stumm, während Trumm und Drunn kurz weiterstritten, ob Trieest und die restlichen Skrale „seiner“ Sippe auf Rache aus waren.

Sie kamen zum Schluss, dass dies nicht der Fall war, und schlossen mit den Worten:

„Nun, Trieest, deine Sippe ist ohnehin verschwindend klein, sodass diese Entscheidung kaum Relevanz hat. Nichtsdestotrotz würde es uns freuen, wenn du ihn leiten könntest. Ein Held von Andor und Halbskral als Häuptling, das könnte tatsächlich helfen, erste friedlichere Beziehungen zwischen den Andori und den Skralen aufzubauen.“

Trieest schnaubte auf. Friedliche Beziehungen waren nun mal schwer aufzubauen, wenn zwei Gruppen bei jeder Gelegenheit mordend übereinander fielen.

„Und ist das etwa unsere Schuld?!“, brauste Drunn auf, als hätte sie Trieests Gedanken gehört, „Seit Jahrzehnten werden unsere besten Krieger von finsteren Mächten unter ihre Kontrolle gezogen. Die wenigsten Skrale konnten sich einem direkten Befehl des Drachen Tarok widersetzen, und nun, wo er tot ist, zwingt uns der Hunger zu Überfällen und Raubzügen! Ganz zu schweigen vom Dunklen Magier Varkur und dem Nekromanten Hademar und dem Schwarzen Herold ... die alle sind noch da draußen, und sie sehen in uns Skralen kaum mehr als Futter für die andorischen Schwerter, bis alle anderen Völker ausgerottet sind – was bei ihrer aktuellen Taktik nie der Fall sein wird. Ich habe per se nichts gegen die Andori. Ich träume von einer Welt, in der wir alle in Frieden zusammenleben könnten.“

„Tust aber nicht viel, um diese Welt zu realisieren“, warf Trumm ein. „Erst recht nicht, wenn du weiter Menschen kostest.“

„Kann ich denn etwas dafür, dass sie so viel leckerer als Ziegen sind?!“, grummelte Drunn.

Trieest fiel siedend heiß wieder ein, mit welcher Art von Wesen er es hier zu tun hatte, und musste sich beherrschen, nicht sein Schwert zu ziehen.

Trumm stöhnte auf. „Lass das! Selbst wenn du uns ernsthaft schaden könntest, würdest du durch unseren Tod keine Skral-Sippen vom Randalieren und Morden abhalten. Und falls du vorhast, eines Tages diese Sippen in eine friedlichere Richtung zu lenken, wirst du unsere Stimmen brauchen. Drunn sagt ja schon jetzt, sie wäre für eine solche Entwicklung, auch wenn ihre Taten anderes besagen.“

„Weil unser akutes Überleben wichtiger ist! Wir müssen zuerst an die Unseren denken! Uns geht es nicht gut genug, um unsere Raubzüge einzustellen. Eines Tages wird es das aber.“

„Drunn ist eine verschrobene Träumerin. Seit dem Unterirdischen Krieg jagen Zwerge und Menschen uns, und das wird vermutlich auf ewig so weitergehen. Skrale und Andori werden wahrscheinlich nie in Frieden leben. Aber selbst kleinste Wahrscheinlichkeiten können zur Realität werden können, und ich will ihre Hoffnungen nicht zu früh enttäuschen. Vielleicht können die Andori ja in eurer Sippe etwas Gutes sehen. Und dann sind wir dem Ideal schon einen Schritt näher.“

Trieest fragte: „Dann muss ich die Skrale behalten?“

„Müssen musst du nichts in diesem Leben, außer sterben. Es würde uns freuen, wenn du sie leiten könntest. Lässt du sie im Stich, werfen wir sie halt einer anderen Sippe zu.“

Trieest wusste nicht genau, was er antworten sollte. Dies war irrelevant, da Trumm bereits ohne seinen Input weitersprach:

„Aber du kümmerst dich ja gar nicht darum, oder, Trieest? Du redest dir doch ein, einfach wissen zu wollen, wo sich Jarid aufhält, ja? Nun, als Zeichen unseres Vertrauens werden wir dir dieses Wissen anvertrauen. Wir besitzen gewisse Möglichkeiten.“

Trumms Blick glitt hinüber zum funkelnden Teich. Sie rührte mit ihrer Hand darin, patschte den Säbelfisch zu Seite, kniff ihre grünlich dampfenden Augen zusammen und meinte:

„Seltsam. Ich kann Jarid nicht sehr. Sie muss sich in einem von außergewöhnlicher Magie getarnten Ort befinden. Falls sie überhaupt noch lebt.“

Trieests Magen verkrampfte sich.

„Steht zur Seite, du kannst mit dem heiligen Tümpel ohnehin nicht richtig umgehen“, grummelte Drunn, stieß Trumm zur Seite und platschte selbst in der Oberfläche herum. Dann murmelte sie jedoch: „Doch auch ein blinder Skral fängt hin und wieder ein Kind. Keine Ahnung, wo sich deine Jarid befindet. Du Armer. Na dann, husch, husch, raus mit euch, die Audienz ist vorüber.“

Trieest dankte den Skralinnen für ihre Zeit und ihren Versuch, und folgte Calrai aus der Höhle.

„Heute Abend findet ein Skralreigen statt!“, rief Trumm ihm nach, „Wir würden euch gerne dort dabeihaben.“

Sobald sie aus der Hörweite der beiden Skral-Schamanen vor der Höhle getreten waren, meinte Trieest: „Tja, du hattest recht. Die beiden sind wirklich ... eigenartig.“

Calrai nickte.

„So“, fragte er dann betont ausdruckslos, „Du bist also ein Held von Andor?“

„Das ist ein Titel, der wenig bedeutet“, wischte Trieest die Herausforderung beiseite. „König Thorald verlieh ihn nach dem Tod seines Vaters ganz vielen. Nicht alle sind so sehr in die Organisation eingebaut wie die überall bekannten. Jarid und ich ... kurz nach unserer Ernennung reisten wir mit Feuermeister Lifornus und Feuerwächterin Tenaya in den Osten, um an irgendeiner heiligen Stätte in der Barbarensteppe meinen Lavastein zu untersuchen. Brachte nichts, und als wir zurückkamen, hatte sich der Orden der Helden bereits in den Königsfrieden begeben. Wir folgen so ziemlich unseren eigenen Abenteuern. Zum Beispiel ...“ Trieest ging im Geiste Abenteuer durch, die er und Jarid erlebt hatten und nicht das Ermorden von Kreaturen beinhielten. Die Liste war erschreckend klein. „... zum Beispiel die Suche nach Oktarok. Betrunkene Fischer erzählten von einem feuerspeienden fliegenden Riesenkraken. War jedoch nur eine Illusion einiger Nixen.“

Calrai nickte erneut, wirkte aber noch nicht sonderlich überzeugt.

Die restlichen drei Skrale erhoben sich allesamt aufgeregt von ihren Tageslagern, als Trieest und Calrai näher traten.

Calrai blickte Trieest erwartungsvoll an. Dieser seufzte und sprach:

„Wenn ich das richtig verstanden habe, haben Trumm und Drunn nichts dagegen einzuwenden, wenn ich diese Sippe leitete. Aber auch nicht, wenn es anders wäre. Ihr wisst, wie ich dazu stehe. Aber ich verstehe auch, wenn ihr euch euren Traditionen folgend mir verpflichtet fühlt. Eine seltsame Situation, das Ganze. Vielleicht ist es am geschicktesten, wenn wir einige Nächte ... öhm, Tage darüber schlafen und uns danach entscheiden, ob ihr euch nicht doch lieber einer anderen Sippe anschließen wollt. Ich will inzwischen einige Botschaften an unsere Bekannten aussenden ...“

... um Andori nach Jarid zu fragen, fügt er im Stillen hinzu. Zerknirscht dachte er, dass er kaum mehr als das tun konnte. Wenn Jarid von den beiden Skralinnen nicht gesehen hatte werden können, dann ... nein, sie musste noch am Leben sein. Sie musste einfach.

Und dann gab es ja auch noch Lysbetts Warnung! Siedend heiß fielen ihm die Worte des Echos der gestorbenen Weinhändlerin ein. Die dunkle Macht des wilden Hraaks, die laut Lysbett irgendwo in der Gegend in ihrem Körper umherstapfte und Schaden anrichtete. Konnte er diese Gefahr vor lauter Sorgen um Jarid vernachlässigen?

„Botschaften absenden? Wie stellst du dir das vor?“, fragte Calrai vorsichtig.

„Na, wir gehen zum nächstbesten Falkner ...“, begann Trieest, „Nun, okay, vielleicht wäre dieser eher etwas erschreckt ob unseres Aussehens. Aber mich kennt man doch ... nun, auch nicht alle. Wie sendet ihr Skrale üblicherweise Nachrichten aus?“

„Na, wir teilen sie Trumm und Drunn mit, und die teilen sie den anderen Sippen mit“, lachte Calrai, „Oder wir reden einfach so mit ihnen. Manchmal, wenn es ganz wichtig ist, suchen wir die Gebirgsbarbaren auf. Einige Drachenkultisten dort sind uns nicht ganz so unwohl gesinnt wie diese hochnäsigen Flachländler. Einige Grehon richten Krarks ab, mächtige Riesenvögel, die für uns Nachrichten übermitteln können. Hui, diese scharfen Krallen! Und ihre Federn können sogar noch schärfer sein! Was wünschte ich mir, einen von diesen aufziehen zu dürfen. Die Kultisten lassen uns bitter bezahlen dafür, dass sie unsere Notmeldungen per Krark an die wichtigsten Skral-Häuptlinge übermitteln lassen, und abgesehen von verrosteten Waffen und vertrockneten Salben besitzen wir doch ohnehin kaum etwas von Wert. Aber wenn die Not ruft ...“

„Was du sagen willst, ist, dass wir von hier aus nicht so einfach mit den Menschen kommunizieren können?“

Calrai schüttelte betrübt seinen Kopf.

„Na dann. Was tun wir?“, fragte Trieest in die Runde.

„Können wir hierbleiben und am Skralreigen teilhaben?“, fragte Kurgat mit leuchtenden Augen. Etwas schüchtern fügte er an: „Es ist nicht so, als wollte ich selbst Nachwuchs für mich. Aber es ist immer so ein Erlebnis! Die letzten Male ließ uns Shron nicht daran teilnehmen, weil er es nicht aushielt, mit seinem gesunkenen Ruf anderen Skral gegenüberzutreten.“

„Tut, was ihr wollt“, sagte Trieest, „Ich werde mich wohl kaum wohlfühlen in einem Skralreigen. Aber wir können hier in der Nähe rasten. Ich bleibe hier und beobachte aus der Ferne.“

Und diese verrückte Situation verarbeiten.

Kurgat stieß seine Faust in die Luft.

Die ganze nächste Stunde wurde Trieest von vier begeisterten Skralen in die Tradition des Skralreigens eingeführt. Offenbar kamen verschiedensten Skralsippen zusammen, um in dieser heiligen Nacht ihren Nachwuchs zu empfangen. Der Skralberg war wohl durchzogen von uralter Drachenmagie, welche derjenigen ähnelte, welche die Kreaturen aus Krahal überhaupt erst geformt hatte. Manche uralte Naturgeister konnten diese Magie spüren, formen, nutzen. Uralte Feuergeister konnten aus den Überresten verstorbener Tiere junge Skralkinder formen. Ein für Trieest unglaublich gruseliger Prozess, der Skralen selbstverständlich erschien. Angeblich nicht so lustig wie die leibliche Zeugung von Nachwuchs, doch erheblich rascher als Ausbrüten von Eiern. Trieest beschloss, nur aus der Ferne zuzusehen. Er richtete sich auf seinem Hügel ein und sah zu, wie der Himmel sich verdunkelte und die Sterne zu blinken begannen.

Und er begann zu weinen. Furcht um Jarid, um seine Zukunft, vor dem Bösen in Lysbett. Alles, was ihm widerfahren war, prasselte auf ihn ein.

Er dachte, er wäre allein, bis auf einmal eine Stimme hinter ihm ertönte.

„Ist es okay, wenn ich dich berühre?“, fragte Calrai vorsichtig.

Trieest nickte.

Calrai trat näher und nahm ihn sanft in die Arme. Trieests Herz unter der feuersteinbefreiten Brust begann, schneller zu schlagen. Feine Feenflügel flatterten in seinem Bauch.

Als er einatmete, vernahm er von Calrai nicht mehr der Geruch nach verfaultem Fleisch, vielmehr roch er Gras und Erde, Metall und alter Stoff ... Er atmete schneller. Es war zu viel.

Abrupt löste sich Trieest von Calrai und stolperte einige Schritte zur Seite. Calrai hielt inne und zog sich rasch zurück.

„Verzeih mir...“, setzen beide gleichzeitig an, nur um wieder abzubrechen.

Calrai wandte sich ab und setzte zum Gehen an.

„Nein“, murmelte Trieest, „Bitte, blieb hier. Wenn du willst.“

Calrai drehte sich wieder zu ihm um, ein fragender Ausdruck im Gesicht.

„Wie kann ich helfen?“

„Ich... ich weiß auch nicht. Es ist einfach zu viel. Ich vermisse sie so sehr“, flüsterte Trieest, während flammende Tränen aus seinen Augen tropften. Ein wenig unbeholfen setzte sich Calrai neben ihn und tätschelte seinen Kopf, sorgsam auf Trieests Reaktion achtend. Trieest lächelte schwach und umarmte Calrai, drückte ihn fest an sich. Calrais Atem stockte kurz, dann erwiderte er die Umarmung kräftig. Calrais schuppiger Schwanz umschlang Trieests Beine, ganz sanft. Verschlungen blieben sie liegen.

Zum ersten Mal seit ihrem Verschwinden glaubte Trieest, Jarid ein bisschen weniger zu vermissen.\bigskip







In der Ferne versammelten sich die Kreaturen. Trumm und Drunn führten ihre Schamanen zu einem alten Steinkreis. Aus allen Richtungen strömten Kreaturen herbei. Trieest sah Trolle, die wilde Wardraks an Eisenketten hielten, Fluggors, die Gors in Windeseile herbeiflogen (von denen nicht alle ihren Mageninhalt bei sich behalten konnten), gar den einen oder anderen Menschen, Agren und Zwerg. Es war eine gloriose Versammlung. Noch vor wenigen Tagen hätte er sich Gedanken gemacht, wie er diese ganze Gesellschaft auf einmal auslöschen konnte. Ihre Grogfässer vergiften oder so. Mörder waren sie, die meisten da unten am Hügel, die ausgelassen um ein Feuer tanzten. Mörder und Ungeheuer. Wie er selbst.

Calrai hatte sich irgendwann von ihm gelöst und war zur Versammlung spaziert, immer wieder einen Blick nach hinten werfend.

„Noch ein Platz frei hier?“, meldete sich eine heisere Stimme neben Trieest. „Ich beobachte die Feierlichkeiten auch von weiter weg.“

Trieest fuhr herum. Hinter ihm stand ruhig und still, als wäre er eben erst aus dem Schatten gewachsen, ein Halbskral. Ein Trieest flüchtig bekannter Halbskral. Der mysteriöse Schattenskral, der der Hexe Reka eine Zeit lang als Schüler gefolgt war. Jarid und Trieest hatten ihn nach dem Kampf um die Rietburg angetroffen, als Reka sich gegenüber einer rabiaten Bäuerin für das Leben des Schattenskrals eingesetzt hatte. Damals hatte Trieest ihn mit Abscheu betrachtet. Nun wusste er nicht mehr, was er fühlte.

Trieest nickte dem Schattenskral zu. Dieser setzte sich in den Schneidersitz, legte seinen stachelbewehrten Schwanz in seinen Schoß und beobachtete aufmerksam den fernen Skralreigen aus sehr menschlich wirkenden Augen.

„Ich kenne dich“, murmelte der Halbskral dann, ohne Trieest anzusehen, „Du warst bei der Befreiung der Rietburg dabei. Zuerst beim Gefecht um den Wehrturm und dann sogar innerhalb der Mauern. So weit traute ich mich nicht. Ich sah dich vorbeiziehen, als Reka mich vor dieser Bäuerin rettete. Hast Münzen gezählt. Ein Söldner bist du?“

Trieest rutschte unruhig hin und her. „Diese Zeiten sind hinter mir. Nun erhalte ich Kost und Logis andershalber.“

„Als Held von Andor, häh? Der Königssohn ernannte dich bald darauf zu einem, erzählte mir Reka. Ich roch schon damals, dass an dir etwas Besonderes war. Konnte es jedoch nicht genau einordnen. Du bist wie ich, oder? Dieser Stein in dieser Brust verdeckte deine wahre Natur, was?“

Trieests Stirn runzelte sich.

„Nun, vielleicht auch nicht“, meinte der Schattenskral, „Vielleicht brachte er dich auch deiner wahren Natur näher.“

„Was meinst du damit?“

Der Schattenskral peitschte mit seinem Schwanz. „Ich will mir nicht anmaßen, zu kapieren, was dieser feurige Klunker dir antat, aber ich kann mir einige Halbskrale vorstellen, die ihre Hand für so ein Ding gäben, das ihr Antlitz menschlicher formt. Ihre wahre Natur unterstützte quasi.“

Auf einmal kam ein eigenartiges Schuldgefühl in Trieest auf. Er hasste seinen Lavastein, für den Forn vielleicht so einiges gegeben hätte. Wenn er funktioniert hätte. „Dieser Stein war kein Segen. Diese Bürde habe ich nicht gewählt. Und gebracht hat sie auch nichts.“

„Dann wurde deine wahre Natur gegen deinen Willen unterdrückt“, sprach der Schattenskral bestimmt, als wäre damit alles gesagt.

„Was willst du damit sagen, Schattenskral?“, fragte Trieest, „Bin ich nun ein Skral oder nicht?“

Der Schattenskral grinste. „Schattenskral nennt man mich nur, wenn man mich noch nicht kennt. Ich bin Scheu.“ Er schlug begrüßend auf Trieests Unterarm.

„Das ... ja, das mag so sein“, antwortete Trieest, immer verwirrter. „Ich bin auch eher scheu.“

„Nein, ich meine, ich bin Scheu. Das ist mein Name. Scheu.“

„Oh, verzeih. Ich bin Trieest. Wer nennt sein Kind denn Scheu?“

„Reka nannte mich so. Sie war es, die mich aufzog.“

Forn räkelte sich und fuhr fort: „Um auf deine Frage zurückzukommen, ob du ein Skral bist oder nicht. Was erhoffst du dir von meiner Antwort? Niemand kann bestimmen, ob du ein Skral bist, außer du selbst. Was fühlt sich richtig an? Fühlst du dich einer Sippe zugehörig? Fühlst du dich hier oder unter den Helden wohler? Es muss keine eindeutige Antwort darauf geben. Doch wenn ein Skral ein unschuldiges Kind verfolgt, gibt es für mich nur einen unter ihnen, mit dem man mitfiebern sollte.“

„Und deswegen siehst du dich nicht als einer ihrer Brüder?“

„Nein. Ja. Vielleicht. Ich tue es einfach nicht. Am Ende ist es auch nicht so wichtig, wie wir uns nennen. Was zählt, sind Taten und Gefühle. Dass du deinen eigenen Weg findest. Egal, ob du nun ein Skral bist oder nicht. Vielleicht findest du unter ihnen deinen Weg. Vielleicht nicht.“

„Ich wurde soeben mit Handkuss von einer kleinen Sippe aufgenommen“, sprach Trieest, „Sie hatten kein Problem mit mir als Halbskral. Du willst sie nicht zufälligerweise übernehmen?“

„Zeiten wandeln sich“, überlegte Forn. „Nicht alle Skral-Sippen sind gleich. Vielleicht hätte ich unter anderen Obersten ein fröhliches Dasein als vollwertiger Skral erlebt, statt verstoßen zu werden. Vielleicht hätte mich irgendein Schamane aufgenommen und ausgebildet. Was bringt es, über andere Welten zu sinnieren? Ich lebe in dieser hier, und ich will nichts mehr mit ihnen zu tun haben. Sie sind meine Geschwister nicht.“

„Und doch bist du hier, um dem Reigen zuzusehen?“

„Und doch bin ich hier“, nickte Forn.

Beide blickten stumm in die Ferne. Dort, inmitten des uralten Steinkreises, waren Drunn und Trumm indes sehr nahe aneinander getreten. Überraschend gewandt begannen die beiden Skralinnen einen langsamen Tanz zu Musik, die nur sie zu hören schienen. Andere Skrale sprangen mit ein. Leises Gesumme schwebte zu Forn und Trieest herüber. Einen kurzen, verrückten Augenblick lang überlegte Trieest, ob er den Schattenskral zu einem Tanz einladen sollte.

„Wirklich keine Lust, selbst teilzunehmen?“, hakte Trieest nach. „In einer ausgelassenen Gesellschaft ein bisschen Optimismus für die Zukunft schnappen? All die Zeit einsam im Walde muss doch aufs Gemüt schlagen.“

„Nee“, verneinte Forn. „Du missverstehst mich. Ich sehe gerne dem Skralreigen zu, um zu ahnen, wie es um die Skrale stehtt. Es freut mich zu sehen, wie es Jahr um Jahr weniger werden. Irgendwann wird der letzte Skral gemordet haben. Vielleicht erleben wir diesen Tag noch.“

Trieest schluckte schwer. Forn verzog seinen Mund und fuhr fort. „Verzeih, das war ein Dunkler Gedanke. Ich würde auch so nicht an Tänzen teilnehmen wollen. Mir behagten große Gesellschaften ohnehin nicht, und einen optimistischen Blick in die Zukunft kann ich nicht teilen. Ich bin der Schattenskral. Ich brauche nicht im Rampenlicht zu stehen, um glücklich zu sein. Ich brauche nun die Einsamkeit, die ich einst verfluchte. Einsamkeit und eine Aufgabe, wie Reka sie mir gab. Das gibt mein Leben einen Sinn. Was gibt deinem Leben so etwas?“

„Andere Personen. Gemeinsam mit ihnen etwas zu erreichen. Leben zu retten, Höfe zu reparieren, Bedrohungen vorherzusehen und abzuwehren.“ Trieest war überrascht darüber, wie rasch seine Antwort gekommen war, und wie leer seine Worte klangen. Er und Jarid hatten viele großartige Taten vollbracht, sicher, aber wie viele davon nur, weil ihn der brennende Lavastein in seiner Brust angetrieben hatte?

„Na eben, das ist bei mir auch so“, brummte Forn, „Nur, dass es mich nicht stört, wenn mich keiner bei meinen Taten sieht. Außerdem wollen die uns da unten nicht mittanzen stehen. Die meisten sehen uns beide immer noch als abscheuliche Mutationen.“

„Obwohl sie kein Problem damit haben, dass Menschen unter ihnen sind? Sieh, da drüben tanzt doch eine.“

„Einige Kreaturen haben auch damit ein Problem. Aber weniger. Ein Skral mit Menschenaugen, das ist für die meisten einfach nur abscheulich, die würden uns nicht einmal anfassen. Ein reiner Mensch oder Zwerg, das geht in Ordnung, die kann man ruhig verspeisen. Und sie brauchen manche Menschen als Drachenkultisten. Schließlich können bloß Menschen den Geist eines Drachen in sich aufnehmen, so sagen die Oberen. Geister gefallener Drachen in Menschen und Leibe gefallener Drachen in uns Kreaturen, nur gemeinsam vermochten diese Gruppen es, sich Taroks Befehlen zu widerstehen.“

„Es gab Kreaturen, die sich der Führung des Drachen widersetzten?“

Forn grinste. „Du hast keine Ahnung, was alles für Konflikte hier im Gebirge abgingen, während ihr Helden euch nur um die Sicherheit eurer Burgen und Höfe kümmertet, oder? Nehal zum Beispiel, das ist ein Drache, den so gut wie jeder im Skralreigen mag. Einer der Alten, der es gut mit uns allen meinte. Sein Beseelter, Nehamal, ist vielleicht irgendwo da unten am Tanzen, und singt in den höchsten Tönen von ihm. Tarok hingegen war bloß ein verbittertes Biest, das den Häuptlingen, die ihm nicht folgen wollten, seinen Willen aufzwang. Nicht alle hier mögen Tarok..“

„Und doch folgten ihm viele freiwillig.“

„Dieselben, die uns auch widernatürliche Dinger schimpfen und verstoßen würden, wenn wir uns zu ihnen gesellten.“

„Sollen sie doch. Es ist eine heilige Nacht. Sie werden kein Blut vergießen.“

„Sie könnten uns dennoch schaden. Aber tu dir keinen Zwang an, Trieest, wenn du gehen willst, dann gehe. Es geschehen noch Zeichen und Wunder. Ha! Ein Held von Andor, der unter dem Skralreigen tanzt.“

„Dieser Titel wird von König Thorald leichtfertig vergeben. Wer weiß, wenn du dich anstrengst, wird er ihn eines Tages auch dir verleihen.“

„Der gibt noch eher freiwillig die Krone ab“, lachte Forn. „Du bist ein Träumer. Trieest. Das gefällt mir. Meine Träume waren lange dunkel.“ Er schauderte, als er mit starrem Blick hauchte: „Gelbe Augen in der Dunkelheit. Scharfe Zähne und spitze Klauen. Das Gefühl, nicht der Jäger, sondern der Gejagte zu sein. Mein eigenes Wimmern, das Wimmern eines Ausgestoßenen, dessen Einsamkeit mehr schmerzte als die blutenden Wunden in meiner Seite. Manchmal frage ich mich, wie ähnlich ich dir geworden wäre, wenn ich von Anfang an unter Menschen aufgewachsen wäre. Da, jetzt machst du mich auch zu einem Träumer.“

Forn hob eine schuppige Augenbraue: „Aber, wenn du deine neue Sippe im Stich lassen willst, wäre jetzt die Gelegenheit dazu. Ich kann dir sogar einen Geheimgang zeigen.“

Forn führte Trieest zurück zur nun verlassenen Höhle der Skral-Hexen. Er führte ihn tief in das Gewölbe hinein, bis zur einer moosbewachsenen Felswand. Diese klopfte er ab und horchte. Trieest vernahm nichts Besonderes, doch Forn nickte zufrieden.

„Hier ist es“, sprach er. Mit spitzen Krallen folgte er schwachen Rillen in der Wand, welche unter seiner Berührung leicht aufleuchteten. Er formte ein Dreieck mit einer Spitze nach oben. Dann flüsterte er eine Botschaft in einer uralten Sprache, die Forn nicht verstand. Rot leuchtende Linien entflammten entlang der gesamten Wand und formten einen Torbogen, der Trieest und Forn um ein Vielfaches überragte.

Mit einem Knirschen öffnete sich die Tür gegen innen und tauchte Forn und Trieest in rotes Licht. Weiter hinten im Gang glaubte Trieest, Runen zu erkennen, die in die Wände geritzt waren und von denen das rote Leuchten ausging. Das beruhigte ihn. Demnach waren diese Wände eindeutig von Zwergen bearbeitet worden, wenn nicht sogar von ihnen geschaffen. Er war sicher hier. Es war nur seltsam, dass nirgendwo Fackeln hingen.

„Zwergentüren“, nickte Forn zufrieden, „Ein unterirdisches Netz aus Gängen, welche die Gegensätze von Feuer und Wasser vereinen, geschaffen von den Vorfahren der Schildzwerge. Nur wenige wissen davon, und noch weniger nutzen sie. Reka mag sie, um rasch vom westlichen ins östliche Rietland zu gelangen. Keine Ahnung, ob die Skral-Hexen wissen, was für ein Zugang hier direkt in ihrem Schlaflager liegt. Auch nicht so wichtig. Lass dich von der Tür nicht täuschen, weiter hinten wird der Gang so eng, dass die Hexen gar nicht durchpassen würden. Führt direkt in den Wachsamen Wald, falls du dort glücklicher wirst.“

Der Wachsame Wald war gut. Vermutlich trieb die Finstere Lysbett dort noch ihr Unwesen. Und mit ein wenig Glück war Jarid nahe zum Baum der Lieder getunnelt. Falls es ihr gut ging. Falls sie ihn in seinem jetzigen Zustand überhaupt noch annehmen würde. Furcht umfasste sein Herz.

Forn legte Trieest seine Hand auf die Schulter und sprach überraschend schwer: „Dies ist eine wichtige Entscheidung, die du hier triffst. Wende dich von deinen Skralen ab oder entscheide dich für sie. Sei rasch und schließe die Tür wieder, ehe Drunn und Trumm dich hier finden – diese Zwergentür ist anders als andere nicht durch einen Tarnzauber geschützt, während sie offen steht.“

„Und du? Was tust du, Forn?“

„Mein Pfad führt mich weiter ins Gebirge. Ich hoffe aber, dass sich unsere Wege wieder unter freundlichen Umständen kreuzen.“ Bereits zum Gehen abgewandt, drehte sich Forn ein letztes Mal zurück. „Hast du dich schon je gefragt, warum ich nicht den Trank der Hexe brauen kann? Ich war lange Rekas Schüler. Mein Vorgänger, dieser Meres, mit dem sich Reka einst so schlimm verstritt, dass er sich nie wieder traute, ihr unter die Augen zu treten, er bekam Rekas geheimste Rezepte beigebracht. Und meine Nachfolgerin, die kleine Chada, mit der braut Reka bereits oft über großen Kesseln, auch wenn sie ihr noch nicht die letzte Zutat verraten will. Doch mit mir braute sie nie Hexentränke. Nicht, weil sie sich vor meiner Skralseite fürchten würde, und nicht, weil Meres es ihr vergrault hätte. Sondern weil ich mich nicht gleich dafür interessierte. Pflanzen zu sammeln und wachsen zu lassen, das ist viel eher mein Ding als ätzender Rauch und große Feuer. Sie ist weise, die alte Reka. Sie hat mir eine Wahl gelassen und mich auf demjenigen Pfad angefeuert, den ich mir aussuchte. Und nun stehst du an einer Schneise, Trieest, und ich kann dir nur sagen, was sie mir damals mitteilte: Welchen Pfad du auch begehst, du kannst dort glücklich werden und Gutes tun. Doch einen Pfad musst du entlanggehen, sonst bleibst du stehen.“

„Was soll das heißen?“, fragte Trieest.

Forn winkte ihm zu und ließ ihn im rötlichen Schein der offenen Zwergentür stehen.\bigskip







\textit{Die finstere Feuerkriegerin betrat die Rote Grotte. Sorgsam strich sie über den Lavastein, der die Höhlenwände in unregelmäßigen Abständen durchzog. Das Böse hatte die Erinnerungen der Feuerkriegerin gesehen. Diese Höhle sollte von einem beständigen Summen, einem Rauschen von uralten Stimmen erfüllt sein. Stattdessen war sie totenstill.}

\textit{Das Böse erhob die fremde Stimme der Feuerkriegerin: „Ich bin hier, um mit einem Toten zu sprechen. Ich hatte nie die Gelegenheit, mich zu verabschieden.“}

\textit{Leise echoten die Worte durch den leeren Raum. Antizipation lag in der Luft.}

\textit{„Meine Rachepläne von damals versagten, und als ich meine Präsenz endlich wiedervereint hatte, warst du bereits verstorben. In den Jahren und Aberjahren meiner Existenz, gefangen hinter einem roten Glas, gezwungen, durch fremde Sinne diese Welt zu erblicken, versiegte mein Rachedurst und meine Wut auf dich. Komm, Bruder, melde dich ein letztes Mal. Sprich zu mir, auf dass ich mit deinen Worten zur Burg zurückkehren und den mir versprochenen angestammten Platz einnehmen kann. Du weißt, dass dein nutzloser Sohn deinem Reich schadet. Mit mir an seiner Seite ...“}

\textit{Es verstummte. Keine Reaktion. Keine Stimme, wie es sie in Trieests Lavastein vernommen hatte. Wut machte sich in ihm breit. Wie konnten diese Toten es wagen, es zu ignorieren?! Es kontrollierte den Tod! Er war das mächtigste Wesen, dass je diese Höhle betreten würde, eine wahre Naturgewalt, und sie hielten sich für besser als es!}

\textit{Wut übernahm seine Gedanken. Wieder und wieder schlug es gegen die Rote Grotte, bis ihre Seitenwände Risse kriegten und die Fäuste der Feuerkriegerin brachen.}

\textit{Erst, nachdem das Böse aus der Höhle gestolpert war, erhoben sich die Echos der Gefallenen wieder und wurden wie immer zum Geräusch des brandenden Meeres, das allen kühlen Lavasteinen inne war.}
















\newpage
\section{Die entlegenste Insel im Hadrischen Meer}





\az{Jahr 61}

\textit{Halle des Ältestenrats, 61 a.Z.}\bigskip



Mit einem letzten Lächeln lehnte sich der grantige Kord wieder an die Stuhllehne. Der Ältestenrat hingegen brach in völliges Chaos aus. Gemurmel, Gebrabbel, Geflüster und Rowindas durchdringende Stimme, die sich an ihre Tochter wandte:

„Jarid, du hast als Wassermagierin einen Eid geschworen, Danwar und seine Einwohner zu beschützen! Du kannst uns nicht einfach verlassen!“

Jarid wich dem Blick ihrer Mutter aus und schwieg. Rowinda kannte das alte Gesetzt genausogut wie sie. Als Einwohner Danwars konnte Trieest den Schutz des Ordens der Wassermagier beanspruchen. Wenn Jarid sich Trieest auf seiner Reise anschließen wollte, so konnte sie der Ältestenrat nicht aufhalten.

Die straffte ihr Gewand und blickte dem Ältestenrat stolz entgegen.

„Führt mich zur Roten Grotte.“\bigskip









\az{Jahr 68}

\textit{Sieben Jahre später.}\bigskip



Jarid schlotterte vergeblich gegen ihr gefrorenes Gefängnis. Kaltes Eis umfasste und unterkühlte sie. Nur ihr Kopf war frei. Vor ihr stand Jivin, die kleine Wassermagierin aus dem mindestens fünften Zirkel, die sie gefangen genommen hatte, wie auch das Böse ihren Körper gefangen genommen hatte.

Etwas weiter weg hatten drei Feuerkrieger mit tiefschwarzen Augen und Lavasteinen in ihren Rüstungen vier weitere Wassermagier mit ihren Rankenschwertern gefesselt, darunter Jarids Mutter Rowinda. Die Feuerkrieger starrten regungslos in die Ferne, genau wie Jivin.

Jarid beschwor weitere Geisteskraft in das sie umgebende Eis und versuchte, Wärme aus der Umgebung in einen bestimmten kleinen Teil zu lenken. Wassermagier waren schließlich nicht auf ihre Handgesten angewiesen, um Wasser zu lenken. Diese verstärkten die Verbindung bloß.

Plötzlich klärte sich der Blick der kleinen Wassermagierin wieder. Ihr Mund verzog sich zu einer wütenden Miene und sie schien sich einen Moment fassen zu müssen, ehe sie grunzte:

„Ach, du bist ja auch noch hier. DAnn war mein Besuch hier zumindest keine völlige Verschwendung. Wer außer dir wäre schon am besten in der Lage, Trieest aufzuspüren? Ich bin an seinem Lavastein interessiert. Und einem bestimmten Echo, das darin nachhallt. Aber ich komm‘ ja ganz in Plapperlaune, das braucht dich doch alles nicht zu kümmern.“

Jivin zupfte mit spitzen Fingern den schwarzen Edelsteinsplitter aus ihrem Unterarm. Hellrotes Blut ergoss sich über selbigen, aber das schien sie nicht groß zu kümmern.

Dann beugte sie sich vor und massierte das Eis, das Jarids rechten Arm einschloss. Ein kleiner Teil schmolz prompt und legte eine Stelle um Jarids Handgelenk frei, kaum größer als eine Münze. Die kleine Wassermagierin riss Jarids Ärmel achtlos auf und hob den schwarzen Kristallsplitter hervor, direkt auf Jarids Unterarm gerichtet.

Jarid konnte sich denken, was sie vorhatte, und sie konnte sich denken, was geschehen würde, sobald der schwarze Kristallsplitter erst einmal in ihrem Arm steckte. Furcht lähmte ihren Geist, wie bereits ihr Körper gelähmt war. Sie schoss ihre Angst nieder, schloss ihre Augen und verdrängte die Eiseskälte aus ihren Gliedern. Es zählte nur das hier und jetzt. Sie war eins mit dem Wasser. Und das Wasser wollte nicht vereist sein. Die Umgebung war zu heiß dafür. Das Eis wollte schmelzen, das Wasser wollte fließen, und Jarid konnte das Eis an einigen konkreten Stellen erlauben, zu fließen ...

Es knackte und knirschte, und Jarid riss ihre Hand aus dem bröckeligen Eisblock, der sie eingeschlossen hatte. Die kleine Wassermagierin riss ihre Augen auf und stieß einen Fluch in einer kehligen Sprache aus, die Jarid nicht verstand. Zu langsam!

Jarids Hand, noch immer zur Hälfte in Eis eingeschlossen, schlug gegen die Faust der kleinen Wassermagierin. Eissplitter explodierten. Jivid taumelte zurück. Ihr Griff löste sich und der kleine schwarze Kristallsplitter vollführte eine schöne Parabel durch die Luft, ehe er von einer Dampfsäule aus dem löcherigen Boden erfasst wurde und aus Jarids Blick verschwand.

Die drei Feuerkrieger standen nicht mehr anteilslos abseits. Stattdessen stürzten sie nach vorne, auf die Stelle zu, wo der Splitter wohl gelandet war. Sie kamen ein, zwei Schritte weit, ehe sie einer nach dem anderen zur Seite kippten und am Boden zuckend liegen blieben. Rotes Leuchten drang aus den Lavasteinen in ihrer Brust, wie als Trotz gegen die Tatsache, dass die Edelsteine vor wenigen Momenten noch von einer tiefen Dunkelheit erfüllt gewesen waren. Hinter ihnen fielen die Rankenschwerter von den gefangenen Wassermagiern ab.

Auch Jivin war zur Seite gekippt, als das Böse seinen Halt über sie verlor. Jarid brach aus dem Eiswall hervor, der sie am Boden gehalten hatte. Dann war ihre Mutter auch schon bei ihr und schickte die letzten Eisstücke schwungvoll zur Seite.

„Wie schlimm verletzt bist du?“, fragte Rowinda leise und tastete Jarids Bauch ab. Jarid biss ihre Zähne zusammen und meinte: „Es geht schon. Ich war bereits bei einem Heiler. Zumindest einem Alchemisten. Einem danwarischen, sogar.“

Rowinda blickte sie zweifelnd an, hakte aber nicht nach. Stille breitete sich aus zwischen Mutter und Kind. Rowindas Augen waren milchiger, doch hielt sie ihr weißes Haar immer noch genauso hochgesteckt wie früher. Jarid versuchte, nicht daran zu denken, wie lange es her war, seitdem sie ihre Mutter gesehen hatte. Der Gedanke daran erfüllte sie mit Schuldgefühlen.

„Was weißt du über dieses Wesen? Was wollte es von dir?“, fragte Rowinda, ohne auf die lange Absenz ihrer Tochter einzugehen.

„Es hat gesagt, es wolle zu Tri, wegen eines Echos. Wir haben bereits vor Kurzen gegen es gekämpft. Damals konnte es aber nur einen Geist auf einmal kontrollieren. Ich vermute, dass es in Tris oder meinen Erinnerungen Danwar gesehen hat und deswegen hierher kam. Vielleicht konnte es sich hierher teleportieren?“

Rowindas Gesicht verhärtete sich beinahe unmerklich bei der Erwähnung von Trieest. Betont kühl fragte sie:

„Das heißt, dass sein Prozess des Wandels noch nicht beendet ist?“

„Mutter, nicht jetzt. Was hat dieses Wesen hier angestellt?“

Rowinda nickte entschlossen: „Du hast ja Recht. Aber viel zu erzählen gibt es nicht. Ich weiß nicht, wann es hier ankam und in welchem Körper. Es konnte Personen bei Berührung mit seinem schwarzen Kristallsplitter kontrollieren, und Feuerkrieger sogar länger, wenn es ihre Lavasteine kontaminiert hatte. Es verlangte, dass wir dich hierher bringen. Wir haben beschlossen, uns nicht unnötig zu widersetzen und den richtigen Moment zur Auflehnung aufzuwarten.“

„Und der kam nie?“, fragte Jarid etwas spöttisch, ehe sie sich fasste und fortfuhr: „Tri und ich fanden diesen Kristall im Schädel eines uralten Ungeheuers, des wilden Hraaks. Er zerbrach, und doch scheint das Böse ungebrochen. Es scheint durch diesen Splitter andere Personen steuern zu können. Das ist nicht gut, hoffentlich gibt es nicht noch weitere solcher Splitter.“

„Fokussiere dich auf das, was du lenken kannst, Jarid. Lass die restlichen Sorgen sein. Dieser Splitter befindet sich noch hier in der Nähe, oder? Wie können wir ihn davon abhalten, weiteren Schaden anzurichten?“

Jarid zuckte schwach mit den Schultern: „Keine Ahnung. Wichtig scheint bloß zu sein, den Kontakt zwischen dem Kristall und gesteuerten Personen zu unterbrechen. Tri kam wieder vollständig zurück, und es steht zu hoffen, dass es den anderen auch so geht. Ich hoffe nur, dass Trieest nicht rückfällig wurde, als das Böse diese Feuerkrieger übernahm.“

Rowinda blickte wachsam um sich und beobachtete die gestürzte kleine Wassermagierin und die umgefallenen Feuerkrieger. Diese richteten sich inzwischen langsam wieder auf und sahen sich einigermaßen verwirrt um. Keine Spur aggressiven Verhaltens. Die Lavasteine leuchteten wieder in einem warmen Rot. Die Welt schien wieder heil zu sein.

„Die Kontrolle zu seinem Hauptleib zu unterbrechen, scheint auch die Kontrolle über die Feuerkrieger gekappt zu haben. Sehr praktisch, damit dürften auch die restlichen Gefangenen dieses Parasits befreit worden sein. Gut gemacht, Jarid.“

In Jarids Augen blickend, schien Rowinda sich zum ersten Mal wirklich auf ihre Tochter zu achten. Sie senkte ihren Kopf und meinte:

„Grün sind die Wogen der Wellen. Es freut mich, dich wieder zu sehen, Jarid. Und es tut mir leid, dich hierher getunnelt zu haben. Es war zu gefährlich. Wir hätten uns wehren sollen.“

Die passende Begrüßungsfloskel war schnell entgegnet. Ungelenk richtete sich Jarid auf und überblickte das Becken von Quodlon.

„Wir müssen den Splitter finden.“

Rowinda blickte sich besorgt um: „Meinst du etwa, der Splitter allein könnte etwas Finsteres anrichten? Kann das Wesen ihn auch jetzt noch bewegen?“

Als hätte es darauf erwartet, schoss ein klitzekleines schwarzes Etwas aus dem Nebel hervor und grub sich in Rowindas Rücken. Diese grinste schmerzverzerrt, langte hinter sich und zog etwas hervor. Einen kleinen, altbekannten schwarzen Kristallsplitter, von dessen Spitze ein Blutstropfen rötlich schimmerte.

„Zu langsam, Jarid. Viel zu langsam“, kicherte das Böse aus Rowindas Mund und verzerrte ihn zu einem Lächeln. Finster-Rowinda tat einen mächtigen Satz über Jarids Kopf und landete im Becken von Quodlon. Eine gewaltige Welle schwappte auf und trug sie weg von Jarid, näher zu der Gruppe der restlichen Wassermagier und Feuerkrieger, die sich noch verwirrt vom vorherigen Geschehen erholten.

Laut pochte das Blut in Jarids Ohren, als sie Furcht und Ärger erneut unterdrückte und zur Handlung schritt. Jarid sprang nach vorne und ergriff selbst Kontrolle über einen Wasserberg, ließ sich davontragen und surfte ihrer Mutter nach, quer über das Becken von Quodlon. Nebenbei krümmte sie ihre Finger und versuchte, die Welle zu teilen, auf welcher der Körper ihrer Mutter ritt. Das Böse drehte sich elegant um die eigene Achse und schnippte verächtlich. Jarids Wellenberg zerbarst. Feurig heißes Quellwasser umschlang sie.

Jarid schrie auf und verband im Geist das heiße Wasser um sie herum mit den Überresten des Eisblocks, der am Uferrand vor sich hin schmolz. Ihre Muskeln verspannten sich, während die Wärme durch sie durch glitt. Schwarze Schlieren umschlagen den Rand ihres Gesichtsfelds. Das Wasser, das ihren Körper umspülte, kühlte sich gerade lange genug ab, damit Jarid einen tiefen Atemzug tun und sich sammeln konnte. Dann wurde sie eins mit dem Becken von Quodlon. Sie war der Dampf, der über dem Becken stieg. Sie war die Wogen, die sich auf der Oberfläche kräuselten. Sie war die Schwingungen, die die Wassermassen in Bewegung hielten. Das Wasser wusste nicht, dass der Boden des Beckens speziell bearbeitet worden war, um bestimmte Resonanzen zu ermöglichen. Das Wasser wusste auch nicht, wie man diese auslöste. Aber sie war nicht nur Wasser, nein, sie war auch Jarid, die Magierin, und Jarid wusste, was sie zu tun hatte.

Sie lenkte ihre ganze Konzentration auf das Wasser. Versuchte, all die Wassermassen gleichzeitig in ihren Gedanken zu haben. Sie stellte sich jeden Tropfen vor, von der Oberfläche bis hinab zur tiefsten Stelle. Und als sie das geschafft hatte, gab sie jedem dieser Tropfen in Gedanken einen kleinen Schubs.

Dann noch einen.

Und noch einen.

Ein dumpfer Klang ertönte, als das Becken von Quodlon als riesige Klangschale zu schwingen begann. Jarid spritzte blitzschnell aus der Mitte des Beckens hervor und schwappte ans Ufer, wo sie wieder zu einem Menschen zusammenfloss. Sie war schwach und ihr war kalt, aber das war jetzt nicht wichtig. Wo war ihre Mutter? Wo waren die restlichen Feuerkrieger?

Im allgegenwärtigen Dampf neben dem Becken war es nur schwer zu erkennen, doch da vorne regten sich definitiv Schemen. Jarid versuchte, den Nebel zur Seite zu lenken, doch ihre Kräfte gehorchten ihr nicht mehr. Jeder Atemzug schmerzte, und sie wollte nur noch liegen. Nicht jetzt, rief sie sich zu.

Da trat eine Gestalt aus dem Nebel. Eine großgewachsene Feuerkriegerin. Ihr linkes Auge fehlte und diese Gesichtshälfte war vollkommen von vernarbter Haut überzogen, welche an angeschmolzenes Wachs erinnerte. Jarid erkannte sie. Sie war dabei gewesen, als die damals Kleine ihren Unfall gehabt hatte. Nun hatte sie es doch noch ihre Ausbildung zur Feuerkriegerin abgeschlossen. Davon zeugte der Lavastein in ihrer Brust. Doch war dieser gerade so finster, als wäre ein Schatten darin eingezogen. Jarids Herz sank.

„Ah, das bist du ja“, krächzte die Feuerkriegerin fröhlich, „Ich dachte schon, ich hätte dich verloren. Das wäre zu schade gewesen.“

Die Feuerkriegerin zerrte an etwas hinter ihr und schleuderte unter einigem Ächzen einen bewusstlosen Feuerkrieger nach vorne. Dieser trug noch einen rot schimmernden Lavastein, doch half ihm dieser gerade relativ wenig. Ein Rankenschwert steckte tief in seinem blutigen Hals

„Lauf nicht weg“, rief die Feuerkriegerin grinsend zu Jarid.

„Ich bin gleich voll bei dir“, antwortete Finster-Rowinda, welche hinter der Feuerkriegern hervorschritt und den schwarzen Kristallsplitter gegen den roten Lavastein des sterbenden Feuerkriegers hielt. Das rote Leuchten wich schwarzen Schatten, als würde eine finstere Essenz vom Kristallsplitter in den Lavastein gekippt werden. Der Feuerkrieger setzte sich auf, riss belanglos das Rankenschwert aus seinem Hals und sah schon fast gelangweilt zu, wie sein Lebenssaft herausschwappte.

„Beim ersten Mal dachte ich noch, dass es wirklich etwas Besonderes wäre, über die Macht der Nekromantie zu verfügen. Aber alles verliert seinen Charme, wenn man es zu oft macht“, intonierte er.

Neben ihm stellten sich Finster-Rowinda und die vernarbte Feuerkriegerin auf.

Jarid hob ihre Hände, aber das Wasser wollte ihr weiterhin nicht gehorchen. Noch war sie zu erschöpft.

„Ich mache es dir einfach“, meinte das Böse dreistimmig. „Ergib dich mir und lass mich deinen Körper führen. Hilf mir, deinen Trieest zu finden. Danach gehen wir unsere getrennten Wege. Niemand muss zu Schaden kommen.“

„Versprechende Worte eines Wahnsinnigen haben keinerlei Wert!“, erwiderte Jarid erheblich selbstsicherer, als sie sich fühlte. Feuerkrieger mochte das Böse über die Lavasteine viele auf einmal steuern können, doch Wassermagier konnte es wohl immer nur einen auf einmal führen. Wenn das Böse ihre Mutter kontrollierte, mussten die restlichen Wassermagier noch frei sein. Wenigstens einer hatte sich doch bestimmt gegen die Feuerkrieger wehren können. Das nächste Dorf lag kaum einen Steinwurf weit von hier. Hatte jemand bereits Alarm geschlagen und Hilfe geholt?

„Wenn du dich nicht ergibst, werde ich gezwungen sein, dich mit Gewalt zu ergreifen“, erklang die Stimme des Bösen, „Und deine liebe Mutter wird natürlich die Narne hinuntergehen müssen. Was für eine Schande das doch wäre.“

Zwei scharfe Rankenschwerter wurden auf Rowinda gerichtet, die Klingen wie flackernde Flammen umherwabernd. Jarids Mutter lächelte, ergriff beide Schwerter mit je einer Hand und zog sie zu ihrer Brust. Blut rann aus ihren Fäusten hervor. Ein zutiefst beunruhigendes Bild.

„Sei kein Dummkopf, Spatz“, sagte sie liebevoll, „Tritt nach vorne und gib mir deinen Arm.“

Jarid zögerte. Das Leben ihrer Mutter stand auf der Kippe. Aber sie erinnerte sich auch noch daran, wie es gewesen war, von diesem Bösen kontrolliert zu werden, auch nur für einen kurzen Augenblick. Diese Machtlosigkeit ... was auch immer der Plan des Bösen war, es konnte nichts Gutes heißen. Unentschlossenheit machte sich in ihr breit.

„Na kommt schon, Mädel. Ober bedeutet dir Rowinda etwa gar nichts?“, meinte die verbrannte Feuerkriegerin. Sie trat ungeduldig nach vorne und griff nach Jarid. Etwas klickte in Jarid. Sie packte den Arm ihrer Angreiferin, drehte ihn auf den Rücken und zwang die Feuerkriegerin in die Knie.

Der singende Klang eines Rankenklingenstichs ertönte. Ein Wasserschwall schoss an Jarid vorbei und schlug die Feuerkriegerin endgültig beiseite. Der Urheber des Wasserschwalls, ein großgewachsener blonder Wassermagier mit einer blauhäutigen Backe im ansonsten bleichen Gesicht, trat aus dem Nebel und packte Jarid am Arm. „Nichts wie weg von hier! Hier, stütz dich auf mich.“

Während Jarid davonstolperte, sah sie aus dem Augenwinkel, wie zwei Rankenschwerter im Körper ihrer Mutter steckten. Finster-Rowinda lächelte, überreichte den schwarzen Kristallsplitter an eine Feuerkriegerin und knallte prompt zu Boden, zusammengeklappt wie eine Puppe, deren Fäden durchgeschnitten worden waren. Ihre glasigen Augen starrten nichts sehend in die Ferne.

Tot.\bigskip







„Lass mich zurück und schlage Alarm!“, krächzte Jarid,

„Veila hat sich bereits auf den Weg gemacht. Sie holt Hilfe. Die Orden müssten jeden Moment hier sein“, erwiderte der Wassermagier, „Und wir sind Danware, wir lassen niemanden der unseren zurück.“

Außer ihr mögt sie nicht, dachte Jarid in Gedanken an Trieest. Stumm stolperte sie weiter, nur weg vom Becken von Quodlon.

„Lasst den grantigen Kord rufen. Diese Situation verlangt nach einem Seher.“ Auch wenn Kord als hochklassiger Seher im Voraus wusste, wo er von Hilfe sein würde, hatte er die faule Angewohnheit, nur auf persönliche Einladung aufzukreuzen.

Der Wassermagier wurde kaum merklich langsamer und blickte Jarid aus haselbraunen Augen an. „Kord ist schon über ein Jahr tot. Das Fieber hat ihn erwischt. Man sagt, in seinem Sterbebett habe er tatsächlich friedlich gelächelt.“

Jarid versuchte verzweifelt, weitere Gedanken an Unheil und Tod zu verdrängen. Ihre Mutter, Kord, diese Orden, sie hatten ihr alle nichts mehr zu bedeuten, ihr Leben lag nun außerhalb Danwars. Sie war eine Heldin von Andor, und als solche hatte sie sich um das Wohl der vielen zu kümmern. Viele, die in Gefahr sein würden, wenn das Böse Danwar verlassen konnte. Sie konnte kaum zur Verteidigung beitragen, aber vielleicht konnte sie mit den wenigen Informationen, die sie über das Böse besaß, von Hilfe sein. Wenn diese Informationen zur richtigen Person gelangen konnten.

„Wo ist der Hüter des Wissens?“

„Im jährlichen Koma des Bewahrens. In den nächsten Tagen nicht ansprechbar.“

„Was ist mit Feuermeisterin Nidwal?“

„Die Verschrobene? Lebt meines Wissens immer noch abseits des Dorfes in ihrer Steinkammer.“

Jarid brachte ein schwaches Lächeln zustande. Mutter Natur hatte sie nicht völlig im Stich gelassen.. „Bringt mich zu ihr.“\bigskip







Feuermeisterin Nidwals weißer Bart war wie üblich zu einer vollendeten Kugel frisiert und mit irgendwelchen Wachsen in Position gehalten, ein Kontrast zu ihrem vollkommen kahlen Schädel. Auch wenn Nidwal nun schon an die einhundert Sonnenumrundungen erlebt haben musste, war die alte Sternguckerin noch immer gut zu Fuß.

Sie war eine der vielen Feuerkriegerinnen, die ihren Lavastein nicht als Bürde für ein begangenes Vergehen in ihrer Brust tragen mussten, sondern ihren Lavastein als Auszeichnung für ihre abgeschlossene Ausbildung in einer maßangefertigten Rüstung tragen durften. Nidwal könnte ihren Lavastein mitsamt der durch ihn angetriebenen Rüstung problemlos ablegen, doch manche munkelten, dass sie ihre Rüstung selbst zum Schlafen anbehielt. Andere munkelten gar, dass sie ohnehin nie ganz schlafen würde, sondern immer höchstens ihren halben Körper in einen Ruhezustand bringen würde. Jarid selbst hatte sie schon mehrmals angetroffen, wie sie mit einem geschlossenen und einem offenen Auge an eine Wand lehnte und mit nur einer Hand irgendwelche Zeichen in eine Steintafel ritzte.

Gerade war Nidwal allerdings mit beiden Körperhälften wach und klopfte den Boden vor ihrer Hütte mit einem komplizierten Werkzeug ab. Es erinnerte an einen Hammer, bestand jedoch aus mehreren beweglichen Teilen und seine Spitze war mit Zahnrädern besetzt.

Nidwal blickte auf und lächelte höflich, als sie Jarid erblickte.

„Na, wen haben wir denn hier?“, rief sie laut. Vom Dach kam ein großer weißer Rabe herabgesegelt, der eine Runde um Jarid und den Wassermagier drehte, ehe er auf Nidwals Schulter landete und ihr etwas ins Ohr krächzte.

„Ah, Jarid! Blau sind die Weiten des Meeres“, rief Nidwal fröhlich, „Morgentau hast du dich bei der Zeremonie zum dritten Zirkel benannt, oder? Lange nicht mehr gesehen! Und du ...“

Nidwal blickte mit gerunzelter Stirn zu Jarids Begleiter hoch. Erneut krächzte der weiße Rabe auf ihrer Schulter ihr etwas ins Ohr. Nidwal nickte betrübt: „Deinen Namen kennen wir nicht, aber dich habe ich bestimmt schon mal gesehen, als du noch ein ganzes Stück kleiner warst als jetzt. Wie lautet dein Name?“

„Nicht relevant“, meinte der Wassermagier abweisend.

Nidwal reagierte gespielt empört: „Nicht relevant?! Mein Junge, Namen sind stets von Relevanz. Jeder Name hat eine Bedeutung, und jeder Name ist es wert, erfahren zu werden. Es wäre unhöflich von mir, nicht ...“

„Ja, schon“, unterbrach der Wassermagier den ihm entgegenströmenden Redeschwall, „Es ist nur ...“

Er gestikulierte ungelenk zu Jarid hinüber. Nidwals rot glühende Augen verengten sich leicht, als ihr Jarids blutgetränktes Kleid auffiel. Sie rappelte sich auf und klopfte sich schwarzen Staub vom Mantel, welcher unter ihrer Rüstung hervorragte. An der Art, wie sich ihr Kiefer dabei lautlos hin und her bewegte, erahnte Jarid, dass ihr Geist gerade auf Höchstleistung ratterte. Vermutlich ging sie gerade die besten Heilmittel in Reichweite durch. Und fragte sich, warum Jarid wegen einer Bauchwunde zu ihr kommen würde.

„Ein bösartiges Wesen ist in Danwar eingedrungen“, versuchte Jarid, die Geschehnisse so kurz und klar wie möglich zusammenzufassen, „Es lebt in einem schwarzen Kristallsplitter und übernimmt die Kontrolle über alle, zu denen es direkten Kontakt hat.“

„Hm“, machte Nidwal, den Blick weiterhin sorgenvoll auf Jarids Bauchwunde gerichtet.

„Es kann auch längere Zeit Kontrolle über einen Feuerkrieger ergreifen, falls es mit dessen Lavastein in Berührung kam.“

„Hm“, machte Nidwal erneut und langte gedankenverloren an ihren eigenen Lavastein.

„Es kann auch in den Erinnerungen der von ihm kontrollierten Personen umherforschen.“

„Hm“, machte Nidwal und kratzte sich am weißen Kugelbart.

„Es sucht nach dem Echo eines Gefallenen, das es beim Kontakt mit Trieests Lavastein hörte.“

Nidwal schwieg, abwartend, ob noch mehr kam. Als dem nicht so war, begann sie mehrere Fragen auf einmal: „Warum hört es nicht einfach ... hat es noch Kontakt mit ... wo ist dein ...“

Der Rabe krächzte laut etwas in Nidwals Ohr, das verdächtig nach „Trieest“ klang, und Nidwal beendete ihre Frage: „Wo ist Trieest?“

Jarids Stimme stockte und sie schüttelte ihren Kopf. Trieest war weit weg von hier, vielleicht schwer verletzt durch den Zweikampf mit diesem riesigen Skralhäuptling, vielleicht tot.

„Das Böse hat Kontrolle über mindestens vier Feuerkrieger ergriffen und Jarid über das Becken von Quodlon hierher holen lassen. Wir haben uns ihm gebeugt“, sprach der Wassermagier nun kleinlaut.

„Wo sind sie jetzt?“

„Wer weiß das schon? Auf der Suche nach Jarid, vermute ich. Wohl auf dem Weg ins Dorf.“

„Hm.“

„Das Böse ist halb wahnsinnig“, fand Jarid nun ihre Stimme wieder, „Es steckte lang in einem unförmigen Biest fest, welches sich nicht ausdrücken konnte. Dem wilden Hraak. Kurz nach dessen gewaltsamen Ende vernahm das Böse das Echo einer Stimme. Nun scheint es wie besessen auf der Suche danach zu sein. Doch gleichzeitig ergötzte es sich an seiner Kontrolle, am Schmerz, das es anrichten konnte, am ...“

Jarids Stimme schwankte wieder, doch dieses Mal fand sie sie rasch wieder: „Es hat Rowinda getötet, weil ich mich ihm nicht ergeben hatte.“

Der weiße Rabe musterte Jarid aus roten Augen und flüsterte Nidwal etwas längeres ins Ohr. Nidwals Blick wurde weich: „Oh Jarid, du musstest eine schwere Entscheidung treffen. Das ist nicht deine Schuld.“

„Ich habe keine Entscheidung getroffen. Ich habe nur zugelassen, dass er meiner Mutter etwas antut! Aber solche Gedanken nützen jetzt nichts.“

Die Feuermeisterin legte ihren Kopf schief: „Es ist nicht zwingend hilfreich, deine Schmerzen zu ...“

„Nicht jetzt! Wir müssen etwas unternehmen, ehe es die Lavasteine aller Feuerkrieger übernimmt und sie nur zum Spaß an der Freude in ihr Schwert laufen lässt!“, sprach Jarid etwas lauter, als nötig war.

Nidwald schüttelte ihren Kopf: „Jarid, du bist verletzt und am Ende deiner Kräfte. Und ich bin schon seit Jahren am Ende der meinen. Wir können kaum etwas unternehmen. Warum bist du zu mir gekommen?“

„Ihr seid Feuermeisterin Nidwal! Ihr habt mir schon so oft aus der Patsche geholfen. Ihr habt so viele Danware angeleitet. Ihr versteht das Leben und die Lebenden. Ihr müsst doch wissen, was sich gegen dieses Böse tun lässt!“

„Soll ich das Böse suchen und bekämpfen gehen, während ihr das hier löst?“, fragte der Wassermagier unruhig.

Nidwald zuckte mit der Schulter: „Das weiß ich wohl kaum besser als du. Aber dieses Böse klingt so, als wäre kein einzelner seinem Kollektiv im Kampf gewachsen. Es kostet dich kaum etwas, noch ein wenig länger hier zu bleiben.“

Der Wassermagier schaute sich hilflos um.

„Warum bist du zu mir gekommen, Jarid?“, fragte Nidwal erneut, noch etwas sanfter.

Und Jarid sprach: „Ich bin gekommen, weil ich etwas tun muss. Ich habe das Böse entkommen lassen, ich habe es hierher gelenkt, ich habe nicht darauf geachtet, dass es noch am Leben sein könnte. Ich bin dafür verantwortlich, dass es hier ist, und ich muss etwas tun, damit es keinen weiteren Schaden anrichtet.“

Nidwal blickte Jarid tief in die Augen und meinte traurig: „Oh, Jarid. Du hast einen analytischen Geist, der sich nur selten selbst überlistet. Was das Böse tut, ist seine Schuld, nicht die deine. Du musst nichts tun. Und du musst doch gewusst haben, dass ich nichts gegen dieses Böse ausrichten konnte, was zwei Dutzend Feuerkrieger im Dorf nicht besser könnten.“

Jarid schwieg, während ihr Geist raste. Sie erinnerte sich zurück an das kleine Mädchen, das sie einst gewesen war. Das zu Nidwal gerannt war, weil ihre Schultafeln in den Lavasee gefallen waren, als sie nicht auf ihre Tasche geachtet hatte. Dem Nidwal geholfen hatte, als es gezweifelt hatte, ob der Orden der Wassermagier wirklich sein Pfad sei. Das mit Nidwal Apfelkuchen gebacken hatte, um sich von der Beerdigung seines Vaters zu erholen.

Nidwal hatte stets die richtigen Worte gefunden, um sie zu beruhigen. Konnte es sein, dass sie gar nicht nach einem Mittel gegen das Böse suchte, sondern nur eines gegen ihren aufgewühlten Gemütszustand?

„Das wollte ich damit nicht sagen“, meinte Nidwal hastig, als sie Jarids Gesichtsausdruck sah, „Du hast dich daran gewöhnt, anderen zu folgen. Du überließt Trieest die Entscheidungen über Eure Abenteuer, um nichts über die Prophezeiung der Roten Grotte zu verraten. Nun wendest du dich an mich, in der Hoffnung, dich leiten zu können. Danke für dein Vertrauen in mich. Aber du hast mich schon seit Klein auf ein Podest gestellt, dem ich nicht gewachsen bin. Ich bin nur eine alte verrückte Feuermeisterin, die sich gerade ernsthaft überlegt, ob sie wegen dieses Bösen von Danwar abhauen soll.“

„Ich werde nicht abhauen“, meinte der junge Wassermagier nun stolz, „Ich werde kämpfen bis zum letzten Schweißtropfen, wenn es sein muss. Wir sind Danware, wir lassen unsere Heimat nicht im Stich!“

Nidwal lächelte nur traurig. Sie haderte offenbar damit, ihre Heimat im Stich zu lassen.

„Meisterin Nidwal!“, sprach Jarid aufgebracht, „Wie könnt Ihr nun einfach aufgeben wollen? Ihr wusstet immer die richtigen Worte, um so viele Danware auf glückliche Pfade zu lenken. Ihr müsst doch auch die Worte kennen, die auch Euch selbst auf den richtigen Pfad leiten.“

„Ach, Jarid, Kommunikation ist selten so einfach. Es mag die richtigen Worte geben, die dich und mich wieder mit Hoffnung erfüllen würden. Es mag auch die richtige Abfolge von Worten geben, die dieses böse Wesen von seinen Taten abbrachte und seine Kraft auf konstruktivere Projekte leitete. Ihm die Fehler seiner Logik aufzeigten.“ Nidwal gluckste. „Bei der Anzahl Gesprächen, die ich in meinem langen Leben bereits geführt habe, habe ich die benötigten Sätze vermutlich alle schon einmal so ähnlich gesprochen, vielleicht gar in der richtigen Reihenfolge. Aber die richtigen Worte zu kennen, ist eine Kunst, die man nur durch passendes Verständnis des Gegenübers kriegen kann. Und ich habe keine Ahnung, wie dieses Wesen denkt. Wo fängt man bei einem wahnsinnigen Edelstein überhaupt an?“

Der blonde Wassermagier horchte auf: „Jarid, sagtet Ihr, dass diese Böse die Erinnerungen von all denen liest, die es übernimmt?“

„Ich habe es selbst gespürt.“

„Und Meisterin Nidwal, stimmt es, dass Ihr viele Jahrzehnte lang den unterschiedlichsten Danwaren als Seelsorger und Ratgeber zur Verfügung standet? Dass Ihr auf Jahrzehnte der Weisheit und guten Ratschläge zurückblicken könnt?“

„Nein!“, rief Jarid. Auch der weiße Rabe auf Nidwals Schulter krächzte protestierend auf, wobei sein Vogelgesicht natürlich keine Regung zeigen konnte. Nidwals Gesicht hingegen durchlief rasch viele Emotionen, von Neugierde zu Überraschung, zu Freude, zu Sorge bis hin zu schlussendlich kalter Entschlossenheit. Sie zupfte an ihrem weißen Kugelbart.

„Lasst uns das Böse suchen.“

Jarid schüttelte ihren Kopf. „Bitte, Meisterin Nidwal, gebt nicht auch noch Ihr Euch ihm hin.“

Nidwal breitete theatralisch ihre Arme aus. „Personen sind darauf trainiert, Widersprüche in ihrem Denken zu erkennen und auszumerzen. Wenn dieses Böse seinen Geist mit denen übernommener Körper teilt, muss es zwar deren Meinungen und Gedanken gut von den seinen trennen können, sonst hätte es sich schon längst in ihnen verloren. Und doch sollte es in der Lage sein, meine Werte und Erfahrungen zu sehen und davon zu profitieren, schneller, als es eine Konversation je könnte. Es ist einen Versuch wert. Und zum Davonlaufen bin ich eigentlich ohnehin zu alt. Ich muss mich von diesem Bösen kontrollieren lassen, um es davon zu überzeugen, dass das Leid, das es anrichtet, nicht sinnvoll ist.“\bigskip







„Was siehst du?“, fragte Jarid den großen Wassermagier zum dritten Mal. Er reichte ihr wortlos das Fernrohr. Durch es hindurch sah Jarid den entfernten Dorfplatz. Pflastersteine um einen dampfenden Brunnen. Triste Steinhäuser vor kargen Gärten. Die Danwarische Architektur hatte sie nicht vermisst.

„Meisterin Nidwal ist erst vor ...“, er blickte auf ein kompliziertes mechanisches Gerät an der Wand von Nidwals Hütte, das verschiedene bewegliche Kugeln vor einem Sternenhimmel-Hintergrund zeigte, „Sieben? Sie ist vor sieben Minuten aufgebrochen? Selbst auf ihrem Schlitten wird es länger dauern, bis sie das Dorf erreicht. Falls das Böse überhaupt dort ist.

„Wo sonst sollte es sein?“, fragte Jarid harsch, ehe sie sich auf ihre Manieren besann, sich entschuldigte und dem Wassermagier höflich das Fernrohr zurückreichte.

Der Wassermagier reichte ihr ein rotbraunes Döschen. „Ich glaube, das ist Goldsalbe? Habe ich neben dem Fernrohr in Meister Nidwals Ramschschachtel gefunden. Ist nicht nur für Feuerkrieger gut.“

Jarid öffnete das Döschen und erblickte die erwartete golden glänzende Creme. Sofort entfaltete sich in ihrer Nase der altbekannte Geruch nach Feuer und Metall. Zwischen spitzen Fingern zerrieb sie einen kleinen Teil der Paste und ließ das entstehende Puder über ihre Bauchwunde rieseln. Augenblicklich verebbte der Schmerz.

„Danke dir“, meinte Jarid seufzend. Sie blickte den Wassermagier an und fragte: „Jetzt aber bin ich doch gespannt: Wie lautet dein Name eigentlich?“

Er setzte zu einer Antwort an, unterbrach sich aber und meinte stattdessen: „Meisterin Nidwal ist schnell wie der Blitz! Sie ist bereits am Dorfplatz angekommen. Sie ist von ihrem Schlitten gestiegen und hat ihre Arme ausgebreitet. Sie ruft etwas.“

„Lass mich das Fernrohr sehen, ich habe mal ein wenig Lippen lesen gelernt“, rief Jarid. Dann fiel ihr ein: „Oder warte, hast du in Meisterin Nidwals Ramschschachtel zufälligerweise ein Gerät gesehen, das wie ein langes dünnes Rohr mit einem eisernen Ohr am Ende aussieht?“

Der Wassermagier nickte, und huschte bereits wieder ins Innere der Hütte. Als er zurückkehrte, hatte er das gewünschte Gerät bereits bei sich.

„Sie zeigte es mir zur Aufmunterung, als meine Base einmal ...“, fing Jarid an, ehe sie sich fing, „Nun, das ist eine Geschichte für ein anderes Mal. Schau, du kannst es hier und hier an das Fernrohr anschrauben, dann hier diese Öffnung schließen und dann das Fernrohr auf die Szene auf dem Dorfplatz richten.“

„Komm her, du finsterer Wicht. Ich ergebe mich dir!“, erschall Meisterin Nidwals krächzende Stimme aus der Metallkonstruktion. Ein wenig blecherner als in Realität, und mit allerlei Hintergrundrauschen, aber immer noch verständlich.

Der Wassermagier war zusammengezuckt und blickte mit großen Augen das Gerät an. Jarid lachte bloß:

„Meisterin Nidwal ist wirklich stolz auf diese Erfindung. Sie nennt es ihr Fernohr.“

Der Wassermagier stimmte kurz in ihr glockenhelles Lachen ein.

„Der Erste Feuerschmied Melek und die thermische Thermosta, diese fröhliche Tüftlerin aus dem rauchenden Tal, boten Meisterin Nidwal angeblich bereits 5 Goldstücke, 73 Silberstücke und 27 Lavadeute für detaillierte Pläne vom Fernohr an. Doch soweit ich weiß, lehnte Nidwal immer ab. Ein solches Meisterwerk soll es nur einmal geben, meinte sie. Böse Zungen – insbesondere die des grantigen Kords – behaupteten, dass sie selbst nicht mehr wisse, wie man ein solches Fernohr ... “

„Wer bist du, der du mich rufst, statt dich mir zu widersetzen versuchen?“, erschall eine keuchende Stimme aus dem Fernohr, die definitiv nicht zu Nidwal gehörte. Sofort erstarb das Gelächter der beiden Wassermagier.

„Die kleine Jivin spricht, die vorhin schon einmal durch das Böse gesteuert wurde“, teilte der blonde Wassermagier nach einem Blick durch das Fernrohr mit. „Das Böse muss sie wieder erwischt haben.“

„Dann hat sie den schwarzen Kristallsplitter“, kombinierte Jarid, „Das Böse konnte bislang nur einen Wassermagier auf einmal kontrollieren, darum hat es mich auch nicht selbst nach Danwar geholt.“

„Warum fragst du danach, wer ich sei, wenn du es ohnehin gleich wissen wirst?“, erklang Nidwal sanfte Stimme aus dem Fernohr.

Das Böse klang fast trotzig. „Was auch immer dein Plan ist: Du kannst mich nicht aufhalten. Ich werde triumphierend über die gesamte Welt herrschen. Das Schicksal will es so.“

„Dann sei es so“, meinte Nidwal schicksalsergeben.

Der kleine Wassermagier kommentierte weiter: „Jivin ist auf Nidwal zugetreten. Sie greift nach ihrem Lavastein!“

Jarid schloss ihre Augen und sandte ein weiteres Stoßgebet an Mutter Natur. Sie hätte Meisterin Nidwal nie davonziehen lassen dürfen. Oder wenigstens den Wassermagier mitnehmen lassen.

„Weitere Feuerkrieger sind um die beiden herum aufgetaucht. Allesamt mit schwarzen Lavasteinen, soweit ich das erkennen kann. Meisterin Nidwals Lavastein wird dunkler. Jetzt ist er völlig schwarz. Das Böse hat Nidwal!“

Jarid musste sich sehr zusammenreißen, um dem jungen Wassermagier das Fernrohr zu lassen.

„Was tut es jetzt?“, wollte sie wissen.

„Nichts Besonderes. Es steht einfach nur da. Sie stehen alle einfach nur da.“

Nichts als Rauschen war aus dem Fernohr zu vernehmen. Dann ein dumpfes Plumpsen.

„Sie sind alle auf die Knie gefallen!“, rief der Wassermagier aufgeregt, „Jivin hält sich sogar ihren Kopf. Was auch immer das Böse in Meisterin Nidwals Erinnerungen erleben, es hat es erreicht.“

„Nicht zu fassen“, hauchte Jarid.

„Jetzt stehen sie wieder auf! Sie laufen im Gleichschritt aus dem Dorf weg.“

„Wohin?“

„Schwer zu sagen. Sie fliehen nicht, aber sie nehmen es auch nicht gemütlich. Sie folgen dem Weg ... zu den Flaschenzügen! Zu den Lastenzügen am Kopf! Sie wollen zum Hafen im Bassin!“

„Wollen sie die Insel verlassen? Was tun wir nun?“

„Was können wir überhaupt tun? Beobachten und den richtigen Leuten berichten gehen.“

„Wollen wir sie nicht aufhalten?

„Mit Verlaub, Jarid, aber Ihr seid nicht unschwer verletzt. Mit der Geschwindigkeit der Armee des Bösen können wir nicht mithalten. Lasst uns lieber den restlichen Orden informieren.“\bigskip







In den nächsten Minuten tauschte der junge Wassermagier einige Brieflilien mit einer ranghöheren Wassermagierin im Inselinnern aus. Vermutlich berichtete er von den Vorfällen im Dorf und wurde darüber informiert, dass der Orden die Lage bereits im Blick hatte, schließlich waren mehrere Wassermagier dem Scharmützel beim Becken von Quodlon entkommen.

Jarid befürchtete, dass der Orden wenig ausrichten konnte. Weitere Feuerkrieger würden nicht mehr in die Nähe des Bösen gelassen und die Wassermagier würden kaum mehr Widerstand leisten, nachdem das Böse mit den Leben der bereits unter seiner Kontrolle stehenden gedroht hatte. Der Orden hatte seit Jahrzehnten keine Landgefechte mehr durchgeführt, und erst recht nicht gegen einen Gegner, der die eigenen Taktiken in- und auswendig kannte.

Niemand wusste, was es vorhatte oder warum. Sie wussten bloß, oder glaubten zu wissen, dass ein Vorgehen gegen das Böse den raschen Tod aller Besessenen zur Folge haben konnte. Und niemand wollte das riskieren, solange das Böse keine offensichtliche Gefahr mehr darstellte und augenscheinlich von Danwar abzog.

Während der junge Wassermagier also relativ ratlos mit einigen anderen Ordensmitgliedern kommunizierte, blickte Jarid rastlos durch das Fernrohr auf den Abzug des Bösen.

Es war gruselig, zu sehen, wie die mindestens zehn Feuerkrieger in absoluter Synchronie und mit vollkommen gleichgültigen Gesichtsausdrücken durch die felsige Landschaft Danwars zogen. Inzwischen sah Jarid nur noch ihre Hinterköpfe. Einholen würde sie sie nicht mehr.\bigskip







Unschöne Erinnerungen kamen über Jarid, als sie zum ersten Mal seit sieben Jahren vor den versammelten Ältestenrat trat.

Der Rat sah die Lage leider ähnlich fatalistisch wie der junge Wassermagier. Einige Mitglieder schienen gar erfreut über die Nachricht, dass das Böse drei Schiffe eingenommen hatte und in den Süden losgesegelt war. Nach Andor. Ihr Entschluss schien felsenfest: Der Rat würde keine Unterstützung nach Andor senden.

Das war schon schade. Wassermagier könnten die Ankunft der Schiffe des Bösen an der andorischen Küste verhindern, Feuerkrieger könnten mit ihren vielseitigsten Gaben, von schwelenden Feuerkugeln über Heilende Flammen bis hin zu ganzen Feuerwänden, stark aushelfen. Doch die beiden Orden würden tatenlos auf Danwar verbleiben. „Um der dringlicheren Bedrohung durch Feuergeister, Magmabestien, Ascheskelette und Strumtrolle Herr zu werden.“

Pah! Jarid hatte in ihrem zugegebenermaßen kurzen Aufenthalt hier in Danwar nicht einmal die Fußspuren einer Kreatur erblickt. Sie wusste um die typische Untätigkeit der Orden. Ein wenig enttäuscht war sie dennoch über die ausgebliebende Hilfeleistung. Insbesondere, als sich auch der junge blonde Wassermagier von ihr abwandte und etwas von der Priorität der Sicherheit Danwars nuschelte.

Jarid blickte ihn kopfschüttelnd an: „Du auch?“

Er blickte ihr entgegen und meinte etwas fester: „Wir sind Danware. Wir sorgen uns um die Sicherheit Danwars. Das muss unser Hauptfokus bleiben. Der Verlust dieser Feuerkrieger wiegt natürlich schwer, aber wir machen die Lage nicht besser, wenn wir ihnen hinterhersegeln und womöglich noch mehr verlieren. Andor hat seine Helden. Wir haben nur uns.“

„Noch sind unsere Feuerkrieger nicht verloren! Wenn wir das Böse, diesen Kristallsplitter, vernichten, können sie wieder frei sein. Unsere Wasserspäher haben berichtet, dass das Böse seine Gesellschaft den Wachsamen Wald ansteuern lässt. Wir müssen die Bewahrer warnen!“

„Wo waren die Bewahrer, als die varatanische Flotte Kurs auf Danwars Lavasteine nahm? Sie scherten sich keinen Lavadeut um die Stabilität unserer Insel und wir kümmern uns nicht um die Sicherheit ihres großen hohlen Baumes. Sie wollten das. Jetzt ernten sie halt, was ihre Vorfahren einst säten.“

„Dann denke halt nur an Danwar, aber denke daran! Dieses Böse ist wie eine Plage, welches jede Person, jedes Wesen, das es will, übernehmen kann! Was, wenn es einen König befällt? Den Urtroll? Die Mächte des Meeres höchstpersönlich? Was, wenn es die Ketten seines Gefängnisses sprengt und alle Wesen übernimmt, die es finden kann? Was, wenn es sich danach an Danwar erinnert und die Insel im Meer versenkt?“

„Du befindest dich in einer Angstspirale. Was, wenn es einen Fehler macht und in wenigen Tagen bereits Geschichte ist?“

Jarid schluckte eine frustrierte Antwort herunter und blickte den Magier ein letztes Mal flehend an. Als dessen Reaktion nur darin bestand, noch ein klein wenig beschämter zu Boden zu blicken, wandte sich Jarid energisch ab und dem versammelten Rat der Wassermagier zu. Vom niedrigsten Novizen bis zum höchstrangigen Absolventen waren verschiedenste Menschen vertreten. Gar ein Mitglied des sechsten Zirkels war anwesend. Doch auch sie alle schüttelten bloß ihre Köpfe. Danwar würde kein Kontingent aussenden, um die Gefahr durch das Böse zu dämmen. Höchstens eine Botschaft der Warnung an den Orden der Bewahrer wurde in Betracht gezogen.

„Vergesst es“, murmelte Jarid, „Das mache ich selbst.“

Immerhin war der Ältestenrat so überaus großzügig, sieben Mitglieder zum Becken von Quodlon zu beordern, auf dass man Jarid von dort aus sicher und schnell über den Ozean zum Brunnen neben dem Baum der Lieder senden könnte. Dort wären doch bestimmt einige Helden von Andor, die sich um dieses Problem kümmern konnten.

„Ich bin auch eine Heldin von Andor!“, zischte Jarid, „Und wir sind keine Götter! Wir können jede Unterstützung brauchen, die ihr erübrigen könnt, ja, sind auf diese angewiesen.“

„Und Danwar ist auf jeden einzelnen Feuerkrieger und Wassermagier angewiesen, sollte dieses Böse je hierher zurückkommen“, meinte die alte Freiga. Sie hatte nach Rowindas Tod ihren Platz als Ratsälteste eingenommen und blickte höchst selbstgefällig auf Jarid hinunter: „Inzwischen sind wir über seine Kräfte vorgewarnt, doch das heißt nicht, dass wir Danwar nicht weiter beschützen werden!“

Jarid entgegnete nichts.

„Bist du sicher, dass du mit deiner Wegreise nicht einmal bis zu Rowindas Beerdigung warten willst? Deine Helden und dein Trieest kommt bestimmt einige Tage ohne dich aus.“

Jarids Herz verkrampfte sich erneut. Ihre Mutter war tot, und Jarid würde nicht einmal ihre Bestattungszeremonie abwarten. Das würde Mutter Natur verstehen, machte sich Jarid klar. Und was die Priester des Flammenden Gottes und die restlichen Danware sagen würden, konnte ihr auch egal sein. Nichtsdestotrotz wässerten ihre Augen wieder einmal. Jarid ballte ihre Fäuste und drängte die aufschwellenden Tränen wieder in ihre Drüsen. Jetzt war nicht der Moment dafür.

Die Erwähnung von Tri machte ihr auch zu schaffen. Dieser Skralhäuptling mit der Eisernen Maske hatte ihn sicherlich schlimm verletzt. Jarid war allein. Sie würde sich allein durchschlagen müssen. Der Gedanke beunruhigte sie zutiefst, und so schob sie ihn zur Seite. Später. Jetzt hieß es erst einmal, sich in Andor auf die Ankunft der Schiffe des Bösen vorzubereiten.

Jarid konnte sich durch das Ende der Versammlung zusammenreißen.

Zum zweiten Mal in sieben Jahren verließ sie Danwar wütend nach einem vergeblichen Gespräch mit dem Ältestenrat. Im Gegensatz zu damals war kein Trieest mehr an ihrer Seite, und keine Mutter winkte ihr beim Abschied zu. Kurz hatte sie noch überlegt, Trieests Mutter Talemma aufzusuchen. Doch wollte sie keine alten Wunden aufreißen. Wenn das alles durch wäre, hatte sie noch genug Zeit, ihr die womöglich traurigen Neuigkeiten über Trieest zu berichten.

Mit der Unterstützung sieben unkorrumpierter Wassermagier überstand Jarid die Wasserreise durch das Becken von Quodlon zurück nach Andor unbeschadet.

Kaum war sie zurück in Andor aus dem nun dampfenden Brunnen nahe des Baums der Lieder gestiegen, fiel sie von Schluchzern gerüttelt zusammen.\bigskip







Bewahrer Tion rollte Jarid geschwind entgegen. Sein Gefährt, ein sogenannter Fahrstuhl – ein eleganter Stuhl mit zwei großen hölzernen Wagenrädern auf den Seiten – war Jarid bereits bekannt, und so beachtete sie es kaum. Stattdessen richtete sie ihre Aufmerksamkeit auf den verwahrlosten Anblick, den Tion bot. Sein hageres Haupthaar und sein langer Bart waren ungekämmt, sein Priesterkleid eiligst über sein Nachtgewand gestülpt. Er machte sich gar nicht erst die Mühe, sein Gähnen zu unterdrücken, und sprach müde:

„Unser oberster Priester Melkart ist in den östlichen Landen unterwegs. Verhandlungen mit der Einhorn-Sippe des Yetohe-Stammes. Ihr werdet mit mir vorstellig werden müssen.“

Jarid brachte einen höflichen Knicks und ein nur leicht gezwungenes Lächeln zustande. Dann weihte sie den Hohepriester der Mutter möglichst rasch ein in die Geschehnisse der vergangenen Tage. Meister Tion konnte ein ausgezeichneter Zuhörer sein, und seine anfängliche Müdigkeit verging wie im Flug, sobald Jarid den finsteren Kristall erwähnte, und erst recht, als Jarid auf die Gefahr der drei anrückenden Schiffe des Bösen hinwies.

Tion zückte eine kleine Glocke aus seinem Amtsgewand und schüttelte sie. Noch bevor ihr heller Schall verklungen war, rannte ein Novize an Tions Seite und beugte sich zu ihm nieder. Auch dessen Haar war verstrubbelt und ungemacht, doch die Augen glühten vor Eifer.

„Ihr wünscht, Meister Tion?“

„Lass die Teleskope an der Aussichtsplattform ganz oben am Baum besetzen, Komu. Haltet das Hadrische Meer im Auge. Wir erwarten ungewünschten Besuch. Drei Schiffe, die dir Jarid hier bestimmt gerne beschreiben wird.“

Novize Komu nickte und raste zurück ins Innere des Baumes der Lieder, kehrte nach einigen Augenblicken ebenso hastig wieder zurück und verschnaufte gerade lange genug, um Jarid nach einer Beschreibung der danwarischen Feuerschiffe zu fragen. Komu wiederholte Jarids Erklärungen zur Überprüfung Wort für Wort und hastete dann ebenso rasch wieder davon.

„Was füttert ihr euren Novizen? Zappelbohnen?“, fragte Jarid verschmitzt, „Wenn alle Anwärter nur halb so eifrig wären ...“

Tion schmunzelte nur und bat Jarid dann, ihn ans andere Ende dieses Balkons um den Baum der Lieder zu begleiten. Während Jarid Tions Fahrstuhl vor sich hin schob, murmelte Tion bedacht:

„Gedächtnisse sind eine eigenartige Sache. Ich selbst war mir lange absolut sicher über mein Wissen zu meiner Vergangenheit. Und doch stolperte ich kürzlich über eine alte Kiste verworfener Notizen meiner Selbst, aus einer Zeit, wo ich noch kein Hohepriester war. Ich weiß, dass diese Texte von mir geschrieben werden mussten, und doch kommt mir so vieles von damals fremd vor. Das Faszinierendste war eine Liste voller Stichwörter, die mein vergangenes Ich mit erinnerungswürdigen Momenten verband, größtenteils fröhlichen oder lustigen. Über die Hälfte dieser Begriffe sagte mir überhaupt nichts mehr. Und es ist erst einige Jahre her ... ich schweife ab, also lasst mich direkt sein: Jarid, seid Ihr Euch sicher, dass Euer Gedächtnis an diese Ereignisse noch Eures ist? Eine Seuche von Gedankenschindern ... nicht auszudenken, was geschähe, wenn der Baum der Lieder kompromittiert wurde.“

„Ich kann mir nicht sicher sein“, sprach Jarid, „Niemand kann das. Doch habe ich definitiv nicht im Sinne dieses Bösen gehandelt.“

„Könnt Ihr euch da sicher sein?“

„Sicherer als Ihr.“

„Dem ist wohl wahr.“

Tion hielt inne und holte Luft: „Last mich eine Geschichte erzählen. Das tu ich gerne.“

Jarid blickte ihn erwartungsvoll an. Tion räusperte sich und sprach salbungsvoll: „Eines Tages, als ich noch weniger graue Haare auf dem Kopf und mehr Muskelmasse an meinen Beinen hatte, schob ich gerade Wache am Rande des grünen Radius, als ein verwundeter Mensch auf mich zutrat. Er sah übel aus, Schnittwunden am Gesicht, eine Schnatte am Bein, keinerlei Besitztümer außer ein halb zerrissener Umhang ... dieser Mann erzählte mir eine haarsträubende Geschichte von verlorenen Schätzen und anstehenden Geburten, warum er dringend in den Norden reisen müsste. Ich verschaffte ihm einen Platz auf einem Handelsschiff. Als ich meinen Oberen vom Vorfall berichtete, war das Schiff schon lange abgefahren. Von Zeit zu Zeit erinnere ich mich ihn und frage mich, wie es ihm wohl geht.“

„Wisst Ihr, ob seine haarsträubende Geschichte stimmte?“

„Sie war ganz dreist gelogen. Beim Menschen handelte es sich um einen Soldaten in den Diensten des Königs, der beim Stehlen entdeckt und von der Burg verbannt worden war. In jener Zeit kam die Verbannung von der Burg einem Todesurteil gleich, doch durch seine kämpferischen Fähigkeiten hatte sich dieser geschickte Schwertkämpfer zu uns durchgeschlagen. Nun, nicht nur durch seine kämpferischen Fähigkeiten, sondern auch durch seine Honigzunge. Dieser Lügner. Als ich dies erfuhr, ging ein wenig von meinem Vertrauen verloren. Und das gewann ich nie wieder.“

Tion blickte Jarid vielsagend argwöhnisch an. Sie wurde immer unwohler.

„Ihr schweiftet wieder ab, also lasst mich nun direkt sein: Ihr könnte mir vertrauen, Meister Tion. Ich versuche nicht, Euch zu hintergehen. Ich bin hier, um Euch zu warnen.“

„Ihr wärt nicht die erste, die einen falschen Alarm auszulösen versucht, damit sich ein Komplize in die Schwarzen Archive schleichen kann“, meinte Tion. Dann aber blickte er mit einem ernsten Gesichtsausdruck zu Jarid hoch und sprach feierlich: „Doch vertraue ich auf meine Menschenkenntnis, und die vertraut auf Euch. Wir werden Ausschau halten nach diesen feindlichen Kriegern. Erst, wenn sie ausbleiben, werden wir Euch vor den Rat der Bewahrer stellen.“

Jarid schluckte schwer.

Gemeinsam blickten die beiden ins Dunkel der Nacht, welches natürlich wie immer nicht völlig dunkel war. Der rote Mond und die weißen Sterne erleuchteten den Wachsamen Wald. Das Sternbild des Hornfalken schwebte da oben, wenn sie sich nicht irrte. Ein gutes Omen. Das mochte Jarid am andorischen Himmel. Über Danwar hingen oft dunkle Dampfwolken, die es unmöglich machten, in den dunkelblauen Kosmos zu spähen. Nur selten konnten die Danware das Firmament erkennen. Erst im Süden war Jarid klar geworden, warum die Bewahrer so viel mehr Sterngucker und Sternzeichenleser hervorgebracht hatten.

Jarids ausschweifende Gedanken, die sich erfreulicherweise einmal nicht um Tri und Rowinda gedreht hatten, kamen zurück ins Hier und Jetzt, an den Baum der Lieder an der Seite des Tion, als ein schneeweißer Rabe angeflattert kam und sich vor ihr auf der Brüstung niederließ.

Jarid hatte diesen Raben erst kürzlich gesehen, auf der Schulter von Feuermeisterin Nidwal, ehe selbige sich dem Bösen ergeben hatte. Der Rabe krächzte aufgeregt und hüpfte auf und ab.

„Versteht Ihr ihn?“, fragte Tion, seine buschigen Augenbrauen erhoben.

Jarid schüttelte ihren Kopf. „Raben zu verstehen, ist ein Talent, das nicht jedem gegeben ist. Aber mir scheint, sie will uns etwas mitteilen.“

„Ist die Armee des Bösen vielleicht bereits hier?“

Jarid spähte ins Dunkel und wünschte sich die leuchtenden Dunkelaugen der Feuerkrieger.

„Nein, das kann nicht sein. Falls sie sehr schnell unterwegs waren, könnten sie vielleicht inzwischen das Ufer erreicht haben, aber das hätten Komu und die anderen an den Teleskopen bestimmt gesehen.“

„Prüft das nach!“, rief Tion und deutete auf die Wendeltreppe im Innern des Baums der Lieder. Jarid sauste davon. Hinter sich hörte sie Tions Fahrstuhl knarzen, als er ihr folgte, gefolgt von einem Bimmeln seiner kleinen Glocke.

Komu war wach und ganz aufgeregt im Ausguck ganz oben im Baum der Lieder. Die zwei anderen Wache haltenden Bewahrer bestätigten, erheblich weniger enthusiastisch, dass keine Schiffe erblickt worden waren. Auch wenn sie wegen der Dunkelheit und den dichten Nebelschwaden über dem Hadrischen Meer zugegebenermaßen nicht sonderlich weit blicken konnten.

Begeistert versuchte Komu, Jarid noch zu erklären, wie das überlange Fernrohr am Aussichtspunkt funktionierte. Es klang wirklich faszinierend. Jarid nahm sich vor, zurückzukehren, sobald die Lage hier geklärt war. Jetzt gab es leider Wichtigeres. Jarid sauste die Wendeltreppe wieder hinunter und berichtete bei Tion, welcher inzwischen von einer weiteren Novizin eine Rampe hochgeschoben wurde. Nidwals weißer Rabe war immer noch dort, flatterte erneut aufgeregt mit den Flügeln und ließ sich dann von der Brüstung fallen.

„Vielleicht will sie, dass ich ihr folge!“

Tion legte seinen Kopf schief und kratzte sich am Bart. Dann sprach er entschlossen: „Dann tut das, Jarid Morgentau. Doch gebt acht. Mit weisen Raben ist nicht zu spaßen.“

Jarid nickte ihm zu eilte die gewundene Treppe hinunter bis aufs Bodenlevel. Sie sauste durch den Torbogen dem Baum der Lieder hinaus. Da flatterte der weiße Rabe ungelenk in der Luft herum. Just bevor Jarid ihn erreicht hätte, raste er ein bisschen tiefer in den Wald hinein.

Jarids Bauch meckerte und ihre alte Wunde platzte wieder auf, doch Jarid rannte weiter, stets knapp auf der Spur von Nidwals weißem Raben. Ihr Bauch mochte protestieren, doch ihr Bauchgefühl versprach ihr, dass, was auch immer der Rabe ihr zeigen wollte, wichtig war. Es sollte recht behalten.

Zunächst rannte Jarid in zwei Skrale hinein, die ihr prompt zwei verrostete Klingen an die Kehle hielten, und Jarid verfluchte ihre untypische Unbedachtheit. Dann aber erklang ein kehliger Befehl aus dem tieferen Dunkel. Die zwei Skrale senkten ihre Waffen und traten zurück.

Jarid riss mit Mühe etwas Wasser aus dem Matsch zu ihren Füßen und formte es zu einem spitzen Eiszapfen in ihren Händen. Wirklich Schaden anrichten würde dieser nicht können, aber vielleicht reichte es, um die Skrale einzuschüchtern.

Über ihr schrie der weiße Rabe warnend auf.

Dann zeigte sich jemand vorsichtig aus dem Schatten zwischen zwei Mammutbäumen.

Zunächst erkannte sie ihn nicht, wirkte er bloß wie einer der übrigen Skrale. Breit, grobschlächtig, leuchtend weiße Augen. Dann aber erkannte sie ihn. Dieses Kinn. Die Art, wie er mitten in einem Schritt vorsichtig stehen blieb. Das Rankenschwert in seiner Hand. Die Rüstung an seinem Körper. Und die Kuhle in seiner Brust, wo früher ein Lavastein gesessen hatte. Ein riesiger Stein fiel von ihrem eigenen Herzen.

„Tri!“, rief sie fröhlich. „Du lebst! Du bist deine Bürde losgeworden! Komm, lass dich ansehen!“

Trieest verblieb im Schatten, von seinem Gesicht neben der Silhouette immer noch nur die nun weiß leuchtenden Kreaturenaugen erkennbar.

„Bei den roten Wogen des Feuers, Tri, ich bin’s! Mutter Natur sei Dank, dir geht es gut! Und du bist hier! Wir können dich dringend benötigen hier. Dich und ... deine neue Skral-Sippe?“, schlussfolgerte sie rasch, „Die restlichen Helden sind alle bei der Gedenkfeier an der Rietburg. Ihr seid unsere letzte Hoffnung.“

„Iarr ... Jarid“, sprach Trieest krächzend, als würden ihm die Silben im Hals stecken bleiben.

Endlich, endlich trat er aus dem Schatten der Bäume ins rötliche Mondlicht. Jarid verspürte gemischte Gefühle bei seinem Anblick. Sie hatte so oft Kreaturen mit Bosheit und Leid in Verbindung gebracht, und so war es durchaus beunruhigend, so viele Züge von Skralen in ihm zu erkennen. Die spitzen Zähne und das schüttere Haar waren alles andere als gewohnt. Doch dahinter steckte unweigerlich immer noch der Tri, den sie kannte und liebte. Der Tri, der nie ihre Seite verlassen würde und ihr schon so oft das Leben gerettet hatte. Der Tri, der genau wusste, welche Worte ihr helfen konnten, wenn sie wütend war, und der gerade nicht wusste, wie er ihr helfen sollte, wenn sie on Trauer erfüllt war. Der Tri, der ihr seit nun sieben Geburtstagen je eine besondere Feder geschenkt hatte, weil er wusste, dass sie sie gerne studierte. Der Tri, dem sie sieben Jahre falsche Hoffnungen gemacht hatte, was seinen Prozess des Wandels anging. Der Tri, den sie sieben Jahre angelogen hatte.

„Du sagtest, du wüsstest, wo der Lavastein mich zu einem Menschen machen würde. Du musst gewusst haben, dass der Lavastein mich nie zu einem Menschen machen würde“, sagte Trieest als Erstes. Keine Frage, aber auch keine Anschuldigung. Eine Feststellung. Vielleicht mit einem Unterton von Verletztseins, aber auch einfach Unverständnis. „Du musst gewusst haben, dass all mein Leiden umsonst war.“

Ja, er war verletzt. Jarid spürte ihre Augen wässerig werden, und beförderte ihre Tränen rasch zurück in die Drüsen, aus denen sie fließen wollten. Jetzt war nicht die Zeit dafür.

„Bist du sicher, dass du das jetzt besprechen willst?“, fragte sie vorsichtig. Tri schien mit den passenden Worten zu kämpfen und antwortete nicht.

„Ich ... ich bin froh, dass du den Lavastein loswurdest“, quetschte Jarid hervor, „Und es tut mir leid. Es tut mir alles so leid. Aber eine Armee von finsteren Feuerkriegern ist auf dem Weg hierher. Ich würde das gerne unsere Priorität machen.“

Tri schnaufte tief durch und nickte dann. Jarid öffnete ihre Arme breit und Tri erwiderte die Umarmung. Fester als gewöhnlich, aber Jarid störte sich nicht daran. Seine Haut fühlte sich härter und rissiger an als zuvor. Das war in Ordnung. Alles war in Ordnung. Tri lebte. Sie hatten einander wieder. Jetzt mussten sie nur noch gegen die Armee des Bösen bestehen. Erneut wallten Tränen in ihr auf und sie bändigte sie hinunter, ehe sie fallen konnten.\bigskip







\textit{Das Böse sinnierte vor sich hin, während seine Schiffe dem Wachsamen Wald näher kamen. Am Horizont sah es hohe Bäume aufragen. Wie an einen entfernten Schatten erinnerte sie sich an die Silhouetten. Dahinter lag es, das Land seiner Hoffnung und seines Hasses. Viele Jahre waren vergangen, seit es erstmals davon gehört hatte. Jahre, in denen es sich großes Wissen angeeignet hatte. Und nun kehrte es zurück, um weitere Geheimnisse zu entdecken. Doch mit dem Horizont näherten sich auch kalte Zweifel: Konnte es hier überhaupt jene Antworten finden, die es suchte?}

\textit{Ratschläge und Einschätzungen dieser Feuermeisterin Nidwal wirbelten durch seine vielen Köpfe. Weise Worte, weise Verständnisse, die es zutiefst erschüttert hatten in ihrer Klarheit.}

\textit{Was konnte es schon bringen, diesem einen Echo einer längst toten Stimme nachzutrauern? Es hatte zuviel Zeit damit verbracht, in die Vergangenheit zu blicken. Dabei war die Zukunft das einzige, worum es sich zu scheren hatte. Wenn es ein Reich regieren wollte, eroberte es es einfach. Zuvor wollte es nur noch sich dieses elenden Gefängnisses entledigen.}

\textit{„Was für eine Verschwendung dies war!“, knurrte es, „Was habe ich mir nur dabei gedacht, nach Danwar zu reisen? Mein Ziel liegt südlicher!“}

\textit{Im Baum der Lieder gab es viele Geheimnisse, einige gar in den Schwarzen Archiven. Vielleicht könnten die ihn aus diesem elenden Kristallsplitter befreien. Und wenn nicht, gab es ja immer noch diesen blinden Seher, der an der Küste lebte.}

\textit{Bald würde es frei sein und die Welt zu seinem Spielball machen. Alle Reiche standen ihm offen. Es könnte zu Varkur zurückkehren und an seiner Seite neue Pläne für eine machtvollere Zukunft entwickeln. Es könnte abwarten bis zur Zeit dieses Zwergenmechanikers und herausfinden, wie er an seine wundersame Sphäre gekommen war – so genau erinnerte es sich leider nicht mehr an dessen Erinnerungen. Es könnte den Urtroll übernehmen und mithilfe dessen Stärke die vermaledeiten Krahder ihrem gerechten Ende zuführen. So viele Möglichkeiten. Aber alles zu seiner Zeit.}

\textit{Das Böse drehte sich seinen finsteren Feuerkriegern zu und sah Jivids Körper dutzendfach in ihren Gesichtsfeldern. Es befahl ihnen im Geiste:}

\textit{„Taucht!“}



































\newpage
\section{Ein Wiedersehen}



\az{Jahr 61}

\textit{Rote Grotte, 61 a.Z.}\bigskip




Das Seil, welches entlang des Höhlengangs gespannt war, endete abrupt. Jarid blieb stehen und öffnete ihre Augen. Rotes Licht blendete sie.

Die Rote Grotte. Eine von Stimmen erfüllte Höhle, mit von Lavastein durchzogenen Wänden. Hin und wieder suchten ausgewählte Bewohner Danwars diesen Ort auf, um den Stimmen zu lauschen und Wissen über ihre Zukunft zu erhalten.

Jarid hatte natürlich gewusst, dass die Rote Grotte rot war, aber nicht, dass sie so rot war. Und so laut. Während sie sich den Gang zur Grotte hin getastet hatte, war das Flüstern der Stimmen der Toten immer lauter geworden. Inzwischen war es ein ohrenbetäubendes unverständliches Wirrwarr aus Gemurmel, Geflüster und Geschrei. Ein kulminierendes Rauschen, welches an das Wogen des Meeres erinnerte, an Jarids Trommelfelle presste und ihren Geist verwirrte.

„Ich bin hier, um meine Prophezeiung zu erfahren“, rief Jarid laut, ohne ihre Stimme zu hören. „Ich werde Danwar verlassen, an der Seite des frischernannten Feuerkriegers Trieest. Was wollt ihr mir mitteilen?“

Die Stimmen flüsterten eine Zeit lang ungestört weiter. So traf es Jarid unerwartet, als sie sich urplötzlich vereinten und wie ein riesiger Chor mit hunderten von Stimmen sprachen:

„Sei vorsichtig mit deinen Wünschen, Jarid Morgentau. Du wünschst dir, durch die Reise fort von hier Gutes zu tun in der Welt und etwas zu erreichen. Das wird dir auch gelingen. Doch wird dein Trieest nie seinen Prozess des Wandels beenden. Er wird nie ein Mensch werden. Nie einer von uns. Doch musst du dafür sorgen, dass er das nicht erfährt. Sollte er es erfahren, wird es seine Hoffnung zerstören. Er würde sich das steinerne Herz aus seiner Brust reißen, und er würde elend verenden. Das darfst du nicht zulassen, Jarid Morgentau. Erhalte seine Hoffnung am Leben, während er mit seiner Bürde kämpft. Wenn die Zeit reif ist, so sage ihm, dass du wissest, dass sein Prozess eines Tages ein erfolgreiches Ende habe. Sage ihm das, und verlängere so sein Dasein. So lange wie möglich.“

Jarid sank zu Boden und versuchte, das Gehörte zu verdauen. Denkbar schwer, wenn weiterhin dutzende von Stimmen an ihr Ohr drangen und leise wisperten:

„Sein Leid hat einen Sinn, Jarid Morgentau. Deine Lügen haben einen Sinn. Das wirst du uns glauben.“

„Warum sagt ihr mir das? Ich hätte ihn auch sonst begleitet und seine Hoffnung aufrecht erhalten! Warum lasst ihr mich lügen, warum lasst ihr denn mich leiden?“

„Dir ein leidloses Leben schenken zu können, liegt nicht in unserer Macht. Auch wir sind nicht perfekt. Doch auch dein Leid hat einen Sinn, Jarid Morgentau. In unseren Augen, und eines Tages auch in den Deinen.“

Dann löste sich der Stimmenchor wieder in einen unverständlichen Chor vielseitigster STimmen auf.

Und Jarid saß da, umgeben von rotem Schein und Lärm, und schluchzte unhörbar vor sich hin.\bigskip







\az{Jahr 68}

\textit{Sieben Jahre später.}\bigskip




Trieest leerte seine Blase. Es fühlte sich anders an als sonst. Weniger beschämend, mehr bestimmend. Markierten Skrale üblicherweise ihr Revier? Besaßen Skrale überhaupt ein Revier? Es gab noch so viel von seinem potentiellen zukünftigen Leben, über das er nicht Bescheid wusste.

Nicht zuletzt hatte er keine Ahnung, wie er mit Jarid verfahren sollte. In eine Skral-Sippe würde sie kaum passen. Doch so etwas hätte Trieest vor einigen Tagen auch von sich selbst behauptet. Wollte sie überhaupt noch etwas mit ihm zu tun haben. Und er mit ihr?

„Schlaf stärkt die Muskeln und die Sinne. Beide werden wichtig sein für die bevorstehende Schlacht“, grunzte Calrai an Trieests Seite.

„Was machst du dann noch hier?“, erwiderte Trieest unsanft.

„Dir helfen. Ich freue mich darüber, dass du uns nicht im Skralreigen stehen lassen hast. Wir sind dir auf deinen Wunsch hier in den Norden gefolgt, und werden an deiner Seite kämpfen. Wenn deine Begleiterin recht hat, steht uns ein unangenehmer Kampf mit mächtigen Gegnern bevor. Ein Häuptling kann als gutes Beispiel vorangehen. Du brauchst Schlaf.“

Trieest blieb stumm.

„Dich beschäftigt etwas“, meinte Calrai, „Die Art, wie du deine Augen verengst, verrät dich. Kannst du etwas tun, um damit es dich nicht mehr beschäftigt?“

„Vermutlich schon.“

„Wirst du es tun?“

„Ich weiß es nicht. Das ist alles sehr neu für mich.“

Tri verabschiedete sich aus Calrais Umklammerung und ging weiter zu Jarid. Sie saß etwas abseits der Sippe an einem Baum und spann an einer Brieflilie. Dem Wunsch der Skrale nach war sie nicht mehr zum Baum der Lieder zurückgekehrt. Jarids und Trieests Versicherungen hin oder her, die Bewahrer waren nie gut auf Skrale nahe ihrer Dörfer zu sprechen.

Trieest hielt sich im Schatten zurück. In dem Augenblick, in dem Jarids Geruch seine Nasenflügel erreichte, machte sich ein unangenehmes Gefühl in seinem Magen breit. Hunger. Hunger auf Fleisch. Durst nach Blut. Er riss sich zurück.

„Warum so zwielichtig, Tri?“, fragte Jarid leise, „Tritt aus dem Schatten.“

Trieest verfluchte erneut das Skralblut in seinen Adern. Wie hatte er in seiner Kindheit diesem Drang nur so lange widerstanden? Oder war es erst die Unterdrückung durch den Lavastein gewesen, die ihn so stark gemacht hatte?

„Grün sind die Wogen der Wellen“, sprach Trieest.

Jarid erwiderte müde: „Und Grün sind die Lichter des Nordens. Beschäftigt dich etwas?“

„Wer weiß schon, was die Zukunft bringt? Wir gehen bald gegen Nekromantie vor. Ich würde diese Angelegenheit lieber vorher hinter uns bringen, sonst lässt sie mich nicht in Ruhe. Hast du Zeit?“

Jarid legte ihren Kopf schief und hörte auf, an ihrer Brieflilie zu spinnen: „Diese Brieflilie wird vermutlich ohnehin zu spät kommen. Was würdest du gerne wissen?“

Tri ließ sich auf den Waldboden plumpsen und sammelte seine Gedanken.

„Ist es wahr, dass du wusstest, dass ich nie ein Mensch werden würde?“

„Das ist wahr. Zumindest glaubte ich den Echos der Roten Grotte, als sie dies behaupteten“, sprach Jarid betont neutral.

Trieest hielt seine Stimme auch betont neutral, als er hervorwürgte: „Dann ist es auch wahr, dass du mich jedes Mal angelogen hast, wenn du mich dazu ansporntest, weiter zu reisen?“

„Das ist leider auch wahr. Tri, ich mochte das genauso wenig wie du jetzt, und ich habe auch stark darunter gelitten.“

„Krarkdreck sind sie, diese Stimmen der Toten! Sie sollen uns anleiten, nicht manipulieren!“ Trieest fasste sich wieder und folgte mit: „Es fühlt sich schrecklich an, zu wissen, dass all unsere Mühen umsonst waren.“

„Ich kann es positiver sehen. Wir haben vielen Personen geholfen, nicht zuletzt uns selbst. Die Echos meinten, dass es dich brechen würde, wenn du vorzeitig erfahren hättest, dass du ...“

Jarid stockte. Dann holte sie tief Luft und sprach mit gesenktem Blick: „Sie meinten, dass du dich umbringen würdest, wenn du dies frühzeitig erführest.“

Da, es war raus. Jarid lehnte sich zurück und holte zitternd Luft.

Trieest machte große Augen und suchte nach den passenden Worten, während sein Gesicht verschiedenste Emotionen durchlief. Zähnefletschend knurrte es dann: „Diese elenden Lügner! Diese hochwohlgeborenen Stimmen der Gefallen, die sich für besser als uns halten! Vermutlich fühlen sie nicht mal was. Und ... und du hast dich immer noch nicht entschuldigt.“

„Verzeih mir! Entschuldige abertausendmal, dass ich dich angelogen habe, Trieest, ich wollte das nicht, aber ich würde es wieder tun.“

„Unsere Vorstellung von Entschuldigungen gehen auseinander.“

„Unsere Vorstellungen der Wahrheit aber nicht. Du hast gelitten, viel und lange und unnötig, das will ich nicht kleinreden, aber mir ging es auch nicht gut dabei! Und wir sitzen beide im selben Boot hier. Wir haben dieselben Ziele. Wir sind auf demselben Stand. Wir können die Vergangenheit Vergangenheit sein lassen und uns auf die Zukunft konzentrieren.“

„Ich mag es nicht, Puppen dieser Echos zu sein! Und ich mag es nicht, dir nicht mehr unbeschränkt vertrauen zu können! Ich mag das alles nicht“, grummelte Trieest mit verschränkten Armen.

„Ich doch auch nicht, Tri. Aber ich glaube, von jemanden zu verlangen, dass diese Person dich niemals unter irgendwelchen Umständen anlügen würde, ist eine unrealistische Erwartung. Immerhin tat ich es zu deinem Besten. Und immerhin tat ich es, während ich mir so gut wie absolut sicher war, dass es einen guten Grund dafür gab. Wer sonst kann schon ... klar, klar, das war nicht zwingend das Geschickteste, was ich jetzt hätte sagen können. Tri, Tri! Bitte geh nicht weg. Tri, schau mich an. Es tut mir leid. Es tut mir aufrichtig leid, ich wünschte, es wäre nicht so gewesen. Was hättest du an meiner Stelle getan? Dir die Wahrheit gesagt, auf dass du dich vielleicht umgebracht hättest? Dieser Lavastein war darauf programmiert, deinen Prozess des Wandels fortzuführen, und wir besaßen nicht die Möglichkeiten, ihn von allein zu entfernen.“

„Pah. Wir hätten auch ohne diesen Lavastein unsere Heldentaten vollbracht.“

„Hätten wir, Tri?“

Trieest schwieg und mahlte mit seinem Unterkiefer.

„Warum würde die Rote Grotte uns das antun?“, fragte er dann, um den Druck von Jarid zu nehmen, „Vielleicht, wenn du mir helfen kannst, irgendeinen Sinn in diesem Quark zu erkennen, kann ich eher meinen Frieden damit finden. Kann es wirklich nur gewesen sein, dass ich diese Bürde tragen musste, um so vielen Personen zu helfen? Steppenländler, Andori, Taren und Tulgori, alles Personen, die den Danwaren und ihren Echos kaum etwas bedeuten?“

„Es könnte auch etwas viel Spezifischeres sein. Wärst du den Lavastein früher losgeworden, hättest du mich zum Beispiel nach dem Kampf gegen den Hraak nicht retten können. Und ...“

Jarid riss ihre Augen auf.

„... und der bösartige Seher hätte uns überlistet! Derjenige, der dich auf die Insel Narkon schicken wollte! Wenn ich die Prophezeiung der Roten Grotte nicht auf den Weg bekommen hätte, dann hätte ich nicht gewusst, dass der Seher dich belog, als er behauptete, dass du auf Narkon deinen Prozess des Wandels beenden würdest! Nur wegen dieser Prophezeiung reisten wir nicht auf dieses verfluchte Stück Fels. Wer weiß, was uns alles zugestoßen wäre!“

„Und das konnten die Echos aus der Roten Grotte dir nicht direkt verraten?!“

„Dann wären wir vielleicht nie an diesen Punkt gekommen.“

„Vielleicht nicht. Ich hasse diese Unklarheit in allen zukunftskennenden Wesen. Diese Macht, die sie über einen haben.“

„Sobald das alles durch ist, lassen wir alle Seher weit hinter uns, Tri.“

Tri schluckte. „Ja, was werden wir tun, wenn das alles hinter uns ist?“

Er ließ die Frage offen im Raum stehen. Jarid hatte sich bislang wohl gezwungen gefühlt, ihm zu folgen und ihm beim Tragen dieser elenden Bürde des Lavasteins zu helfen. Jetzt war sie frei. Würde sie nach Danwar zurückkehren wollen? Für Trieest gab es dort nichts mehr.

„Eines weiß ich sicher, nach Danwar zurückkehren werde ich nicht. Ich war dort, Tri, und es ist keinen Lavadeut besser geworden.“

Kurz wandte sie ihren Blick ab und ballte ihre Faust. Tri kannte diese Geste gut genug, um zu wissen, dass sie gerade etwas unterdrückte. Eine Träne? Kurz haderte er mit sich selbst, doch dann lehnte er sich wieder nach vorne und fragte vorsichtig: „Ist etwas in Danwar vorgefallen?“

„Sie ist tot!“, schluchzte Jarid auf, „Mama ist tot! Wegen mir!“

Das war mehr, als Trieest erwartet hatte. Fast wie automatisch öffnete er tröstend seine Arme und Jarid ließ sich hineinfallen.

„Das ist zugegebenermaßen schrecklich“, stammelte er. Was sollte er auch sagen? Jarid davon zu überzeugen, dass es nicht ihre Schuld war, wie es vermutlich war? Dass sie sich jetzt auf anderes zu konzentrieren hatten? Sie abzulenken? Er brummelte eine alte danwarische Melodie und umarmte Jarid sicher und fest. Auch er konnte kaum seine Tränen zurückhalten. Ein gewaltiges Gewicht löste sich von seinem Herzen, eines, dass lange dort festgesessen hatte.

„Jarid, ich verstehe und vergebe dir.“

Stille. Jarid schniefte.

Dann sprach sie ein neues Thema an: „Danke. Und du, wie ist es dir ergangen? Wirst du ab jetzt ein Skral-Häuptling sein?“

„Einer von ihnen? Einer unter Mördern und Vergewaltigern? Nicht wirklich. Und vielleicht schon. Diese vier Skrale, Calrai, Kurgat, Tran und Bark, die wollen mir folgen. Denken, dass sie an meiner Seite mehr Chance auf Ruhm, Ehre und ein langes, sicheres Leben hätten. Ich stelle sie dir näher vor, sobald dieser Scheißdreck um diesen bösen Kristall hinter uns liegt.“

„Ich will nicht so tun, als hätte ich nicht gewisse Vorurteile, die es schwer machten, diesen Skralen zu vertrauen. Aber ich bin gespannt, ob das unser zukünftiger Pfad sein wird. Weiter weg vor der Zivilisation mit einigen Skralen an unserer Seite.“

„Na, wenn das hier vorbei ist, darfst du dich gerne uns anschließen. Wobei das Leben in einer Skral-Sippe vielleicht nicht das Wahre für dich ist.“

„Wir werden sehen, Tri. Das ist eine Aufgabe für die zukünftigen Wirs. Ich werde auf jeden Fall in deiner Nähe bleiben, wenn das für dich in Ordnung ist. Mit Danwar bin ich fertig.“

„Ich auch. Was diese Ältesten mir angetan haben ... Ich weiß, Träume sind Schäume, aber manchmal sehe ich mein jetziges Gesicht in einer Pfütze und bin glücklich mit dem, was ich erblicke. Wenn der Rat ein wenig offener gewesen wäre ...“

„Oh, Tri ... komm her, lass dich knuddeln, du Großer! Und lass dich nicht unterkriegen von der Welt! Wir packen das schon!“

Wassertropfen perlten an seinen neuen Schuppen ab. Jarid weinte! Das tat sie sonst nie. Trieests drückte sie noch etwas fester. Lange blieben sie so verschlungen. Dann löste sich Jarid langsam.

„So, und jetzt muss ich diese Brieflilie an Jirid absenden.“ Sie wischte sich die letzten Tränen vom Gesicht und lachte kurz auf, „Wobei das wohl nicht viel nützen wird. Selbst wenn die Helden die Nachricht gleich kriegen und sie noch wach und fit sind, dauert es über einen Tag, bis sie endlich hier wären. Und nach der Gefallenenzeremonie ist der große Brunnen neben der Rietburg bestimmt völlig geleert. Demnach könnte nicht einmal Jirid hierher tunneln. Ich befürchte, wenn das Böse mit seiner Horde aus kontrollierten Feuerkriegern hier eintrifft, werden keine weiteren Helden hier sein, um es zu erwarten.“

„Nicht alle Helden von Andor verbringen ihre Zeit mit Feiern“, erklang ein Flüstern aus dem Gebüsch hinter Trieest. Dieser zuckte zusammen, als ein dunkel gewandeter Bursche aus dem Baum trat und sich neben sie stellte. Ein schwarzer Köcher aus Wardrakleder hing an seiner Seite, die dazugehörige Waffe war wohl unter seinem Umhang verborgen.

Ein Held von Andor, wenn auch einer, der lieber Zeit im Schatten als im Licht verbrachte. Trieest hatte erst selten mit ihm gesprochen, doch hatte er sein Kampfgeschick bereits einige Male bewundert.

Jarid fasste sich als erste wieder: „Arbon! Wie lange versteckst du dich schon hier?“

„Keine Sorge, ich habe bloß das Ende eurer melodramatischen Gespräche mitbekommen. Ha! Da lohnte sich es doch, dem Totenfest fernzubleiben! Das kriegt Fenn so was von unter die Nase gerieben ... Ich meine natürlich: Ich hörte, eine große Gefahr kommt auf uns zu, und ich, Hogo, der tapfere Andori, werde mich auf die Seite des Guten stellen!“

„‚Hogo‘, soso? Hast du dich immer noch nicht mit Melkart getroffen und eine Begnadigung ausgehandelt?“, fragte Jarid interessiert, „Wenn eines Tages eine Steintafel von deinen Heldentaten berichten soll, soll sie dann etwa vom tapferen Hogo aus dem Rietland erzählen?“

„Wenn irgendwann meine Taten niedergeschrieben werden, so werde ich diese große Aufgabe sicherlich selbst in die Hände nehmen“, sprach Arbon in seinem besten überheblichen Bewahrer-Ton, „Und aushändigen werde ich sie an die besten Barden diesseits des Fahlen Gebirges, auf dass jene sie in allen Weiten des Landes erschallen lassen. Der ehemalige Königsbarde Grenolin. Oder die gute Gilda aus ihrem gemütlichen Gasthaus. Oder der Spielmann Pasco in Bechtholds Diensten. Welcher Name mir von diesen dann am Ende zugesprochen wird, kann mir völlig egal sein. Ich tu das ja nicht Ruhm und Ehre wegen. Meine Blutsfamilie sind neben Reka und einigen Schwarzen Wachen doch die einzigen, die mit ‚Arbon‘ etwas anfangen könnten, und denen schulde ich nichts. Aber ich komme ja ganz ins Plappern, und um mich geht’s hier ja ganz und gar nicht. Was nun, was tun?“

Jarid und Trieest blickten sich amüsiert an. Jarid meinte schließlich:

„Wir warten, bis die Schiffe des Bösen am Horizont auftauchen. Dann rüsten wir uns für einen Kampf. Unser Ziel muss sein, den schwarzen Kristallsplitter zu vernichten, von dem aus das Böse seine Feuerkrieger kontrolliert.“

„Würde dieses Böse den Kristallsplitter nicht lieber anderswo sicher aufbewahren, sodass wir ihn nicht erreichen können?“

„Es ist nicht nur böse, sondern auch hochmutig. Es wird so unvorsichtig sein.“

Während Trieest Arbon auf den neusten Stand brachte, beendete Jarid den finalen Schliff an ihrer Brieflilie und sandte sie gen Westen. Die Lilie verengte sich zu einem kleinen, blau leuchtenden Körnchen und segelte durch die Luft, stetig ihrem Ziel entgegen.

„Was haben diese kontrollierten Feuerkrieger für Fähigkeiten? Können sie mit ihren geschwärzten Lavasteinen auch andere Wesen beeinflussen?“, fragte Arbon gerade nachdenklich.

Trieest zuckte mit den Schultern. Jarid meinte:

„Vermutlich nicht. Sonst hätten sie diese Fähigkeit schon in Danwar eingesetzt. Doch auch so sind sie gefürchtete Gegner. Das sind keine Halunken, sondern voll ausgebildete Feuerkrieger von Danwar. Dieser Orden ist für seine Kampfstärke bekannt.“

„Ist er das?“, fragte Arbon, „In diesen Landen hier hat man bislang vor allem von Trieest gehört, und während sicherlich keiner sein Kampfgeschick in Frage stellen will, ist er wohl keineswegs ein typischer Krieger Danwars.“

Trieest wandte sich ab. Solche Gespräche machten ihn immer nervös. Warten war noch nie seine Stärke gewesen.

Jarid meldete indes an: „Schadet den Feuerkriegern nicht zu viel. Das sind immer noch die Körper guter Menschen, die bloß besessen wurden.“

„Bist du dir sicher, dass man ihren Geist wieder in den Körper zurückführen kann, sobald die Verbindung zum schwarzen Kristallsplitter gebrochen wurde?“

„Ihr Geist hat den Körper nie verlassen!“

„Sagst du. Wie lange hat das Böse Trieest kontrolliert? Zwei Minuten? Mit Verlaub ...“

„Selbst wenn man ihnen schadet, wird das nicht viel helfen“, mischte sich Trieest wieder ein. „Das Böse ist ein Nekromant. Es konnte Lysbetts Körper weit nach ihrem Tod kontrollieren. Es vermag, selbst tödliche Wunden in von ihm kontrollierten Personen zu ignorieren.“

Arbon machte große Augen: „Sie sind auch noch unsterblich?! Brauchen wir Drachenrelikte, um ihnen zu schaden?“

„Ich konnte die Kontrollierten auch so bekämpfen.“

„Oh nein“, entfuhr es Trieest.

„Was?!“, fragten Jarid und Arbon im Gleichklang.

„Wenn das Böse die kontrollierten Körper irgendwie magisch vor dem Tode bewahren kann ... was spricht dann dagegen, dass es seine Schiffe außerhalb des Ufers im Nebel ankern lässt und mit seinen Kriegern unter dem Wasser hierher spaziert? Untote können nicht ertrinken. Vielleicht schlichten sie weit unter dem Meeresspiegel in den Wachsamen Wald. Wir hätten nicht nur nach Booten Ausschau halten sollen.“

Arbon begriff als Erster, zückte seine Arcuballiste und rannte in Richtung des Baums der Lieder. Jarid stieß noch ein entsetztes „Vielleicht sind sie bereits hier!“ hervor, ehe sie Arbon nachrannte. Trieest holte tief Luft und stieß einen beeindruckenden – wenn auch noch etwas unsauberen – Ruf der Skrale aus, um Calrai und die übrigen Skrale zu alarmieren. Dann setzte er Jarid und Arbon nach.\bigskip







Der Baum der Lieder lag ruhig da, das rötliche Mondlicht glitzerte auf seinen silbernen Blättern. Das Dorf war stumm. Nur die Tatsache, dass keine Wachposten mehr das Eingangstor des Baumes bewachten, störte das idyllische Bild ein wenig. Arbon, Jarid und Trieest hetzten mit gezückten Waffen die gewundene Wendeltreppe hoch – niemand.

Auf den Balkon hinaus – niemand.

Um die halbe Balustrade herum – niemand.

Die Raumflucht betreten – da war jemand! Vor der zweiten Tür links beugten sich zwei spitzohrige Gestalten über eine dritte Person, welche verzweifelt versuchte, davonzukrabbeln. Dem neben ihr liegenden Fahrstuhl nach war das wohl Hohepriester Tion.

„Tion hat Zugang zu den Schwarzen Archiven!“, flüsterte Arbon, „Sie wollen irgendwelche Geheimnisse erlangen. Doch keine Sorge, Tion ist ein äußerst verpflichteter Bewahrer, er wird ihnen nichts verraten.“

Wie um Arbons Worte Lüge zu strafen, nickte der zusammengesunkene Tion ergeben. Zitternd wurde er in die Höhe gehoben, während das Rankenschwert des einen finsteren Feuerkriegers stets auf seinen Hals gerichtet blieb. Und mit bebender Stimme rief Tion: „Schwarze Wachen! Öffnet die Tore!“

„Als ob!“, grummelte Arbon und zückte seine Balliste. Zwei Schüsse später fiel Tion wieder zu Boden, neben ihm zwei danwarische Leichen. Diese blieben jedoch nicht lange tot. Und sie blieben nicht lange allein.

Aus dem Dunkeln trabten weitere Feuerkrieger und stürzten sich auf die Helden. Tion kroch ächzend durch das geöffnete Tor ins Innere der Schwarzen Archive. Sobald er in Sicherheit war, verriegelten die Schwarzen Wachen die Tür von außen wieder und zückten Messer.

Jarid spreizte ihre Hände. Es knirschte und knarzte, dann brachen riesige Wassermassen aus den Löschfässern, welche oben an verschiedenen Ästen des Baums der Lieder gehangen hatten. Mit geschlossenen Augen lenkte Jarid das Wasser durch den Raum. Die noch stehenden Feuerkrieger wurden von den Beinen gehauen und in die Tiefe gerissen, während sich das Wasser galant um unsere Helden und Wachen teilte. Leider wurden auch einige Dokumente und Chroniken mitgeschleppt.

„Hui, wo bleibt deine Rücksicht auf die Feuerkrieger?“, fragte Arbon schnippisch, „Und wo ist der schwarze Kristallsplitter?“

Jarid antwortete nicht, sondern setzte sich einen umherklappernden Helm auf, den eine Feuerkriegerin verloren hatte. Sie stählte sich für ihren nächsten magischen Akt.

Kampfeslärm drang von unten herauf. Skrale und Feuerkrieger hatten sich in ein Gefecht verstrickt.

„Calrai!“, rief Trieest und rannte die Treppe wieder hinunter.

„Fang mir einen von ihnen! Ich brauche einen Lavastein!“, rief Arbon ihm hinterher. Trieest hoffte ganz fest, dass es sich um einen Plan gegen diese Wesen handelte, statt um einen innigen Sammlerwunsch, der gerade im ehemaligen Bewahrer aufgekommen war.

Kaum war er aus dem unteren Tor am Baum der Lieder herausgerannt, stürzte sich auch schon ein Feuerkrieger auf ihn. Orange-rote Feuerschlieren umringten ihn und gruben sich in Trieests Rüstung. Dieser packte den Angreifer mit beiden Händen und warf ihn auf den Waldboden, welcher vom Auftreffen der Löschfässer noch tropfnass war. Selbiges Wasser kroch prompt am Feuerkrieger hervor und drückte seinen Körper tiefer in den Matsch. Jarid war ihm gefolgt!

Der finstere Feuerkrieger wand sich am Boden, doch sein Blick war leer. Das Böse fokussierte sich wohl gerade nicht auf ihn.

„Danke sehr für das Testsubjekt!“, rief Arbon fröhlich, ehe er dem strampelnden Feuerkrieger auf die Brust stand und den dunklen Lavastein in dessen Brust beäugte. Er zog einen kleinen grünen Runenstein aus seiner Manteltasche und hielt ihn dagegen. Das leise Summen des Runensteins wurde ein wenig stärker, sonst geschah aber nichts. Arbon schüttelte seinen Kopf, zog einen roten Stein aus seinem Umhang und brachte diesen näher zum Lavastein des Feuerkriegers. Diesmal schien eine anziehende Kraft vom Lavastein auszugehen, denn der rote Kristall flutschte aus Arbons behandschuhten Händen und knallte klirrend gegen den schwarzen Lavastein, woraufhin sich die Schwärze prompt in den roten Kristall ergoss.

„Drachenmagie, also?“, murmelte Arbon verwirrt, „Das sollte mit Untoten eigentlich überhaupt nicht harmonieren.“

„Häuptling, Obacht!“, erklang Kurgats tiefe Stimme von hinten. Trieest konnte sich gerade noch rechtzeitig umdrehen, um eine anstürmende finstere Feuerkriegerin zu sehen. Sie sah vom Sturz vom Baum noch ziemlich lädiert aus. Einige Knochen waren bestimmt gebrochen, doch nicht ihr Kampfeswille. Trieest zückte sein Rankenschwert und brachte die Feuerkriegerin zunächst zu Fall, dann zu völliger Bewegungsunfähigkeit. So viel zur Kampfstärke des Ordens, zu Strategien und Formationen. Das Böse musste seine Mühe damit haben, so viele Puppen auf einmal zu steuern, und ließ sie eigenständig wilde Angriffe machen, die leicht zu überwältigen waren. Worauf fokussierte es sich, was konnte wichtiger sein als diese Schlacht hier?

Trieest musste zwei weitere anstürmende Feuerkrieger in ihre Plätze verweisen, ehe er sich zu Arbon zurückdrehen konnte. Dieser war gerade daran, Jarid zu instruieren:

„... These besteht darin, dass dieses Böse mithilfe von Drachenmagie in seinem Kristallsplitter eingeschlossen wurde, und diesen Käfig auf andere Kristalle ausdehnen kann, solange er den Kontakt zur gelenkten Person nicht verliert. Letzteres würde die Verbindung zerstören. Es macht die Lavasteine der Feuerkrieger damit quasi zu Drachenrelikten und seine Untote nicht mehr unverwundbar. Das ist wirklich faszinierend. Nein, da rüber. Lenk den Wasserstrahl in die Fugen meines Relikts. Ja, genau so. Wenn diese Kristalle irgendwie über seinen Geist verbunden sind, dann sollten wir in der Lage sein, diese Verbindung zu verfolgen und herauszufinden, wo es seinen Hauptkörper mit dem Kristall versteckt. Ja, jetzt verbinden. Gut machst du das!“

Arbon streute eine Prise eines seltsamen pink glitzernden Pulvers aus einer Manteltasche über seinen roten Stein, welcher mit Jarids Hilfe von einer dünnen Wasserschicht überzogen war und durch eine wässerigen Doppelhelix Kontakt mit dem Lavastein des Feuerkriegers am Boden hatte. Hin und wieder flackerte der rote Stein tiefschwarz auf.

„Komm her, Trieest! Du hast als einziger von uns bislang einen Kontakt mit einem Lavastein aufgebaut, und die Wassermagie sollte dich vor dem Einfluss seines Geistes schützen. Berühre mein Drachenrelikt und sage mir, was du siehst!“

Trieest war niemand, der sich lange mit Diskussionen über das für und wider solch seltsamer Anweisungen aufhielt. Wenn jemand anderes einen Plan hatte, war dieser für gewöhnlich nicht zu missachten, und Trieest hatte ohnehin nicht den Durchblick, den es für Kritik brauchte. So stürzte er rasch zum immer noch im Boden festgehaltenen Feuerkrieger und griff fest nach dem Drachenrelikt. Er spürte kurz die Nässe von Jarids Wasser, dann die Härte des überraschend warmen Steines. Dann wurde alles dunkel.\bigskip



\textit{Es war ein seltsames Gefühl, als würde Trieest durch ein dutzend Augenpaare gleichzeitig gucken und ihre Gesichtsfelder überlappend sehen. Da waren ein, zwei Feuerkrieger, die gerade in ein Gefecht mit seinen Skralen verwickelt waren. Sie sahen sich gegenseitig. Er sah Calrai, dessen zeremonieller Speer sich in sein Gesichtsfeld bohrte. Zweifelsohne hatte Calrai soeben einen finsteren Feuerkrieger erledigt.}

\textit{Trieest blickte in sein eigenes verzerrtes Gesicht, das mit geschlossenen Augen über ihm schwebte, seine eigene Hand, das Drachenrelikt in seiner großen Faust fast verschwindend.}

\textit{Kein Zweifel: Er sah das, was das Böse sah. Und er hörte das, was das Böse hörte, so viele überlappende Stimmen, konnten die alle von ihm sein? Hörte er auch seine Gedanken? Wo war der Hauptkörper? Trieest suchte die verschiedenen Gesichtsfelder ab, aber sie alle zeigten bloß den Baum der Lieder, den andorischen Himmel, pure Schwärze und Matsch ...}

\textit{Da!}

\textit{Eine dunkle Hütte, winzige Fenster, kaum erleuchtet. Eine hochgewachsene Gestalt in einem eleganten dunklen Mantel, welche am Boden kniete. Und die Hand einer kleinen Feuerkriegerin, welche einen schwarzen Kristallsplitter von der Stirn der Gestalt nahm.}

\textit{Ein verzweifeltes grelles Lachen einer hohen Stimme, die Trieest nicht kannte: „Ich bin zu früh! Es wäre auch zu schön gewesen ...“}

\textit{Die Gestalt am Boden blickte nicht auf, als sie murmelte: „Wer ... was bist du? Was willst du von mir?“}

\textit{Das Gesichtsfeld drehte sich leicht, als das Böse den Kopf seiner Puppe in den Nacken legte und wütend knurrte. Dann sprach es: „Ich will frei sein, Leander! Wenn die Schwarzen Archive mir nicht weiterhelfen, dann werde ich es halt aus dir erfahren. Überlege gut, denn dein Leben hängt davon ab. Wie würdest du eine Seele aus einem mit Drachenmagie versiegelten Kristall befreien?!“}

\textit{„Das weiß ich auf die Schnelle kaum, Drachenmagie ist nicht meine Stärke“, versuchte dieser Leander sein Glück.}

\textit{„Das weiß ich, sonst wüsste ich die Antwort bereits, als ich in dir war“, grummelte das Böse, „Doch nun müssen wir es auf die altmodische Art machen. Du kennst deine Erinnerungen am besten, alter Mann, wo würdest du zuerst nachforschen? Irgendetwas musst du doch bereits wissen, lange wird es nicht mehr dauern!“}

\textit{Leanders hilfloses Gelaber wurde zu einem leisen, unverständlichen Gemurmel. Das Böse hörte schon nicht mehr hin. Stattdessen zeigten nun mehrere Gesichtsfelder den knienden Trieest aus verschiedenen Blickwinkeln.}

\textit{„Trieest, was machst du denn hier in meinem Kopf?!“, antwortete das Böse fast fröhlich.}\bigskip



Trieest stolperte zurück und öffnete seine Augen.

Jarid, Arbon und einige verletzte Feuerkrieger um ihn herum starrten ihn an. „Und was ist mit deinem Lavastein geschehen?“, sprachen die Krieger in einem verwirrenden Chor. „Zu schade, der hätte mich eigentlich interessiert.“

Trieest ignorierte sie. „Das Böse befindet sich in einer Hütte. Es hat eine Gestalt ausgefragt. Es will von ihr erfahren, wie es aus dem Kristall befreit werden kann. Auch wenn ... irgendetwas ... noch zu früh sei. Ihr Name ist ... Leander?“

„Der alte Leander!“, rief Arbon, „Er lebt in einer Hütte nahe des Meeres. Er konnte mir schon einige Male aushelfen, und er scheint stets an uraltem Wissen interessiert. Es würde Sinn ergeben, wenn das Böse bei ihm nach einem Ausweg aus seinem Gefängnis suchte.“

„Zumal es so schnell nicht ans Wissen der Schwarzen Archive kommen wird“, meinte Jarid mit einem zufriedenen Blick auf die Lage um den Baum der Lieder. Die finsteren Feuerkrieger kämpften tapfer, doch hatten viele von ihnen schon gebrochene Beine vom Sturz vom Balkon, und auch wenn die Untoten scheinbar weder Blut noch Muskeln zum Bestehen und Bewegen benutzten, schien ihre Integrität stark davon abzuhängen, dass ihre Knochen heile waren. Die meisten Feuerkrieger waren bereits zusammengeklappt und von Skralen, Bewahrern oder Schwarzen Wachen entwaffnet worden. Calrai fesselte soeben die strampelnde Feuermeisterin Nidwal mit einigen Lumpen.

Der Lärm hatte zahlreiche Dorfbewohner geweckt, und auch wenn die meisten grau gewandten Bewahrer bloß mit großen Augen dem Geschehen zuguckten, waren inzwischen schon einige grün gewandte Bogenschützen auf die Lichtung getreten. Sie schienen sich nicht so sicher zu sein, ob sie nun die Feuerkrieger, die Skrale, Trieest oder etwa Arbon anzielen sollten. Hohepriester Tion meldete sich von oberhalb des Balkons und versuchte, die Situation zu entschärfen.

Die Skrale zogen sich auf einen Wink Calrais ins Unterholz zurück. Arbon schien plötzlich ebenfalls wie vom Erdboden verschluckt. Noch während Trieest sich nach ihm umsah, winkte Jarid in eine bestimmte Richtung.

Trieest rannte von dannen. Hinter ihm schlug ein grün gefierter Pfeil in einem Baumstamm und jemand rief „Obacht, Skrale!“ Sie konnten Idioten sein, diese Bewahrer. Immerhin hatten sie die Lage am Baum der Lieder mehr oder minder unter Kontrolle. Eher minder, wenn er das plötzliche Gebrüll weiterer finsterer Feuerkrieger von der Lichtung her richtig interpretierte. Aber der schwarze Kristall hatte Priorität. Und der befand sich Arbons Worten zufolge vor ihnen, in Richtung Norden.\bigskip







Trieest rannte Jarid hinterher. Diese Tätigkeit fühlte sich zugleich altbekannt und doch wie eine neue Erfahrung an. Sei letztes Lauftraining mit Jarid war noch nicht einmal so lange her, kaum einige Tage vor ihrem Zusammenstoß mit dem wilden Hraak. Und doch schien es so anders. Keine Muskeln waren geschwächt. Kein Lavastein brannte in seiner Brust. Seine Füße und Beine fühlten sich kräftig an, wie sie über den Waldboden trommelten, und seine Brust vermochte plötzlich, so viel mehr Luft einzuatmen.

Nicht, dass alles an seinem neuen Dasein so positiv gewesen wäre. Skrale schienen kaum zu schwitzen, und als Halbskral blieb einem Großteil von Trieests neu schuppiger Haut die nötigen Poren auch verwehrt. So überholte Trieest Jarid großspurig, nur um kurz darauf überrascht stehen bleiben zu müssen und nach Luft zu hecheln. Von da an ließ er es langsamer angehen.

Ein verschnörkelter Weg führte vom Baum der Lieder in den Norden. Irgendwo da oben musste ein Handelshafen liegen, doch dieser war nicht ihr Ziel. Jarid deutete nach links, um Trieest auf einige zerbrochene Äste aufmerksam zu machen. Jemand war hier ins Unterholz abgehauen. Arbon hatte ihnen eine Spur hinterlassen!

Kurz darauf trafen Jarid und Trieest auch auf Arbon selbst, wie er in einem Gestrüpp kniete und sich sein Knie hielt. Der Übeltäter war klar: Eine tückische Wurzel ragte hinter ihm aus dem Boden. Jarid und Trieest sollten nie erfahren, ob dies bloß ein unglücklicher Unfall oder der Unbill eines der zahlreichen launischen Waldgeister in dieser Gegend gewesen waren, doch nach einer kurzen Sicherstellung, dass Arbon nicht ernsthaft verletzt war, rannten sie ohne ihn weiter, dem Weg entlang zur Hütte dieses Leanders entgegen.\bigskip







Das stürmische Hadrische Meer war schon von Weitem zu hören. Trieest öffnete seinen Mund und spürte Meeressalz auf seiner Zunge. Sie waren nahe.

Da vorne öffnete sich das Unterholz und gab einen Blick frei auf den Waldrand. Grünes Gras säumte ein lauschiges kleines Plätzchen, und im Waldrand war ein besser ausgebauter Weg zu erkennen, der vermutlich über einen Umweg Leanders Hütter mit dem Rest der Zivilisation verband.

Sie waren angekommen.

Laut brandete die Gischt des Hadrischen Meeres an die Küste des Wachsamen Waldes. Eine kleine Hütte stand grau und unscheinbar am Waldrand. Im grauen Licht des ersten Morgens wirkten die letzten Baumstämme des Waldrandes schwarz wie die Stäbe eines Käfigs.

Jarid und Trieest mussten nicht nachprüfen, ob der Besitzer der Hütte zuhause war. Denn dieser stolperte soeben schreiend von der Hütte weg, eine amüsiert grinsende kleine Wassermagierin auf seinen Fersen. Die Wassermagierin erblickte Jarid und Trieest als erste, verengte ihre Augen und ballte ihre Fäuste, woraufhin eine gewaltige Welle kalten Salzwassers vom Meer ins Land spülte und diesen ... Leander? ... von den Füßen holte und seine Kapuze beiseite wischte.

Die Wassermagierin stellte sich demonstrativ zwischen Leander und die Neuankömmlinge, doch diese konzentrierten sich nicht besonders auf sie. Jarid blieb als Erste überrascht stehen. Trieest rannte einige Schritte weiter, doch da hob Leander seinen Kopf und fragte: „Wer da?! Schnell, holt Hilfe! Kommt nicht näher! Sie darf euch nicht berühren!“

Und da erkannte auch Trieest ihn. Dieses Gesicht hatte er schon Jahre nicht mehr gesehen, und doch würde er es so bald nicht vergessen. Lügen und weitere Lügen hatte es ihm aufgetischt, auf dass er auf seinen Zielen folgen würde. Einzig Jarids Eingreifen hatte ihn damals bewahrt. Jarid, die nur aufgrund der Stimmen aus der Roten Grotte diese Lügen als solche hatte erkennen können.

Es war der blinde Seher aus der Taverne.

„Du!“, rief Jarid.

Leanders Gesichtsausdruck war unter seiner Kapuze nur schwer zu erkennen, doch schien er sich noch mehr zu versteifen und verhaspelte sich einige Male, bis er krächzend erwidern konnte: „Grün sind die Wogen der Wellen, Jarid Morgentau. Dann ist Trieest vermutlich auch in der Nähe? Bitte, seid so lieb und helft mir gegen dieses Ungetüm.“

„Du bist doch bekanntlich ein Seher, weißt du denn nicht, wie das hier ausgeht?!“, fragte Trieest bissig, ohne die Begrüßung zu erwidern. Dann biss er sich selbst auf die Zunge und nickte Jarid zu. Die beiden teilten sich und schritten in weitem Umkreis von links und von rechts auf die kleine Wassermagierin und den Seher zu.

Leander setzte zu einer Antwort an: „Grün sind auch deine Wogen der Wellen, Trieest. Du wurdest vom grantigen Kord verwöhnt. Ich bin ein Seher, aber ich sehe längst nicht alles. Unlängst konnte ich deine magisch-mächtige Präsenz gar mit meinem inneren Auge spüren, doch nicht einmal das sehe ich nun noch. Was ist mit dir nur geschehen, hast du deinen Lavastein verloren?“

Trieest ignorierte ihn und schritt noch näher.

„Keinen Schritt näher, oder ich entledige mich des Blinden!“, rief das Böse aus dem Mund der kleinen Wassermagierin.

„Du dir keinen Zwang an, an ihm liegt mir nichts. Doch benötigst du ihn nicht, um deinem kristallenen Gefängnis zu entkommen?“, erwiderte Jarid bissig und schritt unbeirrt weiter.

„Ich kann allen Anwesenden hier sehr nützlich sein, wenn ihr mich bloß am Leben lasst“, mahnte Leander mit hoher Stimme.

Das Böse hingegen lachte bloß verzweifelt: „Ich werde dich nicht umbringen. Ich will es nicht. Und ich kann gar nicht. Jarid! Trieest! Wollt ihr nicht wissen, warum dieser Seher euch auf eine Selbstmordmission senden wollte? All seine Geheimnisse können die meinen sein ... und damit auch die euren, wenn ihr es euch nicht mit mir verscherzt.“

Trieest knackte seine Fingerknöchel und sein Rankenschwert vereinte sich mit einem melodischen Klang zu einer einzelnen schnurgeraden Klinge.

„Ich sehe, da ist jemand nicht zum Scherzen aufgelegt, häh? Und du, Jarid? Immer noch wütend wegen deiner Mutter? Ich verstehe es. Ich habe auch lange, viel zu lange emotional an meiner Familie gehängt. Aber sie ist unwichtig. Eines Tages wirst auch du das akzeptieren müssen.“

Jarids Schritte wurden kaum merklich kürzer. Trieests nahmen an Tempo zu.

„Ihr beide könnt mich wirklich nicht in Ruhe lassen, oder? Ach, die Karten sind wirklich gegen mich gerichtet. Gönnt ihr mir etwa meine Freiheit nicht? Ich bin mächtig und werde die Welt mithilfe meiner Macht zum Guten wenden! So lasst uns doch reden ...“

Inzwischen hatten Jarid und Trieest das Böse fast erreicht. Wo versteckte es den schwarzen Kristallsplitter?

Das Böse knurrte aus dem Mund der kleinen Wassermagierin: „Könnt‘s kaum erwarten, mich abzustechen, häh? Kann ich verstehen. Früher war ich geduldiger, doch all die lange Zeit in meinen vielen Kerkern und Gefängnissen, echten wie magischen, machte mich ebenfalls ungeduldig! Also dann, bringen wir’s hinter uns!“

Die kleine Wassermagierin stieß Leander von sich und stürzte sich auf Jarid. Trieest beschleunigte seinen Schritt, Jarid riss ihre Hand nach oben. Ein kleiner Ball aus Wasser bildete sich scheinbar aus dem Nichts, folgte ihr und zerplatzte am Arm der kleinen Wassermagierin. Diesen Trick kannte Trieest bereits. Unterkühltes Wasser. Es gefror sofort und hinterließ üble Verbrennungen.

Das Böse schrie auf, doch jetzt war Trieest bei ihm und warf es zu Boden.

Leander kroch davon.

Es raschelte im Unterholz und eine eigenartige Prozession trat nach vorne. Calrai und die restlichen Skrale waren da, um ihnen zu Hilfe zu kommen! Kurgat starrte besorgt auf das Rauschen der Wellen am Meeresufer und hielt sich zurück. Tran und Bark trugen Arbon auf ihren Schultern herbei, welcher eine behelfsmäßige Schiene um sein Bein trug. Und eine gespannte Arcuballiste auf die kleine Wassermagierin richtete. Sie hob ihre Augenbrauen.

Arbon betätigte seine Balliste in genau dem Moment, in dem die Wassermagierin ihre Hand in seine Richtung öffnete. Ein Bolzen schoss aus der Arcuballiste, ein schwarzer Kristall schnellte aus der Hand der Wassermagierin. Die beiden Geschoße bewegten sich blitzschnell, doch schien es Trieest, als würde die Zeit stillstehen und er könnte ihren schönen Flug mitverfolgen. Sie kreuzten sich in der Mitte, so nahe beieinander, dass zwischen ihnen ein Haar hätte zerrieben werden können. Dann bohrte sich der Armbrustbolzen in die Stirn der kleinen Wassermagierin und der schwarze Kristallsplitter in Arbons Stirn.

Trieest hätte so einiges erwarten können, aber sicherlich nicht, dass das Böse aufschrie. Arbon schwankte, fiel zu Boden und stieß einen kehligen Fluch aus, gefolgt von einem ungläubigen „WAS?! Aber ... das geht doch ... Nein! NEIN!“ Das Böse riss irgendein relevantes Teil aus der Arcuballiste, schleuderte sie zu Boden und warf den schwarzen Kristall an Calrai, welcher verwirrt nebendran stand und ihn instinktiv auffing. Trieest stöhnte auf.

Calrai wiederum fletschte seine Zähne, während seine weißen Augen sich schwarz verfärbten, und gab Fersengold.

Oder besser gesagt, er versuchte es – doch Arbon wetzte herum und langte nach Calrais echsiger Ferse. Calrai klappte zusammen, und da waren Jarid und Trieest schon bei ihm. Calrai sprach aus schwarz geifernden Lippen irgendeinen Fluch in einer Sprache, welche längst hätte vergessen werden sollen.

„Gebt acht, ihr wollt doch nicht Trieests neue Fl ...“, setzte das Böse an.

Arbon unterbrach es, hob seine angeschlagene Arcuballiste demonstrativ in die Höhe und verzerrte sein Gesicht zu einem gehässigen Grinsen: „Das hätte ich an deiner Stelle nicht getan.“

Calrais Augen wurden zu dünnen Schlitzen. Dann breitete das Böse seine Arme aus, heulte den Himmel an und sprach in der Skral-Sprache: „Geister des Feuers, Geister der Erde, ich rufe euch an, auf dass ...“

Trieest wusste genug über Calrais Schamanen-Talent, damit er sich zu fürchten begann. Nebelschwaden zogen wie aus dem Nichts auf und ballten sich um Calrai herum.

Auch Arbon schien plötzlich genug von den Spielchen zu haben. Er richtete sich zu seiner vollen Größe auf, griff in seine Manteltasche und zog etwas hervor. Er öffnete seine Faust und präsentierte... \textit{drei kleine, grau gefärbte Holzwürfel.}\bigskip



\textit{„Wir haben insgesaaaamt ... Jarids sieben Stärkepunkte plus Arbons zwölf sind 19. Plus Trieest mit seinen sieben Stärkepunkten ergibt 26!}

\textit{„Und Jarid beginnt!“}

\textit{„Ich würfle ... eine Eins! Eine Zwei! Und ... nochmals eine Eins! Sorry, Leute, das war nix.“}

\textit{„Nehmen wir da die zwei Einsen oder doch die eine Zwei?“}

\textit{„Die zwei Einsen! Wollen ihren Helm doch nicht verkommen lassen.“}

\textit{„Wir sind bei 28!“}

\textit{„Arbon ist dran!}

\textit{„Na, dann schauen wir mal. Vier, das nehmen wir noch nicht. Eins, das wollen wir ganz und gar nicht.“}

\textit{„Das schaut nicht gut aus.“}

\textit{„Jetzt aber! Sechs, das lassen wir doch sehr gerne liegen!“}

\textit{„Schön gemacht!“}

\textit{„34!“}

\textit{„Hip, Hip, Hurra! Earas Rang ist da!“}

\textit{„Das ist schon über Finster-Calrais Stärkepunkten! Wir könnten tatsächlich eine Chance haben!“}

\textit{„Aber hallo, hat jemand je daran gezweifelt?!“}

\textit{„Jetzt noch Trieest.“}

\textit{„Die Spannung steigt ...“}

\textit{„Und roll!“}

\textit{~ Klacker ~}

\textit{„Sorry, ein Würfel landete schief. Den muss ich wiederholen, die anderen sind Vier und Vier.“}

\textit{„Ach, hätten wir doch Trieest den Helm gegeben.“}

\textit{„Der letzte Würfel war eine Fünf! Und mit Trieest Sonderfähigkeit addieren wir noch die eine Vier dazu!“}

\textit{„Warte, warte, Trieest kann seine Sonderfertigkeit ohne Lavastein doch gar nicht mehr nutzen!“}

\textit{„Auweia, stimmt, da müssen wir uns noch eine neue einfallen lassen. Forn kopieren?“}

\textit{„Wäre langweilig. Halbskrale sind nicht alle gleich, und Trieest Gaben brauchen vielleicht etwas Zeit, um wiederzuerwachen. Ohne Lavastein also eine Fünf von Trieest.“}

\textit{„Ein Kampfwert von 39 insgesamt.“}

\textit{„Wer übernimmt Finster-Calrai?“}

\textit{„Wartet, wartet, wir haben die Unterstützung der Skralsippe vergessen! Mit ihren +3 haben wir sogar einen Kampfwert von 42!“}

\textit{„Das gefällt mir eigentlich besser, als wenn wir mit Trieest einen anderen Würfel hätten addieren können.“}

\textit{„Hat die Skralsippe überhaupt noch Platz? Feld 52 sieht schon recht überfüllt aus für mich.“}

\textit{„Das dürfte knapp werden.“}

\textit{„Platztechnisch?“}

\textit{„Zeittechnisch ebenfalls.“}

\textit{„Ach kommt, sonst hängen wir einfach noch ein paar Kampfrunden an.“}

\textit{„Geht nicht, Trieest ist schon in der letzten Überstunde. Und auf ihn kommt’s ziemlich drauf an.“}

\textit{„Dann muss es halt einfach jetzt klappen.“}

\textit{„Ich übernehme sonst einen von Finster-Calrais Würfeln.“}

\textit{„Und ich den anderen.“}

\textit{„Ich einen dritten.“}

\textit{„Perfekt. 30 Stärkepunkte, weil 3 Helden. Uuuund los geht’s!“}

\textit{„Von mir kommt ... eine Zwei!“}

\textit{„Das beginnt ja schon toll ... eine Zwei von mir!“}

\textit{„Noch ist alles offen ...“}

\textit{„Jetzt einfach keine Sechs, einfach keine Sechs ... eine Fünf!“}

\textit{„Na toll!“}

\textit{„Damit hat das Böse genau 35!“}

\textit{„Sieben Willenspunkte runter, das geht perfekt auf!“}

\textit{„Das reicht! Wir haben es!“}

\textit{„Finster-Calrai ist Geschichte!“}

\textit{„Halleluja! Hurra!“}

\textit{„N-Karte vorlesen! N-Karte vorlesen!“}

\textit{„Geduld, Geduld!“}\bigskip



„Das ist für Mama!“, rief Jarid und schleuderte Finster-Calrai einen letzten Wasserschwall entgegen. Dieser stürzte zu Boden. Der schwarze Kristallsplitter rollte aus seiner Hand. Ehe sich dieser wieder verselbständigen konnte, griff Arbon in seine Manteltasche und stülpte einen hölzernen Würfelbecher darüber. Trieest fragte sich nicht, wozu er diesen bei sich hatte, Arbons Mantel schien allerlei nützliche Sachen zu enthalten. Stattdessen kniete sich Trieest neben Calrai und beobachtete ihn angespannt.

Calrais Augen öffneten sich. Er und Trieest blickten einander an. Ein schwaches Lächeln trat auf Calrais Lippen. Das Weiß seiner Augen blieb weiß.

Jarid atmete schwer ein und aus. Die Skrale blickten sie mit einer Mischung aus Ehrfurcht und Furcht an. Leander hatte sich verkrochen. Arbon kniete noch immer über seinem Würfelbecher und murmelte leise vor sich hin.

„Überlasst den Splitter mir“, sprach er beruhigend, „Ich weiß schon genau, wie ich ihn neutralisieren kann.“

Keiner achtete sich groß auf ihn. Alle hingen ihren eigenen Gedanken nach. Arbon kroch zur kleinen Wassermagierin mit dem Armbrustbolzen im Kopf beugte und bestätigte ein wenig schuldbewusst ihren Tod.

Jarid war es, die schließlich das Wort ergriff: „Alle fit und unverletzt? Wer gönnt sich noch einen Siegestrank am Baum der Lieder? Eine gewisse Bewahrerin hat mir einmal zugeflüstert, dass an diesen Ästen nicht nur Löschfässer gelagert werden. Und die erste Runde geht auf mich! Ich habe‘ noch was gut bei Tion!“

Weder Arbon noch die Skrale oder der Halbskral schienen erpicht darauf.

Jarid revidierte ihren Vorschlag: „Und zunächst gehe nur mal ich vor und erkläre ihnen, dass ihr alle nicht zu fürchten seid.“

Arbon war auf einmal nirgendwo mehr zu sehen. Kurgat, Tran und Bark blickten einander sehr unsicher an.

Calrai hielt Trieest seine Hand entgegen. Trieest zog ihn hoch und gab ihm Heilkräuter zum Unter-die-Zunge-Legen, bis sein Gesicht wieder eine natürliche Farbe annahm. Und dann liefen sie, Halbskral und Skral, fünffingrige Hand in vierfingriger Hand, dem Baum der Lieder entgegen.

Die restliche Sippe folgte ihnen.

Heute würde es am Lagerfeuer äußerst viel zu erzählen geben.

Alles war wieder gut.\bigskip







Beim Baum der Lieder war alles andere als alles gut. Zwar hatten die Feuerkrieger offenbar nach der Trennung des Kristallsplitters rasch die Kontrolle über ihre Lavasteine und Körper zurückerlangt, doch waren manche Knochen gebrochen und manche Lungen voller Wasser. Die Heiler des Baums der Lieder gaben ihr Bestes, so viele wie möglich zu retten, und die gute Larissa arbeitete sich fast ihn eine erschöpfungsbedingte Ohnmacht hinein, doch ließen zwei weitere Feuerkrieger in der folgenden Nacht ihr Leben und eine letzte am nächsten Tag.

Nach ausführlicher Beratung zwischen den Hohepriestern Tion und Gända sowie den überlebenden Danwaren wurden die gestorbenen Feuerkrieger und die kleine Wassermagierin Jivin auf Flößen die Narne hinuntergeschickt, aber auf brennenden Flößen, damit ihre Körper den danwarischen Glauben nach dem Flammenden Gott zurückgeschenkt werden konnten. Jarid sandte kurz darauf ein eigenes, leeres, brennendes Floß die Narne herunter, in Gedenken an ihre Mutter Rowinda. Sie selbst dachte weniger an den Flammenden Gott, sondern mehr an Mutter Natur. Die Priester vom Baum der Lieder hatten abgefärbt. Was Rowinda wohl sagen würde, wenn sie wüsste, dass ihre Tochter nun einer ihr fremden Göttin huldigte? Schuldig flüsterte Jarid die Floskeln des Flammenden Gottes. Trieest legt ihr eine schwere Hand auf die Schulter und sprach die Floskeln nach, auch wenn Jarid wusste, dass sie für ihn ähnlich leer wirkten.

Die legendäre Feuermeisterin Nidwal hatte überlebt. Ihr weiser weißer Rabe wachte argwöhnisch über ihren Schlaf. Mochten ihre Beine sie nicht mehr tragen, ihr Mundwerk war weiterhin ungeschlagen, und ihr Geist wach. So erzählte sie den anderen Genesenden tagein, tagaus Geschichten, Sagen und Legenden von Danwar. Von Heldenmut und der Schönheit in den kleinen Dingen, von Tragödien bis hin zu Witzen, von wahren Begebenheiten bis zu fantastischsten Erfindungen. Gelegentlich versuchte sie sich auch an der einen oder anderen Ballade, doch erkannten die Zuhörer bald, dass Mutter Natur bei allem Talent, das sie in Nidwals Erzählstimme gegossen hatte, ihr Tongefühl ausgelassen hatte.

Die restlichen Danware erfreuten sich an allem, auch einem eher schief tönenden Gesang. Doch nicht nur die Danware fanden darin Erleuchtung. Die folgenden Tage gesellten sich auch immer mehr Bewahrer an Nidwals Bett. Sie lauschten nicht nur, manche schrieben auch eifrig mit, allen voran der Novize Komu.

Als der Oberste Bewahrer Melkart einige Wochen darauf endlich von den Verhandlungen mit dem Einhorn-Clan der wilden Völker des Ostens zurückkehrte, waren die Danware schon längst wieder in den Norden zurückgesegelt. Immerhin war eine ganze Kiste voller frischer Pergamente eigens über danwarische Erzählkultur zum Studieren für Melkart und die übrigen Hohen Bewahrer zurückgeblieben. Die Bewahrer hatten angeboten, dass die Abreisenden einige dieser Papiere zurück nach Danwar mitnehmen konnten, doch diese lehnten dankend ab, mit der Begründung, dass derart entflammbares Material auf einer Insel voller Lavasteine und Feuerkrieger nichts zu suchen hatte. Die auf Danwar gebräuchlichen Steintafeln wären viel weniger anfällig auf allerlei Schadensquellen.

Wie zur Bekräftigung dieser Behauptung hatte einer der Feuerkrieger eine feurig heiße Hand an einen Pergamentstapel gelegt und beinahe ein Regal voller Aufzeichnungen über die Ära des Sternenschilds in Brand gesetzt. Nur das urplötzliche Eingreifen eines Wassergeists hatte ein unkontrolliertes Entflammen verhindern können. Das war Vara, die Verstoßene, gewesen. Sie war im Auftrag der Heldin Kheela von der Gedenkfeier an der Rietburg hierher gesandt worden, um ihre Dienste anzubieten, kaum hatte Jarids Brieflilie Jirid auf der Gedenkfeier erreicht.

Varas Dienste waren zu diesem Zeitpunkt zwar nicht mehr gebraucht worden, doch fand Jarid ungemeine Faszination an diesem Wesen und versuchte, sich damit zu unterhalten, jetzt, wo sie endlich Zeit dafür hatte. Wassergeister waren natürlich nichts Neues in Danwar, und auch mit Vara hatte Jarid bereits Kontakt gehabt, aber noch nie in einem Moment des Friedens, wo man sich nach Lust und Laune auf die Erforschung interessanter Phänomene konzentrieren konnte. Ein Wassergeist, der sich so lange schon außerhalb seines Mediums aufhielt? Einer, der schon so lange dieselbe Form hatte? Einer, an dem (laut Jarid) unverkennbar Spuren danwarischer Magie zu spüren waren? Mit vermutlich so einigem Wissen über mehrere Generationen interessanter Andori in seinem Innern abgespeichert? Nein, das war außergewöhnlich. Und auch wenn Vara sich nicht als sonderlich gesprächig herausstellte, schien Jarid dennoch einiges Interessantes über sie herausfinden zu können. Trieest hatte sie diesem plötzlichen Wissensdurst überlassen. Nach all dem, was vorgefallen war, sollte sie sich gut etwas ablenken. Und er hatte auch seine eigene Geschichte zu erleben.

Denn Trieest selbst verbrachte viel Zeit bei seinen Skralen. Nachdem die Bewahrer sie argwöhnisch und nach einiger Überzeugungsarbeit Jarids am Dorfrand hatten kampieren lassen, wirkte eine Zukunft in andorischer Zivilisation nicht mehr so unmöglich, wie sie einst geklungen hatte. Dennoch sahen sie für sich selbst als nächstes eine Zeit im Gebirge vor sich, und sei es nur, damit Kurgat aufhörte, in furchterfüllten Tönen vom Meer zu schwurbeln. Oder, wenn schon nicht mitten durchs Graue Gebirge, so könnte die kleine Sippe am Rande des Rietlands umherziehen. Irgendwo, wo Trieest sie zu Heldentaten anleiten konnte, ohne dass sie gleich voll unter Menschen leben mussten. Irgendwo, wo sie vielleicht weitere Skrale finden konnten und auf die von Trumm und Drunn erträumten friedlichen Zeiten hinarbeiten konnten.

Jarid wollte sich mit Jirid, Kar und Pyros an der Rietburg über seine Situation austauschen, versprach, sie anschließend jedoch wieder aufzusuchen und zu überlegen, wie es fortan mit ihnen allen weitergehen sollte.

Niemand wusste, was die Zukunft bringen würde. Doch Trieest war zuversichtlich, dass sie von Glück erfüllt sein würde.\bigskip







\textit{Arbon kniff seine brennenden Augen zusammen.}

\textit{„Hast du je daran gedacht, ein fahrender Händler zu werden? Ein bisschen das Land zu sehen? So findest du bestimmt auch viel mehr Kundschaft als in dieser versteckten Hütte!“}

\textit{Naraven winkte husten ab. „Ach nein, ich komme schon so über die Runden. Ich brauche niemanden außer mir selbst, ich bin zufrieden hier allein. Solange ich nicht ernsthaft krank werde, kann ich mich ja auch selbst heilen. Und solange mir kein unbekannter Schrecken im Wald auflauert. Und ich will gar nicht erst daran denken, was geschieht, sobald ich so alt bin, dass meine geistigen Kapazitäten sich verabschieden ... das ist ein Problem für mein zukünftiges Selbst. Na ja, jetzt, wo ich so darüber nachdenke, hätte so ein Dasein als fahrender Reisender vielleicht schon etwas. Wobei dies ganz andere Risiken mit sich bringt. Überfälle von Kreaturen und barbarischen Bergkriegern, Abgaben an die bereisten Reiche ... ich labere wieder. Was habt Ihr Leander erzählt?“}

\textit{„Ein Märchen über eine von einem Fluch besessene Wassermagierin, das er hoffentlich glaubt, oder das ihn zumindest nicht zu weiteren Nachforschungen antreibt“, murmelte Arbon. „Am besten erinnert er sich gar nicht mehr an den kleinen Kristallsplitter. Ich traue ihm nicht mit einer solchen Macht. Womit wir wieder beim Thema wären.“ Er nickte der silbernen Masse im Topf vor ihm zu. „Verrate mir eines, o weiser Naraven: Wie konnte dieser Verstand, dieser Geist, dieser Seele, wie auch immer man es nennen will, wie konnte dies überhaupt in diesem Steinsplitter überdauern? So ein Kristall ist ja ganz ohne das Fleischige und Flüssige, das ein menschlicher Geist zum Überleben in einem Körper bräuchte.“}

\textit{„Magie, mein lieber Hogo, Magie“, sagte Naraven schulterzuckend, „Hast du schon mal einen Feuergeist gefragt, wie er seinen Verstand in einem flüchtigen Körper aus stetig reagierenden Partikeln behält? Es muss definitiv eine Erklärung geben, aber die ist weit außerhalb meines Fachgebiets. Nun, das hier ist eigentlich auch außerhalb meines Fachgebiets, aber hier kann ich wenigstens etwas Alchemistisches beisteuern.“}

\textit{Naraven zog mit einer Zange etwas Zuckendes aus dem Topf und ließ es in eine mit Runen übersäte Klangschale fallen. Die Masse war eine Art Körper. Eine wabernde silberne Masse, die sich weigerte, scharfe Konturen anzunehmen. Nichtsdestotrotz glaubte Arbon, Arme und Beine ausmachen zu können. Aufrecht wäre das Wesen ihm nicht einmal zum Knie gekommen. Aktuell lag es jedoch bloß strampelnd in der Klangschale.}

\textit{„So, das war’s!“, sprach Naraven, „Nicht reinbeißen, die Inhaltsstoffe sind ebenso selten wie ungenießbar. Tinte des Oktohan, Asche eines Takuri, ein wenig Lavastein-Pulver, ich erspare mir die weiteren Details. Auf jeden Fall sollte diese Masse jegliche eingefasste Magie gut abschirmen können. Fünf weitere Goldstücke, und ich setze den Kristallsplitter gleich ein.“}

\textit{„Bitte tu das, so etwas könnte ich ohnehin nicht“, grummelte Arbon. Sorgfältig zog er aus seiner Manteltasche den in dicken Stoff eingeschlossenen finsteren Kristallsplitter hervor. Naraven griff mit spitzen Fingern nach dem Stoffstück, stach den Splitter in die silberne Masse und begann, mit einem grün leuchtenden Stab auf die Klangschale zu hauen. Die darin eingefassten Runen leuchteten schwach auf und summten laut los. Die Masse waberte und wurde einen Augenblick formlos, dann festigte sie sich. Der schwarze Kristallsplitter war nicht mehr zu sehen.}

\textit{Naraven rieb sich die Hände: „So, jetzt ist der Stein eingeschlossen. Solange er in dieser Form bleibt, kann er nichts und niemanden mehr kontrollieren. Dieses Siegel ist mein Meisterwerk. Um das zu brechen, bräuchte man ein Feuer, so heiß wie Drachenodem. Und so eines gibt es seit dem Tod des letzten Drachen nirgendwo in der bekannten Welt. Die Gefahr durch dieses Böse wurde neutralisiert. Und mit ein bisschen Glück lernt es bald, seinen neuen Körper zu steuern. Dann sollte es mit dir kommunizieren können.“}

\textit{Arbon nickte bloß und zückte seinen Geldbeutel.}

\textit{Während Naraven in einen Nebenraum schlüpfte, um irgendwelche letzten stabilisierenden Mittel zu holen, wandte sich Arbon der silbernen Masse vor ihm zu.}

\textit{Leise flüsterte er: „Du hast etwas in meinem Geist gesehen, das du nicht sehen solltest. Geheimnisse aus den Schwarzen Archiven, die kein Verstand lange halten kann, ohne verrückt zu werden. Deine Weltsicht mag auf den Kopf gestellt sein, doch bald schon wirst du all dies wieder vergessen oder rationalisiert haben. Ich weiß, wie wir Menschen so ticken.“}

\textit{Er holte tief Luft.}

\textit{„Nichtsdestotrotz muss ich anmerken, dass du niemandem je auch nur einen Hauch dessen davon erzählen darfst, was du in meinen Erinnerungen erhaschtest. Glaube mir, das würde nicht gut für dich enden ... Hademar aus dem Königshause Brandur.“}

\textit{Eine leise, quiekende Stimme erhob sich aus der blubbernden Masse, die den Kristallsplitter umgab. „Welche Geheimnisse? Wer ist Hademar?“}

\textit{„Hah! Wenn du dich länger dumm stellst, verlässt du diese Hütte nicht mehr. Ich sah deine Erinnerungen, als du die meinen kostetest. Dieser neue Körper ist um einiges beweglicher als dein letzter, doch schirmt er deine Macht ab. Du bist mir ausgeliefert. Dein Wissen der einzige Grund, warum du noch lebst. Falls man deine Existenz überhaupt ein Leben nennen kann. Du kennst die Künste der Krahder und der Hadrier, und du verknüpftest sie beide. Diese Kunst könnte uns Helden sehr nützlich sein, auch wenn die meisten im Orden sie wohl aus Prinzip ablehnen würde. Doch was sie nicht wissen, macht sie nicht heiß. Nun sprich, hast du mir etwas zu erzählen?“}

\textit{Inmitten der silbernen Masse bildete sich ein Mund, der mit urplötzlich tieferer, bedrohlicherer Stimme weitersprach.}

\textit{„Du kennst meinen Namen, doch hast du keine Ahnung, wer ich wahrlich bin. Ich bin das Böse, das leidet und Leiden schafft. Ich habe unter den Krahdern gelitten und ihre Dunkle Hexerei erlernt. Ich habe die verbotenen Schriften der Akademie von Hadria studiert. Drachenmagie aus den Knochen im Grauen Gebirge fließt stetig durch meinen Geist. Ich habe Untote mit einem Fingerschnippen beschworen und den Urtroll mit einem einzigen Blitz vertrieben, und selbst mit einem Bruchteil meiner ehemaligen Macht konnte ich mir den Willen so vieler anderer unterwerfen. Ich war ein Mensch, ein Geist, ein Kristall, ein Monster und eine Armee. Ich trage die Erinnerungen unzähliger Feuerkrieger in mir. Ich reiste durch die Zeit und habe länger gelebt als all deine Großeltern zusammen. Ja, ich habe dir etwas zu erzählen, Arbon, der du von deiner Familie nicht gewollt warst. Und ich habe dir eine Frage zu stellen. Gegeben der, der ich bin ...Wie lange denkst du, du könntest mich hier eingesperrt lassen?“}

\textit{Arbon unterdrückte das Verlangen, nachzufragen, woher sich Hademar anmaßte, seine Großeltern zu kennen. Stattdessen grinste er. Die Identität war bestätigt. Nun musste er diese Nuss nur noch knacken.}




\newpage
\section{Ein Magischer Epilog}

\az{Jahr 72}

Bereits von Weitem war zu erkennen, dass neben der Taverne zum Trunkenen Troll eine mächtige Steppenechse mit einem lächerlich dünnen Seil an einen Pflock angebunden war. Das schien die Echse aber nicht groß zu stören, denn diese ergötzte sich lautstark schmatzend an einem Riesenbüschel getrockneten Rietgrases, das in einem Trog vor ihr aufgeschichtet war. Als wäre das noch nicht genug Beweis gewesen, deuteten zwei riesige, prall gefüllte, leuchtend farbige Stoffsäcke am Tuch auf Sabris Rücken darauf hin, wer sich soeben in der Taverne aufhielt.

Eine aufgeregte Fischerin hatte Trieest kürzlich berichtet, dass eine Untergruppe der Helden von Andor vor wenigen Tagen ein finsteres Ritual an irgendeinem Bannkreis hatte aufhalten könnten – „... und das so nahe an meiner Fischerhütte! Was, wenn die Dorfkinder über diesen Kreis gestolpert wären?!“ – und nun überall Feierlaune herrschte.

„Sabri!“, rief Jarid fröhlich und rannte auf die Steppenechse zu. Diese drehte nicht einmal ihren Kopf, sondern stocherte weiterhin stur mit ihren unterarmlangen Riesenzähnen im Stroh herum. Jarid klopfte dem Tier energisch an die Seite und begann, Sabris ledrige Haut an einer ganz bestimmten Stelle hinter einem Ohr zu kratzen. Ein wohliges dumpfes Brummen drang aus dem dicken Echsenhals. Mehr konnte Jarid ihr jedoch nicht entlocken, dafür war sie zu stark mit ihrem Futter beschäftigt.

Trieest grinste bloß und wanderte an ihr vorbei ins Innere der guten Stube.

Die Taverne war wieder einmal platschvoll.

Vor dem großen Kamin lag wie so oft eine Schlafende Katze und wärmte ihr weiches Fell. Im Kamin selbst hatte sich ein putziger kleiner Feuertakuri niedergelassen. Hin und wieder scharrte der golden leuchtende Vogel mit seinen Flügeln ein wenig in der Asche, woraufhin Glut aufstob.

Na, wenn Turr hier war, war seine Hüterin vermutlich nicht allzu weit entfernt. Ein Blick zwei Tische weiter links verriet, wo sie sich aufhielt. Eigentlich hätte man nur auf ihre Stimme hören müssen, denn die stach schon etwas angeschwipst über den Grundlärm in der Taverne hinaus:

„... ein Dutzend von Irils Runensteinchen nimmst, dann kannst du das Dutzend zum Beispiel als einzelne Linie anordnen, oder als zwei Sechserreihen, oder als drei Viererreihen, und das geht alles perfekt auf. Aber, sieh her, wenn ich nur einen einzelnen weiteren Stein dazu nehme, dann werden es dreizehn, und die Dreizehn kannst du nur als einzelne Linie darstellen, nicht als Rechteck mehrerer Linien gleicher Länge. Darum nennen wir die Dreizehn eine Linienzahl.“

Der Blick auf die Tulgori wurde von einigen vorbeistürmenden Tavernengästen verdeckt, doch das Geräusch von polternden Steinchen war weiterhin zu vernehmen.

„Was, wenn ich die Dreizehn in zwei Siebnerreihen aufteile, und diese Rietgrasblüte hier markiert die Null?“, fragte eine klirrend kalte, aber nicht unfreundliche Stimme.

„Das geht doch nicht, dann haben die beiden Reihen ja nicht gleich viele Steine darin!“, protestierte Aćh lautstark.

„Nur wenn du die Null nicht mitzählst“, erwiderte ihr Gegenüber ruhig. Trieest erkannte ihn nicht zuletzt daran, dass sein Sitzplatz über und über mit Eiskristallen bedeckt war. Offenbar hatte man seinen Stuhl (eventuell auch aus genau diesem Grund) ganz in der Ecke des Raumes platziert. Überraschend energisch zeigte Ijsdur mit seinem Zeigefinder auf eine Anordnung von kleinen farbigen Steinchen auf der Tischoberfläche und sprach demonstrativ: „Schau her. Null, Eins, Zwei, Drei, Vier, Fünf, Sechs, jetzt springen wir in die zweite Reihe und sehen Sieben, ...“

Ein kleiner Schauer glitzernden Reifs verteilte sich über die Steinchen und Ijsdur hätte zweifelsohne bis Dreizehn weitergezählt, wenn ich Aćh ihm nicht ins Wort gefallen wäre:

„Wenn wir immer bei der Null zu zählen beginnen würden, würde sich auch so viel anderes ändern. Doch bei diesem Problem, so, wie es festgelegt wurde, beginnen wir immer bei der Eins. Die Zahlen, die wir so nur als Linie darstellen können, die nennen wir Linienzahlen, weil wir sie nur als einzelne Linie darstellen können, und das ergibt Sinn. Dreizehn ist einfach eine Linienzahl, und wenn du das anders siehst, verstehst du einfach nicht, was ich mit diesem Begriff meine.“

Breit grinsend gönnte sie sich einen weiteren Schluck aus ihrem großen Metkrug.

Ijsdur blieb still und kratzte sich nachdenklich an seinem Geweih, woraufhin noch mehr Eiskristalle zu Boden rieselten.

„Was ist mit der Eins?“, warf eine melodische Stimme ein, „Um den Ava rum gibt zwar es nicht viele, die sich mit dem Zählen um des Zählens willen befassen, aber ich kannte da mal eine alte Totemschnitzerin, welcher lieber den lieben langen Tag lang über solche Sachen nachgedacht hätte, statt praktische Anwendungen dafür zu finden. Wenn man sie fragte, hätte sie dir erzählt, dass man das Wirken der Götter eher in solchen Überlegungen finden könnte statt bei einem Blick in die Wunder der wahren Welt. Was für ein Mensch ...“

„Du schweifst wieder ab, Barz.“

„Verzeih mir. Mein Mundwerk mag manchmal murmeln, bis es meinen Geist abgehängt hat. Wenn ich mich richtig erinnere, hatte diese Schnitzerin auch schon einmal von solchen Zahlen berichtet, aber sie gab ihnen einen anderen Namen und sie sagte, dass die Eins nicht zu denen gehörte. Dabei ist die Eins doch eine Linienzahl!“

Während Barz‘ Mund laberte, falteten seine Hände aus einem Stück Pergament eine Faltfigur. Ein Vogel, eventuell gar ein Takuri? Ehe Trieest genauer hingucken konnte, hatte Barz sein Werk auch schon wieder entfaltet.

„Ein einzelner Stein ist doch keine Linie!“, meldete sich Aćh wieder zu Wort. „Die kleinste Linienzahl muss die Zwei sein. Und von dort geht’s weiter mit Zwei, Drei, Fünf, Sieben ...“

Vom Kamin her ließ ihr Feuertakuri Turr einen freudigen Schrei aufklingen. Er sprang auf und verteilte Asche im Schankraum.

„Der hat auch schon genug von diesem Gelaber“, lachte eine kratzige Stimme, „Darf ich meine Runensteinchen nun wieder einpacken?“

„Tu, was du nicht lassen kannst“, sprach Aćh. Einiges Ruckeln war zu hören, als eine Zwergin sich von ihrem extra hohen Stuhl aufrichtete und über den Tisch langte, um einige farblose Steinchen mit eingeritzten Runen zurück in ihre zugehörigen Löcher auf einer über und über mit Runenmustern verzierten Metallscheibe zu befördern. Währenddessen zückte Aćh eine grobe Steinflöte und spielte eine kurze fröhliche Melodie. Der Takuri, der darauf und daran gewesen war, aus dem Kamin zu stürmen, legte seinen Kopf schief und trampelte noch einige Male trotzig auf der Asche herum, ließ sich dann aber wieder fügsam ins Kaminfeuer sinken und wälzte sich darin.

„Aber eine Eins ist doch eine Linie, also sollte eine Eins auch eine Linienzahl sein“, brachte Barz kopfschüttelnd das Gespräch wieder ins Rollen.

„Du brauchst mindestens zwei Punkte für eine Linie“, hielt Iril nun im Einpacken der Runensteine ein. „Erst wenn ich zwei Löcher in den Boden meißle, ist klar, welche Linie ich dazwischen gezogen haben will.“

„Die Richtung der Linie ist doch durch den Tisch und die Rillen im Holz bereits klar“, erklärte Barz. „Du würdest sie doch nicht plötzlich schief drauflegen, das würde sich falsch anfühlen.“

Entschieden legte Iril einen ihrer grauen Steine zurück auf den Tisch und stieß ihn an. Dieser drehte fröhlich für einige Augenblicke, ehe er flach auf dem Tisch landete.

„In welche Richtung zeigt dieser Stein?“, fragte Iril demonstrativ.

Barz antizipierte ihren Punkt und sprach: „Solange kein zweiter Stein da liegt, \textit{vermutlich} in Richtung der Tischrillen.“

„Vermutlich?“

„Man weiß doch nie so ganz, was gilt und was geschehen wird. Aber wenn jemand den ersten Stein dort hinlegt, steht zu vermuten, dass der nächste Stein entlang der Tischrille platziert werden wird. Sobald der zweite Stein dann hingelegt wird, werden wir sehen, ob die Vermutung richtig war. So funktionieren Experimente. Aber du darfst nicht im Voraus wissen, was ich vermute, sonst kannst du das Ergebnis verfälschen.“

„Das hier ist auch kein Experiment, sondern ein Dialog“, meinte Iril, „Zwerge und Menschen handeln im Gegensatz zu Steinen unberechenbar.“

„Nicht unberechenbarer als eine Pulvermischung, die man noch nicht zur Gänze versteht. Also eigentlich jede Pulvermischung.“

„Ach komm, Barz, im Gegensatz zu einem Menschen kannst du bei einer Pulvermischung irgendwann alle Geheimnisse entschlüsselt haben. Einen Kulturschaffenden kannst du nie komplett durchschauen.“

„Ich glaube, du verwechselst die Natur meiner Pulvermischungen mit der deiner Runenmagie. In deiner idealisierten Vorstellung der Runen magst du alles, was es über ein gewisses Symbol zu wissen gibt, verstanden haben. Aber in der realen Welt, wo du deine Runen ungelenk in Metall eingravierst oder auf deine Haut stempelst, kannst du auch die nie zur Gänze verstehen. Musst du auch nicht. Aber letzten Endes unterscheidet in Bezug auf die Unmöglichkeit des Verstehens nichts, aber auch gar nichts einen Menschen oder einen Zwerg von einem Stein, oder einer eingeritzten Rune, oder einem magischen Pulver.“

Iril legte ihre Stirn in Falten, wohl um Barz‘ viele Worte Revue passieren zu lassen.

Dies gab Aćh die Gelegenheit, sich wieder einzumischen: „Ach nein? Warum kann dann jeder Mensch meine Musik verstehen, aber kein einziger Kieselstein?“

Aćh zog ihre tulgorische Steinflöte hervor und brachte eine weitere wilde Melodie zustande. Ein paar Tische weiter drüben johlte eine Runde laut auf.

„Woher willst du wissen, dass kein Kieselstein ...“

„Ich hab’s!“, unterbrach Ijsdur die Runde, „Ich hab’s endlich! Wir müssen die dreizehn Runensteine nicht so quadratisch hinlegen, sondern in einem Dreiecksmuster: In die erste Zeile einen, in die zweite zwei, in die dritte drei, in die vierte vier und in die fünfte wieder drei. Dann können wir die Dreizehn anordnen, ohne die Null mitzuzählen. In dieser Dreiecksform ist ohnehin weniger unnötiger Platz zwischen den Steinchen als in quadratischen Reihen.“

„Das ist eine schöne Form ...“, „... aber das ändert nichts daran, dass man die Dreizehn ...“, „... nicht in gleich lange Linien aufteilen kann!“, widersprachen Iril und Aćh gleichzeitig.

„Warum sollte das überhaupt irgendjemand wollen? Ist das wieder einfach so eine Eigenheit der hiesigen Zivilisation, genau wie ...“

„Wenn du dich noch einmal darüber beschwerst, dass du hierzulande seltsam angeguckt wirst, wenn du nur mit einer Toga rumläufst, erzähle ich dir mal etwas über die komplizierten Kleidervorschriften der Runenmeister der Silberländler“, murmelte Iril.

„Und ohnehin“, ergänzte Barz, „Sind solche Kleidervorschriften viel weniger willkürlich als diese Steinchen- und Zahlen-Rätsel. Letztere dienen nur zum Vergnügen, erstere haben hingegen viele Anwendungen. Kleidung kann Status und Eigenheiten des Gegenübers kommunizieren, kann dich etwa gegen Verletzungen oder Kälte schützen ... oh. Vergiss den letzten Punkt wieder.“, unterbrach sich Barz.

Ijsdur setzte zu einer Erwiderung an, da ...

„Tri“, drang Jarids glockenhelle Stimme an sein Ohr, „Wie lange willst du noch auf den fremden Tisch starren? Du könntest dich einfach zu ihnen setzen.“

„Ah, nein, ich will sie nicht stören. Und mir ist auch nicht zwingend nach Gelaber“, sprach Trieest, „Ich war nur abgelenkt. Hauptsächlich geht es mir jetzt darum, diese Bestellung abzuholen, zur Sippe zurückzukehren und dann den restlichen Tag in Ruhe genießen.“

„Da bin ich ganz bei dir. Wenn nur ...“

Jarid erstarrte und sprach: „Uh-oh. Ich bin mir nicht so sicher, ob wir so viel Ruhe kriegen können, wie wir gerne hätten. Guck mal dort rüber, aber möglichst unauffällig.“

Trieest folgte Jarids Zeigefinger und schnappte nach Luft, als er erkannte, auf wen Jarid zeigte

„Ist das etwa ...“

Jarid nickte grimmig. „Das ist er. Leander, der lügnerische Seher. Sitzt hier im Trunkenen Troll, unterhält sich mit Garz und genießt den Abend, als hätte er nichts Mieses angestellt. Haben wir mit ihm noch ein Hühnchen zu rupfen?“

„Nein, ich will nur nicht ...“ Urplötzlich trat eine Erinnerung vor Trieests inneres Auge. Eine dunkelblaue Frau mit langem, weißem Haar, welche vor einer Feuerwand schwebte. Er korrigierte sich. „Doch, eigentlich habe ich noch etwas, was ich ihm ausrichten soll. Pass einfach auf, dass wir nicht zu lange bei ihm verbleiben.“

Trieests Blick schweifte zurück zu den Magischen Helden am anderen Tisch. Soeben wurden neue Getränke serviert und Barz demonstrierte theatralisch, dass er Met mit dem passenden Pülverchen in Wasser verwandeln konnte. Große Begeisterung kam nicht auf, aber zumindest Ijsdur erfreute sich daran. Er langte nach Barz‘ Bogen, doch dieser hielt ihn auf. „Bitte nicht, Ijsdur, beim letzten Mal vereiste die Sehne und zerbrach. Verzeih, du bist manchmal einfach nicht am geschicktesten mit großen wärmeempfindlichen Gegenständen.“

Trieest blendete die Magischen Helden aus und näherte sich Leander. Jarid folgte ihm neugierig. Das Gesicht des Sehers drehte sich in seine Richtung, also konnte er ihn sehr wohl ankommen hören. Seine Miene blieb aber gelangweilt ausdruckslos, offenbar hatte er ihn diesmal nicht erkannt. Lag das daran, dass er nicht mehr einen Lavastein in seiner Brust trug. Hatte Leander damals vor einigen Jahren dessen Präsenz gespürt?

Gildas fröhliche Stimme donnerte über den Gastraum hinweg.

„Neue Runde für alle Helden! Ijsdur, trink’s schnell aus, ehe es dir wieder einfriert! Moin, Trieest, keine Getränke für deine Jungs?“

Trieests Jungs waren inzwischen in der Anzahl auf das Doppelte angewachsen. Sogar ein alter weiser Schamane namens Obron hatte sich ihnen angeschlossen und war daran, Calrai als seinen Nachfolger auszubilden. Doch fühlten sie sich im Rietland immer noch nicht so wohl, erst recht nach einer üblen Konfrontation mit einem Trupp Soldaten des Königs. Nicht, dass die anderen Skral-Sippen besser auf sie zu sprechen gewesen wären. Aber alles zu seiner Zeit.

Trieest nickte der Gastwirtin zu. „Nein, meine Jungs wollen lieber draußen bleiben. Auch wenn ihr ihnen noch so oft sagt, sie mögen hier willkommen sein. Ich mein, ich kann’s verstehen. Kurgat kann sogar schon von einem persönlichen Gefecht mit Barz erzählen, als er diesem früher im Grauen Gebirge aufgelauert war ... wobei das weniger ein Gefecht war und mehr ein benommenes In-Andor-Aufwachen. Ich schweife ab.“

Nun, wo auch Leander Trieests Identität vernommen haben musste, fluchte der Seher unter seinem Atem auf.

„Abend, Trieest“, murmelte er, „Dann ist Jarid wohl auch nicht weit entfernt?“

„Abend, Leander“, erwiderte Jarid. Niemand von ihnen nutzte die danwarischen Höflichkeitsfloskeln.

„Abend, ihr beiden“, sprach der Handelszwerg Garz fröhlich, als wäre ihm völlig entgangen, wie eisig die Stimmung zwischen dem Seher und den Danwaren war. „Gilda sagte mir, ihr wärt dabei gewesen, als die Weinhändlerin Lysbett starb?“ Mehr zu sich selbst murmelte er: „Sie war doch noch so jung, die Lysbett.“ Dann guckte er auf und fuhr in seinem typisch hämischen Ton fort: „Die Kleine ist mir eine Zeit lang umhergefolgt, wisst ihr? Ich nahm sie mit auf meinen Reisen, weil so ein kleines Kerlchen einfach mehr Sympathie unter den Leuten weckt als ein armseliger geiziger Handelszwerg, wie ich es bin. Hat so einiges von mir gelernt, hat sie, häh? Aber gut, früher oder später musste sie diese Welt verlassen, und weniger Konkurrenz ist immer gut fürs Geschäft! Weißt du zufälligerweise, wo sie ihre Drachenfässer lagerte? Diese momentane Marktlücke gedenke ich rasch und kompetent zu füllen!“

Einzig Garz‘ Augen, welche viel stumpfer als sonst dreinblickten, verrieten den wahren Gemütszustand des Handelszwergs. Trieest winkte ab.

Der Seher lehnte sich zurück, zog ein Stück Holz und ein Messer aus seiner Tasche und begann, desinteressiert zu schnitzen. Obwohl er eine Augenbinde trug, verfehlte sein Messer kein einziges Mal sein Ziel.

Trieest schluckte all seine Wut auf Leander herunter und überlegte, wie er Garz vom Tisch vertreiben konnte. Jarid war nicht so feinfühlig.

„Was war eigentlich dein Problem?!“, herrschte sie Leander an. „Du hättest uns beinahe auf eine mysteriöse Insel gelockt, auf der uns wer weiß was erwartet hätte! Du hast Trieests Zukunft gesehen und dennoch beschlossen, ihm eine falsche Hoffnung zu schenken!“

„Du etwa nicht? Hast du ihm nicht ebenso die ganze Zeit weisgemacht, er könne diese Bürde durch gute Taten loswerden?“, gab der Seher unwirsch zurück.

Garz machte große Augen. „Was geht denn hier ab? Alter Leander, ich hab‘ dich nicht mehr so mürrisch gesehen, seitdem der olle Leam dich abwies. Deine Angelegenheit gehen mich natürlich nichts an, aber Geschäfte laufen einfach besser, wenn man seine Karten offen auf den Tisch legt, weißt du? Willst du mir erzählen ...“

„Halts Maul, Handelszwerg!“, zischte Leander grimmig. Dann atmete er einmal tief durch und sprach: „Es stimmt. Ich verspüre eine große Wut auf dich, Jarid. Du hast mit deinen Worten mehr Schaden angerichtet, als du denken kannst. Aber es bringt niemandem etwas, diese Lage von vorne aufzurollen. Schließlich wusstest du nicht, was du tatst.“

„Du wusstest hingegen sehr wohl, was du tatst! Ein Lügner und Betrüger der schlimmsten Sorte bist du! Warum bist du nicht einfach so auf uns zugekommen und hast uns um Hilfe gebeten?“

„Ich wusste nicht, ob ich euch überzeugen könnte mit der Wahrheit. Mit der Lüge wäre es mir problemlos gelungen. Hättest du nur ein klein wenig länger in diesem Keller nach Wasserfässern gesucht, so hätte ich Trieest davon überzeugt, dir nichts von meinen Worten erzählen, und ihr beide wärt fröhlich nach Narkon aufgebrochen.“

„Und dort ... gestorben? Verflucht worden? Versklavt worden? Was wolltest du auf Narkon erreichen, Leander?!“

„Das ist meine eigene Sache. Ich weiß nicht einmal, warum ich mich vor dir zu rechtfertigen versuche. Ich will bloß noch sagen, dass ihr diese Insel beide wieder gesund und munter verlassen hättet ... und auf dem Weg einige andere Leben gerettet hättet. Diese sind nun höchstwahrscheinlich verloren!“

„Das bringt doch jetzt nichts mehr“, begann Trieest, „Verzeih, Garz, aber könnten wir dich dazu einladen, eine Runde auf unsere Kosten zu trinken? Ein wenig abseits dieses Tisches? Es gibt da etwas ...“

„Ja, ja, auf dass der Handelszwerg ja nichts Interessantes mitkriegt und an die falschen Ohren verkauft“, grummelte Garz. Dann spazierte er jedoch folgsam davon.

„Und glaube ja nicht, dass nur Jarid auf dich wütend wäre“, grummelte Trieest, „Doch schlucke ich dies herunter, weil ich glaube, eine wichtige Botschaft für dich zu haben. Und wer weiß, vielleicht änderst du dich eines Tages noch.“

Leander schnaubte spöttisch: „Wenn ich ein Goldstück hätte für jede Person, die ich in meinem langen Leben belogen habe, hätte ich genug Geld, um ein königliches Kopfgeld auszusetzen. Ihr beide seid nichts Besonderes. Ihr werdet mich nicht ändern.“

„Nicht, solange du uns nicht lässt. Aber das ist auch nicht unser Ziel.“

„Was ist denn Euer Ziel?“

Jarid blickte ihn aufrichtig interessiert an: „Ja, Trieest, was ist unser Ziel eigentlich? Was hast du mit ihm vor?“

Trieest atmete tief durch und meinte dann: „Ich fühle, ich soll dir noch etwas ausrichten. Ich glaube ...“ Trieest Stimme wurde unsicher und leiser. „Leander, ich glaube, ich habe deine Mutter getroffen. Sie richtet aus, dass sie dir verzeiht, für deine vergangenen und deine zukünftigen Taten. Dir und deinem Bruder. Ich soll dir sagen, dass du dich nicht so sehr auf ihn fokussieren musst. Deine Pläne sind unnötig. Du wirst ihn auch so wiedersehen, frei und unverflucht. Es besteht keinen Grund, weiteres Leid anzurichten. Also ... ziemlich sicher jedenfalls. Wer kann die Zukunft schon so genau wissen?“

Leanders Kopf drehte sich ruckartig und seine leeren Augen starrten Trieest andächtig an. Mit bebender Stimme antwortete er schlussendlich: „Das erscheint mir höchst unwahrscheinlich. Meine Mutter ist schon seit vielen Jahrzehnten tot. Ein namenloser Grabstein in Werftheim zeugt davon.“

„Die Wahrheit meiner Worte musst du selbst beurteilen.“

„Dann beurteile ich sie als unwahr!“

„Urteile nicht zu hastig, du selbstgerechter Verfechter der Logik. Die letzten Augenblicke eines Toten hängen manchmal noch lange am Todesort. Und Echos von Toten können in gewissen Steinen noch lange nach ihrem Ableben kurz nachhallen, das müsstest du wissen. Du blaue Blüte in der Dunkelheit. Bitte, richte nicht noch mehr Pein an, als du bereits hast.“

Leanders Mundwinkel zuckten, als er diese Worte vernahm. Ansonsten regte er sich nicht. Ein nasser Fleck breitete sich unter seiner Augenbinde aus.

Und Jarid und Trieest ließen ihn allein sitzen, warfen den Magischen Helden einen letzten Blick zu und begaben sich dann nach draußen, wo ihre Skrale bereits auf sie warteten.











\begin{chapterbox}
    \chapter{Qurun (2021)}
    \label{Qurun (2021)}
    \az{Jahr 71}
    
    Der Seher Leander wird von der Stammesältesten der Agren ins Graue Gebirge berufen. Ihr mysteriöser Auftrag schickt gleich mehrere mächtige Entitäten auf Kollisionskurs. Sie alle sind auf der Suche nach etwas. Nach einem alten Kameraden. Nach Freiheit. Nach Seelennahrung. Nach einem Platz in dieser Welt. Nach einer Möglichkeit, einen Rivalen zu beseitigen. Und... nach dem perfekten Kessel aus Zwergenstahl?
\end{chapterbox}


\section{Prolog}

\az{Jahr 71}

Der Raum war kreisrund und leer bis auf einen Altar in seiner Mitte und eine riesige Gestalt am gegenüberliegenden Ende des Raums. Dort lag ein versteinerter Drache, die Überreste des einst so edlen Kardòl, den von Rissen durchzogenen Kopf in seinen stacheligen Schwanz verbissen. Aber ich war nicht seinetwegen gekommen.

Rotes Mondlicht fiel vom Himmel durch einen kleinen Riss in der Höhlendecke auf den kalten Fels. Es beleuchtete den Altar ebenso wie den weißen Edelstein, den ich soeben darauf gelegt hatte. Ich konnte den Raum und die Farben natürlich nicht sehen, aber ich hatte in den letzten Tagen genug oft davon geträumt, um sie mir gut vorstellen zu können.

Dieser weiße Edelstein war außergewöhnlich. „Drachenherz“ hatten die Zwerge ihn genannt und ihn aufgrund seiner Größe für etwas derart Besonderes gehalten, dass sie mich zweifelsohne für den Rest meines Lebens in den Tiefminen schuften lassen würden, sollten sie je herausfinden, dass ich ihn mir angeeignet hatte. Narren waren sie! Versehen mit den richtigen Zwergenrunen, besaß der Kristall vor mir nun viel interessantere Eigenschaften als seine Reinheit und Größe. Eigenschaften magischer Natur. Selbst jemand wie ich, der noch nie ein Talent für Zaubersprüche und mächtigeren Hokuspokus gezeigt hatte, konnte diese Eigenschaften zu seinem Vorteil nutzen.

Ich fühlte die Präsenz des Bösen, noch ehe es seinen Mund öffnete. Meine Nackenhaare stellten sich auf und ein Schauer lief über meinen gesamten Körper. Ich versuchte, mir nichts anmerken zu lassen, rückte meine Augenbinde zurecht und schritt langsam auf den Altar zu.

„Du hast mich gerufen?“, sprach eine tiefe, heisere Stimme aus den Tiefen der Erde. Wenn ich mich nicht irrte, waren das Worte aus der dunklen Sprache Krahds. Es gibt keine Sprache auf dieser Welt, derer ich nicht zumindest teilweise mächtig wäre. So stellte es für mich kein Problem dar, das Böse zu verstehen. Dennoch antwortete ich nicht und schaute verwirrt drein. Zu zeigen, dass ich die Sprache der Krahder beherrschte, würde nur unnötig Alarmglocken läuten.

„Wer hat mich gerufen?“, wiederholte die tiefe Stimme nun in der Sprache der Andori, mit leicht gebrochenem Akzent. Es schien, als würde der Urheber der Stimme die Sprache zwar kennen, sie aber zutiefst verabscheuen. Und das schien natürlich so, weil es so war.

Andächtig antwortete ich: „Es ist also wahr! Nicht alle Geister der Gefallenen ziehen weiter. Ihr seid der lebendige Beweis für meine These! Nun, so lebendig, wie es den Umständen entsprechend geht. Gepriesen sei Mutter Natur!“

Die tiefe Stimme gluckste auf und ich grinste innerlich. Das Böse schien mir die Rolle des begeisterten Gelehrten abzukaufen.

„Hoffentlich bin ich nicht umsonst in diese Kammer tief ins Graue Gebirge gestiegen. Was willst du von mir, du blauer Wicht?“

Äußerlich ließ ich mich verunsichern, innerlich triumphierte ich. Stockend ließ ich die Worte meinen Mund verlassen: „Ich... ich dachte... nun... wenn ihr es geschafft habt, Gevatter Tod ein Schnippchen zu schlagen... vielleicht... vielleicht könntet ihr mich lehren? Es mir beibringen?“

Ich verstummte. Die Kunst, den Tod zu überdauern, ist eine äußerst nützliche Fähigkeit, die ich nur allzu gerne selbst besäße. Ich hatte meine Lebensdauer bereits mit Tränkchen und Süppchen aus der gesamten bekannten Welt verlängert, dennoch nahmen mit jedem Jahr die Falten in meinem Gesicht zu. Noch. Aber ich war nicht in Eile. Ich hatte Zeit. Sobald das Böse erst einmal in meiner Gewalt wäre, würde ich sein Wissen langsam, aber sicher aus ihm herauspressen können.

Das Böse lachte krächzend auf: „Hast du das Zeug, den Tod zu überlisten? Bist du bereit, all das, was du besitzt, dafür aufzugeben? Bist du bereit, jede mögliche Grenze zu überschreiten, die man überschreiten könnte?“

„Ja!“, antwortete ich mit gespielt zitternder Stimme, „Jede Grenze außer der zum Reich des Todes!“

„Guuuut“, zischte die Stimme des Bösen. „Dann habe ich eine Aufgabe für dich. Tritt ins Mondlicht, mein neuer Freund.“

Die Luft wurde kaum merklich wärmer, als ich ins Licht des roten Mondes trat und auf den Blutsteinaltar kletterte. Eine Spannung lag in der Luft und der Geruch nach Schweiß und Blut drang in meine Nase.

„Was nun?“

„Es ist ganz einfach. Alles, was du tun musst... ist sterben!“

Ein Windstoß fuhr durch die Höhle und wehte mich beinahe vom Altar. Natürlich hatte das Böse nicht vorgehabt, mir das Geheimnis des ewigen Lebens zu verraten. Ich spürte, wie es versuchte, von meinem Körper Besitz zu ergreifen, wie sein Geist in den meinen schlüpfte... und spürte, dass etwas nicht in Ordnung war, als es statt der erwarteten Todesangst kalte Entschlossenheit vorfand.

Schnell jetzt! Ich ließ mich zu Boden sinken und umfasste den weißen Kristall fest mit beiden Händen. Wer weiß, was geschehen wäre, wenn ich ihn verfehlt hätte. Das Drachenherz nach oben haltend, schrie ich zwei Sätze in der Alten Sprache.

Für einen kurzen Augenblick geschah nichts und ich fürchtete schon, die Kontrolle über meinen Körper verloren zu haben. Dann würde es nicht mehr lange dauern, bis das Böse meinen Geist ausgelöscht hätte. Doch da wurde mir das Böse auch bereits wieder aus den Adern gesaugt. Der Windstoß in der Höhle versiegte und der weiße Kristall in meinen Fingern wurde abrupt eiskalt.

Ich atmete tief durch. Es war mir gelungen! Ich hatte ihn gefangen!

Nun galt es nur noch herausausfinden, wie lange er eingesperrt bleiben musste, ehe er mir mitteilen würde, was ich wissen wollte.





\newpage
\section{Der hellsichtige Geisterjäger}

„Tut mir schrecklich leid, dass du dich meinetwegen so abmühen musst“, entschuldigte sich Leander, „Es war bestimmt unangenehm für dich, so weit in den Norden vorzudringen.“

„Macht Ihr Witze?“, entgegnete Jorn, der Agren, begeistert, „Es war grandios, den sagenhaften Baum der Lieder und die Weiten des Rietgrases mit eigenen Augen zu sehen. Könnt Ihr Euch das vorstellen – Unterstände oberhalb der Erde zusammenzubasteln statt in den Gängen unter der Erde zu wohnen? Ach, was erzähle ich da. Natürlich könnt Ihr Euch das vorstellen, wohnt Ihr doch selbst in so einer Hütte. Habt Ihr denn nie Angst, dass die über Euch zusammenbricht? Das wäre ein ziemlich ironisches Ende, meint Ihr nicht auch? Vorsicht, Stolperstein.“

Leander sandte ein Stoßgebet an Mutter Natur, dass sie ihn vor dem Geschwafel seines Begleiters retten möge. Vorsichtig tastete er mit seinen Füßen den großen Stein vor sich aus und suchte Halt. Dann schwang er sich darüber. Etwas knirschte unter seinen Füßen.

Jorn zog scharf Luft ein: „Auweia, ist das etwa... nun, er oder sie ist jedenfalls schon lange tot, da müsst Ihr euch keine Sorgen machen. Vielleicht ein halber Krieger aus dem Unterirdischen Krieg? Ich glaube, in diesem Helm einen mächtigen Schild der Schildzwerge eingraviert zu erkennen. Habt Ihr Euren Fuß vertreten? Könnt Ihr weiterlaufen?“

Leander grummelte etwas, tastete nach Jorns Hand, ergriff sie und lief rasch weiter.

„Nicht so schnell, sonst zieht Ihr uns noch beide in einen Abgrund“, warnte Jorn, „Verzeiht, wir Agren benutzen die Alte Zwergenstraße so selten, dass sie kaum instand gehalten wird. Wir hätten auch gar nicht die Mittel dazu. Kreaturen benutzen sie viel häufiger. Wer nicht gerne zu Wargorfutter wird, hält sich meistens von hier fern.“

Jorn schwieg für einen kurzen Augenblick. Leander vermutete, dass er sich gerade nervös nach Wargors umsah. Leider konnte Leander das Rauschen der Bäume und die Musik der Toten in den Schluchten des Grauen Gebirges nur kurz genießen, ehe sich Jorns Mund erneut öffnete.

„Nicht, dass wir nicht auch mit einem Wargor klarkommen könnten. Also, ich ja nicht. Aber Ihr besitzt ein wenig Kampfgeschick, oder? Ich habe euch mit dem Messer umgehen sehen. Schöne Verzierungen. So etwas finden wir höchstens noch als verrosteten Überrest aus den uralten Kriegen. Ein wenig mehr nach links, Ihr kommt vom Weg ab.“

Leander schwieg, wie er es schon für den Großteil des Wegs getan hatte. Er strich mit seinem Stab über die schön angeordneten Steine, aus denen die Alte Zwergenstraße aufgebaut war. Hoffentlich waren sie bald am Ziel.\bigskip







„Da wären wir, o Seher. Es war mir eine Ehre, Euch bis hierhin zu begleiten. Auch wenn Ihr nicht der gesprächigste wart, habe ich Eure Gesellschaft außerordentlich genossen“, plapperte Jorn. Leander verzerrte seinen Mund zu einem Grinsen und murmelte ein gequältes „Gleichfalls“, ehe er Jorn losließ und mit seinem Stock nach der nächstgelegenen Wand klopfte. Der Klang war dumpf. Ein mit Kräutern und Moos überwachsener Eingang zu einer kleinen Agrenhöhle.

Vorsichtig tastete sich Leander tiefer in die Höhle hinein. Selbst mit gesenktem Kopf stieß der großgewachsene Narkonier immer wieder gegen die erdige Höhlendecke. Warme Luft wehte ihm entgegen und umwehte seine abgemagerten Wangen. So stolperte er weiter auf die Wärmequelle zu, wo er Holz knacken hörte. Ein Lagerfeuer. Dem Geruch nach wurden gerade getrocknete Kräuter geräuchert. Den Geräuschen (oder besser dem fehlenden Geplapper) entnahm Leander, dass sich höchstens eine andere Person hier befand. Er zog tief Luft ein. Der bittere Geruch von Sternkraut trat in seine Nase und ließ ihn aufhusten. Die Ohren gespitzt, glaubte er, rasselnde Atemzüge zu erkennen.

Er räusperte sich.

Der rasche Atemzug, den er nun vernahm, verriet ihm, dass sein Gegenüber ihn nicht erwartet hatte. Die Agren fing sich allerdings rasch und begann mit ihrer rauen Stimme zu sprechen:

„Verzeiht mir, o Seher, ich hatte nicht gedacht, dass Ihr so schnell hierher fändet. Ich bin Rhona, die Stammesälteste dieses Agren-Stammes. Tretet zu mir und nehmt meine Hand, ich führe Euch zu einer passenden Sitzgelegenheit.“

Die Sitzgelegenheit stellte sich als einen ein klein wenig weicheren und klein wenig weniger kalten Teil des kargen Höhlenbodens heraus, aber Leander war nicht verwöhnt und ließ sich nach dem Gewaltmarsch ins Graue Gebirge nur allzu gerne zu Boden sinken. Er ächzte, als die Muskeln in seinen dünnen Beinen zuerst protestierten und sich dann entspannten.

Rhona murmelte etwas vor sich hin und bot dann an: „Ihr müsst hungrig und durstig sein. Wollt ihr frisches Quellwasser? Ich könnte es mit einigen Edelgelb-Blüten aufbessern. Dieses Gemisch hat sich kürzlich diesseits der Korn-Schlucht äußerst beliebt gemacht.“

Leander winkte dankend ab: „Bemüht Euch nicht, Stammesälteste.“

„Bitte, nennt mich Rhona. Die Formalitäten brauchen Euch nicht zu kümmern. Ihr seid schließlich hier, weil wir Eure Hilfe brauchen.“

Endlich kommen wir zur Sache, schmunzelte Leander innerlich. Jorn hatte alles Mögliche verlauten lassen, außer auch nur den kleinsten Hinweis darauf, warum Rhona ihn hatte rufen lassen. Leander hatte natürlich einige Vermutungen angestellt und verschiedenste Steine und Pülverchen eingepackt, dennoch war er aufrichtig gespannt auf Rhonas Gründe.

„Dann erzählt einmal, Rhona. Ich kann mir vorstellen, dass es für Euer Volk einige Probleme bereitete, nach der Befreiung aus der Gefangenschaft dieses finsteren Nekromanten wieder in Eure Höhlen zurückzukehren und eine neue Normalität aufzubauen. Aber das ist nun auch schon wieder mehr als ein halbes Jahrdutzend her. Ich wüsste nicht, wobei gerade ich Euch behilflich sein könnte. Und warum gerade Euch. Kümmert sich sonst nicht der Stammesälteste Grone um den Kontakt mit Menschen von außerhalb des Grauen Gebirges?“

Rhona schwieg. Wahrscheinlich war es nicht angenehm für sie, zurück an die Zeit der Gefangenschaft unter dem finsteren Hademar zu denken. Dann sprach sie: „Um ebenjenen Nekromanten geht es. Und nun ja, Ihr redet nicht mit Grone, sondern mit mir, weil Ihr nicht hier seid auf Gesuch von jemandem außerhalb des Grauen Gebirges. Ihr seid hier, weil Euch jemand vorgeschlagen hat, der eng mit unserem Stamm in Kontakt steht. Wisst Ihr, der Wolfskrieger hat von Euch geschwärmt.“

Das war interessant. Leander hatte bislang mit Orfen nur wenig Kontakt gehabt, und noch weniger von sich selbst preisgegeben. Einige Male hatte Leander Orfen im Wachsamen Wald angetroffen, einmal sogar in Begleitung dieses riesigen schwarzen Wolfs, des Königswolfs. Bei der Gelegenheit hatte Leander versucht, Orfen zum Plaudern über die Herkunft der magischen Fertigkeiten des Königswolfs zu bringen. Er hätte geradesogut auf Zwergenstahl beißen können. Nur zu den Gerüchten, dass im Grauen Gebirge der Geist eines bösartigen Krahders herumspukte, hatte Orfen etwas verlauten lassen. Offenbar hatten ihn die exakten Fragen Leanders beeindruckt. Sah er etwa eine Art Experte für Geisterkunde in ihm? Auf jeden Fall glaubte Leander nun zu wissen, warum er hierhin gerufen wurde. Er kramte in seinem Mantel nach dem weißen Runenstein, der ihm in diesem Fall behilflich sein könnte, und atmete erleichtert auf, als er ihn ertastete.

„Ihr vermutet, dass der Geist des Nekromanten sich noch immer in dieser Welt aufhält?“, riet Leander ins Blaue hinaus.

„Ihr vermutet richtig“, krächzte Rhona, „Es scheint, als hätte ich den Richtigen gerufen. Auf mein Bauchgefühl ist meistens Verlass.“

„Nun, das muss sich erst noch zeigen.“, sagte Leander vorsichtig. Die Gelegenheit, Kontakt mit dem Geist des Nekromanten aufzunehmen, würde ihn natürlich mit großer Freude erfüllen. Er konnte kaum glauben, über was für Wissensschätze der Nekromant verfügen musste. Aber die Helden von Andor hatten den Nekromanten aus der Winterburg verjagt, und diese waren leider meistens sehr gründlich im Auslöschen von Gegnern aller Art.

„Wie kommt ihr auf diese Vermutung, dass der Geist des Nekromanten noch in dieser Sphäre verweile?“, fragte Leander, innig hoffend, dass die Antwort nicht nur in einem Bauchgefühl Rhonas bestünde.

„Ich vermute nicht“, korrigierte ihn Rhona, „Ich weiß es. Finstere Träume plagen mich schon seit Jahren. Seit unserer Befreiung aus der Winterburg, um genau zu sein. Finstere Träume von einer Gestalt mit einer Kapuze. Erst waren es nur Schemen, doch in letzter Zeit kann ich ihre Form deutlich erkennen. Sie wird stärker.“

„Hademar, der Nekromant?“

„Ich weiß es nicht. Aber ich weiß, dass seit einigen Wochen sämtliche Agren, die sich zu tief in die Überreste der Zwergenbauwerke des Grauen Gebirges vorwagen, spurlos verschwunden sind. Fünf an der Zahl waren es. Fünf verschiedene Bauwerke. Und in jeder der fünf Nächte erschien mir im Traum diese Gestalt in einem roten Umhang. Sie hielt einen Stab umklammert, an dem Agrenschädel hingen, nach jedem Vorfall einen mehr. Ihre finstere Kraft nimmt stetig zu.“

Leander legte seinen Kopf schief.

„Fünf kurz aufeinanderfolgende Unfälle sind natürlich Grund zur Sorge, könnten aber reine Werke des Zufalls sein. Und die Träume... besitzt Ihr überhaupt die Gabe des zweiten Gesichts?“

„Ich weiß, wann ich meiner Intuition vertrauen kann. Spottet nicht, o Seher. Ich bitte Euch bloß, dieser Sache nachzugehen. Falls der Nekromant weiterhin hier umgeht, sind wir nicht sicher. Er dürstet danach, unser Volk zu versklaven und einzukerkern. Nicht einmal zu seinem eigenen Vorteil nutzen will er uns, wie er es mit den Arpachen tat. Es ist reine Rache, nach der er strebt. Rache dafür, dass wir einst seinen Bruder unterstützt hatten, als dieser sich von ihm abwandte und ihn in Krahd zurückließ.“

„Ich werde sehen, was ich tun kann“, murmelte Leander enttäuscht. Fünf verschwundene Agren und einige Träume einer Greisin. Dafür hatte sie ihn die lange Reise ins Graue Gebirge antreten lassen? Nicht, dass er die Macht der Träume nicht kannte, aber das Lesen der Sprache der Gegenwart und der Zukunft war eine höchst komplexe Angelegenheit, deren Studium Jahre des Übens bedurfte, ehe man auch nur halbwegs konsistente Resultate liefern konnte. Bauchgefühle waren hier fehl am Platz.

Die alte Reka wäre da anderer Meinung gewesen, erinnerte Leander sich. Die beiden hatten sich schon tiefgehend über Prophezeiungen und Visionen zerstritten. Im Gegensatz zu ihm verließ sich Reka ausschließlich auf Träume und Bauchgefühle. Nun, Leander wusste ja, wer von ihnen beiden stets korrekte Vorhersagen machte und wer sich in seinen Interpretationen oft hatte fehlleiten lassen.

„O Seher?“, fragte Rhona plötzlich mit einem alarmierten Unterton in der Stimme und unterbrach Leanders Gedanken.

Leander gab einen fragenden Laut von sich.

„Mögt Ihr mir mitteilen, warum ihr ein Herz des Nehal mit Euch tragt?“

Leander erschrak und ließ hastig den weißen Runenstein in seiner Manteltasche los. Dieser weiße Edelstein war einer der größten Edelsteine, den die Schildzwerge je in Cavern gefunden hatten, und einer der beiden Edelsteine, die in die Statue zu Ehren des Drachen Nehal am nördlichen Ende der Korn-Schlucht eingelassen worden waren. Sie waren schon vor Jahrhunderten im Innern der Statue bei Nehals Stein an der Stelle platziert worden, wo sich bei Drachen die Herzen befinden.

Die beiden Kristalle waren kürzlich von einem dreisten Dieb namens Ken Dorr aus der Statue gebrochen worden und hatten ihren Weg in den Besitz eines gierigen Handelszwergs, Garz genannt, gefunden, welcher zumindest einen davon für eine horrende Summe an Leander verhökert hatte. Das andere Drachenherz war wahrscheinlich wieder zurück in Cavern gelangt, wo es nun wie ein Augapfel gehütet wurde. Doch im Gegensatz zu den Schildzwergen wusste Leander um die Magie, die in diesem Edelstein aus den Tiefen der Erde konserviert war. Die nötigen Runen, um diese Magie freizusetzen, hatte er schon längst in seine Oberfläche geritzt. Und so hatte er den Runenstein mit ins Graue Gebirge genommen, weil er geahnt hatte, dass er seine Kräfte hier würde nutzen können.

Hatte die Stammesälteste die Magie in seiner Manteltasche gespürt? Oder war auch das nur ein Bauchgefühl gewesen, eine Vermutung, um zu zeigen, dass sie doch über eine gewisse Gabe des zweiten Gesichts verfügte? Leander drehte sich zur Seite und überlegte sich eine angemessene Antwort, da fuhr Rhona bereits fort.

Ihre Stimme hatte einen drohenden Unterton angenommen und zitterte leicht: „Wenn ich mich nicht stark irre, sollte sich dieser Edelstein in einer Statue zu Ehren des ehrenwertesten Drachen der uralten Zeit befinden, nicht im Mantel eines düsteren Sehers aus dem Drachenland. Es scheint mir respektlos gegenüber Kreatok und Nehal, das Mahnmal an die Macht der Drachen und die Fragilität von Bündnissen aus Gründen der Gier zu schänden.“

Leander suchte in seinem Kopf nach besänftigenden Worten, als ihm plötzlich ein beunruhigender Gedanke kam. Er hatte Gerüchte über das Graue Gebirge gehört, Gerüchte, die Orfens Antworten damals hatten zu einer konkreten These verfestigen können. Aus dieser These folgerte Leander nun eine Vermutung für die konkrete Situation, in der er sich befand. Warum die Agren den weißen Edelstein hatte spüren können. Warum er sie so enervierte. Das war keine Wut, das war... Furcht. Wenn Leanders Vermutung der Wahrheit entsprach, dann nahm der Hilferuf der alten Agren plötzlich ganz andere Dimensionen an.

Glücklicherweise gab es einen direkten Weg, seine Vermutung zu widerlegen, sollte sie falsch sein. Leander ignorierte das aufgebrachte Gezeter der Stammesältesten, zog den weißen Edelstein aus seiner Tasche hervor, hielt ihn dorthin, wo er das Rhonas Gesicht vermutete, und sprach einen Satz in der Alten Sprache.

Wie erhofft, unterbrach die Stammesälteste ihre Tirade sofort, heulte auf und stolperte zurück. Leander vernahm hastige Schritte von weiter draußen und eine ihm unbekannte Agren schrie auf: „Rhona! Was ist geschehen?!“

„Raus mit dir!“, zischte Rhona zurück, „Ich bin bloß hingefallen. Wenn ihr Jungspunde Euch nur ein einziges Mal nicht zu viel Sorgen um mich machen würdet...“

Langsamere, schlurfende Schritte zeugten davon, wie die fremde Agren den Raum beschämt wieder verließ.

Leander konnte es nicht lassen, den Besserwisser zu spielen, beugte sich vor und flüsterte zu Rhona: „Das Drachenherz ist in meinen Händen besser aufbewahrt als im Körper einer Steinstatue. Habt Ihr mich nicht rufen lassen, um den alten Zwergenruinen einen Geist auszutreiben? Dieser Edelstein wurde mit den passenden Runen versehen und ist nun in der Lage, eine nicht mehr in ihrem eigenen Körper verweilende Seele zu greifen und in seinem kristallenen Innern einzusperren. So kann man fremde Geister fangen. Alles, was es dafür braucht, sind die richtigen Worte in der Alten Sprache. Aber das wisst Ihr alles natürlich bereits, sonst wärt Ihr nicht so hektisch zurückgewichen. Euer Glück, dass ich nicht die richtigen Worte gesprochen habe.“

Die Reaktion der Stammesältesten hatte Leander mit ziemlicher Sicherheit erfüllt, dass er gar nicht mit der echten Stammesältesten sprach. Nur mit einem Geist, der ihren Körper führte. Und da er abgesehen von Hademar nur von einem möglichen Geist im Grauen Gebirge vernommen hatte, wagte er einen Tipp abzugeben, mit wem er hier sprach: „Nomion. Der erste Krahder. Es ist mir eine Ehre, Eure Bekanntschaft zu machen.“

Sein Gegenüber blieb still, was Leander großzügig als Bestätigung seiner Vermutung interpretierte. So dachte er laut weiter:

„Was könnte den Geist des Entdeckers der Dunklen Hexerei dazu bewegt haben, den Körper dieser Stammesältesten einzunehmen und nach der vermutlich einzigen Person im Lande zu rufen, die ihn tatsächlich erkennen könnte? Warum sich dem durchaus beträchtlichen Risiko aussetzen, von mir eingefangen zu werden? Hochmut könnte es natürlich sein, Hochmut, wie man es von den Riesen kennt. Ihr hieltet es für unwahrscheinlich, dass ich Eins und Eins zusammenzählen könnte, und wenn ihr nicht so direkt nach dem Drachenherz gefragt hättet, hätte es wohl selbst mich ein bisschen länger gekostet, den richtigen Schluss zu ziehen.“

Die Stammesälteste gab ein Schnauben von sich. Leander fuhr genüsslich fort: „Ihr seid dieses Risiko willentlich eingegangen. Ihr braucht also etwas von mir. Ihr riefet mich hierher und gabt mir den Auftrag, die Ruinen der alten Zwergenbauwerke nach einem Unheil abzusuchen. Da liegt der Schluss nahe, dass in diesen Ruinen tatsächlich etwas umgeht. Etwas, oder jemand. Jemand, vor dem sich selbst der große Nomion fürchtet?“

Leander legte seinen Kopf schief. Da endlich meldete sich die Stammesälteste wieder zu Wort. Ihre Stimme klang plötzlich finster, als würde sie jedes Wort wütend durchkauen und dann erst ausspucken: „Wie wagst du es, so mit mir zu reden? Ich bin Nomion, der Hexer aus Krahd! Der Meister des Urtrolls! Ich gebiete über eine Naturgewalt und ich fürchte mich vor niemandem!“

Leander klatschte in seine Hände und verstaute den weißen Runenstein wieder in seiner Manteltasche. „Warum nicht gleich so?“, fragte er fröhlich. Ohne auf eine Antwort zu warten, führte er seinen Gedankenschluss zu Ende:

„Nomion, der Meister des Urtrolls, musste sich tatsächlich vor niemandem fürchten, als der Urtroll noch das mächtigste Wesen im Grauen Gebirge war. Aber das war einmal. Der Urtroll ist ein Schatten seines früheren Selbst. Jahrhunderte hat er geschlafen, und als er das letzte Mal geweckt wurde, war es nicht von Euch. Er wurde von den Helden von Andor besiegt und von einer Fee kontrolliert, aber auch diese sind es nicht, vor denen Ihr Euch fürchtet. Nein, ihr fürchtet Euch vor demjenigen, der es vermochte, Euren Urtroll mit einem einzigen grellen Strahl aus purer Magie zu vertreiben.“

„Nomion fürchtet sich vor niemandem!“, keifte Nomion erneut auf. Leander musste ein Glucksen unterdrücken. War dies etwa wirklich der bösartige Herrscher, der einen Kult von Krahder-Hexern gegründet und die Drachen beinahe vollständig ausgelöscht hatte? Dessen Körper nur von Tarok selbst hatte vernichtet werden können und dessen Geist sich selbst danach noch an diese Welt hatte klammern können? Der Zahn der Zeit schien auch an ihm genagt zu haben.

„Sagt mir, Nomion“, fragte Leander nun, „Ist es Hademar von Krahd gelungen, die Lehren der Dunklen Hexer mit geheimen Wissen aus der Akademie von Hadria zu kombinieren? Hat der Nekromant, gestärkt durch die Drachenmagie, es geschafft, dem Tod ein noch größeres Schnippchen zu schlagen, als Ihr es vermochtet? Vermutet Ihr etwa, dass sein Geist dem Euren überlegen ist?“

„Hademars Körper wurde von den Helden von Andor vernichtet, so wie Tarok den meinen zu Asche vergehen ließ“, knurrte Nomion, „Nicht aber sein Geist. Hademars Geist streift immer noch in diesen Landen umher, setzt sich langsam zusammen und wird mit jedem vergangenen Tag stärker. Meine Kräfte sind nicht mehr, was sie einst waren, und so ließ ich dich rufen, weil ich dir die Gelegenheit geben will, Hademar ein für alle Mal auszulöschen.“

Ein Feigling ist Nomion, wie alle Krahder, dachte Leander im Geheimen. Äußerlich sprach er: „Eine spannende Gelegenheit ist das zweifelsohne, o Erster der Krahder. Doch verratet mir dies: Warum sollte ich sie ergreifen?“

Nomion quetschte seinen nächsten Satz hervor und Leander hörte förmlich, wie er dabei die Lippen der Stammesältesten zu einem hämischen Grinsen verzerrte: „Mach mir nichts vor, Leander, Seher von Narkon. Ich habe von dir gehört. Du verzehrst dich doch förmlich nach vergangenem Wissen, und wer auf dieser Welt außer Hademar kennt sich schon sowohl in der unsrigen als auch in der nordischen Magie aus? Sein Fang wäre eine Goldgrube für dich. Hopp, hopp, Silberländler. Dein Ziel hast du doch bereits festgelegt, sobald ich dir verraten habe, dass Hademars Geist noch in der Nähe verweilt und eingefangen werden kann. Möge die Jagd beginnen.“

Leander trommelte mit seinen Fingern auf seinem Holzstab herum. Nomion hatte recht, Leander würde sich nicht im Traum die Gelegenheit entgehen lassen, Hademars Geist zu fangen und seine Kenntnisse auszupressen. Doch mochte er es nicht im Geringsten, ausgespielt zu werden. Normalerweise war Leander es, der andere Personen zu seinen eigenen Zwecken manipulierte. Dass Nomion den Spieß nun umdrehte, gefiel ihm nicht im Geringsten. Leanders eigenen Zwecke bestanden oft darin, heldenhafte Gestalten an seiner Stelle in den gefährlichen Norden Silberlands zu schicken. Heldenhafte Gestalten wie den melancholischen Schwarzen Barden Orril. Oder wie den Feuerkrieger Trieest aus dem fernen Danwar. Wenn damals diese Wassermagierin Jarid nur nicht dazwischen gekommen wäre...

Leander musste an Rhona denken. Wie lange hatte sie gemeinsam mit dem restlichen Volk der Agren unter dem Joch Hademars leiden müssen, nur um nun unter Nomions Kontrolle zu stehen? War sie wohl wach? Konnte sie fühlen, was um sie vorging? Beim Gedanken an ihr Leiden wallten Schmerzen in Leander auf.

Schnell schüttelte Leander diesen Gedankengang beiseite. Die Stammesälteste bedeutete ihm nichts. Er hatte selbst schon so viele Personen kontrolliert, wenn auch nie so direkt wie Nomion. Die Frage, die ihn vielmehr beschäftigen sollte, war, wann und wie öffentlich Nomion ihren Körper verlassen würde. Wenn der Agrenstamm Wind davon bekäme, dass Leander gewusst hatte, was mit ihrer Stammesältesten vorgegangen war, und nichts dagegen getan hätte, dann würden die Agren wohl kaum mehr so freundlich mit ihm umgehen. War das ein verkraftbarer Verlust?

Während Leanders Gedanken rasten und kalkulierten, wandte er sich wieder an Nomion und erwiderte: „Hademars Kenntnisse über die Dunkle Hexerei umfassen doch höchstens das wenige, was er als Sklave in Eurem Tempel aufgeschnappt hatte. Sie können unmöglich so tief gehen wie die Euren, der diesen Tempel doch überhaupt erst errichten ließ. Gibt es einen Grund, warum ich Euch nicht an Ort und Stelle im Drachenherz gefangen nehmen und Euer Wissen auspressen sollte?“

Urplötzlich wurde es kalt um Leander. Jedes Haar auf seinem Körper stellte sich auf. Er hörte die Stammesälteste röchelnd husten und spürte die Gegenwart von etwas Fremden. Eine Macht, die groß genug war, um einen jeden Geist zu versklaven. Leander nahm wahr, wie \textit{Etwas} in seinen Geist eindrang und seine Gedanken in Eiseskälte tauchte. Er stolperte alarmiert zurück und stieß mit seinen Kopf gegen die Höhlendecke.

Sein Körper fing sich von selbst. Von Horror erfüllt, spürte Leander seine Zunge in seiner Mundhöhle umherwandern und seine Zähne entlanggleiten. Seine eigenen Lippen öffneten sich und er hörte seinen eigenen Mund krächzend sprechen: „Drohst du mir etwa, Leander von Narkon?!“

Leander versuchte zu antworten, versuchte, zurückzuweichen, versuchte, in seiner Tasche nach dem Drachenherz zu langen. Alles erfolglos, sein Köper gehorchte ihm nicht mehr.

Wie ein Fremder in seinem eigenen Kopf vernahm er, wie sein Mund sich erneut verdrehte und sprach: „Na los, versuch schon, mich im Edelstein einzufangen.“

Ein grässliches Lachen entsprang Leanders Kehle, als Nomion fortfuhr, nun aus Leanders Mund statt aus Rhonas: „Es kann nur eine Seele auf einmal im Drachenherz gefangen sein. Hademar wird stetig stärker und muss möglichst bald gebändigt werden. Seine Kenntnisse werden dich für Jahre beschäftigen. Das muss deine Priorität sein. Später, wenn du dich an ihm ergötzt hast, magst du ihn vernichten und dich wieder auf die Suche nach mir machen. Aber wisse, dass du mich nur dann finden wirst, wenn ich es wünsche. Denn alles geschieht, wie Nomion gebietet!“

Leanders Magen drehte sich auf den Kopf, als Nomions Kälte seinen Körper verließ. Röchelnd stürzte Leander zu Boden und hustete sich die fremde Seele aus dem Leib. Ein letztes Lachen vernahm er noch („Ich hoffe für dich, dass du dich bei Hademar geschickter anstellst als gerade eben!“), dann huschte ein Windhauch an ihm vorbei und Nomions Präsenz war nicht mehr da.

Leander und die Stammesälteste waren beide alleine und lagen schwer atmend am Höhlenboden. Leander rührte sich als erster und fragte: „Stammesälteste! Rhona! Seid Ihr verletzt? Geht es Euch gut?“

Die Stammesälteste antwortete nicht in Worten, sondern stieß einen langanhaltenden Schrei aus, der all ihre Qualen ausdrücke, die sie nicht auszuformulieren vermochte.

Leander wich schockiert zurück. Wie lange war sie die Gefangene Nomions gewesen?

Erneut hörte Leander hektische Schritte die Höhle betreten und eine fremde Agrenstimme fragte: „Rhona, Rhona, so sprecht mit mir! Was ist los?“

Die Stammesälteste schluchzte auf und schrie: „Was... was ist geschehen? Welcher Dämon... bei der Mutter der Erde, wer war das?!“

Leander wandte sich an die Neuankömmlinge und sprach: „Rhona wurde von einem bösartigen Geist erfüllt, doch dieser ist weitergezogen. Er kann ihr nicht weiter schaden.“

Weitere Schritte, dann wurde Leander plötzlich am Kragen gepackt: „Was hast du mit Rhona gemacht, du Finsterling?!“

„Wie ich bereits sagte, sie wurde einem finsteren Geist kontrolliert. Ich hatte nichts damit zu tun, ich habe sie vielmehr befreit.“

Agren waren genau wie die Mehrzahl der Menschen, Zwerge und Taren von Emotionen erfüllt, die ihnen das Zusammenleben vereinfachten und sie offener für Leanders Manipulationen machten, aber leider auch oft von rationalen Handlungen abhielt. Das musste Leander erneut erleben, als ihn plötzlich etwas Spitzes in die Seite stach. Verächtlich schlug er der Agren die Klinge aus der Hand, da schlug etwas anderes, hölzernes, feste gegen sein Schienbein. Leander fluchte auf. Den Schmerz zu ignorieren vermochte er natürlich, aber sein Bein war kein Baumstamm und gab unter dem Schlag nach. Leander ging zu Boden. Sein Kopf schlug gegen den Höhlenboden, der nun doch nicht mehr so weich wirkte, wie als Leander sich zuerst darauf niedergelassen hatte.

Alles wurde farbig. Leander sah eine Vision der Winterburg sich um ihn herum aus dem Boden erheben und konnte durch die Mauersteine hindurch den Körper der Stammesältesten erkennen, der in einer kleinen Zelle saß. Dann flog die Gittertür auf und eine seltsame Schattengestalt trat ein, die sich stetig deformierte und zwischen der Silhouette eines Menschen und der eines Bären hin- und herzuwechseln schien. Der Mund der Stammesältesten verzog sich zu einem Lächeln und alles wurde wieder schwarz.

„Der Mensch im Bären und der Bär im Menschen wird kommen, um Euch zu retten“, flüsterte Leander, ohne diese Worte gewählt zu haben.

„Zurück, tretet zurück!“, erklang die herrische Stimme Rhonas, „Er spricht die Wahrheit, nicht er hat mir etwas angetan.“

Leander bemerkte erleichtert, dass ihn keine Stiche und Schläge mehr trafen. Vollkommene Stille erfüllte die Höhle, oder zumindest derart prominente Stille, wie sie in der Präsenz mehrerer Personen auftreten konnte. Manchmal war Leander froh darüber, dass seine Augen aufglühten, sobald er eine Vision empfing. Das erlaubte seinem Umfeld, zu erkennen, dass er nicht einfach Lügenmärchen erfand.

Als die Stimme der Stammesältesten erneut erklang, war sie ganz nahe an Leander getreten und hauchte ihm förmlich ins Gesicht: „Was... was habt Ihr soeben gesagt? Was habt ihr gesehen, Seher?“

„Es tut mir so leid“, antwortete Leander, „Ihr werdet noch ein drittes Mal in Gefangenschaft geraten. Ich habe es gesehen. Doch fürchtet Euch nicht. Der Mensch im Bären und der Bär im Menschen wird kommen, um Euch zu retten. Und danach werdet Ihr den Rest Eures Lebens in Freiheit genießen können.“

Leander richtete sich abrupt auf und richtete seinen Mantel. Er tastete sein Schienbein ab. Nichts schien gebrochen und den Stich in seine Seite fühlte er nicht einmal mehr.

„Stab!“, verlangte er herrisch. Von seiner Rechten wurde ihm sein Stab in die Hand gedrückt.

„Aus dem Weg!“, brachte er noch heraus, dann stolperte Leander mehr aus der Höhle, als dass er schritt.\bigskip







Erst als Leander sich einige Minuten von der Höhle der Stammesältesten entfernt hatte, hielt er inne und verschnaufte. Er rückte seine Augenbinde gerade und fühlte überrascht Tränen seine Wangen runterrollen. Die Vision hatte ihn stärker getroffen, als er angenommen hatte. Leander sank zu Boden.

Die Wahrheit war, dass dies das erste Mal seit Langem gewesen war, dass ihm endlich wieder eine Vision erschienen war. Natürlich hatte er gewusst, dass er seine Gabe der Voraussicht nicht einfach so verloren hatte. Dennoch war in den letzten Monaten ein immer heftigerer Samen der Furcht und des Zweifels in ihm gewachsen. Die Bestätigung, dass er noch immer das Zeug zum Seher hatte, erfüllte ihn mit Erleichterung und Euphorie. Die Hoffnung, seinen Bruder doch noch retten zu können, erwachte wieder in ihm.

Leander erinnerte sich daran, als wäre es gestern gewesen, was er sich damals davon erhofft hatte, die Sprache der Zukunft zu erlernen: Ein Gefühl für die Zukunft, eine Gewissheit, was ihn erwarten würde. Seinen Tod würde er voraussehen, und zwar so weit herausgezögert, wie nur irgendwie möglich, denn wenn es ein vermeidbarer Tod wäre, so könnte er ihn ja vermeiden und hätte ihn gar nicht erst gesehen. Allerlei geniale Ideen und Pläne aus der Zukunft würde er empfangen, und funktionieren würden sie, weil, wenn sie noch verbessert werden könnten, so hätte er sie ja bereits verbessert vorhergesehen. Ja, damals hatte Leander noch gedacht, dass seine seherischen Fähigkeiten ihn zu einer Art Gottheit machen würden, fast allwissend, fast unaufhaltsam.

Doch es war nicht dazu gekommen. Alles, was Leander vernehmen konnte, waren kurze Einblicke in die Zukunft, Schemen, Gestalten, Wortfetzen. Auch wenn er die Sprache der Zukunft erlernt hatte, war ihre Stimme oft leise und schwer verständlich. Und am Schlimmsten war, dass er nicht einmal kontrollieren konnte, wann ihn eine Vision ergriff. Zufällig waren sie nicht, dafür tauchten sie zu oft in passenden Situationen auf. Ob eine bewusste höhere Macht dahintersteckte, die ihm diese Visionen zuschickte? Diese würde ihn dadurch gut manipulieren können. Das war ein Bild, das Leander nicht gefiel. Und wer würde ihm schon nur so spärliches Wissen zukommen lassen wollen und vor allem zu welchem Zweck? Andererseits hatte Leander auch keine andere eindeutige Gesetzmäßigkeit hinter den Visionen erkennen können, allerhöchstens Trends. In ereignisreichen Jahren tendierten mehr Bilder der Zukunft dazu, ihn zu erreichen. Insbesondere wenn Leander selbst betroffen oder in Gefahr war. Dem gegenübergestellt hatte Leander in den letzten Monaten gleich zwei brenzlige Konfrontationen mit marodierenden Gorlots erlebt, ohne auch nur eine Vision zu empfangen. Die Sachlage blieb unergründlich.

Auf jeden Fall empfand Leander ungemeine, ja, gar überwältigende Freude darüber, nun endlich wieder einen Blick in der Zukunft empfangen zu haben, auch wenn dieser sich nur um die unwichtige Stammesälteste gedreht hatte.

Leander richtete sich auf und versuchte, diese Gefühle und diese Gedanken an Vergangenes zu unterdrücken.

Es gab im Moment Wichtigeres zu tun.

Er hatte einen Nekromanten zu fangen.











\newpage
\section{Nervige Hexen, zwei an der Zahl}


Leander tastete den Boden nach einer Schlafgelegenheit ab. Sein Stock stieß auf Knochen, Knochen und noch mehr Knochen. Schädel, Rippen und Beine von Menschen, Zwergen und Schafen zugleich übersäten die Knochengrube. Hier und dort konnte man sogar den magischen Fußknochen eines waschechten Drachen finden.

Eigentlich sollte Leander in Hochstimmung sein. Er hatte soeben (im Prolog) geschafft, Hademars stetig stärker werdenden Geist zu einem Blutsteinaltar der Zwerge unter dem Grauen Gebirge zu rufen und im Licht des roten Mondes ins Drachenherz zu sperren. Nun leuchtete dieser einst strahlend weiße Edelstein in einem tiefdunklen Rot. Zumindest hatte ihm Jorn der Agren wortreich von der satten Farbe geschwärmt. Jorn hatte zudem unter der steinernen Oberfläche hin und wieder dunkle Schwaden zu erkennen geglaubt, welche willkürlich im Innern des Drachenherzes umherwaberten. Hademars Geist musste sich wohl erst an seinen neuen kristallenen Körper akklimatisieren und herausfinden, wie er seine schwarze Seele im Kristall verformen konnte, um mit der Außenwelt zu kommunizieren. Nun, Leander würde ihn darin unterstützen, sobald er ihn erst einmal zu seiner Hütte im Wachsamen Wald zurückgebracht hatte.

Siegestrunken war Leander zurück nach Andor aufgebrochen, diesmal ohne die Begleitung Jorns. Doch hier, in der Knochengrube, wo er sein Nachtlager aufschlagen würde, konnte er einfach nicht fröhlich bleiben. Leander trauerte. Er trauerte um die Drachen, deren kollektives Wissen, deren Magie, deren Geschichten und deren Gedanken ein für alle Mal ausgelöscht worden waren, als diese elenden Helden von Andor den letzten ihrer Art erschlagen hatten.

Leander erinnerte sich nur allzu gut daran, wie Schreie durch den Wachsamen Wald gedrungen waren und wie ein Geruch nach Rauch und Schwefel das ganze Land erfüllt hatte. Das war einer der Momente gewesen, wo sich Leander seine Sehkraft zurück gewünscht hatte. Taroks Flug über das Land hätte er nur zu gerne wahrgenommen. Wie die riesige Echse ihren Schatten über das Rietland warf, wie sie vom Himmel stieß... aber dann waren ihre Flügel von Pfeilen durchlöchert, ihre Schnauze von Äxten zerkratzt und ihre Herzen von Schwertern durchstoßen wurden. Barbaren waren sie doch, diese Helden! Und nicht nur Fenn.

Leander hätte einen friedlichen Weg gefunden, Tarok zu besänftigen. Den Zugang zu Krahal, all dieses Wissen, verloren für immer. Was für eine Schande!

So trauerte Leander lange um Tarok, ehe er endlich einschlief.\bigskip







Leander erwachte aufgrund der Kälte. Kälte war keine Besonderheit im Grauen Gebirge und für Leander wäre es nicht das erste Mal gewesen, dass sein pelziger Umgang des nachts zur Seite gerutscht wäre und ihm die Wärme seines Körpers verwehrt hätte.

Doch so kurz nach seinen Begegnungen mit den Geistern Nomions und Hademars war sich Leander umso stärker bewusst, wie sich eine unnatürliche Kälte anfühlte. Und die Kälte, die gerade Leanders Gliedmaßen schlottern ließ, war definitiv keinen natürlichen Ursprungs.

Leanders Vorgehen war rasch und methodisch.

Sein erster Gedanke galt Hademar, falls dieser es geschafft hatte, dem Drachenherz zu entrinnen. So tastete Leander in seiner Manteltasche nach dem Runenstein und bestätigte erleichtert dessen Vollständigkeit und Unversehrtheit.

Leanders zweiter Gedanke galt Nomion. War der Geist des ersten Krahders zurückgekehrt, um dem eingekerkerten Hademar ein für alle Mal den Garaus zu machen? Ein perfider Plan, der einem hinterlistigen Krahder alle Ehre machen würde, doch würde Nomion es wohl kaum riskieren, dass Leander seine Ankunft bemerkte. Leander konnte Hademar jederzeit aus dem Edelstein befreien und dann würde Nomion einem wahnsinnigen Nekromanten gegenüberstehen, der die Macht der Drachenmagie für sich nutzen konnte.

Nein, diese unnatürliche Kälte schien weder von Hademar noch Nomion zu stammen. Leander ging im Geiste sämtliche Begegnungen durch, die ihn an die jetzige erinnerten, und fand nach einem kurzen Augenblick einen Treffer. Genau diese Form von schleichender Kälte, die ihn an eine mondlose Nacht und an ein schattiges Tal erinnerte, diese Form von Kälte hatte er schon einmal gespürt. Aber das war lange her gewesen. Und Leander hätte schwören können, dass ihre Urheberin bereits vor über einem halben Jahrdutzend von diesen elenden Helden von Andor vernichtet worden war, tatsächlich erst wenige Tage vor Taroks Tod.

Schlaftrunken richtete sich Leander auf, griff nach seinem Holzstab und rief: „Alte Freundin, was schleichst du dich so an? So zeige dich doch! Wie hast du es denn geschafft, dem Getümmel bei der Rietburg zu entkommen?“

Mit Schrecken erkannte Leander, dass etwas Eiskaltes seine Füße umspülte und langsam seine Beine hochstieg.

„Lass das!“, fauchte Leander mit einer ein bisschen zu hohen Stimme, um einschüchternd zu wirken, „Ich weiß, du bist die Dunkelheit in Person und willst alles verschlingen, was sich dir entgegen stellt, aber ich kann dir so viel nützlicher sein, wenn du mich fürs Erste verschonst. Wir hatten eine solche Konversation schon einmal! Ich kam dich in deinem Verlies an der Rietburg besuchen. Hast du denn alles vergessen in der Zwischenzeit?“

Leander hatte sie tatsächlich im Verlies der Rietburg kennengelernt, als er sie in der Rolle eines mürrischen Gelehrten aus dem Wachsamen Wald aufgesucht hatte, um Kenntnisse über ihre Kräfte zu erlangen. Sie war seit den Trollkriegen dort angekettet und halb wahnsinnig gewesen, doch hatte sie sich ihm gegenüber aus ihm unverständlichen Gründen vollkommen freundlich verhalten. Nun wirkte sie überhaupt nicht mehr freundlich, als eine flüsternde und doch durchdringende Stimme erklang: „Wir kennen uns nicht, Fremder, doch sei getröstet: Sobald ich mir deine Seele einverleibt haben werden, werde ich dich kennen.“

Ja, dann ist es zu spät, dachte Leander trocken.

„Shan!“, rief er aus, „Ich will dir nicht wehtun, aber ich werde es tun, wenn ich muss.“

Shan, die Schattenhexe, lachte nur leise vor sich hin, während ihre Schattententakel sich noch stärker um Leanders Beine wanden.

Genug war genug. Leander konnte sich zwar nicht erklären, warum Shan sich nicht an ihn erinnerte, aber ihm war klar, warum sie hier war. Die Schattenhexe dürstete es nach Seelen und Hademars von Magie nur so strotzende Seele war im Drachenherz auf engsten Raum gequetscht. Eine so dichte, so mächtige Seele musste auf die hungrige Shan wie ein Leuchtfeuer wirken, das sie magisch anzog. Praktischerweise war der Kristall, dessen Inhalt Shan derart anzog, auch einer der einzigen effektiven Schutze gegen Shans Präsenz: Ein Edelstein Caverns, in welchem das Licht der Welt gespeichert war.

Natürlich förderte Hademars Präsenz mit all ihrer Dunklen Hexerei und Dunklen Magie das Licht des Drachenherzens nicht im Geringsten. Dennoch hoffte Leander, dass genug von der ehemaligen Reinheit des weißen Edelsteins übrig war, um Shan zu vertreiben.

So klaubte Leander zitternd das erkaltete Drachenherz aus seiner Manteltasche und hielt es Shans Stimme entgegen. Prompt erklang ein durchdringendes Heulen, als Shan zischte:

„Es brennt! Steck es weg! Weg!“

„Weiche zurück, und ich stecke den Stein weg, Shan!“, rief Leander aus, „Ich weiß, dass diese Seele unglaublich kräftig ist, aber du kannst sie nicht haben. Zumindest noch nicht.“

Shan heulte erneut auf, aber ihre Tentakel ließen endlich Leanders Beine los und der Seher stolperte zurück. Fast war er froh, dass er nicht sehen konnte, wie nahe Shans Schatten von allen Richtungen an ihn herandrangen, als er das Drachenherz in seinem Mantel verbarg, aber immer noch in seinem Griff behielt.

„Verflucht seist du, Fremder, verflucht! Mögest du nie wieder Frieden finden, bis...“, wisperte Shan.

Leander lachte auf. Er selbst hatte oft genug versucht, Seekönig Varatan zu verfluchen, als dass er sich von Shans Drohungen beeindrucken ließe: „Flüche bedeuten nichts, sofern sie nicht von einer Person mit königlichem Willen gesprochen werden.“

Und Personen mit königlichem Willen hatten oftmals keine Ahnung, was sie mit einem Fluch auslösten. Varatan, der Seekönig, hatte vor langer Zeit einen Fluch über eine ganze Insel verhängt, der seine Bewohner langsam in den Wahnsinn trieb. Unter diesen Bewohnern befand sich auch Callem, Leanders Bruder. Und Leander hatte seither immer wieder heldenhafte Gestalten nach Silberland geschickt, ohne dass auch nur eine davon hätte herausfinden können, wie man den Fluch brechen konnte. Stattdessen waren sie alle ihm verfallen.

„Ist das so?“, fragte Shan bedeutungsvoll, „Ist es so, dass nur ein willensstarker König einen Fluch verhängen kann? Wie kommt es dann, dass auf mir ein Fluch der Dunkelheit liegt? Die Dunkelheit selbst hat doch keinen eigenen Willen. Oder etwa doch?“

„Manchmal ist es besser zu schweigen, als etwas zu behaupten, von dem man nichts versteht“, stieß Leander müde hervor, „Ziehe von hinnen, Shan! Suche dir eine andere Beute und lass uns in Ruhe. Oder soll ich den Runenstein erneut hervorholen?“

„Es schmerzt, es schmerzt so sehr, diese Seele ziehen zu lassen, und es schmerzt, in ihrer Nähe zu bleiben“, wimmerte Shan, ehe sie fortfuhr: „Hüte deine Zunge, Fremder! Ich verstehe mehr von Flüchen, als du meinen magst! Ich kenne gleich mehrere Verfluchte, und an einem davon zieht ein Fluch, deren Signatur dem ganz ähnlich ist, der schwach an dir haftet. Vielleicht gar identisch.“

Leander zuckte zurück. Er wusste, dass er Varatans Fluch erschreckend nahe gekommen war, damals, als er nach Varatans Angriff auf die Schwarze Kogge die Silberberge erklommen hatte, um Callem zu suchen. Wie aus dem Nichts war das durchsichtige, verfallene Gesicht Varatans vor ihm aufgetaucht und hatte ihn angestarrt. Die Manifestation des Fluchs war jedoch nicht näher gekommen. Inzwischen wusste Leander, dass die Silberberge das Ende des Reviers des Fluchs markierten und dieser sie nicht überqueren konnte. Doch hatte Leander gedacht, dass diese verstörende Begegnung nicht gereicht hatte, um von ihm markiert zu werden.

Er schüttelte seinen Kopf. Das war egal. Das war alles egal. Wichtig war doch nur...

„Beweis es“, flüsterte Leander verschwörerisch, „Zeig mir, dass du dich auskennst, o Shan. Sage mir, wie man Varatans Fluch brechen kann.“

Shan lachte auf: „Das ist doch einfach. Töte Varatan! Dann löst sich der Fluch.“

Leander lachte auf: „Wenn es doch nur so einfach wäre! Varatan ist bereits vor langer Zeit umgekommen.“

In Gedanken ergänzte Leander, dass er sonst ja schon längstens einen Weg gefunden hätte, Varatan persönlich davon zu überzeugen, den Fluch aufzuheben.

Dann fuhr er fort: „Der Fluch wird auch nach seinem Tod weiterhin aufrecht erhalten. Kann das sein?“

„Ja, natürlich“, zischte Shan in aller Seelenruhe, als wäre es das Selbstverständlichste auf der Welt, „Die Nachkommen deines Fluch-Urhebers nähren diesen ebenfalls – sofern sie die nötige Willenskraft dazu besitzen. Varatans Willenskraft lebt in seiner Blutlinie weiter.“

Leander stutzte. Konnte die Lösung tatsächlich so einfach sein? Musste er nichts tun, als Varatans Nachkommen zu vernichten? Er fluchte innerlich auf beim Gedanken, dass er so nahe daran gewesen wäre.

Seine Gedanken flogen zurück in die Zeit seiner Jugend.

Sein Bruder Callem war früh von zu Hause aufgebrochen. Voller Tatendrang und einer unbeschreiblichen Gier nach Ruhm. Leander hingegen war zu Hause geblieben. Hatte lesen und ein wenig die Kunst des Heilens erlernt. Dann hatte er Interesse an Sprachen entwickelt. So hatte er die umliegenden Inseln besucht und ihre Sprachen erlernt. Überrascht hatte er damals festgestellt, dass von der Existenz seiner Heimatinsel auf den restlichen Nebelinseln niemand wusste.

In Varatanien hatte Leander sich niedergelassen und zu einem Physikus ausbilden lassen. So hatte er seine Kenntnisse mehren können und gleichzeitig den Kontakt mit Callem aufrecht erhalten, wenn dieser nicht gerade im Auftrag des Seekönigs das Hadrische Meers befuhr.

Doch dann war alles schief gelaufen. Callems Geist war von Kenvilar, der tückischsten Macht des Hadrischen Meeres, vergiftet worden. Er hatte sein Schiff vernichtet und eigenhändig den Großteil seiner Mannschaft getötet. Unter Kenvilars Führung hatte er eine neue Besatzung versammelt, die Besatzung der gefürchteten Schwarzen Kogge. Und gemeinsam waren diese Unruhestifter durch das Hadrische Meer gezogen, hatten Handelsschiffe überfallen und Waren geraubt.

Aufgrund seiner brüderlichen Verbindung mit Callem war Leander auf den Nebelinseln in Verruf geraten, sobald herausgekommen war, dass Callem noch lebte, in den Bann des Bösen gelangt war und nun nach der Krone der Nordmeere strebte. Daraufhin war Leander auf einem Schiff gen Hadria gereist. Er hatte zwar nie ein Talent für Zaubersprüche und mächtigeren Hokuspokus gezeigt, doch das Wissen der Akademie von Hadria und die Gespräche mit dem Gelehrten Hombudt, dem Zauberer des Raums im Turm der Magie, waren für ihn Grund genug gewesen, sich eine Zeit lang in Nordgard niederzulassen.

So hatte Leander nur durch Hörensagen erfahren, wie Varatan die Schwarze Kogge bis in ihr Versteck verfolgt hatte, Leanders und Callems Heimatinsel. Wie Varatan die Insel Narkon getauft und seinen unsäglichen Fluch darüber verhängt hatte, ungeachtet der restlichen Bewohner der Insel.

Hingegen hatte Leander direkt miterlebt, wie Varatan Jahre später in den Norden gereist war. Er hatte den grünlichen Rauch aus der Hadrischen Unterwelt aufsteigen sehen, als Orweyn mit einem gewöhnlichen Falken und drei gewöhnlichen Waffen in den Horun hinabgestiegen und mit dem riesigen Falken von Yra und drei magischen Waffen zurückgekehrt war. Er hatte aus sicherer Entfernung durch ein Fernrohr beobachtet, wie Varatan mit den drei magischen Waffen stolz die Mächte des Meeres herausgefordert hatte und seine Flotte von den Mächten vernichtend geschlagen worden war. Ja, Leander hatte miterlebt, wie Varatan nach Orweyns Opfer – oder besser gesagt Orweyns Massaker an der Zauberergemeinschaft – seine Königswürde abgelegt hatte. Leider hatte dies nicht gereicht, um seinen Fluch über Narkon aufzuheben.

Leander war auch dabei gewesen, als Varkmar, Sohn des Varatan, unzufriedene Zauberer in Nordgard um sich geschart und den Zaubererorden des Feuers gegründet hatte. Varkmar selbst hatte ihm den entscheidenden Tipp gegeben, wie Leander die Sprache der Zukunft erlernen könnte. Es war eine Form der Dunklen Magie, die ihn befähigte das zu sehen was nicht ist, sondern erst noch kommt. Doch wie immer hatte die Dunkle Magie ihren Tribut gefordert. In gleichem Maße wie Leanders Fähigkeiten wahrzusagen zugenommen hatte, waren seine Augen für das hier und jetzt schwächer geworden, bis ihr Licht ganz erloschen war.

So nahe war Leander an Varkmar gewesen! Eine einzige Phiole schwarzen Vypera-Gifts, ein kleiner Stich mit einer Nadel, und Varatans Fluch wäre gebrochen gewesen.

Doch stattdessen hatte sich Leander, enttäuscht über die vagen Visionen des erst noch Bevorstehenden und bestürzt über den Verlust seines Augenlichts, aus der kalten und klammen Eiswelt Hadrias ins warme Andor zurückgezogen und sich im Wachsamen Wald eine Hütte ergaunert. Die Bewahrer vom Wachsamen Wald betrachteten ihn nun wohl ähnlich wie die Hexe Reka als eigenbrötlerischen mürrischen Weisen, der hin und wieder aufkreuzte, um Wissen zu erfahren oder zum Rechten zu sorgen, und auch nach Jahren kaum einen Tag gealtert wirkte.

So war Hadria aus dem Blickfeld des Sehers geraten, und nun hatte Leander keine Ahnung, wie viele Kinder Varkmar Zeit seines Lebens in die Welt gesetzt hatte. Oder wie viele Kinder diese wiederum gezeugt hatten. Es würde potentiell sehr schwer werden, Varatans Blutlinie zu verfolgen. Er verwarf den Gedanken und wandte sich seiner aktuellen Konfrontation mit der Schattenhexe zu.

„Ich danke dir für diesen Hinweis, Shan“, sagte Leander, „Aber nun muss ich dich endgültig bitten, mich zu verlassen. Meine Seele wirst du nicht kriegen. Und auch das in diesem Edelstein eingesperrte Leben bleibt dir verwehrt. Zumindest heute.“

Ein demonstrativer Griff in seine Manteltasche, wo noch immer das leuchtende Drachenherz steckte, genügte. Die Schattenhexe kreischte ein letztes Mal auf („Die Dunkelheit wird dich bald erneut heimsuchen. Ich werde wiederkehren, wenn du erst einmal ungeschützt bist!“), dann wurde Leander merklich wärmer, als ihre Präsenz diesen Ort verließ. Er rubbelte an seinen Armen und Beinen und versuchte so mehr oder minder erfolgreich, seine Gliedmaßen vom Zittern abzuhalten. Schlussendlich legte er sich sorgsam wieder neben den Knochen der zahlreichen Drachenopfer in der Knochengrube nieder und versuchte, seinen Atem zu beruhigen, während seine Gedanken kreisten.\bigskip







„Das darf doch nicht wahr sein“, fluchte Leander. Nach der langen Reise war er endlich wieder im Wachsamen Wald eingetroffen, hatte sich mit einigen den grünen Radius beschützenden Bewahrern ausgetauscht – keine besonderen Vorkommnisse im Land in seiner Abwesenheit – und sich danach noch auf seine Hütte gefreut, die grau und unscheinbar am Waldrand stand, direkt an der Küste des Hadrischen Meeres. Das Rascheln der Blätter hatte die Luft erfüllte und das entfernte Rauschen der Wellen überdeckt. So hatte Leander gelauscht. Der Wind spielte schon überall sein Lied auf einem anderen Instrument. Die Gerüche des Waldes waren ihm in die Nase gestiegen. Für einen Augenblick hatte ihn das Gefühl erfüllt, zu Hause zu sein.

Doch dann hatte er seine Stirn gerunzelt und gerade eben geflucht. Ein feiner Geruch hatte seine Nase erreicht, der ganz und gar nicht ins restliche Sinnesbild passte. Ein Dampf war es, der Anteile von verdorbener Krallenflechte enthielt, aber auch nach süßen Bärenbeeren roch. Je näher Leander durch den lauschigen Waldweg auf seine Hütte zustapfte, desto lauter wurde das Rauschen des dahinter liegenden Hadrischen Meeres, aber desto stärker wurde auch dieser Geruch nach Kräutern und Rauch.

Kein Zweifel: Jemand kochte gerade in seiner Hütte ein Süppchen! Leander kannte nur eine, die so dreist wäre, ungefragt in sein Heim einzudringen und sich etwas zusammenzubrauen.

Mit seinem Stock schwang er die angelehnte Tür auf und marschierte in ihr Inneres. Der Geruch nach Drachenbohnen stieg ihm in die Nase und ließ seinen Magen knurren, der sich die letzten Wochen primär mit Apfelnüssen und Sternkraut hatte begnügen müssen. Leander ignorierte den Eindringling in seinem Heim, so wie der Eindringling ihn ignorierte. So stapfte der Seher am Kamin vorbei, vor welchem er ein unheilvolles Blubbern vernahm, legte seinen Mantel ab, schlüpfte aus dem darunter liegenden varatanischen Brustpanzer und warf sich in seinen Schaukelstuhl.

„Nicht dorthin!“, ertönte Rekas warnende Stimme, „Maro genießt das Schlafplätzchen.“

Leander korrigierte seine Flugbahn und schrammte bloß gegen die Lehne des Schaukelstuhls, woraufhin ein wütendes Zischen von Rekas Schlange ertönte.

„Ich hoffe, ich störe dich nicht, Leander“, sprach Reka fröhlich weiter, „Ich brauchte ein Plätzchen, um dieses Rezeptlein zu perfektionieren, und du besitzt nun mal einen der besten Zwergenstahlkessel, die dieses Land zu bieten hat. Als du nicht anzutreffen warst, dachte ich, ich riskiere es. Ertappt!“

Leander konnte sich ein Zucken seiner Mundwinkel nicht verkneifen, dass er das Schweigeduell gegen die Kräuterhexe gewonnen hatte. Und er wusste es besser, als einen Streit mit Reka über die Nutzung seines Kessels anzuzetteln. Die Hexe wäre doch mitten in der Nacht in seine kleine Hütte gestürmt und hätte ihn aus dem Bett geschleudert, wenn ihr der Kopf danach stünde, seinen Kessel zu benutzen. Immerhin hinterließ sie üblicherweise als Ausgleich dafür ein kleines Geschenk wie ein Tränkchen oder eine uralte Schriftrolle, die Leander dann beim Baum der Lieder entziffern lassen konnte. Sie wusste, an welchem Wissen Leander sich erfreute, und so tolerierte er ihre Gegenwart. Auch wenn sie für seinen Geschmack zu viel vor sich hin brabbelte.

„Ich bin auch hier!“, meldete sich eine helle Stimme zu Wort, die definitiv nicht zu Reka gehörte, „Ich bin Chada, Rekas Schülerin. Also, vermutlich wisst Ihr das ja bereits. Reka meinte, es wäre in Ordnung, wenn...“

Leander blendete den Rest von Chadas Geplapper aus und knirschte die Zähne zusammen. Chada, die Heldin von Andor, deren herannahende Pfeile das letzte gewesen waren, was Taroks glutrote Augen erblickt hatten. Chada, die Heldin von Andor, die einen Kundschafter der Krahder aus dem Süden lieber erschlagen hatte, statt ihn gefangenzunehmen und nach Informationen auszufragen. Chada, die Heldin von Andor, die selbstgerecht Hademars Kenntnisse über die Verbindung von Dunkler Magie und Dunkler Hexerei aus dieser Welt gelöscht hätte, wenn sein Geist nicht glücklicherweise ohne seinen Körper hätte überleben können.

Nein, Leander und Chada hatten wahrlich nicht dieselbe Weltanschauung. Auch wenn Leander natürlich froh darüber war, dass die Helden von Andor diese Lande vor großen Übeln verteidigten, konnte er sich einfach nicht mit ihrer stolzen Zerstörungswut anfreunden. Es war, als kümmerten sie sich nicht einmal darum, dass mit jedem Streich ihrer Schwerter und jedem Pfeil im Herzen eines Gegners bessere Handlungsoptionen verloren gingen, tiefgehendes Wissen verloren ging, die Chance auf eine bessere Zukunft schwand. Und Chada war die stolzeste Heldin von allen, mit ihrem ehrwürdigen Auftreten und ihrer unüberwindbaren Willenskraft. Leander schüttelte seinen Kopf.

„Bis morgen seid ihr drei verschwunden, in Ordnung? Und bis dahin möget ihr eure Worte reduzieren. Mir steht der Kopf gerade nicht nach einem Schwatz“, murmelte Leander müde und ließ sich in einen anderen Stuhl sinken.

„Wann tut er das je?“, lachte Reka und warf irgendetwas in den Topf, woraufhin das ‚Wumpf‘ einer semi-großen Stichflamme zu vernehmen war.

„Fackel mir nicht das Dach ab!“, konnte sich Leander nicht verkneifen, „Das Rietgras ist staubtrocken und dieser Unterschlupf hat mich einiges gekostet.“

Die Hexe kicherte und schlug mehrmals mit einem metallischen Gegenstand gegen den Kochtopf.

Leander versuchte, seine strapazierten Glieder etwas zu entspannen, doch konnte er in der Gegenwart der hektisch umherhantierenden Hexe und ihrer etwas leiser, aber nicht weniger hektisch umhereilenden Schülerin kaum zur Ruhe kommen. Innerlich brannte er schon darauf, den Edelstein in seiner Manteltasche zu untersuchen, doch konnte er das kaum, während die alte Reka in seinem Haus umherhuschte.

„Ein sonderbarer Stein ist es, den du da bei dir trägst“, unterbrach Reka seinen Gedankengang, „Willst du ihn mir mal zeigen? Ich könnte dir verraten, welche Formen der Schatten unter seiner Oberfläche nimmt. Das ist wohl eine Grundlage für eine erfolgreiche Kommunikation mit der darin eigesperrten Seele.“

Wovon wusste diese Hexe denn nicht Bescheid?! Leander winkte dankend ab und zog zum Beweis das Drachenherz hervor. Chada machte ein staunendes Geräusch, auch wenn sie unmöglich wissen konnte, was in diesem Stein vorging.

„Sei still und lausche“, sprach Leander zu Reka, „und du wirst das leiseste Summen vernehmen. Schon bald wird er den Stein soweit unter Kontrolle haben, dass er ihn vibrieren lassen und so zur Kundgebung verwenden kann. Augen sind nicht vonnöten, um Töne zu hören. Selbst, wenn er sich nicht differenziert auszudrücken vermag, könnte ich ihm zum Beispiel den Code des Seefahrers Morseus beibringen. Ich brauche dich nicht, um mit ihm zu sprechen.“

„Wie du meinst“, antwortete Reka und klatschte irgendetwas, das nach einem großen Kochlöffel klang, in den Kessel. Sie flüsterte irgendwelche Anweisungen zu Chada, welche auf eine bestimmte Art im Kessel zu rühren begann, während Reka ein bisschen weiter weg in etwas Raschelndem herumkramte.

Leanders Gedanken schweiften wieder zu Varatans Fluch. Heute würde es zu spät sein, aber am nächsten Morgen könnte er zum Baum der Lieder aufbrechen und einen der dortigen Priester nach dem Stammbaum des Seekönigs fragen. Bestimmt wusste jemand etwas darüber, wie viele Kinder und Kindeskinder Varkmar gehabt hatte.

„Eigentlich“, setzte Reka an und unterbrach Leanders Gedankengang, „Eigentlich gibt es noch einen Grund, warum ich hier bin.“

Leander hätte sich schon innerlich an die Stirn geschlagen, wäre da nicht Rekas urplötzlich ernste Stimme gewesen, die ihn aufhorchen ließ.

„Eine alte Frau mit zu viel Zeit kommt viel um in diesen Landen, und da vernimmt man natürlich die einen oder anderen Gerüchte.“

„Was für Gerüchte hast du vernommen, Reka?“

„Man munkelt, Meres marschiere wieder durch diese Gefilde“, stieß Reka hervor, „Du hast ein Talent dafür, mächtige Gestalten aufzuspüren, Leander. Falls du ihn sehen solltest, könntest du ihm ausrichten, dass ich ihn gerne sehen würde? Es gibt da etwas, was ich zu berichtigen habe.“

Leander legte seinen Kopf schief. Das war ja aufrichtig interessant. Meres der Hexer war seines Wissens nach Rekas Schüler gewesen, dessen Hexerei die Hexenkunst Rekas bei weitem übertreffen konnte, aber stets unvorhergesehene Folgen mit sich zog. Meres hatte sich vor mehr als einem Jahrdutzend mit seiner Lehrmeisterin zerstritten – Rekas Hütte war dabei draufgegangen – und war im Zorn außer Landes gezogen. Nach dem Tod des Drachen war er zurückgekehrt, hatte ein Dutzend Tulgori aus dem bislang unbekannten Land westlich des Fahlen Gebirges mitgebracht, aus Versehen die Statuen des Bruderkrieges zum Leben erweckt und sein Leben im Kampf gegen diese Kreationen aus Stein riskiert, vielleicht gar gelassen.

Leander war nie davon überzeugt gewesen, dass Meres damals in Cavern umgekommen war. Er konnte es dem Hexer nicht verübeln, die Helden von Andor damals verlassen zu haben, nachdem sie ihn keines Blickes gewürdigt hatten. Und Reka hatte recht: Wenn Meres nun zurückgekommen war und Reka und die Helden mied, so hatte Leander wahrscheinlich die beste Chance, ihn aufzuspüren. Leander konnte den Mann zwar nicht gut durchschauen, aber er wusste, dass es einen Weg geben musste, ihn nach Narkon zu schicken. Würde ein so mächtiger Hexer wie Meres es schaffen, eine solche Kampfstärke gegen Varatans Fluch aufzubringen, dass auch der Wille aller Nachkommen Varatans gemeinsam nicht ausreichen würde, um ihn weiter wirken zu lassen?

Reka sprach hastig weiter: „Nein, tu... vergiss einfach, was ich gesagt habe. Es freut mich, dass du wieder im Lande bist, Leander.“

Die Kräuterhexe in Verlegenheit? Das war auch eine Seltenheit. Leander versuchte vergeblich mit seinem müden Geist, sich einen Reim darauf zu machen. Reka stapfte zur Tür hinaus, woraufhin Maro, die treue Seele, von Leanders Schaukelstuhl plumpste und der Hexe zischend hinterherschlängelte.

Chada wandte sich Leander zu, entschuldigte sich wortreich dafür, in seine Hütte eingedrungen zu sein, versprach, dass sie ihm dafür einen Gefallen schuldig wäre, und schlich dann leise zur Tür hinaus, einen netten Abschiedsgruß auf den Lippen.

Leander blieb noch einige Momente in seinem Sessel, um sicherzugehen, dass die Hexe nicht einfach kurz Wasser lassen war. Als sie und ihre Begleiterinnen nach dem Viertel einer Stunde noch nicht zurückgekehrt waren, übermannte ihn allerdings seine Neugierde und er erhob sich, um vor seinen Kamin zu schlurfen und nach dem Kessel zu tasten.

Der Dampf über dem Kessel roch lecker und unbedrohlich. Leander traute sich, einen Löffel in die Flüssigkeit zu tauchen und daran zu nippen. Seine Geschmacksnerven frohlockten, als er in Rekas Gesüpp einen der besten Eintöpfe erkannte, die er in seinem Leben je gekostet hatte.

Hatte sie ihm die ganze Zeit einfach nur ein Gericht gekocht? Ein Geschenk Rekas dafür, dass er Meres übermittelte, dass Reka ihn sehen wollte?

Versteh einer diese Hexen.

Nichtsdestotrotz war Leander dankbar für ihre Gabe.\bigskip







Wie geplant reiste Leander am nächsten Tag an den Baum der Lieder. Hohepriesterin Gända hatte frei und war so nett, ihn ins Archiv zu führen.

„Sagt noch einmal, nach wem Ihr sucht.“

„Varatan, der Seekönig. Ich nehme doch stark an, dass Ihr hier Aufzeichnungen über seine Familie lagert?“

„Selbstverständlich“, erwiderte Gända und Leander hörte, wie sich ihre Schritte hurtig entfernten.

„Warum interessiert Ihr euch für den alten Meereskönig, wenn ich fragen darf?“, erklang ihre durchdringende Stimme von der anderen Seite eines Regals.

„Ihr Jungspunde und Eure Neugier“, rief Leander aus. Er mochte es, einen mürrischen alten Gelehrten zu spielen und er wusste, dass diese Bezeichnung Gända ein bisschen Perspektive verleihen konnte. Gända war eine der altehrwürdigsten Bewahrer des Ordens und sehr stolz darauf, doch konnte sie nicht leugnen, dass Leander den Baum der Lieder schon aufgesucht hatte, als sie noch nichts als eine junge Novizin gewesen war.

„Ich habe kürzlich in einem Manuskript von ihm gehört“, flunkerte Leander das Blaue vom Himmel herunter, „Es gab ja schon viele Meereskönige von Varatan, aber da sein Reich unter seiner Herrschaft neu aufblühte und gar nach ihm ‚Varatanien‘ genannt wurde, lässt sich sein außergewöhnlicher Einfluss auf die Kultur der Nebelinseln nicht verweigern. Halb Werftheim will doch mit ihm verwandt sein! Und doch ist mir nur von einem einzigen Kind bekannt, das er in die Welt gesetzt haben soll. So etwas weckt doch das Interesse eines jeden wahren Gelehrten, wie wir es beide sind.“

„Gefunden!“, ertönte Gändas kühle Stimme. Ihre leisen Schritte auf dem Holzboden inmitten des Baums der Lieder kamen wieder näher.

„Varkmar, der erste Zauberer des Feuers, war der erste und einzige Sohn des Varatan.“

So weit, so gut, ganz wie erwartet, dachte Leander. Ein einziger Sohn, der inzwischen wahrscheinlich ohnehin schon tot war.

„Fahrt fort, mein Kind“, sprach er angespannt.

„Nun, viel ist über Varkmar nicht verzeichnet“, murmelte Gända beschämt, „Es gab stark widersprüchliche Schriften über sein Leben, aus den stark gefärbten Perspektiven der großen Zaubererorden Hadrias. Ich muss Euch wohl nicht berichten, wie konfliktreich es damals in Hadria zuging.“

„Irgendwelche neutralen Perspektiven?“, fragte Leander.

Er hörte es rascheln.

„Es gäbe hier einen Bericht, den die Heldin Eara aus dem Hohen Norden einst niedergeschrieben hatte. Das war zur Zeit, als sie einige Zeit lang hier in diesen Archiven hauste und unsere Aufzeichnungen zu den verschiedensten Zaubersprüchen studierte“, hängte Gända nicht ohne Stolz in der Stimme an.

„Eine Zauberin des Turms, und dann auch noch lange Zeit, nachdem es geschah?“, fragte Leander. Es war verblüffend, dass Gända auf eine solche Quelle überhaupt Wert legte.

Ein wenig spitzer als vorher fuhr Gända fort: „Nun, Eara ging sehr differenziert auf... jedenfalls berichtet sie von Varkmars Familie. Er hatte eine Tochter und einen Sohn.“

Leander horchte auf. Bei zwei Enkeln lag es auch noch im Bereich des Möglichen, beide Blutlinien aufzuspüren und auszulöschen.

„Und ihre eigenen Familien?“

„Nun, die Tochter...“, raschelte Gända erneut in den Pergamenten herum, „Die Tochter, Nika, verstarb jung, als Varkur das Siegel zum Eisernen Turm sprengte und die magischen Waffen zu erringen versuchte.“

Leander kam schwach eine Erinnerung auf, dass er diese Geschichte vermutlich schon einmal gehört hatte. Eine menschliche Schwäche war es, auf die er nicht stolz war, aber seitdem er Hadria verlassen hatte, hatte er sich nicht mehr groß um diese abgeschiedenste Insel des Hadrischen Meeres gekümmert. Er verband einige gute Erinnerungen mit dem Ort, aber noch mehr schlechte. Das hatte das traumatische Miterleben eines massenmörderischen Mobs so an sich. Leander schauderte beim Gedanken daran, wie Orweyn einst durch Hadria gewütet war.

„Und der Sohn?“, fragte er.

Gända stockte, und meinte dann: „O Leander... der Sohn Varkmars ist doch Varkur, der Dunkle Magier. Wusstet Ihr das nicht?“

Leander blieb stumm. Dann schlug er sich gegen seine Stirn. Wie hatte er so blind sein können?! Flammensohn, Flammenmeer, Flammentod! Es ergab solchen Sinn, Varkur passte so natürlich in die Abfolge.

Leander hatte von Varkur bislang nur als einen sich der Dunklen Magie verschrieben habenden Zauberer aus Hadria gedacht, der in der weiten Welt seine Chance auf Herrschaft und Rache suchte. Nun sah er, wie falsch er gelegen hatte. Varkur war weit mehr als das. Varkur trug das Erbe Varatans in sich, und stärkte den Fluch über Narkon mit seinem übermächtigen Willen.

Leander murmelte einige Dankesworte an Gända und verabschiedete sich von ihr. Seine Gedanken waren nicht mehr bei der Sache, sondern kreisten nur noch um einen: Varkur, den Enkel Varatans. Schon seit jungen Jahren aus Hadria verbannt. Höchstwahrscheinlich kinderlos. Und ungemein gefährlich.

Varkur musste sterben, damit Callem befreit werden konnte.

Und Leander würde dies irgendwie einrichten müssen.








\newpage
\section{Alte und neue Pläne}


Während Hademar im Laufe der nächsten Wochen langsam lernte, mit seinem kristallenen Gefängnis umzugehen und das Drachenherz zum Ruckeln zu bringen, verbrachte Leander den Großteil seiner Zeit am Baum der Lieder. Die Hoffnung, Callem zu befreien, hatte neuen Elan in sein vorher teilweise zielloses Leben gebracht und ließ ihn nun Gända, Tion und alle anderen Bewahrer, die sich seiner annehmen wollten, nach Texten über Varkur ausfragen.

Varkur war Leander vorher nie sonderlich interessant erschienen – ein Magier aus dem Hohen Norden, der hier und da auftauchte und vergeblich versuchte, seine Macht zu mehren. Doch nun, da Leander von Varkurs Verbindung zu Varatan wusste, war jeder noch so kleine Hinweis auf seinen möglichen Aufenthaltsort wertvoll.

Seinen letzten großen Auftritt hatte der Dunkle Magier gehabt, als er in der Ära des Sternenschilds den Helden den Sternenschild vor der Nase wegzuschnappen versucht hatte. Dieser letzte Angriff war wohl ein verzweifelter Plan des Hadriers gewesen, sein Schicksal zum Guten zu wenden, nachdem die Helden Tarok aus dieser Welt befördert hatten. Einer ziemlich ausführlichen Aufzeichnung über die erste Eroberung der Rietburg zufolge war Varkur überhaupt erst nach Andor gekommen, um den Drachen Tarok zu zähmen und gen Hadria zu reiten. So hatte er Rache an den Zauberern nehmen wollen, die ihn einst verstoßen hatten. Doch sein Hadrisches Stundenglas war zerborsten und Varkur hatte ein neues Ziel für seinen Hass gefunden: Brandur und die Helden von Andor. Hier und dort hatte er versucht, Taroks Kreaturen umzulenken und für seine Pläne zu missbrauchen, aber nie waren diese Pläne von Erflog gekrönt gewesen. Und dann waren Brandur und Tarok gestorben, und Varkur hatte plötzlich zwei große Ziele weniger gehabt.

Nun, nachdem die Helden ihm den Sternenschild entrissen hatten, hatte er wohl definitiv einen neuen Plan finden müssen, mit dem er seine gewünschte Rache an den Helden und an Hadria vollziehen könnte. Aber seltsamerweise hatte er seit damals keine weiteren offen Attacken mehr durchgeführt.

Mit gefurchter Stirn ging Leander aber und abermals die gehörten Textstellen in seinem Kopf durch. Varkur befand sich höchstwahrscheinlich nicht mehr in Andor, sonst hätten die Helden ihn bereits aufgespürt. Aber er würde doch auch nicht einfach irgendwo ins Gebirge geflohen und dort gestorben sein, oder? Er musste ein Versteck besitzen und an seinen nächsten Plänen feilen, und was seine Pläne auch waren, es mussten große sein. Nun, wo Tarok nicht mehr war, wagte sich Varkur vielleicht an den Urtroll? Könnte er diesen etwa ohne zusätzliche Zeit eines hadrischen Stundenglases zu bändigen lernen? Oder hatte er sich analog zu Meres ins ferne Tulgor zurückgezogen, um dort neue Technologien zu meistern?

Leander griff gedankenverloren nach dem Drachenherz und schüttelte den Edelstein, woraufhin ein leises Klingen ertönte. Hademar reagierte!

„Was meinst du, wo sich dein Varkur aufhält?“, fragte Leander den Stein leise. Hademars Singsang schwankte unkontrolliert und brach dann ab. Noch war er nicht soweit, sprechen zu können. Aber das war in Ordnung. Gut Ding will bekanntlich Weile haben. Früher oder später würde er rudimentär mit der Außenwelt kommunizieren können. Und Leander würde da sein, um ihm zuzuhören.

Hademars Geist war tatsächlich eine von Leanders letzten Hoffnungen darauf, Varkur aufzuspüren. Schließlich waren der aus Krahd geflohene Nekromant und der aus Hadria verbannte Dunkle Magier lange Zeit Komplizen gewesen. Die genaue Natur ihrer Beziehung blieb Leander fürs Erste verborgen, da kaum mehr als ein einzelner von den Helden abgefangener Brief Hademars an Varkur von ihrer Bekanntschaft zeugte. Wenn diesem uneingeschränkt Glauben zu schenken war, so hatte Hademar nach seiner Flucht aus der Gefangenschaft der Krahder dem Land Andor, das von seinem ihn zurückgelassen habenden Bruder Brandur regiert wurde, den Rücken zugekehrt und war nach Hadria aufgebrochen. Die Krahder hatten sein Talent für die Dunkle Hexerei geweckt gehabt und nun hatte es Hademar danach gedürstet, mehr zu erlernen.

Doch die Zauberer des Turms hatten Hademar abgewiesen. Kein Wunder, diese elenden Narren fürchteten sich doch vor allem, das potentiell gefährlich sein konnte, also nebst ihren eigenen Schatten auch vor so gut wie allen frischen Ideen. Im aus der Feste von Yra verstoßenen Varkur hatte Hademar allerdings einen Seelenverwandten gefunden. Die beiden hatten viel Zeit in Nordgard verbracht und Rachefeldzüge gegen die Zauberer der Feste von Yra geplant. Varkur hatte Hademar Zugang zur Bibliothek der Akademie von Hadria verschafft und Hademar Varkur vom Drachen Tarok tief unter dem Grauen Gebirge erzählt. So waren sie gemeinsam ausgezogen. Während Varkur den offenen Kampf mit Andor gesucht hatte und prompt unterlegen war, hatte sich der Pfad des vorsichtigeren Hademars schon früher von Varkurs getrennt. Der Nekromant hatte sich zunächst in der Winterburg eingerichtet, wo er die umliegenden Agrenvölker versklavt und langsam Drachenmagie aus den Zwergentürmen im Grauen Gebirge gezogen hatte. Letzten Endes war Hademar erheblich stärker gewesen als Varkur, aber selbst das hatte ihn nicht vor dem Zorn der Helden retten können.

Und nun lag Hademars Geist in Leanders Hand, wusste vielleicht, wo Varkur sich aufhielt, konnte es Leander aber nicht verraten. Noch nicht. Doch Leander war geduldig. Er konnte warten. Und Varkur würde ihm nicht davonlaufen.\bigskip







Ironischerweise spürte Leander nie Varkur auf. Stattdessen spürte Varkur Leander zuerst auf.

Das erste, was Leander spürte, war ein leiser Schauer, der seinen Arm entlangglitt. Er richtete sich in seinem Bett auf und hoffte, dass es nicht schon wieder Shan wäre, die ihn in seinem Schlaf störte.

Es war nicht Shan. Soviel war urplötzlich klar. Leander spürte ihn herannahen. Jedes Haar an seinem Körper stellte sich auf und eine Schwere überkam seinen Körper von allen Seiten zugleich. Dann begann die Pein. Die gesamte Länge von Leanders Arm verkrampfte sich und flammende Schmerzen strahlten bis in seinen Kopf. Leander hatte genug von Varkurs Angriffen gehört, dass er dieses Phänomen als den schmerzinduzierenden schwarzen Nebel erkennen konnte, mit dem sich der Dunkle Magier nur allzu gerne umgab.

Mit zusammengebissenen Zähnen versuchte Leander, den Schmerz zu ignorieren und seinen Gegner zu lokalisieren. Er drehte seinen Kopf in die Richtung, aus der er Schritte zu hören glaubte, und stolperte dann in entgegengesetzter Richtung aus seinem Bett.

Für einen kurzen Augenblick ließen die Schmerzen nach, dann war der Nebel des Magiers wieder um ihn herum, noch stärker als vorher. Der Mantel der Agonie stülpte sich über Leander, riss den dürren Seher in die Höhe und schleuderte ihn zurück auf seine Schlafstelle. Leander hörte etwas knirschen. Hoffentlich war das nur das Bett gewesen und nicht sein Rücken.

Dann war da urplötzlich ein Gestank nach Blut und faulem Fleisch, den Leander sonst nur von Kreaturen kannte. Er brauchte einen Augenblick, um zu erkennen, dass das Varkurs rasselnder Atem war. Der Dunkle Magier befand sich direkt vor ihm!

„Wo ist er?“, sprach eine zischende Stimme leise und langsam, „Wo ist Hademar?“

Leander hätte beinahe aufgelacht. Wenn er gewusst hätte, dass Varkur sich noch um seinen ehemaligen Komplizen sorgte, hätte er Hademars Kristall noch viel prominenter als Lockmittel eingesetzt. Nur war dies eine gänzlich unerwünschte Konfrontation, denn auch wenn er Varkur hatte aufspüren wollen, war Leander im Moment planlos und hoffnungslos davon entfernt, Varkur zu töten. Seine Priorität musste nun darin liegen, diese Situation mit seinem Leben zu verlassen. So griff Leander ächzend in seine Manteltasche und präsentierte Varkur gefügig das Drachenherz, den weißen Runenstein mit Hademars Seele darin.

„Töte mich nicht, o Sohn des Varkmar“, stieß Leander hervor, „Ich bin der Einzige, der Hademars Geist wieder aus diesem Stein befreien kann.“

„Du unterschätzt mich“, hauchte die kalte Stimme Varkurs. Schuppige Klauen klaubten Leander das Drachenherz aus klammen Fingern. Dann verdichtete sich der schmerzvolle Griff von Varkurs Nebels um Leander Kehle. Dieser ächzte auf und schnappte nach Luft, versuchte vergeblich, weitere Worte hervorzubringen, doch stattdessen floss nur noch mehr des betäubenden dunklen Nebels in seinen Hals hinein und stoppte seine Luftzufuhr.

Leander wusste, dass er nur wenige Augenblicke bei Bewusstsein bleiben konnte. Er musste handeln, und zwar schnell. Im direkten Kampf gegen den Dunklen Magier hatte er keine Chance, aber so...

Mit seiner letzten Kraft hob Leander seine Hand und stach mit spitzen Fingern dorthin, wo er Varkurs Gesicht vermutete. Er schrammte gegen Schuppen und Horn, doch dann traf er auf etwas Glibbriges, das Leander als Varkurs Augen identifizieren konnte.

Der Dunkle Magier heulte auf und schleuderte Leander von sich. Wohl wissend, dass dies seine letzten Sekunden in dieser Welt sein könnten, ignorierte Leander seinen protestierenden Körper, hob sich in die Höhe und stolperte zur Tür hinaus in den nächtlichen Wachsamen Wald hinein.

Schon hatte sich Varkur von der Überraschung erholt und brüllte ihm etwas dämonisch Klingendes hinterher. Leander indes erhob seine eigene Stimme und schrie:

„Hilfe! Hilfeeeeeee!“

Der Wachsame Wald blieb stumm, bis auf eine Eule, welche weit entfernt antwortete. Leander wägte ab, ob er erneut schreien sollte, entschied sich dann dagegen, weil dies Varkur auf seine Position aufmerksam machen könnte, und rannte vom ihm bekannten Waldpfad weg.

Blind stolperte der Seher durch das Unterholz, schrammte hier und dort gegen Äste und Wurzeln, die ihm ins Gesicht schlugen und seine Füße beinahe zum Stolpern brachten. Beinahe.

Die Hände ausgestreckt, versuchte Leander, sich zumindest ein klein wenig vor Hindernissen zu bewahren. So kam es, dass Leander immerhin nicht kopfsvoran mit einem Baum kollidierte, sondern den Aufprall etwas abfedern konnte. Trotzdem schleuderte ihn dieses unglückliche Treffen zu Boden, wo er hustend und prustend nach Luft rang, während die Schmerzen aller Schnitte und blauen Flecken, die der Wald ihm verpasst hatte, auf ihn einprasselten und nur schwer zu ignorieren waren.

Wie weit hatte er sich von seiner Hütte entfernt? War Varkur ihm auf den Fersen?

Seine Brust hob und senkte sich schnell, während Leander seinen Atem kontrolliert beruhigte und sich so flach wie möglich an den Baumstamm kauerte, gegen den er gerannt war.

Vielleicht war er Varkur entkommen. Vielleicht war dieser so sehr auf Hademar fokussiert, dass ihn Leander kaum kümmerte, oder zumindest so lange nicht kümmerte, dass Leander hatte entkommen können.

Leander Herz sank in sein niederes Nachtgewand, als sein Körper von einem erneuten Schauer erfasst wurde und etwas Schmerzvolles, Nebliges sich um seine Hüfte schnallte. Mit einem Ruck wurde er aus dem Unterholz gerissen und strampelte vergeblich. Seine Füße streiften kaum mehr die oberen Enden der Sträuche.

„Du dachtest doch nicht wirklich, dass es so leicht wäre, einem Dunklen Magier zu entkommen?!“, lachte Varkur finster. Seine Stimme klang ein wenig weiter weg als vorhin. Er hatte gelernt, Leanders Finger seinen Augen nicht zu nahe kommen zu lassen.

„Ich... ich kannte deinen Vater“, stieß Leander verzweifelt hervor, „Und deinen Großvater! Ich könnte dir von ihren Geheimnissen verraten, dir Macht liefern. Hademar weiß, wie man den Tod umgehen kann. Wünschst du dir nicht, deine Schwe...“

Für einen kurzen Blick hatte sich der Griff um Leanders Körper gelockert, doch nun wurde er noch fester als vorhin. Varkur brüllte hasserfüllt: „Ich habe keine Väter! Und ich wünsche mir keine Händel mit dem Tod! Ich bin eine Kreatur der Finsternis! Alles Menschliche in mir wurde schon vor Urzeiten ausgelöscht!“

„Mit Verlaub, das ist eine Unwahrheit“, erklang eine tonlose aber nicht unfreundliche Stimme. Leander kannte sie nicht. Varkur offenbar auch nicht, denn der Dunkle Magier fuhr herum und knurrte:

„Wer da?!“

Der Nebel des Schmerzes um Leander löste sich plötzlich in Luft auf, woraufhin der Seher unsanft auf den Boden knallte. Sein Schädel brummte, doch stemmte er sich ungeachtet dessen in die Höhe und stolperte so rasch wie nur möglich von Varkur weg.

„Dunkle Magie ist in diesen Landen nicht gern gesehen. Und nächtliche Überfälle erst recht nicht. Ich muss dich bitten, dich unverzüglich zu ergeben, Unruhestifter“, erklang dieselbe monotone Stimme erneut. Kein Bewahrer wäre so dreist, sich Varkur direkt im Kampf zu stellen.

Varkur lachte auf und Leander hörte einen gellenden Aufschrei. Vermutlich hatte Varkur seinen Nebel auf den Neuankömmling fokussiert. Leander dankte seinem scheinbar todesverliebten Helfer innerlich und zog sich vorsichtig weiter aus dem Kampf zurück. Da erklang das ‚Wumpf‘ einer Stichflamme und der Schrei des Neuankömmlings versiegte. Stattdessen vernahm er, wie Varkur auffluchte.

Etwas knisterte und den Gestank verbrannten Fleischs drang an Leanders strapazierte Nase. Dann ein weiteres ‚Wumpf‘, das Rascheln von Stoff und... waren das Katzen? Ja, das war eindeutig das Miauen dutzender Katzen, direkt von der Stelle, wo er den Kampf zwischen Varkur und dem offensichtlich magiebegabten Neuankömmling vermutete. Leander kratzte sich am Kopf.

Das Miauen ließ nicht nach, dafür das Geräusch magischer Attacken, welches urplötzlich versiegte. Leander duckte sich auf den Boden und versuchte, so unsichtbar wie möglich zu bleiben. Leise Schritte näherten sich ihm ungeachtet seines Versteckversuchs. Einerseits verriet ihm das, dass er trotz seiner Mühen von weitem sichtbar gewesen war. Andererseits wusste er, dass Varkurs Schritte nicht so klangen. So vermutete Leander, dass es der mysteriöse Neuankömmling war, der sich auf ihn zubewegte. Und er hatte schon eine Vermutung, um wen es sich handeln könnte.\bigskip







„Der Magier hat sich verzogen. Ihr seid in Sicherheit“, erklang die tonlose Stimme des Fremden. Gedankenverloren fügte er hinzu: „Eigentlich wollte ich ihn in seinem eigenen Nebel eingangen. In der Steppe Tulgors fand ich eine Blume, deren Saft Wasser- und auch Säuretropfen blitzschnell einfrieren können. Ich hatte keine Ahnung, warum auf einmal all diese Katzen aufgetaucht sind. Eine faszinierende Reaktion...“

Seine Stimme brach ab, als sein Fokus sich auf etwas anderes richtete, sein Mantel und urplötzlich ein lautes Schnurren vom Boden ertönte. Die Pause erlaubte Leander, sich hoheitsvoll aufzurichten und zu vermuten: „Seid Ihr Meres, der Hexer von Andor?“

Sein Gegenüber klang das kleinste Häuchlein von überrascht, als es antwortete: „Den Hexer von Andor, so kennt man mich hier noch? Ja, ich bin Meres, oder zumindest war ich das einst. Ich bin mir nicht mehr sicher, wer ich bin. Meres. Haamun. Ein Schüler. Ein Hexer. Ein Reisender. Ein Unruhestifter. Ein Beschützer.“

„Das klingt alles sehr nobel“, bekräftigte Leander abgelenkt, während er sich sein malträtiertes Nachthemd glattstrich, „Ich bin Leander der Weise, und ich danke dir außerordentlich für dein tapferes Eingreifen. Ohne dich gäbe es mich wohl nicht mehr. Lass mich dich zum Dank in meine Hütte bitten und dir eine Gegenleistung überreichen.“

Meres blieb eine Zeit lang still und murmelte dann: „Die Helden von Andor verlangen keine Gegenleistung für ihre Dienste, oder?“

Leander blickte ihn überrascht an: „Meinst du etwa, du wärst... öhm, nein, höchstens einen deftigen Eintopf als Dank habe ich sie schon annehmen sehen. Und natürlich die königlichen Kopfgelder für überwundene Kreaturen.“

Unweigerlich schweiften Leanders Gedanken zurück zu demjenigen Tag, als er mit Hademar in der Manteltasche aus dem Grauen Gebirge zurückgekehrt war und Reka in seiner Hütte aufgefunden hatte. Er hatte ihr damals versprochen gehabt, Meres auszurichten, dass sie mit ihm reden wollte. Vermutlich hatte die Hexe im Sinn, ihren Zwist mit ihrem ehemaligen Schüler beizulegen, und wer konnte es ihr verübeln?

Es stand zu vermuten, dass Reka früher oder später davon erfahren würde, wenn Leander Meres Rekas Wunsch verschweigen würde. Sein Versprechen zu brechen, würde also in einem Verlust des wenigen bisschen an Vertrauens resultieren, das die Hexe noch in ihn steckte. War ihm das die kleine Chance wert, dass Meres Varatans Fluch überwältigen konnte?

Leander überdachte die Sachlage kurz und kam zum Schluss, dass ein Mittelweg das optimale Vorgehen wäre. Er würde Meres zunächst zu Reka schicken. Aber das bedeutete nicht, dass er ihn vorher nicht auf den richtigen Pfad locken könnte.

„Hör mal, Meres, Reisender und Beschützer. Nach was suchst du hier in diesem Lande?“, setzte er an.

Stille, bis auf das anhaltende Schnurren der Katzen um Meres herums. Dann...

„Ich weiß es nicht. Ich nehme an, ich suche meinen Platz auf dieser Welt.“

„Sehr poetisch, o Hexer. Hast du diesen Platz bereits gefunden?“

„Ich habe vieles gefunden, und vieles hinter mir gelassen“, kam die monotone Antwort wie aus der Achter- oder Arcuballiste geschossen, „Aber noch an keinem Ort fühlte ich mich wirklich zuhause. Nicht viele verstehen mich. Vielleicht ist das mein Los.“

Sehr schön, dachte Leander innerlich und fuhr fort.

„Für einen jeden gibt es eine Bestimmung. Die richtigen Menschen für einen zu finden, kann den Unterschied zwischen einem leeren und einem erfüllten Leben machen.“

Leander verdrängte die Gedanken an die brüderliche Verbindung, die er und Callem geteilt hatten. Nicht einmal von Reka hatte er sich je so verstanden gespürt. Aber jetzt musste er sich auf Meres konzentrieren:

„Vielleicht könnte ich dir helfen, herauszufinden, wo du dich auf die Suche machen musst. Ich glaube zu spüren, wo sich eine gute Gemeinschaft für dich aufhalten könnte. Ich bin nämlich ein Seher.“

„Na sieh einer an, ein Seher?“, erwiderte Meres höflich, „Ich halte es für unwahrscheinlich, dass du mir zu zeigen vermagst, was nicht einmal ich selbst herauszufinden vermochte.“

„Na komm, was hast du schon zu verlieren?“

Und Leander packte Meres bei den Schultern und führte ihn zu seiner Hütte zurück.\bigskip







Varkur ließ sich langsam zu Boden sinken. Der schwarze Rauch der Dunklen Magie setzte sich um ihn herum und färbte den Schnee des Grauen Gebirges in einem dunklen Ascheton.

Er musste nießen. Schnee und Rauch stoben gleichermaßen auf und verborgen den Magier von potentiellen neugierigen Blicken.

Katzen. Ausgerechnet Katzen hatte dieser unvorsichtige Hexer herbeirufen müssen! Wie konnte das sein, dass Varkur eine solche Macht besaß, und seine schon beinahe vollständig echsenhafte Nase noch immer so stark auf diese elenden Biester reagierte?!

Erneut erschütterte ein Niesreiz Varkurs Körper und sein magischer Nebel trug die Vibrationen in seine Umgebung. Er würde zurückgehen müssen und sich um diesen halbgaren Hexer kümmern, der glaubte, mit Kräutern und Tränklein an seine Künste ankommen zu können. Er würde ihm lehren, was es hieß, sich auf ein Duell einzulassen mit Varkur, dem größten Dunklen Magier seit Orweyns Zeiten! Aber im Moment war das nicht seine Priorität.

Hatschi!

Wütend vor sich hin murmelnd riss Varkur einige Sternkraut-Büschel aus dem Untergrund und zerquetschte sie in seiner Faust, bis ein grünlicher Saft aus dieser heraustropfte. Varkur schmierte sich die nasse Masse in die tränenden Augen und zuckte zusammen, als sein Körper ihm wieder schmerzlich daran erinnern ließ, dass der Seher ihn kürzlich in ebenjene gereizten Augen gestochen hatte. Immerhin konnte der Sternkraut-Saft den Juckreiz etwas lindern.

Varkur hasste es, einen Körper zu haben. Ständig schmerzte etwas oder ging zu Bruch. Ein freier Geist, so wie Hademar es ihm vorgeschwärmt hatte, das wollte er sein. Aber noch besaß er nicht die nötigen Kenntnisse dazu, körperlos fortzubestehen. Noch musste er mit dieser Hülle klarkommen.

Varkur schloss seine schmerzenden Fenster zur Seele und tauchte erleichtert in eine angenehme Dunkelheit ein. Ohne groß nachzudenken, dirigierte er den schwarzen Rauch der Dunklen Magie, ließ ihn zerfasern, sanft die Umgebung ertasten, ihm ein Bild übermitteln.

Nicht weit von ihm entfernt krochen zwei Gorlots aus einer Höhle, nichts von seiner Gegenwart ahnend. Diese gelenkigen Gestalten waren weitaus geschickter als gewöhnliche Gors (kein Wunder, sie hatten ja Hände) und waren dazu noch gescheiter. Aber sie hatten weder die natürliche Rüstung von Gors noch das taktische Geschick von Skralen, von der Geschwindigkeit von Warkreaturen oder der Stärke von Trollen ganz zu schweigen. Sie waren schlicht und ergreifend zu gewöhnlich, als dass sie Varkur in seinem Heer gewinnbringend einsetzen könnte, und Varkur konnte nur so und so viele Kreaturenanführer auf einmal dirigieren. Aber wenn gerade nichts Besseres zur Verfügung stand...

Varkur drang in den Geist des einen Gorlots ein und dieser wurde von einem einzigen Gedanken erfasst.

Der Gorlot erhob den tarenähnlichen Kopf mit dem übergroßen Unterkiefer in den Himmel und schnüffelte. Dann jaulte er auf und rannte auf Varkur zu. Sein Kumpane grunzte verwirrt und setzte ihm dann nach. In Windeseile waren die beiden Kreaturen bei Varkur angekommen, die eine willig wartend, die andere ein rostiges Messer zückend, ein Überbleibsel aus dem Unterirdischen Krieg, das der Gorlot sich angeeignet hatte.

Der dichte schwarze Nebel der Dunklen Magie machte kurzen Prozess mit dem zweiten Gorlot, während Varkur das Drachenherz hervorholte, den einst weißen, nun blutroten Edelstein mit beiden Händen fest umschloss und versuchte, die verworrenen Fäden der Drachenmagie zu durchdringen, die Hademars Seele im Kristall einsperrten. Als ihm das wie erwartet nicht gelang, versuchte Varkur stattdessen, das verworrene Geflecht der magischen Ströme aufzunehmen und mithilfe Dunkler Magie in den schwachen Geist des Gorlots zu projizieren.

Der Gorlot blieb weiterhin von Varkurs Befehl erfüllt stehen und kratzte sich gedankenverloren am Hinterbein. Mehr geschah nicht. Varkur fluchte auf und konzentrierte sich noch stärker. Schweiß tropfte aus dem menschlichen Teil seiner Haut und perlte an seinen Schuppen ab. Es kam ihm wie eine Ewigkeit vor, bis der Gorlot endlich aufjaulte und auf seinen ungeschützten Hintern plumpste, während das projizierte Geflecht des Geists im Drachenherzens seine Gedanken überschrieb und unterdrückte.

Varkur öffnete seine gereizten Augen und erblickte durch einen Tränenschleier hindurch den Gorlot, wie er sich überrascht umsah und dann sorgsam aufrichtete. Ein krächzendes Stöhnen drang aus seiner tierischen Kehle, als Hademar seine neuen Stimmbänder ausprobierte. Dann öffnete der Gorlot seinen unmenschlich großen Kiefer, entblößte mehrere Reihen spitzer Zähne und sprach:

„Varkur, du alter Sack! Wie hast du nur so lange gebraucht, um mich zu finden?!“

Nicht einmal zu einer Erwiderung ansetzen konnte Varkur, ehe die Verbindung abbrach, Hademars Geist wieder hinter die magischen Mauern des Drachenherzens gezogen wurde und der Gorlot zusammenklappte wie eine Puppe, der man die Fäden durchgeschnitten hatte. Die Kreatur richtete sich zuckend auf, sah so erschrocken drein, wie man mit einem solchen Gesicht erschrocken dreinschauen konnte, und sprang dann keuchend und stolpernd davon.

Varkur richtete seinen Blick auf das Drachenherz in seiner Hand. Es hatte ihn schon solche Unmengen an Energie gekostet, Hademars Geist einige Lidschläge lang den Gorlot steuern zu lassen. Wie in aller Welt wollte er dies über längere Zeit aufrecht erhalten?

Nein, eine langfristige Lösung war das wahrlich nicht.

Aber genug, um mit Hademar zu kommunizieren.

Genug, um herauszufinden, wie er ihn aus diesem Gefängnis befreien konnte.\bigskip







Leander war von gemischten Gefühlen erfüllt. Einerseits war er froh darüber, Meres von Narkon berichtet und den Hexer auf die verfluchte Insel angesetzt zu haben. Es fiel Leander uncharakteristisch schwer, Meres einzuschätzen, doch schätzte er seine Chancen, Meres‘ Interesse geweckt zu haben, nicht schlecht ein.

Andererseits hatte er, der stets stolz „Ich bin ein Seher“ verkündet hatte, es wieder einmal immer wieder versucht und nicht geschafft, auch nur den Ansatz einer Vision zu empfangen. Seit dem Gespräch mit der Stammesältesten hatte er tatsächlich keine Nachricht mehr von der Zukunft empfangen und das leise Gefühl von Unfähigkeit, gepaart mit der Angst, seine Fähigkeit verloren zu haben, setzte wieder ein.

Am Ende hatte er Meres einige Drachenknochen hingeworfen und ein bisschen Hokuspokus darum und um Narkon gesponnen, nur um danach auf Rekas Wunsch, ihn zu sehen, zurückzukommen. Das war mitnichten seine überzeugendste Schauspielrolle gewesen.

Ganz zu schweigen davon hatte er bei dieser ungewünschten Begegnung mit Varkur das Drachenherz und Hademars Seele verloren. Dem Ziel, Varkur zu töten, war er keinen Schritt näher gekommen und nun wusste er mit Sicherheit, dass der Dunkle Magier nicht aus purem Glück das Zeitliche gesegnet hatte. Und da Hademar verschwunden war, konnte er jetzt nicht einmal mehr den Nekromanten nach seinen Kenntnissen oder nach Varkurs Versteck aushorchen.

Leander hoffte, dass Varkur von Meres‘ unvorhersehbaren Hexereien derart abgeschreckt worden war, dass er nicht zurückkehrte, um Leanders Leben endgültig zu beenden. Aber Leander fürchtete sich nicht sonderlich vor einer Rückkehr Varkurs. Einerseits hatte Varkur keinen persönlichen Hass auf ihm, andererseits war Varkur im Moment wohl ohnehin damit beschäftigt, Hademars Geist aus dem Drachenherz lösen zu wollen. Nomion würde nicht glücklich sein, sobald es ihm gelang. Aber das war fürs Erste nur dessen Bier, und vielleicht eines Tages das der Helden von Andor. Leander wollte im Moment vor allem eines: In sein zusammengestürztes Bett fallen und sich heilen.

Ein bisschen krustige Heilsalbe hatte Leander sicher noch übrig. Er schleppte sich in seinen Trankraum, tastete die Regale nach der passenden eingekratzten Inschrift ab und zog dann das Säcklein hervor. Ächzend ließ er sich auf dem erdigen Boden seiner Hütte nieder, öffnete das Säcklein und befeuchtete das körnige Pulver mit etwas Spucke. Es schmeckte bitter. Alsbald tastete Leander seinen Körper nach offenen Wunden ab und schmierte etwas von der körnigen Paste auf alle gefundenen Schnitte und Schrammen.

Da floss eine unnatürliche Kälte in den Raum und ließ die Wundpaste einfrieren.

„Was ist denn jetzt schon wieder?!“, murmelte Leander missmutig und zog sich in die Höhe. Es war noch nicht einmal Morgen!

Wie als Antwort darauf schlängelte sich ein eiskalter Schattententakel um seine nackten Beine und zog ihn unsanft nach vorne. Seine Wunden öffneten sich wieder und der Tentakel zerfaserte sich in mehrere kleinere Ärmchen, welche sich ebenso unsanft um Leanders wunden Körper wanden.

„Shan“, gähnte Leander erschöpft, „Wir hatten doch bereits...“

Er verstummte erschrocken. Das Drachenherz, oder besser gesagt das darin gespeicherte Licht der Welt, das ihn bislang vor Shans Zorn bewahrt hatte, war nicht mehr hier. Varkur hatte es mitgenommen. Konnte es wahrlich sein, dass die seelenhungrige Shan Leander nachgeschlichen war, nur darauf wartend, dass ihn das Drachenherz verließ?

„Jetzt komm schon, die mächtige Seele des Nekromanten ist nicht hier“, sprach Leander unwirsch, in der Hoffnung, die Schattenhexe noch auf Varkurs Spur zu lenken, „Du spürst doch bestimmt, wohin der Dunkle Magier sie bringt. Wenn du dich beeilst, kannst du den Moment abpassen, wenn er die Seele freilässt.“

Äußerlich gab er sich gelassen, ja, leicht genervt. Innerlich musste er seine Panik unterdrücken, denn ohne einen Edelstein war er Shan schutzlos ausgeliefert.

Shan antwortete lächelnd: „Keine Sorge, o Seher. Ich werde dort sein, wenn der Dunkle Magier den Schutz vom Geist des Nekromanten entfernt. Aber zuerst wende ich mich dir zu. Es kommt nicht oft vor, dass mir jemand entkommt. Du hast mein Interesse geweckt und darum werde ich dich mir nun einverleiben. Der Schatten ist hungrig.“

Wie viel musste Leander denn diese Nacht noch erdauern?! Unauffällig tastete er nach seinem Holzstab auf dem Hüttenboden. Der Stab war kein Zauberstab, nur eine Replik, die Eindruck schindete und ihm bei der Fortbewegung half. Aber solange er als stumpfe Waffe taugte, musste er gar nicht magisch sein, um etwas anrichten zu können. Leander wusste, dass die Helden von Andor Shan bei der Befreiung der Rietburg schon einmal vertrieben hatten, und die meisten von ihnen besaßen kein ausgeprägtes magisches Talent. Insofern musste es möglich sein, Shan alleine durch körperlichen Schaden das Weite suchen zu lassen. Also musste es auch für Leander möglich sein, die Schattenhexe zu vertreiben.

„Suchst du das hier?“, erklang Shans leise Stimme, in der Leander nun einen Anflug von Häme zu erkennen glaubte. Ein helles Poltern erklang, als Shan Leanders Stab ergriff und an die gegenüberliegende Wand schleuderte.

Leander stieß sich an der Wand hoch. Sein Messer lag ebenfalls nutzlos weit entfernt. Meres war fortgezogen. Er stand mit dem Rücken zur Wand und Shans Nähe ließ ihn immer mehr frösteln. Schon spürte er die verfluchten Schattenkräfte Shans nach ihm greifen, nahm wahr, wie seine Kräfte schwanden und eine große Last sich auf seinen Geist legte.

Der Seher von Narkon stürzte verzweifelt nach vorne, hinein in die Kälte. Er griff mit klammen Fingen nach dem zerfasernden Körper Shans, versuchte, irgendetwas zu fassen zu kriegen, das er festhalten konnte, dem er Schaden zufügen konnte. Shans Präsenz fing ihn weich, beinahe sanft auf und stülpte sich dann von allen Seiten um ihn, schnürte ihm die Atemluft ab. Sternchen blitzten vor seinem inneren Auge auf und er sank auf seine Knie.

„Shan“, krächzte er auf, „Ich kann dir so viel nützlicher sein, wenn du mich fürs Erste am Leben lässt.“

Die Schattenhexe ließ nicht locker. Leander wusste nicht, was er weiter sagen sollte. Damals, im Kerker der Rietburg, hatte es gewirkt, als wäre Shan für solche Argumente empfänglich gewesen. Doch nun agierte sie nur ihrem Seelenhunger nach. Leander verstand den Drang der Dunkelheit, der auf der Schattenhexe lag, und er wüsste nicht, wie er sie umstimmen sollte, wenn sie nicht auf ihren Verstand hörte.

Wahrscheinlich würde dies sein Ende sein. Vielleicht hatte er deswegen nur noch so selten Visionen empfangen in der letzten Zeit.

Leanders letzter Gedanke galt Callem.\bigskip







Da öffnete sich sein sehendes Auge und Licht strömte auf ihn ein. Leander rang nach der frischen Luft, so illusorisch sie auch sein mochte, und erblickte ein kleines Mädchen mit einem lieben Lächeln auf den Lippen, wie es durch den freien Markt hüpfte. In seinen Händen trug es einen Korb mit frischen roten Äpfeln. Leander zog die satten Farben der Früchte ein.

Seine Mundwinkel zuckten unwillkürlich nach oben beim Anblick dieses unschuldigen Lächelns. Vergessen war der klamme Griff Shans, der ihn in der Realität gepackt hatte und ihm unzweifelhaft gerade das Leben aussog.

„Rosanna! Sanna! Komm, meine Sonne!“, erklang eine freundliche Stimme. Das kleine Mädchen blickte auf und rannte zu einer großgewachsenen Frau, welche sie hochhob und fröhlich im Kreis herumschwang, einmal, zweimal, dreimal.

Da flackerte das glückliche Bild und an der Stelle des kleinen Mädchens sah Leander eine von flirrenden Schatten umgebende Gestalt mit einer finsteren Fratze. Shan war das, und Shan öffnete ihren Mund. Alle Schatten der Welt traten daraus hervor, überfluteten den freien Markt und schwemmten die fröhliche Frau davon. Die Farben des Lichts ergrauten, die Sonne schien ohne Wärme. Nichts blieb mehr übrig außer Shan, die inmitten der Dunkelheit stand und verzweifelt aufschrie. Einen solchen Schrei der Verzweiflung hatte Leander erst kürzlich bei der Stammesältesten vernommen. Sein ganzer Körper erschauerte ob des Klanges.

Oder erschauerte er wegen der Kälte, die immer noch von Shans Griff ausging? Das innere Bild verblasste und hinterließ die übliche Absenz jeglicher optischer Sinnesreize. Er hatte eine weitere Vision gehabt. Aber keine Vision der Zukunft, sondern eine der Vergangenheit. Das war neu. Hatte irgendetwas an Shans einzigartiger Schattenmagie sein sehendes Auge verwirrt und in die Vergangenheit statt die Zukunft blicken lassen? Es war egal.

Rasch zählte Leander Eins und Zwei zusammen und brachte stockend hervor: „Ro... Rosanna! Sanna! Mei... meine Sonne!“

Eigentlich änderte sich nichts, denn noch immer war Leander von allen Seite von Shans Schatten umgeben. Und doch änderte sich auf einmal alles, denn die Schatten drängten nicht mehr bedrohlich auf ihn ein, saugten nicht mehr gierig nach seinen Seelenkräften. Nein, die Schatten und die Dunkelheit waren einfach da, ohne aufdringlich zu sein. Sie schienen zu warten.

„Rosanna...“, setzte Leander erneut an, sich von seiner Vision leitend, „Du... du warst einst ein Mensch. Ein glücklicher Mensch. Ein guter Mensch. Du strahltest wie das Licht der Sonne und die Leute lächelten, wenn sie dich auch nur aus der Ferne erblickten.“

Stille. Dann wurde Leander merklich wärmer, als Shan – Rosannas – Schatten sich zurückzog und ihn auf dem Hüttenboden absetzte.

„Was sagst du?“, erklang die tonlose Stimme der Schattenhexe. Leander wusste, dass sie genau verstanden hatte, was er gestammelt hatte. Er war auf Gold gestoßen.

„Ich dachte bislang immer, du wärst nichts weiter als eine Kreatur der Finsternis, geschaffen, um zu zerstören. Aber dem ist nicht so. Du bist ein Geschenk der Sonne, des Lichts. Die Dunkelheit mag ihren Schatten über dich geworfen haben, der dich hungern und zerstören lässt. Dieser Schatten, dunkler als die Nacht selbst, mag sich vor dich geschoben haben, weil das Weltenlicht nicht da war, um ihn zurückzuhalten. Aber in deinem Kern bist du immer noch das reine Licht des Lebens. Aber in deinem Kern trägst du noch immer die junge Frau von damals in dir. In deinem Kern bist du immer noch Sanna aus den Flusslanden.“

Die Schattenhexe blieb stumm. Leander verstummte ebenfalls, und wagte nicht, sich groß zu rühren.

„Du irrst dich“, sprach Shan dann langsam, „Ich bin nicht die, von der du sprichst. Ich bin der Schatten, der schon immer ein Teil von ihr war und eines Tages endlich entfacht werden konnte.“

Varkur war wütend darüber gewesen, an seine menschliche Herkunft erinnert zu werden. Shan hingegen war ruhig, verdächtig ruhig. Und doch schien sie dem Gedanken gegenüber offen, Leander nicht sofort zu verschlingen. Irgendwo in ihrem Innern gab es einen Teil von Shan, der noch als die gute Sanna von damals erkannt werden wollte. Irgendwo in ihrem Innern gab es ein emotionales Zentrum, an das Leander appellieren könnte.

Leander versuchte es mit einer anderen Taktik.

„Nun, eigentlich ist es egal, wer oder was du es einst warst. Wichtig ist, was du sein wirst. Du trägst den Körper Sannas mit dir. Du könntest sie wieder sein. So sage mir, was ist dein Ziel, Schattenhexe? Willst du alle Seelen Andors verschlingen? Tulgors? Des Barbarenlands? Wenn du so weitermachst, kommst du irgendwann ans Ende der Welt und dein Hunger lässt sich nie mehr stillen. Dann wirst du leiden und dich dir selbst stellen müssen. Hättest du dann nicht lieber vorher eingehalten und einen Platz in dieser Welt gefunden? Wenn du hingegen Mäßigung zeigst und hin und wieder Leben bestehen lässt, kann dieses Leben für neues Leben sorgen und du im Gegenzug so lange satt bleiben, wie du existierst. Es ist noch nicht zu spät, einzuhalten.“

„Das ist alles nicht wahr“, erwiderte Shan ruhig, „Du weißt nicht, ob die Welt einen Rand hat. Ich werde mich nie mit meiner Vergangenheit auseinandersetzen müssen, wenn ich nicht will. Und es ist zu spät, um dem Schatten in mir Einhalt zu gebieten.“

Leander blieb stumm. Ihm fiel auf, dass Shan vom Schatten sprach, als wäre er doch nicht ganz Teil von ihr. Und immer noch griff sie ihn nicht wieder an. Was sollte er nun tun?

„Magst du mir noch einmal von Rosanna erzählen?“, erklang nun Shans geisterhafte Stimme, beinahe vorsichtig.

Leander nickte erleichtert und meinte dann: „Was ich in meiner Vision sah, war ein kleines, glückliches Mädchen, welches in den Landen nicht weit von hier lebte. Ich sah das Glück, dass es in die Welt setzte, und ich sah den Stolz in den Augen ihrer Eltern, als sie an Sannas Seite standen. Sanna war besonders, und nicht wegen des Fluchs, den die Dunkelheit auf sie gelegt hatte, sondern aufgrund der Liebe, die sie der gesamten Welt entgegenbrachte. Sanna mag gelitten haben, und der Schatten mag viel Schaden in ihrem Namen angerichtet haben, ja, der Schatten mag sogar ein Teil von ihr geworden sein. Aber noch ist ihre Geschichte nicht zu Ende.“

Eine Zeit lang blieb es wieder still, und Leander vermochte kaum zu hoffen, dass Shan ihn in Ruhe lassen würde. Da sprach die Schattenhexe: „Du sprichst, um dein Leben zu bewahren, o Seher. Das respektiere ich. Und doch sorgen deine Worte für ein warmes Gefühl in meinem Innern. Ein Funken in der Eiseskälte. Es ist ungewohnt, aber nicht unangenehm. Ich verstehe selbst nicht, warum, aber das muss ich auch nicht. Deine Worte spenden Trost, und das tut gut. Eines Tages werde ich deine Seele in mich aufnehmen. Aber noch nicht jetzt. Gehe von hinnen, Seher. Wenn es dich danach dürstet, den Fluch zu lösen, der dich markiert hast, so mögest du dies irgendwann tun. Aber wenn du dich ihm hingeben willst, dann werde ich da sein, um dich mit deiner neugefundenen Macht zu leiten.“

Diese Einstellung erinnerte Leander schon eher an die Shan, die er vor all diesen Jahren in der Rietburg angetroffen hatte. Er atmete tief durch und sprach: „Danke, Sanna. Ich bin auf dem besten Weg dazu, den Fluch zu brechen. Einzig Varkur muss dafür fallen.“

Ein raues Lachen ertönte. Zuerst vermutete Leander, dass es Rosanna ein weiteres Mal mit Freude erfüllte, ihren alten, vergessenen Namen zu hören. Doch dann fuhr die Schattenhexe fort: „Meinst du, mit dem Ende der Blutlinie wäre ein Fluch schon gebrochen? Du bist naiv. Der Fluch hat seit seiner Entstehung Willenskraft und Magie aus seiner Erschaffer und dessen willensstarken Nachkommen gezapft. Er kann auch alleine noch viele Jahre weiter bestehen und von seiner gespeicherten Kraft zehren. Nein, damit der Fluch gebrochen werden kann, muss sich jemand ihm stellen und seine Reserven aufbrauchen lassen. Ihn überwinden. Direkt.“

Erneut hatte Shan gesprochen, als wäre dies das Selbstverständlichste der Welt. Leanders Erleichterung ob des Überlebens dieser Begegnung mit Shan und ob des Empfangens einer weiteren Vision versiegte. An ihre Stelle traten seine üblichen kalkulierenden Gedankengänge.

Wenn Shan die Wahrheit sprach, dann würde es nicht reichen, Varkur zu töten. Zusätzlich müsste jemand nach Narkon reisen und Varatans Fluch persönlich überwinden, ihn alle seine Reserven zehren lassen. Den erschreckten Berichten der Silberzwerge nach besaß der Fluch die Fähigkeit, all jene, die er berührte, alles Wissen über die Außenwelt zu rauben. Manch ein Silberzwerg hatte seine Kumpanen zu weit in die Silberberge wandern sehen, berührt werden und dann ziellos weiterlaufen, immer weiter in die verfluchte Nebelinsel hinein, bis man sie nicht einmal mehr mit einem Fernrohr mehr erkennen konnte.

Wie konnte man einen Fluch überwinden, ohne sich berühren zu lassen? Indem man ihm davonrannte? Indem man vom Fluch sein Gedächtnis löschen und dann mit vorher niedergeschriebenen Notizen wieder auffrischen ließ? Indem man die Grenze des verfluchten Gebiets Narkons durchbrach?

Leander fiel kaum auf, dass Shan davonschlich. Sein Ziel, Callem zu befreien, war schon wieder weiter in die Ferne gerückt. Wie schon so oft zuvor haderte er mit sich selbst. Sollte er Callem erneut aufzugeben versuchen, sich anderen Projekten zuwenden? Bislang war er früher oder später immer wieder auf dieses eine Ziel zurückgefallen, egal, wie oft er sich einzureden versuchte, dass Callem ihm nichts bedeutete, dass Callem nur eine einzige Person in einer so riesigen Welt war, in der Leander einen Sinn finden konnte.

Gemeinsam in einer lebensfeindlichen Welt aufzuwachsen, konnte zwei Brüder derart zusammenschmieden. Menschliche Emotionen. Nicht einmal Leander war gegen sie gefeit.

Kopfschüttelnd sank er in einen unruhigen Schlummer.\bigskip







Am nächsten Morgen – der wohl schon eher gegen Mittag ging – schälte sich Leander nur mit Mühe aus seinem eingekrachten Bett. Sein ganzer Körper schmerzte von der Konfrontation mit gleich zwei übermächtigen Wesenheiten. Varkur und Shan hatten ihn übel zugerichtet. Ihn und seine Hütte. Seine Möbel lagen nicht dort, wo sie sollten, von seinen Pülverchen und Salben ganz zu schweigen. Und sein Geist fühlte sich schwammig und träge an.

Leanders Gedanken drehten sich weiterhin um Varkur und Narkon, ohne dass er dabei auf einen grünen Zweig käme. Er kam bloß zur frustrierten Feststellung, dass diese emotionalen Gedanken an seinen verschollenen Bruder ihn einfach nicht loslassen wollten.

Wie so oft wünschte Leander sich, einen Hinweis aus der Zukunft zu erhalten, welcher ihm einfach mitteilte, wie er Callem befreien könnte. Dieser Wunsch war einst gar einer der Gründe gewesen, warum Leander die Sprache der Zukunft überhaupt erst erlernt hatte.

Zwei Visionen hatte Leander in den letzten Monaten empfangen, bloß zwei. Die erste, nachdem Nomions Geist ihn besessen hatte, als einige eifrige Agren ihn wegen der verletzten Stammesältesten angegriffen hatten. Die zweite gerade erst gestern, als Shan ihm fast das Leben ausgehaucht hatte. Konnte es sein, dass seine stärkeren Visionen durch Kämpfe ausgelöst wurden? Oder durch lebensgefährliche Situationen?

Leander wusste natürlich, dass nicht alle lebensbedrohlichen Lagen in einer Vision resultierten. Varkurs Angriff hatte keine zur Folge gehabt. Und Leander hatte auch schon Vision in alltäglichen Momenten gehabt. Oder?

Jetzt, wo er so darüber nachdachte, war er sich plötzlich nicht mehr sicher. Er konnte sich zumindest nicht daran erinnern, je eine Nachricht aus der Zukunft empfangen zu haben, während er still in seinem Schaukelstuhl gesessen und vor sich hin sinniert hatte. Konnte es sein, dass sein Körper angeregt sein musste, um empfänglich gegenüber der Stimme der Zukunft zu sein? Waren Visionen umso wahrscheinlicher, je heikler seine Situation war, je mehr er sich fürchtete? Konnte das der Grund sein, warum die Visionen abgenommen hatten und unvorhersehbarer geworden waren, je mehr Leander die Sprache der Zukunft erlernt hatte, je natürlicher ihm diese verbotene Kunst erschienen waren?

Eine gewagte These, kein Zweifel.

Doch führte sie in Leanders Kopf zur Bildung eines Plans. Wenn er wahrlich wahrscheinlicher Visionen in Momenten der Gefahr erfuhr, so konnte er sie vielleicht mit Absicht auslösen, indem er sich einer tödlichen Bedrohung stellte. Vielleicht konnte er gar den Inhalt der Vision beeinflussen. Dies würde ihm erlauben, seine Vermutung auf die Probe zu stellen, und im Erfolgsfall zugleich endlich etwas darüber zu erfahren, wie er Varatans Fluch brechen und Callem befreien könnte. Seine letzten beiden Visionen hatten jeweils in Verbindung gestanden zur Gefahr, die ihn in diesem Augenblick bedroht hatte. Sollte Leander vielleicht eine Gefahr in Verbindung zu Varatans Fluch aufsuchen? Zu Callem?

Leander verwarf die Idee, Silberland zu besuchen und die Silberberge zu erklimmen. Dies wäre eine direkte Konfrontation mit dem Fluch, aber eine, die zu hohe Risiken barg. Niemand wusste genau, wie der Fluch agierte. Lieber würde sich Leander einer Gefahr aussetzen, die er genau einschätzen konnte und wusste, wie er sie zu neutralisieren vermochte.

Da aber kam ihm eine andere Idee: Vielleicht müsste die Gefahr gar nicht direkt etwas mit Callem oder Varatans Fluch zu tun haben. Er könnte sich auch einfach in Gefahr bringen und sich vornehmen, sich erst zu retten, sobald er wusste, was er tun könnte, um Callem zu befreien.

Um die Gefahr zu neutralisieren, würde die Stimme der Zukunft ihm dann etwas über Callems Befreiung verraten müssen.

Würden sich seine seherischen Kräfte so leicht austricksen lassen?







\newpage
\section{Vergangenes und Zukünftiges}



Reka hatte sich nicht sehr verständnisvoll gezeigt und Leander keine Phiole von Maros Gift überlassen. Am Ende hatte Leander gar heimlich bei Rekas Schülerin Chada vorstellig werden müssen – ein Fakt, der ihn nicht mit Stolz erfüllte. Chada hatte ihm liebend gerne ein bisschen von Maros Gift gezapft und überreicht (natürlich ohne zu wissen, dass dies gegen Rekas Wünsche ging). Immerhin war bei solchen Aufträgen auf die Helden von Andor Verlass.

Nun stand Leander wieder in seiner Hütte und hielt in seinen Händen eine kleine Phiole mit schwarzen Schlangengift. Vypera-Gift, um genau zu sein. Als er und Callem noch klein gewesen waren, hatten sie gerne den Stummen Wald Narkons erkundet. Nun, der aufmüpfige Callem hatte das gerne getan, Leander war ihm meistens nur hinterhergestolpert.

So war es eines Tages gekommen, dass Leander auf eine Vypera gestoßen war und sich einen Biss eingeholt hatte. Einzig Callems rasches Eingreifen hatte sein Leben damals gerettet – sein Bruder hatte die Schlange mit einem einzigen Streich seines Dolchs aus dieser Welt befördert und sich dann rasch daran gemacht, Leanders Bein abzuschnüren.

„Nicht aufgeben, Leander. Guck mich an. Bleib bei mir!“, ertönte die ruhige Stimme Callems aus seinen Erinnerungen. Leander versuchte, nicht daran zu denken, wie es Callem wohl ging. Was, wenn ihn in der Zwischenzeit eine Vypera erwischt hatte und Callem sich nicht mehr daran erinnerte, wie man mit einer solchen Wunde umging? Hatte Varatans Fluch Callem und dessen Crew schon in den Wahnsinn getrieben? Waren sie überhaupt noch am Leben? Leander verwarf die aufblubbernden Sorgen gekonnt, wie er es schon so oft getan hatte, und wandte sich wieder dem Gift von Rekas silberner Hausschlange zu.

Vypera-Gift wirkte paralysierend und konnte zum Tod führen. „Schwarzes Eis“ wurde es in den zwielichtigen Kreisen des Nordens genannt, und nur ein rechtzeitig eingenommenes Gegengift konnte einen vor größeren Schäden bewahren. Leander hatte sich fest vorgenommen, das Gegengift nicht einzunehmen, ehe er erfuhr, wie er Callem befreien konnte.

Das Gegengift stand ein bisschen abseits in einem Flakon, den Leander einem fahrenden Händler namens Naraven abgekauft hatte. Trotz dessen Anwesenheit war das Unterfangen, das Leander plante, ein äußerst gefährliches. Das schien ja von Nöten zu sein, um seine Visionen zu triggern. Es konnte nur hoffen, dass er, wenn er sich nun Vypera-Gift einverleibte, eine Vision Callems empfangen würde. Es mochte eine exakte Wissenschaft gewesen sein, die Sprache der Zukunft zu erlernen, aber die Visionen der Zukunft auszulösen, das konnte er mit seinen aktuellen Kenntnissen noch nicht in genaue Regeln gießen. Er erinnerte sich daran, wie er sich einst mit Reka darüber gestritten hatte, wie schwammig die Lehre des Zukunftlesens doch war. Vielleicht hatte doch ein Ansatz der Wahrheit in den Worten der Kräuterhexe gelegen.

Sorgfältig zückte Leander seinen Dolch, wägte ein letztes Mal ab zwischen den Risiken seines Vorhabens, seinem Vertrauen in dessen Gelingen und seiner emotionalen Verbindung zu Callem. Dann atmete er tief durch, nannte sich selbst einen Narren und fuhr den Dolch durch seinen linken Handrücken. Diese Extremität könnte er wohl am problemlosesten verlieren, zum Laufen brauchte er beide Beine und rechts war seine dominante Seite.

Die Wunde brannte in der kühlen Abendluft. Leander stellte sich vor, wie sein blaues Blut auf den Boden tropfte. Alsbald steckt Leander sich einen Lederriemen in den Mund, biss darauf und ließ sorgfältig einige Tropfen des schwarzen Vypera-Gifts in die offene Wunde tropfen.

Normalerweise war Leander sehr geschickt darin, Schmerzen zu unterdrücken, aber das Vypera-Gift bot ihm diesbezüglich keine schlechte Herausforderung. Hinzu kam noch, dass Leander beim Kontakt mit dem Gift der Vorfall aus seiner und Callems Kindheit wieder vors innere Auge geführt wurde, was alles andere als angenehm war. Ein unterdrücktes Stöhnen entfuhr seiner Kehle.

Schon breitete sich ein Gefühl der Steifheit in Leanders Hand aus und strahlte in seinen Arm.

Dann wurde alles farbig.\bigskip







Es war kalt. Und es war laut. Schnee und Wind peitschten in Leanders Gesicht, zerrten seine Kapuze von seinem Kopf und ließen sein langes schwarzes Haar umherwehen. Haar, das er schon seit Jahrzehnten nicht mehr besaß.

Er bibberte. Er schien sich in Hadria zu befinden, dem magischen Land von Eis und Schnee. Aus der Ferne war ein animalisches Röhren zu hören. Schreie und Rufe. Meereskreaturen, die dem aufgewühlten hadrischen Meer entstiegen. Nachtgors, die sich an den Schafherden und Einwohnern Hadrias weideten. Ein Trauerfest.

Doch nicht nur das Gebrüll der Kreaturen und die Hilferufe der Hadrier vernahm Leander. Nein, weiter draußen im Meer, so weit draußen, dass man es kaum mehr durch die Schneeschauer erblicken konnte, zuckten Blitze aus dem Himmel und erhellten die Silhouetten mehrerer gut gerüsteter Schlachtschiffe. Von dort waren weitere, eigenartigere Geräusche zu vernehmen:

Der Wind heulte mit einer Stimme, die Flüche vom Himmel schleuderte.

Das Meer flüsterte mit einer zischenden Stimme, die den Kreaturen Schwachstellen an den Schiffen verriet.

Und aus den Tiefen des Abyss erklang ein dumpfes Tröten, welches Leander gehofft hatte, nie wieder in seinem Leben zu vernehmen.

Arkteron, Kenvilar und Oktohan.

Die drei Mächte des Meeres waren entfesselt worden.

Leanders Vision zeigte ihm Hadria zu jener Zeit, als Seekönig Varatan die Mächte herausgefordert hatte. Seine Seefeste Klippenwacht war damals zu einer Ruine degradiert worden. Sein Volk hatte nach Werftheim flüchten müssen, auch wenn diese Nebelinsel damals noch nicht so geheißen hatte. Und die Heere der Mächte des Meeres hatten ihren Blick auf den Ort gerichtet, von dem Varatan seine magischen Waffen erhalten hatte. Hadria wäre in jenen Tagen beinahe überrannt worden. Und auch wenn damals das grausame Opfer Orweyns die Mächte hatte besänftigen können, hatten die Mächte Schnee und Eis mit sich gebracht, welches Hadria bis an den heutigen Tag noch bedeckten.

Warum erinnerte diese Vision ihn an jene schreckliche Zeit? Leander war damals dort gewesen, hatte den Angriff der Mächte mit eigenen Augen gesehen. Leander mochte es nicht, an diese Zeit zurückzudenken. Es war ein Gemetzel gewesen, ein sinnloses Gemetzel.

Orweyn, der mächtigste aller Zauberer, der einst den drei magischen Waffen in der hadrischen Unterwelt ihre Kraft verliehen hatte, hatte als erster erkannt, dass Varatans hochmütige Schlacht gegen die Mächte des Meeres nicht mehr gewonnen werden konnte. Er hatte als erster erkannt, dass die Dunkle Magie, die den magischen Waffen ihre Kraft verlieh, stets ihren Preis forderte. Und so hatte Orweyn sich vorgenommen, die magischen Waffen und jeden, der von der Dunklen Magie wusste, aus dieser Welt zu schaffen. Er hatte in Windeseile den Eisernen Turm erbauen lassen. Und kaum hatte dieser magische Turm über der Feste von Yra gethront, hatte Orweyn eine Versammlung der Festenbewohner einberufen, die das Schicksal Hadrias maßgeblich bestimmen sollte.

Leander erinnerte sich an eine andere Versammlung, die zeitgleich stattgefunden hatte. Er war als interessierter Gelehrter nach Hadria gezogen, nachdem Callem sich der finsteren Kenvilar und der Piraterie verschrieben hatte und Leanders Familie in Varatanien in Verruf geraten war. Leander hatte auch nur als interessierter Gelehrter der Versammlung einiger Magier in Nordgard beigewohnt, die sich ob der nicht abschwächen wollenden Angriffe der Kreaturen zu beraten ersuchten. Mit einem Krachen war Orweyn in der Runde erschienen, seinen grün schimmernden Hammer in der Hand und ein wirres Glitzern in den Augen.

„Die Mächte des Meeres müssen mit einem Opfer besänftigt werden!“, hatte er gegen den heulenden Wind aus dem Schlund Arkterons gebrüllt, „Ein jeder, der von der Dunklen Magie Kenntnisse besitzt, möge mir folgen! Die Magischen Waffen müssen weggesperrt und unser Wissen über sie getilgt werden. Wir versammeln uns bei der Feste von Yra und beenden diese Perversion der Natur ein für alle Mal!“

Eine junge Zauberin war aufgestanden und hatte protestiert. Orweyn hatte nicht lange gefackelt und seinen Hammer auf die Adeptin geschleudert. Leander hatte von Glück reden können, dem ausbrechenden tödlichen Getümmel lebendig entkommen zu sein.

Das nächste Mal, als er Orweyn erblickt hatte, hatte dieser sich eine üble Wunde am Bein zugezogen, aber zusätzlich zu seinem Hammer auch Varatans Helm errungen und eine Handvoll ähnlich denkender (oder aus reiner Furcht mit ihm mitziehender) Zauberer um sich geschart. Gemeinsam war diese Horde durch Nordhom gezogen, hatte Häuser gestürmt und jeden Zauberer gemordet, der sich ihnen nicht anschloss. Sie richteten insgesamt mehr Schaden an als die Armeen von Kreaturen, denn letztere hatten sich zu diesem Zeitpunkt größtenteils ins Meer zurückgezogen, wo die finstere Kenvilar verblüfft zugesehen hatte, wie Orweyn ihre Arbeit für sie tat.

Leander hasste Kenvilar dafür, seinen Bruder auf ihre Seite gelockt und dann im Stich gelassen zu haben. Doch noch mehr hasste er Varatan dafür, den Fluch über Narkon gesprochen zu haben. Wie er Orweyn mit Varatans Helm auf dem Kopf erblickt hatte, hatte er gehofft, dass Varatan inzwischen gefallen sei. Doch dem war nicht so gewesen. Varatan hatte den Untergang seines Flaggschiffs, der Tambur, irgendwie überlebt und sich durch das eisige Wasser des hadrischen Meers an die hadrische Küste geschleppt. Und Kenvilar, die Tückische, hatte von ihm abgelassen.

Damals hatte Leander sich gefragt, warum Kenvilar Varatan nicht nachgetragen hatte. Eine ihrer Töchter war genau wie Callem Teil der Besatzung der Schwarzen Kogge und nun schon seit Generationen auf Narkon gefangen. Hätte Kenvilar Varatan damals den Rest gegeben, so wäre die Besatzung und ihre Tochter vermutlich schon längst wieder auf freiem Fuß.

Heute hingegen fragte sich Leander, ob Kenvilar damals geahnt hatte, welche ungeahnten Ausmaße von Tod und Verderben Varatans Brut in diese Welt setzen würde. Vielleicht war die Menge an Pein, die der Dunkle Magier Varkur in diesen Landen verbreiten würde, in ihren Augen wertvoller als die Freiheit ihrer Tochter? Wer konnte schon wissen, ob ihre Tochter ihr überhaupt etwas bedeutete? Wer konnte schon wissen, ob sie überhaupt wollte, dass die Magischen Waffen unter Verschluss blieben? Vielleicht hätte sie es geliebt, wenn ein mächtiger Zauberer mithilfe der Waffen zu einer vierten Macht des Meeres aufgestiegen wäre.

Orweyn war sich jedenfalls sicher gewesen, dass die einzige Lösung für die drohende Gefahr darin bestand, die Magischen Waffen und alle überlebenden Dunklen Magier im einzig dafür erbauten Eisernen Turm einzuschließen. Nur einer hatte sich vor seinem Mob verstecken können. Zoren hieß er, ein einsamer Einzelgänger, welcher sich bereits seit einem gewaltigen Streit mit dem großen Orweyn von der Zauberergemeinschaft zurückgezogen und in seiner Spitzburg verschanzt hatte. Paranoid, wie er war, hatte er sich beim ersten Anzeichen von Trubel an einem anderen geheimen Ort versteckt, wo selbst die Zauber seiner einstigen Kollegen ihn nicht hatten aufspüren können. Leander hatte mit angesehen, wie Zoren in der Ruhe nach dem Sturm fassungslos durch die Gegend gestolpert war. Anschließend hatte er sich in seine Spitzburg zurückgezogen und war der Legende nach seither nie mehr hervorgetreten. Was er wohl den lieben langen Tag so trieb?

Leander hatte so lange in seinen Erinnerungen geschwelgt, dass ihm erst jetzt wieder bewusst wurde, dass er sich immer noch in einer Vision befand. Einer Vision, die durch Vypera-Gift induziert worden war. Wie lange war er bereits hier? Wie erging es seinem Arm? Er musste weiterziehen.

Die Vision klärte sich ein wenig und Leander Blick fiel auf den Innenhof der Feste von Yra. Niedere Zauberer rannten schreiend umher und trauerten über den Leichnamen ihrer gefallenen Kameraden. Dort drüben umarmte der junge Koraph schluchzend den Körper eines jungen Mannes und versuchte verzweifelt, eine Wunde an seiner Seite zu schließen, die den Zauberer schon längst aus dem Reich der Lebenden befördert hatte. Weit oben in den Himmel ragte der Eiserne Turm, und ein türkises Glühen umgab ihn.

Da oben stand Orweyn, vor dem Eingang zum Turm, seinen Hammer der Stärke locker an seiner Seite haltend, Varatans Helm der Macht auf dem Kopf und Varlion das Flammenschwert hoch in den Himmel gereckt. Er schubste gerade Hombudt, den Zauberer des Raums, ins Innere des Eisernen Turms.

Neben Orweyn rollte ein Braumeister der Feste von Yra ein riesiges Eichenfass ins Innere, aus dessen Seite einige Eiszapfen stachen. Über was für einen sagenumwobenen Inhalt mochte dieses Fass wohl verfügen?

Leander wusste genau, was als nächstes geschehen würde: Orweyn würde gen Himmel blicken und seine letzte Prophezeiung verkünden, mit den Magischen Waffen in den Eisernen Turm steigen und den Turm von innen verschließen, sich und alle überlebenden Dunklen Magier darin festsetzen und damit dem sicheren Hungertod überlassen. Der Turm würde verschlossen bleiben, bis sein Eingang Jahrzehnte später durch Varkur gesprengt werden würde – und auch dann würden die Magischen Waffen dort drinnen verbleiben.

Wie erwartet, reckte Orweyn seinen Hammer und das Flammenschwert gen Himmel und begann, zu seine Prophezeiung zu intonieren:

„Wenn Feuer und Turm miteinander ringen, das Ende aller naht, denn Qurun wird sie bezwingen.“

Qurun war ein hadrisches Wort für das Ende der Welt, die Personifizierung des Bösen, soviel wusste Leander. Doch nun änderte sich die Vision abrupt. Leander verlor kurzzeitig die Orientierung. Als er wieder aufblickte, sah er immer noch den Innenhof der Feste von Yra, doch aus einen anderen Perspektive. Und er nicht mehr Leander. Er war Orweyn. Er blickte durch die Augenlöcher in Varatans Helm der Macht auf eine gelb gefärbte Welt und hielt die beiden anderen magischen Waffen in seiner linken und rechten Hand. Leander spürte Orweyns Entschlossenheit und irre Gewissheit ebenso wie das Brennen der Wunde an seinem rechten Bein, das ihm jemand zugefügt hatte. Ob es Varatan oder ein elender Zauberer gewesen war, wusste er nicht mehr. Es war auch nicht relevant.

Durch Orweyns innere Augen erhaschte Leander einen Blick auf die Vision, die Orweyn damals empfangen hatte, und plötzlich war Qurun nicht mehr nur ein Ausdruck für das Ende der Welt, nein, er hatte ein Bild dazu vor sich: Eine riesige, schattenhafte Gestalt mit spitzen Klauen, langen Zähnen und Augen wie glühender Glut. Gewundene Hörner erhoben sich aus dem schwer im Schatten auszumachenden Schädel, und riesige Flügel trugen das Biest durch die Lüfte.

Unter dem grässlichen Wesen Qurun sah Leander Zauberer des Turms und des Feuers sich gegenseitig bekämpfen, auch wenn Orweyn ihnen diese Begriffe noch nicht zuordnen konnte. Sie schleuderten magische Blitze und Drachenfeuer aufeinander, während über ihren Köpfen Qurun lachte und sich auf dem Eisernen Turm niederließ.

Von Schrecken und Grauen erfüllt, zog sich Orweyn aus dieser düsteren Vision zurück, trennte sich wieder von Leander und verschwand im Innern des Eisernen Turms. Er sprach einige Worte in der Alten Sprache und das riesige Tor fiel krächzend ins Schloss, oder besser gesagt versiegelte sich magisch, denn ein Schloss gab es an diesem Tor gar nicht. Es war nie gedacht gewesen, dass dieser Eingang je wieder geöffnet wurde.

Die Schreie und das Stöhnen der Zauberer im Turm verstummte augenblicklich, das der Zauberer außerhalb blieb hingegen bestehen. Eine türkise Aura umwogte den Eisernen Turm nun, zog Schnee und Eis mit sich und hob diese Schwaden weit in den Himmel. Und wie diese himmelblaue Wogen in den Himmel getragen wurden, schwächte der Sturm, der über Hadria wogte, leicht ab. Die Mächte des Meeres waren besänftigt worden.

Doch Leander achtete nicht darauf. Nein, Leander sah vor seinem inneren Auge weiterhin die Vision, deren Anfang Orweyn gesehen hatte. Er sah, was Orweyn gesehen hätte, wenn er nur ein klein wenig länger mit dem Versiegeln des Turms gewartet hätte. Leander erblickte einige kleine Gestalten, die sich diesem grässlichen Geschöpf namens Qurun näherten und es zurückdrängten. Dort, dieser kleine grüne Bogenschütze, der Varatans Helm der Macht auf dem Kopf trug und gleich drei Pfeile auf einmal im schattigen Leib versenkte, das musste Chada aus dem Wachsamen Wald sein. Dieser Zauberer des Turms, der Varlion das Flammenschwert auf Qurun richtete und einen mächtigen Feuerstrahl in dessen Brust beschwort, das war Eara aus dem Hohen Norden. Oder der Krieger mit dem kalten Blick, der Orweyns Hammer der Stärke auf das Schattenwesen schleuderte, das war höchstwahrscheinlich Thorn aus dem Rietland. Die Helden von Andor, hier in Hadria, mit den Magischen Waffen gegen Qurun kämpfend? Kurz war Leander überrascht, dann lachte er auf. Nun, wer wäre schon so todesmutig, wenn nicht sie? Und wer hatte bessere Chancen, sich Qurun zu stellen?

Dann flackerte die Vision erneut etwas und zeigte Leander das Ende des Kampfes. Qurun sank besiegt zu Boden, die Schatten versiegten und anstelle des riesigen Ungetüms lag nur noch eine kleine Gestalt in einem grauen zerrissenen Gewand im kalten Schnee.

Leander erkannte im grässlichen Geschöpf namens Qurun den ältesten Widersacher der Helden von Andor.

Varkur, der Dunkle Magier.

Er war Qurun gewesen.

Und die Helden hatten ihn besiegt.

Varkur richtete sich mit letzter Anstrengung aus, spuckte Blut und rief etwas in den heulenden Wind. Dann sank Varkur, Sohn des Varkmar, in dunklen Nebel und schied aus diesem Leben.

Leanders sehendes Auge verschloss sich.\bigskip







Nur langsam kam Leander wieder zu Besinnung. Als erstes merkte er, dass er auf seinem Hüttenboden lag und ihm der bittere Geruch nach Sternkraut in sein Riechorgan stach. Dann erinnerte er sich an seinen Plan, eine Vision über Callem zu erhalten. Der Plan hatte funktioniert. Das schwarze Gift hatte es tatsächlich geschafft, eine Vision zu triggern! Eine ungewöhnlich lebhafte und ungewöhnlich lange war es gewesen, und erneut hatte er in die Vergangenheit statt in die Zukunft geblickt.

Erschrocken wurde ihm bewusst, dass er zwar Maros Gift, nicht aber das Gegenmittel eingenommen hatte. Wie viel Zeit war vergangen? War seine Hand noch zu retten? Rasch versuchte er, sich aufzurichten, da traf ihn eine Hand unsanft auf der Brust und hinderte ihn daran.

„Ts, Ts, Ts“, erklang ein kritisches Schnalzen, „Wenn ich dir deinen Wunsch nach Schlangengift verweigere, hat das einen Grund. Das heißt nicht, dass du stattdessen einfach mein Mädel danach fragen sollst.“

Reka. Natürlich hatte Reka ihn gefunden.

Leander sank seufzend auf den Boden zurück. Erleichtert, dass sein Leben nicht mehr in Gefahr war. Aber auch furchtsam vor den Worten der Hexe. Und diese war noch nicht fertig mit ihm. Sie sprach weiter, und zum ersten Mal konnte Leander nicht einmal den Ansatz von Humor in ihrer Stimme erkennen:

„Du kannst von Glück sprechen, dass Chada mir rechtzeitig davon berichtet hat. Wenn ich einige Minuten später hier aufgekreuzt wäre, hätte ich dich nicht wieder zurückbringen können.“

„Ich habe doch gar nichts gesagt“, erklang die protestierende Stimme Chadas aus dem Hintergrund. Sie erinnerte Leander an die Vision, die er soeben erblickt hatte. Chada und die restlichen Helden würden Qurun besiegen. Und Qurun würde Varkur sein. Interessante Informationen, zweifelsohne, insbesondere da sie ihm erlaubten, sich keinen ausgeklügelten Mordplan für Varkur ausdenken zu müssen. Egal, was er tun würde, Varkur würde nicht ewig am Leben bleiben. Dennoch hatte ihm die Vision nicht das gezeigt, was er sich gewünscht hatte. Nichts zu Callem und wie man ihn von Varatans Fluch befreien könnte. Ganz hatte er seine seherischen Kräfte also nicht austricksen können.

„Hast du etwas zu deiner Verteidigung zu sagen?“, sprach Reka. Leander versuchte, nicht daran zu denken, dass er entgegen ihres Wunsches Meres nach Narkon geschickt hatte, und entgegnete stattdessen:

„Wenn ich dich so sehr störe, warum hast du mich überhaupt gerettet?“

Reka blieb einen Moment stumm. Dann meinte sie: „Nenn mich verrückt, Leander, aber ich habe das Gefühl, dass du deine große Rolle in den Legenden von Andor noch nicht gespielt hast. Ich weiß, dass du viel Übles angerichtet hast in deiner langen Lebzeit. Aber ich glaube, dass du auch noch viel Gutes anrichten wirst.“

Leander gluckste schwach auf. Dafür konnte er die Hexe wirklich verrückt nennen.

Diesmal hatte sie auch keinen Eintopf für ihn übrig, als sie ihn verließ. Nein, Leander musste ihr danken dafür, dass sie sein Leben gerettet hatte. Und das tat er auch. Anschließend dirigierte er sie und Chada aber auch schon wieder unwirsch aus seiner kleinen Hütte hinaus und sortierte seine Gedanken im Angesicht seiner jüngsten Vision.

Er wusste nun, wie Varkur sterben würde. In Hadria. Durch die Hand der Helden, wie könnte es anders sein? Nur wie bald? Die Helden wirkten nicht so, als würden sie Andor so bald verlassen wollen. Varkur könnte sie natürlich in den Norden locken. Aber warum würde er das tun?

Leander studierte und knobelte an diesem Rätsel herum, bis ihm endlich ein Ansatz einfiel. Die Krahder aus dem Süden. Das Riesenvolk hatte vermutlich keine Ahnung davon, dass Tarok gefallen war, dass der Weg nach Andor für sie offen stand, ja, hatte vermutlich nicht einmal eine Ahnung davon, dass da im Norden ein Königreich lag, welches von aus Krahd geflohenen Ambacus gegründet wurde, und dass Mitglieder dieses Königsreichs einen der ihren erschlagen hatten. Sobald sie davon erfuhren, würden sie wohl mit Vergnügen Rache nehmen wollen.

Die meisten Krahder waren nun mal Feiglinge. Folglich müsste jemand die furchterregenden Helden von Andor aus dem Weg schaffen, damit sie in Andor einfielen. Und jemand müsste die Krahder über den freien Weg in den Norden informieren. Diese Rollen könnte Varkur übernehmen. Und dies gäbe Varkur einen guten Grund, in den Norden zu reisen und die Helden hinter sich herzulocken.

Kurz verzog Leander seine Mine, als er an die vielen Bauern der Andori denken musste, von denen bestimmt einige bei einem solchen Überfall der Krahder umkommen würden. Er unterdrückte die Gedanken. Die Bauern bedeuteten ihm nichts, und Krahder in Andor wären ganz nebenbei keine schlechte Gelegenheit für Leander, dunkles Wissen zu ergattern. Aber der Angriff der Krahder wäre nicht einmal das wichtigste, sondern nur ein Grund für Varkur, in den Norden zu ziehen. Wichtiger war, dass sie Helden von Andor Varkur in den Norden folgten und dem Dunklen Magier dort ein Ende bereiteten. Dass sie es möglich machten, dass Varatans Fluch gebrochen werden konnte.

Und wenn die Helden erst einmal im Norden fertig waren, würden sie zurückkehren wollen, vermutlich in Eile, wenn es in Andor dann nicht allzu rosig zu und her ginge. Und wenn ihr Schiff auf Narkon kenterte... Könnte Leander einen Pakt mit einer Macht des Meeres eingehen? Könnte er Arkteron, Kenvilar oder Oktohan dazu bewegen, das Schiff der Helden in die Stachelklippen Narkons zu dirigieren? Wer könnte schon mit Sicherheit Varatans Fluch die letzten Kraftreserven aufzehren lassen, wenn nicht die Helden von Andor?

Aus den wirren Gedanken in Leanders Kopf kristallisierte sich immer deutlicher ein grandioser Plan heraus. Erschöpft sank er in seinen Schaukelstuhl und atmete schwer durch. Es würde riskant werden. Aber um Callem zu befreien, hatte er schon viel Riskanteres durchgezogen.

Leander massierte seine Schläfen. Shan war vermutlich gerade auf der Suche nach Varkur, um dabei zu sein, wenn dieser Hademars Seele aus dem Drachenherzen befreite. Und Shan konnte die magischen Spuren ihres Schattens erheblich schlechter verbergen als Varkur die seinen. Wenn er Shans Spur folgte, würde er höchstwahrscheinlich auf Varkur stoßen können.

Danach müsste Leander seinen Plan nur noch Varkur unterbreiten. Nun, zumindest den Teil, der nicht Varkurs Tod beinhaltete. Am besten so, dass der Dunkle Magier das Gefühl hatte, von selbst drauf gekommen zu sein.

Entschlossen griff Leander nach seinem Stab.\bigskip







Varkur, der Dunkle Magier, halb Mensch, halb dunkler Nebel, erreichte das Graue Gebirge. Langsam schälte sich sein Körper aus dem Nebel. Mit geschlossenen Augen stand er da, der eisige Wind schlug ihm ins Gesicht und der Schnee hüllte ihn mehr und mehr ein. Er war in Hochstimmung.

Einige Wochen hatte er nun schon vergeblich damit verbracht, Hademars Geist aus diesem verfluchten kleinen Steinchen zu befreien versuchen. Sein Kopf schmerzte von all den niederen Kreaturen, in die er Hademars Geist schon projiziert hatte, nur um kurze, unverständliche Antworten zu erhalten.

Hademars Seele, sein Geist, oder wie auch immer man das nennen wollte, hatte sich stets kontaktfreudig gezeigt. Selbst wenn er nicht gerade einen Gorlot steuerte, konnte der Nekromant den schwarzen Schemen unter der roten Edelsteinoberfläche des Drachenherzens mit einigem Geschick verformen. Ohnehin verkraftete Hademar die Zeit in Gefangenschaft überraschend gut. Aber vermutlich war er sich schon daran gewöhnt, körperlos zu sein. All die Jahre, die er seit dem Verlust seines Körpers damit verbracht haben musste, die Überreste seines Geistes langsam zusammenzusetzen und erstarken zu lassen... Varkur wollte sich nicht vorstellen, wie sich das hatte anfühlen müssen.

Varkur vermisste es, Hademar von Gesicht zu Gesicht gegenüberzustehen. Die beiden hatten gemeinsam gute Zeiten erlebt in Hadria. Ehe sie auseinander getrieben worden waren vom Schicksal. Oder besser gesagt von ihren persönlichen Rachefeldzügen und bösartigen Plänen. Dort waren sie nie auf einer Wellenlänge gewesen. Varkur hatte rasch handeln wollen, Kreaturen zu offenen Angriffen nutzen, den Drachen Tarok zähmen, die Völker fremder Länder gegeneinander locken, Könige umbringen und am Ende mit einer riesigen Armee unter seiner Kontrolle gen Hadria ziehen.

Hademar hingegen hatte sich so stark wie nur irgendwie möglich vom für ihn verhassten Andor fernhalten und im Geheimen seine Macht mehren wollen. Alte Schriften studieren, leere Festen erkunden, mächtige Formen der Magie meistern. Erst wenn er sich (fälschlicherweise) sicher gewesen war, dass niemand gegen ihn ankommen könnte, erst dann hatte er sich offenbart.

Am Ende war aus dem Dunklen Duo eine schwache Bekanntschaft geworden. Hin und wieder hatten sie sich Briefe gesendet und den jeweils anderen darüber aufgeklärt, wie viel besser ihre Angehensweise sich für sie bezahlt machte.

Am Ende hatte es für sie beide nicht funktioniert. Die Helden von Andor waren ihnen beiden in die Quere gekommen, immer und immer wieder. Varkurs Gesicht verzerrte sich beim Gedanken an sie zu einer grimmigen Fratze.

Einst waren Hademar und er zwei begeisterte Studenten der geheimen Magieformen in Hadria gewesen. Nun war Varkur ein halber Echsenmensch, umgeben von einem ihn verzehrenden brennenden Nebel, und Hademar war nichts weiter als ein schwarzer Schatten unter der Oberfläche eines kristallenen Runensteins, der hier und dort durch eine besessene Kreatur einige Worte sprechen konnte.

Aber nicht mehr lange. Hademar war es gelungen, Varkur in den letzten Wochen einige konkrete Hinweise zu übermitteln.

„Licht.“ „Mond.“ „Altar.“

Solche Wörter hatten die Gorlots, Wargors und Skrale gekrächzt, ehe die Verbindung gebrochen worden und sie wieder zusammengeklappt waren.

Varkur hatte einen genügend geübten Blick für die fließenden Bänder der Magie, um zu spüren, dass mächtige Drachenmagie Hademar in diesem Stein einschloss. Nach und nach hatte er es mit Hademars Hinweisen geschafft, den Prozess herauszufinden, der dies bewerkstelligt hatte. Das Licht des roten Mondes hatte den Stein in einem Ritual blutrot gefärbt. Ein Alter aus purem Roteisen hatte geholfen, dieses Licht zu kanalisieren. Das bedeutete, dass Zwerge involviert waren. Diese unterirdischen Ratten! Varkur hätte sie schon längst aus ihren engen Gängen ausräuchern sollen.

Heute hatte Hademar Varkur endlich in die unterirdische Kammer gelenkt, wo sein Geist eingesperrt worden war. Tief unter das Graue Gebirge war Varkur geflogen, hinein in die Korn-Schlucht, ins unterirdische Reich der Schildzwerge aus der Zeit, wo sie noch nicht einmal Schildzwerge genannt worden waren, nach links, dann zweimal rechts, nach oben und dann noch weiter nach unten, stets dem schwarzen Pfeil folgend, den Hademars schwarze Seele im Edelstein für ihn formte.

Nun stand er in einem kreisrunden Raum. Dieser war leer bis auf den Blutstein-Altar in seiner Mitte und eine riesige Gestalt am gegenüberliegenden Ende des Raumes. Dort lag ein versteinerter Drache, die Überreste des einst so edlen Kardòl, den rissigen Kopf in seinen stacheligen Schwanz verbissen. Aber Varkur war nicht seinetwegen gekommen.

Die Spuren des magischen Rituals waren nicht mehr frisch, aber dennoch deutlich zu erkennen. Hier war Hademar im Drachenherz eingeschlossen werden. Schlampige Arbeit, aber dennoch schwer zu durchbrechen. Varkur schüttelte seinen Kopf und platzierte das Drachenherz vorsichtig auf dem großen Alter aus purem Roteisen. Blutsteinaltar hatten ihn die Zwerge genannt, nach dem Blut der gefallenen Drachen, welches dem Stein seine Eigenschaften verliehen hatte.

Der rote Mond schien nicht, durch den gezackten Spalt in der Decke konnte Varkur nur den klaren Sternenhimmel erkennen. Das Sternbild des Hornfalken, wenn er sich nicht irrte. Ein gutes Omen. Er brauchte das Licht des roten Mondes nicht, um Hademar zu befreien.

Varkur schlug seine Ärmel zurück und enthüllte echsenhafte Klauen, die er knackend spreizte. Zeit, seinen alten Freund zurückzuholen.

Er ritzte sich seinen Handrücken auf und spritzte sein schwarzes Blut über den Altar. Sein Mund öffnete sich und enthüllte spitze Zähne, als er grinste und ansetzte, die entsprechenden magischen Worte der Alten Sprache auszusprechen, die Hademar befreien würden.

Urplötzlich wurde es kalt um Varkur. Der Dunkle Magier spürte, wie \textit{Etwas} in seinen Geist eindrang und seine Gedanken in Eiseskälte tauchte. Alarmiert stolperte er nach vorne und stieß gegen den Blutsteinaltar.

Sein Körper fing sich von selbst. Von Horror erfüllt, spürte Varkur seine lange Zunge in seiner Mundhöhle umherwandern und seine scharfen Zähne entlanggleiten. Seine eigenen Lippen öffneten sich und er hörte seinen eigenen Mund krächzend sprechen: „Tut mir leid, Hadrier. Ich kann das nicht zulassen.“

Varkur versuchte zurückzuweichen, zum Schlag auszuholen, seinen Stab zu greifen, den er achtlos beiseite geworfen hatte.

Nichts konnte er tun. Alles erfolglos, sein Körper gehorchte ihm nicht mehr.

Einzig der Nebel der Dunklen Magie gehorchte ihm noch. In Varkurs Auftrag wallte dieser durch den Raum, spaltete Stein und brach Felsen, suchte nach einem Gegner, dem er sich stellen konnte. Doch es gab keinen solchen Gegner. Denn der Gegner befand sich in Varkurs Kopf.

„Du verstehst das bestimmt“, sprach sein Mund nun, und wie ein Fremder in seinem eigenen Körper hörte Varkur die pure Finsternis aus sich sprechen, „Hademar ist ein mächtiger Nekromant. Wird er nun befreit, könnte er zur mächtigsten Person dieser Welt werden. Das darf ich nicht zulassen.“

War das die Dunkle Magie, die da aus ihm sprach? Hatte sie endgültig seine Menschlichkeit in sich aufgesogen, ihn endlich endgültig in ihre Gewalt gekriegt? Varkur ließ sich ergeben sinken. Nun, nicht wirklich sinken, er hatte ja keine Kontrolle über seinen Körper mehr. Aber er hörte auf, gegen die fremde Macht anzukämpfen. Die Dunkle Magie brannte schon seit Jahrzehnten in ihm, und je länger er von ihr Gebrauch gemacht hatte, desto öfter hatte sich dieses zweischneidige Schwert gegen ihn gewandt und ihm mehr seiner Menschlichkeit genommen. Er war dazu verdammt, früher oder später als ausgebrannte Hülle der Dunklen Magie zu dienen. Dieses Schicksal hätte ihn ohnehin irgendwann erreicht. Nur schade, dass es so viel früher als später der Fall gewesen war.

Ergeben nahm Varkur wahr, wie seine Hände sich öffneten und seinem Zauberstab geboten, zu ihm zu springen. Sein Stab tat wie geboten, und die Seelen der unzähligen darin eingefangenen gequälten Geister stöhnten auf, als die finstere Macht in ihre Energie griff und ihre Kraft entfesselte. Dann hoben sich Varkurs Hände, den Zauberstab fest umschlossen. Seine Kehle brach in einem dämonischen Lachanfall aus, als der Stab auf das Drachenherz niederfuhr, fest entschlossen, Hademar ein für alle Mal auszulöschen.

Stattdessen blieb der Stab kaum eine Handbreit über dem Drachenherz in der Luft stehen, als wäre er auf eine unsichtbare Wand getroffen. Varkurs echsenartige Augen rollten ohne seine Führung in ihren Höhlen umher und versuchten, im Dunkel die Ursache dieser Störung zu erkennen. Als wäre dies überhaupt möglich gewesen, so wurde der Raum noch kälter, und vor allem eines, dunkler. Kein Stern war mehr durch die Ritze in der Decke zu erkennen.

Immer finsterere Schatten flossen in den Raum und Varkur erkannte, dass das, was seinen Stab davon abgehalten hatte, das Drachenherz zu zersplittern, ein finsterer Tentakel aus purem Schatten war, der aus einem schattigen Gebilde zu strömen schien, welches sich soeben in den Raum zwängte.

Zwei große gelbe Augen leuchteten in dem Gebilde auf und die Schatten formten sich zu einem dunklen Schemen, der entfernt an die Gestalt einer Frau erinnerte. Varkur wäre zurückgezuckt, hätte er können, so sehr erinnerte ihn diese Erscheinung an seine N...

„Es ist sie nicht“, sprach er in Gedanken zu sich selbst, „Es kann nicht sie sein.“

Da sandte der Dunkle Magier seinen Nebel aus, der schwarz und bissig wie ein Tier auf die Schattenfrau zuglitt und sie umwallte. Oder vielmehr versuchte, sie zu umwallen. Die Schattententakel der Schattenhexe wogten und verformten sich zu einem breiten Schild aus purer Dunkelheit, welcher sich zwischen die Frau und seinen Nebel stellte und diesen zurückdrängte. Wie war das möglich?

Varkurs Körper schnellte nach vorne, als er von der finsteren Macht, die ihn kontrollierte, zum direkten Angriff dirigiert wurde. Varkurs Krallenhände stürzten sich auf den Schattenschild und durchbrachen dessen Oberfläche, sorgten für einen Spalt, durch den sein Nebel zur Hexe dringen konnte. Sie schrie nicht auf, aber ihre ganze Form zitterte und verschwamm. Gemeinsam stürzten sich Varkur und die Macht, die seinen Körper führte, auf die Schattenhexe.

Nun, da er sie genauer erkennen konnte, atmete Varkur erleichtert auf. Das war nicht N... Ni... es war sie nicht.

Die Schatten der Schattenhexe zerfaserten und flossen um Varkur herum, dessen Körper herumschleuderte und die Schattenhexe zu fassen versuchte.

„Narr“, drang eine flüsternde und doch durchdringende Stimme aus dem Schatten, „Ich bin auf deiner Seite, Dunkler Magier. Ich will genau wie du den Nekromanten aus dem Stein befreien.“

Schwarze Schemen flossen an Varkurs Körper hoch und verfestigten sich um seine Hand- und Fußgelenke. Mit übermenschlicher Kraft zerrte die Schattenhexe an ihm und bugsierte ihn wieder in die Nähe des Drachenherzens, Hademars Edelstein, der noch immer auf dem Blutsteinaltar lag.

Varkur ließ seinen Dunklen Nebel abziehen. Wenn diese Schattenhexe wirklich Hademar befreien wollte, waren sie wahrlich auf derselben Seite. Er spürte, wie seine Muskeln von der fremden Macht in seinem Körper versteift wurden und sich erfolglos gegen die Kontrolle der Schattenhexe zu wehren versuchte.

Schon hatten Varkurs Klauenhände den Edelstein wieder ergriffen und in die Höhe gehoben. Unruhig nahm Varkur war, wie die Schatten der Schattenhexe schwächer wurden, je näher sie dem Edelstein kamen. Das Licht der Gemme schwächte sie. Nichtsdestotrotz vermochte sie immer noch, seinen Körper zu führen. Ein unterdrückter Aufschrei entfuhr ihr, dann hielt sie den Stein mit Varkurs Händen in die Höhe und begann, einen Satz in der Alten Sprache zu sprechen.

Varkurs Kehle entfuhr ein Aufschrei und Varkurs Magen stellte sich auf den Kopf. Sein Körper sackte zusammen und ein keuchender Husten schüttelte ihn, als eine plumpe, blau-graue Masse aus seinem Körper floss und sich vor ihm zusammensetzte. Dann war es also doch nicht die Dunkle Magie gewesen, die ihn kontrolliert hatte. Nur ein fremder Geist.

Auf dem kalten Steinboden liegend, erkannte Varkur, wie der Geist kurz eine konkrete Gestalt annahm: Eine schwebende Masse von grauen Tüchern, auf der ein kleiner gemeiner Krahder-Kopf saß. Dann richtete der Geist eine siebenfingrige Hand auf die Schattenhexe und brüllte: „Wer wagt es, sich mir zu widersetzen?! Ich bin Nomion, der Meister des Urtrolls! Eine kleine Schattenhexe kann mir nicht in die Quere kommen!“

Nomions Form zerfaserte und stürzte auf die Schattenhexe zu. Er drang mühelos durch ihren Schattenschild und verschwand in der Dunkelheit, die die Schattenfrau umgab. Selbige kreischte auf. Ihre gesamte Gestalt waberte und zitterte, als sie sich gegen die Kontrolle des Krahder-Hexers zu wehren versuchte.

Gut, dachte Varkur, soll er nur versuchen, sie zu kontrollieren. Es kostete Nomion bestimmt einiges an Energie, einen neuen Körper zu übernehmen. Irgendwann müssten seine Reserven aufgebraucht sein. Und die Schattenhexe schien ihm einen angemessenen Kampf zu bieten.

Die ganze Welt drehte sich unschön, als Varkur seinen Körper in die Höhe stemmte. Erleichtert streckte und dehnte er sich. Sein Körper gehörte wieder ihm. Zeit, das zu tun, warum er überhaupt hierhergekommen war.

Varkur ergriff das Drachenherz, hob es in die Höhe und setzte erneut zum Satz in der Alten Sprache an, den Hademar aus seinem Gefängnis befreien würde.

Da wurde ihm von einem Schattententakel das Drachenherz aus der Hand geschleudert. Frustriert wollte er sich umdrehen, da umschlangen bereits weitere Tentakel seine Arme und Beine und zogen ihn nach vorne. Das Gesicht der Schattenhexe erschien vor ihm, doch nun waren ihre Züge noch unmenschlicher und ihre Ohren spitz wie die eines Krahders, als eine viel tiefere, doch immer noch geisterhafte Stimme aus ihrer Kehle erklang:

„Wie schon gesagt: Ich kann das nicht zulassen, Hadrier. Hademar darf nicht befreit werden.“

Offenbar war es Nomion gelungen, den Geist der Schattenhexe zu vergiften, ohne dass sie im Gegenzug den seinen hätte auslöschen können. Nomion schlang einen weiteren Schattententakel um Varkurs Kehle und drückte zu. Varkur hatte selbst gesehen, wie rasch die Absenz von Luft einem Köper das Leben ausquetschen konnte, und haderte nicht lange mit einer Antwort.

Verzweifelt saugte er die Dunkle Magie in sich ein, viel mehr, als er für eine lange Zeit aus ihr geschöpft hatte. Er spürte, wie in dieser lebensbedrohliche Situation ein letzter Damm in ihm brach, wie die Magie durch sein Adern jagte. Und wie sie ihm erlaubte, die Schatten zu packen, die ihn umgaben.

Mit letzter Kraft griff Varkur nach den Schatten der Schattenhexe, aber nicht mit seinen Klauen, sondern mit der Dunklen Magie. Er zog sie zu sich, leitete sie durch den Blutsteinaltar, verband sie mit der allgegenwärtigen Drachenmagie und versuchte verzweifelt, so viel wie möglich in sich aufzunehmen, um Nomions Angriff widerstehen zu können. Sein schwarzer Nebel drang in die Schatten ein, doch anders als vorher wurde er von diesen nicht abgewehrt. Stattdessen verbanden sich sein Nebel und die Schatten der Schattenhexe. Varkur zog und zerrte an der neuen Verbindung. Für einen kurzen Augenblick löste sich der Druck auf seine Kehle, flossen die Schatten von dem Körper der Schattenhexe weg und auf ihn zu. Varkur glaubte, inmitten der Dunkelheit die Gestalt einer jungen Frau mit einem traurigen Lächeln auf den Lippen zu sehen. Dann ballten sich die Schatten wieder um sie und der Körper der Schattenhexe richtete sich wieder auf, noch größer als zuvor. Der Druck auf Varkurs Kehle verstärkte sich wieder.

Erneut zerrte Varkur an dieser seltsamen Verbindung seiner Dunklen Magie und des Schattens der Schattenhexe, noch stärker als vorher. Diesmal fühlte er, wie etwas vor ihm auseinanderriss, das Ziehen einer anderen Macht am Schatten versiegte und plötzlich unmöglich zu bändigende Mengen an Dunkler und Schattenmagie durch seinen Körper strömten, ihn transformierten.

Varkurs Muskeln schwollen an und seine Haut verschuppte vollends. Ein Prozess, den die Dunkle Magie schon seit Jahrzehnten an ihm vornahm, wurde in Sekundenschnelle ins Extrem getrieben. Vor Varkurs glühenden Augen wuchsen seine Klauenhände auf die zweifache, dreifache, vierfache Größe an. Seine Halsmuskeln spannten sich und schoben den kläglichen Rest von Nomions Schattententakeln mühelos zur Seite. Seine Kopfhaut spannte sich, als seine Hörner in die Höhe schossen, seine Kapuze zerrissen und die Schatten des Raumes durchbrachen.

Varkur fühlte, dass er plötzlich so viel mehr war als sein stetig wachsender Körper und der allgegenwärtige Nebel der Dunklen Magie. Er fühlte den verfluchten Schatten im Raum, wie er über den kalten Steinboden strich und die Dunkelheit liebkoste. Fast war es Varkur, als flüsterte dieser Schatten zu ihm: „Ich bin dein. Nutze mich!“

Probehalber befahl er dem Schatten im Geiste, sich um ihn zu versammeln. Prompt floss der Großteil der magischen Dunkelheit, die diesen Raum unter dem Grauen Gebirge erfüllt hatten, zusammen und umgab den sich transformierenden Körper Varkurs, verband sich mit seiner Schuppenhaut, stärkte seine Klauen, maskierte sein Gesicht.

Der Körper der Schattenhexe waberte weiterhin vor ihm, aber im Vergleich zu den Schatten, die Varkur nun umgaben, sahen die wenigen Fetzen von Dunkelheit, die noch um die Schattenhexe wogten, beinahe lächerlich aus. Varkur hatte der Schattenhexe soeben den größten Teil ihrer Macht entrissen. Die Schattenhexe, oder vielmehr ihr Körper, heulte auf und sank zusammen. Nomions Geist löste sich von ihr und nahm ein weiteres Mal Gestalt an, nur um kreischend das Weite zu suchen.

Die Schattenhexe richtete sich auf und blickte Varkur aus leeren Augen an.

„Wie... was?“, stammelte sie schwach.

Varkurs Körper saugte weiterhin ungehindert Dunkle Magie in sich auf, gestärkt durch die Kraft der Schattenseelen. Seine Klauenhände wurden zu schwer, um aufrecht stehen zu bleiben, und so stürzte Varkur auf sie. Seine unförmigen langen Hörner stießen bereits gegen die Decke des Zwergenraums. Die Schatten, die einst der Schattenhexe gehört hatten, wurden zu riesigen Schwingen, die sich hinter ihm erhoben und den ganzen Raum ausfüllten, ja, gar aus dem Raum hinausquollen. Und als er seine Stimme erhob, um die Schmerzen in seinem Innern auszudrücken, erklang an ihrer Stelle ein tiefes animalisches Röhren.

Varkur fiel kaum auf, dass nun auch die Schattenhexe hastig das Weite suchte. Er drehte und wendete sich, versuchte, im staubigen Steinraum Hademars Edelstein zu finden. Es gelang ihm nicht. Das Drachenherz war verschwunden. Und selbst wenn er es finden könnte, könnte er es in seinem jetzigen Zustand wohl kaum transportieren.

Varkur tauchte in den Strom der Dunklen Magie, der ungehindert in ihn hineinströmte, und versuchte, ihn zu versiegen zu lassen. Er spürte, wie seine Wandlung ein klein wenig stagnierte, aber es war, als würde er versuchen, den Fluss des Tatru mit einem einzelnen Holzstamm aufzuhalten. Erneut stieß Varkur mit seinen Hörnern gegen die Höhlendecke. Wenn diese Transformation so weiterginge, würde er bald hier unter dem Gebirge feststecken. Er musste hier raus, und zwar rasch!

Varkur brach durch die Tür des Raumes in den dahinter liegenden Gang. Ein Paar Gorlots heulten auf und sprangen davon, so schnell sie konnten. Varkur war schneller. Er vermochte sich nicht mehr an den genauen Weg erinnern, den er genommen hatte, um hierher zu kommen. So schrammte er blindlings durch die unterirdischen Gänge des Zwergenreichs, schlug Türen auf und Säulen um, kratzte tiefe Krallenspuren in uralten Stein und schlug mit seinen Hörnern Löcher durch die Höhlendecke. Seine Schattenflügel wogten um ihn herum, wie es einst sein finsterer Nebel getan hatte, zertrümmerten Stein, pulverisierten Fels und brachen Gorlot-Knochen, wo immer sie konnten.

Ein dumpfes Dröhnen erklang, als hinter ihm ein Gang nach dem anderen einstürzte. Varkur legte noch weiter an Tempo zu und holperte und polterte auf allen vieren durch das verlassene Zwergenreich, in der Hoffnung, die freie Luft zu erreichen, ehe sein hünenhafter Echsenkörper zu groß war, um sich fortbewegen zu können.

Er erreichte eine Sackgasse, prallte mit voller Wucht in eine Wand – er spürte es kaum, so sehr federten seine neuen Schatten den Aufprall ab – und hinter ihm stürzten Tonnen von Geröll hinunter, als der Gang zusammenstürzte und den Rückweg blockierte.

Es war stockfinster, nass und kalt. Varkurs Körper schwoll weiterhin an und presste gegen die Höhlenwände. Es wurde immer schwerer, sich noch zu bewegen. Erst als er quasi den ganzen Raum ausfüllte und physisch nicht mehr weiterwachsen konnte, spürte Varkur, wie der Strom der Dunklen Magie in ihn versiegte. Er befand sich nun in einem Raum irgendwo unter dem Grauen Gebirge, der seit Jahrzehnten, vielleicht gar Jahrhunderten von keiner Zwergenseele betreten worden war, eingequetscht in einem unmenschlichen Echsenkörper, umgeben von Dunkler Magie und fremden Schatten, die verzweifelt gegen den von allen Seiten auf ihn drückenden Felsen ankämpften.

Es grauenvolles Knirschen zog sich durch den Fels und Varkur hätte sich seine nun spitzen Riesenohren zugehalten, hätte er sich noch bewegen können. Er fürchtete, das ganze Graue Gebirge würde gleich auf ihn einprasseln und ihn endgültig aus dieser Welt holen. Varkur, der Dunkle Magier, erdrückt von einem Berg. Was für ein lächerliches Ende!

Krachend gab die eine Seitenwand des engen Raums nach, in dem Varkur eingeschlossen war. Varkurs purzelte aus der neu entstandenen Öffnung in Felswand ins Freie (und in den freien Fall), während der kleine Raum hinter ihm kollabierte. Die Höhle hatte direkt an der Korn-Schlucht gelegen und Varkur stürzte unkontrolliert in deren Tiefe!

Varkur ächzte, während der Wind an seinen Schwingen riss, doch konnte er seinen Fall mit den Schattenflügeln aufhalten. Tief unter sich glaubte er, einen riesige steinerne Gestalt mit einem abgebrochenen Horn in der Schlucht liegend zu erkennen. Dann zog sein Blick nach oben und blieb auf der Ruine der Zwergenbrücke Tiefenfall liegen, die die tief darunter ruhende Gestalt der Legende nach vor ihrer Fertigstellung bereits wieder eingerissen hatte. Mit seinen mächtigen Schwingen schleuderte er sich in die Höhe und katapultierte sich aus der Korn-Schlucht heraus.

Erleichtert stellte Varkur fest, dass der Strom der Dunklen Magie versiegt war und seine deformierte Schattengestalt nicht mehr immer größer wurde. Tatsächlich begann sein Körper schon damit, einen Großteil der Dunklen und Schattenmagie freizugeben. Varkur schrumpfte, die Schatten um ihn herum zerflossen, und mehr torkelnd als elegant landete der Dunkle Magier am Rande Tiefenfalls.

„Keine Sorge, wir sind nicht für immer verschwunden“, schien diese ungebändigte Kraft mit den flüsternden Stimmen hunderter Seelen zu wispern, „Wir werden wieder an deiner Seite stehen, wenn du uns wieder brauchst.“

Die fremden Seelen verflogen völlig und hinterließen Varkurs kleinen menschlichen Körper auf dem nackten Fels liegen. Es war eiskalt. Er drehte sich elend zur Seite und kauerte sich zusammen.

Da fiel sein Blick auf eine dunkle Gestalt, welche in einiger Entfernung auf einem Stein saß und ihn unter einer schweren Kapuze heraus interessiert beobachtete.

Varkur klaubte sich zusammen und stemmte seinen schmerzenden Menschenkörper in die Höhe. Die dunkle Gestalt erhob sich zeitgleich mit einer einzigen fließenden Bewegung von ihrem Stein und griff nach etwas Langem neben ihr. Die Luft flimmerte, als wäre es Hochsommer, dabei war das Graue Gebirge genauso von Eis und Schnee erfüllt, wie es Hadria beim Angriff der Mächte des Meeres gewesen war.

Varkurs Herz füllte sich zu dem Grade mit Furcht, zu dem der Dunkle Magier überhaupt noch zu einer solchen Gefühlsregung imstande war. Denn an der Seite dieser Gestalt erkannte er einen grünlich schimmernden Hammer, der eigentlich im Hohen Norden im Eisernen Turm eingeschlossen sein sollte. Er kannte den Namen dieses Hammers, und er kannte den Namen desjenigen, der ihn selbst in die Esse gehalten hatte. Orweyn, der mächtigste aller Zauberer.

Die Hadrische Unterwelt wandelte alles, was durch sie wanderte. Einst war Orweyn mit drei gewöhnlichen Waffen und einem Falken in die Hadrische Unterwelt gegangen und mit drei magischen Waffen und dem riesigen Falken von Yra zurückgekehrt. Doch an Orweyn war den Legenden nach niemandem eine Veränderung aufgefallen. Konnte es sein, dass die Unterwelt auch ihn gestärkt hatte? Dass sie ihm Ewiges Leben verliehen hatte oder sonst erlaubt hatte, die langen Jahrzehnte im Eisernen Turm zu überleben?

Damals, als der junge, schwache Varkur versucht hatte, an diesen Hammer zu kommen, damals, als er das Siegel zum Eisernen Turm gesprengt hatte, war da jemand aus dem Turm geschlüpft? Konnte es sein, dass der Dunkle Magier nun tatsächlich vor Orweyn stand?

Varkur ballte seine Echsenklauen zu Fäusten und rief im Geiste seinen Stab, doch dieser befand sich wohl immer noch unter Tonnen von Stein begraben in den Tiefen des unterirdischen Zwergennetzwerks.

„Mein Qurun!“, rief der schimmernde Orweyn theatralisch, „Das Ende Hadrias, wie ich es vorhergesehen hatte! Wenn Feuer und Turm miteinander ringen, so wirst du sie bezwingen!“

Varkur wusste, was Orweyn in seinen letzten überlieferten Lebenstagen angerichtet hatte, welcher Aufgabe er sich verschrieben hatte. Wenn dieser Magier noch lebte, so nur, um die Spuren der Dunklen Magie ein für alle Mal auszulöschen. Keine sonderlich schöne Aussicht für einen Dunklen Magier.

Noch während Varkur versuchte, sein Gleichgewicht unter Kontrolle zu kriegen und sich kampfbereit hinzustellen, die Schattenseelen wieder hervorzurufen, verschwand das seltsame Schimmern um die fremde Gestalt und enthüllte, dass sie nicht Orweyns Hammer der Stärke in ihrer Hand trug, sondern bloß einen Holzstab, wie ihn die Zauberer von Hadria ihren Speichelleckern austeilten, sobald sie sie in ihren Augen für würdig empfangen.

Und als die Gestalt ihre Kapuze zurückschlug, erkannte Varkur auch, dass das nicht Orweyn sein konnte. Die Person war glatzköpfig wie er, doch trug sie eine Augenbinde und ihre Haut schimmerte bläulich im Licht der aufgehenden Sonne. Das war Leander, der Seher von Narkon, der Hademars Seele in diesen kleinen weißen Kristall eingesperrt hatte. Dem Kristall, der nun entweder unter Tonnen von Stein begraben war oder sich in den Händen von Nomion oder dieser Schattenhexe befand. Nichts davon war eine schöne Aussicht. Und nichts davon würde es Varkur leicht machen, ihn wieder aufzuspüren. Immerhin konnte er sich nun diesen elenden Seher vorknöpfen, ohne dass ihm ein einen kolossalen Kollateralschaden in Kauf nehmender kauziger Kräuterkundler in die Sternkrautsuppe spucken würde.

„Du!“, rief Varkur aus und rief seinen dunklen Nebel hervor.

„Haltet ein, o Qurun“, sprach Leander mit einer Stimme, die nun wieder nach dem Seher und nicht nach dem heiseren Orweyn klang.

Qurun. Das Wort hallte in Varkurs Kopf nach. Er mochte diesen Klang.

„Wir müssen uns überhalten, o Qurun“, fuhr der Seher fort, „Über die elenden hadrischen Zaubererorden. Über die hirnlosen Helden von Andor. Und über Krahd.“

Varkur blieb unentschlossen stehen. Dann überbrückte er die Distanz zwischen sich und dem Seher mit einem gewaltigen Satz und sprach zum erschrockenen Seher:

„Sprich.“

\begin{center}
    Weiter geht es in \hypref{Der Giftzwerg und das Drachenherz (2023)}.
\end{center}












\newpage
\section{Epilog: Eines Hexers Abschied}



Die CAPELLA stach langsam vom Hafen am Wachsamen Wald in die Hadrische See. Meres, der Hexer von Andor, stand an der Reling und blickte gen Norden. Die See war unerwartet ruhig, fast als würde sie ihn erwarten. Nebelfetzen wogten über die Wasseroberfläche. Hier und dort konnte er vereinzelt Nixenschwänze ausmachen, als die scheuen Wasserbewohner vor dem Pfad des Schiffs auswichen .

Meres schnippte unruhig vor sich hin, woraufhin grüne Flämmchen über seine Handfläche flackerten. Er hatte sich nicht an Rekas Wunsch gehalten und brach nun in den Norden auf, ohne vorher seine einstige Lehrmeisterin aufzusuchen. Nunmehin war es schon über vierzehn Jahre her, dass Meres sich im Zorn von seiner einstigen Lehrmeisterin abgewandt hatte. Ihre verschiedenen Sichten der Welt und der Hexenkunst hatten schon einige Male zu Spannungen zwischen den beiden geführt, und die direkte Art der beiden hatte diese Spannungen kaum senken können. Sie hatten auch ihre guten Momente gehabt, wenn sie über einem neuen Trank getüftelt oder ein seltenes Kraut aus dem Barbarenland untersucht hatten. Meres wusste auch, dass die guten Momente die schlechten aufgewogen hatten. Dennoch hatten die beiden sich zu oft gegenseitig nicht ausgehalten.

Und dann war es geschehen. Als Reka verlangt hatte, er müsse eines seiner feurigeren Experimente sofort beenden, hatte Meres zu widersprechen begonnen und die Hexe kurzerhand dieses einfach selbst zu löschen versucht. Dabei konnte man diese Form der Glut nicht mit Wasser löschen. Im Gegenteil. Das hätte sie doch wissen müssen! Aber wie man in Andor so schön sagte: Selbst Brandur war nicht gegen Fehler gefeit.

Grünes Feuer war alles, an das Meres sich danach noch erinnerte. Grünes Feuer war überall gewesen und hatte die Hütte der Hexe vollends niedergebrannt. Die Kinder der Umgebung, die am nächsten Tag bei der Hütte aufkreuzten, um von Reka Lesen, Schreiben und Rechnen (von vielen Eltern abschätzig „Bewahrer-Zeug“ genannt) beigebracht zu bekommen, hatten nur noch grünlich schwelende Ascheresten vorgefunden. Und sowohl die Hexe als auch Meres hatten einander die Schuld in die Stiefel geschoben. Nicht, dass das inzwischen noch relevant wäre.

Jahrelang hatte sich Meres von Andor, dem Königreich der Hexe, ferngehalten. Hin und wieder war er zurückgekehrt, aus Faszination oder einem momentanen Wunsch. Zwischendurch hatte er sogar Reka einige Male erblickt gehabt und die Gefühle des Zorns und des Schams waren jedes Mal erneut hochgekommen, wenn auch immer schwächer. Und so hatte sich Meres immer wieder ab- und neuen Zielen zugewandt.

Neuen Zielen wie dieser mysteriösen Nebelinsel Narkon, von der der Seher Leander Meres erzählt hatte. Meres hatte gefühlt, dass der zwielichtige Seher irgendwelche Hintergedanken hatte, gar eine persönliche emotionale Bindung dazu, was mit dieser Insel geschah. Meres hatte ihm klargemacht, dass er sich nicht auf eine Reise in den Norden begeben würde, ohne zu wissen, warum er dorthin geschickt würde. Und dass er es nicht mochte, hintergangen zu werden.

Da hatte Leander ihm die Wahrheit aufgetischt und beschämt gesagt, dass ein Fluch über der Insel lag und dass eine für Leander sehr wichtige Person dort festsaß, vermutlich, ohne sich überhaupt an ihn zu erinnern. Meres hatte Mitleid gekriegt. Er verfügte über Tränke, Salben und Kräuter, von denen niemand sonst Bescheid wusste. Wer, wenn nicht er, könnte diesen Fluch zu brechen versuchen? Wer, wenn nicht er, würde sich den Gefangenen des Fluches annehmen?

„Meres! So warte doch!“

Meres erstarrte und wurde aus seinen Gedanken gerissen. Das war Rekas Stimme gewesen, und sie kam von hinten, vom andorischen Ufer. Die raue Stimme, die er so oft als zeternd in Erinnerung hatte, klang nun wieder weich und einladend. Er haderte mit sich selbst. Die CAPELLA stach gen Norden, bald schon würde sie zu weit entfernt von der Küste sein, als dass er sich noch sinnvoll mit Reka unterhalten könnte. Er könnte sie wieder vergessen, das alles hinter sich lassen. Für seine Gemütslage wäre das wahrscheinlich am besten.

Aber er könnte auch die Gelegenheit nutzen, sich der Hexe zu stellen.

Meres machte abrupt kehrt, rannte an der Seite des Schiffes entlang zum Heck und spähte in Richtung des Wachsamen Waldes. Im grauen Licht des ersten Morgens wirkten die letzten Baumstämme des Waldrandes schwarz wie die Stäbe eines Käfigs. Und vor diesem Käfig stand sie.

Reka.

Sie trug ihren üblichen zerrissenen grauen Mantel und winkte hektisch dem abziehenden Schiff hinterher. Der Wind trug ihre krächzende Stimme klar und deutlich an Meres‘ Ohren.

„Meres! Halte ein! Es tut mir leid! Ich will dich nicht vertreiben!“

Sie hielt inne. Meres stockte ebenfalls. Selbst die CAPELLA selbst blieb plötzlich stehen. Meres blickte überrascht zu Kapitänin Mondrianne hinüber, welche neben ihrer Tochter am Steuer saß.

Mondrianne meinte belustigt: „Tu dir keinen Zwang an, Hexer, aber ich hab die alte Reka meiner Lebtag noch nie so schnell anrennen sehen. Ich will wissen, wie das ausgeht. Die CAPELLA fährt erst weiter, wenn ihr eure Angelegenheit geregelt habt.“

Meres‘ Blick wanderte wieder zu Reka. Er wusste nicht, was er antworten sollte. Er empfand schon lange keine starken Emotionen gegenüber Reka mehr, oder zumindest redete er sich das ein. Auch ihm tat es leid, was zwischen ihnen vorgefallen war.

„Ich habe einen Fehler begangen“, schrie Reka weiter. Kurz hielt sie inne, wohl ob der Mehrdeutigkeit ihres Zugeständnisses, dann meinte sie: „Ich habe dem Seher Leander von dir erzählt. Er manipuliert dich. Narkon ist eine Falle!“

„Bin ich zu weit weg, um meine Gedanken zu lesen?“, rief Meres ruhig zurück, „Meine Entscheidung steht fest.“

„Ach ja. Deine Meinung hat sich schon immer verhärtet, wenn man dir mit Rat zu kommen versucht. Aber du wirst es alleine schlicht nicht schaffen, dieser Falle zu entkommen.“

„Immer meinst du, die Zukunft bereits zu kennen, Reka. Keiner kann das mit Sicherheit. Ich will zumindest versuchen, diesen leidenden Wesen zu helfen. Das ist das, was ein Held tun würde.“

Reka blieb still.

Warum hatte die Hexe ihr Wiedersehen so publik machen müssen? Andererseits gaben ihm die hinüberspähenden Zuschauer auf dem Schiff zumindest einen guten Grund, das Gespräch schnell hinter sich zu bringen. Und was hatten Reka und er sich auch groß zu sagen? Sie wussten beide, was damals vorgefallen war. Sie kamen nicht groß miteinander klar, aber sie hassten einander auch nicht mehr. Dafür war zu viel Wasser die Narne hinuntergegangen in der Zwischenzeit. Jetzt war seine Gelegenheit, einen Schlussstrich zu setzen.

„Auch mir tut es leid“, sprach Meres schlicht. War das eine gute Antwort? Eine sinnvolle? Eine hilfreiche?

Stille. Er wandte seinen Blick ab und schnippte wieder unruhig vor sich hin, woraufhin kleine kalte grüne Flämmlein an seinen Fingerspitzen auftanzten.

Rekas Stimme ertönte, und zum ersten Mal glaubte Meres, wieder ein bisschen von der Zuneigung zu spüren, die Reka ihm früher immer entgegengebracht hatte.

„Mir bleibt wohl nichts anderes übrig, als dir viel Glück zu wünschen auf deinem weiteren Weg“, sprach Reka sanft, „Das ist eigentlich auch das, weswegen ich dich überhaupt erst wiedersehen wollte. Ich will...“

Die sonst so selbstsichere Reka stockte an den Worten, als sie fortfuhr: „Ich will nur ausdrücken, dass du Andor nicht mehr meinetwegen meiden musst. Wir haben beide Fehler begangen, aber diese sind lange her. Wir können sie hinter uns lassen, wenn wir das beide wollen. Ich werde hier sein, wenn du Lust hast, von deinen Reisen und deinen Erkenntnissen zu berichten. Es würde mich sogar freuen Das nächste Mal, wenn du zurückkommst. Falls. Du verstehst schon. Viel Glück mit Narkon, Meres. Du wirst es brauchen.“

Meres wich weiterhin ihrem Blick aus.

Doch er lächelte.








\begin{chapterbox}
    \chapter{Der Giftzwerg und das Drachenherz (2023)}
    \label{Der Giftzwerg und das Drachenherz (2023)}
    \az{71 und 72}

    \begin{center}
        Fortsetzung von \hypref{Der Giftzwerg und die Sphäre (2020)} und \hypref{Qurun (2021)}
        
        einst versteckt unter dem Code 79082
    \end{center}
    
    Kjall, der diebische Sammler wertvoller Artefakte aus allen möglichen Zeiten, nutzt eine Konfrontation zwischen Varkur, Nomion und Shan. Er erringt Varkurs Zauberstab für sein geheimes Museum, lässt dabei aber auch spontan einen blutroten Edelstein mitgehen – und bereut diese Tat prompt, als \textit{Etwas} mit ihm Kontakt aufnimmt.
\end{chapterbox}

\az{Jahr 71}

Vorsichtig schlich sich Kjall an die Felsspalte heran und spähte in die Tiefe. Er erinnerte sich noch gut an den Vorfall, auch wenn es für ihn selbst schon einige Zeit her war. In der heutigen Nacht würde das gesamte Graue Gebirge gehörig durchgerüttelt werden. Den damaligen Gesprächen der Zwerge nach, die das Beben miterlebt hatten, würde ein gewaltiger Teil des verlassenen unterirdischen Zwergenreichs nahe der Korn-Schlucht einstürzen. Die Tiefminen würden besonders viele Flammen spucken, Kjall selbst würde sich nur knapp hinter einem Roteisenschild ducken und retten können. Einige seiner Kameraden würden nicht so glücklich sein.

Auch viele der höheren Schildzwerge waren erzürnt gewesen über die heutige Zerstörung uralter unterirdischer Bauwerke, auch wenn selbige bereits seit Jahrzehnten verlassen waren. Kjall sah die Lage ähnlich. Und ebenso vertraute er den ehemaligen Handwerkern der Schildzwerge, jahrhundertelang stabil bleibende Gänge zu bauen. Sie waren verlassen gewesen, weil das Volk der Schildzwerge leider schon lange nicht mehr so zahlreich war wie damals und weder alle Gänge vor finsteren Kreaturen verteidigen konnte noch wollte – was würde man mit all dem Extraplatz auch anfangen – aber sie waren definitiv nicht verlassen gewesen, weil Einsturzgefahr bestanden hätte.

Nein, Kjall stimmte den alten Rauschebärten zu, die felsenfest davon überzeugt waren, dass da Fremdeinwirkung im Spiel gewesen war. Fremdeinwirkung hieß, dass jemand Mächtiges sich eingemischt hatte und diese Gänge hatte einstürzen lassen. Und jemand Mächtiges im Grauen Gebirge bedeutete nun, wo Kjall viele Jahre älter war, dass er sich eventuell ein weiteres mächtiges Souvenir für sein geheimes Museum gönnen könnte. Schließlich war er nun ausgestattet mit der Sphäre, einem mächtigen Artefakt, welches es ermöglichte, Zeitportale in Vergangenes und in Zukünftiges zu öffnen,

So war Kjall in die Vergangenheit gehopst und schlich nun vorsichtig der Korn-Schlucht entlang, in der Hoffnung, den Urheber des Einsturzes ausfindig machen zu können. Und dem war so gewesen. Ein altbekanntes Schaudern hatte ihm schnell zu verstehen gegeben, dass ein mächtiges Wesen präsent war. Jemand, den Kjall kannte. Jemand, der sich hoffentlich nicht allzu sehr um Kjall scheren würde, da sie ihn noch gar nicht kannte. Kjalls alte, schattige Freundin, die einst als Teil von Kjalls Schar mit ihm in die Vergangenheit reisen würde, war offenbar heute in der Gegend unterwegs. Kjall bewegte sich vorsichtig in die Richtung, aus der dieses unwohle Gefühl stammte, welches Shan stets wie ein Schleier umgab. Dese Form von schleichender Kälte, die ihn an eine mondlose Nacht und an ein schattiges Tal erinnerte. Sie war unverwechselbar. Bald schon beobachtete Kjall, wie eine Masse aus festem Schatten in eine Felsritze ins Innere des uralten Zwergenreichs schlüpfte.

Kjall konnte seinen Augen kaum glauben, als er durch die kleine Felsspalte in einen tiefen Raum blickte und erkannte, wer dort unten an einem Blutsteinaltar stand, einen mächtigen roten Kristall in die Höhe hob, und verzweifelt mit Kjalls schattiger Freundin rangelte. Ein gewisser mächtiger Dunkler Magier, dessen Stab sich nur äußerst gut in Kjalls Kollektion machen würde. Von diesem roten Kristall ganz zu schweigen. Kjall rieb sich die schwieligen Hände.

Seit seinen ersten, unbeholfenen Zeitreisen hatte Kjall sich ein rigoroseres Protokoll zusammengestellt, wie er in einer solch lebensgefährlichen Situation vorgehen würde:

1. Wenn er sich selbst nicht aus der Zukunft auftauchen sah, würde er in die Vergangenheit reisen und irgendetwas Verrücktes, Aufsehenerregendes tun, das sein vergangenes Ich definitiv hätte wahrnehmen müssen (z.B. eine Steinflöte zücken und einen wilden Flötentanz aufzuführen beginnen).

2. Wenn er sich selbst aus der Zukunft auftauchen sah und seinem zukünftigen Ich etwas Unschönes geschah, so würde Kjall dessen Vorgehen genau beobachten, an diesen Moment zurückreisen und dann irgendetwas Aufsehenerregendes tun, das sein zukünftiges Ich definitiv nicht getan hatte (z.B. mitten in den Kampf zwischen seiner schattigen Freundin und dem Dunklen Magier springen und dabei laut die Melodie von Steinsang aus voller Kehle grölen).

3. Wenn er sich selbst aus der Zukunft auftauchen sah und sein zukünftiges Ich erfolgreich die lebensbedrohliche Gefahr umging und die zu erringenden Gegenstände errang, so würde Kjall dessen Vorgehen genau beobachten und dieses dann erfolgreich kopieren.

Aufgrund der Art und Weise, wie Zeitreisen mit der Sphäre von Cavern funktionierte, konnte Kjall ja nur in dieselbe Zeitlinie reisen, aus der er kam. Folglich konnte bloß dritte Fall eintreffen. Denn damit die Zeitreise-Logik nicht gebrochen wurde, musste Kjall sich wie sein zukünftiges Ich verhalten. Und Kjall hatte sich fest vorgenommen, die von seinem zukünftigen Ich beobachteten Handlungen nur zu kopieren, wenn der zukünftige Kjall glücklich mit dem Ergebnis war. So konnte er quasi das Schicksal erpressen: Ich verhalte mich nur so, wie du willst, wenn das Ganze so herauskommt, wie ich es will.

Ohne dass er je einen Finger dafür krümmen musste, wurde ihm quasi vom Schicksal selbst sein eigener bester Plan zur Ergatterung seltener Objekte aus Gefahrensituationen auf dem Silbertablett serviert. Was für ein Geschenk die Macht der Zeitreise doch war. Was für ein Narr er gewesen war, als er sich bei seiner ersten Zeitreise über die Unerschütterbarkeit der Zeitlinie geärgert hatte! Kjall kicherte. Damals war ihm dieses Korsett wie eine Limitation vorgekommen, das Xolls Leben vor ihm verwehrte, heute hingegen wie ein Geschenk. Solange er die Sphäre und genügend Treibstoff besaß, konnte ihm nichts Böses passieren.

Kjall hatte das Glückslos des Schicksals gezogen und lebte in der unglaublich unwahrscheinlichen Zeitlinie, in der er an die Sphäre gelangt war. Und die sich deshalb ganz nach seinen Wünschen entwickelte.

Sich strikt an diesen Plan haltend, blickte Kjall in die Höhle hinein, wo sich seine schattige Freundin und der Dunkle Magier gehörig um sich schlugen. Zaubersprüche, dunkler Rauch und Fetzen Fleisch gewordenen Schattens wirbelten umher. Kjall blickte aber nicht auf das Getümmel, sondern suchte nach einer Spur seines zukünftigen Selbsts. Tatsächlich erkannte Kjall kurz darauf durch den Felsspalt, wie sich tief unter ihm in der Höhle ein blaues Zeitportal öffnete und ein zukünftiger Kjall hinaustrat. Kjalls schattige Freundin und der mächtige Dunkle Magier waren zu sehr in ihren Kampf verwickelt, als dass sie sich groß um den Neuankömmling scherten. Vermutlich war er ihnen nicht einmal aufgefallen. Gerade schleuderte ein Schattententakel den roten Kristall aus der Hand des Magiers in eine Ecke des Raumes, wo bereits der Zauberstab des Magiers gelandet war. Kjall grinste und notierte sich den Zeitpunkt auf seinem Chronometer.

Sein zukünftiges Ich massierte seine Schläfen und orientierte sich. Es war nie angenehm, durch das Zeitportal zu reisen, aber immerhin schaffte Kjall es inzwischen, nicht mehr ohnmächtig zu werden bei kurzen Reisen durch den Zeit-Limbus zwischen den Zeitportalen, diesem endlosen Raum aus gleißendem blauem Licht, in dem es keine Schwerkraft zum Orientieren und keine Luft zum Atmen gab.

Dann zwinkerte der zukünftige Kjall dem jetzigen weit über ihm zu und marschierte schnurstracks auf den roten Kristall und den Zauberstab zu. Er ergriff beide, sprang zurück in das Zeitportal, aus dem er gekommen war, und das Portal verpuffte in blauen Wirbeln.

Mission erfüllt.

Kjall wandte sich grinsend ab und suchte in seinem Kopf schon nach dem kürzesten bekannten Weg zur Höhle, in der er sein zukünftiges Ich gerade hatte auftauchen sehen. Durch die Zeit mochte er reisen können, wie es ihm beliebte, doch den Ort seines Auftauchens musste er weiterhin manuell aufsuchen.

Er war frohen Mutes.

Auch dieses Vorhaben von ihm würde von Erfolg gekrönt sein.

Das Schicksal wollte es so.\bigskip




\az{Jahr 72}


Der zukünftige Kjall stolperte aus einem Zeitportal in seine eigene Zeit, fernab der kollabierten Gänge, und hielt prompt inne, um stolz seine beiden soeben errungenen Schätze zu betrachten.

Der schwarze Zauberstab des Dunklen Magiers war ein Meisterstück, auch wenn Kjall ihn selbst nicht verwenden konnte. Doch dieser rote Kristall war es, der Kjalls Interesse besonders geweckt hatte. Er war zwar nicht so groß wie das mächtige Drachenauge, das Fürst Kram – Kjall spuckte aus – in einigen Monaten in den Tiefen der Tiefminen finden würde. Beachtlich war es dennoch, insbesondere aufgrund der blutroten Farbe und des tiefschwarzen Schemens, der unter seiner Oberfläche umherzuwandern schien. Und jemand hatte zwar mehr schlecht als recht uralte Zwergenrunen in seine Oberfläche geritzt. Etwas Magisches hatte dieser Stein an sich, dessen war sich Kjall sicher.

Die glatte Oberfläche kribbelte unter seinen Fingern, fast, als würden diese ertauben. Rasch steckte Kjall den Edelstein in seinen Mantel.

Dann machte er sich auf den Weg in sein geheimes Museum tief in den Tiefminen. Die engen Gänge waren erfüllt von emsiger Bewegung. Dutzende Kjalls rasten umher, auf das Museum zu oder vom Museum weg, mit errungenen Schätzen in ihren Händen oder Plänen zur nächsten Errungenschaft in ihren Köpfen. Kjall hatte sich den heutigen Tag in dieser Zeit ausgesucht, um sein Museum in einem Schlag auf den neusten Stand zu füllen. Er fühlte sich ganz hibbelig, wie jedes Mal, wenn er seine Sammlung betrat. Rund um ihn herum eilten andere Kjalls, kreuzten seinen Weg oder nickten ihm freundlich zu, allesamt ähnlich grinsend wie er. Jeder einzelne von ihnen war er bereits gewesen oder würde er bald sein. Er fühlte sich wie in einem triumphalen Fiebertraum.

Dann war er da, bei seinem geheimen Museum. Er senkte seinen Kopf ehrfürchtig vor der großen Hammeraxt von Xoll Hammeraxt, seinem ehrenwerten Vater. Diese Waffe hatte über Kjalls Werkstatt gehangen, nun jedoch auch in seinem Museum. Sie hatte ihren Platz verdient neben Magischen Waffen, Mächtigen Schilden, Gaben aller Völker der bekannten Welt und dem einen oder anderen verschollenen Gegenstand.

Sechs Tiefminen-Golems, je zwei Versionen jeder seiner Kreationen, eilten im Museum umher, nahmen Schätze entgegen und deponierten sie sorgfältig auf Sockeln. Einmal mehr verfluchte Kjall Fürst Hallwort dafür, diese Wunderwerke der Mechanik, angetrieben von uralten Golemkernen, geschmiedet aus dem reisten Roteisenerz, in der Vergangenheit einfach eingeschmolzen zu haben. Dieser sonst so neugierige Dödel von einem Fürsten hatte das Roteisen wohl für eigene Zwecke nutzen wollen, statt diese Kunstwerke zu studieren. Elend! Doch immerhin hatte Kjall seine Golems jetzt wieder. Und nicht nur sie. Weitere metallene Helfer, solche, die für ihn vor langer Zeit in den Baum der Lieder eingebrochen waren und ihm erstmals die ersten Blätter des uralten Berichts zur Sphäre verschafft hatten, wuselten am Boden umher, beschrifteten die erjagten Schätze akribisch und signalisierten den verschiedenen Kjalls, wohin sie zu gehen hatten. Es war das reinste Chaos. Kjall würde keine Freude daran haben, dies alles zu organisieren, aber diese Aufgabe schob er fröhlich seinem zukünftigen Selbst zu. So wortwörtlich hatte er noch nie prokrastinieren können.

Gerade sah Kjall, wie einer seiner kleinen metallenen Helfer eine von Säure zerfressene Konstruktion auf einem Sockel deponierte. die in etwa die Form eines Arms hatte. Eine Prothese, die Kjall höchstpersönlich angefertigt hatte. Er knurrte grinsend beim Gedanken daran, sie erschaffen zu haben.

Angefangen hatte all dies mit der alten Graella aus den Tiefminen. Sie hatte ihre beiden Beine bei einem Feuerstoß aufgrund einer Grubengasexplosion verloren. Die Heiler hatten sie nur mit Stumpen zurückgelassen. Die Arme konnte so nicht mehr arbeiten, kein Teil der Kumpels mehr sein. Da hatte Kjall sich ein Herz gegriffen. Er kannte sich mit Tiefminen-Golems aus, mit mechanischen Armen und Beinen, und mit eigenständigen Läufern. Es war ein leichtes gewesen, der alten Graella metallene Beine zurückzugeben. Lange hatte er daran gearbeitet und sich an Graellas Reaktion erfreut. Gelegentlich musste sie nachölen, aber ihre neuen Füße waren sogar besser als ihre alten.

Doch dabei war es nicht geblieben. Als nächstes verlor der kleine Halbun einen Finger unter einem Hammer und fragte bei Kjall nach. Dann beschwerte sich ein Zwerg darüber, dass Graella schneller rannte als er selbst, und wollte besondere Zwergenstiefel von Kjall, die ihn gleich schnell machen würden. Dachten sie etwa, dass Kjall sich nur damit auseinandersetzte? Er war immer noch ein Tiefminen-Arbeiter! Er war ein Förderer von Bodenschätzen, ein Erforscher des Wissens, ein Sammler von Trophäen, kein Heiler für die Bedürftigen! Und die eigentlichen Heiler Caverns wollten nichts von seinen unorthodoxen Methoden hören, die verstanden sich bloß auf Fleisch und Blut. Nicht, dass dies Kjall gestört hätte, er wollte auch nicht irgendwelchen vorlauten Knirpsen seine Berufsgeheimnisse verticken. Er wollte einfach nur, dass er in Ruhe gelassen wurde. Die Altvorderen waren schließlich auch ohne moderne Prothesen ausgekommen.

Und es hatte nicht mit Zwergen geendet. Kjall spuckte aus. „Dies ist eindeutig kein Produkt aus zwergischen Werkstätten, dafür hat es zu viel Technik und zu wenig Magie in sich“, hatte diese hochnäsige Priesterin Gända vom Baum der Lieder abschätzig gesagt, als sie Graellas mechanische Beinkonstruktion inspiziert hatte – nur um einige Monate später dann dennoch diesen alten Wolfskrieger zu Kjall zu schicken. Wilselm hieß der Wolfskrieger, und einen ganzen Arm hatte er im Kampf gegen den Schwarzen Herold verloren. Und nun hatte er sich einen ganzen neuen Arm von Kjall gewünscht! Dabei hatte ja noch einen anderen, und er wollte ihn auch gar nicht, um zu kämpfen, sondern, weil es ihm praktischer im Alltag wäre. Was für ein Weichling. Und ein Mensch war es auch noch, einer aus Brandurs Schar, dieses Landräubers, der für Xolls Tod verantwortlich war.

Kjall kicherte bei der Erinnerung daran. Ja, er hatte dem Wolfskrieger Wilselm eine Konstruktion gebastelt. Nur nicht genau so, wie dieser wollte. Und schon bald hatte dieser sich nicht mehr danach erfreut.

Sein metallener Helfer lenkte Kjall zu einem freien Wandstück, in das ein Set aus Haken eingelassen war. Ehrfürchtig platzierte Kjall den Zauberstab des Dunklen Magiers darauf und trat einige Schritte zurück. Lange konnte er den Schatz jedoch nicht mehr bewundern, ehe er von seinem metallenen Helfer gleich wieder aus dem Raum gescheucht wurde, um Platz für weitere Tiefminen-Golems und Kjalls zu machen.

Kjall grinste, als er im Weggehehen seinen Blick über seine neusten Errungenschaften schweifen ließ: Eine andorische Flöte, die Brandur selbst gespielt hatte. Ein hadrischer Kompass, der den Seekrieger Ruuf damals zur mythischen Insel Danwar geleitet hatte. Boords Hammer, mit dem er die große Statue in der Halle der vier Schilde aus dem Stein gehauen hatte. Die Kapuze des Hral, die ihren Träger beinahe mit dem Schatten verschmelzen ließ. Ein Takuri-Spiegel aus der fernen Metropole Agarb.

So viele Möglichkeiten, was Kjall sich als nächstes beschaffen konnte.

Das Hauptaugenmerk des geheimen Museums waren natürlich die Sockel der vier mächtigen Schilde aus uralter Zeit. Drei hatte Kjall bereits erbeutet, und sie standen auf Hochglanz poliert nebeneinander auf ihren angestammten Sockeln: Der Sternenschild, der Bruderschild und der Silberschild. Nur der Feuerschild fehlte noch, stattdessen war auf dessen Sockel eine Skizze des Schwertmeisters Harthalt abgebildet, wie er den markanten dreieckigen Schild führte. Ob dieser Lackaffe überhaupt gewusst hatte, was für einen Schatz er da getragen hatte?

Dieser Schild würde Kjalls Sammlung vervollständigen, sein kleines Museum perfektionieren, seinem Lebenswerk die Krone aufsetzen. Sobald Kjall genug von seinen kleineren heutigen Errungenschaften hatte, würde er den Feuerschild aus der Vergangenheit holen und ins Museum bringen, aber erst am morgigen Tag. Diesen Moment der Perfektion und Erfüllung wollte er sich nicht durch ein Gewusel aus Kopien seiner Selbst, seiner Golems und seine metallenen Helfer stören lassen, und erst recht nicht wollte er sich diesen Moment, auf den er so lange hingearbeitet hatte, spoilern. Nein, den Feuerschild würde er allein holen, andächtig, nachdem er die Lust am ganzen Gewusel hier verloren hatte. Er war schon gespannt, wie viele Magische Waffen er bis dahin erringen konnte. Von Varlion dem Flammenschwert hatte er bereits eine Vermutung, wann er es erringen konnte, und Carlion das Kälteschwert stand sogar bereits in seiner Scheide in der hinteren rechten Ecke des Museums. Die Fürstenkrone der vier Schilde fehlte ihm ebenso wie die Rietgraskrone, doch die Krone der Nordmeere lag sicher verwahrt in einer staubfreien Vitrine hinter ihm. Mehr aus Spitzfindigkeit hatte er sogar den legendären goldenen Eintopftopf der Mutter des Zwergenfürsten Kram errungen, den er in einer Ecke als Spucknapf verwendete. Krams Mutter, wie hieß sie nun schon wieder? Murna? Oder doch eher Bairen? Egal, es war nicht wichtig.

Kjall selbst verstand nicht ganz, was hier alles abging, was für Artefakte hier alle noch in seinem geheimen Museum eintreffen und von emsigen metallenen Helfern einsortiert würden. Er war jedoch überzeugt davon, dass alles nach Plan lief.

Gedankenverloren zog er den roten Kristall hervor, den er in der Vergangenheit errungen hatte. Seine Fingerspitzen kribbelten. Dunkle Schemen pulsierten unter der Oberfläche. Was war dieses Objekt nur? Wo sollte er es hinstellen, wie sollte er es beschriften?

Direkt neben Kjall öffnete sich ein blaues Portal. Heraus stolperte eine weitere verschwitzte, doch grinsende Version seiner Selbst, die in ihrer Hand eine eiserne Kugel schwenkte, welche grünlich dampfte. Eine Kugel von Kreatoks Fallengas, direkt aus der großen Schmiede des Wunderzwergs höchstpersönlich! Leider wusste Kjall bereits, dass dieses Artefakt (oder zumindest sein Inhalt) bald schon wieder sein Museum verlassen mussten. Aber da ragte auch ein Fangzahn von Nehal höchstpersönlich aus der Tasche seines anderen Selbsts, und dieser würde ein schöner dauerhafter Bestandteil seiner Kollektion werden. Hach, die glorreichen Urzeiten des Zwergenreichs vor dem Unterirdischen Krieg. Vielleicht könnte Kjall dort Ferien machen gehen, wenn er hier fertig war. Er nickte seinem anderen Selbst anerkennend zu, welches zurückzwinkerte.

Aus einer dunklen Ecke von Kjalls Museum löste sich eine kleine Gestalt, die Kjall gerade mal bis zum Knie reichte. Ihr Körper war eine wabernde silberne Masse, die sich weigerte, scharfe Konturen anzunehmen. Nichtsdestotrotz glaubte Kjall, Arme und Beine ausmachen zu können, als dieses mysteriöse Wesen mit raschen Schritten auf das sich schließende Portal in die tiefe Vergangenheit zustürzte und darin verschwand. Es vergingen nur Augenblicke, bevor das Portal sich mit einem leisen Plopp wieder schloss.

Kjall blickte sich argwöhnisch seinem anderen Selbst nach, um zu sehen, was dieses davon hielt, doch war jenes bereits weitergelaufen und hatte sich in die Menge der Kjalls gemischt. Und länger konnte sich Kjall nicht mit diesem eigenartigen Vorfall befassen, denn in diesem Augenblick regte sich etwas tief in ihm. Seine Hand kribbelte und juckte. Kjall blickte argwöhnisch auf den roten Kristall, den er darin hielt, und befahl seinen Fingern, sich zu lösen und das Ding fallen zu lassen.

Seine Finger gehorchten ihm nicht mehr.

Kjall versuchte, sich zu rühren, um Hilfe zu schreien, irgendetwas zu tun, außer still und erstarrt im Raume zu stehen und regelmäßig ein- und auszuatmen. Es gelang ihm nicht. Da war eine gewisse Kälte, die durch seinen Arm pulsierte und bis in seinen Kopf vordrang. Kjall erschauderte. Da war Etwas, das sich gegen seinen Geist schmiegte. Gedanken, Erinnerungen und Gefühle wallten in ihm auf, die ihm unbekannt, fremd, falsch erschienen. Eine Präsenz war da, ein Dunkles Etwas, das ihn sondierte. Die Berührung dieser fremden Entität war schmerzhaft. Sie wirkte gleichzeitig zaghaft, als wäre das Wesen vorsichtig, und rabiat, als wäre das Wesen ungeübt in dem, was es hier tat. Und unzweifelhaft gingen all diese seltsamen Empfindungen vom roten Edelstein in Kjalls Händen aus.

Warum half ihm niemand? Warum schlug ihm kein zukünftiger Kjall den Kristall aus seinen Händen? Warum hatte ihn keiner gewarnt?

Eine tiefe Stimme scholl durch Kjalls Kopf, lauter als jede Stimme, die er in seinem Leben je gehört hatte. „WAS?! WIE KANN ICH ... WAS ... WER BIST DU?“

Kjalls ganzer Körper zitterte angespannt, doch abgesehen davon vermochte er, keinen Finger zu rühren. Die dumpfe Stimme schrie weiter: „AH, MEIN KRISTALLENES GEFÄNGNIS ... VARKUR GELANG ES BEREITS, MEINE GITTERSTÄBE ZU LOCKERN. EIN TEIL MEINER SELBST VERMAG ES, HERAUSZUFLIESSEN, UND DA BIST DU, UM MEINE GEDANKEN AUFZUFANGEN. WELCH WUNDER! WER BIST DU? EIN ZUKÜNFTIGER? EIN ZEITREISENDER? DIESE SPHÄRE ... DIESE MACHT!“

„Eine Macht, die dir nicht zusteht. Wer oder was bist du?“, dachte Kjall so laut, wie er konnte. Das Wesen beachtete ihn nicht, sprach jedoch immerhin erheblich leiser weiter.

„Du hast mich gerettet! Du hast mich vor Nomion bewahrt! Du hast mich Varkur und Shan entrissen! Du hast mir eine neue Zukunft ermöglicht. Du bist ...“ Der Druck auf Kjalls Schädel nahm zu. „... du bist Kjall, Sohn des Xoll. Und du hast einen Blick auf diese Welt, auf diese Zeit, wie ich selten einen wahrnahm. Wir können gemeinsam Großes erreichen, wir beide.“

Kjall ignorierte das Gefasel des Wesens. Er interessierte sich nicht dafür, wer dieser Nomion war, und er verfluchte Varkur und Shan gerade dafür, dass sie seine Aufmerksamkeit auf sich gezogen und ihn in diese Situation gebracht hatten. Kjalls Arm zuckte zur Seite, jedoch ohne den Kristall fallen zu lassen. Er hatte sich ein klein wehren können, entgegen dieser Beherrschung! Nun war er wieder erstarrt, doch wenn er sich nur ein klein wenig mehr anstrengte ...

Das Böse in ihm kicherte. „Doch zunächst einmal müssen wir uns an einen sicheren Ort begeben, wo ich diese neue Fähigkeit trainieren kann. Die Erstreckung meines Geistes auf Körper, die mein kristallenes Gefängnis berühren, das wird mir sehr zu Diensten sein ... und dass ich von allen Personen das Glück hatte, auf dich zu treffen, einen Zeitreisenden! Varkur und ich haben lange über diese Möglichkeiten gesprochen, damals in unserer gemeinsamen Zeit in Nordgard. Varkurs Mentor Koraph war äußerst fasziniert vom Studium der Zeit, auch wenn er diese nie so sehr wie Kirr beherrschte. Varkur berichtete von Hadrischen Stundengläsern, die die Zeit verlangsamen konnten, und davon, wie er und sein Mentor Koraph nächtelang darüber getratscht hatten, was es denn bedeuten würde, wenn man die Zeit nicht nur komplett anhalten, sondern vielleicht sogar so sehr verlangsamen könnte, dass sie rückwärts abliefe. Sie kamen zum eindeutigen Schluss, dass Zeitportale, wie sie sich beispielsweise in den Tiefen des Horun zu manifestieren scheinen, einen nur in dieselbe Zeitlinie befördern können, aus der man stammt. Und dass es theoretisch keinem bekannten Gesetz der Natur widerspräche, einen Abstecher ins Vergangene oder Zukünftige zu machen ... doch diese Sphäre, dieses Wunderwerk ... ich hätte es mir nie auszumalen gewagt, dass es so einfach wäre!“

Wie von selbst bewegten sich Kjalls Arme, verschoben Hebel und Schalter auf seiner Sphäre und öffneten ein rotierendes Zeitportal in eine einige Jahre zurückliegende Vergangenheit. Und auch wenn Kjall sich so stark wie möglich dagegen wehrte, konnte er nicht verhindern, dass sein Körper sich hineinwarf.





\az{Jahr 48}

Kjall wurde nicht ohnmächtig. Sein Körper plumpste durch das andere Zeitportal hinaus und schlug hart auf felsigem Boden der Vergangenheit auf. Das Portal wirbelte noch kurz vor sich hin, ehe es sich zusammenzog und mit einem leisen Plopp verschwand.

Dann war alles still.

Totenstill.

Noch immer spürte Kjall diese fremde Macht, die seinen Geist beherrschte. Wie ein dunkler Schemen am Rande seines Gesichtsfelds, der versuchte, selbiges zu überdecken und seinen Geist schlafen zu schicken. Doch noch schien dieser fremde Geist, überwältigt von der Zeitreise durch den Limbus, nicht die Macht dazu zu haben, ihn zu unterdrücken.

Kjall blickte sich um. Eine äußerst unangenehme Erfahrung, nicht bestimmten zu können, wohin seine Augen rollten. Dies war natürlich derselbe Raum, der sein geheimes Museum werden würde, doch befanden sich hier in der Vergangenheit noch keine errungenen Schätze darin. Noch nicht einmal das zertrümmerte Hadrische Stundenglas, welches Kjall nach der ersten Schlacht um die Rietburg ergattern würde. Stattdessen lagen auf den Sockeln erste aus Roteisen gefertigte Werkzeuge in allen erdenklichen Formen. Werke, auf die Kjall durchaus mit Freude zurückblicke – von ihrer Herstellung stammten viele der Brandnarben, die er mit Stolz trug. Dennoch waren diese Werke nichts im Vergleich zu den Tiefminen-Golems, die er später erschaffen würde.

„Welch ... widerliche Erfahrung, diese Reise durch den Zeit-Limbus“, sprach die finstere Macht mit seiner Kehle, jedes Wort verzerrt, als wäre es noch nicht an seine Stimmbänder gewöhnt. Kjalls Mund spuckte aus. Wie von selbst setzten sich Kjalls Beine in Bewegung, stolperten los durch die Tiefen der Tiefminen, auf Pfaden, die dieses Wesen aus seinen Erinnerungen gelesen haben musste, denn es verirrte sich nicht einmal auf feuerstoßgefährdeten Pfaden und ging nicht einmal einen Umweg.

Kjall hoffte darauf, dass ihm jemand entgegenkommen und ihn retten würde, irgendein Kumpel aus den Tiefminen, Radan, Drak, ja, selbst einer Begegnung mit Kram würde er erleichtert entgegensehen. Doch die Gänge blieben leer.

An einem kleinen Höhlenausgang irgendwo in den nördlichen Ausläufern des Grauen Gebirges kamen sie ins Freie. Ein mondloser Sternenhimmel beschien hohe Gipfel und vereinzelte Bäume. Die Finsternis der Nacht lag über den Gipfeln und Hügeln. Kalte Luft drang in seine Lungen und versorgte seinen vom Dauerlauf geschundenen Körper mit frischer Energie.

Kjall hatte inzwischen aufgegeben, mit dieser fremden Macht verhandeln zu wollen, die seinen Körper immer selbstsicherer dirigierte. Angst keimte in ihm auf, als er zu verstehen versuchte, warum kein zukünftiges Selbst ihn retten kam. War er von nun an verdammt, ein Sklave dieser fremden Macht zu sein? Waren alle zukünftigen Kjalls, die er in seinem geheimen Museum gesehen hatte, von diesem Wesen gesteuert worden?

„So“, sprach das Böse heiser in Kjalls eigener Stimme. „Du hast dich bislang beinahe brav an meine Anweisungen gehalten. Doch sind wir auch einige Male fast gestolpert, weil du dich mir zu widersetzen versuchtest. Wenn ich deinen Körper kontrolliere, dann doch bestimmt auch das weiche Ding, das in deinem Schädel herumschwimmt und dich schlafen lassen kann? Lassen wir uns doch ...“

Etwas pochte in seinem Kopf. Furcht wallte in Kjall auf, als eine unheimlich starke Müdigkeit aus dem Nichts durch seinen Körper schoss und seine Augen zufielen. Wenn dieses Ding seinen Geist einschlafen lassen und seinen Körper wie den eines Schlafwandlers fernsteuern könnte, wäre es erst recht um seinen Widerstand geschehen! Wobei, wenn er sich ohnehin nicht gegen sie wehren konnte, war ein Traum oder traumloser Schlaf womöglich Jahren der Agonie vorzuziehen.

Doch, ehe er eine Entscheidung fällen oder sich dem Mantel der Ungewissheit und Ohnmacht hingeben konnte, wurden Kjall und sein Tormentor durch ein fremdes, fernes Singen abgelenkt.

„Still“, knurrte das Böse, während es langsam zur Quelle dieser Geräusche schlich. Kjall, welcher ohnehin nicht hätte schreien können, horchte bloß. Das war nicht nur eine Stimme, das war ein ganzer dissonanter Chor, der in einer ihm fremden Sprache vor sich hin dudelte.

Dann lief Kjalls Körper um einen Felsen herum und sie beide erblickten in der Ferne die Urheber des gruseligen Gesangs.

Schwarz ragte der Dunkle Tempel in den grauen Himmel auf. Der riesige Turm war ein vielstöckiger Bau, in Kjalls eigener Zeit nur noch eine Ruine, von den rachsüchtigen Helden von Andor bis auf die Grundmauern eingerissen, um sein ätzendes Geisterfeuer einzudämmen.

Kjall wusste, dass düstere Drachenkultisten und geheimnisvolle Greifenanbeter der Grehon gelegentlich gemeinsam dunkle Messen in diesem geschichtsträchtigen Gemäuer abgehalten hatten. Genau wie jetzt. Eine Ansammlung an Gestalten in Dunklen Kapuzen hatte sich auf der Turmspitze versammelt, unterhalb eines gewaltigen ... war das ein Skelett? Befand sich ein riesiges Skelett auf der Turmspitze? Was auch immer diese Kultisten hier taten, Kjall hätte gerne so wenig wie möglich damit zu tun. Hoffentlich sah die fremde Entität dies ebenfalls so. Jeder konnte doch verspüren, dass unter dem Dunklen Tempel eine finstere Macht lauerte.

Doch das Böse in seinem Körper stolperte langsam näher, auf den überwachsenen Eingang des Dunklen Tempels hinzu, als wäre es magisch angezogen davon.

Und dann, als es nur noch einige Schritte davon entfernt war, geschah etwas Unerwartetes. Der gruselige Gesang der Kultisten wandelte sich in ekstatisches Geschreie. Ein dumpfes Dröhnen drang aus dem felsigen Boden. Hitze brandete auf. Und schwarz-silbernes Geisterfeuer brach an unregelmäßigen Stellen aus dem Boden rund um den Dunklen Tempel herum hervor und stieb als dunkel schimmernde Stichflammen in den Nachthimmel.

Kjall wusste nur von einem Artefakt, das schwarz-silbernes Feuer hervorrief, in welchem selbst Drachenknochen verkohlten. Nur die wenigsten kannten seine Geschichte, doch er hatte sie von seinem Vater Xoll schon oft erzählt bekommen. Dies war ... unerwartet. Doch erfreulich. Vielleicht musste Kjall sich gar nicht mit diesem Lackaffen Harthalt streiten, um den Feuerschild zu erringen. Voller Staunen und Ehrfurcht betrachtete Kjall das tödliche Feuerspiel vor ihm.

Und selbst die dunkle Macht, die seinen Körper kontrollierte, schien für einen Augenblick von einem ehrfürchtigen Gefühl der Bestimmung befallen zu sein. Seltsame Eindrücke schwammen durch Kjalls Geist, Bilder und Eindrücke, die er nicht einordnen konnte. Kristallsplitter sanken in einen See voller umherwirbelnder Farben, wie er sie noch nie zuvor gesehen hatte. Blutrote Flammen schmolzen wabernde Wolken von einem sternenübersäten Himmel. Ein gewaltiger Schmiedehammer presste einen krummen Turm in ein golden glühendes Feld.

„Es ... es ist wunderschön“, krächzte Kjalls Stimme. Für den flüchtigsten Augenblick verlor das fremde Ding den Halt über ihn. Auf einmal, wider allen Erwartens, gelang es Kjall, seinen Arm zum Zucken anzuregen. Er zwang seine Finger, sich zu öffnen und den roten Kristall loszulassen.

Klirrend fiel der Edelstein zu Boden. Kjall stolperte zitternd zurück, tastete sich fahrig ab, horchte in sich hinein. Keine Spur mehr des Bösen. Was auch immer in diesem Kristall gesteckt hatte, es hatte ihn nicht mehr in seiner Gewalt.

Das schwarz-silberne Feuer erlosch. Der Dunkle Tempel lag einen Herzschlag lang ruhig da, dann ertönte Jubeln und Klatschen von seiner Spitze, als die Kultisten einen nächsten Singsang anstimmten.

Kurz überlegte Kjall, sich dem roten Kristall wieder zu nähern. Dann jedoch schüttelte er seinen Kopf, ließ den Kristall Kristall sein und raste davon, bis er an einem stillen Örtchen ein Zeitportal in die Zukunft öffnen konnte. Was auch immer dies für eine dunkle Entität gewesen war, er hoffte, dass sie in der Vergangenheit ein baldiges Ende finden konnte. Ärger anrichten konnte dieses Ding in dieser Zeitlinie nicht mehr, das wäre ihn doch sicherlich schon aufgefallen.

Nein, er selbst hatte sich wichtigeren Dingen zuzuwenden.

Der Feuerschild wartete auf ihn.











\begin{chapterbox}
    \chapter{Ein König auf Abwegen (2021)}
    \label{Ein König auf Abwegen (2021)}
    \az{Jahr 74}
    Die Rietgraskrone und die durch sie verliehene Verantwortung lasten schwer auf König Thorald. Sein Vertrauen in seinen manipulativen Ratgeber Ken Dorr schwindet. Seine geliebte Bäuerin hat er schon lange nicht mehr besucht. Zwei mächtige Schilde verstauben unter einer Klappe hinter dem Thron. Und hinter der Quelle des Likko versteckt sich ein langer Gang in ein Land voller Geächtete, Söldner, Sheriffs ... und einem fremden König.
\end{chapterbox}



\section{Thorald trifft eine Entscheidung}

\az{Jahr 74}

Die Nacht war bereits lange angebrochen und ihre finsteren Schatten schmiegten sich an die Mauern der Rietburg, als Thorald den Eingang erreichte. Er kniff die Augen zusammen, während er am Ewigen Feuer vorbeischlich, welches feuerrot in einer eisernen Schale vor dem großen Tor umhertollte. Er wollte nicht wahrhaben, dass kleine violette Flämmchen aufzischten, wenn er sich der Burg näherte. Er war doch keine Gefahr für das Königreich. Er war der König, bei Mutter Natur! Der König von Andor. Kein dahergelaufener Dieb.

„Man... man öffne die Tore! Ich bin heimgekehrt!“, entsprangen die krächzenden Worte den aufgerissenen Lippen des Thronerben. Die ganze Welt schwankte, als hätte man sie aus den Angeln gehoben. Thorald musste sich gegen das Tor stützen, um nicht dem feuchten Rietgras entgegenzustürzen. Schließlich gab er auf und sank daran herunter. Der Boden war matschig und kalt, aber es war so unendlich viel einfacher, sich der Erde zu ergeben, als sich gegen sie zu stemmen. Er hätte sich von der guten Gilda etwas Met für die Heimreise mitgeben lassen sollen, als Stärkung gegen die schändliche Schwerkraft. Aber so, wie er Gilda kannte, hätte sie ihm den Met glatt verweigert. Er lachte müde auf bei dem Gedanken. Sich dem König von Andor verweigern, eine einfache Tavernenwirtin!

„Wache! Ich erbitte um Einlass!“, lallte Thorald erneut, und endlich polterte etwas auf der anderen Seite der Burgmauer. Ein Gesicht blickt über die Brüstung. Armond, der Anführer der Rietgarde. Ausgerechnet heute musste Armond Nachtschicht haben, dieser pflichtbewusste Besserwisser.

„Mein König“, sprach Armond knapp, als er Thorald ins Innere der Burg führte. Er bot Thorald bereitwillig seinen Arm als Stütze an. Gleichzeitig bedachte er ihn mit einem Blick voller Abscheu und Mitleid, bei dem Thorald am liebsten auf der Stelle im Boden versunken wäre.

„Ich finde den Weg alleine“, blaffte Thorald den Gardisten an, „Ist ja nicht das erste Mal, dass ich... Ihr wisst schon.“

„Ja, das ist leider nicht das erste Mal“, gab Armond resigniert zurück. Er setzte bereits an, zu seinem Posten zurückzukehren, als er stutzte.

„Mein König, ist das Blut? Seid Ihr verletzt? Seid Ihr auf Kreaturen gestoßen? Ich sagte Euch doch, dass Ihr Euch von einer Leibgarde begleiten lassen sollt, wenn Ihr...“

„Schweigt stille, Ihr... Ihr La... La... Langeweiler. Ich wurde vom besten Schwertmeister ausgebildet, den dieses Königreich je ge... gesehen hat... da werde ich doch noch hoffentlich mit einigen ga... ga... garstigen Gors klarkommen können.“

Die Welt schwankte erneut und Thorald sackte auf ein Knie. Er wusste, dass eine Ausbildung bei Harthalt kein Garant dafür war, eine Begegnung mit Kreaturen unbeschadet zu überstehen – Harthalt selbst hatte sein Leben im Kampf gegen einen einzelnen Skral verloren, und war dabei noch nüchtern gewesen. Thorald wusste auch, dass es nicht geschickt war, sich ohne Begleitung zum Trinken zu schleichen. Aber in Begleitung von jemandem, der sich seines Königs schämte, wären seine Ausflüge noch viel unerträglicher, als sie es ohnehin schon waren. Und es war nicht so, als wäre das Land auf seine Anwesenheit im Thronsaal angewiesen, um am Laufen zu bleiben.

Dass er heute Nacht auf zwei grantige Gors gestoßen war, war bloß Pech gewesen. Pech, das ihn vielleicht sein Pferd gekostet hatte. Er sollte einen Boten aussenden, um das umliegende Rietland nach seiner schönen Stute abzusuchen. Mit etwas Glück war sie nur durchgegangen und unversehrt geblieben. Morgen, nahm er sich vor, morgen würde er das tun. Ach, wenn doch nur Thorn noch hier wäre. Der Bursche wusste, wie man mit Pferden umgeht. Und er hatte stets aufmunternde Worte für Thorald gehabt. Nun, zumindest so lange, bis Thorald ihn als nichts als einen dummen Bauernsohn mit einem Schwert bezeichnet hatte. Eine weitere Beziehung, die Thorald gekappt und es dann bereut hatte. Thorald schwelgte noch einige trunkene Augenblicke in Erinnerungen an Thorn, dann rappelte er sich ächzend auf.

„Mein König, wenn Ihr verletzt seid, so muss ich darauf bestehen, dass Ihr mich zum Heiler begleitet“, versuchte Armond erneut sein Glück.

„Lasst den alten Alduin doch schlafen“, befahl Thorald, „Mir geht es doch... so... so gut wie neugeboren.“

Armond schüttelte bloß den Kopf. Ein Kontrollblick verriet ihm allerdings, dass der König noch sämtliche Finger und Ohren besaß und dass die Flecken auf seinem Wams eher nach Wein als nach Blut aussahen. So ließ er ihn widerwillig ziehen.\bigskip







Thorald stapfte durch die Rietburg. Der Mond warf sein Licht auf die Abdrücke von Thoralds edlen Stiefeln im Matsch. Dreck spritzte an seine ohnehin schon besudelten Leinenhosen. Die Burg lag friedlich da. Die meisten Leute schliefen in Ruhe, während ihr König betrunken heimkehrte.

Eine Eule schrie in der Ferne, und Thorald zuckte zusammen. „Nur ein Eule, nur ein einfacher Flattervogel“, murmelte er zu sich selbst.

Langsames, rhythmisches Hämmern traf auf seine Trommelfelle. Hatte sich Warguth tatsächlich diese Tageszeit ausgewählt, um seine Schmiedekunst zu perfektionieren? Seitdem Wulfron die Schmiede an Warguth abgetreten hatten, war die Qualität der darin produzierten Schwerter laut Ken merklich gesunken – und Ken hatte ein Auge für so etwas. Das war aber noch lange kein Grund, bis mitten in die Nacht hinein verbissen weiterzuwerkeln. Thorald grummelte. Der Hall in seinen Ohren war alles andere als angenehm. Wie konnten die restlichen Bewohner der Rietburg auch nur ein Auge zutun dabei?

Tatsächlich, da war Warguth, tief in seine Arbeit versunken. Funken sprühten, während sein Schmiedehammer in regelmäßigen Intervallen auf den Amboss niederfuhr. Warguths schwarzer Hund lag neben ihm und hob den Kopf, als er Thorald vorbeistapfen sah. Ein tiefes Knurren entsprang seiner Kehle. Nicht einmal die Hunde mochten Thorald noch.

Nun fiel auch Warguth die Präsenz seines Königs auf und der junge Schmied hielt in seinem Werk inne. Er sah Thorald entgegen, wie dieser als elendes Häufchen den Trampelpfad entlangtorkelte. Kurz wirkte es, als wollte Warguth etwas sagen, doch dann wandte er sich bloß wieder seinem Schwert zu und hämmerte stirnrunzelnd weiter. Thorald empfand das Bedürfnis, etwas zu erwidern, doch die passenden Worte fielen ihm nicht ein und ganz abgesehen davon konnte er es kaum erwarten, in sein wohliges weiches Bett zu fallen und die Welt um ihn herum zu vergessen. Also zog er weiter.\bigskip







Zum Schlafen sollte Thorald so schnell nicht kommen, denn in seinen Gemächern erwartete ihn ein ganz und gar nicht gut gelaunter Ken Dorr. Ken hatte natürlich kein Auge zugetan und setzte sich mit verschränkten Armen auf, sobald Thorald durch die Tür stolperte.

„Thorald“, setzte Ken an, sein Tonfall von Missgunst und Unmut geradezu triefend. Thorald brachte eine wegwerfende Handbewegung zustande und brummelte etwas in seinen Bart hinein, während ihn noch mehr Schuldgefühle überkamen.

„Wir haben schon einige Male darüber geredet. Ich glaubte, mich klar ausgedrückt zu haben. Dein Verhalten ziemt sich nicht eines Königs. Alleine zum Umtrunk auszureiten und ohne Pferd und ohne Besinnung zurückstolpern zu kommen – was denkst du dir nur immer dabei?“

Thorald ignorierte Kens lauter werdende Stimme und schüttelte seine verdreckten Stiefel ab. Zudem wandte er seinem engsten Ratgeber den Rücken zu, damit dieser möglichst nicht die Blutflecken auf seinem Wams bemerkte.

„Währenddessen macht sich die ganze Rietburg Sorgen machen um dich, aber das hat dich noch nie gekümmert. Dir hätte wer-weiß-was alles zustoßen können! Was würde deine Mutter sagen? Und Brandur? Warum bist du nur so furchtbar unverbesserlich? Was muss ich tun, um zu dir durchzudringen?“

„Jetzt zünd‘ mal nicht gleich das ganze Rietgras an!“, fuhr Thorald zurück, „Allen anderen Andori ist es doch völlig egal, was mit mir geschieht. Meine Regen... Regent... Regentschaft zeichnet sich doch bloß dadurch aus, dass ich dumme Entscheidungen treffe, die sich schlecht auf das Königreich auswirken. Meinetwegen sind so viele Zwerge im Kampf um Cavern umgekommen. Die hassen uns jetzt noch mehr. Und es war ich, der um die Beachtung und Liebe der Bevölkerung buhlte und darum die Helden von Andor in den Norden sandte, um nicht in ihrem Schatten zu stehen. Wenn die nächste Trollhorde aus dem Gebirge hier aufkreuzt, sind wir ihr schutzlos ausgeliefert. Alle diese verlorenen andorischen Leben werden auf meine Kappe gehen! Selbst die Steu... Steuern waren keine gute Idee, die haben die Flussländler gegen uns aufgebracht. Ich wollte meinem Vater nacheifern und in die Legenden von Andor eingehen, aber inzwischen glaube ich... ich glaube, dass ich einfach nur ein Tor bin und nicht für dieses Amt geschaffen.“

Manchmal wünschte sich Thorald die Zeit zurück, in welcher er nur ein kleiner, naiver Junge gewesen war. Damals hatte er eine Selbstsicherheit besessen, die an Hochmut gegrenzt, ja, in manchen Momente diese Grenze gar überschritten hatte. Nichts hatte seine Selbstsicherheit damals trüben können. Natürlich hatte er nicht weniger Fehler gemacht als heute – er war es gewesen, dessen voreiliges Ausziehen der Rietgarde die Rietburg für ihre erste Eroberung schutzlos zurückgelassen hatte. Aber zumindest war er sich damals der Konsequenzen seiner Handlungen nicht bewusst gewesen. Und zumindest war er damals noch von fröhlichen Ratgebern und Lehrern umgeben gewesen. So vieles hatte sich verändert seit damals.

Thorald mühte sich vergeblich mit seinem verdreckten Wams ab, grunzte erschöpft und ließ dann endlich die Schwerkraft ihr Ding tun. So fiel er mehr neben Ken ins Bett, als dass er hineinstieg, und verteilte vergorenen Wein und vergossenes Gorblut darüber.

„Vielleicht sollte ich abdanken. Vielleicht kommt ja jemand und erhebt Anspruch auf den Thron. Irgendein entfernter Verwandter, der die Lage richten kann.“, murmelte er in seine Strohmatratze. Kurz fühlte er in seiner Brust den Stich der verfluchten Klinge des Schwarzen Ritters, mit der sein Onkel ihn einst hatte umbringen lassen wollen. Ob Hademar wohl Kinder hatte, die einen Anspruch auf den Thron Andors geltig machen und ihn aus diesem elenden Amt befreien konnten?

Thorald wusste nicht, ob Ken sein Gemurmel gehört hatte. Auf jeden Fall atmete Ken tief durch, seine Stirn glättete sich und er kratzte gedankenverloren an der Narbe an seiner rechten Wange herum. Von Gezeter und Standpauke keine Spur mehr.

„Mein König... du bist zu hart mit dir selbst“, sprach Ken fest und ließ sich neben Thorald nieder, „Unter unserer Führung ist das Königreich erblüht wie noch nie. Die Anstürme der Kreaturen haben endlich nachgelassen. Die Vorratskammern sind voll. Die diplomatischen Beziehungen mit Tulgor sind auf einem Höhepunkt. Das Volk liebt dich, Thorald. Du als König hast ihnen die Ära des Friedens gebracht, für den sie so lange kämpften. Du stärkst alle Personen in deiner Nähe. Du bist ein Held. Der Thron Andors ist sicher in deiner Hand.“

Thorald wälzte sich herum und blickte Ken an. Er wollte ihm widersprechen, wusste er doch sehr wohl, welchen Schaden seine voreiligen Entscheide immer wieder über das Reich gebracht hatten. Aber Kens Lobe fühlten sich stets so gut an. Vielleicht konnte er ja doch ein klein wenig Verantwortung für den momentanen Frieden übernehmen. Über acht Jahre davon. Das hatte es in der Geschichte Andors noch nie gegeben.

Aber nein, wenn er ehrlich mit sich war, dann waren die meisten guten Entscheidungen aus Thoralds Regentszeit Kens Kopf entsprungen. Ken war es, dem das Volk den Königsfrieden verdankte. Und Ken war noch nicht fertig mit ihm.

„Deine Probleme, Thorald, haben nicht mit deiner Regierung zu tun, sondern mit deiner Verantwortungslosigkeit. Der Druck der Rietgraskrone macht dir zu schaffen, natürlich, wem würde er nicht? Aber sich alleine aus der Burg zu schleichen wie ein ungezogener Bengel, das ist unter deiner Würde. Es muss doch bessere Möglichkeiten geben, mit deiner Frustration umzugehen. Wenn du diese nicht finden kannst, wirst du früher oder später daran zu Grunde gehen. Thorald, du musst stark bleiben. Wir stehen das durch.“

Thorald seufzte. Rekas Warnung schoss wieder einmal ihm durch den Kopf. Und so begann er, vor sich hin zu murmeln:

„Ken, du weißt, dass dein Rat mir von allen am teuersten ist. Ernsthaft. Wenn noch welche von Wulfrons Heldenbroschen übrig wären, würde ich dir höchstoffiziell eine verleihen für deine Dienste für das Königreich“

Kens Mund verzog sich zu einem breiten Lächeln. Thorald brauchte mehrere Anläufe, um die bösen Worte hervorzustoßen, die ihm auf dem Herzen lagen.

„Aber ich komme nicht um... umhin, zu befürchten, dass deine Worte leer sind. Dass du mich nicht um m... m... meiner selbst willen willst. Du hast doch bloß Angst davor, dass, wenn ich abdanke, ein anderer an meiner Stelle tritt. Einer, der von selbst klarkommt mit all den Schwierigkeiten des Königsamtes. Einer, der sich nicht so leicht kontrollieren lässt wie mich.“

Ken lächelte nun nicht mehr. Thorald ebensowenig.

„Ken... die alte Reka hat kürzlich mir eine Warnung überreicht. Dass mich einer manipuliere. Einer, der im Geheimen bereits die Rietgraskrone trage. Dass einer davon träume, an meiner Stelle auf dem Thron zu sitzen. Dies... dies kann nur auf dich zutreffen. Ich mag kein weiser König sein, aber ich weiß zumindest so viel: Du sollst... du sollst mich nicht kontrollieren. Und das kannst du nicht. Nicht mehr länger. Ich bin ein freier Mensch und dein Rat... dein Rat ist nur so viel... ein Rat halt. Lass mich dir zur Abwechslung einen Rat geben: Es geht dich nichts an, wenn ich mich in der Nacht einen Umtrunk gönne. Und es gehört dir nicht, mir zu widersprechen, wenn... wenn ich etwas... wenn ich die Krone zum Wohle des Reiches abtrete... ooo... Ken, mir ist kotzübel.“

Ken stand brüsk auf und starrte auf Thorald nieder, welcher immer noch in seinem verdreckten Wams dalag und vor sich hin murmelte. Kurz verharrte er regungslos, doch dann ließ er sich auf ein Knie fallen und flüsterte: „Mein König! Wie kannst du auch nur daran denken, dass es mich nach deiner Krone verlangte? Ich sehe doch, wie sehr du an diesem Los zu leiden hast. Diese Reka... wie alt ist sie nun schon? Einhundert Jahre? Mehr? Sie hat nicht mehr alle Tassen im Schrank, das ist es. Sieht hinter allem Intrigen und will vermutlich ihre eigene politische Macht mehren. Du kannst ihr nicht vertrauen. Mir schon. Deinem treuen Ken Dorr. Komm schon, Thorald, lass nicht zu, dass deine Zweifel dich übermannen.“

Thorald schüttelte bloß seinen Kopf und schluchzte erstickt etwas in sein Kissen hinein, von dem er nicht einmal selbst wusste, was es bedeuten sollte. Die Übelkeit in seinem Innern nahm stetig zu.

„Das reicht jetzt, Thorald“, sprach Ken mit einem bedeutungsschwangeren Unterton in der Stimme. „Du hast dich lange genug geplagt. Komm mit.“

Ken packte Thorald beim Unterarm und zog den protestierenden König aus seinen Gemächern heraus.

„Bei Mutter Natur, habe ich nicht... nicht soeben gesagt, dass du mich nicht kontrollieren sollst“, murrte Thorald.

„Ja, das hast du“, räumte Ken ein. Mit einem schiefen Grinsen fügte er dann hinzu: „Aber ganz ehrlich, willst du lieber hier drinnen reihern?“\bigskip







Ken führte den König aus Brandurs Turm hinaus, mitten auf die Brücke zwischen dem Turm und dem Thronsaal.

Thorald kniete nieder und blickte über die Brüstung, sein Magen rumorend. Dann spuckte er. Ken stand neben ihm, die Hand auf die Schulter, aber abgewandt. Da kauerte er, der König neben seinem Ratgeber, sich über die Brüstung gebeugt von einem der höchsten Punkte des Reichs in die Tiefe übergebend. Der Mond stand immer noch hoch am Himmel. Tief unter den beiden lag das Dorf der Rietburg still da. Auch das rhythmische Hämmern in Warguths Schmiede hatte aufgehört.

„Besser raus als rein“, warf Ken hilfreich ein, „Bist du endlich fertig?“

Thorald stöhnte schwach. Ken klopfte ihm ungeduldig auf die Schulter und zog ihn dann in die Höhe.

Seine Worte waren hart, doch seine Stimme war urplötzlich wieder weich geworden: „Ich kann dieses jämmerliche Getue nicht länger mit ansehen. Ich weiß, was du brauchst. Komm mit. Na, komm schon, hat dich mein Rat je in Schwierigkeiten gebracht?“

Ken schleifte den Thronerben mehr von der Brücke, als dass er ihn führte. Mit einem eleganten Schlüssel öffnete Ken die Tür zum Thronsaal und führte Thorald hinein. Es war stockfinster, aber Ken schien sich gut auszukennen, denn er verschwand schnurstracks im Dunkeln. Thorald ließ erneut die Schwerkraft auf sich wirken und sank an einer Säule neben dem Eingang zu Boden. Es spuckte aus, aber der säuerliche Geschmack in seinem Mund wurde nicht weniger. Im Schatten fiel es wenigstens nicht auf, wie sehr die Welt wackelte.

Ein Knirschen und Knarren weiter hinten im Thronsaal ließ Thorald aufhorchen.

Bläuliches Licht flammte auf und warf den Schatten einer großen schwarzen Silhouette bis vor seine Füße. Thorald brauchte einige Momente, um zu verstehen, was er sah: Ken hatte die versteckte Klappe hinter dem Thron geöffnet und wuchtete nun den strahlend hell scheinenden Sternenschild dahinter hervor.

„Jetzt komm schon her“, ächzte der Kahlköpfige, „Der Schild scheint sogar zu spüren, dass du ihn jetzt brauchst.“

Thorald schüttelte bloß den Kopf: „N... Nein, Ken, das mache ich nicht noch einmal durch. Mehrere Monate habe ich jetzt ohne den Sternenschild durchgehalten... er kann regelrecht süchtig machen.“

„Hör für einmal auf, dich zu sorgen, und greife nach der Hoffnung, du tumber Troll! Du willst sie doch!“

Thorald schüttelte weiterhin den Kopf, doch das Licht des Sternenschilds zog ihn schon beinahe magisch an. Ehe er sich versah, stand er neben der verborgenen Klappe und blickte auf den Bruderschild hinunter, in dessen verdreckter Verzierung sich der mächtige Schild der Hoffnung widerspiegelte. Und was für ein Licht dieser aussandte! Thorald musste seine Augen zukneifen, um nicht zu erblinden. Fast wie im Traum streckte er seine Hände aus und berührte das sternförmige Emblem des Schilds.

Thorald hatte einen schwachen Anstieg von Stolz in seiner Brust erwartet, wie es die letzten Male gewesen war, bei denen er sich auf den Schild gestürzt hatte. Doch dem war nicht so. Kaum hatten Thoralds Fingerspitzen die Schildoberfläche berührt, glühte der Schild noch heller auf und Thorald fühlte, wie ein riesiges Gewicht von seinen Schultern genommen wurde.

Es war wieder wie beim ersten Mal, als er vorsichtig den Sternenschild berührt hatte. Seine Kopfschmerzen und seine Übelkeit waren immer noch da, aber völlig in den Hintergrund getreten. Sie waren nicht wichtig. Nichts war wichtig außer diesem einen Moment, in dem Thorald den Sternenschild hielt und mit zusammengekniffenen Augen in dessen grell leuchtende Oberfläche starrte. Ein leises Summen war zu hören, mit einer wunderschönen Melodie, welche selbst Grenolins klare, reine Stimme übertraf. Thorald glaubte, im Schild Schemen zu erkennen. Oder wurden diese Bilder vor seinem inneren Auge abgespielt?

Er sah eine Masse von Menschen, welche sich wie ein riesiger brauner Wurm über das Rietland auf die Rietburg zubewegten. Freudenschreie und Marschlieder drangen schwach an sein Ohr. Und angeführt wurde die Prozession von zwei in grüne und blaue Gewänder gekleideten Gestalten, die gemeinsam auf einem edlen weißen Pferd ritten.

Je mehr sich Thorald auf diese Personen konzentrierte, desto klarer wurde das Bild. Schwarzes Haar, ein Bogen an ihrer Seite, und war das etwa ein schwarzer Hund, welcher neben dem Pferd her trottete? Ganz eindeutig, das war Chada, die Heldin von Andor. Zeigte der Sternenschild ihm die Zukunft? Würden die Helden von Andor nach Andor zurückkehren? Aber warum aus dem Süden? Und war das etwa... ja, das war unverwechselbar die goldene Rietgraskrone, welche auf dem Kopf Chadas thronte!

Natürlich! Thorald hatte so lange versucht, aus dem Schatten der Helden von Andor zu treten und Brandurs Fußstapfen auszufüllen, dass er ihr Talent, mit Schwierigkeiten umzugehen, ganz vergessen hatte. An sie könnte er das Königsamt Andors guten Gewissens abtreten. Dann wäre dieser Albtraum endlich vorüber.

Jawohl, dass Chada in dieser Vision die Rietgraskrone trug, war ein eindeutiges Zeichen. Thorald würde sein Amt abtreten und die Krone Andors an die Fürsten Andors abgeben. Chada würde eine würdige Nachfolgerin für den Thron sein. Trotzdem war es bemerkenswert, dass Chada und nicht etwa der gute Thorn, den Thorald schon viel länger kannte und mochte, die Krone trug. Thorald dachte zurück an die Zeit, als er als junger Erwachsener Chada zum ersten Mal getroffen hatte. Damals hatte er schnurstracks davon geträumt, sie eines Tages zu heiraten. Unrealistische Prinzenträume eines selbstverliebten Angebers und Schürzenjägers, natürlich, aber konnte es sein... dass vielleicht eine klitzekleine Chance bestand?

Das Bild verblasste und das leise Summen des Schilds ließ gemeinsam mit dem Glühen nach. Der Sternenschild hatte fürs erste seine Kraft verbraucht. Aber er hatte seine Wirkung getan. Die Hoffnung, die Thorald verspürt hatte, ließ nicht nach.

Die Worte Rekas hallten in seinem Kopf nach: „Du musst vorsichtig sein, Thorald. Ich habe es im Traum erblickt. Eine dunkle Gestalt dürstet nach deiner Thron, und sie wird dir eher Schaden zufügen, als dass sie die Rietgraskrone auf einem anderen Kopf sieht.“

Damals hatte Thorald Rekas Worte als abergläubiges Geschwätz abgetan, aber heute, hier und jetzt, in einer für ihn so unüblichen Klarheit, die der Sternenschild ihm geschenkt hatte, war er sich nicht mehr so sicher. Jetzt, wo er so darüber nachdachte, entstammten nicht nur viele seiner guten Entscheidungen Kens Rat, sondern auch seine schlechteren. Die Helden von Andor hatte er auf Kens Rat in den Norden geschickt. Hatte Ken Thoralds Ansehen zu mehren versucht oder seine stärksten Beschützer außer Reichweite bringen wollen? Und wie oft hatte Ken ihn bereits zu einem Umtrunk angeregt, nur um ihm danach eine Standpauke über dieses Laster zu halten? Wie oft hatte er ihm bereits den Sternenschild überreicht und in den siebten Himmel gelobt? Thorald erschauerte. Sein Selbstbild war wie Wachs in Kens Händen. Seine Entscheidungen auch. Das konnte nicht gut sein. Warum erkannte er das erst jetzt?!

Als das Leuchten des Sternenschilds vollständig erlosch, blieb der Thronsaal kurzzeitig in beinahe vollständige Dunkelheit getaucht und nur das Nachbild des leuchtenden Schildes war erkennbar. Dann flammte wie aus dem Nichts eine Flamme auf und beleuchtete die tiefen Furchen im Gesicht Kens und die Narbe an seiner rechten Wange. Ken hängte die soeben angezündete Fackel in eine Halterung an einer Säule, schnappte Thorald den inzwischen erkalteten Sternenschild aus den Händen und bugsierte ihn vorsichtig in die versteckte Klappe hinter dem Thron.

„Flausen ausgetrieben?“, sprach er Thorald an, den Rücken immer noch zugewendet, „Siehst du nun wieder, dass der Thron Andors dir zusteht und dass noch Hoffnung für das Reich besteht? Dass eine Abdankung nicht nötig ist?“

Thorald nickte schicksalsergeben und Ken klopfte ihm aufmunternd auf die Schulter. Thorald zuckte vor seiner Berührung zusammen.

„Lass...“, krächzte Thorald, „kannst du... ich glaube, ich brauche etwas Zeit für mich selbst.“

Kens Augenbrauen zogen sich zusammen, aber er nickte: „Natürlich. Es ist auch wahrlich an der Zeit, dass ich mir endlich eine Mütze Schlaf gönnen kann.“

Er verbeugte sich und huschte geschwind aus dem Raum. Thorald blieb alleine zurück. Alleine mit seinen Gedanken.

Die Vision des Sternenschilds zeigte ihm deutlich, dass Andor eine gute Zukunft bevorstehen konnte, auch oder gerade ohne Thoralds Führung. Die Krone Andors musste nicht für immer auf seinem Haupt bleiben. Er würde sie an einen würdigeren Nachfolger abtreten, sobald die Helden von Andor aus dem Norden zurückkommen würden. Allzu lange konnte es ja nicht mehr dauern. Er war erst der zweite König dieses jungen Landes, niemand konnte ihm vorschreiben, dass er das Königsamt nicht übergeben könnte. Und dann würde er keinen Schaden mehr anrichten können. Müsste keine langen Sitzungen mit den Anführern der Armee und den Vertretern des Volkes haben. Hätte einfach nur Zeit für sich selbst. Würde er arbeiten müssen? Vielleicht könnte er bei der Ausbildung der Krieger helfen. Es würde bestimmt ein Plätzchen für ihn geben.

Nur sollte er vielleicht lieber vorsichtig sein und Ken nicht darauf aufmerksam machen, was er vorhatte. Ein Trunkener Tor war er, warum nur hatte er Ken von Rekas Warnung erzählt?

Thorald wankte die Säulen des Thronsaals entlang. Da, an der Wand, hing ein Gemälde seiner Eltern. König und Königin, Hand in Hand. Er war aus Krahd geflohen und sie stammte aus den Flusslanden. Sie war friedlich die Narne hinuntergegangen, als Thorald noch nicht einmal ein Schwert hatte führen können. Brandurs Tod war ihm hingegen traumatisch tief in seine Erinnerungen gebrannt. Thorald spürte erneut die Panik, die er ihn damals ergriffen hatte. Späher des König hatten berichtet, wie sich in den Ausläufern des Grauen Gebirges ein Drache aus der Knochengrube erhob. Es war erst Stunden her gewesen, dass sein Vater gefallen war. Der Kampf zur Befreiung der Rietburg war noch vollends im Gange gewesen. Damals hatte er gedacht, dass Tarok sein Ende sein würde. Nur der tapfere Einsatz der todesmutigen Helden hatte ihn gerettet. Wäre es besser gewesen, wenn er damals umgekommen wäre? Wie hatten seine Eltern es bloß geschafft, ein Königreich zu führen, ohne dabei den Verstand zu verlieren? So viele unterschiedliche Wünsche anzuhören und gerechte Entscheidungen zu treffen?

Als Thoralds Blick auf den jungen schwarzhaarigen Burschen auf dem Gemälde fiel, welcher zwischen seinen beiden Eltern stand und stolz in die Welt hinausschaute, da kamen sie endlich, die Tränen. Und wie sie kamen. Als wäre ein riesiger Staudamm gebrochen, so wie Thoralds naiver Stolz schon lange gebrochen war. Der rote Mantel des Königs, den er mit Wein, Met und Gorblut besudelt hatte, lag schwer auf ihm. Brandurs Fußstapfen waren zu groß für ihn gewesen, nicht als ein blasser Schatten seines Vaters war er. Und die Stärke seiner Regierung war nun abhängig von einem Berater, welcher ihm vielleicht lieber die Krone mit Gewalt abnehmen würde, als seinen allumfassenden Einfluss über ihn schwinden zu sehen. Es war zum Verzweifeln.

Thorald stolperte zur Schatulle, in der die Rietgraskrone für zeremonielle Anlässe verstaut war. Er klappte den Deckel hoch und betrachtete die goldene Krone, deren dünne Zacken im Fackellicht umherzuwiegen schienen wie das Rietgras im Wind.

Ein Bild erschien vor seinem inneren Auge: Sein Vater, wie er mit der Krone schief auf dem Kopf und einem Wanderstock in der Hand durch das Rietgras strich. Im Jahre seines Todes hatte Brandur noch versucht, die Wunden der Vergangenheit zu heilen und Frieden mit Fürst Hallgard von den Schildzwergen zu schließen. Ein Frieden, der jahrzehntlang unmöglich erschienen war, weil Brandur den Schildzwergen den Sternenschild einst vorenthalten und dann diesen an den Verräter Rudnar verloren hatte. Kram hatte sich nach Brandurs Eskorte zur Mine dafür eingesetzt, dass der Bruderschild bei den Helden von Andor bleiben konnte, statt in einer Schatzkammer der Schildzwerge zu verstauben. Und nun verstaubten diese beiden mächtigen Schilde in einer Klappe hinter dem hölzernen Thron Andors. Weil Thorald ihren Einsatz verwehrt hatte, als Kram beim Kampf um Cavern danach verlangt hatte. Aus Furcht davor, betrogen zu werden, und weil er den Sternenschild in seinen Zeiten der Hoffnungslosigkeit für sich selbst gebraucht hatte.

Aufgrund dieser Entscheidung waren vermutlich viele Zwerge gefallen. Mit dem Bruderschild hätte man Kram aus weiter Ferne Stärke schenken können, bis die restlichen Helden in Cavern eingetroffen wären. Vielleicht hätte man mit Sternenschild gar Hallgards Tod abwenden können. Nicht, dass Thorald per se viel an den goldgierigen Zwergen läge. Aber sie waren praktische Handelspartner und Thorald hatte es wieder einmal geschafft, mit seinen Entscheidungen die Beziehungen und somit das Schicksal Andors zum Schlechteren zu neigen. Er fluchte und wischte sich das tränennasse Gesicht ab.

Kram war inzwischen nicht nur ein Fürst von Andor, sondern auch der Fürst von Cavern geworden, das Oberhaupt aller Schildzwerge. Aber wenn Thorald seinen Informanten vertrauen konnte, so hatte Krams Ernennung nicht die Anerkennung aller Schildzwerge genossen. Eine Splittergruppe, die sich selbst die „Wahren Schildzwerge“ nannten, hatte schon einmal versucht, ihn zu stürzen, und wer vermochte zu sagen, ob sie es nicht ein zweites Mal versuchen würden? Ihr Anführer war Gerüchten zufolge der stärkste Zwerg, den die Minen je gesehen hatten, und er hasste Menschen aus ganzer Seele. Zumindest hatten Kens Kontakte das behauptet.

Thorald fühlte erneut den Stich des Schwarzen Ritters unter seinem Herzen. Kram war es gewesen, welche die lebensrettenden Zauberhutpilze für Thorald beschafft hatte. Und Kram war es gewesen, der sich mit einem Schild todesmutig zwischen einen Feuerstrahl Taroks und den am Boden liegenden Prinzen geworfen hatte. Kram hatte ihm gleich mehrere Male das Leben gerettet. Thorald schuldete ihm so vieles. Hatte er da nicht eine gewisse Verpflichtung, Kram zu unterstützen? Auch wenn viele Zwerge sich nicht um das Schicksal der Menschen zu kümmern schienen, so war Kram derjenige, der es doch tat. Das Reich konnte nur davon profitieren, wenn Thorald Kram als Fürsten Caverns unterstützte.

Thorald richtete sich abrupt auf und setzte die Rietgraskrone auf seinen Kopf. Ihm war soeben eine geniale Idee gekommen. Brandur hatte es schon einmal geschafft, mit einem Marsch zur Mine einen ihm abgeneigten Fürsten der Schildzwerge friedlich zu stimmen. Thorald konnte es ihm nachtun. Er hatte bereits mehr als bewiesen, sich selbst verteidigen zu können. Und im Gegensatz zu Brandur damals besaß Thorald jetzt sogar etwas, das die Schildzwerge von ganzem Herzen wollten.

Sein Blick schweifte zurück zur geheimen Klappe hinter dem Thorn. War es das wirklich wert, die beiden mächtigen Schilde aufzugeben?

Ja, beschloss Thorald, das war es. Die Beziehung Andors zu Cavern brachte für beide Reiche nur Vorteile, und Kram brauchte Symbole wie die mächtigen Schilde, um die Anerkennung im gesamten Zwergenvolk zu erhalten. Ganz nebenbei war es ein Akt der Rebellion gegen Ken. Ken hatte immer darauf beharrt, dass die mächtigen Schilde in der Rietburg bleiben sollten. Thorald würde ihm hierdurch zeigen, dass immer noch er der König dieses Königreichs war.

So viele Entscheidungen Thoralds hatten schlechte Nachwirkungen nach sich gezogen. Dies war seine Chance, dem Reich, seinem Vater und sich selbst zu beweisen, dass er nicht bloß ein Taugenichts war. Er war der König von Andor, und er würde noch in dieser Nacht nach Cavern aufbrechen. Mit den beiden mächtigen Schilden im Gepäck und der Rietgraskrone auf dem Kopf. Alleine, wie sein Vater vor ihm. Ein Friedensangebot und ein Aufstand gegen Kens Kontrolle.

Die mächtigen Schilde würden ein Geschenk als Wiedergutmachung der Beziehung ihrer beiden Völker sein. Kram würde sie mit Freuden annehmen. Die restlichen Helden würden bald darauf aus dem Norden zurückkehren, auf dass man ihnen die Rietgraskrone überlassen könnte.

Alles würde gut kommen.








\newpage
\section{Thorald kommt vom Weg ab}


Thorald schreckte aus unruhigen Träumen hoch. Das war eigenartig. Eigentlich hatte er schon seit Jahren keine Albträume mehr gehabt. Nicht, seitdem diese gruselige, schattenhafte Gestalt mit den weißen Augen, dem langen weißen Haar und den zwei langen Schwertern ihn endlich nicht mehr Nacht um Nacht im Traume verfolgt und gefragt hatte, wer der Erbe seines Vaters Willens sei. Und diese Gestalt war Thorald inzwischen schon seit über einem Jahr nicht mehr erschienen.

Thoralds unruhige Träume waren nicht das einzige Eigenartige an seinem Zustand. So bemerkte Thorald rasch, dass er sich nicht in seinem gemütlichen Bett in der Rietburg befand, sondern in einer von Morgentau feuchten Böschung am Ufer der Narne lag. Seine Füße waren so kalt, weil sie bereits zum Teil vom Narnenwasser überflutet wurden.

Rasch richtete sich Thorald auf. Die Erinnerungen von letzter Nacht brachen über ihn herein ein und dröhnten seinen Kopf mit Schmerzen zu. Ein Blick auf seine Umgebung bestätigte es: Er war tatsächlich mitten in der Nacht nach Cavern aufgebrochen. Der Bruderschild und der Sternenschild lagen in einem durchnässten Sack an der Böschung, einige Schritte von ihm entfernt. Er hatte es offenbar irgendwie bis über die Markbrücke geschafft, ehe er hier zusammengebrochen war.

Der König räkelte sich und versuchte anhand des Sonnenstands zu erkennen, wie spät es war. Keine Chance. Zu seiner Linken sah er die mächtige Zwergeneiche hoch in den Himmel ragen, zu seiner Rechten erkannte er eine blau gewandete Frau, welche am anderen Flussufer saß und ihn interessiert zu mustern schien. Thorald langte sich instinktiv an den Kopf, und zog scharf Luft ein. Die Rietgraskrone saß nicht mehr da. Er musste sie gestern verloren haben. Ach, warum war er nur zu dieser wahnwitzigen Tat aufgebrochen? Ein einfacher Bote oder sogar ein Falke mit einer Einladung zu Schildübergabverhandlungen hätte doch gereicht.

Aber gut, jetzt, wo er schon einmal hier war, konnte er die Mission auch rietgraskronenfrei zu Ende führen. Thorald kämpfte sich die matschige Böschung hoch und spürte die warme Sonne auf seinem Gesicht. Von hier aus waren es nur noch wenige Stunden bis zum südlichen Eingang Caverns und der Weg war gut ausgebaut.

Es wäre so einfach, die beiden Schilde jetzt Kram abzuliefern.

So einfach.

Aber nun, ein wenig nüchterner, war sich Thorald plötzlich nicht mehr sicher, ob er sie wirklich abgeben wollte.\bigskip







Thorald wandte sich vom Weg nach Cavern ab und folgte dem geschlängelten Flusslauf der Narne gen Norden. In einem Büschel Gras am Ufer fand er einen hübschen grünen Runenstein, der in Thoralds Faust leise vor sich hin summte. Thorald steckte ihn als gutes Omen ein und machte sich auf den weiteren Weg.

Ein Trampelpfad führte von hier aus über die Bogenbrücke in den Wachsamen Wald, doch war dieser nicht Thoralds Ziel. Thorald selbst war sich seines Ziels nicht ganz bewusst, ehe er in der Ferne eine ärmliche Bauernkate am Fuße des Grauen Gebirges erkannte.

Elgas Hof.

Thorald schund sich innerlich. Warum hatte er sich nicht bei ihr gemeldet? Wie viel Zeit war seit ihrem letzten Treffen vergangen? Thorald hatte Elga seit ihrer Heirat immer seltener besucht, und doch erinnerte er sich nun, als wäre es erst gestern gewesen, an eine Zeit zurück, in der er so viel Zeit wie nur möglich bei ihr verbracht hatte. Elga hatte ein hartes Leben gehabt, alleine ihren Bauernhof bestellend, und sie hatte in Thorald nie bloß einen verwöhnten reichen Prinzen gesehen. Ihre gemeinsame Zeit war viel mehr als nur ein Techtelmechtel gewesen, sondern sehr wertvoll für Thorald als Person. Als er nach Brandurs Tod mit der Krone gehadert hatte, war Elga es gewesen, die ihn davon überzeugt hatte, das Amt zu übernehmen. Als die Frau des alten Erwan von gegenüber verstorben war, hatte Thorald eine prunkvolle Beerdigung veranstalten lassen, wie es das östliche Rietland noch nie gesehen hatte. Dann aber hatten die Pflichten eines Königs Thorald mehr und mehr in die Rietburg beordert. Ken hatte mehr und mehr seiner Zeit in Anspruch genommen. Elga hatte ihre eigene Familie gegründet. Und so waren sie auseinandergetrieben worden. Wie es ihr wohl ergangen war in all dieser Zeit?

Kurz kicherte Thorald darüber, wie absurd die Situation war. Der König von Andor klopfte mit Kopfschmerzen bei einer einfachen Bäuerin an der Tür, mit den zwei mächtigsten Artefakten des Landes bei sich und höchstwahrscheinlich bald einem verärgerten Suchtrupp der Rietgarde auf den Fersen. So nahm er seinen Mut zusammen und pochte an die verfallene Holztür.

Es dauerte nicht lange, bis die Tür aufflog und Thorald in zwei blitzende braune Augen starrte, die definitiv nicht zu Elga gehörten. Die Frau blickte kurz überrascht drein, als sie Thorald erkannte, fasste sich allerdings ebenso rasch und deutete dann einen Knicks an: „Euer Hoheit!“

Thorald fragte sich, wie viel Ironie hinter der Höflichkeitsgeste steckte.

„Juna! Es ist schön, dich... ich... ist Elga hier?“, fragte Thorald angespannt.

Juna legte ihren Kopf schief und meinte fröhlich: „Du wirst lachen, aber ich bin mir nicht einmal sicher.“

Sie drehte sich um und rief laut ins Innere der Kate: „Elgalein! Bist du drinnen? Dein König ist hier!“

Es blieb einen Moment lang still. Juna wandte sich bereits wieder Thorald zu und setzte zu einer Absage an, als doch noch Schritte ertönten und die Tür zur Gänze aufgerissen wurde. Da stand Elga und lächelte Thorald breit an. Thorald versuchte, ein Lächeln aufzusetzen, aber es wollte ihm nicht vollends gelingen.

Elga winkte ihn dennoch hinein.\bigskip







So saßen sie bald beide an Elgas altem Holztisch, an dem sie schon so oft gesessen hatten. Thorald hatte die mächtigen Schilde und sein Schwert zur Seite gelegt, konnte seine Hände nun nicht still halten und kratzte an der rauen Tischplatte herum. Elga stand still da und musterte ihn aufmerksam, aber ruhig. Juna tigerte hinter Elga umher und blickte immer wieder verstohlen zu Thorald hinüber.

Es war Elga, die das Wort ergriff.

„Ach, Thorald, du hast schon besser ausgesehen. Was ist mit dir geschehen?“

Thorald fragte sich, ob diese Worte sich auf seinen jetzigen verdreckten Zustand bezogen und die zwei Schilde im großen Sack neben seinem Holzstuhl, oder darauf, wie er Elga mit der Zeit weniger und weniger besucht hatte. Er beschloss, sich erst einmal um ersteres zu kümmern, und antwortete knapp.

„Ich war auf dem Weg nach Cavern. Ich hatte gestern betrunken die stolze Idee, die mächtigen Schilde ihren... ‚rechtmäßigen‘ Besitzern zurückzubringen.“

Elga nickte bloß. Juna hatte aufgehört, hin- und herzuwandern, und starrte Thorald nun direkt an. Thorald warf ihr einen knappen Blick zu und fuhr dann fort, versuchte, die passenden Worte zu finden.

„Die Wahrheit ist... ich... ich weiß nicht, was ich tun soll. Ich habe nicht viele gute Entscheidungen getroffen, und die meisten guten verdanke ich nur meinem Berater. Es steht meinetwegen nicht gut um die Zukunft des Landes und die Helden von Andor sind wegen mir weit weg. Ich dachte, dass das Volk mich dadurch mehr ehren würde, aber dieses wendet sich immer mehr von mir ab. Ich halte das nicht mehr aus.“

Er beugte sich vor und atmete tief durch. Elga flüsterte etwas zu Juna, welche sich umdrehte und langsam aus dem Raum lief, aber dabei immer wieder mürrisch nach hinten sah. Offenbar hätte es sie brennend interessiert, wie Thoralds Geschichte weiterging.

„Ich glaube... ich glaube, dass die Krone Andors auf einem anderen Haupt besser sitzen könnte als auf meinem Brummschädel“, brummte Thorald dann, „und das vergesse ich nicht einmal im größten Trinkgelage. Ich... ich will sie abtreten. Aber das kann ich erst, wenn die Helden zurückkehren. Und... ich beginne, mich vor Ken zu fürchten. Ich wüsste nicht, an wen ich mich wenden sollte, wenn er nicht an meiner Seite wäre und mich beraten würde, aber ich werde das Gefühl nicht los, dass er etwas... er hat eine finstere Seite, eine manipulierende, und diese kommt immer öfter zu Vorschein.“

Juna betrat den Raum erneut. Sie trug eine Platte mit Milchkrügen und Käsescheiben bei sich. Thorald spürte erst jetzt, wie trocken seine Kehle war und wie sehr sein Magen knurrte. Juna vollführte erneut einen kleinen Knicks und bot ihm die Fressplatte an. Thorald langte zu und bedankte sich überschwänglich.

Elga sprach, und ihre klangvolle Stimme beruhigte Thorald etwas: „Das Los der Krone hast du noch nie leicht getragen, auch wenn du das früher gerne geglaubt hast. Bei all den Trinkereien, in die du stürzt, bei dem makellosen Ruf deines Vorgängers, bei deinen Ambitionen und bei deinem listigen Berater, da ist es wahrlich kein Wunder, dass es dir schwer fällt, dich so um dein Land zu kümmern, wie du es dir gerne wünschtest... oder deine Beziehungen.“

Thorald blickte beschämt auf, doch konnte er in Elgas blitzenden Augen keine Wut erkennen, bloß einen Hauch von Traurigkeit. Noch etwas, was sie so stark von Ken unterschied.

„Es tut mir leid, dass ich mich nicht mehr bei euch gemeldet... Wie... wie geht es euch? Wie ist es euch ergangen?“

„Wir kommen klar“, meinte Juna mit einem aufrichtigen Lächeln, griff sich einen Batzen Käse und zog sich dann langsam wieder aus dem Raum zurück.

Elga bestätigte: „Wir kommen klar.“

Dann sanken ihre Mundwinkel: „Gerüchte von verschwundenen Bauern im östlichen Rietland und von wilden Wölfen an den Ufern der Narne gehen um. Der alte Erwan von nebenan ist gerade verstorben. Man munkelt, die Wölfe hätten ihn gerissen. Aber seine Kinder sind stark. Wir gucken immer wieder bei ihnen vorbei. Sie scheinen ebenfalls klarzukommen.“

„Das... das tut mir leid“, brachte Thorald hervor, „Ich hätte hier sein sollen. Kann ich... lässt sich eine Beerdigung...“

„So sehr uns das Angebot doch ehrt, so glaube ich nicht, dass eine weitere prunkvolle Beerdigung das ist, was seine Kinder sich wünschen. Ganz abgesehen davon hat euer Wolfskrieger schon eine schöne Zeremonie abgehalten.“

Thorald nickte überrascht. Orfen konnte er sich schlecht bei einem Beerdigungsritual an der Narne vorstellen – aber hinter dessen rauer Fassade versteckten sich offenbar so einige verborgenen Talente.

„Dann... braucht ihr... kann ich euch sonst irgendwie helfen? Ich könnte eine Patrouille der Rietgarde hier vorbeischicken, um sich diese Wölfe einmal anzuschauen.“

„Danke, aber ich glaube nicht, dass das nötig ist. Juna meistert den Umgang mit ihrem Stab immer besser. Bald wird sie soweit sein, die Wölfe friedlich zurückzutreiben – sollten diese sich überhaupt je einmal mit Hintergedanken auf unseren Hof wagen.“

Hatte Thorald wieder einmal versucht, mit Geschenken und Gaben sein schlechtes Gewissen zu besänftigen? Elga hatte ihn schon einmal auf diese Tendenz angesprochen, lang, lang war’s her.

Eine kurze Zeit lang blieb es still, während Thorald sinnierte. Dann fuhr Elga fort: „Was du tun könntest, ist, wieder öfters hier aufzukreuzen. Das Leben eines Königs besteht natürlich nicht aus sehr viel Freizeit, aber es würde uns freuen, dich wieder einmal hier begrüßen zu dürfen, statt dass du Gildas Met oder gar erneut dem Starkbier der Zwerge verfällst. Dann hättest du auch etwas Zeit weg von diesem Ken. Nach den wenigen Gerüchten, die uns von der Rietburg zufliegen, hat er dich ziemlich im Griff. Wie man sich doch in einem Menschen täusche kann. Ich mag mich noch gut an die Zeit erinnern, als du ihn kennenlerntest und stundenlang von ihm schwärmen konntest. Damals erschien er noch wie der freundliche junge Bursche, der gerade zum richtigen Zeitpunkt zu dir gefunden hatte. Er beriet dich gut und verschaffte dir kleine Siege, die dein Selbstvertrauen stärkten. Er schaffte es, dass dir die Fußstapfen deines Vaters nicht so riesig vorkamen, als dass du darin versinken müsstest.“

Thorald schluckte. An die schöne Zeit, als er Ken kennengelernt hatte, wollte er nicht zurückdenken. Zu kompliziert war sein Verhältnis mit Ken inzwischen geworden. So wandte er sich von diesem Thema und sprach stattdessen Elgas Vorschlag an, sich wieder häufiger bei ihnen blicken zu lassen: „Das... nein, ihr könnt mich nicht einfach wieder in eure Leben lassen... du sollst mir nicht einfach verzeihen. Ich habe schlimme Fehler begangen und dich und Juna im Stich gelassen. Ihr solltet wütend sein auf mich... alle anderen sind’s doch auch...“

Elga lächelte traurig: „Ach, Thorald... keiner leugnet, was für Schäden du in deinem Leben angerichtet haben magst. Aber du bist immer noch eine wertvolle Person mit einem guten Kern. Dich in ein Loch der Schuldgefühle reinzuwerfen, könnte vielleicht einigen Genugtuung verschaffen, wäre aber ziemlich sicher nicht hilfreich für dich oder das Königreich. Vielleicht bist du ja hier, um ein klein wenig Hilfe dafür zu erbitten, aus diesem Loch herauszuklettern? Wir helfen gerne, wo wir können.“

„Ich brauche mehr als nur ein klein wenig Hilfe“, murmelte Thorald, „Und ich verdiene sie ohnehin nicht. Ich will... ich will einfach nur die Krone loswerden. Nun, das hat mein betrunkenes Ich gestern Nacht tatsächlich geschafft. Die Rietgraskrone habe ich auf dem Weg hierher verloren. Aber das Amt abgeben steht noch an... ach Elga, ich weiß einfach nicht weiter.“

Elga stützte sich auf ihr Kinn und murmelte: „Hui, du steckst wirklich in einem unschönen Mentalität fest.“

Thorald klaubte weiter an der hölzernen Tischplatte herum und wich Elgas Blick aus, als diese fortfuhr: „Aber das ist in Ordnung. Jeder braucht Hilfe von Zeit zu Zeit, und dass du hierhergekommen ist, ist schon ein guter Anfang. Kleine Schritte führen zum Ziel, und du bist es wert, diesen Weg zu gehen. Zuerst einmal wollen wir doch herausfinden, was du wirklich mit den mächtigen Schilden anstellen willst, und ob dir Ken Dorrs Nähe mehr bringt oder schadet. Magst du hier bleiben für den Moment? Ich muss diesen Nachmittag noch etwas Saatgut streuen und natürlich die Kühe melken – da könntest du gerne aushelfen. Am Abend könnten wir uns dann die Zeit nehmen, deine Optionen zu besprechen. Schilde abgeben oder behalten. Krone abtreten oder akzeptieren. Ken Dorr bessere Umgangsformen beibringen oder von dir fernhalten. Wir hätten auch ein Gästebett für die Nacht.“

Thorald lehnte sich zurück. Solche Worte hatte er vermisst, ohne sich dessen überhaupt bewusst zu sein. Ein respektvolles Gespräch über seine Entscheidungsmöglichkeiten, was mehr konnte er sich wünschen. Er nickte langsam.

Da stürzte Juna durch die Tür hinein, ein blankes Schwert in der einen und ein knorriger Holzstab in der anderen Hand. Sie sprach aufgebracht: „Ich unterbreche euch zwei Turteltakuri ja nur äußerst ungern, aber da draußen marschiert eine ganze Schar von Pferden an. Ein Glatzkopf führt sie an. Den mögen wir nicht.“

Thorald sprang auf, während Elga ihren Stuhl zur Seite schob und sich sorgsam erhob. Schon war von draußen ein Wiehern zu hören. Schwere Stiefel klatschten auf den Dreck vor der Hütte und Schritte kamen immer näher.

Juna packte ihren Stab fester und flüsterte zu Elga: „Ich könnte es mit einem magischen Blitz versuchen. Gestern konnte ich ihn halbwegs kontrolliert feuern.“

Elga grinste schief und meinte: „Letzten Endes ist das doch ein lebendiger Mensch mit Wünschen und Träumen, mit dem wir es hier zu tun haben. Ein hochrangiger noch dazu. Ein magischer Blitz scheint mir als Begrüßung etwas... übertrieben? Was meinst du, Schatz?“

Juna nickte mit gesenktem Blick und legte ihren Stab enttäuscht zur Seite.

Thorald richtete sich seine Frisur.

Es klopfte.

Wie schon bei Thorald zuvor öffnete Juna die Tür. Thorald erkannt sofort Kens schnarrende Stimme: „Ist er hier? Richtet ihm aus, dass er etwas vergessen hat!“

Ein metallisches Klingen ertönte. Thorald vermutete, dass Ken soeben die Rietgraskrone zu Boden geschleudert hatte. Das konnte heiter werden. Ken war mal wieder in einer üblen Laune.

Die Tür zur Bauernkate wurde aufgerissen und Thorald erkannte die mächtige Silhouette seines engsten Ratgebers, der ins Dunkel der Hütte starrte und blinzelte. Seine Augen mussten sich wohl zunächst noch an das wenige Licht anpassen.

„Thorald? Bist du da?“, sprach er unwirsch.

Thorald räusperte sich und versuchte, so ehrwürdig wie möglich „Ja“ zu antworten. Kens Stirn furchte sich und seine Stimme sprang eine ganze Oktave in die Höhe, als er rief: „Wo hast du Trampeltier die mächtigen Schilde gelassen?! Sag bloß, dass du sie auch noch verloren hast!“

„Sie sind hier“, antwortete Thorald, „Und bitte, lass den Ton stecken. Du musst nicht so...“

„Ich spreche, wie mir der Schnabel gewachsen ist! Komm Thorald, auf zurück in die Rietburg! Zuerst schläfst du dir deinen Rausch aus und dann unterhalten wir uns einmal gründlich darüber, wie viel Schaden...“

„Ne, Ken, ich war gerade dabei...“

„Wie bitte?“

„Nein, Ken, ich komme noch nicht mit. Ich... ich bin immer noch dein König, verflucht noch mal! Und wenn ich die mächtigen Schilde lieber ihren Urhebern schenke, als dass du sie noch länger zu meiner Manipulation nutzt, dann hast du mir da nicht zu widersprechen!“

Jetzt war es raus. Thorald atmete schwer und wurde sich bewusst, dass er sich nun wieder sicher darin war, die mächtigen Schilde an Fürst Kram abtreten zu wollen.

Ken hingegen schüttelte bloß seinen Kopf und murmelte etwas unter seinem Atem. Thorald besah angespannt, wie Ken nach seinem Schwertgriff langte. Überlegte er sich gerade, ob es geschickter wäre, den unterwürfigen Ratgeber zu mimen oder zu mehr Beleidigungen und Gewalt zu greifen? Thorald wollte für keine der beiden Seiten Kens hier blieben.

So sprang Thorald auf, griff sich den Sack mit den beiden mächtigen Schilden – sein Schwert klapperte zu Boden – raste an Ken vorbei und stürzte ins Freie. Seine Augen brauchten einen kurzen Augenblick, um sich zu orientieren. Während das grelle Leuchten des Sonnenlichts abschwoll, erkannte Thorald die Schemen zweier weiterer Männer in der Rüstung der Rietgarde, die sich gerade zwei Pferden zuwandten, sowie ein drittes, etwas abseits stehendes Pferd, welches wohl Ken zur Anreise genutzt hatte.

Thorald mochte nicht mehr der beste Reiter des Landes sein, aber sofern Kens Handlanger zwei durchschnittlichen Gardisten waren, konnte er ihnen locker das Wasser reichen. Zudem kannte er Kens Pferd gut. So hopste er in der allgemeinen Verwirrung rasch auf Kens Rappen zu, schwang sich mehr oder minder elegant auf dessen Sattel und gab Fersengold.

Thorald lachte wie verrückt, als ihm einmal mehr die absurde Lage bewusst wurde, in der er sich befand. Der König des Landes, wie er mit zwei mächtigen Schilden vor seinem Berater davonpreschte. Er wollte einfach nur weg von alledem, doch nicht einmal bei Elga hatte er Ken und der Krone entkommen können. Wohin nun?

Thorald erinnerte sich noch verschwommen an die gute alte Zeit, als sein Vater Brandur mit ihm den Baum der Lieder aufgesucht hatte und Thorald mit einem gewissen Merrik davongeschlichen war, um die Quelle des Likko zu untersuchen. Sie hatten dort eine geräumige Höhle gefunden, welche sich hinter der Quelle verbarg. Merrik hatte die Höhle mit Freude zu skizzieren versucht, während Thorald die Höhlenwände mit kruden Zeichnungen bekritzelt und dann Merrik mit kalten Wasser beworfen hatte. Gute alte Zeiten. Thorald verspürte Nostalgie und hielt die Höhle für einen guten Ort, um sich zu verstecken und kurz zu verschnaufen. Und so tat er es.\bigskip







Thorald hatte ein bisschen Mühe damit, sein Pferd durch den Wasserfall vor dem Höhleneingang zu locken, aber mit ein wenig Zucker und ein wenig Zerren konnte er auch dieses Hindernis überwinden. Ganz nebenbei hatte der Wasserfall den Effekt, Thoralds verdreckte und verschwitzte Kleidung in Wasser zu tränken. Zumindest einen kleinen reinigenden Effekt musste das doch haben.

Im Innern der Höhle angekommen, ließ Thorald sich erst einmal fallen und rieb sich den schmerzenden Schädel. Draußen hatte ihn die warme Nachmittagssonne gewärmt, doch hier drinnen war es nun überraschend kühl. So schlang er seinen Mantel enger und schwang seine Arme, um sich etwas Wärme zu verschaffen. Dann wandte er sich Kens Pferd zu und versuchte, dessen Fell trocken zu rubbeln. Ungeschickt, wie er war, löste sich der Sack mit den zwei Schilden, welcher scheppernd auf den Höhlenboden polterte.

„Beim Barte des Urtrolls!“, fluchte Thorald, aber das laute Rauschen des Wasserfalls versicherte ihm, dass er von draußen nicht gehört werden konnte. Bestimmt war Kens Suchtrupp ihm bereits dicht auf den Fersen. Sollte er bei Melkart Exil suchen? Nein, er war der König von Andor, und er sollte erhobenen Hauptes zur Rietburg zurückkehren und seine Angelegenheiten richten. Noch immer plante er, die Krone an die Helden abzugeben, sobald diese aus dem Norden zurückkehrten. Wie lange das noch dauern konnte... schwer zu sagen. Gildas Gerüchten zufolge hatten sie viel Zeit damit verplempert, verschiedenen Nebelinseln auszuhelfen. Zumindest hatten sie als Belohnung den Sturmschild verliehen bekommen, den dritten mächtigen Schild, das war doch etwas. Ach ja, genau, die beiden mächtigen Schilde hatte er an Kram übergeben wollen, ehe ihn seine Beine zu Elgas und Junas Hof gelenkt hatten. Das Treffen mit Elga war sicherlich kein Fehler gewesen und hatte ihm ein wenig stark benötigte Zuversicht verleihen können, jetzt jedoch war es an der Zeit, sich wieder auf seinen Plan zu konzentrieren.

Ja, er würde die beiden mächtigen Schilde nach Cavern bringen. Sobald es draußen wieder dunkler geworden war. Damit Ken und sein Trupp ihn nicht vorher fanden und davon überzeugten, zurück zur Burg zu kommen und das Vorhaben aufzugeben. Das war ein sinnvoller Plan. Nur dem Warten sah Thorald nicht mit Freude entgegen. Warten war noch nie seine Stärke gewesen. Sein Magen knurrte bereits erneut und sein Blick wanderte über das Höhleninnere, in der Hoffnung, etwas Essbares zu finden. Irgendeine nette Pflanze oder einen Höhlenpilz.

Jetzt erst fiel Thorald auf, wie gut er das Höhleninnere erkennen konnte. Als er als kleines Kind hier gewesen war, war es stockfinster gewesen, aber nun konnte Thorald problemlos die hintere Höhlenwand ausmachen... sowie einen Gang, der davon wegzuführen schien. Aus diesem schien schwach grünliches Licht, als würden sich weiter hinten grün schimmernde Lichtquellen befinden. Er konnte doch unmöglich so nahe am Geheimen See sein, oder?

Ein unterirdischer Gang war nicht zwingend eine gute Nachricht. Höhlengänge bedeuteten potentiell die Präsenz von Höhlenwichten, oder schlimmer gar, Arpachen. Im besten Fall führte der Gang aber über eine Zwergentür direkt ins Herzen von Cavern, und Thorald müsste nicht einmal bis in den Abend warten, um die mächtigen Schilde dort abzuliefern. Er wog kurz seine Optionen ab und beschloss, zumindest einige Minuten in den Höhlengang vorzudringen.

Sein Pferd war rasch wieder mit den beiden mächtigen Schilden bestückt. Der Höhlenboden war relativ flach, somit hatte er kein Problem damit, die Stute in einen erleuchteten Höhlengang zu führen. Sie sträubte sich zunächst, war aber gut dressiert und folgte dann seinem Zügel.

Thorald wusste nicht, wie lange er den Gang entlangstrich. Der grünliche Schimmer wurde manchmal schwächer, dann wieder stärker, und von Zeit zu Zeit glaubte Thorald, Runen zu erkennen, die in die Wände geritzt waren und von denen das schwache grüne Leuchten ausging. Das beruhigte ihn. Demnach waren diese Wände eindeutig von Zwergen bearbeitet worden, wenn nicht sogar von ihnen geschaffen. Er war sicher hier. Es war nur seltsam, dass nirgendwo Fackeln hingen. War dieser Gang etwa so alt, dass die Zwerge beim Graben noch nicht das Geheimnis des Feuers besessen hatten? Thorald versuchte zurückzudenken an die Geschichtslektionen, die ihm einst erteilt worden waren. Schwertkunde hatte er deutlich lieber gelernt gehabt und die Kopfschmerzen vom gestrigen Gelage meldeten sich schon wieder deutlicher zu Wort, also gab er bald auf, die Runen zuordnen zu wollen.

\jdh{1193}

Bei seinem Brummschädel brauchte Thorald auch ziemlich lange, um zu erkennen, dass der Gang sich plötzlich weitete und wieder in den Wald öffnete. Thorald spitzte die Ohren, aber andere Pferde oder auch nur Schritte eines Suchtrupps Kens waren so wenig zu hören, wie sie nicht zu sehen waren. So wagte Thorald es, sein Pferd langsam aus dem Höhlenschatten ins Licht zu führen und tief durchzuatmen. Die Luft im Höhlengang war stickig gewesen, hier draußen im Freien war es viel angenehmer. Es lag allerdings ein ungewohnter süßlicher Duft in der Luft, den Thorald nicht näher zuordnen konnte. Kurz überlegte er sich, zurück in den Höhlengang zu kehren, aber die durch die Blätter knapp erkennbare Sonne warf schon lange Schatten. Bald würde es dunkel sein, da konnte sich Thorald doch schon jetzt auf den Weg nach Cavern machen.

Oder sich zumindest daran machen, seine Orientierung zu finden. Er befand sich irgendwo im Wachsamen Wald, also sollte er, wenn er sich einfach nach Süden bewegte, früher oder später wieder am Likko ankommen. Oder?

War das hier überhaupt der Wachsame Wald? Spuren von Bewahrern sah er nämlich nirgends. Und nach Mammutbäumen sah das Gewächs hier auch nicht aus. Wuchsen im Barbarenland andere Baumsorten? Oder konnte er die ganze Strecke bis zum Südlichen Wald unterirdisch überquert haben? Das schien ihm alles unwahrscheinlich.

Kurz überlegte Thorald, einen Baum zu erklimmen und die Gegend zu überblicken, sah dann aber davon ab, als er sich daran erinnerte, wie solche Kletterpartien in seiner Kindheit üblicherweise geendet hatten. So schwang er sich auf sein Pferd und ritt langsam weiter.

Kaum hatte Thorald ein bisschen Distanz zwischen sich und den Höhlenausgang gebracht, ertönte ein lautes Knirschen. Überraschte blickte Thorald nach hinten und erkannte, dass der Höhleneingang, aus dem er soeben getreten war, Teil eines riesigen steinernen Gesichts war, das in einen Felsen gemeißelt worden war. Thorald war offenbar aus dessen offenem Mund getreten, doch nun schloss sich ebendieser Mund knirschend und verschloss den Höhleneingang.

Erstaunt und erschrocken ließ Thorald sein Pferd näher treten. Sobald er einige Schritte vom Gesicht entfernt war, öffnete sich der Mund des riesigen Gesichts wieder und enthüllte den Höhlengang. Eigenartig. Dieses Mal glaubte Thorald zusätzlich, ein lautes Summen zu vernehmen. Er suchte in seinen Taschen nach dem Urheber des Geräuschs und fand den kleinen grünen Runenstein, den er früher am Narnenufer gefunden hatte und nun vor sich hin vibrierte. Vorsichtig ließ Thorald das Pferd einige Schritte zurückweichen, woraufhin das Summen des Runensteins versiegte und das Steingesicht seinen Mund wieder schloss. Ein faszinierender Mechanismus, um den Eingang zu verstecken, kein Zweifel, aber von den Runenmeistern der Zwerge war auch nicht weniger zu erwarten. Thorald zog an den Zügeln und führte das Pferd vom steinernen Gesicht weg.

Da ertönte ein lautes Plätschern und wie aus dem Nichts ergoss sich ein Schwall Wasser darüber. Ein wahrer Wasserfall verdeckte das steinerne Gesicht mit dem Höhleneingang nun vollständig. Thorald beachtete den Vorgang kaum und versuchte stattdessen orientierungslos, sich für einen optimalen Pfad zu entscheiden. Er wollte schließlich nach Cavern, und diese Mine lag südlich des Wachsamen Waldes. Süden war links von dort, wo sich die Sonne hinbewegte, oder?\bigskip







Thorald erreichte bald die nächste Waldlichtung und musterte interessiert einen blau schimmernden Runenkreis am Waldboden. Er ließ sein Pferd kurz anhalten und versuchte erneut erfolglos, sich zu orientieren. Da trat ein wahrer Hüne von einem Mann auf die Lichtung und bewegte sich schnurstracks auf Thorald zu. Der Hüne trug einen blauen Mantel mit einem über seine Schultern drapierten kurzen braunen Umhang, aber was darauf saß, zog Thoralds Aufmerksamkeit viel eher auf sich. Eine krumme Nase stach aus einem kantigen Gesicht hervor, welches von dichtem braunen Haar umrahmt wurde. Der Hüne stammte in jedem Fall nicht aus der Rietgarde oder zu Ken Dorrs Handlangern. Vielleicht ein Holzfäller?

„Zum Gruße, der Herr“, rief Thorald freundlich, „Mir scheint, ich habe mich etwas verlaufen. Mögt Ihr mir mitteilen, in welcher Richtung der Likko fließt?“

„Schweig stille, Adliger, ‘s gibt kein‘ ‚Likko‘ hier“, brummte der Hüne. Er blieb vor Thoralds Pferd stehen und zog in einer fließenden Bewegung zwei mächtige Schwerter von seinem riesigen Rücken. Inzwischen sah er relativ bedrohlich drein. Als Thorald an seinen Gürtel griff, stellte er erschrocken fest, dass da kein Schwert befestigt war. Er musste es bei seinem hektischen Aufbruch von Elgas und Junas Bauernkate vergessen haben. Trolldreck!

Der Hüne griff nun sanft nach den Zügeln von Thoralds Pferd und verkündigte freundlich: „Falls du’s noch nicht kapiert hast: Das hier is’n Überfall. Ich klau‘ jetzt dein‘ Sack.“

Der Hüne wollte doch tatsächlich den Sack mit dem Sternenschild und dem Bruderschild klauen. Thorald hätte beinahe aufgelacht.

„Weißt du denn nicht, wer ich bin?“, fragte er nervös.

„Irgendein stinkreicher Faulpelz, das biste“, meinte der Hüne selbstsicher und griff frech nach dem Sack mit den mächtigen Schilden.

Jetzt hatte Thorald genug. Geschwind löste er den Sack vom Sattel des Pferds, schwang sich aus selbigem und warf sich einige Schritte vom Hünen entfernt in Pose. Kurz überlegte sich Thorald, davonzurennen, aber dann würde er nur noch verirrter sein. Und wenn er sich diesen kräftigen Hünen ansah, würde Thorald wahrscheinlich die Schilde zurücklassen müssen, um ihm entkommen zu können. Die Schilde auch noch zu verlieren war eine Schande, das Thorald einfach nicht zulassen konnte. Es war schade, dass Thorald sein Schwert bei Elga und Juna vergessen hatte, aber das sollte nicht heißen, dass er nicht einem dahergelaufenen Rabauken das Wasser im Faustkampf reichen konnte.

„Ich bin dein König, du tumber Troll!“, protestierte Thorald, „Und wenn du diese Schilde willst, so musst du durch mich hindurch.“

„Wie du willst“, schmunzelte der Hüne. Er legte seine beiden Schwerter sorgfältig zu Boden („D‘mit ich dich nicht aus Versehen ernsthaft verletz‘. Wir sind kein‘ Bösen, wir woll‘n nur dein Hab und Gut“) und bewegte sich dann überraschend wendig auf Thorald zu.

Thorald hob seine geballten Fäuste und wich vorsichtig zurück.

„‘Der König‘, als ob ich nich‘ lach‘“, murmelte der Hüne und griff nach Thorald.

„Es stimmt!“, erwiderte Thorald trotzig, während er dem Hünen aus dem Griff glitt und ihn mit seinen Fäusten bearbeitete, „Und wenn du mich nicht sofort gehen lässt, wirst du vor den Rat der Bewahrer kommen, dass schwöre ich!“

„Der König ist leider weit weg von hier“, murmelte der Hüne unberührt, „Und wenn die Krone mich fängt, krieg‘ ich ohnehin schon den Strick. Der steht allen Geächteten zu.“

Thorald bemühte sich, die Kniekehle des Hünen zu treffen, aber dieser ließ sich davon nicht groß beeindrucken. Dann fand die Faust des Hünen ihr Ziel und Thorald wurde schwarz vor den Augen.













\newpage
\section{Thorald trifft vom Weg Abgekommene}

„Edler Herr, seid Ihr verletzt?“

Thorald rieb sich zum zweiten Mal an diesem Tag den schmerzenden Kopf und erwachte aus der Dunkelheit. Über ihm gebeugt stand ein Soldat in einer ihm unbekannten Uniform und stupste ihn mit einem Speer an. Rot war seine Kleidung, und rot-weiß war das spitze Schild, das er fest in seiner anderen Hand hielt.

„Wo... wo bin ich?“, stammelte Thorald.

„Im Forest seid Ihr, und allem Anschein nach hat Euch einer der Geächteten übel erwischt. Diese Bastarde haben es schon seit Wochen auf die Adligen der Grafschaft von Nottingham abgesehen.“

Die Wache legte ihren Speer ab und kniete neben Thorald hin: „Geht es Euch gut? Wurdet Ihr am Kopf erwischt?“

Thorald nickte bloß und fragte dann: „Konntet Ihr die magischen Schilde sicherstellen? Und wisst wenigstens Ihr, dass ich Euer König bin?“

Die Wache blickte verdutzt drein und murmelte dann: „Auweia. Öhm. Kommt, Ihr müsst Euch erholen. Ich besitze leider nicht die Heilkünste, die benötigt werden, um mit einer solchen Kopfverletzung umzugehen. Stützt Euch auf mich, ich bringe Euch zur Burg. Unser Heiler wird sich Euer annehmen können.“

„Ja, die Rietburg ist ein gutes Ziel, ich weiß, wo die steht“, lallte Thorald. Sein Blickfeld wurde wieder unscharf, und die Stimme der Wache hallte, als befänden sie sich immer noch in der Höhle bei der Quelle des Likko.

„Ja, ja, zur Burg mit uns. Auf jetzt. Nein, lasst Euch nicht fallen. Bei Gott, Ihr seid ja in einem Zustand...“

Danach nahm Thorald nur noch verschwommen war, wie Sprachfetzen an sein Ohr drangen.

„Warum müsst Ihr auch so schwer...

„Kommt, Alrich, helft mir kurz...“

„Dieser Karren ist konfisziert im Namen des...“

Plötzlich musste Thorald nicht mehr laufen, sondern konnte sich auf ein Strohbett sinken lassen. Aber das war gar kein Strohbett, das schaukelte ja viel zu heftig. Wie eigenartig.

So sank Thorald in einen unruhigen Schlaf.\bigskip







Als Thorald aufwachte, brauchte er einen Moment, um seine Erinnerungen zu sortieren und zu verstehen, was geschehen war. Es gelang ihm nicht vollständig. Er bemerkte, dass sein Kopf nicht mehr schmerzte und stattdessen von einem weichen Samtkissen aufrecht gehalten wurde.

Überrascht schlug Thorald seinen Augen auf und blinzelte gegen die grellen Lichtstrahlen. Dann fiel ihm auf, dass er sich in einem ihm vollkommen unbekannten Raum befand. Und dass eine ihm vollkommen unbekannte Person auf einem Stuhl neben seinem Bett saß und ihn aufmerksam musterte. Der Mann trug einen eleganten, königlich roten Mantel und feuerrotes Haar, das Thorald an Fenn erinnerte. Seine Hände hatte er gefaltet und seine Aufmerksamkeit galt ganz dem erwachenden Thorald.

Thorald beschloss, die unangenehme Stille zu unterbrechen. Er räusperte sich, hustete und stammelte dann: „Ich... ich danke Euch ganz herzlich für die Gastfreundschaft. Ich war nur auf der Durchreise durch den Wald, als ich von einem groben Hünen überfallen wurde.“

Die Person auf dem Stuhl nickte: „Das wurde mir von der Wache auch berichtet. Ich bin der Sheriff von Nottingham, und es ist meine Aufgabe, diese Geächteten und ihren Anführer Robin Hood zu fangen und zu richten.“

„Sehr erfreut, Sheriff“, nickte Thorald, „Ich bin der Thorald von Andor“. Ein seltsamer Name war das, Sheriff. Und auch von einem Nottingham hatte er noch nie gehört. Konnte es sein... vor vielen Jahren hatten die Andori erfahren, dass das Fahle Gebirge ihnen ein ihnen bislang unbekanntes Land verborgen hatte, das mythische Tulgor, Land der Architektur und Technologie, der Temm und der Takuri. Konnte es sein, dass die Andori von der Existenz eines weiteren Landes nicht wussten? Auf jeden Fall lag Thorald auf einem Bett in einem ihm fremden Gemäuer. Einer Burg? Thorald wusste von keiner bewohnten Burg außer der Rietburg. Wie fern von seinem Heimatland befand er sich nur?

„Thorald von Andor, soso“, murmelte Sheriff, „Ihr seht eindeutig wie ein Adliger aus, wenn auch ein ziemlich verdreckter. Schon alleine Euer Umhang dürfte mehr wert sein, als eine Wache in einem Monat verdient. Seltsam ist bloß, dass ich noch nie von einem Adelsgeschlecht namens ‚Andor‘ gehört habe. Und ich rühme mich, in den Adelsgeschlechtern ziemlich bewandert zu sein.“

Thorald wusste, dass er vorsichtig vorschreiten musste. Sheriff schien eine mächtige Person zu sein, und mächtige Personen tendierten oft zu einem Hang für Willkür. Es schien ihm nicht falsch, auf die Symmetrie ihrer Situation hinzuweisen.

„Nun, Sheriff“, sprach Thorald, „Mir scheint das Ganze ebenso seltsam. Ich habe noch nie von einem Nottingham gehört. Wo befinde ich mich denn überhaupt?“

„Im Nottingham Castle, dem Herzen von England!“, kam die Frage wie aus der Pistole geschossen.

„England?“

Der Sheriff schüttelte lachend seinen Kopf: „Ich habe ja schon gehört, dass einen Schlag auf den Kopf einem Manne den Geist vernebeln kann, aber das ist schon unerhört, was ihr mir hier auftischt. Woher glaubt Ihr denn zu stammen, wenn ihr nicht einmal England kennt?“

„Wie ich bereits sagte: Ich stamme aus Andor, dem Drachenland.“

Der Sheriff verschluckte sich fast vor Glucksen, als er das hörte: „Drachen?! Als nächstes erzählt Ihr mir noch von Feen und Einhörnern!“

„Von den Drachen habt Ihr also schon gehört!“, rief Thorald freudig auf, „Dann werdet ihr ja wissen, dass die Drachenspezies vor langer Zeit schon ausgelöscht wurde bis auf den letzten. Dessen Hort... da direkt nördlich davon liegt Andor!“

Sheriff schlug sich auf die Schenkel: „Haltet ein, Thorald von Andor. Ich mag solche Geschichten so gerne wie der nächste, aber irgendwann wird es einfach zu viel. Ich hoffe, dass sich Euer Geist bald wieder beruhigt, auf dass wir einige Informationen über die Geächteten erfahren mögen.“

Wie konnte es sein, dass der Sheriff von Drachen, nicht aber von Andor gehört hatte? Hatte der Unterirdische Krieg vielleicht nur alle Drachen in der Nähe Andors ausgelöscht, aber weiter weg lebende verschont? Die Gerüchte über einen Eisdrachen im Hohen Norden würden dem ja auch entsprechen. Nicht, dass Thorald diesen Gerüchten Glauben schenken würde. Aber Ken mochte es, darüber zu diskutieren.

Da fiel Thorald etwas anderes auf: „Ihr behauptet, noch nie von Andor gehört zu haben. Wie kommt es dann, dass Ihr die andorische Sprache sprecht? Die Sprache, die den flüchtigen Ambacus aus Krahd einst von den Bewahrern gelernt wurde?“

Thorald besann sich, einst von einem legendären Trank gehört zu haben, welcher seinem Trinker erlaubte, Strukturen in gesprochener Sprache zu erkennen. War es vielleicht möglich, mit Runenmagie einen ähnlichen Effekt zu erzeugen? Runenmagie wie derjenigen, die den langen Gang zwischen Andor und dieser seltsamen Welt geziert hatte? Aber würde er die Sprache dieser Menschen dann so klar verstehen können? Und warum in diesem schrecklichen Akzent?

„Nein, mit ‚andorischer Sprache‘ hat das gar nichts zu tun. Ihr sprecht die Sprache Englands, wenn auch mit einer, wage ich zu behaupten, abscheulichen Betonung“, schmunzelte Sheriff.

War es möglich, dass sie tatsächlich dieselbe Sprache sprachen? Das würde, nein, musste bedeuten, dass sie gemeinsame Wurzeln hatten. Dass ein Kontakt zwischen den Welten bestand, oder dass gar Menschen aus Andor dieses Land besiedelt hatten, wenn nicht umgekehrt. Die Implikationen davon mochte sich Thorald gar nicht ausmalen. Nein, das hätten die Bewahrer vom Baum der Lieder mit all ihren Aufzeichnungen doch unmöglich unter Verschluss halten können.

Sheriff unterbrach Thoralds Überlegungen: „Auf welchem Pfad seid Ihr hierhergekommen? Zu Fuß? Zu Wagen?“

„Hoch zu Ross“, fuhr Thorald fort, „durch einen unterirdischen Gang. Erst hielt ich ihn für das Werk von Zwergen, aber die Zwerge hatten noch nie von einem England berichtet... vielleicht war das ein Feenpfad durch die Feenwelt? Solche können große Distanzen überwinden.“

„Ich sagte doch bereits: Haltet ein mit den Märchen!“, rief Sheriff nun mit einem bedrohlicheren Unterton in der Stimme, „Und bleibt bei der Wahrheit! Nichts von Feen, Zwergen, Magie oder dergleichen. Diese existieren nicht!“

Da erkannte Thorald endlich, dass Sheriff all diese Wesen und gar die Magie selbst für Hirngespinste zu halten schien, und verstummte überrascht. Sheriff war eindeutig ein reizbarer Mensch, und solange er Thoralds Erzählung nicht glaubte, war es vielleicht geschickter, sie nicht weiter auszuführen. Aber lügen wollte er auch nicht. Ach, wenn doch nur Ken hier wäre, der wüsste, was zu tun wäre.

Beim Gedanken an Ken spürte Thorald ein feines Stechen in seiner Brust, aber er ließ ihn nicht los, sondern versuchte, sich im Geiste Kens schnarrende Stimme vorzustellen:

„Ein sehr skeptischer und ignoranter Mensch ist das. Kein Wunder, dass er dir nicht glaubt, vor einem Jahrzehnt hättest du auch niemandem die Existenz von Tulgor abgekauft. Du musst etwas finden, was ihn überzeugt, dass du nicht auf den Kopf gefallen ist. Tische ihm keine alten heiteren Geschichten über Feen und Zwerge auf, sondern demonstriere ihm irgendetwas etwas Neues, Dunkles, vielleicht gar etwas Magisches?“

Da hatte Thorald eine neue Idee: Er griff in seine Hosentasche und zog den grünen Runenstein hervor, den er bei seiner Reise zu Elgas und Junas Hof gefunden hatte.

„Und was sagt Ihr zu diesem Prachtstück hier? Ein stärkender Stein, geprägt mit der Macht der Runen. Er war es, der mir den Gang in dieses Land überhaupt öffnete. Ein klassisches Beispiel für Magie!“

Sheriff griff nach dem Stein, wog ihn in der Hand und warf ihn dann zu Thorald zurück: „Ist das Futhorc? Ein hübscher Stein, kein Zweifel, aber ich erkenne darin keine Spuren von Magie.“

Jetzt, wo er es sagte, musste Thorald ihm recht geben. Der Stein glomm nicht mehr und als Thorald ihn an sein Ohr hielt, summte er auch nicht leise, wie die Runensteine es sonst taten. Er lag einfach nur still da. Als wäre die Magie in seinem Inneren erloschen.

Natürlich hatte Thorald schon erlebt, dass die Runensteine an verschiedenen Orten mit ihrer Umgebung mehr oder weniger resonierten – an den Eingängen zu den unterirdischen Zwergengängen waren sie beispielsweise mächtiger als über dem offenen Meer – aber dass er einen Runenstein fest in seiner Hand halten konnte, während dieser komplett erloschen blieb, das hatte er noch nie erlebt. Was war das hier für eine Umgebung?

„Keine weiteren Geschichtchen mehr?“, lachte Sheriff und behauptete stolz, „Das magischste, was ich je in meinem Leben erlebt habe, war eine präzise dressierte Krähe. Da draußen züchtet jemand Krähen und schickt sie auf uns los. Wenn wir eine von denen einfangen könnten, oder besser gar, zu ihrem Hüter verfolgen, dann wäre dies überaus hilfreich. Ich schwöre beim Herrn...“

„Beim Herrn?“

„Ach, kommt schon, Ihr glaubt doch nichts ernsthaft, dass ich Euch abkaufe, noch nie vom Herrn gehört zu haben.“

„Ich will Euch wirklich keinen Bären umbinden.“

„Bären hat es also dort, woher Ihr kommt?“

„Natürlich gibt es in Andor Bären!“

„Aber Gott kennt man dort nicht? Seid Ihr etwa ein Ketzer?“

„Ich habe noch nie in meinem Leben eine Kerze gezogen“, versicherte Thorald Sheriff.

„Ihr müsst Euch dringendst mal mit Father Egbert unterhalten, werter Thorald von Andor. Naja, sobald dieser sich wieder eingekriegt hat“, gluckste Sheriff, „Im Moment ist er nicht so gut auf die Krone zu sprechen, seitdem wir das goldene Kreuz der Kirche für die Kriegssteuer einziehen mussten.“

„Ach ja, Steuern. Das Volk mochte sie bei uns auch nicht sonderlich.“, murmelte Thorald bedrückt und erinnerte sich daran, wie Ken ihn zum Erheben größerer und größerer Steuern ermuntert hatte. Die Staatskasse Andors war nun praktisch am Überlaufen, aber das Volk war nicht besser auf die Krone zu sprechen. In den Flusslanden gab es gar Schreie nach Unabhängigkeit. Obwohl Thorald versprochen hatte, das Gold zum Wohle Andors zu nutzen, sobald sich eine Gelegenheit ergab.

Sheriff schein das nicht so zu sehen, denn er gluckste auf und stimmte zu: „Natürlich mag das Volk die Steuern nicht, die wollen ja lieber ihren Wohlstand für sich behalten. Aber Wohlstand der einzelnen Bürger besiegelt nun mal keine Bündnisse fürs ganze Volk. Und arme Schlucker sind abhängiger vom guten Willen der Krone. So können wir unsere schöne Macht beibehalten. Diejenigen, die nicht genügend leisten oder sich uns widersetzen, werden zu Geächteten, und diese wiederum erhalten früher oder später das, was sie verdienen: Den Tod!“

Thorald spürte ein flaues Gefühl in seinem Magen. Dieser Sheriff hatte nicht das Geringste Mitgefühl für seine Untergebenen. Daran erkannte man einen Tyrannen, das hatte ihm Brandur schon oft gesagt.

„Hört mal, werter Herr Sheriff...“, setzte Thorald an.

„Sheriff ist kein Name, sondern mein Titel“, brummelte der Sheriff, „Wenn ihr mich schon bei Namen nennen müsst, so nennt mich William von Wendenal.“

„Werter Herr von Wendenal“, setzte Thorald erneut an, doch blieb dann stumm. Sich mit dem Sheriff über die Moral seiner Taten zu unterhalten, wäre ein guter Weg, selbst auf die Feindesliste dieses Tyrannen zu landen. Stattdessen fuhr er fort:

„Ich könnte Euch vielleicht von der Wahrheit meiner Geschichte überzeugen, wenn ich Euch zum Höhlengang in mein Heimatland führte und dieses mit dem Runenstein öffnete. Wie klingt das in Euren Ohren?“

Sheriff lächelte mit kalten Augen: „Ha! Als ob ich so viel Zeit zur Verfügung hätte. Viele Angelegenheiten in der Grafschaft erfordern meine Aufmerksamkeit. Der hohe Prinz John hat sich höchstpersönlich hier in Nottingham einquartiert, könnt ihr es euch vorstellen? Was für eine Ehre! Seit der König in die Kreuzzüge gezogen ist, regiert der Prinz das Land an seiner Stelle. Darum müssen wir doppelt so hart durchgreifen, Wachpatrouillen verdoppeln und Räuber und Diebe ohne Gnade dem Tod überweisen. Und darum haben wir keinerlei Mannen für derlei Hirngespinste aufzuwenden.“

Thorald sagten eine Menge dieser Worte nichts, aber innerlich verfestigte sich sein Gefühl, dass er als König Andors nicht einmal so schlecht gewesen war. Zumindest hatte er nicht seine eigenen Untertanen ermorden lassen.

„So glaubt mir doch, warum würde ich Euch auf den Arm nehmen wollen?“, fuhr Thorald fort, der spürte, dass er ohne Beweise dieses Castle nicht mehr lebend verlassen würde, und der sich sicher war, dass ein Wiederholen seiner Behauptungen diese glaubhafter machen würde, „Genau dieser Runenstein hier öffnete den Ausgang zum Höhlengang zwischen unseren Reichen. Der war mit Runen verziert, Runen wie der auf diesem Stein hier. Der Ausgang hatte die Form eines riesiges steinernen Gesichts, und hinter seinem Mund liegt der Gang nach Andor...“

„Ein steinernes Gesicht?“ horchte der Sheriff plötzlich auf, „Hat es geweint?“

„Öh... vielleicht? Es war jedenfalls Wasser dort...“

Leise flüsterte der Sheriff: „‘\textit{Trittst du in den Kreis der Runen ein, wird der Strom der Tränen versieget sein.}‘ Ist das möglich, dass... ?“

Der Blick des Sheriffs, der Thorald nun traf, war urplötzlich kalt und entschlossen.

„Ich lasse einen Trupp bereitmachen. Führt sie zum steinernen Gesicht.“\bigskip







Thorald fühlte sich endlich wieder ein wenig königlich, als er auf einem edlen Pferd von Nottingham Castle aus aufbrach, in Begleitung zweier Wachen und des besten Söldners, den man für gutes Gold anheuern konnte. Guy von Gisbourne hieß er, und offenbar verdingte er sich als Jäger dieser Geächteten, die Thorald die mächtigen Schilde abgenommen hatten – wenn auch bislang bloß mit mäßigem Erfolg. Thorald zweifelte daran, dass ein so grimmiger Bursche dem geschickten Fenn aus dem Barbarenland im Fährtenlesen etwas vormachen konnte.

Der Sheriff betrachtete die Abreise des Trupps vom Torbogen aus, sein rotes Gewand in starkem Kontrast zum blauen Himmel. Insgeheim nahm Thorald sich vor, ein Wams in derselben Farbe in Auftrag zu geben. Die Farbe hatte etwas.

Auf dem Weg aus der Burg heraus blieb Thoralds Blick an einem großen Holzgestell hängen, an dem ein langes Seil mit einer Schlaufe am Ende hing. Thorald fragte Gisbourne, ob es sich dabei um ein Kunstwerk oder ein Werkzeug handelte. Gisbourne lehnte sich aus seinem Sattel und flüsterte Thorald ins Ohr: „Das ist der Galgen, an dem der Henker alle Gesetzesbrecher aufknüpft. Dieses Schicksal steht vielleicht auch dir bevor. Der Sheriff mag dir deine Wahnsinngeschichte abkaufen, aber so leicht trügt man mich nicht! Ich habe ein Näschen dafür, wo die Geächteten auftauchen, und diese Nase schlägt Alarm, wenn ich mich auch nur in deine Nähe komme. Sollte dein Tipp ins Leere schlagen, so lasse ich dich gefesselt und gekettet vor Prinz Johns Urteilspruch bringen. Und dann werde ich genüsslich zusehen, wie du am Galgen zappelst, bis dir die Luft ausgeht“

Thorald schluckte tief. Er hatte die letzten Jahre stark damit gekämpft, ein schlechter König zu sein, aber im Vergleich mit diesem Herrscher kam er sich ganz goldig vor. Ein Werkzeug zum Ermorden der eigenen Untertanen, so prominent zur Schau gestellt?! In was für einer Welt war Thorald hier nur gelandet?

„Meister Gisbourne...“, versuchte er, seine adlige Haltung zu bewahren, „Ich führe Euch hier freiwillig zu diesem Gang, in der Hoffnung, zurück in mein Land zu kommen. Ein wenig Dank wäre von Eurer Seite nicht fehl am Platze“

Thoralds zog rasch wieder den Kopf ein, als Gisbourne bloß knurrte und eine unwirsche Geste in seine Richtung machte. Was für eine grausame Kreatur dieser Söldner doch war. Thorald beharrte: „Ich kann Euch versichern, dass ich nicht lüge. Es mag sein, dass ich den Weg auf die Schnelle nicht finde, aber wenn wir nur lange genug suchen, werden wir die Stelle schon finden. Den Eingang kann man auf jeden Fall nicht übersehen. Seht, wir folgen den Radspuren des Wagens, der mich zur Burg gebracht hat. Folglich liegt der Gang in dieser Richtung. Haltet einfach Ausschau nach einem Runenkreis oder einem steinernen Gesicht.“

Thorald hob seine Hand und zeigte grob in die Richtung, in welcher er das Gesicht vermutete. Ein Rascheln ließ seinen Blick ins Unterholz gleiten, wo sich soeben eine grau gewandte Gestalt mit einem großen Buckel in den Schatten zurückzog. War das Reka, die Kräuterhexe? War sie etwa auch dem Tunnel in dieses fremde Land gefolgt?

„Keine Ablenkungen“, zischte Gisbourne, „Auf, auf! Die Geächteten kommen näher, das spüre ich. Wenn wir uns beeilen, können wir sie vielleicht überraschen.“

Da fühlte Thorald, als hätten die Baumwipfel Augen und würden auf den kleinen Trupp des Königs starren, wie er durch den Sherwood Forest ritt. Ein Blick nach hinten verriet ihm, dass die beiden Wachen ebenfalls vorsichtig die Bäume inspizierten. Nur Guy von Gisbourne ritt mit starrem Blick nach vorne voran.

Sie waren kaum eine Viertelstunde weiter gezogen, da schrie Gisbourne plötzlich auf. Thorald hatte seine Schwierigkeiten damit, sein Pferd zu zügeln, da zischten auch schon mehrere irgendetwas an seinem Kopf vorbei.

Im Licht, das die sinkende Sonne durch die Wipfel warf, erkannte Thorald mit Schrecken, wie die beiden Wachen hinter ihm zu Boden sanken. Aus der einen ragten zwei Pfeile, von denen der eine den anderen beinahe vollständig gespalten hatte.

Thorald wurde bleich. Hilflos suchte er die Umgebung ab, nach einem Zeichen, wer denn hinter diesem Überfall steckte. Gisbourne starrte die meisterhaft geschossenen Pfeile an und flüsterte: „Das ist Robin Hood. Das ist ihr Anführer!“

Ein irres Glänzen schlich sich in seine Miene, als er ausrief: „Zeig dich, Robin, wo versteckst du dich?“

Als wäre es seine Antwort, surrten zwei weitere Pfeile durch die Luft und schlugen gegen die Brustplatte des Söldners. Gisbourne keuchte, aber die Pfeile hatten seinen Brustpanzer nicht durchschlagen und fielen wirkungslos zu Boden.

Gisbourne zog einen verdreckten Helm hervor und setzte ihn auf. Lilane Federn ragten stolz daraus hervor und nur noch seine wahnsinnig glänzenden Augen waren von ihm zu erkennen. Jetzt würde es erst recht schwer werden, ihn zu treffen.

Thorald fluchte und glitt von seinem Pferd, in der Hoffnung, am Boden vor einem Pfeilhagel sicherer zu sein. Er landete unsanft in einer Ansammlung von Waldpilzen, welche prompt zu stäuben begannen und seine Nase reizten. Thorald nießte einmal, zweimal laut und rollte sich aus den Pilzen auf den matschigen Waldboden. Es stank nach Pferdedreck.

Als Thorald sich aufrichtete, sah er zwei Gestalten entgegen, welche in die hell erleuchtete Lichtung traten. Die eine war schlank, aber großgewachsen, in einen himmelblauen Umgang samt Kapuze gekleidet, und hielt einen mächtigen Bogen auf Gisbourne gespannt. War das dieser legendäre Robin Hood, der Anführer der Geächteten? Die andere Gestalt trug ihr langes feuerrotes Haar offen und stützte sich auf eine schwere Axt. Wenn Thorald sich nicht täuschte, wehte der Wind gerade ein bisschen zu viele Blätter um sie herum, als dass es ein Zufall sein könnte. Ein schwarzer Rabe stürzte sich aus dem Himmel und flog der Gestalt ohne Kapuze in ihre ausgestreckte Hand, woraufhin sich Thorald an die Worte des Sheriffs erinnerte:

„Da draußen züchtet jemand Krähen und schickt sie auf uns los.“

„Zurück!“, erklang ihre helle, gebieterische Stimme, „Ziehet von hinnen und wir schenken Euch Euer Leben!“

„Zur Hölle! Wo ist Robin Hood?!“, zischte Gisbourne und presste seine Fersen in die Flanken seines Pferdes. Dieses wieherte, richtete sich auf seine Hinterbeine und sah ganz gefährlich aus. Dann ließ es seine Vorderbeine wieder auf den Boden prallen und nahm geifernd Kurs auf die beiden Geächteten. Gisbourne zückte seine blanke Klinge und richtete sie auf die Gestalt mit dem Raben.

Der himmelblaue Bogenschütze feuerte auf den Söldner, einmal, zweimal, aber wieder prallten die Pfeile nutzlos an Gisbournes Rüstung ab. Mit einem frustrierten Grunzen schulterte der Bogenschütze seinen Bogen wieder und rief: „Gisi? Darf ich dich Gisi nennen? Die Welt wird sich erfreuen an den Spottliedern, die ich über dich verfassen werde... sobald du erst einmal das Zeitliche gesegnet hast!“

Der Bogenschütze zog ein langes Messer hervor und richtete es auf den anstürmenden Gisbourne, scheinbar ungerührt von dessen Ansturm. Gisbourne gab sich ähnlich ungerührt, zügelte sein Pferd und bellte vom Rücken seines Gauls einen knappen Befehl. Wie aus dem Nichts tauchten hinter den beiden Geächteten zwei weitere rote Wachen auf und fuchtelten mit langen Speeren. Die Frau mit dem Raben wich zurück und sprang auf den nächsten Baum, während ihr Rabe der Wache ins Gesicht flog und allen Anschein nach ansehnlichen Ärger bereitete.

Dann aber wurde der Rabe davongescheucht und die beiden Wachen drängten den übriggebliebenen Bogenschützen in Thoralds Richtung. Ein Speer verfing sich in der himmelblauen Kapuze, riss sie herunter und enthüllte ein faltenloses Gesicht mit einem strubbligen schwarzen Haarschopf. Das war doch noch kaum ein erwachsener Bursche! Thorald sah zu, wie der Geächtete gefährlich unkontrolliert mit seinem langen Messer herumfuchtelte, und beschloss, einzugreifen. So griff er nach einem am Boden liegenden Büschel von Ästen und schleuderte diese ungelenk von sich. Nicht viele der Äste trafen ihr Ziel, doch zu seiner Freude sorgte sein Eingreifen dafür, dass der Bursche sich umdrehte und die beiden Wachen sich auf ihn stürzen konnten. Nach einem kurzen Gerangel wurde das Messer des Burschen fortgeschleudert. Gisbourne sah dem ganzen Vorgehen aus einiger Entfernung grinsend zu. Thorald richtete sich zu seiner vollen Größe auf und wischte sich den Dreck von seinen Kleidern.

Ein dumpfer Schlag direkt neben ihm ließ sein Herz wieder in die Hose sinken.

„Haste mich vermisst?“, ertönte eine tiefe Stimme. Der Hüne, der Thorald erst vor Kurzem so freundlich in diesem Reich begrüßt und ihm die beiden mächtigen Schilde entwendet hatte, war auch hier! Thorald schrie mutig auf und gab Fersengeld, aber kaum zwei Mannslängen weiter erwischte ihn der kräftige Griff des Hünen dennoch.

Der Hüne warf Thorald ohne Mühe über seine Schulter. Jetzt konnte Thorald nur noch den verschwommenen Waldboden ausmachen, wie er hin- und herschaukelte, während der Hüne mit großen Schritten davonrannte. Ein mächtiger Satz, dann ein Knarzen wie von einem Baumstamm, dann... Stille. Ein Schrei erklang, in dem Thorald die Stimme Gisbournes zu erkennen glaubte, aber er schien weit weg.\bigskip







Thorald wurde sanft zu Boden gelassen. Das war kein Waldboden, sondern... Bretter? Eine Brücke in den Bäumen? Ehe Thorald sich genauere Gedanken dazu machen konnte, hatte sich der Hüne bereits über ihn gebeugt und platzierte seine breite Hand fest auf Thoralds Mund.

„Hübsch still sein, gell?“, wies er Thorald an. Dieser hustete und blickte sich panisch um, aber Widerstand schien ihm im Moment nicht das geschickteste Vorgehen zu sein.

Es raschelte neben ihm und die Gestalt mit dem Raben trat in sein Blickfeld. Überrascht erkannte Thorald, dass es sich um eine Dame in einem langen Kleid handelte. Das Bild passte so gar nicht zur mächtigen Axt, die sie mit zwei Händen führte.

„Sie haben Will geschnappt“, fluchte sie.

„Un‘ wir ha’m den hier“, meinte der Hüne,

„Aber wir hängen unsere Gefangenen im Gegensatz zu ihnen nicht einfach so, wenn sie nichts dagegen tun.“

Der Hüne kratzte sich am Kinn und antwortete dann: „Wenn Gisi ihn geschnappt hat, gibt es so hurtig nichts, was wir für ihn tun könn‘. Komm, Marian, wir bringen den hier zu Robin, und dann guck’n wir weiter, wie wir Will aus dem Castle kriegen.“

„Deine Hoffnung hätte ich gerne“, schüttelte die Dame – Marian – ihren Kopf.

Der Hüne meinte schlicht: „Der Herr ist mit uns.“

Marian schluckte tief, nickte dann aber: „Na gut, dann auf zu Robin mit ihm. Zuletzt hat er auf der anderen Seite der Lichtung Position bezogen. Suchen wir ihn dort.“

„Nicht nötig“, erklang eine leise Stimme dicht neben dem Trio, und alle drei zuckten zusammen, wenn auch Thorald am stärksten. Er versuchte erfolglos, seinen Kopf so zu drehen, dass er den Neuankömmling erkennen konnte.

Der Neuankömmling indes beugte sich so nahe an Thorald heran, dass dieser seinen Atem spüren konnte, und murmelte: „Ich bin Robin von Loksley.“

Thorald stammelte ein „Sehr erfreut“, das durch die Hand des Hünen kaum zu verstehen war.

Robin fuhr ungerührt fort: „Wir sind etwas in Eile, darum komme ich gleich zum Punkt: Wer seid Ihr, und warum wisst Ihr, wo unser geheimes Lager liegt?“

Thorald schluckte. Er wusste doch gar nicht, wo ihr geheimes Lager lag. Und so skeptisch, wie der Sheriff auf seine Geschichte reagiert hatte, war es vielleicht nicht das Geschickteste, einem Haufen Halunken seine Herkunft zu erklären versuchen.

Schon entfernte der Hüne seine Hand von Thoralds Mund. Thorald krabbelte ein wenig zurück und richtete sich auf. Ja, er stand auf einer Art Baumbrücke, durch deren Ritzen er den erstaunlich weit von ihm entfernten Waldboden erahnen konnte. Ihm wurde etwas schwindelig und er richtete seinen Blick lieber auf die drei Geächteten, die ihn gespannt anblickten. Neben ihnen sah Thorald nun den Sack, den der Hüne Thorald bei ihrer ersten Begegnung abgenommen hatte. Die beiden mächtigen Schilde waren hier, direkt in seiner Reichweite!

„Nun?“, fragte Robin erneut. Sein Ton war nicht freundlicher geworden.

„Ich... ich bin ein Reisender aus einem fremden Land“, gab Thorald vorsichtig von sich, „und ich habe wirklich nicht die geringste Ahnung, wo Euer geheimes Lager liegt. Und ebenso wenig Interesse daran.“

Der Hüne gluckste: „Wir würd’n dir ja gerne glauben, aber dass du den groben Gisi direkt in uns’re Richtung geführt has‘, sagt ‘was andres als deine Stimm‘.“

Robin nickte: „Wenn der Hüter des Waldes mich nicht vorgewarnt hätte, hättest du Guy und seine Truppen direkt zu uns gebracht.“

„Aber... aber ich bin doch erst seit einigen Stunden hier! Wie soll ich denn ein geheimes Lager gefunden haben, welches nicht einmal der beste Söldner des Landes aufspüren konnte?“

Robin wandte sich den anderen beiden zu: „Er lügt nicht... könnte es sein, dass er vielleicht ebenfalls...“

Marian unterbrach ihn: „Was ist das für ein fremdes Land, aus dem du stammst?“

Ach, was soll’s. Etwas ausdenken wollte Thorald sich nun wahrlich nicht, das war nie seine Stärke gewesen.

So begann er: „Mein Heimatsland heißt Andor. Ihr mögt noch nicht von ihm gehört haben, aber es existiert. Wirklich. Und ich bin dort der K... eine wichtige Persönlichkeit. Also wagt es nicht, mir etwas anzutun. Ihr wollt keinen Ärger mit uns anfangen.“

Seine Drohung ignorierend, sahen sich Robin und Marian wissend an, und der Hüne gluckste erneut. Dann sprach er: „Witzig, dass du‘s so sagst. Wir ha’m den Namen nämlich g’rad vorhin schon mal g’hört.“

Als wäre das ihr Stichwort gewesen, schwang sich eine dunkel gewandete Gestalt auf die Plattform in den Bäumen und kam neben Robin zum Stehen.

Das Licht der untergehenden Sonne traf auf einen kahlen Schädel. Thorald erkannte überrascht Ken Dorr, seinen treuen Ratgeber, wie er ungerührt neben den drei Geächteten stand und ihnen Rapport gab: „Sie haben Will ins Castle gebracht. Ich habe es mit eigenen Augen gesehen.“

Robin erwiderte ebenso ungerührt: „Dann hast du bestimmt nichts dagegen, wenn ein paar fremde Augen das bestätigen lassen.“

Er gab ein Handzeichen. Marian nickte und sprang von der Baumbrücke. Eine Rabe stürzte aus dem Himmel und folgte ihr.

Ken Dorr fixierte Robin aus kalten Augen und lachte auf: „Endlich jemand mit Verstand, Robin von Loksley! Es wäre töricht gewesen, meinem Rat einfach so zu vertrauen. Schön, zu sehen, dass...“

„Still, Kennard. Zu dir komme ich noch.“

Ken verstummte und zog sich ein bisschen zurück, die Hand scheinbar locker auf seiner Hüfte liegend, aber gefährlich nahe an seiner Schwertscheide. Sein Gesicht war wie eine grimmige Maske. Er gab nicht zu erkennen, ob er Thorald erkannte. Aber es musste Ken Dorr sein, diese Narbe an seiner rechten Wange würde Thorald im Schlaf wiedererkennen.

„Das is‘ gut“, meinte der Hüne nun, „Solang‘ Will nicht in dieser verfluchten Festung Blackgarden steckt, ha’m wir ‘ne gute Chance, ihn da wieder ‘rauszuhauen.“

Robin nickte und tigerte auf der kleinen Brücke hin und her, während er sich am Kopf kratzte:

„Wenn Kennards Bericht stimmt, müssen wir uns beeilen. Irgendwie in die Burg schleichen, bevor Prinz John zu Will kommt – oder der Henker. Wir brauchen ein Seil.“

„Colin der Zimmermann könnte noch ein‘s bei sich haben.“

„Wollen wir die Dorfbewohner wirklich noch mehr in Gefahr bringen?“

„‘S‘is immerhin Will, um den’s hier geht.“

Thorald blickte zu Robin und dem Hünen, wie sie verschiedene Pläne besprachen. Wie gemeine Diebe wirkten sie nicht, erst recht nicht, als sie die Sicherheit der Dorfbewohner zu besprechen begannen und festlegten, wie viel Proviant sie den Hungernden verteilen konnten. Thorald schlich sich langsam rüber zu Ken. Dieser wirkte im Gegensatz zu den Geächteten wie ein waschechter Gauner, wie er mit einem schiefen Lächeln auf den Lippen mit einer Goldmünze umherspielte. Thorald lag immer noch ungut im Magen, was in Andor zwischen ihm und Ken vorgefallen war. Aber darum konnte er sich auch später noch kümmern. Im Moment war es einfach nur eine unglaubliche Erleichterung, ein bekanntes Gesicht zu sehen.

Ken versuchte eine Zeit lang, Thorald zu ignorieren, aber als dieser zu nahe an ihn getreten war, setzte er ein gequältes Lächeln auf uns sprach theatralisch:

„Kennard von Dorr ist mein Name. Fremder in einem fremden Land, wie du, so scheint es.“

Endlich fiel der Groschen bei Thorald, dass Ken nicht preisgeben wollte, dass sie sich bereits kannten, und er nickte theatralisch:

„Thorald ist der meine. Ummm... Ihr... Ihr stammt ebenfalls aus Andor?“

Ken verdrehte die Augen ob dem fehlenden Schauspieltalent, nickte aber brav.

„Ja, zu euch beiden und euren seltsamen Namen kommen wir noch“, wandte sich Robin ihnen beiden wieder zu, „Aber Will aus der Gefangenschaft zu befreien hat im Moment Priorität.“

Der Hüne nickte zustimmend.

Ein Rabe kam angeflogen und krähte zweimal.

„Nun denn, sieht so aus, als ob du die Wahrheit gesagt hast, Kennard. Will wurde ins Nottingham Castle gebracht“, nickte Robin, „John und ich machen uns auf den Weg dorthin. Kennard, kannst du dich um unseren Adligen kümmern, bis wir zurückkommen? Danach entscheiden wir, was wir mit ihm und seinem Gespür für versteckte Lager tun wollen.“

Ken schien kurz etwas erwidern zu wollen, nickte dann aber brav und legte Thorald possessiv eine Hand auf die Schulter. Thorald schauderte leicht.

Robin packte den Sack mit den mächtigen Schilden. Dann sprangen er und John von der Baumbrücke – es ertönten ein leiser und ein mächtiger Aufprall – und zogen davon. Ken wartete einige Augenblicke und wandte sich dann mit einem breiten Grinsen Thorald zu:

„Ich nehme alles zurück, was ich über seine Weisheit sagte. Es war töricht von ihm, uns zwei Fremde zusammen zurückzulassen.“

Es war, als würde eine große Last von Thoralds Schultern fallen. Das hier war Ken, wie er ihn kannte und liebte. Der weise Ken, der stets einen Plan hatte und die Lage zum Bessern wenden konnte.

„Nun, du hast ja auch so getan, als würden wir uns nicht kennen. Und sie haben gerade Wichtigeres im Kopf“, relativierte Thorald. Resigniert ergänzte er: „Wegen mir ist dieser Will in Gefangenschaft geraten und wird jetzt vor den Folterknecht und dann den Henker kommen. Ich habe wieder einmal alles falsch gemacht.“

„Machst du Witze?“, rief Ken, „Das war doch ihre Schuld. Sie dachten, dass du sie direkt zu ihrem Lager führst, obwohl du dieses noch nicht einmal gesehen hast. Allein aufgrund der Warnung eines uralten Waldverrückten! Wer aufgrund einer derart geringen Sachlage bereits zu einer Tat schreitet, ist ganz und gar selbst für deren Scheitern verantwortlich.“

Thorald war die Spitze in Kens Worten nicht entgangen, die sich nur allzu gut auf Reka übertragen ließ. Rekas Warnung bezüglich Ken fiel ihm wieder ein. Er ignorierte sie geflissentlich, wie es einem König gebührte, unterdrückte seine Schuldgefühle, wie er es schon so oft getan hatte, und stellte stattdessen eine ebenso dringende Frage: „Wie... wie hast du mich überhaupt gefunden?“

Ken gluckste: „Die Hufspuren im Schlamm vor der Quelle des Likko waren kaum zu übersehen. Du bist nicht wirklich geübt im Verstecken, oder? In der Höhle musste ich dann nur der Spur des tropfenden Pferds folgen und bin durch einen langen Gang zu diesem steinernen Gesicht gekommen. Hat eine Zeit lang gebraucht, bis ich den Ausgang öffnen konnte. Zum Glück hatte ich einen Casamatuc zur Hand.“

Ken Dorr schüttelte ein kleines Werkzeug aus seinem Ärmel und überreichte dieses Thorald zur Inspektion. Tatsächlich, das könnte ein Casamatuc sein! Das waren Zwergenwerkzeuge, mit denen man so gut wie jedes Schloss öffnen konnte. Sie waren relativ selten, Thorald hatte bislang nur in Erzählungen von ihnen gehört. Wie Ken an diesen Casamatuc gekommen war, wollte Thorald lieber nicht nachfragen.

Während Ken ihm fröhlich den Casamatuc zeigte, konnte Thorald nicht umhin, sich zu wundern, wie Ken so gelassen sein konnte. Der zornerfüllte Schemen, der Ken im Eingang zu Elgas Hütte gewesen war, schien vollkommen verschwunden. Oder zumindest unterdrückt. Thorald wollte lieber nicht zu lange darüber nachdenken, sondern führte stattdessen seine eigene Geschichte aus.

„Ich hatte natürlich keinen Allzweckschlüssel, doch bemerkte ich tatsächlich erst, dass das steinerne Gesicht sich verschließen kann, nachdem ich hindurchgetreten war. Der Mund scheint sich automatisch zu öffnen, sobald man einen Runenstein in seine Nähe bringt. Der ganze Gang war ja auch mithilfe von Runenmagie beleuchtet. Aber kaum war ich ein bisschen weitergezogen, wurde ich auch gleich von diesem Hünen, John, überfallen und dieser hat mir... mit... mit einem fiesen Trick die Schilde abgeknöpft. Wie... wie hast du dich bei den Geächteten eingeschlichen?“

Kens Blick trübte sich.

„Ach Thorald, ich habe dir doch schon einmal davon erzählt. Ich musste mich in meiner Kindheit in der Gasse rumschlagen, und habe da das eine oder andere Talent aufgegriffen. Den gefürchigen Dieb zu spielen, gehörte leider dazu. Dieser Will Scarlet ist wie ein Waldgeist vor mir aufgetaucht und wollte mich überfallen, aber da habe ich ihm gezeigt, dass man Ken Dorr nicht so einfach beraubt.“

Ken grinste kurz, dann wurde seine Miene aber wieder finster.

„Ein Glück, dass ich dich so schnell finden konnte. Ich werde nicht lügen, Thorald, dein irrsinniger Umgang mit den mächtigen Schilden hat mich ziemlich wütend gemacht. Aber jetzt müssen wir uns gerade auf Wichtigeres fokussieren. Dieses Steingesicht, durch welches wir hierher kamen, liegt nicht weit von hier. Wir können zurück nach Andor und mit der gesamten Rietgarde zurückkehren, um uns die Schilde zu erkämpfen.“

Der Gedanke war verlockend. Thorald vermisste das Rietland schon jetzt. Dort könnte er sich endlich wieder sicher fühlen. Saubere Kleider anziehen. Sich weder mit tyrannischen Herrschern noch Geächteten herumschlagen. Nur... nur mit Ken. Aber dann müsste Thorald auch mit leeren Händen zurückkehren, während die mächtigen Schilde in diesem unbekannten Land saßen und weiß Mutter Natur was anstellten. Und die Rietgarde von der Rietburg abzuziehen war ein Fehler, den Thorald nur einmal machen würde. Die erste Befreiung der Rietburg hatte genug Leben gekostet. Ganz zu schweigen davon, dass eine fremde Armee im Sherwood Forest den Bewohnern von Nottingham Castle sicherlich nicht gefallen würde, und ihre Wachen waren nicht schlecht ausgerüstet, nach dem wenigen, was Thorald im Castle hatte erkennen können. So viel Verstand von der Kriegskunst hatten ihm seine Lehrmeister eingebläut.

Dann schoss Thorald wieder das Bild durch den Kopf, wie dieser junge Geächtete, Will, wegen Thoralds Angriff strauchelte und von den roten Wachen zum grimmig grinsenden Guy von Gisbourne geschleift wurde. Nein, Thorald würde nicht fliehen. Ganz abgesehen von den mächtigen Schilde hatte er durch seine Taten Will dem Tode verurteilt. Dem unrechten Tode. Als König war es seine Pflicht, Unrecht entgegenzuwirken!

Thorald schüttelte seinen Kopf: „Ken, das Volk hier leidet unter einem tyrannischen Herrscher. Ohne Will fehlt den Geächteten ein wichtiger Streiter. Will wird wegen mir vor eine Mordmaschine kommen. So etwas habe ich meiner Lebtag nicht gesehen. Und im Gegensatz zu den Geächteten habe ich einen sicheren Zugang in die Burg. Nur ich kann ihm jetzt noch helfen.“

„Ist bei dir noch alles in Ordnung im Oberstübchen?“, spottete Ken, „Du willst mir doch nicht ernsthaft sagen, dass du dich jetzt um eine dahergelaufene Bande von Aufmüpfigen scherst! Du suchst doch nur wieder nach einer Möglichkeit, deinen Stolz zu bewahren! ‚Thorald, der gute und gerechte König‘. Diese Kutsche ist doch schon lange abgefahren. Hör auf, dir etwas vorzumachen, und komm mit mir zurück nach Andor, lass die Rietgarde dies handhaben!“

Thorald blieb trotzig: „Meine Meinung steht fest. Wir sind... ich bin Schuld daran, dass Will gefangen genommen wurde. Ein junger Mann ist das, ein guter Mann, der noch sein ganzes Leben vor sich gehabt hatte. Es ist meine Pflicht, ihm da wieder rauszuhelfen.“

Ken griff sich bloß an den Kopf und stöhnte.

„Na komm“, meinte Thorald, und klopfte Ken auf die Schulter, „Wir haben uns lange genug vom Schicksal herumschubsen lassen. Lass uns aktiv werden.“








\newpage
\section{Thorald tut etwas Mutiges}

Die Nacht war bereits lange angebrochen und ihre finsteren Schatten schmiegten sich an die Mauern von Nottingham Castle, als Thorald den Eingang erreichte.

„Man... man öffne die Tore! Ich bin entkommen!“, entsprangen die krächzenden Worte seinen aufgerissenen Lippen.

Eine Wache in ihrem typischen roten Gewand beugte sich über den Torbogen und verschwand dann gleich wieder dahinter. Thorald hört leises Getuschel von oben und versuchte, noch ein wenig mitleiderregender auszusehen.

„Ist das etwa dieser elende Bettler?“

„Bist du blind? Der ist doch viel älter.“

„Wir können doch nicht einem dahergelaufenen...“

Thorald erkannte, dass er seine Taktik ändern musste. Er richtete sich auf, so gut es ging, warf sich in Pose und ließ mit tönender Stimme verlauten:

„Mein Name ist Thorald vom Königreich Andor, und ich habe Informationen über den Aufenthaltsort der Geächteten! So lasset mich ein!“

Wieder beugten sich neugierige Wachen-Gesichter über den Torbogen, diesmal mehrere.

Schließlich hörte Thorald eine Stimme: „Holt den Sheriff!“

Er entspannte sich. So weit, so gut.\bigskip







Der Sheriff war ganz und gar nicht glücklich darüber, mitten in der Nacht geweckt worden zu sein. Er trug ein vollkommen rotes Nachthemd mit einer eleganten Mütze (Thorald merkte sich in Gedanken die Farbe, damit er in Andor selbst ein solches Gewand in Auftrag geben konnte) und rümpfte die Nase, als der mit Schlamm bedeckte Thorald in sein Arbeitszimmer spazierte. Seine Augen leuchteten aber rasch auf, als Thorald von einem fantastischen erfundenen Gefecht mit den Geächteten berichtete, an dessen Ende Thorald knapp mit seinem Leben davonkam – und nun eine noch genauere Vorstellung davon hatte, wo sich das geheime Lager der Geächteten befinden könnte. Gleich morgen in aller Herrgottsfrühe würde Guy von Gisbourne einen neuen Suchtrupp anführen, versprach der Sheriff.

Er packte zwei Karaffen Wein aus einer Truhe hervor und bot eine davon Thorald an.

Dieser lächelte: „Jetzt sprecht ihr definitiv meine Sprache!“\bigskip







Der Heiler von Nottingham Castle war auch nicht sonderlich glücklich darüber, aus dem Schlaf gerissen zu werden, aber wie könnte er dem Sheriff einen Wunsch abschlagen? Thorald wurde gründlich auf Wunden untersucht und schließlich durchgewinkt. Daraufhin ließ der Sheriff ihm ein Bad bereiten und ein frisches Kleidungsset auslegen. Endlich, endlich, konnte Thorald sich etwas entspannen und die Ereignisse des letzten Tages Revue passieren lassen. So kam es, dass er fast einnickte, ehe er sich an den wahren Grund erinnerte, weswegen er sich zurück ins Nottingham Castle geschlichen hatte. Dieser Grund saß wahrscheinlich gerade einige Stockwerke über ihm in einer klammen Zelle und bibberte dem Tod entgegen. Nun, hoffentlich nicht mehr lange.

Thorald bedankte sich ausschweifend beim Sheriff von Nottingham für das Bad und die Kleidung und fragte dann genauso schwungvoll nach einer Schlafgelegenheit. Der Sheriff gähnte todmüde und wies gedankenverloren auf einen kleinen Häuserkomplex: „Da drin wird man bestimmt angemessene Unterkunft für dich finden können.“

Thorald bezweifelte das, aber das war im Moment auch nicht wichtig. Er bewegte sich auf das Haus zu und ließ seinen Blick sorgfältig über die Burgmauern wandern. Die wenigen Wachen, die dort Wache hielten, beachteten das Innere der Burg nicht. Der Sheriff zog sich in seine Privatkammer zurück. Der Weg war frei.

Nottingham Castle war etwa von derselben Größe wie die Rietburg, aber die Architektur dieser Burg erschien ihm dennoch verwirrend. Thorald brauchte einige Versuche, um eine offene Tür in den großen Turm zu finden, in dem Ken den Gefangenentrakt vermutet hatte.

Schlimmstenfalls, falls er entdeckt wurde, könnte er sich auf Schlafwandeln berufen. Oder darauf, dass er nur den Abort suchte?

Glücklicherweise musste Thorald keine halbherzigen Ausreden einsetzen, denn keine Menschenseele kreuzte seinen Weg, bis er endlich vor einer schweren Eichentür stehen blieb, welche vielsagend mit einer massiven Eisenkette verschlossen war. Dahinter hörte Thorald ein leises Gemurmel. Er versuchte, einen Blick durch die Spalten zu werfen, aber es fiel ihm denkbar schwer.

Schlussendlich wagte er es, seine Stimme erklingen zu lassen, lehnte sich gegen die Tür und wisperte: „Wer ist da?“

Stille. Überraschte Stille? Thorald überlegte bereits, weiterzuziehen, als eine tiefe Antwort aus dem Innern erklang: „Hier sitzt Will Scarlet.“

Innerlich triumphierte Thorald. Äußerlich fischte er mit zitternden Fingern den Casamatuc aus seiner Tasche, das Zwergenwerkzeug, mit dem Ken durch das steinerne Gesicht in diese Welt hatte treten können.

„Halte still, Will, ich hole dich hier raus“, flüsterte Thorald durch die Zellentür. Ein Klick, zwei Klicks, beim dritten klemmte es, etwas Rütteln, ein Klack, eine Drehung, und schon hatte der Casamatuc sein Werk vollbracht und das schwere Eisenschloss fiel polternd zu Boden. Thorald vernahm ein Ächzen von der anderen Seite der Tür und dann ein heller Aufschrei: „Nein, nicht!“

Diese Stimme war viel höher gewesen als die tiefe Stimme, die ihm vorhin geantwortet hatte. Thorald musste sich nicht lange Gedanken über dieses Rätsel machen, denn schon wurde die Tür vor ihm aufgerissen und der König von Andor blickte dem Urheber der tiefen Stimme in die glühenden Augen. Das musste er sein: Prinz John, von dem der Sheriff gesprochen hatte. Der neue Herr von Nottingham Castle, mit einem blutigen Messer in der Hand, welches unschön gut zu zwei Schnitten im Gesicht des jungen Mannes passte, der sich am anderen Ende der Zelle ins Stroh kauerte.

„Nein, rette dich!“, rief der junge Mann mit der schwarzen Strubbelfrisur nun erneut. Er trug immer noch einen himmelblauen Umhang. Das war unzweifelhaft Will Scarlet, der Bogenschütze, für dessen Festnahme Thorald gesorgt hatte.

„Nun, das ist ja eine hübsche Überraschung“, grinste Prinz John süffisant, „Und... sind das etwa Kleider des Sheriffs? Ihr seid mir so manche Erklärung schuldig. Aber keine Sorge, die werdet Ihr mir liefern.“

Wie Prinz John da auf ihn zuschritt, mühelos mit dem kleinen Foltermesser in der Hand spielend, fühlte sich Thorald plötzlich an die Zeit als kleiner Junge erinnert, als er zum ersten Mal einem Troll gegenübergestanden war. Und ihm wurde schmerzlich bewusst, dass er wieder einmal keine Waffe bei sich trug.

Aber das hier war anders. Thorald war kein kleiner Junge mehr. Sein Gegenüber war kein Troll, sondern bloß ein bösartiger Mensch. Und Thorald brauchte keine Waffen, um sich zu verteidigen. Diesmal würde er nicht wegrennen.

So nahm Thorald all seinen Mut zusammen, hob seine Fäuste und schlug nach der Magengegend des Prinzen. Es war kein schöner Stil, und Harthalt wäre nicht stolz darauf gewesen, wie er seine Deckung vernachlässigte, aber es reichte, um den überraschten Prinzen keuchend zurückzudrängen. So waren manche Prinzen halt. Viel Worte, weniger dahinter.

„Renn, Will, renn!“, rief Thorald.

Will rappelte sich auf und grinste schwach: „Ich habe die letzten Stunden in dieser engen Zelle nicht umsonst Kräfte gespart. Lass dich nicht abhängen!“

Schon war Will an Thorald vorbei und aus der Tür geflitzt. Gute Güte, mit dieser Geschwindigkeit könnte der Bursche einem Skral einen ansehnlichen Wettkampf bieten!

Thorald war so beeindruckt, dass er beinahe selbst vergaß, die Beine in die Hand zu nehmen. Beinahe. Der hasserfüllte Blick Prinz Johns war am Ende doch genug, ihn aus seiner Starre zu reißen und Reißaus nehmen zu lassen.

Weiter vorne am Burgtor regten sich die Wachen, als sie erkannten, dass im Innern der Burg Spannenderes lief als außerhalb – erst recht, als Prinz John aus der Gefangenzelle zu taumeln kam und Zeter und Mordio schrie.

Will drehte sich abrupt um und rannte zurück, auf Thorald zu: „Gibt es einen Plan? Ist Robin hier?“

Thorald versuchte, wieder zu Atem zu kommen, und winkte ab. Wie lange war es nun schon her, dass er an einem Reitgefecht teilgenommen hatte? Wann war er so außer Kondition geraten?

Will ließ die Schultern sinken: „Es sieht nicht rosig aus. So einfach kommen wir nicht von der Burgmauer runter. Falls ich das überlebe, werde ich ein Lied auf deinen törichten Todesmut machen.“

„Törichter Todesmut ist richtig“, bekräftigte Thorald, „Ich lasse nicht zu, dass du meinetwegen umkommst.“

Will schaute ihn verdutzt an und schien sich unsicher, was er mit dieser Aussage anfangen sollte. Ehe er zu einer Entgegnung ansetzen konnte, packte Thorald ihn kurzerhand und warf sich gemeinsam mit ihm über die Brüstung der Burgmauer.

Der Schreck stand Will immer noch ins Gesicht geschrieben, als er sich hustend und prustend aus dem Heuwagen an der Mauer schälte. Ein Glück, dass Ken wie versprochen den Heuwagen hierhin schieben hatte können.

Will setzte an, das Weite zu suchen, doch Thorald hielt ihn zurück.

„Warte!“

„Was?! Wenn wir hier bleiben, dann werden wir gefunden.“

„Sie werden uns nicht sehen... hoffentlich. Sie werden annehmen, dass wir das Weite gesucht haben.“

„Aber früher oder später werden sie doch...“

„Psst. Sie kommen!“

„Was sollen wir tun?“

„Nichts. Warte.“

„Aber wir können nicht für immer im Verborgenen bleiben. Wir müssen uns offenbaren!“

„Nicht jetzt...!“

„Wann dann?“

„Nicht jetzt! Bleib im Schatten!“

So warteten Thorald und Will im Heuwagen, während sich die Tore von Nottingham Castle öffneten und Wachpatrouillen förmlich hinausströmten. Die wütende Stimme des Prinzen würde Thorald bestimmt noch lange im Ohr bleiben: „Findet mir die Geächteten! Und bringt mir diesen Thorald von Andor!“

Thorald von Andor saß indes im Heuwagen direkt unter der Nase des Prinzen und fühlte sich trotz seiner seltsamen Umgebung zum ersten Mal seit langem wieder wie ein König. Er hatte Ken getrotzt und war seinen eigenen Wünschen gefolgt. Er hatte Gutes getan und einen unschuldigen Mann aus den Fängen eines Tyrannen befreit. Und das fühlte sich verflixt gut an.\bigskip







Kurz vor Sonnenaufgang wagten Thorald und Will, den Heuwagen zu verlassen und sich in Richtung des Sherwood zu begeben. Dort wartete auch schon Ken im Schatten und beglückwünschte die beiden zur gelungenen Flucht. Will führte die beiden Andori auf eine Baumbrücke und die drei Fliehenden kletterten sorgsam durch die Baumkronen des Sherwood, während unter ihnen planlose Wachen die Waldpfade entlangschlichen.

Lange dauerte es nicht, bis sie auf Robin und Marian stießen. Die beiden unterhielten sich gerade über die Chance, dass im Laufe des nächsten Tages eine Kutsche durch den Sherwood Forest reisen würde.

„Länger als zwei Tage können wir nicht warten, sonst ist es definitiv um Will geschehen“, sprach Marian gerade, „Wir sollten uns auf die Suche nach einem Seil machen und die Kutschen Kutschen sein lassen.“

„Schon seit Tagen hat keine Kutsche mehr den Sherwood passiert, jetzt muss praktisch eine auftauchen. Wir kennen ihre Strecken und wir können sie abpassen“, entgegnete Robin, „Mit John an unserer Seite sollten wir den Kutscher problemlos überwältigen können. Das Dorf liegt weit weg und ein Seil zu finden könnte zu lange dauern. Vielleicht sind die Kutschen unsere letzte Hoffnung.“

Marian schüttelte ihren Kopf: „Du und John könnt die Kutschen abpassen. Ich schleiche mich lieber ins Dorf. Bei der aktuellen Flussströmung bin ich mit dem Boot so schnell bei Megan und Alex, da wird die Sonne noch nicht einmal aufgegangen sein. Nur um Gisi müssen wir uns Sorgen machen...“

„Verkleide dich. Oder nimm die Schilde mit“, schlug Robin vor und hantierte an einem Sack auf seinem Rücken herum, „Dieser himmelblaue Dreiecksschild scheint mir besonders wirkungsvoll. Ich kann nicht bestimmen, warum dies so ist, aber in seiner Präsenz fühle ich mich sicherer. Und nicht nur mich. Nenn mich abergläubisch, aber seitdem wir diese Schilde herumtragen, ist die Hoffnung im Land unnatürlich schnell gestiegen.“

Robin kniete sich nieder und öffnete den Sack. Unüberraschenderweise enthüllte er den Sternenschild und den Bruderschild, die beiden mächtigen Schilde, die Little John Thorald kurz nach seiner Ankunft in England abgeknüpft hatte.

„Wie gerufen“, lächelte Ken, und sprang zu den beiden hinüber. Thorald und Will taten es ihm gleich.

Thorald erkannte, wie die Gesichter von Robin und Marian rasch zahlreiche Emotionen durchliefen. Schrecken, als die drei Gestalten auf ihre Baumbrücke sprangen. Entspannung, als sie Ken und Thorald erkannten, kurz gefolgt von Überraschung, als der geschundene Will dazukam. Erleichterung, als sie schlussfolgerten, dass die beiden Andori Will auf eigene Faust befreit hatten. Dann wieder Schrecken und Wut. Moment mal, Schrecken und Wut?

Jetzt war es an Thorald, überrascht zu sein. Während sich Robin und Marian langsam und vorsichtig aufrichteten, blickte sich Thorald ebenso vorsichtig um und erkannte die Quelle ihres Zorns: Ken Dorr hatte sein Schwert gezückt und hielt es dem verletzten Will Scarlet an die Kehle.

„Beim Barte des Urtrolls, was tust du da?!“, zischte Thorald Ken zu. Ken ignorierte ihn und hielt Will weiterhin fest im Griff, während er das Wort an Robin und Marian richtete: „Geächtete! Wir haben euren Will für euch aus dem Castle befreit und werden euch nun für immer verlassen. Wir hätten aber vorher gerne unsere Schilde zurück. Wenn ihr also so freundlich wärt...“

„Ken, das ist doch Wahnsinn!“, sprach Thorald. Als Ken ihn weiterhin ignorierte, wandte sich Thorald stattdessen an Robin und Marian: „Lasst nicht zu, dass diese Situation hier eskaliert. Wir... wir wollen keinen Ärger. Ken meint das nicht so, er ist nur... nur ein sehr vorsichtiger Mensch. Bitte... gebt uns einfach die beiden Schilde und wir sind auf und davon und Will ist sicher.“

„Und unser Pferd wollen wir auch zurück!“, schnarrte Ken.

„Vergiss doch das verflixte Pferd!“, fauchte Thorald zurück. Flehend blickte er die beiden Geächteten an. Wie er befürchtet hatte, hatte Robin bereits seinen Bogen und Marian ihre Axt fest gepackt und starrten Ken hasserfüllt an. Ken deeskalierte die Angelegenheit nicht im Geringsten, als er sein Schwert Will noch stärker an den Hals drückte und „Die Schilde, aber schnell!“ rief. Will wimmerte.

Ein Glück, dass wenigstens Robin und Marian noch Verstand in der Birne hatten. Marian ließ ihre Axt fallen und nickte: „Natürlich, nehmt die beiden Schilde. Sie bedeuten uns nichts.“

„Na los, Thorald, greif sie dir!”, befahl Ken.

Robin blieb angespannt, griff aber nicht ein, als Thorald, die Hände beschwichtigend gehoben, nach vorne lief und die beiden mächtigen Schilde einpackte. Sein Herz schlug rasend schnell und er achtete darauf, keine hektischen Bewegungen zu machen. Marian tat es ihm gleich. Robin schien sich ganz und gar nicht zu bewegen, sondern hatte seine Augen starr auf Ken gerichtet.

Schließlich war die Schildübergabe abgeschlossen. Ken befahl den beiden Geächteten, ihre Waffen auf den Waldboden zu werfen und dann weit zurückzutreten. Robin und Marian handelten wie angewiesen. Da ließ Ken Will frei, stieß ihn nach vorne, sprang selbst auf den Waldboden und sauste davon. Thorald hatte sich im Castle etwas von seinen Anstrengungen erholen können, warf den drei Geächteten einen letzten entschuldigenden Blick zu und sprang dann selbst auf den Waldboden.

Keine Wachen in Sicht. Auch kein Guy von Gisbourne. Die mächtigen Schilde schleppte er auf dem Rücken. Der Weg nach Andor kannte er. Sein Pferd hatte er verloren und ein paar neue Feinde hatte er sich gemacht, aber abgesehen davon war nun alles bereit für die Rückkehr nach Andor.

So nahm Thorald seine Beine in die Hand und setzte Ken nach.\bigskip







Als Thorald in die Nähe des Wasserfalls trat, hinter dem ein steinernes Gesicht den Höhlengang nach Andor verschloss, erklang ein vertrautes leises Summen aus seiner Tasche. Er wühlte noch während des Rennens darin und sah mit Freude, wie der kleine grüne Runenstein immer heller zu leuchten begann, je näher sie dem Höhleneingang kamen.

Als Thorald dem Wasserfall den Runenstein entgegenhielt, knirschte es dahinter laut. Das dort liegende Gesicht hatte offenbar gerade seinen steinernen Mund geöffnet. Die beiden Andori traten raschen durchs kühle Nass in den Gang, der sie hoffentlich nach Andor zurückbringen konnte.

\az{Jahr 74}

Die Rückkehr verlief ziemlich ereignislos. Pferdelos trotteten der König und sein engster Ratgeber durch den schummrig erleuchteten Gang und ließen die Ereignisse des vergangenen Tages Revue passieren. Zumindest tat Thorald das, und vermutete, dass es Ken ähnlich ging. Immerhin kam es nicht jeden Tag vor, dass man in eine fremde Welt gelangte und dort ein buchstäbliches Beispiel eines tyrannischen Königs unter die Nase gehalten kriegte. Vielleicht hatte das Schicksal doch noch Pläne für Thorald und versuchte ihm so klarzumachen, dass es doch nicht das übelste wäre, die mächtigen Schilde für sich zu behalten? Immerhin schadete er seinem Volk nicht absichtlich. Wer konnte schon wissen, was Krams Nachfolger mit den mächtigen Schilden anstellen könnte, sollten etwa die „Wahren Schildzwerge“ Kontrolle über Cavern erringen. Andererseits konnte es auch nicht so weitergehen wie bislang, oder Ken würde Thorald mit Worten und den Schilden immer weiter manipulieren.

Wenn er also die mächtigen Schilde nicht aus seinem Leben entfernen würde, dann...

So traf Thorald schweren Herzens eine Entscheidung. Er brach das Schweigen und seine Stimme hallte durch den breiten Stollen: „Die mächtigen Schilde sind genau das, mächtig. Ich mag nicht geeignet sein zum König, aber immerhin bin ich nicht bösartig. Das wissen wir. Von den Zwergen wissen wir das nicht. Nicht auszudenken, was der Sternenschild in den falschen Händen anrichten könnte. Es ist meine Pflicht, sie bei mir zu behalten.“

Ken blickte ihn für einen Augenblick überrascht an und schien etwas entgegnen zu wollen, zuckte dann aber mit den Schultern und nickte.

Thorald fragte sich, was in seinem Kopf vorging. Aber er wusste besser, als nachzufragen – Ken vertrug solche Fragen nicht sonderlich gut. Thorald haderte selbst noch damit, die richtigen Worte zu finden, und so kam es, dass die beiden Andori bereits erschöpft durch die Höhle bei der Quelle des Likko gestrauchelt und aus dem großen Wasserfall auf der Andor-Seite des unterirdischen Gangs getreten waren, als Thorald auffiel, dass Ken ziemlich sicher noch auf eine Entschuldigung dafür wartete, dass Thorald mit den mächtigen Schilden vor ihm davongerannt war.

„Entschuldige, Ken“, murmelte er leise.

Ken lachte nur auf: „Du ziehst alleine aus der Rietburg los, reitest mir davon, bringst uns alle in diese Bredouille und denkst, eine einfache Entschuldigung würde das alles in Ordnung bringen?“

Thorald schüttelte seinen Kopf: „Nein, Ken. Ich... ich habe so lange versucht, mich in deinen Augen als würdig zu erweisen und du behandelst mich immer mehr wie der letzte Dreck. Nicht nur mich, sondern auch alle anderen. Was hast du dir nur dabei gedacht, die Geächteten zu bedrohen? Wir haben Glück, dass Robin dir nicht gleich zwei Pfeile in den Rücken gejagt hat. Und ich... ich bin deine Manipulationen und Ausbrüche einfach satt. Du bist ein gescheiter Ratgeber, aber... aber du bist kein guter Partner. Ich glaube, dass es... dass es mir guttun würde, mich zumindest eine Zeit lang von dir fernzuhalten. Und ich entschuldige mich dafür, dass ich das wohl tun werde.“

Ken packte Thorald an den Schultern und sah ihn mit großen Augen an: „Gute Güte, Thorald, woher kommen denn all diese Ideen auf einmal? Ich bin immer noch den treuster Ratgeber und deine beste Hilfe! Du brauchst mich!“

„Ich kann auch andere Ratgeber finden, die mich besser behandeln. Elga war das einmal, bis du ins Bild kamst. Und ich brauche dich erst recht nicht, um mich zu kontrollieren und meine Fehler im Schach zu halten. Ich habe Will befreit und das Einzige, was dabei schiefgelaufen ist, war dein brutales Eingreifen.“

Ein Anflug von Furcht flackerte in Kens Augen auf, wurde dann aber gleich wieder von Wut überdeckt: „Sei kein tumber Trottel von einem Troll, Thorald! Meinst du, eine einzige gute Tat würde all deine Schlampereien gutmachen und zukünftige Fehler abwehren? Ganz abgesehen davon konntest du Will nur mit meinem Casamatuc befreien, und nur weil ich den Karren mit dem Heu vor die Burg geschoben habe! Du brauchst mich! Deinen treuen Ratgeber, deinen Diener, deinen ergebenen Ken Dorr!“

Kens letzte Worte hatten schon beinahe trotzig geklungen.

Thorald stockte, fuhr dann aber fort: „Nein, Ken, es ist gerade andersherum: Du brauchst mich, Ken. Du bist von meiner Macht abhängig, und du hast zu lange versucht, mich zu kontrollieren, und mein Wohlbefinden aktiv missachtet. Nicht weiter. Ich ziehe jetzt zurück zur Rietburg, lagere die mächtigen Schilde dort und bleibe gerade so lange im Amt, bis die Helden aus dem Norden zurückkehren und ich ihnen das Amt übertragen kann. Ich habe viele Fehler gemacht und vielen das Leben gekostet. Aber diese Entscheidung ist kein Fehler. Manche... manche Leute sind einfach nicht für das Königsamt gemacht. Ich mag einer davon sein. Du bist das aber ebenfalls.“

Thorald blieb einen kurzen Moment ergriffen stehen und genoss das Licht der andorischen Sonne auf seinem Gesicht. Die Luft roch einfach anders hier. In einiger Ferne sah er Elgas und Junas Bauernhof am Fuße des Grauen Gebirges stehen. Wahrscheinlich lag die Rietgraskrone immer noch irgendwo dort herum. Thorald ließ seinen Blick über die nebelumwobene Narne hinweg zur Rietburg schweifen. Lieber direkt zur Burg zurückkehren und die Schilde in Sicherheit bringen. Der König von Andor würde von einem Abenteuer zurückkehren und zufrieden sein. Die Rietgraskrone konnte er immer noch später abholen. Bei Elga und Juna war sie wahrscheinlich sogar sicherer als in der Burg.

„Ich habe Freunde in der Burg, viele Freunde“, zischte Ken und unterbrach so Thoralds Gedanken, „Ich habe in meiner Zeit als Soldat des Königshauses Seite an Seite mit vielen Männern der Rietgarde gekämpft, während du über ihren Köpfen Wein gesoffen hast! Viele waren erzürnt über meine Verbannung durch das Hause Brandurs, das du repräsentierst!“

Kens Stimme hatte einen finsteren Unterton angenommen.

„Was meinst du, Thorald, auf wessen Seite sich die Garde stellen würde, sollte ich mich öffentlich gegen dich stellen? Du willst es nicht darauf ankommen lassen!“

Thorald verwarf die Drohung: „Du willst das doch ebenso wenig. Die Schildzwerge würden sich nur liebend gerne unter einem Vorwand auf unsere mächtigen Schilde stürzen. Eine Revolution käme ihnen nur allzu gerne recht. Und die Helden... die Helden würden dich nicht auf dem Thron stehen lassen. Das ist immer noch ihr Land.“

„Du begehst einen großen Fehler, Thorald!“

„Ich ziehe von hinnen, Ken. Es ist vorbei.“

Thorald wandte sich energisch von Ken ab und stapfte in Richtung der Bogenbrücke, die beiden mächtigen Schilde auf seinem Rücken tragend. Er fühlte sich... erleichtert, irgendwie. Als wäre ein großes Gewicht von seinen Schultern genommen worden. Der Sternenschild auf seinem Rücken summte beruhigend. Alles war gut.








\newpage
\section{Epilog}

„Du brauchst mich!“, rief Ken Thorald ein letztes Mal hinterher. Thoralds Gestalt wurde immer kleiner, während der Prinz davonschritt und seinen Ken nicht weiter beachtete. Ken stampfte mit seinem Fuß auf, schritt ihm aber nicht hinterher. Er durfte nun nicht planlos handeln. Sein Kopf ratterte, während sein Blick noch lange auf seinem König verblieb.

Schritte.

„Da seid ihr ja endlich!“, knurrte Ken seine beiden Handlanger aus der Rietgarde an, welche keuchend angerannt kamen, „Ihr beiden habt ja was verpasst.“

„Wo seid Ihr denn gewesen, Herr?“, frage der eine.

„War das der König, den ich soeben abstürmen gesehen habe?“, fragte der andere.

Ken Kiefer zitterte, als er die Worte ausstieß: „Ich... ich werde ihm nachgehen. Ich werde ihn wieder umstimmen. Ich kann das. Er... er muss auf mich hören, alleine ist er der Krone nie und nimmer gewachsen... er begeht gerade den größten Fehler seines Lebens... und wenn er das bald nicht eingesehen hat, dann... dann..., wenn nicht. Aber er wird es tun.“

Ken knurrte weiter vor sich hin, und seine beiden Handlanger tauschten nervöse Blicke aus, blieben aber stumm. Als Ken das Wort wieder ergriff, triefte seine Stimme nur so von Zorn und Bosheit.

„Thorald darf auf keinen Fall die Rietburg erreichen und die Stadtwache gegen mich aufbringen. Wir müssen ihn irgendwie ablenken, seine Wut auf etwas anderes weisen. Elga und Juna, diese Bäuerinnen am Fuße des Gebirges, deren Gesellschaft Thorald der meinen so sehr vorzuziehen scheint – sie haben die Rietgraskrone noch. Holt sie euch und bringt die Krone zur Burg. Sollten den Bäuerinnen unterwegs ein... Unfall zustoßen, dann wäre das natürlich ganz tragisch.“

Die leere Blick seiner Handlanger klärte sich. Mit solchen Worten wussten sie schon besser umzugehen.

„Nun stellt euch vor, dass jemand mit einer blutigen Krone ins Rietland zurückkehrte... da könnte der Prinz ja auf die Idee kommen, dass die räudigen Wölfe des Südens Elga und Juna erwischt hätten. Er, der stets nach großen Taten dürstet, würde seinen Hass auf die Wölfe lenken, wieder auf meinen Rat hören und diesem folgend zur Jagd blasen lassen. Diese sollte ihn von mir ablenken und wieder etwas in die Realität zurückholen. Danach würde er hoffentlich wieder auf meinen Rat hören. Und falls er sich weiter gegen mich wehrte... Nun, wenn ein König nicht weiß, was gut für sein Land ist, wenn er eine Gefahr für sich selbst und seine Gemeinschaft anstellt, dann muss man sich seiner entledigen. Ich hatte gehofft, dass es nie dazu kommen würde, aber ich bin ein Realist. Ich werde euch entsprechende Anweisungen zukommen lassen. Falls niemand in der Burg von meinem Zerwürfnis mit Thorald erführe und sein Dahinscheiden natürlich wirkte, so würde keiner in der Burg es wagen, meine Position als Statthalter von Andor in Frage zu stellen. Aber zunächst einmal gilt es, den König auf eine Wolfsjagd zu hetzen. Los, los, ab jetzt. Auf zu den Bäuerinnen!“

Kens Handlanger sahen einander ein letztes Mal fragend an und zogen dann hastig von hinnen. Ken lehnte sich zurück und sog tief Atem ein. Ja, Andor roch einfach anders als dieser seltsame Wald, in dem er und Thorald gelandet waren. Dieser verfluchte Wald in diesem verfluchten fremden Land! Er würde noch einmal zur Höhle reisen müssen und den Zugang zerstören. Einstürzen lassen Zur Gänze vernichten. Nicht auszudenken, was geschähe, wenn die Armee dieses anderen Königs in Andor einfallen könnte.

Ken schüttelte sich und brach auf. Auf, zurück nach Andor. Auf, zurück zur Rietburg. Er musste seine Karten jetzt gut ausspielen, um am Ende noch obenauf zu enden. Aber ja, alles konnte noch gut kommen.\bigskip







Eine kurze Zeit lang lag die Wiese neben der Quelle des Likko still da. Aber verlassen war sie nicht. Denn sobald Ken außer Hör- und Sichtweite getrabt war, löste sich eine grün gewandete Gestalt aus dem Schatten und sah sich vorsichtig um, den Bogen stets im Anschlag.

Robin von Loksley hatte nicht alles verstanden, was dieser Kennard von Dorr mit seinen Handlangern besprochen hatte. Aber er hatte genug gehört, um zu wissen, dass Bauern in Gefahr waren. Und wie könnte er sich guten Herzens einen Helden und Beschützer der Armen nennen, wenn er nicht Menschen in Not beistehen würde, egal, aus welcher Welt sie stammten? In einer war Robin ja schon ein Geächteter. Zeit, es auch in einer zweiten zu werden. Vielleicht würde dies ja endlich seinen Albträumen ein Ende bereiten.

So wandte sich Robin um und winkte zur Quelle des Flusses zurück. Prompt schlichen seine Gefährten aus dem Schatten hervor und schlossen sich Robin an. Gemeinsam folgte die Gruppe im Geheimen Kennards Handlangern in Richtung Bauernhof.

Robin zählte fünf Pfeile in seinem Köcher. Er lächelte.

Mehr würde er auch nicht brauchen.









\part{Aufzeichnungen analogen Autors aus allerlei alternativen Andorversen}


\begin{chapterbox}
    \chapter{Andor-Märchen (2019)}

    Hier begann mein Pfad zur Veröffentlichung andorischer Fanfiction: An einem warmen Sommerabend, an dem mich ein Kommentar der Schlafenden Katze dazu anspornte, das Märchen vom gestiefelten Kater andorisch zu verpacken. Bald darauf folgten einige weitere andorisch umgedichtete Sagen, die in ihrer Gesamtheit eine kleine Märchensammlung bilden. Von gescheiten Streifenmardern, den ersten Taren, Schatten finsterer Krahderfürsten und Arpachen hütenden Jungdrachen.
\end{chapterbox}

\extramarks{}{ }

\section{Der gestiefelte Streifenmarder}


Es war einmal – zu einer Zeit, als Werftheim noch von einem Dunklen Magier regiert wurde – ein Werftheimer, der sein Brot damit verdiente, Schiffe zu bauen. Dieser alte Herr war mit drei Söhnen gesegnet worden, und als er verstarb, vermachte er seinem ältesten Sohn seinen kleinen Werftbetrieb, welchen dieser weiterführen sollte. Seinem zweitältesten Sohn vererbte er seinen kleinen Kahn, und dieser zog damit in die weite Welt hinaus. Dem jüngsten Sohn hingegen hinterliess er nichts weiter als den kleinen Streifenmarder, welchen er als Haustier gehalten hatte.

Der jüngste Sohn war enttäuscht, sah keinen weiteren Zweck im Streifenmarder und war schon kurz davor, ihn zu braten und für einige Goldstücke zu verkaufen. Doch der Streifenmarder sprach zu ihm: „Wenn du mich leben lässt, und mir deine Stiefel schenkst, damit ich die weite Welt bereisen kann, so sollst du reicher werden, als du es dir je hättest träumen können.“ Und der jüngste Sohn schenkte dem Streifenmarder seine Stiefel, woraufhin dieser von dannen zog.

Der Streifenmarder war geschwind in seinen neuen Stiefeln unterwegs, sammelte Muscheln an den Küsten von Werftheim und brachte diese zu den Silberzwergen im Norden, um sie gegen gutes Gold einzutauschen. Ein Teil dieses Goldes brachte wiederum er dem Seekönig von Varatanien „als grosszügiges Geschenk des edlen Fürsten von Werftheim“. Der Seekönig von Varatanien hatte noch nicht von einem solchen Fürsten gehört, doch da er gerade dabei war, einen geeigneten Gatten für die Seeprinzessin von Varatanien zu finden, liess er ein Schiff in See stechen, um den Fürsten von Werftheim kennenzulernen.

Indes war der gestiefelte Streifenmarder bereits flink wie der Wind nach Werftheim zurückgekehrt, und verteilte den Rest seines Geldes unter den Bewohnern, unter der Bedingung, dass sie dem Seekönig erzählen würden, dass alle ihre Häuser dem „Fürsten von Werftheim“ untertan wären. Dann lief er zum jüngsten Sohn und befahl ihm, seine dreckigen Kleider auszuziehen und sich splitternackt in die tosende See zu werfen. Der jüngste Sohn tat, wie ihm aufgetragen.

Als nun der Seekönig von Varatan sich Werftheim näherte, trat der gestiefelte Streifenmarder zu ihm und sprach: „Seht, böse Piraten haben das edle Schiff des Fürsten gestohlen, ihm alle Kleider geraubt und ihn über Bord geworfen. Dort schwimmt er!“ Und der nackte jüngste Bruder wurde an Bord des Schiffes des Seekönigs gebracht und mit den edelsten seeköniglichen Kleidern ausgestattet.

Doch der Seekönig fragte die Anwohner von Werftheim: „Wem gehören all diese Häuser hier?“ Die Werftheimer antworteten: „Sie sind alle im Besitz des edlen Fürsten von Werftheim.“ Und nun glaubte der Seekönig dem Streifenmarder, dass der jüngste Bruder der Fürst von Werftheim sei.

Der dunkle Magier aber, der Anspruch auf die Insel Werftheim erhob, mochte es gar nicht, dass jemand sich als Fürst seines Herrschaftsgebiets ausgab, und er stützte herab wie ein Raubvogel, um dem gestiefelten Streifenmarder ein Ende zu bereiten. Der gescheite Streifenmarder forderte ihn heraus und spottete: „Du behauptest, der mächtigste Dunkle Magier weit und breit zu sein, doch kannst du dich nicht einmal in einen Würfel verwandeln.“ „Kann ich wohl“, spieh der Dunkle Magier wütend zurück, und verwandelte sich als Demonstration seiner Macht in einen kleinen Würfel. Doch das war ein Fehler gewesen: Der gestiefelte Streifenmarder zögerte keine Sekunde: Er stibitzte den Würfel, und der Dunkle Magier war nicht mehr.

Der Seekönig von Werftheim vermählte daraufhin den jüngsten Sohn als Fürsten von Werftheim mit seiner Tochter, der Seeprinzessin, und als der Seekönig starb, wurde der jüngste Sohn der nächste Seekönig von Varatanien. Und der gestiefelte Streifenmarder ward sein engster Ratgeber. Die Moral von der Geschicht – die kenn’ ich wirklich nicht. Doch wenn sie alle nicht gestorben sind, so leben sie noch heute.




\newpage
\section{Die Schöne und der Tarus}


Vor langer, langer Zeit war das Lande Andor primär von Trollen bevölkert. Die Rietburg war noch nicht gebaut, die Taverne zum Trunkenen Troll stand noch nicht, der Trunkene Troll selbst war höchstwahrscheinlich noch nicht mal geboren. Vereinzelt und versteckt konnte man Bauernkaten sehen, oftmals in der Nähe der Narne oder im Schatten von Bäumen, auf dass man sich im Falle eines Trollüberfalls rasch in Sicherheit bringen konnte.

Dazumals lebte im Lande ein armer Bauer namens Marus, der hatte sechs Töchter. Waren die älteren fünf Töchter alle selbstverliebt und gemein, so war die jüngste und ihm liebste Tochter grosszügig und nett. Als Marus eines Tages mit seinem kleinen Schiffchen in die nordische See stechen wollte, um einen Teil seiner Ernte in Varatanien zu verkaufen, baten ihn alle Töchter um Mitbringsel. Die älteren fünf Töchter baten um teure Perlen, Golddublonen, Silberketten, bestes Holz aus dem Wachsamen Wald, gar Nixenstaub. Der Vater versprach ihnen all dies. Die jüngste Tochter jedoch, bescheiden, wie sie war, bat Marus bloss um eine Sturmrose.

Das Boot des Vaters geriet in einen Sturm und kenterte. Marus verlor all seine Ladung und strandete auf einer ihm bislang unbekannten felsigen Insel östlich von Varatanien. Nun konnte er seinen Töchtern keine Geschenke mehr kaufen, geschweige denn überhaupt einen Gewinn aus den untergegangenen Vorräten an Winterweizen ziehen. Existenzängste überkamen ihn.

Als der Vater sich so auf der Insel umsah, erkannte er, dass in deren Mitte, geschützt vor der tosenden See durch die hohen Klippen, ein kleines Tal lag, augenscheinlich unbewohnt, da komplett überwuchert. Und da, in der Mitte des kleinen Tälchens, erkannten die geübten Augen Marus’ eindeutig einen einzelnen Blumenstrauch, welcher aus einem Findling spross. Eine Sturmrose!

Kaum hatte der arme Marus sich in Bewegung gesetzt, um diese Pflanze zu pflücken und so wenigstens den Wunsch seiner jüngsten Tochter zu erfüllen, da sprang von einer Felswand ein mächtiges Wesen herunter. Es überragten den doch nicht allzu kleinen Bauern um mindestens zwei Köpfe, hatte gebogene, lange Hörner und war mit nichts als einem Lendenschurz bekleidet. Ein Tarus!

Ihr müsst wissen, damals waren Taren noch nicht als die liebevollen, friedfertigen Naturliebhaber bekannt, als die wir sie heute zu schätzen wissen. Das gemeine Volke und selbst die Bewahrer vom Baum der Lieder hatten nie von einem Tarus gehört, und dieses Exemplar hier war mehr Tier denn Mensch, hatte es doch so lange in diesem Tale festgesessen. Es trat vor den Bauern und sprach: „Du Unwürdiger! Wolltest du Dieb doch glatt das letzte bisschen Schönheit aus diesem mutterseelenverlassenen Tal entfernen, und mich so für alle Ewigkeiten verdammen!“

Und schon hätte er sich auf den völlig überrumpelten Marus gestürzt, hätte dieser nicht beschwichtigend seine Arme in die Luft geworfen und ausgerufen: „Es war in keinster Weise mein Ziel, diesem Tal oder Euch einen Schaden zuzufügen. Ich bin bloss in des Schicksals Ungnade gefallen, wie es mir scheint, und darf nun wohl nicht mal mehr den letzten Wunsch meiner jüngsten Tochter erfüllen – denn bloss ihretwillen wollte ich diese Sturmrose pflücken.“

Da horchte der Tarus auf, und bot dem Vater an, ihn wieder gehen zu lassen, wenn dieser im Gegenzug dafür seine liebste Tochter an seiner Stelle auf die Insel senden würde, sobald sie die Volljährigkeit erreichen würde, auf dass sie die Insel nie mehr verlassen sollte. Der Vater weinte und haderte, doch wusste er, dass seine Töchter ohne ihn nicht durchhalten würden. Da willigte er ein, und der Tarus überreichte Marus ein neues, besseres Schiff, sowie genug Gold für Vorräte für einen gesamten Winter, nein, sogar zwei ganze Winter!

Und Marus kehrte nach Andor zurück, und als seine jüngste Tochter volljährig geworden war, trennte er sich schweren Herzens von ihr und sandte sie zum Tarus auf seiner geheimnisvollen Insel, abgeschottet vom Rest der Welt. Natürlich litt die Tochter darunter, nicht mehr bei ihrer Familie zu sein, doch im Gegensatz zu ihren Schwestern stellte sich der Tarus als gar nicht gemein heraus, das Essen, welches der Tarus bereitete, war köstlicher als alles, was sie je gekostet hatte, die Umgebung war wunderschön, und wie es sich herausstellte, hatte der Tarus Zugang zu einer überragenden Sammlung Danwarer Schriften, sodass die jüngste Tochter ganze Mondzyklen damit verbringen konnte, über die umliegende Welt zu lernen. Bloss verlassen durfte sie die Insel nicht, da war der Tarus ganz strikt.

Jede Nacht träumte die jüngste Tochter von einem wunderschönen Prinzen, der sie innigst bat, ihm zu helfen. Die Tochter glaubte, dieser Prinz müsse wohl ein anderer Gefangener auf dieser Insel sein, und beschloss, ihn zu finden. Und doch war ihre Suche erfolglos. Indes aber näherten sie und der Tarus sich langsam an, und als sie den Tarus bat, sie wenigstens einmal, zur Zeit des grossen andorischen Erntefests, ihre Familie besuchen zu lassen, willigte er ein. Er schenkte ihr sogar ein neues Boot und zusätzliche Nahrung, warnte sie aber: „Kehre unbedingt innert eines Mondzyklus zurück, oder ich werde vergehen.“

Doch als die jüngste Tochter fröhlich zu ihrer Familie zurückkehrte, erblickte sie Neid in den Gesichtern ihrer Geschwister. Sie hatte ein vergleichsweise gutes Leben auf der Insel gehabt, bemerkte sie nun. Gutes Essen, spannende Lektüre, angenehme Gesellschaft. Und als der Mondzyklus beinahe vorüber war, hatte sie in einer Nacht einen anderen Traum, einen Traum vom Tarus, welcher sie innigst bat, zu ihm zurückzukehren, oder er würde sterben.

Und die jüngste Tochter erkannte, dass sie Gefühle für den Tarus empfand, und teleportierte sich mithilfe eines magischen Spiegels vom Volke der Takuri zurück ins Tal des Tarus. Und dieser erzählte ihr seine Geschichte:

Der Prinz aus ihren Träumen war wahrhaftig er! Einst war er ein Mensch von den Nebelinseln gewesen, doch hatte er keinen Respekt vor den Naturgeistern gezeigt. Gar viele Sturmrosen hatte er von dieser Insel ausgerissen, und so hatten die Wiesengeister ihn verflucht. Er war gewachsen, seine Haut hatte sich verdunkelt, und ihm waren seine langen Hörner gewachsen. Zunächst hatte er sich gefreut, stärker, schneller, ausdauernder zu sein. Aber dann hatten ihm die Naturgeister erklärt, dass er von nun an ein Teil dieser Insel sein musste, sie nicht mehr verlassen durfte, bis jemand ihn wahrhaft lieben würde – und der erlösende Kuss musste stattfinden, ehe die letzte Sturmrose auf der Insel verdorrte.

Da blickte die jüngste Tochter auf die Sturmrose, und sie war verdorrt, sie war zu spät! Der Tarus war nun für immer auf dieser Insel gefangen! Bitterlich weinten sie beide, die jüngste Tochter gestand dem Tarus ihre Gefühle, und sie küssten sich. Da ging eine Veränderung in der jüngsten Tochter vor, und auch sie wuchs, ihre Haut verhärtete sich und Hörner brauchen aus ihren Schläfen hervor. Sie ward zu einem weiblichen Ebenbild des Tarus!

Die beiden schickten Falken an Marus und dessen Familie und boten ihnen an, das trollverseuchte Land zu verlassen und zu ihnen auf die Sturminsel zu ziehen. Und als der alte Vater Marus und die fünf anderen Töchter die Insel betraten, wurden auch sie von den Wiesengeistern in Taren verwandelt. Gleich erging es allen Menschen, die über die nächsten Jahre auf der Insel stranden sollten und länger als einige Tage dort blieben. Langsam wuchs eine richtige Gesellschaft dort heran, alle gemeinsam wartend darauf, dass eine weitere Sturmrose wachsen sollte, da sie erst dann die Insel endlich wieder verlassen dürften.

Der Legende nach ist dies die Geschichte, wie die ersten Taren entstanden und warum sie in ihrem Sturmtal so zurückgezogen vom Rest der Welt leben. Und wer’s nicht glaubt, bezahlt `nen Goldtaler. Oder noch besser, gibt `ne Runde aus!





\newpage
\section{Der Schatten}



Hier fängt die Erzählung an. Sie berichtet von einem uralten Krahderfürsten, welcher mit starker Hand über sein Reich herrschte. Alle noch so kleinen Dörfer in der Gegend waren ihm untertan und selbst über das Wetter gebot er – oder so munkelte man zumindest.

Die menschlichen Bewohner der umliegenden Städte nannten sich Ambacus, die Unfreien, da der Krahderfürst und die seinen ihnen vorgaben, wie sie zu leben hatten, sich ihrer Ernten, Besitztümer und Arbeitskraft bedienten und ihnen verbaten, auch nur davon zu träumen, in bessere Lande zu fliehen. Was konnten die Ambacus schon dagegen tun? Sie waren schwach und ausgehungert, viele, aber nicht organisiert, und die Krahder waren mächtig und gewissenlos – eine gefährliche Kombination.

Doch diese Erzählung soll nicht von den Ambacus berichten, sondern von ebenjenem Krahderfürsten, welcher das Oberhaupt des Landes war und alleinigen Anspruch auf die Knechtschaft der Ambacus erhob.

Es begab sich, dass sich dieser Krahderfürst eines Tages auf seinem steinernen Thron fläzte und es sich gut ergehen liess. Seine Bediensteten brachten ihm gerade sein viertes Stück Fleisch auf einem Silbertablett, als sein Blick auf seinen eigenen Schatten fiel, den das flackernde Fackellicht auf die Felswand warf. Der Krahderfürst bewegte seinen Körper und der Schatten folgte jeder seiner Bewegungen. Das enervierte den Fürsten. Seine Macht konnte nicht angetastet werden, jede Menschen-, Zwergen- und Drachenseele in- und ausserhalb seines Reichs fürchtete ihn, und doch war da sein Schatten, welcher auf demselben Thorn sass, dieselben Handlungen durchführte und dieselbe Autorität ausstrahlte. Und er konnte sich drehen und wenden, wie er wollte, der Schatten würde stets mit ihm verbunden bleiben. Der Krahderfürst sah seinen Schatten nun als Rivalen an und befahl ihm, von dannen ziehen.

Am nächsten Morgen war sein Schatten verschwunden.\bigskip



Jahrzehnte darauf liess es sich der schattenlose Krahderfürst erneut gut ergehen, als eine Gestalt an das Burgtor seiner Festung klopfte. Es war der Schatten, welchen der Krahderfürst vor so langer Zeit vertrieben hatte. Er war klein und schwach geworden in der Zeit, und so amüsierte sich der Krahderfürst in Überlegenheit, und liess ihn ein.

Der Krahderfürst hatte sich dem Bösen, dem Hässlichen, den Lügen verschrieben, und sein ganzes Reich war voll davon. Doch der Schatten berichtete ihm von anderen Welten, die er besucht hatte, Lande voller Schönheit und Unversehrtheit, Lande, in denen nicht das Recht des Stärkeren wirkte. In diesen Landen war der Dunkle Schatten des Krahderfürsten eingegangen, nunmehin schwach und ein Bruchteil seiner einstigen Stärke.

Doch im Reich des Krahderfürsten vermochte der Dunkle Schatten sich wieder mit Dunkelheit und Bosheit aufzutanken, zu wachsen, und im Verlaufe seines Aufenthalts in der Festung seines ehemaligen Herrn gedieh er auf seine ursprüngliche Grösse.

Und es begab sich, dass der Krahderfürst in dem Masse schwächer und dünner wurde, in welchem sein ehemaliger Schatten wieder wuchs und erstarkte. Nun fürchtete sich der Krahderfürst, denn würde er nicht seine momentane Stärke beibehalten können, so dürfte er bald schon mit Überfällen aus den umliegenden Krahderreichen rechnen, oder noch schlimmer, mit einem Aufstand der ihm Unterlegenen.

So schlug der Schatten ihm vor, selbst in die umliegenden Lande auszureisen, Unheil und Verderben über sie zu bringen, und sich so wieder zu kräftigen. Die einzige Bedingung war, dass der Schatten den Krahderfürsten spielen durfte, während der Krahderfürst selbst den Schatten spielen sollte. Der Krahderfürst willigte ein.\bigskip



Derart reisten der Krahderfürst und sein Schatten durch seine verdorbenen Lande und darüber hinaus, der Schatten spielte den Fürsten und der Fürst spielte seinen Schatten. So kamen sie eines Tages an eine von den Krahdern eingenommene Zwergenfestung, so hoch, dass ihre höchsten Zinnen selbst im wärmsten Sommer von Schnee und Eis bedeckt waren. In dieser Festung lebte eine unvermählte Krahderkaiserin, und als der Krahderfürst sie erblickte, verfiel er ihr.

Er und sein Schatten reisten nicht mehr weiter durch die Lande, sondern verblieben am Hof der Krahderkaiserin und machten ihr denselbigen, während der Fürst weiterhin seinen Schatten spielte und der Schatten den Fürsten. Und wie die Zeit so verging, so verging auch dem Krahderfürsten die Lust darauf, nur einen Schatten zu spielen. Er wollte selbst mit dieser Krahderkaiserin sprechen und sie umwerben, nicht bloss Zuschauer der Handlungen seines Schattens sein.

Da tat der Krahderfürst zum ersten Mal in seinem langen Leben etwas durchaus Gewagtes, nahm all seinen Mut zusammen und schlich sich des Nachts, während sein Schatten schlief, aus dessen Zimmer, um als sich selbst, den mächtigen Krahderfürsten, vor die Krahderkaiserin zu treten. Doch o Schreck! – als der Krahderfürst mit pochendem Herzen an den Wachtrollen der Kaiserin vorbeistapfte und an die Tür ihres Schlafgemachs klopfte, öffnete ihm sein eigener Schatten die Tür. Sein Schatten war ihm zuvorgekommen und stand nun grösser und mächtiger denn je vor ihm!

Der Schatten sprach zu der Krahderkaiserin:

„Sieh doch nur, wie schrecklich! Das Gehirn eines Schattens ist wahrlich leicht zu verwirren – glaubt doch mein Schatten nun tatsächlich, er sei mich, und ich sei sein Schatten!“

„Wie grauenvoll“, sprach die Krahderkaiserin mitleidig, „Und es gibt nichts, was wir tun könnten, um ihn vom Gegenteil zu überzeugen?“

„Nichts“, nickte der scheinheilige Schatten traurig, „Wir können es bloss auf uns nehmen, ihn von seinem verwirrten Elend zu erlösen.“\bigskip



Nun ist wichtig, zu erkennen, dass zwei mögliche Enden zu dieser Erzählung im Umlauf sind. In der Version, die die Krahder von ihren Sklaven den zukünftigen Königen ihres Reichs erzählen lassen, bäumt sich der Krahderfürst auf, erwürgt seinen zwielichtigen Schatten mit blossen Händen und erwählt die Krahderkaiserin zu seiner Gemahlin, welche ihm noch so gerne in sein düsteres Reich zurückfolgt, wo die beiden bis zu diesem Tage immer noch nicht gestorben sind, und somit noch heute glücklich leben.

In einigen Städten der Ambacus allerdings, wenn keiner der Krahder in der Nähe war, flüsterten die Mütter ihren Kindern ein ganz anderes Ende zu:\bigskip



Der Krahderfürst erschrak, als er die bösen Worte seines ehemaligen Schattens hörte. Er wollte nicht sterben! Auf die Knie vor seinem Dunklen Schatten fiel er, und gab unter Tränen zu, sich geirrt zu haben. Er sei wahrlich der Schatten, und der andere, der vor ihm stand, sei der wahre Krahderfürst. Möge man ihn doch bitte verschonen, so würde er tunlichst seiner Pflicht als Schatten Folge leisten und nie wieder aufbegehren.

Der Schatten grinste, und der Krahderfürst legte sich nieder und spielte erneut den Schatten, gedemütigt und verzweifelt, doch blieb ihm keine andere Wahl.

Der Schatten und die Krahderkaiserin vermählten sich bald darauf und kehrten glücklich in das düstere Reich des Krahderfürsten zurück, welches nun das Reich des Schatten war. Und der Fürst würde bis ans Ende seines Lebens den Schatten seines Schattens spielen müssen. Und wenn er nicht gestorben ist, so ist er heute noch geknechtet.\bigskip



Gute Nacht!




\newpage
\section{Einköpfchen, Zweiköpfchen und Dreiköpfchen}



Es waren einmal drei Drachengeschwister, die von Mutter Natur ganz besonders gezeichnet worden waren, denn sie alle trugen ein goldenes Schuppenkleid, was selbst unter den uralten Drachen des Grauen Gebirges eine Seltenheit war.



Da war Einköpfchen, das jüngste der dreien, auf dessen schlankem Schlangenhals ein einzelner güldener Drachenkopf thronte. Einköpfchen mochte die Gesellschaft der Bergzwerge und Höhlenwichte und liebte es, diese Wesen beim Umherwuseln auf und unter den Bergen zu beobachten. Beim Baden im See in der Tiefe der Krahal-Schlucht wagte es sich jeweils bloss Schritt für Schritt, ganz langsam, ins kalte Wasser. So eines war Einköpfchen.



Da war Zweiköpfchen, das mittlere der dreien, aus dessen robustem Reptilrumpf gleich zwei Hälse und zwei Drachenköpfe entsprangen. Zweiköpfchen mochte die Lüfte, das Fliegen durch die weissen Wolken und das Spüren des Windes auf seiner Haut. Es sprang meistens gleich übermütig und elegant in die eiskalten Tiefen des Sees am Grunde der Krahal-Schlucht. So eines war Zweiköpfchen.



Da war Dreiköpfchen, das älteste der dreien, dessen – wie der Name gewiss bereits vermuten liess – königlicher Körper von gleich drei mächtigen Drachenköpfen gekrönt war. Dreiköpfchen mochte die Flammen, die Glut, das Dahinschmelzen von Erde, Pflanzen, Stein und selbst Metall unter dem Einfluss des ewigen Drachenfeuers. Es spie jeweils einen mächtigen Flammenstrahl auf den See in der Tiefe der Krahal-Schlucht, bis die Wasseroberfläche zu brodeln begann. Erst dann begab es sich in das nun erhitzte Wasser und genoss dessen Wärme. So eines war Dreiköpfchen.



Nun begab es sich, dass Zweiköpfchen und Dreiköpfchen signifikant mehr Köpfe hatten als der Duchschnittsdrache, und sich in der Drachengesellschaft oft als Aussenseiter vorkamen. Und da muss man leider sagen, dass sie, anstatt zu erkennen, was für Vorteile ihre zusätzlichen Köpfe ihnen bringen könnte, ihre Frustration oftmals am Einköpfchen ausliessen. Sie stibitzten dem den grössten Teil seines Futters, und wenn jemand die Hausarpache hüten musste (ihr müsst wissen, dass die Drachen stets gerne einige dieser Rieseninsekten oberirdisch als Haustiere hielten, um deren Sekret zu ernten), so schoben sie diese Aufgabe auch immer auf das hungrige Einköpfchen.



Zum Glück musste Einköpfchen nicht lange hungrig bleiben, denn eines Tages trat eine alte Agren zu ihm und verriet: „So höre! Sprichst du ‚Arpache, weg, Tischlein, deck’, so wird dir ein unerschöpflicher Gabentisch voller Futter erscheinen, und sprichst du ‚Arpache, meck, Tischlein weg’, so wird der Tisch wieder verschwinden.“ Einköpfchen dankte der weisen Agren und siehe da – kaum dachte es die magischen Worte, verwandelte sich die Arpache in einen riesigen Tisch, der sich buchstäblich krümmte unter der Last der auf ihm ruhenden Speisen. Nun, in der Drachengesellschaft nutzte man eigentlich keine Tische, aber dass die Gerichte darauf essbar waren, wäre jedem Drachen sofort klargewesen. So konnte Einköpfchen sich jeweils beim Hüten der Hausarpache den Bauch vollschlagen, und vor der Rückkehr am Abend die Zauberformel sprechen und aus dem Tisch würde wieder die unscheinbare Hausarpache werden.



Zweiköpfchen und Dreiköpfchen entging nicht, dass Einköpfchen gar keinen Hunger mehr zeigte, und so folgte Zweiköpfchen Einköpfchen oft aufs Feld, um herauszufinden, woher Einköpfchen denn Nahrung erlangte. Doch Einköpfchen war listig, und sang ein altes Drachenschlaflied, und Zweiköpfchens beide Köpfe schlossen ihre Augen und ruhten. Und während Zweiköpfchen so schlief, konnte Einköpfchen ungestört vor sich hin schlemmen.



Die Tage zogen ins Land, und an einem stürmischen Tag bemerkte Einköpfchen mal wieder, dass es beim Hüten beobachtet wurde, weswegen es wieder das uralte Drachenschlaflied anstimmte. Doch oh weh! Es war Gewitterzeit, und Zweiköpfchen war irgendwo weit entfernt am Himmel am Umhersegeln, den Sturm geniessend. Nicht Zwei-, sondern Dreiköpfchen hatte Einköpfchen beobachtet, und nur zwei seiner drei Köpfe waren eingeschlafen! Der letzte Kopf beobachtete das ganze Geschehen, und als Dreiköpfchen den riesigen Gabentisch sah, an dem Einköpfchen sich gütlich tat, wurde es ganz grün vor Neid, flog in, und erschlug die Hausarpache im Jähzorn.



Da weinte Einköpfchen bitterliche Feuertränen, denn es wollte nicht wieder Hunger leiden müssen. Doch die alte Agren trat wieder zu ihr und riet ihr „Begrabe die Überreste der Arpache vor deinem Bau“. Gesagt, getan, da wuchs ein riesiger goldener Mammutbaum über Nacht vor Einköpfchens Bau, und nur Einköpfchen konnte dessen Früchte ernten, da sich seine Äste von allen anderen Drachen wegbogen, auch vor seinen Geschwistern.



So geschah es, dass ein edler Drache aus den Ostlanden durch das Graue Gebirge streifte und sein Interesse durch den goldenen Baum geweckt wurde. Deswegen trat er vor Einköpfchens Bau und verlangte zu wissen, wem denn dieser Baum gehöre. Zweiköpfchen und Dreiköpfchen versuchten sich aufzuplustern und behaupteten, der Baum gehöre ihnen. Doch als der edle Drache ihnen befahl, ihm eine goldene Frucht vom Baum zu pflücken, bog der mächtige Baum sich vor ihnen weg und sie konnten ihn nicht erreichen. „Äusserst skurril“, schmunzelte der edle Drache. Erst Einköpfchen konnte auf den Baum klettern und ein goldene Frucht abbrechen, ohne dass sich der Baumwipfel wehrte, und so bot ihm der edle Drache an, dass Einköpfchen ihn fortan auf seinen Reisen begleiten möge, und die beiden zogen gemeinsam fort und lebten glücklich bis an ihr Lebensende.



Doch dies soll nicht heissen, dass Einköpfchen seine Geschwister nie mehr wiedersah. Denn es trug sich einige Sonnenumrundungen später zu, dass Einköpfchen glücklich durch die Lüfte flog, als es zwei traurige Drachengestalten weiter unten am Boden erkannte. Das waren Zweiköpfchen und Dreiköpfchen, und alle fünf Köpfe, die sie gemeinsam hatten, liessen sie hängen, denn sie litten nun selbst unter Hunger und bereuten, wie sie dem Einköpfchen übel mitgespielt hatten vor so langer Zeit. Und Einköpfchen, die gute Seele, die es war, verspürte keinen Groll mehr und hiess die beiden mit offenen Pranken willkommen, auf dass die drei Geschwister wieder vereint wären und gemeinsam die Welt bereisen konnten.\\bigskip



Nun könnte man natürlich behaupten, diese Geschichte hätte sich irgendein betrunkener Bauer einst einfach so aus den Fingern gesaugt. Doch wenn ihr aus dem Fenster der Taverne in die weite Ferne schaut, erkennt ihr manchmal im Lichte der untergehenden Sonne und des aufgehenden roten Mondes ein leichtes goldenes Glitzern aus einem Wald am Hang des Grauen Gebirges. Und ich garantiere euch, so wahr ich hier sitze – dieses Glitzern stammt von Einköpfchens goldenem Mammutbaum, der da immer noch steht, und immer noch goldene Früchte trägt, und diese noch immer wacker gegen alle Wesen verteidigt, die diese Früchte zu ernten versuchen könnten.

Und da er nicht gestorben ist, lebt er auch heute noch.



\begin{chapterbox}
    \chapter{Heldenhafte Kurzgeschichten}

    \az{diverse Jahre}

    In meiner Zeit im Andor-Forum werkelte ich an viel zu vielen Fan-Held*innen. Da das Geschichtenschreiben mich ebenfalls äußerst vergnügt, entstanden zu manchen dieser Heldenideen kurze Storytexte, welche teils gar aufeinander Bezug nahmen. Hier sind die längsten davon versammelt – und als Zückerchen kommt am Ende mit den Geschenken vom Santa Gor ein bisschen Weihnachtsstimmung auf.
\end{chapterbox}


\section{Lynnra \& Kaleido (2019)}

\az{Jahr 64}

„Sind wir bald da? Ich mag’ einfach nicht mehr weiterziehen!“, exklamierte Kaleido und liess sich demonstrativ ins noch mit Morgentau besetzte Gras fallen. Dort am kalten Boden schien es ihr allerdings auch nicht besser zu gefallen, und deswegen rappelte sich die nun relativ durchnässte Kaleido rasch wieder auf, klopfte sich etwas Staub von den glitzernden Kleidern und blickte zu Lynnra, um zu sehen, ob ihre Aktion wahrgenommen worden war.



Lynnra hatte sich zu Kaleido umgedreht und konnte ein Grinsen nicht unterdrücken. Bei Kaleidos kindlichem Verhalten fiel es ihr nur allzu leicht, zu vergessen, dass sie es hier mit einem uralten Naturgeist zu tun hatte. Oder vielleicht war Kaleido auch eine verwunschene Fee? Irgendetwas in der Art. Lynnra hatte sich noch nie sonderlich gut mit den alten Mythen ausgekannt, und für sie war Kaleidos Klassifikation nie wichtig gewesen.



Als Lynnra Kaleido zum ersten Mal getroffen hatte, hatte Lynnra noch einer stolzen Barbarensippe angehört und Kaleido war noch als golden glänzende Gaswolke in Erscheinung getreten. Reflexartig versuchte Lynnra, die schmerzhaften Gedanken an diese unbesonnene, bessere Zeit zu verdrängen. Ob Kaleido sich überhaupt noch daran erinnerte? Heute war von Lynnras Sippe nur noch sie selbst übrig, alle anderen verschleppt oder ermordet vom Riesenvolk aus dem Süden mit dessen furchtbarer Armee der Toten. Und Kaleido hatte im Verlauf ihrer gemeinsamen Zeit die Gestalt einer unternährten Barbarendame angenommen – sie war im Aussehen praktisch zu Lynnras Spiegelbild geworden, nur stets noch umgeben von einem in allen Regenbogenfarben schimmernden Dunst.



Beim Angriff der Krahder hatte Lynnra sich versteckt gehabt – dafür schämte sie sich heute noch – und war von da an praktisch als Eremitin durch die geschundenen Lande gewandelt, auf der Suche nach Erlegbarem, mit Kaleido als einziger Begleiterin. Die harten Umstände hatten sie ausgezehrt, doch die Blösse, den anderen Sippen ins westliche Feindesland zu folgen und Hilfe zu erbitten, würde sie sich erst geben, wenn wahrlich kein anderer Ausweg mehr in Sicht war.



Dieser Zeitpunkt war nun eingetreten. Eine Fäulnis hatte einen nicht gut gepflegten tiefen Schnitt an ihrem Arm befallen, den sie sich bei einem Zank mit einem grossen grauen Troll um eine mickrige verirrte Ziege zugezogen hatte. Gegen so eine faulende Wunde konnte bloss eine weise Kräuterkundige helfen, und so eine war Lynnra wahrlich nicht. Einst hatte eine vorbeireisende Hexe den Kindern in ihrer Sippe die wichtigsten Heilkräuter zu lehren versucht. Viele hatten ihr an den Lippen gehangen, als sie einen kleinen Jungen von einem bösen Fieber geheilt, und währenddessen alle möglichen Perlen der Weisheit mit der Schar geteilt hatte. Doch Lynnra hatte nicht zugehört. Sie hatte in dieser Zeit eine Flöte aus dem Besitz der Hexe entwendet. Die Klänge, die sie diesem Instrument hatte entlocken können, waren so viel reiner gewesen als alle Töne, die sie mit den Hörner der Barbaren zustande gebracht hatte. Und das Wissen um die Kräuter hätte sie ohnehin nicht behalten. Oder zumindest redete sie sich dies nun ein.



Während Lynnra andächtig in ihren Erinnerungen geschwelgt hatte, war Kaleidos Aufmerksamkeit bereits weitergewandert und an einer kleinen verdorrten Staude Sternkraut hängengeblieben. Auf ihrem Gesicht zeichnete sich Besorgnis ab, und sie nahm die Überreste der Pflanze in ihre Hände. Ein goldener Schimmer umgab sie für einen kurzen Moment, und als Kaleido ihre Hände wieder öffnete, blühte das Sternkraut in voller Pracht und glänzte in allen Regenbogenfarben. Kaleido klatschte enthusiastisch, was Lynnra aus ihrer Nostalgie riss.



Lynnra staunte aufs Neue: Kaleido hatte einer toten Pflanze neues Leben eingehaucht. Schon war es wieder unvorstellbar, Kaleido als irgendetwas Anderes als das wundersame Mysterium anzusehen, das sie war. Wo das Limit ihrer Fähigkeiten lag, wusste Lynnra nicht. Sie hatte Kaleido schon fröhlich Wanzen wiederbeleben, Futterreste zu ganzen Rationen wandeln und düstere Albträume für immer stoppen sehen. Andererseits hatte Kaleido sich geweigert, auch nur eine Schürfung an Lynnras Knie zu heilen, ganz zu schweigen von ihrem faulenden rechten Arm.



Und es war nicht so, als hätte Kaleido Lynnras Bitte damals nicht verstanden – Lynnra vermutete, dass Kaleido zu jedem Zeitpunkt komplett durchschauen konnte, was in Lynnra vorging, trat sie doch von Zeit zu Zeit mit den passendsten Ratschlägen auf den Plan. In anderen Momenten lebte Kaleido aber auch in ihrer eigenen Welt und scherte sich nicht gross um Lynnras Anliegen oder Anwesenheit. Andererseits schien sie sich auch oft Lynnra unterzuordnen. So auch jetzt, wo sie sich einige Schritte weiterschleppte, um dann flehend zu Lynnra aufzublicken und zu fragen: „Ist es noch weit? Können wir eine Pause machen?“



Kaleido grinste schelmisch. Ihr war sehr wohl bewusst, wie sie sich aufspielte. Was war nur ihr Ziel? Aber Lynnra konnte ihr einfach nicht böse sein. Ein Seufzer entfloh ihr, als sie sich wieder ihrer magischen Begleiterin zuwandte. „Schau, dort oben, siehst du diese Hügelkuppe? Dorthin wollen wir, und dann sehen wir das ganze weite Westland. Dann können wir auch erspähen, wohin es die anderen Sippen verschlagen hat.“ Mit einem leisen Plopp verschwand Kaleido in einer Wolke Goldstaub, um Sekunden darauf oben auf der Hügelkuppe wieder zu erscheinen, auf die Lynnra gezeigt hatte. Lynnra stiess ein erstauntes Geräusch aus, und machte sich auf, die letzten rund fünfhundert Schritte zwischen sich und der Hügelkuppe zu Fuss zu überwinden. Kaleido würde sie wohl immer wieder mit neuen Fähigkeiten überraschen. Das Pochen in ihrem Arm liess nach.



Und dann war er da, der Moment, in dem Lynnra sich oben auf dem Hügel ins Gras sinken lassen konnte – ihr verletzter Arm protestierte kurz – und der Sonnenaufgang das Lande Andor in seiner vollen Pracht vor sich liegend bestrahlte. Die Narne glänzte im hellen Licht, umgeben von dichten Nebelschwaden, die Baumkronen des Wachsamen Waldes lagen in Wurfweite, und weit in der Ferne war etwas von Menschenhand Geschaffenes zu erblicken – eine Burg zweifelsohne. Nun, die würden Lynnra und Kaleido tunlichst zu meiden gedenken. Barbarenhütten waren keine in Sichtweite, und eine Hexe war bei dieser Entfernung erst recht nicht zu erkennen, das wäre ja auch zu leicht gewesen. Aber Herausforderungen war Lynnra gewohnt, und mit Kaleido an ihrer Seite vertraute sie auch darauf, diese Herausforderungen zu überwältigen. Nun war es an der Zeit, sich zur Ruhe zu legen. In der Nacht würden sie weiterziehen. Auf, ins geheimnisvolle fremde Land Andor!









\newpage
\section{Najuktaris Erwachen (2022)}


\az{Jahr 65}

Najuk sah die weiße Gestalt, die sich aus den nebelverhangenen Bergen des Fahlen Gebirges löste. Eine Wolke aus Schnee und Eis wirbelte um sie herum, und es schien, als würde sie darauf schweben. Die Erscheinung trug das Antlitz einer Frau. Sie hatte etwas Faszinierendes an sich, und Najuk konnte sich ihrem Bann kaum entziehen. Sie strahlte eine solche Eiseskälte aus, dass er spürte, wie seine Glieder langsam erstarrten. Bewegungslos sah er die durchscheinend anmutende Gestalt auf sich zukommen, und das Letzte was er hörte, bevor ihn die Sinne verließen, war: „Ich bin Siantari. In den tiefen Schluchten des Kuolema scheint niemals die Sonne auf das ewige Eis, und dort ist mein Reich.“

Die nächsten Stunden – oder waren es Tage? – nahm Najuk kaum wahr. Hin und wieder blubberte sein Bewusstsein gerade lange genug herauf, damit Najuk die beißende Kälte um seinen Körper bemerken konnte. Dann umfing ihn immer wieder der gnädige Mantel der Ungewissheit.

Zumindest so lange, bis laute Schreie wie aus weiter Ferne an sein Ohr drangen:

„Wo ist sie hin?“

„Morar hat sie gerade noch hinter diesen riesigen Felsen gesehen!“

„Krah, ich kann für mich selbst sprechen! Sie hat sich ins Zentrum des Sturms verkrochen. Krah! Könnt ihr nicht einfach ihrer Eisspur folgen?“

„Können vor Lachen, Schnee und Eis liegen hier überall gleichmäßig knietief!“

„Zwergenknie zählen nicht!“

Eine eiskalte Handfläche klatschte auf Najuks Brust.

Er öffnete seine Augen und blinzelte, als gleißendes Licht hineinschien. Es wurde etwas erträglicher, als eine Gestalt vor das grelle Leuchten trat. Najuk erblickte das ausdruckslose Gesicht Siantaris, welches sich ihm näherte. Die Herrin des Kuolema-Gebirges sah angeschlagen aus. Eines ihrer Geweihe war abgebrochen und sie schleppte sich auf ihn zu, als ob sie nicht einmal mehr die Kraft hätte, aufrecht zu stehen. Es blieb gerade genug Zeit für Najuk, um festzustellen, dass das umliegende Rietgras völlig verschneit wirkte, ja gar er selbst bis über die Hüfte eingeschneit worden war. Dann presste Siantari auch ihre zweite Hand auf seine Brust und sein Gesichtsfeld verschwamm.

Als nächstes nahm Najuktari wahr, wie er neben dem Geschehen stand und auf Siantari und seinen eigenen bewusstlosen Körper niederblickte. Die Kälte war noch da, doch biss sie ihn nicht mehr. Eine eigenartige Ruhe erfüllte ihn. Er wusste nicht, wie ihm das kommuniziert worden war, doch er verstand nun, dass dies schon immer der Plan gewesen war. Der Dämon des ewigen Eises musste fortbestehen, und dafür musste Siantari Najuk zu ihrem Nachfolger machen. Nur indem sie seinen Körper in diesen Zustand zwischen Tod und Leben brachte, konnte Najuks Geist aus seinem Leib gerissen und vom Eisdämonen Tari erfüllt werden. Und nur so würde Najuktari später seinen menschlichen Körper im ewigen Eis versenken können, damit sein Geist noch für Jahrhunderte den des Eisdämons nähren konnte, während er den langjährigen Kampf gegen die Wärme fortführte. Es war gut so. Es musste so sein. Das Schicksal wollte es so.

Najuktari blickte an sich herunter und sah, wie sich aus plötzlich umherwirbelndem Schnee ein neuer Körper um seinen Geist herum zu formen begann und stetig opaker wurde. Es schien, als ob er dem Eis entwuchs. Siantari blickte ihn an, ein schwaches und doch triumphierendes Lächeln auf ihrem Gesicht. Dann schloss sie ihre Augen und ihr Körper sank langsam in den vereisten Untergrund ein.

Plötzlich blubberte ein letzter Rest Widerstand in Najuktari auf. Eigentlich hatte er sein Leben lang die kalte Jahreszeit nicht gemocht. Er war ein leidenschaftlicher Bäcker, kein Herold des ewigen Eises! Und seine Familie ... er mochte nicht einmal daran denken, was seine Familie denken würde, wenn er nicht mehr nach Hause zurückkäme. Beim Herz der heiligen Mutter, sie mussten sich bereits jetzt schon solche Sorgen machen!

Najuktari ballte die noch leicht durchscheinenden Schneehände seines neuen Körpers zu Fäusten. Es fühlte sich ungewohnt an, aber nicht unangenehm. Nach einem letzten Blick auf seinen komatösen menschlichen Körper rannte er los. Er hatte doch vorhin rufende Stimmen vernommen.

„Zu Hilfe! Man helfe mir!“, rief er, so laut er konnte. Was erheblich lauter als sonst zu sein schien. Seine Stimme klang klirrend und scholl durch den Schneesturm.

Siantari schlug ihre Augen auf und riss sich wieder aus dem Eisboden los. Mit verärgerter Miene hechtete sie nach Najuktari und bekam knapp seine Ferse zu fassen. Najuktari stürzte in den Schnee und wurde, umgeben von kalter Dunkelheit, erneut von innerer Ruhe erfüllt. Was für einen Sinn gab es schon, sich einem Dämon zu widersetzen? Das ewige Eis würde unweigerlich kommen, und nur in einer Rolle als Herr des Eises würde er die kommende Eiszeit überdauern können.

Da schoss ein Lebewesen hinter einem nahe liegenden Felsen hervor, ein rothaariger Mann in grüner Waldläufer-Kleidung. Ohne lange zu fackeln, zückte der Neuankömmling ein Messer, stürzte sich auf Siantari und riss die Herrin des Kuolema-Gebirges von Najuktaris Schneekörper fort.

„Ich hab sie! Hierher!“

Eine weitere, stämmigere Person tauchte hinter dem Felsen auf, und diese war Najuktari gar wohlbekannt. Das war Kram, ein Held von Andor. Kram hielt eine große Axt fest im Griff, die, wie Najutarik überrascht feststellte, komplett vereist war. Er trat einen Schritt vor und machte seine Axt zum Schwung bereit. Die Edelsteine im tulgorischen Knochenhelm auf seinem Kopf begannen, rötlich zu schimmern. Er fühlte sich offensichtlich bedroht durch die Anwesenheit der beiden Gestalten aus Eis vor ihm.

Ruhig und beschwichtigend hob Najutarik seine Hände und sprach: 

„Ich hege keinen Zwist mit euch, Helden von Andor, ganz im Gegenteil, als Andori und erst recht als Geretteter bin ich euch zu ewigem Danke verpflichtet. Mein Name ist Najukt... Najuk, und mein eigentlicher Leib liegt euch zu Füßen. Die Macht eines Eis-Dämons scheint meinen Geist in diesem gefrorenen Gefäß zu halten, obschon der Eis-Dämon und seine finsteren Absichten noch immer bei unserem Vorgänger, der Herrin des Kuolema, zu verweilen scheinen.“

„Na, na, plapper‘ nicht so verschwurbelt, oder wir werden noch ganz misstrauisch“, brummelte Kram mit einem breiten Grinsen. Nach einem Kontrollblick, dass Siantari keine akute Gefahr mehr darstellte, fügte er an: „Und, wie kriegen wir deinen Geist jetzt wieder in deinen Körper hinein?“

Einiges erfolgloses Pröbeln folgte, doch am Ende genügte es, Najuktari auf einem Stückchen zerriebenen Heilkrauts herumkauen zu lassen. Urplötzlich fühlte Najuktari ein Ziehen zurück in seinen leblos daliegenden Menschenkörper, seine Schneegestalt verging in einem Wirbel aus Schneeflocken und Najuk öffnete wieder seine leiblichen Augen. 

Nachdem er sich überschwänglich bei den Helden für seine Rettung bedankt und ihnen lebenslänglich Gratisbrote aus seiner Bäckerei versprochen hatte (Fenn hatte begeistert angenommen, Kram zumindest zu Beginn freundlich protestiert und ein dritter, plötzlich aufgetauchter Held namens Hogo äußerst eloquent abgelehnt), kehrte Najuk so geschwind er nur konnte nach Hause zurück. Seine Familie war wie befürchtet außer sich gewesen vor Sorge um ihn und hatte schon das halbe Rietland nach ihm abgesucht. Entsprechend überglücklich waren sie über seine Rückkehr, aber auch erschreckt ob seiner Schilderung der gruseligen Begegnung mit einer Eis-Dämonin. Als Najuk an diesem Abend mit seinen Liebsten das Brot brach, schienen die erschreckenden Erlebnisse der vergangenen Tage bereits kaum mehr als eine intensive Erinnerung. 

Das lag alles nun hinter ihm, so dachte er.

Erst als Najuk im kommenden Winter vom einem höllischen Fieber erwischt wurde und Najuktari plötzlich wieder in einem Schneekörper vor seinem Krankenbett stand, erkannte er, wie falsch er damit gelegen hatte.







\newpage
\section{Sians Entscheidung (2020)}

\az{Jahr 65}

Sian öffnete ihre Augen und blinzelte, als gleissendes Licht hineinschien. Es wurde etwas erträglicher, als eine Gestalt vor das grelle Leuchten trat. Sian erkannte das forsche Gesicht eines stämmigen Zwergs, der sie finster musterte. Er hielt eine grosse Axt fest im Griff, die, wie Sian überrascht feststellte, komplett vereist war. Sie wich seinem Blick aus, um ihn nicht zu provozieren. Hinter dem Zwerg stand ein grösserer Mensch mit einem breiten Grinsen und einem roten Haarschopf. Dieser sprach: „Das muss jetzt bestimmt furchtbar verwirrend für Euch sein. Verzeiht mir, mein Messer habe ich noch in Euren Bein gelassen, weil ich nicht wusste, ob das Entfernen...“

Er brach seinen Satz ab, als Sian gelassen an ihren Unterschenkel griff und das kleine Messer ohne grosse Umstände aus ihrem Bein zog. Es schmerzte nicht einmal, und die Wunde war bereits vereist. Wer dieser Mann wohl war und warum er ihr Bein erstochen hatte?

Jetzt drängte eine dritte Person nach vorne, die bislang im Schatten geblieben war und sich weiterhin in einen schwarzen Umhang hüllte. „Fenn, vielleicht überlässt du das Reden lieber mir.“

Vor Sian tretend, musterte dieser Neuankömmling sie von Kopf bis Fuss. Er versuchte offensichtlich, gelassen zu wirken, hielt aber seine Waffe – eine seltsame Art Kurzbogen, mit einer raffinierten Mechanik zum Spannen der Sehne ausgestattet – stets schussbereit an seiner Seite. Immer mehr Gedanken rasten durch Sians Bewusstsein. Das letzte, woran sie sich klar erinnern konnte, war die Reise in den Kuolema gewesen, danach waren ihre Eindrücke nur noch schwammig und unvollständig. Viele Jahre der angenehmen Einsamkeit im Eis und Schnee des Kuolema, nur unterbrochen durch die eine oder andere Begegnung mit einer Bergziege, einem Hornbären oder einem törichten Tulgori. Jemanden hatte sie passieren lassen, die anderen waren Teil ihrer vereisten Sammlung geworden. Wie viel Zeit war seither bloss vergangen?

Sichtlich zufrieden mit sich selbst, sprach die Gestalt mit der schwarzen Kapuze nun: „Deinen Dämon sollte ich dir damit ausgetrieben haben, aber das heisst nicht, dass du dich wieder vollkommen zu einem Menschen zurücktransformiert hättest.“

In eine graue, zerknitterte Schriftrolle blickend, führte der Fremde aus: „Solltest du Unwohlsein verspüren oder sollten sich deine Bewegungen langsam versteifen, so wäre das ein Indiz, dass dein Körper mit den Überresten der Eismagie des Dämons nicht klarkommen kann – melde dich dann bitte umgehend wieder bei mir!“

Der Mann mit dem roten Haupthaar und den breiten Grinsen – Fenn – meldete sich auch wieder zu Wort. Ihn mochte Sian instinktiv weniger. Er schien einer zu sein, der zu oft sprach, wenn er schweigen sollte. Ohnehin wollte sie momentan nichts weiter, als in Ruhe denken zu können.

“Und zu denken, dass ich dich einst ausliefern wollte, Arbon“, sprach Fenn weiter, „Du machst dich wirklich gut in einem Team!“ 

„Fenn! Das ist doch doch jetzt bestimmt das zehnte Mal, dass ich dich bitte, mich vor anderen Personen ‚Hogo‘ zu nennen!“, schnauzte der Mann im dunklen Umhang zurück. 

Fenn erwiderte schnippisch: „Wenn du deine Identität wirklich beschützen wolltest, solltest du deine Bewahrerkleidung so schnell wie möglich loswerden. Und früher oder später musst du dich Melkart ohnehin stellen, das weisst du. Auf ewig wirst du ohnehin nicht versteckt bleiben.“

Arbon brummelte etwas in seinen Umhang und war dann still. Die Stille tat Sian gut, half ihr, sich zu fokussieren. Sie betastete sorgfältig die kleinen Hörner an ihrer Schläfe. Diese waren offenbar geblieben von ihrem Dasein als eisige Herrin des Kuolema. Was hatte sich sonst an ihr verändert? 

Noch immer fühlte sie keine Gefühle in ihrem Herzen (und einen Herzschlag hatte sie auch nicht mehr) aber das Verlangen, welches sie früher in den Kuolema gezogen hatte, war nicht mehr. 

Sie war auch nicht mehr von Schnee und Eis umgeben, sondern lag auf trockenem Gras. Goldenes Gras. Dieses gab es in Tulgor nicht. Wo befanden sie sich? Oder war einfach so viel Zeit vergangen, dass sich eine ganz neue Vegetation in Tulgor verbreitet hatte?

Vorsichtig zog Sian ihre Füsse näher an ihren Körper und machte Anstalten, sich aufzurichten.

Dies brachte Bewegung in die restliche Gruppe: 

Fenn schnellte nach vorne, um ihren Arm zu stützen – was auch gut war, da ihr Bein immer noch von seinem Messerstich verletzt war und gleich wieder unter ihr nachgab. 

Der Zwerg, der sich bislang im Hintergrund gehalten hatte, trat einen Schritt vor und machte seine Axt zum Schwung bereit. Die Edelsteine im tulgorischen Knochenhelm auf seinem Kopf begannen, rötlich zu schimmern. Er fühlte sich offensichtlich bedroht durch die Anwesenheit Sians.

Einzig und allein Arbon bewegte sich nicht, sondern versteifte sich vielmehr. Doch Sian konnte er nicht täuschen: Wenn sie die Gruppe tatsächlich hätte angreifen wollen, wäre er der gefährlichste Gegner gewesen, und der erste Pfeil seiner Arcuballiste hätte bereits zwischen ihren Augen gesteckt, ehe der Zwerg auch nur die Hälfte der Distanz zwischen ihm und Sian hinter sich gelegt hätte.

Ein Glück, dass Sian nicht vorhatte, der Gruppe zu schaden. Warum sollte sie auch?

Als Sian sich, von Fenn gestützt, wieder auf dem goldenen Gras niederliess – seltsam, hatte dies vorher auch bereits Rief getragen? – stürzte ein schwarzer Vogel vom Himmel herab und landete auf Fenns Schulter. Das überraschte Sian nicht, sie hatte bereits einige Falkner gesehen, die ihren Vögel solcherlei Kunsttücke beibringen hatten können. Was sie allerdings überraschte, war, dass der Rabe seinen Schnabel öffnete und eine krächzende, menschliche Stimme aus seinem Schlund entschlüpfte:

„Krah! Jetzt beruhigt euch erst mal wieder! Der Körper dieser arme Dame hier hat fünf Jahrhunderte lang einem grausig grausamen Eisdämonen als Gefäss gedient! Kroa! Selbst wenn Arbons kleiner Zaubertrick den Dämon vertrieben haben sollte, heisst das noch lange nicht, dass es ihr wieder gut geht. Sie weiss nicht, wer sie ist, wo sie ist, wann sie ist... Habt doch mal etwas Verständnis. Kro‘ar! Kram, wenn du nicht hier sein willst, dann geh‘ doch einfach woanders hin!“

„Wie immer bist du der einzige, der hier etwas Verstand in der Birne hat“, brummelte Kram. Dann machte er sich vom Acker, immer wieder einen kritischen Blick zurückwerfend.

Sian war fasziniert von diesem schwarzen Wesen vor ihr: „Ein sprechender Rabe? Wie ist denn so etwas nur möglich?!“

„Das ist das, was du dich am ehesten fragst?! Wie der Rabe sprechen kann?“, fragte Arbon ungläubig.

„Tut mir leid, Morar hat natürlich recht, wie unbedacht von uns“, wandte sich Fenn jetzt wieder an Sian, Arbons Einwurf ignorierend. Er setzte sich vor Sian auf den Boden und fragte: „Wie fühlst du dich denn?“

Sian antwortete nicht. Das war auch eine schwere Frage. Sie hatte nur selten etwas wirklich gefühlt, und wenn, war es primär ein Verlangen nach dem kalten Eis und Schnee des Kuolema gewesen. Und auch wenn sie das Eis nicht komplett verlassen hatte, so war dieses Bedürfnis nun komplett versiegt. Was hatte sie jetzt noch für einen Antrieb?

“Wo bin ich und wer seid ihr?“, fragte sie statt einer Antwort auf Fenns Frage.

Fenn stockte kurz. Dann sagte er: „Du bist in Andor, dem Lande der Freien.“ Nicht ohne einen Anflug von Stolz in seiner Stimme fügte er hinzu: „Wir sind die Helden von Andor, der Orden derjeniger, die die Armen und Schwachen beschützen. Wir haben den Angriff des Dunklen Magiers Varkur abgewehrt – mehrmals. Wir haben den bösen Drachen Tarok erschlagen. Wir haben die Rietburg gleich zweimal... Aber gut, ich merke gerade, dass dies dir alles nichts sagt. Wir sind jedenfalls die, die den Dämon besiegt haben. Wir sind die, die dich aufnehmen und dir einen neuen Platz in dieser Welt geben könnten– sofern du willst.“

Sian war unsicher. Es stimmte, sie hatte keinen Platz mehr in dieser Welt.

Sian war einst gestorben, doch jetzt war sie wieder da. Siantari war nicht mehr. Sie war sich ihrer Menschlichkeit wieder sicher. Was sollte nun aus ihr werden?











\newpage
\section{Geschenke vom Santa Gor (2022)}

\az{Weihnachten}

So oft geht der Santa Gor vergessen. :sad:

Es gibt ihn, so glaubt mir doch! Wenn die Tage kürzer werden und der Schnee über das Land fällt, dass fliegt der Santa Gor unter rotem Mondschein über den Nachthimmel, in einer großen Kutsche gezogen von rotnasigen Wardraks, und verteilt gute Gaben an alle, die sie brauchen können. Im Vergleich zu unserem Planeten ist der bekannte Teil der Andor-Welt nicht mal so groß, da klingt es doch sogar etwas realistischer, dass der Santa Gor in derselben Nacht einen kompletten Rundgang darüber drehen kann.

Von den unschuldigen Ambacus in den Fängen der Krahder über die Agren in ihren Höhlen und die Kreaturen auf Land und See (die meisten davon kriegen – wie auch Garz – eingepackte Edelsteine geschenkt), bei jedem landet ein kleines eingeschnürtes Paketlein. Von den Barbaren im Osten über die Bewahrer und Rietländer bis zu den Tulgori im Westen, sie alle kriegen ein gut gemeintes Geschenk in dieser glorreichen Nacht. Lonas erhält einen Kauknochen, Turr eine Quietschemaus, nur Sabri steckt leider (wie auch alle Trolle) in der Winterstarre und geht leer aus. Dafür wirft der Santa Gor sogar den Eisdämonen im ewigen Eis etwas Schönes hin. Karotten, damit sie Schneemänner bauen können?

Die meisten Naturgeister könnten mit einem Geschenk wohl wenig anfangen, doch Vara kriegt hin und wieder einen leuchtenden danwarischen Stein, der wie ein Minatur-Wasserfall aussehen kann.

Der Santa Gor rutscht in Barathrum herunter wie durch einen Schornstein und verteilt links und rechts seine Gaben. Bei Silberhall muss er halt ans große Tor klopfen und seinen Geschenkberg abliefern. Taren erfreuen sich an Silberketten und Rechenschiebern, Werftheimer freuen sich über gutes starkes Holz, Nixen jubilieren über neue Säcklein für ihren Staub und die Geschenke der Danware müssen rasch geöffnet werden, ehe etwaige Lavasteine darin die Verpackung schmelzen.

Von ganz hoch oben wirft der Santa Gor Gaben und Geschenke auf das verfluchte Narkon, sodass selbst die dortigen Gefangenen sich über etwas freuen können.

In Hadria, dem Land der ewigen Kälte, gibt es besonders viele, die sich eine Wärmeflasche wünschen. Und dem Seeriesen wird natürlich eine seltsam geformte Muschel geschenkt.

Und wenn alle Geschenke verteilt sind, gesellt der Santa Gor sich zu den Helden ans Lagerfeuer (oder an ein ewiges Feuer, falls sie sich gerade in Hadria aufhalten), gibt in einer fremden Sprache Grußworte von sich und verteilt rote Zipfelmützen, wie er auch selbst eine trägt. Manchmal gesellen sich da gar andere Kreaturen dazu. Solange der Santa Gor da ist, wird es nicht zu Streitigkeiten kommen.

Es ist ein Geheimnis, wie der Santa Gor an seine Geschenke kommt, ja, wie er sie überhaupt einpackt. Hat er das Schleifenbinden mit Hornklauen gemeistert oder kriegt er Hilfe von einigen freundlichen Feen in seiner Geschenkeschmiede? Woher nimmt er alles Verpackungsmaterial und wohin verschwindet es, wenn es sich ein, zwei Tage später wieder in Luft auflöst?

Warum tut er dies alles? Will er beweisen, dass es auch Gors mit gutem Gewissen geben kann? Freut er sich schlicht daran, Freude zu verbreiten? Ist er ein Gestaltwandler, der sich zur Zeit aus irgendeinem Grund gerade besonders am Körper eines Gors erfreut?

Manchmal, wenn es ganz rau steht um das Schicksal der Welt, kommt der Santa Gor sogar außerhalb der Weihnachtszeit nach dem Rechten sehen. Dann aber ohne fliegende Wardrak-Kutsche. Vielleicht ist er ja auch nur ein Angestellter einer mächtigeren Entität und darf darum die Kutsche nur zu Geschäftszeiten fahren?

Der Santa Gor ist ein Wesen vieler Rätsel. Doch kein Rätsel ist, ob es ihn tatsächlich gibt. Das muss man doch einfach glauben. :P\bigskip



Verspätete frohe Festtage, Freunde! :P






\newpage
\section{Im Schatten des Winters (2024)}

\az{Weihnachten}

\textit{Nur ein einzelner Teil einer kollaborativen Adventsgeschichte.}\bigskip

Das bläuliche Glühen strahlte aus den vier Augen der beiden Kreaturen, die vor der Heldengruppe aus dem Gestrüpp brachen.

Lange Hauer, spitze Hornklauen, dunkle Schuppen – zweifelsohne waren dies Nachtgors aus Hadria, dem fernen Land der Magie. Doch diese beiden Nachtgors schienen nicht an Menschenfleisch interessiert. Der linke Nachtgor hielt sich die linke Schulter, an der schwarzes Blut heruntertropfte. Der andere zog sein rechtes Bein hinter sich her. Sie erstarrten, als sie der vier bewaffneten Helden gewahr wurden.

Vier Heldenaugenpaare starrten in vier pupillenlose Kreaturenaugen.
Eine Eule schrie in der Ferne.

Und Bewegung kam über die Nachtgors. In entgegengesetzte Richtungen humpelten sie davon.

Kheela, die Hüterin der Flusslande, brüllte ihnen hinterher: "Ja, lasst euch hier nicht mehr blicken! Und in den Flusslanden auch nicht!" Dann richtete sie ihre zeremonielle Ghirlada und blickte stolz in die Runde.

Kram, der Tiefminen-Arbeiter aus Cavern, rieb seine Finger, an denen immer noch das getrocknete Blut des verletzten Andori klebte. "Unser Andori hatte nicht einmal sein Schwert gezogen. Nicht er hat diese Nachtgors so zugerichtet. Sind die eigentlich immer so ängstlich?"

Chada, die ehemalige Bewahrerin aus dem Wachsamen Wald, ließ ihren Bogen Audax sinken und furchte nachdenklich ihre Stirn. "Wenn Eara oder Tenaya hier wären, könnten die dir sicher genaueres sagen. Nach dem wenigen, was ich von Nachtgors gelesen habe, sind sie vergleichbar mit den hiesigen Gors, verstehen sich aber umso besser auf schneeige Umgebungen. Der Legende nach sollen sie nach dem prophezeiten Qurun in Scharen aus der Hadrischen Unterwelt dringen, wie es einst unsere Gors nach dem Unterirdischen Krieg aus Krahal taten."

Aćh, die Takuri-Hüterin aus dem fernen Tulgor, ließ ihr goldenes Mondschwert zur Sicherheit gezogen. "Soll Turr ihnen folgen? Oder wollen wir vielmehr herausfinden, wovon diese Nachtgors flohen?"

Nach rascher Beratung kamen die vier Helden zum Schluss, dass die beiden verletzten Nachtgors keine akute Gefahr darstellten. Stattdessen wollten sie sich weiter ins Unterholz des südlichen Waldes vorwagen und versuchen zu ergründen, wer oder was die Nachtgors verletzt hatte.

Die kahlen Bäume warfen lange Schatten über den dunklen Schnee. Es knirschte laut unter den vier Paar Stiefeln, so leise die Helden auch zu schleichen versuchten.

Kram hatte als Zwerg natürlich die beste Sicht in der Dunkelheit, und so hatte er kein Problem, die Fährten der Gors zu erspähen. Problematisch war eher, dass er immer wieder tiefe Schneelöcher einkrachte. Seine sonst so gute Laune verpuffte endgültig, als er zum dritten Mal bis zur Brust im weißen Pulver einsank.

"Und das ausgerechnet heute", grummelte Kram durch seinen vereisten Bart hindurch. "Dieser neue Sternkundige aus Tulgor las dunkle Tage aus seinen Himmelskarten. Wir sollten bald zur Rietburg aufbrechen, wenn wir das Adventsfest noch miterleben wollen."

"Och, mich würde es jetzt nicht allzu stören, König Thoralds neuste langatmige Rede zu verpassen", sagte Aćh. "Die anderen Helden wiedersehen und Flaps streicheln würde ich aber schon gerne."

"Doch ist dies hier nun unsere Aufgabe", sprach Chada bestimmt. "Wir können dem südlichen Wald nicht einfach den Rücken kehren, solange ein unbekannter Schrecken hier lauert."

"Und ohnehin", sagte Kheela versöhnlich, "Wollen wir etwa Fenn erzählen gehen, neben seiner Höhle lauerte eine unbekannte Gefahr, und wir wären einfach unverrichteter Dinge abgezogen? Da stehen wir nicht als gute Freunde da."

Da wollte niemand etwas entgegnen. Noch enger in ihre dicken Wintermäntel eingehüllt, zogen die vier Helden weiter durch die weiße Wunderwelt. Aćh dachte an ihre Mutter Nelímar, die ihr gut zusprechen würde, wenn sie hier wäre. Chada wünschte sich, sie könnte ihren treuen und warmen Zauberwolf Lonas kuscheln. Kheela dachte an ihren Sohn Janis, der hoffentlich gerade in seinem warmen Bett im Hof auf der anderen Flussseite vor sich hin schnarchte. Kram schwelgte in Erinnerungen an den legendären Eintopf seiner Großfamilie. Sein Magen knurrte entsprechend.

"Mir knurrt auch langsam der Magen", gab Kheela zu, "Was würde ich nicht alles für eine warme Suppe geben."

"Mit Blaubachbeeren!", sagte Chada schwärmerisch, "Kombiniert mit den legendären grünen Blättern des Westerwaldes lassen die jedem Gericht eine feine Note verleihen."

"Eure Träume in allen Ehren", meinte Kram, auf einmal flüsternd, "Aber bin ich der einzige, der diesen Gestank riecht? Hinter dieser Schneewehe da vorne ist etwas. Bei den versengten Augenbrauen des Wunderkinds, ich kann den Geruch kaum aushalten!"

Mit seiner Axt schaufelte er mehr Schnee zur Seite, doch die Schneewehe wurde kaum kleiner

"Warte, lass mich. Oder besser gesagt, lass Turr", sagte Aćh. Sie zog aus ihrem Wintermantel eine grob gehauene steinerne Flöte hervor und spielte eine leise fröhliche Melodie. Nicht eine Minute später stieß mit einem klangvollen Schrei ein golden glühender Feuervogel durch die dunklen Wolken und krallte sich in Aćhs Schulter.

"Tja, damit ist das Überraschungsmoment dahin. Turr! Haamun Meza.", kommandiere Aćh, und deutete nach vorne. Mit einem melodischen Kreischen hüpfte der Feuertakuri auf Aćhs Haarschopf – die Takuri-Hüterin ächzte nur leicht unter seinen Krallen – breitete seine Flügel gebieterisch aus und begann, immer stärker zu leuchten. Flirrende Hitze umfasste den Schneehaufen. Unter der Wärme des Feuervogels schmolz er dahin wie Eis im Licht der Morgensonne und gab den Blick frei auf den Ursprung des Gestanks, einen großen schattigen Schemen.

Sie hatten ihr Ziel erreicht.

Vor ihnen wurde ein Bild der Zerstörung deutlich.

Der Schemen stellte sich als glitzernde Holzkutsche heraus, größer als der tollste Troll. Allerdings glich sie aktuell mehr einem Wrack an einer der unzähligen Klippen des Hadrischen Meeres, wie sie verkehrt herum im Boden steckte.

Im aufgewühlten Schnee, umgeben von roten Blutflecken, erkannte Kheela das verlorene Schwert des verletzten Andori. Schwer lag es in ihrer Hand, als sie sich die schartige Waffe an den eigenen Gürtel steckte. Die Handhabung solcher Klingen war nicht ihre Stärke, als Hüterin der Flusslande hatte sie oft repräsentativere Aufgaben. Sollte sie ihren Wassergeist Vara rufen oder damit abwarten, bis ihre Gegenwart wirklich nötig war? Sie umklammerte ihren danwarischen Stab fester und richtete ihren Blick auf das große Gefährt vor ihr.

Es war einst sicherlich eine elegante Kutsche gewesen, nun jedoch stand sie kopfüber in ihren eigenen Einzelteilen. Davor lagen zwei tote Wardraks, zweifelsohne die Urheber des Gestanks. Wenn das Licht des roten Mondes die Heldenaugen nicht täuschten, waren die Echsennasen dieser Wardraks nicht wie üblich schwarz, sondern knallrot. Schwere Ketten befestigten die animalischen Kreaturen mit dem Karren. Tiefe Furchen zogen sich durch ihre Flanken, als hätte ein Wesen mit großen Klauen sie angefallen. Furchen, wie sie der Andori aus der Taverne an der Schulter gehabt hatte. Schwarzer Rauch stieg von den Wunden auf und waberte wie dunkle Schatten um den Ort des Geschehens.

Seltsam war auch, dass es keine Fußspuren oder Kufenspuren im Schnee um die Kutsche gab, die zeigen würden, woher sie gekommen war. Hingegen waren zahlreiche Äste an nahe liegenden Bäumen abgeknickt, als wäre der Karren durch sie direkt vom Himmel gefallen.

Unter, hinter und neben der abgestürzten Kutsche war ein wahrer Berg an Geschenken verstreut. Einige waren noch in Pergament in allen Regenbogenfarben verpackt und mit schönen Schleifen verschnürt, doch die meisten waren angeknickt, plattgedrückt oder gar völlig zerfetzt. Da lagen tropfende Trinkschläuche, zerbrochene Fernrohre, gar ein Falke mit angeknackstem Flügel, neben dem sich Turr sofort beruhigend setzte.

Und neben dem Karren, sicher in einem großen Schneehaufen gelandet, steckte ein weiterer Gor und atmete schwach. Kein Nachtgor war dies, wie an seiner rötlichen Schuppenfarbe erkennbar war. Zudem trug dieser Gor trug nicht nur die losen Lendenschurze, die von diesen Kreaturen auf im Winter zu erwarten war. Nein, dieser Gor war in einen eleganten roten Mantel gekleidet, und auf seinem Kopf trohnte eine lange Zipfelmütze mit einem weißen Bommel.

Chada blinzelte überrascht. "Ist das nicht ... ? Nein, das kann nicht sein, ich dachte stets, er wäre bloß eine Legende, die die Hohepriester von Mutter Natur erzählen, um den Nachwuchs brav zu lassen."

Kheela legte ihren Kopf schief. "Liebe Chada, haben wir in den letzten Jahren nicht zu oft erfahren, dass alle Legenden einen wahren Kern haben? Der Schwarze Herold, der letzte Drache, die Eis-Dämonen aus dem Fahlen Gebirge ... warum auch nicht der Santa Gor?"

Aćh, welche im fernen Land Tulgor aufgewachsen war und in ihrer Zeit hier noch lange nicht alle Sagen Andors vernommen hatte, fragte ein nervös: "Wer oder was ist der Santa Gor?"

Kram holte tief Luft und setzte an: "Die Geschichte vom Santa Gor erzählt mein guter Freund Mart immer gerne zum längsten Dunkeltag, wenn der Winter bevorsteht. Es geht darin um den guten Geist der ..."

Da regte sich die Kreatur mit der Zipfelmütze, richtete sich auf, klopfte sich den gröbsten Schnee vom langen weißen Bart und krächzte in akzentfreiem Andorisch: "Jawohl, ich bin der Santa Gor, o Helden von Andor. Und wie ich befüchte, benötigen ich und meine Nachtgor-Helferlein dringend Eure Hilfe."




% TODO: Ergänze meine Teile vom Abenteuer der Tavernengäste \url{https://legenden-von-andor.de/forum/viewtopic.php?f=11&t=8059}, vielleicht als eigenes Kapitel.






\begin{chapterbox}
    \chapter{Maximal erreichbarer Kampfwert (2022)}
    \az{Jahr 65}

    Wer die Andor-App "Das Geheimnis des Königs" durchgespielt hat, mag sich erinnern, dass dort bisweilen verrückt hohe Kampfwerte vorkommen. Da stellt sich für knobelbegeisterte Andori natürlich die Frage, ob ähnlich enorme Kampfwerte auch im physischen Brettspiel erreicht werden können. Und wie hoch der maximale offiziell erreichbare Kampfwert aktuell ist. :P

    Hier folgt ein konstruierter Spielbericht, der die Erreichbarkeit eines gewissen hohen Kampfwerts zu beweisen versucht. Vermutlich handelt es sich dabei nicht um den allerallerhöchsten offiziell erreichbaren Kampfwert, er sollte sich jedoch hoffentlich in der richtigen Größenordnung bewegen. Am Ende folgen dann noch einige Vermutungen, wie man den hier hergeleiteten Kampfwert vielleicht übertreffen könnte, wenn man noch ein bisschen mehr Hirnschmalz hineinstecken würde.

    Da es ein künstlicher Legendenverlauf ist, werden die Helden natürlich am Ende unnatürlich viele Päsche werfen. Es soll jedoch auch erkennbar sein, dass das Erlangen derart vieler Stärkepunkte um einiges wahrscheinlicher als ein Riesenpasch ist. Darum landen z.B. die Runensteine in dieser konstruierten Partie nicht allesamt auf denselben Feldern. Und die meisten Kämpfe vor dem Endkampf wurden echt ausgewürfelt.

    Ihr seid hiermit herzlich eingeladen, euch vor dem Weiterlesen kurz zurückzulehnen und nachzudenken.
    Was meint ihr: In welcher Legende kann der größte Kampfwert erreicht werden? Mit welchen Helden? Mit welchen Spielvarianten und Mini-Erweiterungen?
    Und vor allem: Was ratet ihr, wie hoch der maximal offiziell erreichbare Kampfwert sein könnte?
\end{chapterbox}












{\parindent0pt


\section{Spielvorbereitung und Tag 1}

\az{Jahr 65}

Wir spielen die Ära des Sternenschilds (vor allem wegen der unglaublich starken Wölfe).

Andere vielversprechende Kandidaten wären L10 im Dienste des Turms (stellt euch +14 Taren an der Seite eines +24 Seeriesen vor) und L17 (dank alter Waffen und allerlei toller Hoffnungskarten).\bigskip



Da mehr Helden höhere Kampfwerte erreichen, spielen wir natürlich mit der maximal möglichen Anzahl: 6!

Unsere 6 Helden sind:

– Iril (für ihre famose SP-Generierung)

– Orfen (für seinen SP-Bonus gegenüber Trollen)

– Darh (für ihren Knochen-Golem)

– Thorn (wegen seiner 4 Würfel)

– Forn (wegen seiner 3 Würfel + Helm)

– Aćh (für ihren Feuertakuri)



Von der Reihenfolge her sortieren wir sie einfach nach Rang:

Orfen, Thorn, Aćh, Forn, Iril, Darh.



Ich war überrascht, dass Bragor und sein famoser Brunnen-SP-Generierungs-Loop mit Irils Runenscheibe nicht nur nicht nötig sind, sondern sogar schlechter für die Gesamtsumme als diese Heldenkonstellation.



Von den Sonderfähigkeiten her hätte ich zunächst lieber Jarid statt Forn gewählt, da Iril mit Jarid famose Synergie zum WP-Verteilen hat. Da es im Grundspiel jedoch nur 3 Helme gibt, müsste dann einer der 4 potentiellen Dreierpasch-Werfer ohne Helm in den Endkampf ziehen. Also kam der von selbst Päsche addierende Forn ins Spiel. In der Partie selbst stellte sich Forn dann mal wieder als absoluter Alleskönner heraus, sodass ich gar nicht mehr weiß, was ich ursprünglich gegen ihn hatte.\bigskip



Wir nutzen die NH-Anpassungen fürs Spiel mit 5/6 Helden:

– Neutrale Zeitsteine

– Verstärkte Kreaturenleiste

– Schwarzer Herold



Ich gehe davon aus, dass es offiziell erlaubt ist, frei zwischen den NH-, DH- und MH-Anpassungen für 5/6 Helden auszuwählen, obwohl diese das Spiel unterschiedlich schwer machen. Falls ihr stattdessen meint, dass durch die Verwendung von Magischen Helden im Spiel mit 5/6 Helden die Verwendung des Bannkreis von Choranats obligatorisch würde, müssten wir leider noch dessen 4 bis 8 Stärkepunkte vom finalen Kampfwert abziehen.\bigskip



Wir spielen mit folgenden offiziellen Spielvarianten und Mini-Erweiterungen:

– Das letzte Lagerfeuer

– Der Bruderschild aus Neue Helden

– Die Wunschbrunnen

– Das Licht der fünften Stunde

– Die weißen Ereigniskarten aus der Bonus-Box

– Die Zwergentüren aus Düstere Zeiten



Mit dem letzten Lagerfeuer können die Helden bei einem zufälligen Buchstaben (in dieser Partie auf B) beliebig viele SP und WP untereinander tauschen.

Der Bruderschild wird zu Beginn zufällig eingewürfelt (in dieser Partie auf 31). Helden können ihn ebenfalls nutzen, um ihre SP mit anderen Helden zu tauschen.

Mit den Wunschbrunnen können Helden volle Brunnen auf ihrem Feld opfern, um sich etwas von der Liste zu wünschen. Zum Beispiel leichter zu zähmende Wölfe. ;)

Das Licht der fünften Stunde schenkt jeden Tag den ersten in der fünften Stunde kämpfenden Helden +5 Kampfwert.

Die weißen Ereigniskarten aus der Bonus-Box bringen zahlreiche tolle Gegenstände wie Kampfaxt, Messer und Brote ins Spiel. Das Bonus-Heft aus der Big Box besagt, dass die weißen Ereigniskarten offiziell auch ohne die Fluggors verwendet werden können. Vermutlich würden wir es auch in Anwesenheit der Fluggors hinkriegen, aber wir müssen uns das Leben ja nicht extra schwerer machen. :)

Die Zwergentüren sind zwar nicht zwingend nötig, aber ich mag sie. Sie schenken uns nicht nur Abkürzungen, sondern auch einen weiteren Flattervogel. :D\bigskip



Wir spielen mit der "schwierigeren" Legendenkarte E. Mehr Kreaturen im Spiel können auch etwas Gutes sein.



Die genauen Positionen von Runensteinen, Waldpilzen, etc. könnt ihr auf dem Bild weiter unten sehen.



Die A2-Karte ist diejenige mit den 3 Trollen. Die Trolle werden im östlichen Rietland eingewürfelt (42, 44, 45). Eine Horde Kreaturen stürmt aus dem Süden auf die Rietburg zu. Iril wird die Legende beginnen.



Die Fürstenaufgabe ist "Burgtor". Orfen wird auf Feld 2 von Baumeister Mard über das schiefe Burgtor informiert, während die restlichen Helden in der Taverne (72) einkehren. Bei Legendenende müssen Holzstämme im Wert von 12 auf der Burg liegen.



Von den Gaben der Andori lassen wir den Hornfalken in der Schachtel. Ein kostenloser Falke wäre toll, aber die anderen Gaben scheinen für unser Ziel einfach besser geeignet, und wir kriegen ohnehin schon eine gratis Eule von den Zwergentüren.



Die Eule geht natürlich an unsere Vogelliebhaberin Aćh.

Die andorische Flöte geht ebenso natürlich ebenfalls an unsere Flötenspielerin Aćh.

Die Fackel von Cavern geht an Thorn.

Das Hadrische Stundenglas, 2 Bonus-SP und 3 Gold gehen allesamt an Darh.\bigskip






\includegraphics[width=\textwidth]{Das Erbe des Wunderkindes/Bilder/Tag 1 Anfang.jpg}

\textit{Anfang von Tag 1.}\bigskip







Dann geht es los mit Tag 1! :P

Wir zielen zunächst darauf ab, möglichst rasch die Wölfe zu rufen und zu zähmen.



Zuallererst spielt Aćh die andorische Flöte. Der Würfel zeigt eine 4! Die WP aller Helden außer Orfen steigen auf 7 +4 = 11. Dann übergibt Aćh die Flöte an Forn.



Aćh nutzt die Eule 2x, um die 30-Feuertür und die 61-Feuertür zu öffnen. Sie frischt die Eule 2x auf. Aćhs WP sinken auf 11 -3 -3 = 5.

Thorn legt die Fackel von Cavern ab und nimmt die Eule von Aćh an. Er nutzt die Eule 2x, um die 5-Wassertür und die 45-Wassertür zu öffnen. Er frischt die Eule 2x auf. Thorns WP sinken auf 11 -3 -3 = 5.

Wir wissen nun genau, welche Zwergentüren miteinander verbunden sind: 5/61, 45/37 und 55/30.

Thorn gibt die Eule an Forn weiter.



Zug 1: Iril dreht an ihrer Runenscheibe.

In Stunde 1 zeigt der Runenwürfel eine 1. Iril dreht die Runenscheibe um 1 +1 = 2 Speichen auf die 3-WP. Irils WP steigen auf 11 +3 = 14.

Nun befindet sich Iril in ihrer untersten WP-Zeile und kann fortan immer, wenn sie will, ihre Runenscheibe unabhängig vom Runenwürfel jeweils um genau 3 Speichen drehen (1 +2, 2 +1 oder 3 +0).



Zug 2: Darh läuft für 1 Stunde (Zeitstein von 1 auf 2) von 72 auf 18. Darh kauft sich einen Helm. Darhs Gold sinkt auf 3 -2 = 1.



Zug 3: Orfen beendet seinen Tag. Er erhält das Sonnenaufgang-Plättchen.



Zug 4: Thorn beendet seinen Tag.



Zug 5: Aćh nutzt ihre tulgorische Steinflöte. Der Feuertakuri (nennen wir ihn Turr) fliegt von 72 auf 58. Vielleicht taucht Lonas ja dort auf.

Aćh beendet ihren Tag.



Zug 6: Forn läuft für 2 Stunden (Zeitstein von 2 auf 4) von 72 nach 35. Er leert den 35-Brunnen. Forns WP steigen auf 11 +3 = 14.

Darh nutzt das Hadrische Stundenglas, um den Zeitstein von 4 zurück auf 1 zu schieben.



Zug 7: Iril wählt ihre Aktion "Runen befragen". Wann immer sie den 35-Brunnen auffrischt, leert Forn diesen sofort. Einige Quellen dazu:

– Irils Sonder-Aktion kann "Runen befragen" genannt werden.\footnote{Aus der Taverne, "Magische Helden und ein paar allgemeine Fragen": [Giftknödel:] die Aktion "Runen befragen" kostet immer jeweils eine Stunde, genau wie jede einzelne Kampfrunde beim Kämpfen. D.h. zunächst rückt der Zeitstein weiter. Dann werden ggf. WP gezahlt. Und dann folgt die Aktion, also das Drehen der Scheibe.}

– Iril kann auch Brunnen auffrischen, auf deren Feld ein Held steht\footnote{Aus der Taverne, "RF zu den MH Barz, Iril und Aćh in Teil III": [Giftknödel:] Quellen werden nur bei Sonnenaufgang nicht umgedreht, wenn ein Held darauf steht. Die Macht der Runen hingegen vermag das.} (Die Quelle bezieht sich auf Quellen :mrgreen:, aber bei Brunnen sollte das analog funktionieren).

– Zwischen zwei Runenscheiben-Drehungen können freie Handlungsmöglichkeiten ausgeführt werden.\footnote{Aus der Taverne, "Einige Fragen zu den Magischen Helden": [Giftknödel:] Grundsätzlich gilt: Jedes Drehen kostet 1 Stunde. Das Drehen der Scheibe außerhalb des Kampfes ist eine abgeschlossene Aktion. Wenn sie jeweils 1 Stunde zahlt, kann sie diese Aktion auch mehrfach hintereinander tätigen.
Freie Handlungsmöglichkeiten und Standard-Segeln können zwischen den Aktionen ausgeführt werden.} Stellt euch vor, wie nützlich dies erst in Verbindung mit Bragors oder Jarids SF ist. Aber auch Forn kann hübsch davon profitieren. :P

In Stunde 2 dreht Iril die Runenscheibe um 3 Speichen auf den 2-Brunnen. Sie frischt den 35-Brunnen auf. Forn leert den 35-Brunnen. Forns WP steigen auf 14 +3 = 17.

In Stunde 3 dreht Iril die Runenscheibe um 3 Speichen auf den 5-Brunnen. Iril frischt den 35-Brunnen auf. Forn leert den 35-Brunnen. Forns WP steigen auf 17 +3 = 20.



Zug 8: Darh läuft für 1 Stunde (Zeitstein von 3 auf 4) von 18 auf 36. Sie findet einen gelben Runenstein.



Zug 9: Forn läuft für 2 Stunden (Zeitstein von 4 auf 6) von 35 auf 24. Er findet einen grünen Runenstein.

Ich verletze hier die goldene Forn-Regel "Nutze niemals eine Stunde für einen Schritt, für den du auch einen WP bezahlen könntest." Und es gibt nicht einmal einen guten Grund dafür. Ich habe Forns SF hier schlicht und ergreifend übersehen. :oops:



Zug 10: Iril wählt ihre Aktion "Runen befragen".

In Stunde 7 dreht Iril die Runenscheibe um 3 Speichen auf die 3-WP. Irils WP steigen auf 14 +3 = 17.



Zug 11: Darh läuft für 2 Stunden (Zeitstein von 0 auf 2) von 36 auf 72. Sie legt den gelben Runenstein ab.



Zug 12: Forn läuft für 2 Stunden (Zeitstein von 2 auf 4) und 4 WP (Forns WP sinken auf 20 -4 = 16) von 24 über die 5/61-Zwergentüren auf 54. Er findet einen blauen Runenstein.

Forn nutzt die Eule, um Iril den grünen und blauen Runenstein zu schicken.



Zug 13: Iril, nun mit allen drei Runensteinen ausgerüstet, wählt ihre Aktion "Runen befragen".

Wichtig: Iril hat nur zwei kleine Ablagefelder und kann nicht alle drei Runensteine gleichzeitig tragen. Da Iril jedoch zwischen den Runenscheiben-Drehungen freie Handlungsmöglichkeiten (wie das Ablegen und Aufnehmen von Runensteinen) ausführen kann, und da sie bei jeder Drehung auf mindestens einer der drei Runenstein-Speichen gar nicht landen kann, ist es Iril dennoch möglich, immer die passenden Runenstein tragen. Ich werde der Übersicht halber nicht immer extra aufschreiben, welche Runensteine Iril ablegt und aufnimmt.

In Stunde 8 (Irils WP sinken auf 17 -2 = 15) zeigt der Runenwürfel eine 3. Iril dreht die Runenscheibe um 3 +2 = 5 Speichen auf die 4-WP. Irils WP steigen auf 15 +3 = 18.

"4-WP" steht dafür, dass diese Speiche Kampfwert 4 hätte. Die 4-WP-Speiche verschafft auch "nur" +3 WP.

In Stunde 9 (Irils WP sinken auf 18 -2 = 16) zeigt der Runenwürfel eine 2. Iril dreht die Runenscheibe um 2 +0 = 2 Speichen auf den blauen Runenstein. Irils SP steigen auf 2 +1 = 3.

In Stunde 10 (Irils WP sinken auf 16 -2 = 14) dreht Iril die Runenscheibe um 3 Speichen auf den gelben Runenstein. Irils SP steigen auf 3 +1 = 4.



Zug 14: Darh passt (Zeitstein von 4 auf 5).



Zug 15: Forn läuft für 1 WP (Forns WP sinken auf 16 -1 = 15) von 54 auf 47. Das 47-Nebelplättchen versteckte eine Ereigniskarte.

Weiße Ereigniskarte 12 wird ausgelöst. Das Brot aus dem Wachsamen Wald – mit eingebackenen Apfelnüssen – ist allseits bekannt. Ein Brot wird auf 55 eingewürfelt.



Zug 16: Iril wählt ihre Aktion "Runen befragen". Sie nutzt dafür einen noch unbewegten Zeitstein.

In Stunde 1 zeigt der Runenwürfel eine 3. Iril dreht die Runenscheibe um 3 +1 = 4 Speichen auf die 4-WP. Irils WP steigen auf 14 +3 = 17.

In Stunde 2 zeigt der Runenwürfel eine 1. Iril dreht die Runenscheibe um 1 +1 = 2 Speichen auf den blauen Runenstein. Irils SP steigen auf 4 +1 = 5.

In Stunde 3 zeigt der Runenwürfel eine 2. Iril dreht die Runenscheibe um 2 +0 = 2 Speichen auf die 3-WP. Irils WP steigen auf 17 +3 = 20.

In Stunde 4 zeigt der Runenwürfel eine 1. Iril dreht die Runenscheibe um 1 +0 = 1 Speiche auf den gelben Runenstein. Irils SP steigen auf 5 +1 = 6.

In Stunde 5 dreht Iril die Runenscheibe um 3 Speichen auf den grünen Runenstein. Irils SP steigen auf 6 +1 = 7.

In Stunde 6 dreht Iril die Runenscheibe um 3 Speichen auf den blauen Runenstein. Irils SP steigen auf 7 +1 = 8.

In Stunde 7 dreht Iril die Runenscheibe um 3 Speichen auf den gelben Runenstein. Irils SP steigen auf 8 +1 = 9.

In Stunde 8 (Irils WP sinken auf 20 -2 = 18) dreht Iril die Runenscheibe um 3 Speichen auf den grünen Runenstein. Irils SP steigen auf 9 +1 = 10.

In Stunde 9 (Irils WP sinken auf 18 -2 = 16) dreht Iril die Runenscheibe um 3 Speichen auf den blauen Runenstein. Irils SP steigen auf 10 +1 = 11.

In Stunde 10 (Irils WP sinken auf 16 -2 = 14) dreht Iril die Runenscheibe um 3 Speichen auf den gelben Runenstein. Irils SP steigen auf 11 +1 = 12.

Volle Stärkeleiste für Iril! :D



Zug 17: Darh läuft für 1 Stunde (Zeitstein von 5 auf 6) von 72 auf 23. Dieser Skral soll noch heute fallen. Sie nimmt dabei die Fackel von Cavern und einen von Irils Runensteinen von 72 mit, legt diese aber direkt wieder auf 23 ab.



Zug 18: Forn läuft für 1 WP (seine WP sinken auf 15 -1 = 14) von 47 auf 46. Das 46-Nebelplättchen versteckte Reka. Reka wird auf 46 aufgestellt. Forn erhält einen Trank der Hexe.



Zug 19: Iril läuft für 1 Stunde (Zeitstein von 6 auf 7) von 72 auf 23. Sie nimmt die schlafende Eule und zwei Runensteine mit.



Zug 20: Darh lädt Iril zum gemeinsamen Kämpfen gegen den 23-Skral ein.

In Kampfrunde 1 (Zeitsteine von 0 auf 1 und von 7 auf 8, Darhs WP sinken auf 11 -2 = 9) zeigt der Runenwürfel eine 1. Iril dreht die Runenscheibe um 1 +0 = 1 Speiche auf das 5-Auge. Die Helden haben einen Kampfwert von 4 SP (Darh) +12 SP (Iril) +4 (Darhs Würfel) +5 (Irils Speiche) = 25.

Der 23-Skral hat einen Kampfwert von 10 SP +4 = 14.

Der 23-Skral ist besiegt.

Belohnung: Darhs Gold steigt auf 1 +4 = 5.

Darh beschwört ihren Knochen-Golem. Er wird samt Skral auf Feld 41 eingewürfelt (oh nein, das bedroht Casimirs Bruder auf 41). Das Golem-Symbol kommt auf E.

Der Erzähler läuft auf B.

Die Wölfe (nennen wir sie Lonas, Rutan und Merla) werden eingewürfelt. Lonas erscheint auf 64, Merla auf 33 und Rutan auf 43. Das Wolfssymbol wird auf Stunde 4 gewürfelt.

Das letzte Lagerfeuer wird ausgelöst. Die Helden verteilen ihre SP und WP neu:



Orfen: 2 SP, 7 WP –> 1 SP, 1 WP

Thorn: 2 SP, 5 WP –> 3 SP, 4 WP

Aćh: 2 SP, 5 WP –> 4 SP, 8 WP

Forn: 2 SP, 14 WP –> 14 SP, 20 WP

Iril: 12 SP, 14 WP –> 1 SP, 15 WP

Darh: 4 SP, 9 WP –> 1 SP, 6 WP



Zug 21: Forn läuft für 1 Stunde (Zeitstein von 1 auf 2) von 64 auf 45.

Forn opfert den 45-Brunnen den Wunschbrunnen. Er wünscht sich schwächere Wölfe.



Zug 22: Iril wählt ihre Aktion "Runen befragen".

In Stunde 3 zeigt der Runenwürfel eine 1. Iril dreht die Runenscheibe um 1 +2 = 3 Speichen auf die 4-WP. Ihre WP steigen auf 15 +3 = 18.

In Stunde 4 (das Wolfssymbol wird ausgelöst; Merla läuft von 33 auf 30, Rutan von 43 auf 39 und Lonas von 64 auf 45) zeigt der Runenwürfel eine 2. Iril dreht die Runenscheibe um 2 +0 = 2 Speichen auf den blauen Runenstein. Ihre SP steigen auf 1 +1 = 2.



Zug 23: Darh wählt ihre Aktion "Knochen-Golem bewegen". Der Knochen-Golem läuft für 1 Stunde (Zeitstein von 8 auf 9, Darhs WP sinken auf 6 -2 = 4) von 41 auf 45.



Zug 24: Forn bekämpft Lonas auf 45.

In Kampfrunde 1 (Zeitstein von 4 auf 5, das Licht der fünften Stunde wird ausgelöst) hat Forn einen Kampfwert von 14 SP +6 (Knochen-Golem) +5 (Licht der fünften Stunde) + 5 (Würfel) = 30. Der gewünschte Kampfwert von 2x14 = 28 ist somit erreicht.

Die Wölfe sind gezähmt und jeder von ihnen ist glorreiche 14 SP wert! Forn ist der Wolfsfreund. Diese Verbindung soll nicht vergessen gehen, denn beinahe ein Jahrzehnt später wird Forn der erste von Lonas aufgespürte Dunkle Held werden – wobei Forn sich dann wieder vor Lonas fürchten wird.



Kurze Tangente: Theoretisch wäre auch ohne Wunschbrunnen eine Wolfsstärke von 14 (d.h. ein Zähmwert 4x14 = 56) möglich, aber nur mithilfe von unglaublichem Glück wie 14-SP-Thorn +7 (Turr) +6 (Knochen-Golem) +5 (Licht der fünften Stunde) +24 (6er-Pasch mit 4 Würfeln) = 56 oder aber mithilfe von Verbrauchsgegenständen wie 14-SP-Eara +7 (Turr) +6 (Knochen-Golem) +5 (Licht der fünften Stunde) +6 (minimum Wassergeist-Würfel) +6 (Trank der Hexe) +4 (Heilkraut) +4 (Heilkraut) +4 (Kampfaxt) = 56.



Zug 25: Iril wählt ihre Aktion "Runen befragen".

In Stunde 10 (Irils WP sinken auf 18 -2 = 16) dreht Iril die Runenscheibe um 3 Speichen auf den gelben Runenstein. Irils SP steigen auf 2 +1 = 3.



Zug 26: Darh beendet ihren Tag.



Zug 27: Forn läuft für 3 Stunden (Zeitstein von 5 auf 8, Forns WP sinken auf 20 -2 = 18) und 8 WP (Forns WP sinken auf 18 -8 = 10) von 45 über 40, 28 und 0 auf 5. Er nimmt im Vorbeigehen die beiden Bauern von 40 und 28 auf 0 mit. Zwei goldene Schilde mehr für die Rietburg!

Forn leert den 5-Brunnen. Forns WP steigen auf 10 +3 = 13.



Zug 28: Iril wählt ihre Aktion "Runen befragen".

In Stunde 9 (Irils WP sinken auf 16 -2 = 14) dreht Iril die Runenscheibe um 3 Speichen auf den grünen Runenstein. Irils SP steigen auf 3 +1 = 4.



Zug 29: Forn läuft für 1 Stunde (Zeitstein von 9 auf 10, Forns WP sinken auf 13 -2 = 11) über die Zwergentür von 5 auf 61. Er könnte zwar noch weiterrennen, wartet aber wohlweislich hier auf das Einwürfeln der C-Heilkräuter.



Zug 30: Iril beendet ihren Tag.



Zug 31: Forn beendet seinen Tag.\bigskip






\includegraphics[width=\textwidth]{Das Erbe des Wunderkindes/Bilder/Tag 1 Ende.jpg}



\textit{Ende von Tag 1.}

\textit{Korrektur: Der Takuri-Marker sollte bereits auf +2 liegen.}


\newpage
\section{Tag 2}


Sonnenaufgang:

Weiße Ereigniskarte 14 wird ausgelöst. Im Wachsamen Wald liegt eine Waffe aus den Trollkriegen versteckt. Auf 54 wird ein Messer eingewürfelt.

Die Kreaturen laufen, die Brunnen werden aufgefrischt, die Zwergentüren werden geschlossen und der Erzähler läuft auf C.

Legendenkarte C wird ausgelöst. Zwei Heilkräuter werden eingewürfelt auf 25 und 16. Ein Skral erscheint zufällig auf 36.

Die Bedrohung ist ... Trommelwirbel ... der Belagerungsturm! Er erscheint auf 84 und wird jeden Sonnenaufgang so viele Felder bewegt, wie Kreaturen auf Spielfeldern stehen.\bigskip


\includegraphics[width=\textwidth]{Das Erbe des Wunderkindes/Bilder/Tag 2 Anfang.jpg}


\textit{Anfang von Tag 2.}

\textit{Korrekturen: Der Takuri-Marker sollte bereits auf +3 liegen. Die beiden 4er-Heilkräuter sollten verdeckt sein. Und was macht denn der Dunkle Tempel da?}\bigskip




Zug 1: Orfen beendet seinen Tag. Er behält das Sonnenaufgang-Plättchen gleich.



Zug 2: Thorn beendet seinen Tag.



Zug 3: Aćh nutzt ihre tulgorische Steinflöte. Turr fliegt von 58 auf 16, wo am nächsten Tag der 36-Skral stehen wird.

Aćh beendet ihren Tag.



Zug 4: Forn läuft für 3 WP (Forns WP sinken auf 11 -3 = 8) von 61 auf 54. Er nimmt den 4er-Holzstamm und das Messer auf.



Zug 5: Iril wählt ihre Aktion "Runen befragen".

In Stunde 1 zeigt der Runenwürfel eine 3. Iril dreht die Runenscheibe um 3 +2 = 5 Speichen auf die 3-WP. Irils WP steigen auf 14 +3 = 17.

In Stunde 2 zeigt der Runenwürfel eine 2. Iril dreht die Runenscheibe um 2 +2 = 4 Speichen auf den grünen Runenstein. Irils SP steigen auf 4 +1 = 5.

In Stunde 3 zeigt der Runenwürfel eine 1. Iril dreht die Runenscheibe um 1 +0 = 2 Speichen auf die 4-WP. Irils WP steigen auf 17 +3 = 20.

In Stunde 4 zeigt der Runenwürfel eine 1. Iril dreht die Runenscheibe um 1 +1 = 2 Speichen auf den blauen Runenstein. Irils SP steigen auf 5 +1 = 6.

In Stunde 5 dreht Iril die Runenscheibe um 3 Speichen auf den gelben Runenstein. Irils SP steigen auf 6 +1 = 7.

In Stunde 6 dreht Iril die Runenscheibe um 3 Speichen auf den grünen Runenstein. Irils SP steigen auf 7 +1 = 8.

In Stunde 7 dreht Iril die Runenscheibe um 3 Speichen auf den blauen Runenstein. Irils SP steigen auf 8 +1 = 9.

In Stunde 8 (Irils WP sinken auf 20 -2 = 18) dreht Iril die Runenscheibe um 3 Speichen auf den gelben Runenstein. Irils SP steigen auf 9 +1 = 10.

In Stunde 9 (Irils WP sinken auf 18 -2 = 16) dreht Iril die Runenscheibe um 3 Speichen auf den gelben Runenstein. Irils SP steigen auf 10 +1 = 11.

In Stunde 10 (Irils WP sinken auf 16 -2 = 14) dreht Iril die Runenscheibe um 3 Speichen auf den gelben Runenstein. Irils SP steigen auf 11 +1 = 12.

Volle Stärkeleiste für Iril, schon zum zweiten Mal! :P



Zug 6: Darh läuft für 1 Stunde (Zeitstein von 0 auf 1) von 23 auf 31. Sie findet den Bruderschild.

Darh nutzt die erste Seite des Bruderschilds, um ihren 1 SP mit Irils 12 SP zu tauschen.



Zug 7: Forn läuft für 1 Stunde (Zeitstein von 1 auf 2) von 54 auf 53. Er findet einen 2er-Holzstamm und einen 2er-Waldpilz. Forns Gold steigt quasi auf 0 +2 = 2.



Zug 8: Iril wählt ihre Aktion "Runen befragen".

In Stunde 1 zeigt der Runenwürfel eine 2. Iril dreht die Runenscheibe um 2 +2 = 4 Speichen auf die 4-WP. Irils WP steigen auf 14 +3 = 17.

In Stunde 2 zeigt der Runenwürfel eine 3. Iril dreht die Runenscheibe um 3 +1 = 4 Speichen auf die 3-WP. Irils WP steigen auf 17 +3 = 20.

In Stunde 3 zeigt der Runenwürfel eine 1. Iril dreht die Runenscheibe um 2 +2 = 4 Speichen auf den grünen Runenstein. Irils SP steigen auf 1 +1 = 2.

In Stunde 4 dreht Iril die Runenscheibe um 3 Speichen auf den blauen Runenstein. Irils SP steigen auf 2 +1 = 3.

In Stunde 5 dreht Iril die Runenscheibe um 3 Speichen auf den gelben Runenstein. Irils SP steigen auf 3 +1 = 4.

In Stunde 6 dreht Iril die Runenscheibe um 3 Speichen auf den grünen Runenstein. Irils SP steigen auf 4 +1 = 5.

In Stunde 7 dreht Iril die Runenscheibe um 3 Speichen auf den blauen Runenstein. Irils SP steigen auf 5 +1 = 6.

In Stunde 8 (Irils WP sinken auf 20 -2 = 18) dreht Iril die Runenscheibe um 3 Speichen auf den gelben Runenstein. Irils SP steigen auf 6 +1 = 7.

In Stunde 9 (Irils WP sinken auf 18 -2 = 16) dreht Iril die Runenscheibe um 3 Speichen auf den gelben Runenstein. Irils SP steigen auf 7 +1 = 8.

In Stunde 10 (Irils WP sinken auf 16 -2 = 14) dreht Iril die Runenscheibe um 3 Speichen auf den gelben Runenstein. Irils SP steigen auf 8 +1 = 9.



Zug 9: Darh läuft für 1 Stunde (Zeitstein von 2 auf 3) von 31 auf 25. Sie findet ein 4-Heilkraut.

Darh nutzt das Hadrische Stundenglas, um den den Zeitstein von 3 zurück auf 0 zu schieben.



Zug 10: Forn läuft für 1 Stunde (Zeitstein von 0 auf 1) von 53 auf 55. Er leert den 55-Brunnen. Forns WP steigen auf 8 +3 = 11.

Forn nimmt den 6er-Holzstamm auf.

Iril schickt die Eule (ohne Inhalt) an Forn

Forn frischt die Eule für 3 WP auf. Forns WP sinken auf 11 -3 = 8.

Forn schickt den Trank der Hexe samt Eule zurück an Iril. Iril legt die schlafende Eule und den Trank der Hexe auf 23 ab.

Forn nimmt das Brot auf 55 auf und schon sind seine kleinen Ablagefelder wieder gefüllt.



Zug 11: Iril wählt ihre Aktion "Runen befragen".

In Stunde 1 zeigt der Runenwürfel eine 3. Iril dreht die Runenscheibe um 3 +1 = 4 Speichen auf die 4-WP. Irils WP steigen auf 14 +3 = 17.

In Stunde 2 zeigt der Runenwürfel eine 1. Iril dreht die Runenscheibe um 1 +1 = 2 Speichen auf den blauen Runenstein. Irils SP steigen auf 9 +1 = 10.

In Stunde 3 zeigt der Runenwürfel eine 3. Iril dreht die Runenscheibe um 3 +0 = 3 Speichen auf den gelben Runenstein. Irils SP steigen auf 10 +1 = 11.

In Stunde 4 zeigt der Runenwürfel eine 2. Iril dreht die Runenscheibe um 2 +2 = 4 Speichen auf die 4-WP. Irils WP steigen auf 17 +3 = 20.

In Stunde 5 zeigt der Runenwürfel eine 2. Iril dreht die Runenscheibe um 2 +0 = 2 Speichen auf den blauen Runenstein. Irils SP steigen auf 11 +1 = 12.

Volle Stärkeleiste für Iril, zum dritten Mal! :D



Zug 12: Darh läuft für 1 Stunde (Zeitstein von 1 auf 2) von 25 auf 23.

Darh übergibt Iril den Bruderschild und nimmt die schlafende Eule auf. Iril nutzt die zweite Seite des Bruderschilds, um ihre 12 SP mit Orfens 1 SP zu tauschen.



Zug 13: Forn läuft für 1 WP (Forns WP sinken auf 8 -1 = 7) von 55 auf 51. Er findet 2 1er-Waldpilze. Forns Gold steigt quasi auf 2 +2 = 4.



Zug 14: Iril wählt ihre Aktion "Runen befragen".

In Stunde 6 dreht Iril die Runenscheibe um 3 Speichen auf den gelben Runenstein. Irils SP steigen auf 1 +1 = 2.

In Stunde 7 dreht Iril die Runenscheibe um 3 Speichen auf den grünen Runenstein. Irils SP steigen auf 2 +1 = 3.

In Stunde 8 (Irils WP sinken auf 20 -2 = 18) dreht Iril die Runenscheibe um 3 Speichen auf den blauen Runenstein. Irils SP steigen auf 3 +1 = 4.

In Stunde 9 (Irils WP sinken auf 18 -2 = 16) dreht Iril die Runenscheibe um 3 Speichen auf den gelben Runenstein. Irils SP steigen auf 4 +1 = 5.

In Stunde 10 (Irils WP sinken auf 16 -2 = 14) dreht Iril die Runenscheibe um 3 Speichen auf den grünen Runenstein. Irils SP steigen auf 5 +1 = 6.



Zug 15: Darh beendet ihren Tag.



Zug 16: Forn läuft für 1 WP (Forns WP sinken auf 7 -1 = 6) von 51 auf 48. Das 48-Nebelplättchen versteckte eine Ereigniskarte.

Weiße Ereigniskarte 16 wird ausgelöst. Im hohen Rietgras verstecken sich Waffen aus der großen Schlacht um die Rietburg (im Lied des Königs?). Bei Betrams Hof (24) wird ein Messer eingewürfelt.



Zug 17: Iril wählt ihre Aktion "Runen befragen".

In Stunde 3 zeigt der Runenwürfel eine 1. Iril dreht die Runenscheibe um 1 +0 = 1 Speiche auf die 4-WP. Irils WP steigen auf 14 +3 = 17.

In Stunde 4 zeigt der Runenwürfel eine 3. Iril dreht die Runenscheibe um 3 +1 = 4 Speichen auf die 3-WP. Irils WP steigen auf 17 +3 = 20.

In Stunde 5 zeigt der Runenwürfel eine 1. Iril dreht die Runenscheibe um 1 +0 = 1 Speiche auf den gelben Runenstein. Irils SP steigen auf 6 +1 = 7.

In Stunde 6 dreht Iril die Runenscheibe um 3 Speichen auf den grünen Runenstein. Irils SP steigen auf 7 +1 = 8.

In Stunde 7 dreht Iril die Runenscheibe um 3 Speichen auf den blauen Runenstein. Irils SP steigen auf 8 +1 = 9.

In Stunde 8 (Irils WP sinken auf 20 -2 = 18) dreht Iril die Runenscheibe um 3 Speichen auf den gelben Runenstein. Irils SP steigen auf 9 +1 = 10.

In Stunde 9 (Irils WP sinken auf 18 -2 = 16) dreht Iril die Runenscheibe um 3 Speichen auf den gelben Runenstein. Irils SP steigen auf 10 +1 = 11.

In Stunde 10 (Irils WP sinken auf 16 -2 = 14) dreht Iril die Runenscheibe um 3 Speichen auf den gelben Runenstein. Irils SP steigen auf 11 +1 = 12.

Volle Stärkeleiste für Iril, zum vierten Mal! :P

Iril legt alle drei Runensteine auf 23 ab.



Zug 18: Forn läuft für 4 WP (Forns WP sinken auf 6 -4 = 2) von 48 auf 0.

Forn lädt alle Holzstämme ab.

Die Fürstenaufgabe ist erfüllt! :D

Forn nutzt die andorische Flöte. Der Würfel zeigt eine 2! Forns WP steigen auf 2 +2 = 4. Da Orfen auf 2 angrenzend zu 0 schläft, steigen auch Orfens WP auf 1 +2 = 3.



Zug 19: Iril beendet ihren Tag.



Zug 20: Forn beendet seinen Tag.\bigskip





\includegraphics[width=\textwidth]{Das Erbe des Wunderkindes/Bilder/Tag 2 Ende.jpg}

\textit{Ende von Tag 2.}

\newpage
\section{Tag 3}


Sonnenaufgang:

Weiße Ereigniskarte 6 wird ausgelöst. Der gute Wille und die Kraft des andorischen Volkes begleitete die Helden – der gute Wille begleitet Aćh und die Kraft begleitet Orfen. Aćhs WP steigen auf 8 +2 = 10. Orfens SP steigen auf 12 +1 = 13.

Die Kreaturen laufen, davon ein Gor und ein Skral direkt in die Burg.

Der 55-Brunnen werden aufgefrischt. Die 61-Zwergentür wird geschlossen. Es sind aktuell noch 9 Kreaturen im Spiel. Der Belagerungsturm läuft 9 Schritte von 84 auf 45.

Der Erzähler läuft auf D.\bigskip


\includegraphics[width=\textwidth]{Das Erbe des Wunderkindes/Bilder/Tag 3 Anfang.jpg}

\textit{Anfang von Tag 3.}\bigskip


Das Tagesmotto dreht sich heute ganz ums Sammeln von Gold durch Vertreibung dunkler Kreaturen. Und darum, den Belagerungsturm mithilfe des Sternenschilds von der Rietburg fernzuhalten.



Forn nutzt die andorische Flöte. Der Würfel zeigt eine 6! Forns WP steigen auf 4 +6 = 10. Da Orfen auf 2 angrenzend zu 0 steht, steigen auch Orfens WP auf 3 +6 = 9.



Zug 1: Orfen läuft für 1 Stunde (Zeitstein von 0 auf 1) von 2 auf 6.



Zug 2: Thorn beendet seinen Tag. Thorn erhält das Sonnenaufgang-Plättchen.



Zug 3: Aćh passt (Zeitstein von 1 auf 2). Sie will gleich noch Turr besänftigen, wenn der Erzähler E erreicht.



Zug 4: Forn, der Wolfsfreund, wählt die Aktion "Wölfe bewegen". Rutan läuft für 1 Stunde (Zeitstein von 2 auf 3) von 39 auf 6.

Darh nutzt das Hadrische Stundenglas, um den Zeitstein von 3 zurück auf 0 zu schieben. Dann legt sie das Stundenglas auf 23 ab. Die kleinen Ablagefelder der lange wachen Helden dürften heute knapp werden.

Darh schickt die Eule mit den drei Runensteinen von 23 an Orfen.



Zug 5: Iril beendet ihren Tag bereits. Ausnahmsweise. ;)



Zug 6: Darh muss passen, da sie noch nicht weiß, wo die E-Kreaturen auftauchen werden. Darh wählt ihre Aktion "Knochen-Golem bewegen". Der Knochen-Golem läuft für 1 Stunde (Zeitstein von 0 auf 1) eine fröhliche Runde von 45 über 44/46/64 auf 45.



Zug 7: Orfen bekämpft den 6-Troll.

In Kampfrunde 1 (Zeitstein von 1 auf 2) hat Orfen einen Kampfwert von 13 SP +14 (Rutan) +10 (großer schwarzer Würfel) = 37.

Der 6-Troll hat einen Kampfwert von 18 SP +6 (Würfel) = 24.

Der Troll ist besiegt.

Belohnung: Orfens Gold steigt auf 0 +6 = 6.

Der Erzähler läuft auf E. Eine Legendenkarte wird ausgelöst. Der Knochen-Golem kommt zurück auf 80. Dafür erscheinen mehr Kreaturen im Südlichen Wald: Gor auf 22, Trolle auf 24 und 34. Und ein bösartiger Wardrak verlässt via 67 das Graue Gebirge.



Zug 8: \textit{Vorsicht! Brennt die Flamme des Takuri zu heiß, endet sein Lebenszyklus. Ein Glück, dass die tapfere Aćh ihn immer wieder beruhigen kann.} Aćh beruhigt Turr. Der Takuri-Marker sinkt von +7 auf +5. Aćhs WP sinken auf 10 -5 = 5.

Aćh beendet ihren Tag.



Zug 9: Forn, der Wolfsfreund, wählt die Aktion "Wölfe bewegen". Merla läuft für 1 Stunde (Zeitstein von 2 auf 3) von 30 auf 24.



Zug 10: Darh läuft für 1 Stunde (Zeitstein von 3 auf 4) von 23 auf 24. Darh nimmt das Messer von 24 auf.



Zug 11: Orfen läuft für 2 Stunden (Zeitstein von 4 auf 6) von 6 auf 16. Das 16-Nebelplättchen versteckte einen Stärkepunkt. Orfens SP steigen auf 13 +1 = 14.

Orfen findet ein 4er-Heilkraut. Er frischt die Eule auf. Orfens WP sinken auf 9 -3 = 6. Orfen schickt das 4er-Heilkraut mit der Eule an Darh.



Zug 12: Forn, der Wolfsfreund, wählt die Aktion "Wölfe bewegen". Rutan läuft für 1 Stunde (Zeitstein von 6 auf 7) von 6 auf 13.



Zug 13: Darh bekämpft den 24-Troll.

In Kampfrunde 1 (Zeitstein von 0 auf 1) hat Darh einen Kampfwert von 12 SP +14 (Merla) +6 (Würfel) = 32.

Der 24-Troll hat einen Kampfwert von 18 SP +6 (Würfel) = 24.

Die WP des 24-Trolls sinken auf 12 -8 = 4.

In Kampfrunde 2 (Zeitstein von 1 auf 2) hat Darh einen Kampfwert von 12 SP +14 (Merla) +3 (Würfel) = 29.

Der 24-Troll hat einen Kampfwert von 18 SP +4 (Würfel) = 22.

Der 24-Troll ist besiegt.

Belohnung: Darhs Gold steigt auf 5 +6 = 11.

Darh beschwört ihren Knochen-Golem. Er wird samt Troll auf 36 eingewürfelt. Das Golem-Symbol kommt auf I.

Der Erzähler läuft auf F.



Zug 14: Orfen bekämpft den 16-Skral.

In Kampfrunde 1 (Zeitstein von 2 auf 3) hat Orfen einen Kampfwert von 14 SP +7 (Turr) +10 (großer schwarzer Würfel) = 31.

Der 16-Skral hat einen Kampfwert von 10 SP +2 (Würfel) = 12.

Der 16-Skral ist besiegt.

Belohnung: Orfens Gold steigt auf 6 +4 = 10.

Der Erzähler läuft auf G. \textit{Nun sieht der Takuri Orfen noch einmal an, und in einem kurzen Augenblick rotglühenden Feuers beendet er sein Leben, so wie er es schon so oft beendet hat.} Er sinkt auf +0.

In dieser Partie denken die Helden auf G über Gildas Lieder nach. Da gleich zwei Helden auf 72 stehen (bzw. liegen), lesen wir direkt auf der Hinweiskarte "Gildas Lied" weiter. Gilda singt den schlafenden Helden das Lied von Nehal und verweist auf den Krallenfelsen (30). Doch die Zeit drängt! Auch Varkur ist auf der Suche nach dem Sternenschild. Haben wir ihn bis I (zwei Erzählerbewegungen) nicht gefunden, wird er ihn triumphierend an sich reißen!



Zug 15: Forn wählt die Aktion "Wölfe bewegen". Lonas läuft für 1 Stunde (Zeitstein von 3 auf 4) von 45 auf 38.



Zug 16: Darh läuft für 3 Stunden (Zeitstein von 4 auf 7) von 24 auf 30. Sie untersucht den Krallenfelsen und erkennt ... 30 +10 +8 = 48!

Mag dies die Stelle sein, an welcher der diebische Rudnar einst während der Trollkriege den gestohlenen Sternenschild versteckte?



Zug 17: Orfen läuft für 1 Stunde (Zeitstein von 0 auf 1) von 16 auf 38.



Zug 18: Forn läuft für 4 Stunden (Zeitstein von 1 auf 5) von 0 auf 48. Triumphierend nimmt er den Sternenschild auf. Varkur dampft wutschnaubend davon.

Forn nutzt den Sternenschild, um das Brunnen-Symbol abzudecken. Dann bewegt sich der Belagerungsturm beim nächsten Sonnenaufgang nicht.



Zug 19: Darh läuft für 1 Stunde (Zeitstein von 5 auf 6) von 30 auf 29.



Zug 20: Orfen bekämpft den 38-Skral.

In Kampfrunde 1 (Zeitstein von 6 auf 7) hat Orfen einen Kampfwert von 14 SP +14 (Lonas) +10 (Würfel) = 38.

Der 38-Skral hat einen Kampfwert von 10 SP +5 (Würfel) = 15.

Der 38-Skral ist besiegt.

Belohnung: Orfens Gold steigt auf 10 +4 = 14.

Der Erzähler läuft auf H.



Zug 21: Forn läuft für 2 WP (Forns WP sinken auf 10 -2 = 8) von 48 auf 13. Das 13-Nebelplättchen versteckte 1 Gold. Forns Gold steigt auf 4 +1 = 5.



Zug 22: Darh läuft für 1 Stunde (Zeitstein von 0 auf 1) von 29 auf 28.



Zug 23: Orfen beendet seinen Tag.



Zug 24: Forn bekämpft den 13-Troll.

In Kampfrunde 1 (Zeitstein von 1 auf 2) hat Forn einen Kampfwert von 14 SP +14 (Rutan) +12 (Würfel) = 40.

Der 13-Troll hat einen Kampfwert von 18 SP +5 (Würfel) = 23.

Der 13-Troll ist besiegt.

Belohnung: Forns Gold steigt auf 5 +6 = 11.

Der Erzähler läuft auf I. Der Knochen-Golem kehrt zurück auf 80.



Zug 25: Darh läuft für 1 Stunde (Zeitstein von 2 auf 3) von 28 auf 36.



Zug 26: Forn, der Wolfsfreund, wählt die Aktion "Wölfe bewegen". Lonas läuft für 1 Stunde (Zeitstein von 3 auf 4) von 38 auf 36. Der Wolf, der schwarze Wolf, hatte Darhs Witterung aufgenommen. ;)



Zug 27: Darh bekämpft den 36-Troll.

In Kampfrunde 1 (Zeitstein von 4 auf 5, das Licht der fünften Stunde wird ausgelöst) hat Darh einen Kampfwert von 12 SP +14 (Lonas) +5 (Licht der fünften Stunde) +4 (Würfel) = 35.

Der 36-Troll hat einen Kampfwert von 18 SP +5 (Würfel) = 23.

Der 36-Troll ist besiegt.

Belohnung: Darhs Gold steigt auf 11 +6 = 17.

Darh beschwört ihren Knochen-Golem, schon zum zweiten Mal bei diesem Troll. Der Knochen-Golem wird samt Troll auf 32 eingewürfelt. Das Golem-Symbol kommt auf M.

Der Erzähler läuft auf J.



Zug 28: Forn, der Wolfsfreund, wählt die Aktion "Wölfe bewegen". Merla läuft für 1 Stunde (Zeitstein von 5 auf 6) von 24 auf 72.



Zug 29: Darh wählt ihre Aktion "Knochen-Golem bewegen". Der Knochen-Golem läuft für 1 Stunde (Zeitstein von 6 auf 7) von 32 auf 72.



Zug 30: Forn beendet seinen Tag.



Zug 31: Darh beendet ihren Tag.\bigskip


\includegraphics[width=\textwidth]{Das Erbe des Wunderkindes/Bilder/Tag 3 Ende.jpg}

\textit{Ende von Tag 3.}

\newpage
\section{Tag 4}


Sonnenaufgang:

Weiße Ereigniskarte 1 wird ausgelöst. Orfen setzt einen Zeitstein von 0 auf 2, um die Kampfaxt zu erhalten.

Kreaturen laufen, davon ein Gor auf den letzten freien goldenen Schild.

Die Brunnen werden wegen des Sternenschilds nicht aufgefrischt. Der Belagerungsturm bleibt auf somit 45 stehen.

Der Erzähler läuft auf K.\bigskip

\includegraphics[width=\textwidth]{Das Erbe des Wunderkindes/Bilder/Tag 4 Anfang.jpg}

\textit{Anfang von Tag 4.}

\textit{Korrekturen: Das Licht der fünften Stunde, Darhs Eule und Forns Flöte sollten alle bereits wieder aufgefrischt sein. Hach, gegen Ende schleicht sich wieder der Fehlerteufel ein :(}\bigskip


Iril nimmt die Fackel von Cavern, das Hadrische Stundenglas und den Trank der Hexe von 23 auf.

Iril nutzt die Fackel von Cavern, um den Troll von 23 auf 72 zu scheuchen.

Aćh nutzt ihre tulgorische Steinflöte. Turr fliegt von 16 auf 72.

Jawohl, der entscheidende Kampf wird gegen einen gewöhnlichen Troll und nicht gegen den Belagerungsturm sein. :D



Zug 1: Thorn passt (Zeitstein von 2 auf 3). Jemand muss auf 72 warten, um Forns Gold entgegenzunehmen, und dieser jemand ist er.

Iril nutzt das Hadrische Stundenglas, um den Zeitstein von 3 zurück auf 0 zu schieben.



Zug 2: Aćh läuft für 1 Stunde (Zeitstein von 0 auf 1) von 72 auf 18.



Zug 3: Forn läuft für 3 Stunden (Zeitstein von 1 auf 4) von 13 auf 72.

Thorn erhält Forns Gold (11, davon 4 Waldpilze).



Zug 4: Iril läuft für 1 Stunde (Zeitstein von 4 auf 5) von 23 auf 72.



Zug 5: Darh läuft für 1 Stunde (Zeitstein von 5 auf 6) von 36 auf 18.

Darh kauft sich 2 SP. Darhs Gold sinkt auf 17 -4 = 13.



Zug 6: Orfen läuft für 2 Stunden (Zeitstein von 6 auf 8, Orfens WP sinken auf 6 -2 = 4) von 38 auf 18.

Orfen kauft sich einen Helm. Orfens Gold sinkt auf 14 -2 = 12.

Aćh erhält den Rest von Orfens Gold. Aćhs Gold steigt auf 0 +12 = 12.

Aćh kauft sich 6 SP. Aćhs Gold sinkt auf 12 -12 = 0. Aćhs SP steigen auf 4 +6 = 10.



Zug 7: Thorn läuft (mit Forns 11 Gold) für 1 Stunde (Zeitstein von 0 auf 1) von 72 auf 18.

Thorn erhält Darhs Gold. Thorns Gold steigt auf 11 +13 = 24.

Thorn kauft sich 11 SP und einen Helm. Thorns Gold sinkt auf 24 -22 -2 = 0. Thorns SP steigen auf 3 +11 = 14.

Alle Stärkeleisten sind voll! :D



Zug 8: Aćh läuft für 1 Stunde (Zeitstein von 1 auf 2) von 18 auf 72.



Zug 9: Forn, der Wolfsfreund, wählt die Aktion "Wölfe bewegen". Rutan läuft für 1 Stunde (Zeitstein von 2 auf 3) von 13 auf 72. Lonas läuft für 1 Stunde (Zeitstein von 3 auf 4) von 36 auf 72.

Alle Unterstützer stehen bereit! :D



Zug 10: Iril wählt ihre Aktion "Runen befragen", um sich optimal auf den Kampf einzustimmen.

In Stunde 5 zeigt der Runenwürfel eine 1. Iril dreht die Runenscheibe um 1 +1 = 2 Speichen auf den 2-Brunnen. Es gibt zwar keine geleerten Brunnen aufzufrischen, aber dafür ist sie perfekte 3 Speichen vom Kampfwert 5 entfernt.



Zug 11: Darh läuft für 1 Stunde (Zeitstein von 5 auf 6) von 18 auf 72.



Zug 12: Orfen läuft für 1 Stunde (Zeitstein von 6 auf 7) von 18 auf 72.



Zug 13: Thorn läuft für 1 Stunde (Zeitstein von 0 auf 1) von 18 auf 72.

Alle Helden sind in Position! :P



Die andorische Flöte wird an unsere Flötenspielerin Aćh übergeben. Aćh spielt die Flöte. Der Würfel zeigt eine 4. Aćhs WP steigen auf 5 +4 = 9. Irils WP steigen auf 14 +4 = 18. Darhs WP steigen auf 4 +4 = 8. Orfens WP steigen auf 4 +4 = 8. Thorns WP steigen auf 4 +4 = 8. Forns WP steigen auf 8 +4 = 12.

Forn teilt sein Brot. Alle Helden erhalten 6x2 = 12 WP.

Alle Willensleisten sind voll! :D



Die Gegenstände werden ein bisschen umverteilt, aber nicht groß.

Aćh: 10 SP, 19 WP, drei Runensteine, Eule

Iril: 12 SP, 20 WP, Trank der Hexe (den Iril auf ihren Speichenwert anwenden darf\footnote{Aus der Taverne, "Magische Helden und ein paar allgemeine Fragen
": [Giftknödel:] Irils Speichenwert entspricht einem normalen Würfelwert. Demnach darf sie den Trank der Hexe darauf verwenden.}), Hadrisches Stundenglas, Fackel von Cavern

Darh: 14 SP, 20 WP, Helm, 2x Messer

Orfen: 14 SP, 20 WP, Helm, Kampfaxt

Thorn: 14 SP, 20 WP, Helm, andorische Flöte und tulgorische Steinflöte, warum auch nicht.

Forn: 14 SP, 20 WP, 2x 4er-Heilkraut, Sternenschild\bigskip



\includegraphics[width=\textwidth]{Das Erbe des Wunderkindes/Bilder/Tag 4 Endkampf.jpg}

\textit{Trollkampf an Tag 4. Schräg liegende Gegenstände werden soeben verbraucht.}\bigskip

Zug 14: Aćh lädt alle Helden zum gemeinsamen Kämpfen gegen den 72-Troll ein.

In Kampfrunde 1 (Zeitstein von 1 auf 7, das Licht der fünften Stunde wird ausgelöst) haben die Helden einen Kampfwert von 10 SP (Aćh) +14 SP (Forn) +12 SP (Iril) +14 SP (Darh) +14 SP (Orfen) +7 (Orfens Trollbonus) +14 SP (Thorn) +14 (Lonas) +14 (Rutan) +14 (Merla) +7 (Turr) +6 (Knochen-Golem) +5 (Licht der fünften Stunde) +12 (Aćhs großer schwarzer Würfel) +18 (Forns Pasch) +8 (2 Heilkräuter) +5 (Iril dreht ihre Runenscheibe um 3 Speichen auf den 5-Brunnen) +5 (Trank der Hexe) +18 (Darhs Pasch) +4 (2 Messer) +18 (Orfens Pasch) +4 (Kampfaxt) +24 (Thorns Pasch) = 261.\bigskip



Ich wiederhole dies gern noch etwas lauter:

{\Large Kampfwert 261!}

Ganz legal (sofern nicht verrechnet)! Und klar, die vielen 6er-Päsche sind sehr unwahrscheinlich, aber die ganzen restlichen Kampfpunkte sind gar nicht so unrealistisch zu erreichen!\bigskip



Der 72-Troll ist natürlich besiegt, der Erzähler läuft auf L und da die Fürstenaufgabe schon längstens erfüllt ist, müssten wir nun nur noch den Belagerungsturm besiegen gehen und könnten die Legende sogar noch gewinnen! :P\bigskip



Falls es heute nicht mehr für den Belagerungsturm reichen sollte, können wir immer einfach...

... den Belagerungsturm mit der zweiten Seite des Sternenschilds am Laufen hindern, ...

... Forn mit der Fackel von Cavern zur Burg flitzen lassen, für 1 Stunde von 0 nach 5 hoppeln, den 5-Brunnen den Wunschbrunnen opfern und sich aufgefrischte Gaben der Andori wünschen, danach für 1 Stunde von 5 nach 4 laufen und den 4-Gor mit der Fackel ein Feld zurückscheuchen, damit am nächsten Sonnenaufgang nur der 1-Gor in die Burg läuft, ...

... einen Helden für 2 Stunden nach 35 hoppeln lassen, den 35-Brunnen den Wunschbrunnen opfern und sich einen freien goldenen Schild für den 1-Gor wünschen ...

... und schon haben unsere tapferen Helden auf M noch einen ganzen Tag Zeit, den Belagerungsturm zu erreichen und zu demolieren. So geht das!






\newpage
\section{Verbesserungspotential}







Zum Nachtmodus:



Ich wünschte, ich hätte mir die Legendenkarte zum Heldenlager rechtzeitig nochmals durchgelesen. Der Nachtmodus hätte hier wirklich tolles Potential gehabt.

Im Sonnenaufgang von Tag 1 auf Tag 2 (bzw. in der Dämmerung von Nacht 1 auf Nacht 2) erhalten alle Helden im Heldenlager je 1 Stärkepunkt. Potentielle +6 SP sind schon sehr praktisch, aber es kommt noch besser: In der Dämmerung von Nacht 3 auf Nacht 4 dürfen alle Helden im Heldenlager untereinander SP und WP tauschen. Das hätte uns erlaubt, Irils volle Stärkeleiste am Ende auch noch mit Thorns leerer Leiste zu tauschen und Irils Stärkeleiste ein fünftes Mal mit Runenmagie hochzupowern. Dann hätten wir erheblich weniger Gold sammeln müssen.

Den endgültigen Kampfwert hätte das allerdings nicht verbessert.\bigskip







Zu Orfen:



Wenn man irgendwie den Belagerungsturm daran hinderte, die 80er-Felder zu verlassen (was bedingen würde, am Tag seines Auftauchens den Erzähler von C auf mindestens G zu pushen und sofort den Sternenschild zu finden, oder aber beinahe alle Kreaturen vom Spielplan zu räumen) und wenn man danach irgendwie alle Helden und Begleiter mit vollen Stärkeleisten und voller Ausrüstung auf ein 80er-Feld mit Belagerungsturm darauf brächte, könnte man den Endkampf dort stattfinden lassen.

Da man explizit nicht den Belagerungsturm, sondern den ihn herumschiebenden Troll bekämpft, sollte Orfen dann einen Stärkebonus von +8 statt +7 kriegen können und den maximalen Kampfwert von obigem Spielbericht um 1 Punkt übertreffen.

Ob das wirklich möglich ist, steht allerdings noch in den Sternen. Eventuell könnte man einen Bauern der Mini-Erweiterung "Die Taverne von Andor" opfern, um den Erzähler von B auf A zurückzusetzen und so einen Tag zu gewinnen, ehe der Belagerungsturm erscheint. Dann dürfte es bei den goldenen Schilden der Rietburg jedoch knapp werden.



Alternativ könnte man auch versuchen, einen gewöhnlichen Troll auf ein 80er-Feld zu befördern. Doch da Orfen mit Helm auf dem Trollfeld stehen muss, und da Feld 83 für Trolle unerreichbar ist (selbst mit Ijsdurs Eisbrücken\footnote{Aus der Taverne, "Einige Fragen zu den Magischen Helden": [Doro:] Selbst der magische Ijsdur kann mit der Fackel eine Kreatur nur auf ein wirklich angrenzendes Feld versetzen. }), würde das bedingen, einen gewöhnlichen Troll (evtl. durch Darh auf Feld 66 eingewürfelt) mit Sternenschild, Zwergenseil und/oder dem Geisterfeuer des Dunklen Tempels tagelang am Laufen zu hindern und mit der Fackel von Cavern immer näher in Richtung 81 schubsen (wobei man immerhin einmal einen Brunnen opfern könnte, um sich eine Auffrischung der Gaben der Andori zu wünschen und somit zwei Fackelnutzungen an diesem Tag zu kriegen).

Ob das wirklich möglich ist, steht allerdings ebenfalls noch in den Sternen. Ich bin mir nicht einmal sicher, ob es regeltechnisch erlaubt wäre, mit Zwergenseil gefangene oder hingelegte Kreaturen durch die Fackel von Cavern zu bewegen.\bigskip







Zu Bragor:



In meinen ersten Theorien zum maximalen Kampfwert war Bragor noch mit von der Partie, damit er sich von Iril jeweils an einem Brunnen hochpowern lassen konnte. Das Witzige daran wäre, dass man dann mit dem letzten Lagerfeuer (oder dem Heldenlager aus dem Nachtmodus) gleichzeitig zwei Helden hätte vollpowern können. In Kombination der beiden Lager oder eines Lagers mit dem 2x SP-Tauschen des Bruderschilds hätte man somit am Ende keinen einzigen Stärkepunkt mit Gold kaufen müssen.

Aber Aćhs maximale 10 SP +7 (Turr) +12 (schwarzer Würfel) = 29 sind eindeutig besser als Bragors maximale 14 SP +12 (schwarzer Würfel) = 26, darum wurde des Taren große Heldentafel gegen der Takuri-Hüterin große Heldentafel eingetauscht.\bigskip







Zu Iril-Alternativen:



Aktuell basiert unsere Strategie zum Füllen der Stärkeleisten hauptsächlich auf Irils Runenscheibe. Iril ist also zwingend mit von der Partie. Im Endkampf steuert Iril 12 SP +5 (Speiche) + 5 (TdH) = 22 bei. Wenn es uns gelänge, alternative Strategien zum Füllen der Stärkeleisten zu finden, könnten wir Iril vielleicht durch einen noch ein klein wenig stärkeren Helden ersetzen.



Kram kann auf Feld 71 mit Gold 1:1 SP kaufen. Genau wie Iril kann auch er einfacher generierte SP über Bruderschild, letztes Lagerfeuer und Heldenlager (Nachtmodus) mit anderen Helden tauschen. Vielleicht ist es irgendwie möglich, genügend Gold zu sammeln, damit Kram statt Iril von der Partie sein könnte.

Falls jemand sich daran versuchen will, könnte die Mini-Erweiterung "Koram, der Gor-Häuptling" helfen. Solange Koram im Spiel ist, haben Gors zwar die SP von Skralen, aber auch ihre Belohnung.

Kram würde im Endkampf maximal 14 SP +6 (Würfel) +6 (TdH, ist ja kein Helm mehr übrig) = 26 liefern, also ganze 4 Punkte besser als Iril.



Noch besser wäre es natürlich, Kheela statt Iril oder Kram antreten zu lassen. Kheela hat im Endkampf maximal 14 SP +7 (Wassergeist-Würfel) +7 (TdH) = 28, ganze 6 Punkte besser als Iril.

Ich habe aber keine Ahnung, wie man ohne Zwergentricks alle Stärkeleisten (fast) vollmachen will.

Ich studierte eine Zeit lang an einem möglichen Loop mit der Mini-Erweiterung "Die Magie von Choranat" herum, bei der zwei Helden abwechslungsweise alleine eine viel stärkere Kreatur (z.B. einen Troll) angriffen und den Kampf nach der ersten Kampfrunde jeweils wieder abbrächen. Der erste Held hätte nur 1 SP und würde immer wieder auf 0 WP fallen, ohne davon wirklich betroffen zu sein (er könnte so sogar beliebig viele Überstunden machen), während der zweite Held jeweils einen Schild vom Barz-Bannfeuer erhalten würde, um sich vor dem WP-Verlust zu schützen, und immer beim Ijsdur-Bannfeuer 1 SP erhalten würde.

Glücklicherweise las ich die Mini-Erweiterung nochmals genau durch und sah das kleingedruckte "eignet sich nicht für das Spiel mit 5 und 6 Helden", ehe ich noch länger versucht hätte, diesen Loop hier umzusetzen.

Nichtsdestotrotz: 1 SP pro 4 Stunden zu generieren, ist wirklich nicht schlecht, und der Loop könnte im Spiel mit 4 oder weniger Helden gut funktionieren, was die Mini-Erweiterung in meinen Augen gleich noch ein Stückchen mehr OP macht. ;)\bigskip







Zu Kirr:



Kirr könnte maximal 14 SP +6 (Zeitstein auf 10) +6 (Würfel) +6 (TdH) = 32 liefern, also satte 10 Punkte mehr als Iril.

Und selbst wenn wir bei der Iril-Strategie bleiben, wäre Kirr mit 14 SP +6 (Zeitstein auf 10. Stunde) +12 (schwarzer Würfel) = 32 Kampfpunkten um ganze 3 Punkte besser als Aćhs 10 SP +7 (Turr) +12 (schwarzer Würfel) = 29.

Wir sollten also Aćh durch Kirr ersetzen.



Die Frage, ob Kirr im Grundspiel (und den Erweiterungen) problemlos gespielt werden kann, würde ich mit einem eindeutigen Ja beantworten, da seine Sonderfähigkeit meines Wissens nach zu keiner einzigen Regelunklarheit führt und da er mir weder zu stark noch zu schwach erscheint.



Doch darf man Kirr höchstoffiziell im Grundspiel nutzen? Die Antwort ist ein ganz klares: Jein. ;)

Von offizieller Seite gibt also meiner Einschätzung nach weder ein eindeutiges Ja noch ein eindeutiges Nein.\footnote{Aus der Taverne, "Ein Schiff so schwarz wie die Nacht ...": [Michael Menzel:] Kirr ist speziell für diese Legende entwickelt worden und ich habe ihn nicht in anderen getestet. Daher wird er auch nicht unabhängig zum Download bereitgestellt, sondern als Teil dieser Legende. Eine weibliche Seite gibt es nicht. Nicht nur aus Zeitgründen, sondern auch weil seine Figur ja bereits in der Reise in den Norden enthalten ist. Ich denke, dass es ihn nur zum Download geben wird, denn wie gesagt, wie er sich in anderen Legenden spielt, kann ich gar nicht sagen.} Kirr wurde zwar nur in der Rückkehr der Schwarzen Kogge getestet, lässt sich aber im Gegensatz zu manchen Dunklen Helden im Norden ohne jegliche Regelambiguität in andere Legenden übertragen.



Nachdem mir Kirr erst relativ spät eingefallen ist (heieiei, den Santa Gor hatte ich zu Beginn extra bedacht, aber Kirr schlicht vergessen :oops: ), und da es nicht 100\% sicher ist, ob er höchstoffiziell im Sternenschild eingesetzt werden kann, lasse ich ihn vorerst außen vor und belasse es bei der Bemerkung, dass man mit Kirr statt Aćh ziemlich sicher noch 3 Kampfpunkte mehr rausschlagen könnte.

}



\begin{chapterbox}
    \chapter{Ein gemütliches Gasthaus voller Geschichten (2025)}
    \label{Ein gemütliches Gasthaus voller Geschichten (2025)}
    \az{82}
    
    Die Wirtin Gilda richtet eine offizielle Erzählbühne in ihrer Taverne zum Trunkenen Troll ein. Zahlreiche Tavernengäste kommen zusammen und tragen verschiedenste Geschichten vor.
\end{chapterbox}


\section{Gildas glamouröse Gedichtbühne}

Im westlichen Rietland des Königreichs Andor, mittig zwischen dem freien Markt und dem Südlichen Wald, lag eine gut und gern besuchte Taverne, deren legendärer Ruf als gemütlichste Gaststätte in ganz Andor verbreitet war: Die Taverne zum Trunkenen Troll. 

An jedem Nachmittag waren wie an so vielen anderen verschiedenste Feiernde versammelt. Dort drüben am Stammtisch saß Galaphil, der Druide des Verlassenen Turmes der großen Fledermäuse, tief in ein Kartenspiel versunken. Hier vorne stand eine große Steinstatue von Rapa Nui, dem Nabel der Welt, und brummelte fröhlich vor sich hin. Und an einem Randtisch saß der Feuerdämon Kar éVarin, tief in eine von Runen übersäte Schiefertafel versunken.

Soeben war die Wirtin Gilda dabei, ihre Galerie zahlreicher geschenkter Zeichnungen zünftig zufriedener Gäste umzubenennen. Kaum einer achtete darauf. Vielmehr waren die anwesenden Andori und Besucher fernerer Länder beschäftigt mit Tratsch und Klatsch, Trinken und Singen, Würfelspiel und komplizierteren Geisterinselspielen. 

Dies änderte sich, als Gilda neben dem Kamin Tische zur Seite rückte. Von hinter dem Tresen hievte sie eine breite Holzkiste hervor und schob jene bis an den Kamin, unter das dort hängende andorische Wappen. Die auf dem Kaminsims schlafende Katze ließ nur durch zuckende Ohren erkennen, dass sie die Veränderung wahrnahm.

Das angenehme Hintergrundrauschen aus miteinander sprechenden Besuchern ebbte ab und nahm, erfüllt vom Raunen rascher Spekulationen, erneut zu, als Gilda ihre Hände an ihrer Schürze abwischte, auf die Holzkiste stand und sich laut räusperte.

"Liebe Andori! Vieles habe ich in den letzten Jahren hier vorgetragen an Liedern und Sagen aus Andor und darüber hinaus. Gesungen, gedichtet, berichtet. Und so vieles habt ihr erzählt an fantastischen Geschichten, wenn ihr von legendären Heldentaten in Andor, von Legenden und Heldentafeln erzähltet. Es wird an der Zeit für eine offizielle Bühne dafür."

Sie gestikulierte zur Holzkiste, welche sie nun demonstrativ betrat. "Wie die Erfahreneren unter euch bereits wissen, finden Reisende aus aller Welt den Weg in meine bescheidene Taverne. Dann und wann geschieht es, dass einer von ihnen sich am Feuer niedersetzt und eine merkwürdige Geschichte erzählt. Ringsherum wird es still und ein jeder lauscht."

Und als hätte er es erwartet, erhob sich bei Gildas letzte Worten ein massiger Tavernenbesucher von seinem drollig kleinen Vierbein und lief zur Erzählbühne. Wie die dreifingrigen Hände und die Hörner klar machten, war dies ein Troll! Gilda stimmte einen Applaus auf ihn an, während erwartende Stille sich unter dem Publikum breit machte.




\section{Der ewige Rat}

Dies war der Tolle TroII, ein überall gern gesehener Barde der Kreaturen. Der ebenso rundum einzigartige wie ideenreiche, Geschichten erzählende, recht außergewöhnliche Troll. Ein Vogelnest trug er auf dem Kopf, ein Geschichtenbuch in seiner Hand und jede Menge Honig in der Stimme.

Und der Tolle TroII erzählte vom Ewigen Rat. Lange hielt er die lauschenden Andori in Bann, mit geschmiedeten Worten, sodass es selbst den Balladenschreiber Grenolin, die begnadete Sängerin Gilda und den aus den fernen Norden angereisten Barden Canour zu Tränen rührte. Der TroII erzählte davon, wie die Krahder besiegt worden waren, die verschleppten Andori befreit, die Helden von Andor trotz aller Opfer einmal mehr siegreich. Von Thorn, der sich nach Frieden sehnte. Von Drukil, den es dazu drängte, seine menschliche Haut abzustreifen. Von Chada, die ihr Erbe antreten  wollte, während Janis von seinem davonlief. Von Eara, die mit ihrer Dunkelheit rang. Von Leander, dessen Visionen seinen Bruder und eine alte Bedrohung zeigten. Von einem lange unterschätzten Feind, der das Erbe der Krahder antrat. Und von einem Ewigen Rat aus lange totgeglaubten Widersachern der Helden, die so viel mehr als nur Andor vernichten wollten\footnote{"Der ewige Rat" von TroII: \url{https://legenden-von-andor.de/forum/viewtopic.php?f=26&t=11465}}.

Tage- und nächtelang dichtete er an seinem Wunderwerk. Über Wochen kehrte er immer wieder in die Taverne zum Trunkenen Troll ein, stellte sich auf die Holzkiste neben dem Kamin und führte seine Geschichte fort. Die Tavernengäste hingen gebannt an seinen Lippen. Weinten Tränen der Trauer und sangen Lieder der Lobpreisung auf sein Talent und seinen Durchhaltewillens. Selbst manche Tiere des nahe gelegenen Südlichen Waldes näherten sich dem Gasthaus und lauschten. Ein bestimmtes Rotkehlchen landete immer wieder auf dem Sims und hörte der Erzählung des TroIIs gespannt von außerhalb des Fensters zu. Von a bis z arbeitete der TroII sich durch das andorische Alphabet, und als er damit durch war, begann er gleich wieder bei A und arbeitete sich noch einmal bis Z durch.  Es dauerte viele Tage und Nächte, bis er seine epische Erzählung einem Ende zuführte. Jenes Ende war unerwartet und hätte man doch so lange kommen sehen können. Es war der abschließendste aller möglichen Abschlüsse und doch so offen für weitere Abenteuer. Es war ... unübertrefflich. Und es war vorbei. Vollbracht. Vollendet.

Und Gilda sprach: "Wunderbar, eine erste Erzählung für die Taverne! Besten Dank, lieber TroII. Und an alle Zuhörer: Selbstverständlich müssen eure Geschichten solche Ausmaße nicht annehmen. Es steht euch vollkommen frei, wie lang eure Geschichten sind."

Die Hände wund vom Klatschen, blieben die Tavernengäste still in Andacht und fragten sich, bei allen Kreaturen der Tiefe, wie könnte man diesem Wunderwerk je nahe kommen? 




\section{Die ersten Barden der Taverne}


Da meldete sich TroII zu Wort und korrigierte, er wäre nicht der erste gewesen, der hier in der Taverne fiktionale Geschichten präsentiert habe. So viele Leute wären bereits hier eingekehrt und hätten lange Geschichten zu erzählen begonnen.

„Erinnert ihr euch noch an den Bogenschützen Chris aus dem Westerwald? Nur dank ihm erfuhren wir von den heldenhaften Abenteuern der Zwergin Bait, die sich, während ihr Lieblingsbruder Kram an der Seite anderer Helden den Kampf um Cavern bestritt, alleine, nur mit ihrem Mut, ihrem Mithrilring und ihrer Streitaxt bewaffnet, an grausamen Kreaturen, lähmenden Spornen und stinkenden Seemonstern vorbei durch Cavern schlagen musste\footnote{"Bait, die Zwergin" von ChrisW: \url{https://legenden-von-andor.de/forum/viewtopic.php?f=11&t=1587}}.

Was war mit Tapsi Tost vom Schicksalsberg, welcher eine selbstgeschriebene Geschichte zu Andor vortrug? Mit allerlei amüsanten Anmerkungen versehene Erzählungen von einer alternativen Ankunft der Helden, einem ersten Zusammentreffen der vier bekanntesten Helden Thorn, Eara, Chada und Kram, sowie der gefürchteten Rückkehr der Skrale nach Andor, die König Brandur, Fürst Hallgard und Hohepriester Melkart allesamt überraschte. Daraufhin folgte eine Erzählung der Heilung des Königs Brandur, in die auch der Bewahrer Arbon und der Fährtenleser Fenn verwickelt wurden. Zwei bekannte Legenden in einem neuen Licht\footnote{Tosts Geschichte von Andor: \url{https://legenden-von-andor.de/forum/viewtopic.php?f=11&t=1920}}.

Der berufliche Geschichtenerzähler Kleativ holte sogar viel früher in der Zeitlinie aus und malte uns brutale Bilder der Höllenwelt Krahd und erster Vorbereitungen zur Flucht von Brandur, Reka, Harthalt, Drak und sogar des zwielichtigen Jari Dorrs\footnote{Kleativs Legende von Andor: \url{https://legenden-von-andor.de/forum/viewtopic.php?f=11&t=2439}}.

Und kaum einer konnte dem alten Barden Bird das Fabelwasser reichen, einem weiteren beruflichen Geschichtenerzähler aus einem fernen Land. Bei einem ausgezeichneten Becher Met setzte er sich alle paar Mondzyklen vor Gildas behaglich warmes Feuer und brachte lyrische Meisterwerke dar, Legenden von Andor, wie sie in seiner fernen Heimat erzählt werden. Anders als die uns bekannten, und doch so ähnlich. Vogelgleich sang Bird von der Flucht von Brandurs tausenköpfiger Schar und dem Kampf des zukünftigen Königs gegen den Drachen Tarok. Davon, wie Baumeister Kram den Tod fünf guter Zwerge verschuldete und wie die junge Chada sich zwischen Bewahren und Bewachen entscheiden musste. Wie Wachhauptmann Thorn nach einem Blutsturm beim Freien Markt zum Rechten sah. Wie die Helden ausgesandt wurden, die Hexe Reka zu finden, Lehrmeisterin der Einsiedlerin Fennah. Eine meisterhafte Erzählung\footnote{Birds Legenden von Andor: \url{https://legenden-von-andor.de/forum/viewtopic.php?f=11&t=3757}}. Ich bin betrübt, dass sie kein offizielles Ende fand.“

Bei den Worten des TroIIs sinnierte die versammelte Tavernengemeinschaft über die Werke der ersten Barden der Taverne und den Einfluss, die sie auf die bisher gehörten Geschichten gehabt hatten und auf die erst noch kommenden haben würden. Im Hintergrund spielte die Fantasy Folk Band ELANE eine Ballade über die Taverne. Und aus Birds Legenden wurde ein Trinklied hervorgehoben, auf das die Andori fröhlich anstimmten:

"\textit{Und wir trinken uns lustig, wir trinken uns toll,}

\textit{wir trinken im Gasthaus Zum Trunkenen Troll,}

\textit{und wir trinken zum Umfall'n, wir trinken uns voll,}

\textit{wir trinken im Gasthaus Zum Trunkenen Troll!}"





\section{Einzigartige Einzel-Erzählungen}




Kaum war das Trinklied ausgeklungen, erhob sich eine Gestalt. Es war Canour, der Barde aus dem hohen Norden. Andächtig betrat er die Holzkiste und sprach: "Wenn wir schon so fröhlich am Singen sind, will ich mich dem anschließen: Mir wurde aufgetragen, im Namen von Naqoy aus dem Süden einen Lobgesang darzubringen, und dies tue ich nun." Und so tat er dies. In astreinem Versmaß und in einem wunderschönen Bariton sang Canour eine Lobpreisung an auf die Wunder der Tavernengemeinschaft\footnote{"Das Lied von Canour dem Barden – ein Lobgesang" von Naqoy: \url{https://legenden-von-andor.de/forum/viewtopic.php?f=11&t=2930}}. Ein wahrer Dichter!

Da bahnte sich bereits wieder der Tolle TroII einen Weg zur Erzählbühne. Durch seinen vielseitigen Einsatz in der Gaststube hatte er Mjölnirs Olymp bezwungen, und so stimmten die Tavernengäste alliterative Lobpreisungen auf ihn an, und er Lobpreisungen auf die Tavernengemeinschaft – oder zumindest wollte er letzteres, doch schien er leider auf einmal seine vorbereitete Ballade in acht Fragmente zerteilt und selbst alle Links dazu über die ganze Taverne (und, wenn es mit der Modertaion geklappt hätte, auch darüber hinaus) verteilt und in den staubigsten Tavernenecken verloren zu haben. Prompt kamen verschiedenste Tavernengäste zusammen und halfen dem TroII dabei, die diversen Teile seines Lieds zu vereinen. Es war ein Rätsel, wie es ihnen so rasch gelang, die Fragmente zu finden, zu entschlüsseln, zu markieren, nach Alter zu sortieren und von Schmutz zu befreien. Doch am Ende waren sie siegreich und der TroII vermochte es, verschmitzt seine vollständige Ballade vorzutragen. Und als den ersten Gästen klar wurde, dass sich in deren ersten Worten mehr verbarg als vorher vermutet, ja, da grinste der TroII noch breiter. Und ihnen allen wurde klar, dass diese gesamte Aktion von äußerst geschickter dreifingriger Hand geplant worden war, und dass im gemeinsamen Geiste der Balladenvereinigung eine weitere Ode der Dankbarkeit kunstvoll versteckt worden war\footnote{"Die Ballade vom TroII, der seinen Link verloren hat": \url{https://legenden-von-andor.de/forum/viewtopic.php?f=11&t=6118}, \url{https://legenden-von-andor.de/forum/viewtopic.php?p=46782\#p46782}, ...}.

Canours und TroIIs Wunderwerke des Gesangs berührten viele. Gildas Singstimme und Grenolins Schreibkünste waren vereint in beiden, und auch der Rätselhunger der Tavernengäste war fürs Erste gestillt. Canour und der TroII setzten sich. Die schwarze Katze, die bislang auf dem Sims des knisternden Kamins geschlafen hatte, zuckte mit ihren Ohren, räkelte sich und trottete dann mit einer Selbstverständlichkeit, wie sie vielen Katzen angeboren war, auf die Holzkiste. Gleich zwei kurze Erzählungen gab sie zum besten, eine von den Folgen einer unterlassenen Erstretterritterschulung mit darauffolgendem Retterrittereinsatz\footnote{"Rettungsdienst im Lande Andor" von der Schlafenden Katze: \url{https://legenden-von-andor.de/forum/viewtopic.php?p=59497\#p59497}} und eine zweite von einem beliebten Kinderspiel mit Stoffwolfswelpen, Drachen und natürlich Junior-Helden\footnote{"Der Ursprung von Andor Junior" von der Schlafenden Katze: \url{https://legenden-von-andor.de/forum/viewtopic.php?p=63055\#p63055}}. Aus dem Publikum erklangen einige fröhliche Kinderstimmen, die zu einem melodischen "Andor, Andor!" anstimmten.

Als nächstes stand ein Fischer von der Narne auf die Holzkiste beim Kamin. Als Thies stellte er sich vor, und berichten tat er von uralten Aufzeichnungen, welche unter dem geheimen Boden einer von Fürst Kram im Verlassenen Turm wiederentdeckten Kiste gefunden worden waren: Von Orfas, Sohn des Ofanus vom vergessenen Volk der Wölfe aus den Wäldern Lupiens, vom Schicksal des Volks der Krähen aus den Corvidischen Wäldern und vom Gräuel der Schar des Drachentöters Brandur\footnote{"Die wahre Geschichte von den Anfängen Andors" von Thies: \url{https://legenden-von-andor.de/forum/viewtopic.php?f=11&t=6860}}.

Während Thies stolz von der Bühne schritt, stolperte ein kleiner Teddybär darauf zu, zog sich unter etwas Mühe auf die Holzkiste und sprudelte Wörter dort über den Santa Gor los\footnote{"Santa Gor" von BBB: \url{https://legenden-von-andor.de/forum/viewtopic.php?p=101625\#p101625}}. An dessen Legende glaubten natürlich nur noch Kinder, nichtsdestotrotz war es schön, sich auszumalen, wie der Santa Gor in einer Kutsche über den Nachthimmel flog und verschiedensten Figuren diverse passende Geschenke bescherte.

Die Kinder, die fröhlich im Chor "Andor, Andor!", gerufen hatten, wanderten hinüber, um mehr vom Santa Gor zu hören. Sie wurden abgelöst vom Musiker Mario, der extra von den Nebelinseln angreist war und dessen wunderschöne Flötenstücke die Taverne schon oft zum Singen -- oder vielmehr Grölen -- animiert hatten\footnote{Ein Lied zu Marios Melodie: \url{https://legenden-von-andor.de/forum/viewtopic.php?p=90472\#p90472}}.

Der Butterbrotbär wurde indes von einer mysteriösen Botschaft abgelenkt\footnote{Mivos Pergament: \url{https://legenden-von-andor.de/forum/viewtopic.php?f=11&t=4695&start=7439}}, ebenjenem leeren Pergament, auf das eines Tages diese Geschichte niedergeschrieben würde. Stammte es womöglich vom tulgorischen Sternkundigen Mivo?

Auf jeden Fall war mit diesen Liedern der Bann endgültig gebrochen. An diesem und an den nächsten Abenden trauten sich immer mehr Andori nach und nach zur Erzählbühne, um dort auch längere Geschichten zu erzählen.

Flederfluse flatterte aus dem fernen Tulgor herbei und berichtete vom Schiffsbruch der \textsc{Antares} und davon, wie der königliche Krieger Marcus und die Trollmagierin Anna ins ferne Tulgor gelangten und von der Felderwirtschaft der Tulgori erfuhren. Die Beschreibung zahlreicher fremder Tierarten faszinierte besonders, von Hybärnen und Baumtrollen über Feuerschrecken und blauschuppigen Feruns bis hin zum legendären Gepodon, über das noch viel gesprochen würde\footnote{Flederfluses Tulgor: \url{https://legenden-von-andor.de/forum/viewtopic.php?p=103305\#p103305}}. 

Da kam auch Kram, der erste aus den Tiefminen stammende Fürst der Schildzwerge. Er richtete seine Schildkrone und erzählte in Melkarts Namen davon, wie es nach dem Fall der Krahder weiterging. Vom brennenden Baum der Lieder, von der Jagd auf einen gefährlichen Mann in Schwarz, von einem wiedergeborenen Tarok und von einem Hilferuf von den tulgorischen Sümpfen der Traurigkeit\footnote{kram1s tulgorische Sümpfe: \url{https://legenden-von-andor.de/forum/viewtopic.php?f=11&t=5762}}.

Und als der Musiker Mario mit seiner Flöte wieder in den Norden abreiste, kam der Barde Julian auf den Plan, welcher die Andori für die restlichen Abende musikalisch belustigen würde mit Balladen vom Geheimnis des Königs und Liedern über das Rietland, das Zwergenreich, das Graue Gebirge und so einige Kämpfen. Das Lied vom Blutstrom war allerdings nicht darunter.

Von einigen Kämpfen berichtete als nächstes auch der königliche Krieger Malin. Jener brachte spannende Spekulationen an den Tisch, darüber, wie die Geschichte Andors sich entwickelt hätte, wenn Brandur sich vor seiner Flucht aus Krahd geopfert und seine Schar somit niemals Tarok besiegt und nie ein Königreich im Drachenland gegründet hätte. Was, wenn die drei Großen Trolle noch immer das Drachenland unsicher gemacht hätten, als der Yetohe-Stamm von den Krahdern dorthin gescheucht wurde? Was, wenn Fenn vom Büffelclan, Tranuk vom Einhornclan, Barz von den Iquar (?) und Grent von den Grehon statt Tenaya und den Grundspielhelden gegen die Ewige Kälte vorgegangen wären? Was, wenn Hademar einen Großangriff auf Krahd statt Andor angeführt hätte? Diesen und so vielen weiteren Fragen ging Malin nach, äußerst spannend, einsichtsreich und gut durchdacht\footnote{"Was wäre wenn ..." von Malin: \url{https://legenden-von-andor.de/forum/viewtopic.php?f=11&t=9697}}.



\section{Gemeinsame Gast-Geschichten}

Stille machte sich breit während den nächsten Abenden in der Taverne, denn wieder einmal war eine Erzählung schwer zu übertreffen. Bis sich eines Tages nicht eine Einzelperson, sondern ein ganzes kollaboratives Kollektiv aus Tavernengästen vor dem Kamin versammelte (denn auf die Holzkiste, die als Bühne fungierte, passten sie nicht alle gleichzeitig). Diese Gruppe berichtete von einem gemütlichen Jubiläums-Sommerabend am Stammtisch der Taverne, welcher abrupt durch einen Giftmord unterbrochen wurde. Sie erzählte von einer verschlüsselten Botschaft über ein anstehendes königliches Turnier und von Gerüchten über den Hexenmeister Noctis, der leider äußerst geschickt darin war, prefide Illusionen zu weben\footnote{"Der Stammtisch der Taverne": \url{https://legenden-von-andor.de/forum/viewtopic.php?f=11&t=5451}}. 

Ein junger Bewahrer namens Albus erinnerte sich an seinen ersten Besuch in der Taverne, damals, während der Ewigen Kälte, als Garz der stellvertretende Tavernenwirt gewesen war. Prompt spannen andere Andori weiter, wie sie damals auf Albus reagiert hatten, und knüpften gar eine weitere kollaborative Geschichte daran an, eine mit Dutzenden Figuren und Aberdutzenden an Beiträgen. Sie berichteten, was die Tavernengäste für Abenteuer erlebt hatten, während die Helden von Andor im Grauen Gebirge den Nekromanten Hademar gejagt hatten. Wie der junge Albus vom diebischen Varkur entführt und im Verlassenen Turm festgehalten wurde. Wie der Trunkene Troll 30 Fässer von Erloths sagenumwobenen Goldmet gegen hochbrisante Informationen zu tauschen ersuchte. Wie der zaubermächtige Agren Ühra Zwist zwischen den Völkern der Nebelinseln säte. Eine ungleich lange und wilde Geschichte voller Großangriffen auf die Taverne, finsteren Intrigen des Dunklen Magiers Varkur und eines uralten Geschenks eines mysteriösen Reisenden mit Katzenaugen: Des Kodex, einer zerbrochenen Landkarte des Andorversums, deren Vereinigung ungeahnte magische Macht zu verleihen vermag. Eine Erzählung ohne Ende, an der man noch heute weiterdichten könnte\footnote{"Abenteuer der Tavernengäste": \url{https://legenden-von-andor.de/forum/viewtopic.php?f=11&t=8059}}.

Und während vor dem Fenster des Gasthauses der erste Schnee fiel, fand sich eine kleinere Gruppe von Tavernengästen auf der Erzählbühne ein, welche gemeinsam eine kürzere kollaborative Geschichte konstruierte. Darin unterbrach ein verletzter Andori ein Weihnachtsfest in der Taverne und einige Helden von Andor (sowie Wassergeist Vara und Feuertakuri Turr) verirrten sich auf der Suche nach gefürchteten Nachtgors im Südlichen Wald, wo sie auf untote Wardraks, eine mysteriöse zerstörte Kutsche und einen bloß spärlich bekleideten Fährtenleser stießen. Und das ausgerechnet vor der großen Adventsfeier König Thoralds und in jener Nacht, in welcher der Santa Gor seine Geschenke verteilen sollte ... falls er überhaupt existierte\footnote{"Im Schatten des Winters": \url{https://legenden-von-andor.de/forum/viewtopic.php?f=26&t=11528}}.


\section{Spannende Story-Sammlungen}

Nachdem so viele Leute bewiesen hatten, dass sie gemeinsam an einer großen Geschichte spinnen konnten, eine Person nach der anderen, kamen andere Gruppen auf die Idee, gemeinsam verschiedene Geschichten spinnen zu lassen. Lose Sammlungen, die sich doch irgendwie ergänzten.

Die Schulkinder, welche bislang immer wieder "Andor, Andor!" gerufen hatten, hatten sich einmal mehr zur Erzählbühne begeben, als es im Schatten des Winters wieder um den Santa Gor gegangen war. Nun trauten sie sich nacheinander auf die Holzkiste und trugen ihre eigenen Ideen an Kurzgeschichten vor, ganze 13 Stück davon. Abenteuer der Helden Bait, Thorn, Pasco und Eara. Earas treuer Falke Korax wurde erstaunlich oft erwähnt. Geschichten von Kämpfen gegen böse Kreaturen, den Dunklen Magier Tarok und den Drachen Tarok, aber auch von zu rettenden Wölfen. In meisterhafter Erzählstimme vorgetragen, mit epischer Musik unterlegt\footnote{Sammlung "Klasse 3e": \url{https://legenden-von-andor.de/forum/viewtopic.php?f=11&t=3443}}. 

Als die Schüler die Bühne wieder freigaben und sich an ihrem Langtisch feinsten Nachtisch (andorische Knusper und Erloths Sonnenbrot mit Zimt und Zucker) gönnten, versammelte Thies von der Narne eine Gruppe von kulinarisch kreativen Tavernengästen. Wie der Fischer verkündete, war er einst von König Brandur höchstpersönlich dazu berufen worden, durchs Land zu reisen und eine Schrift zu den reichhaltigen Rezepten des Reichs zu sammeln. Angeführt von Dharwyn, ergänzt von Waldfaunen, Speerkämpfern, aufgeweckten Katzen und allerlei anderen Andori, präsentierte die Runde ihre liebsten Rezepte aus dem ganzen bekannten Andorversum  -- die meisten davon ergänzt um atmosphärische Kurzgeschichten. Von Tischmanieren der Taren über zahlreiche Einblicke in den Heldenalltag über Kreatoks letztem Schmaus vor dem Unterirdischen Krieg bis hin zu Forns zahlreichen Einblicken in Kreaturenkultur\footnote{"Das Kochbuch der Andori": \url{https://legenden-von-andor.de/forum/viewtopic.php?p=49839\#p49839}}. Die tavernenbesuchenden Bewahrer Phlegon und Tapta jubelten auf, als die ordentlichen Drachenbohnen, ihre Leibspeise, erwähnt wurden. Und generell waren die Vorträge dieser Rezepte derart beliebt, dass selbst nach Abschluss des Vortrags Bewahrer Melkart höchstpersönlich auftrat, um anlässlich des anstehenden Geburtstags der Taverne weitere Rezepte zu ergänzen, von Apfelnusskuchen der Bewahrer\footnote{Melkarts Apfelnusskuchen: \url{https://legenden-von-andor.de/forum/viewtopic.php?p=98408\#p98408}} über andorische Tomatenfladen zu Schokokeksen der Schildzwerge. 

Zum Geburtstag der Taverne lenkte Melkart den Fokus auch auf spannende Geschichten, die der Wachsame Waldläufer Aoto aus dem Wachsamen Wald zu Krahd zusammengetragen hatte. Dieser Wachsame Aoto betrat nun selbst die Erzählbühne und teilte weitere wilde Sagen aus Andor, Krahd, Silberland und anderen Orten. Von Namenskonventionen der Krahder und davon, wie die uralte Ent den Wald "Wachsam" schuf und den Bewahrern den Baum "Lieder" schenkte. Von den zywallischen Bognern Archor und Archiri, vom Druiden Yelaos und dem Nebelmagier Uhain, der nicht gerne als solchen bezeichnet wurde. Aoto blieb auch nicht alleine mit seinen Geschichten. Der Bewahrer Merrik, der Zwerg Brogg und gleich zwei verschiedene Bären schlossen darauf an und berichteten von eigenen Sagen: Seemannsgarn des alten Käpt'n Maleard, der seine \textsc{Capella} an Kapitänin Mondrianne verloren hatte. Ein Bauer, der sich unfreiwillig einem feurigen Skralreigen anschloss. Wie Forn mit Drukils Tod umging. Sogar detaillierte Ausführungen zur fernen nebligen Insel Zywall mit ihrem schwefligen Mhourlgebirge\footnote{"Geschichten von Andor oder dem Rest der Welt", eröffnet vom Wachsamen Waldläufer und ergänzt von anderen Andori: \url{https://legenden-von-andor.de/forum/viewtopic.php?f=11&t=5100}}.

Einer dieser Bären, der tiny toastförmige Teddy mit einer Vorliebe für Butterbrote, blieb für die nächsten Abende gleich auf der Bühne und begann damit, weitere Geschichten zu erzählen und zu verknüpfen. Vom tödlichen Kampf zwischen Kreatok und Tarok. Von Kjalls zweitem -- oder ersten? ;) -- Diebstahl der Zeitreise-Sphäre. Von den Konsequenzen von Taroks Tod und dem ersten Zusammentreffen der Magischen Helden. Von den Geheimnissen der Wassermagierin Jarid und des Feuerkriegers Trieest. Davon, wie der Seher Leander das Gespenst des Nekromanten Hademar einfing. Selbst von König Thoralds unfreiwilligem Ausflug in ein fernes Land mit einem Sherwood Forest. Lose verbundene Kurzgeschichten, von denen noch einige weitere geplant waren\footnote{"Bärig gebutterte Geschichten": \url{https://legenden-von-andor.de/forum/viewtopic.php?f=11&t=5828}}\textsuperscript{,}\footnote{BBBs Erbe des Wunderkindes: \url{https://legenden-von-andor.de/forum/viewtopic.php?f=26&t=11490}}.

Als nächstes präsentierte sich ein in perfekter Synchronie vor die Gäste tretendes Quartett, bestehend aus dem Zauberer und Hofschreiber Lifornus, dem abtrünnigen Krahder Ragomiter sowie zwei rundonischen Thornen namens Taris Norr und Dagain. Von diesem Quartett tauschte einer mit dem anderen in perfekt einstudiertem Rhythmus den Platz und beendete angefangene Sätze des vorherigen. Und so erzählten die vier vom fernen Land Rundon östlich des Lands der Steppe. Vom weisen König Rastak, dem von Krahdern verstoßenen Prinz Argwydd und dem magiebegabten Agren Ühra\footnote{"Rundon" von Dagain: \url{https://legenden-von-andor.de/forum/viewtopic.php?f=11&t=6345}}. Und von einer legendären Zeit der Söldner\footnote{"Zeit der Söldner" von Dagain: \url{https://legenden-von-andor.de/forum/viewtopic.php?f=11&t=9671}}. 




\section{Famose Fan-Filme}


Manche Andori meinten sogar, man solle nicht nur auf diese Holzkiste stehen und von dort Geschichten erzählen, nein, man könnte sogar welche theatralisch aufführen. Ideen wurden herumgeworfen, von Liboa\footnote{Liboas 3x3 Akte: \url{https://legenden-von-andor.de/forum/viewtopic.php?p=81089\#p81089}} zu Hofschreiber Lifornus\footnote{Lifornus' 3 Akte: \url{https://legenden-von-andor.de/forum/viewtopic.php?p=84464\#p84464}}, dessen viertes Quartettmitglied Dagain schlussendlich einen Vorschlag zu Visionen und Verrat\footnote{Dagains Visionen \& Verrat: \url{https://legenden-von-andor.de/forum/viewtopic.php?p=119840\#p119840}} vortrug, einer Variante der Legende von der Heilung des Königs, in der die Helden einen vergiftenden Diener an der Rietburg enttarnen.

Andere bardische begnadet begabte, theatralisch talentierte Truppen hatten schon längst lyrische Stücke einstudiert und inszenierten sie nun, so gut es ihnen auf und vor der kleinen Bühne ging, vor freudigem Publikum. Gregor aus dem beschaulichen Potsdam erzählte von einem magischen Tisch, der Musik machte und Würfelbecher vorhersah. Ein mechanisches Meisterwerk, von dem sogar Liphardus beeindruckt war\footnote{"Andor meets Pixelsense" von Gregor Tallig: \url{https://www.youtube.com/watch?v=QRZDlJ5JaQQ}}.  Mit "Und wenn sie nicht gestorben sind, dann würfeln sie noch heute" schloss Gregor seine Erzählung und setzte sich wieder hin.

Am nächsten Nachmittag bot eine Gruppe um Meister Felix in einem halbstündigen Stück eine theatralische Interpretation der Legende der Tage des Widerstandes dar, vom Kampf der tapferen Helden Mairen, Fennah, Pasco und Liphardus gegen finstere Kreaturen und die noch finsterere Dunkle Magierin Varkur. Für diese Aufführung verließen die Darsteller sogar die Taverne zum Trunkenen Troll und nutzten den naheliegenden Wald samt der alten Burg Reichenau als Kulisse. Als Kreaturen verkleidete Andori zeigten ihre düsterste Seite. Ein Bewahrer vom Baum der Lieder traute zwei verliebte Bauersleute. Runensteine flüsterten verheissungsvoll von vergangenen Zeiten. Und hin und wieder kehrten manche Schauspieler zurück zur Taverne und spielten die erlebte Geschichte als Würfelspiel durch, auf dass die beiden Perspektiven einander perfekt ergänzten, bis hin zum ausgelassenen Siegesfest, welches selbstverständlich in der Taverne durch den Ausschank von Met, Ziegenmilch und Litschi-Limes ergänzt wurde. Ein beeindruckendes professionelles Werk\footnote{"Die Legenden von Andor" von Felix Merten: \url{https://www.youtube.com/watch?v=vXzOKKTNIbM}}.

Dann kam Yoda, in Dunkle Schatten gehüllt. Ein kleines grünes Männlein (ein Temm, vielleicht, auch wenn er selbst diese Spezies stattdessen als Goblins bezeichnete), das auf einen knorrigen Stock gestützt zur Erzählbühne watschelte. Es reiste schon seit langer, langer Zeit hierher, von einem weit, weit entfernten Ort. Einem Ort, in dem das Wort "Andor" ebenfalls groß geschrieben wurde, wenn auch aus ganz anderen Gründen. Und Yoda bot in ganz korrekter Satzstellung und angenehmer Stimmlage zweierlei Vorgeschichten dar, zunächst eine Erzählung von der Flucht von Brandurs Schar aus Krahd und Brandurs Kampf gegen den Drachen Tarok\footnote{"Die Flucht aus Krahd" von darkshadowyoda: \url{https://www.youtube.com/watch?v=Sy-Qx-av-KE}}, danach einen Überblick über den noch viel weiter zurückligenden Wirren des Unterirdischen Kriegs zwischen Zwergen, Drachen, Trollen und Riesen\footnote{"Der Unterirdische Krieg" von darkshadowyoda: \url{https://www.youtube.com/watch?v=Uzz773Fwy_4}}. Wie das Wunderkind Kreatok und der Drache Nehal die vier mächtigen Schilde aus uralter Zeit schmiedeten. Wie Nomion die Dunkle Hexerei entdeckte und der erste Krahder wurde. Düstere Geschichten, voller Humor und Spannung erzählt, ergänzt mit atmosphärischen Klängen, guter Musik und wunderschönen Bildern mmeisterhafter Hand.




\section{Heldenhafte Hintergrund-Historien}


Und dann wurde es wieder still in der Taverne. So vieles war gehört worden, von langen Geschichten über kollaborative Geschichten zu Kurzgeschichtensammlungen. Von Krahd bis Hadria, von Tulgor bis zum Barbarenland, von Zywall im fernen Westen bis Rundon im fernen Osten. Was gab es noch zu erzählen?

Da klopfte es an der Türe. Herein trat Tapsi Tost vom Schicksalsberg, ein gern gesehener und doch lange nicht mehr gesehener Gast der Taverne. Er winkte in die Runde und rief: "Hallo Leute! Ist ne Weile her, seit ich das letzte Mal hier war." 
Und kaum war Tost die neue Erzählbühne aufgefallen und erklärt worden, sprang er bereits stolz darauf und sprach los: "Na? Wie geht es euch so? Habt ihr es schön warm auf euren Stühlen? Dann kuschelt euch mal vor dem Kamin und lasst mich euch etwas über das Waldtost erzählen."
Und so erzählte Tost vom Waldtost, welches alle Butterbrote auf ihre Marmeladenseite fallen lässt\footnote{"Das Waldtost" von Tost: \url{https://die-legenden-von-andor.fandom.com/de/wiki/Benutzer_Blog:Tapsi_das_Waldtost/Das_Waldtost}}. 

Der Feuerdämon Kar éVarin räusperte sich aus der Ecke. Er führte eine flammende Feuervogelfeder, mit welcher er aschene Schriftzeichen in ein vor ihm liegendes Pergament brannte. Wundersamer Weise benötigte er dafür keine Tinte.
"Ich habe bereits Buch geführt darüber, was alles für Geschichten in unserer gemütlichen Gaststube vorgetragen wurden", sprach der Feuerdämon mit kehliger, kratziger Stimme, ohne aufzusehen\footnote{Kar éVarins Liste der Storytexte: \url{https://legenden-von-andor.de/forum/viewtopic.php?f=11&t=4432}}. Dann hielt er inne und ließ seine glühenden Augen über die angespannt wartenden Gäste wandern. "Aber eine Hintergrundgeschichte fehlt noch. Die meine." Dann erzählte Kar von der kleinen Mivar, die zu fasziniert vom Feuer gewesen war, von der Wassermagierin Jirid, die dem Dämon auf der Spur gewesen war, und von \textbf{Flammen}\footnote{"Kar éVarin": \url{https://legenden-von-andor.de/forum/viewtopic.php?f=11&t=4587}}. Ergänzt wurde die Geschichte um eine handvoll verschiedener Quellen, bei denen der Feuerdämon selbst noch herausfinden musste, wie genau sie zu seinem Wandel vom Dämon zum Feuerkrieger passen: Von der Wassermagierin Jarid, vom danwarischen Feuerschmied Melek, vom Kapitän Lunor und einigen anderen\footnote{"Quellen zu meiner Geschichte" von Kar éVarin: \url{https://die-legenden-von-andor.fandom.com/de/wiki/Benutzer_Blog:Kar_éVarin/Meine_Geschichte}}.

"Und meine Geschichte wurde bereits erzählt, doch will ich sie auch hervorheben!", fiepste eine Stimme. Der Butterbrotbär war zurück, und sprach stolz von einem großen braunen Brummbären, der von Gildas besten Butterbroten stahl und mysteriöse Schiefertafeln hinterließ, deren Entschlüsselung die Tavernengäste lange beschäftigt hatte\footnote{"Die Sage vom Butterbrotbären": \url{https://die-legenden-von-andor.fandom.com/de/wiki/Benutzer_Blog:Butterbrotbär/Die_Sage_vom_Butterbrotbären}}. "Und wie soll aus diesem großen Braunbären du kleiner sprachfähiger Teddy geworden sein?", rief jemand ein. Der kleine Bär schwieg grinsend.

Ein weiterer Bewahrer meldete sich zu Wort: "Ich bin Phlegon, Giftknödel, hier inzwischen wohlbekannt, doch war dem nicht immer so. Damals, vor meinen Recherchen zu den Düsteren Zeiten, war ich nur ein unbekannter Adept am Baum der Lieder, einer von zwei jungen Bewahrern. Und damals, im Sommer des Jahres 67 nach andorischer Zeitrechnung, schrieb ich folgendes ..." Und so erzählte Phlegon die Geschichte seines Spitznamens\footnote{Phlegons "Mein wahrer Name": \url{https://legenden-von-andor.de/forum/viewtopic.php?f=11&t=5205}}.

"Auch ich war einst unter einem anderen Namen bekannt", sprach ein Bursche, ein Wandergeselle von überall und nirgendwo. Einst hatte er als Schusterlehrling ohne Schuhe den Namen fusssohle getragen, doch sein wahrer Name ... war Pahl, der Wanderer. Und davon sprach er\footnote{Pahls Name des Wandergesellen: \url{https://legenden-von-andor.de/forum/viewtopic.php?f=11&t=5905}}.

Nachdem Pahl die Bühne verlassen hatte, lief eine Gestalt in einer schweren Rüstung vor die versammelte Tavernengemeinschaft. Gum Jabbar. Mit rauer Stimme berichtete er von seinen Abenteuern und denen seiner Begleiterin Kaylan. Von seiner Verbindung zu Wölfen und seiner und Kaylans Verfolgung wolfstötender Trolle ... wobei der wahre Wolfstöter womöglich gar kein Troll war\footnote{"Wolfstöter" von Gum Jabbar: \url{https://legenden-von-andor.de/forum/viewtopic.php?f=11&t=7073}}.

Große Augen wurden gemacht, als der schon lange verschollen geglaubte Hexer Meres sich vor die Gemeinschaft wagte und mit tonloser Stimme zu berichten begann. Er sei nicht beim Untergang der Schwarzen Kogge gestorben, nein, und er habe einiges verpasst, was sich seither zugetragen hatte. Doch als er zurückgekehrt war, war er von der Taverne zur Rietburg gereist, hatte Königin Chada auf seine überhebliche Art um Vergebung gebeten und den geflüchteten Kapitän Callem gefolgt ... oder wollte er sich ihm nicht lieber wieder anschließen? Eine Geschichte voller Anspielungen auf verschiedene Tavernenmitglieder und rumlose Piraten von fernen karibischen Inseln\footnote{Geschichten zu Meres' Verbleib: \url{https://legenden-von-andor.de/forum/viewtopic.php?f=11&t=7079}}.

Doch gerade als Meres zum Wiedersehen mit Kenvilars Tochter Kentar kam, wurde er unterbrochen vom Bewahrer Albus, der die Taverne betrat und an diesem Abend erneut eine Geschichte zum besten geben wollte. Bis Albus zur Erzählbühne gelangt war, war Meres ohne ein Wort des Abschieds verschwunden. Und so sprach Albus: "Dann will ich euch erneut davon erzählen, was sich zugetragen hatte, als ich diese Taverne zum ersten Mal betrat, damals, während der Ewigen Kälte. Selten ward ich an einem Ort so offen empfangen ... aber selten kommt es auch vor, dass Bewahrer vom Baum der Lieder den grünen Radius verlässt, und nie ohne guten Grund. Auch ich hatte einen guten Grund dafür." Und so erzählte Albus nicht nur von seinem ersten Tavernenbesuch, sondern auch vom Rätsel um das Verschwinden des Obersten Bewahrers Melkart\footnote{"Ein neuer Gast in der Taverne": \url{https://legenden-von-andor.de/forum/viewtopic.php?f=11&t=7950}}.

Nachdem Albus die Bühne wieder verlassen hatte, um mit einem feurigen Phoenix zusammenzusitzen, erhoben sich zwei weitere Protagonisten des "Abenteuers der Tavernengäste" und erzählte von ihren Hintergründen. Da war zum einen Nicopaos Runensteinen, Zauberer aus dem fernen Lande Magix\footnote{Nics "Nicopaos Runensteinen": \url{https://legenden-von-andor.de/forum/viewtopic.php?f=11&t=9153}}. Und da war zum anderen eine Steppenechse aus einem anderen fernen Land, dem Land des Weißen Gebirges\footnote{Steppenechses Macht der Hydra: \url{https://legenden-von-andor.de/forum/viewtopic.php?f=11&t=9160}}.

Die Tür schlug auf und eine Gestalt in einem grauen Mantel huschte herein. Sie schlug die Kapuze zurück und sprach zur Runde: "Ich habe von einer Wirtin gehört mit einer Stimme wie Gold, so warm wie die Ewigen Feuer aus Hadria. Ich würde gerne ein Lied hören, bitte. Kennt jemand ein Traditionelles Lied aus Andor? Mir ist bisher nur das Lied des Königs bekannt." Nachdem die Fremde einige Lieder vernommen hatte, stellte sie sich als Rekas Tochter vor, gesellte sich zur trauten Runde, erzählte von Varkur und vagen Erinnerungen und spekulierte zu ihrer wundersamen Herkunft\footnote{"Rekas Tochter": \url{https://legenden-von-andor.de/forum/viewtopic.php?p=109258\#p109258}}. 

Inspiriert von Rekas Tochter, trauten sich so viele andere Tavernengäste nun einen nach dem anderen hervor und beglückten die Tavernengemeinschaft mit kurzen Einblicken darin, wie sie zur Taverne fanden: Ein Feuervogel aus Danwar, der hadrische Zauberer Kamuna, das Irrlicht loro, ein als namenloser Wanderer auftretender Bewahrer namens Kahru, Dagain aus Rundon, Qurunatobra zweiter Sohn eines ersten Sohns und Freund von Liphardus siebtem Sohn eines siebten Sohnes, Schwertmeister Harthalts ebenso tapferer Sohn Malin ... verschiedenste Leute, von einem klassischen Schildzwerg aus Cavern über den Kundschafter Legolas Grünblatt aus den Elbenwäldern jenseits Krahds bis hin zum lächelnden Spike -- ein weiteres Zeichen dafür, was für ein Treffpunkt vielseitigster Leute Gildas gemütliches Gasthaus geworden war\footnote{"Storys zu Forumsmitgliedern": \url{https://legenden-von-andor.de/forum/viewtopic.php?f=11&t=9194}}.


\section{Und nun ... ?}

Und wieder einmal blieb die Erzählbühne leer, während die Stammgäste am Stammtisch fröhlich miteinander quatschten. In einer Ecke der Taverne saß eine einsame Gestalt an einem kleinen Rundtisch, die Kapuze tief in ihr Gesicht gezogen.

Ein Tablett voller Ziegenmilchkrüge gekonnt balancierend, bahnte sich die Wirtin Gilda einen Weg zum Rundtisch und stellte einen der bauchigen Krüge vor dir ab. "Aufs Haus!", rief sie dir zu, "Die erste ist immer frei für neue Gäste."

Du schlugst höflich die Kapuze zurück. Gilda lächelte dich an und sagte: "Und, hast Du Geschichten gehört, die Dir gefallen haben?" Sie nickte auffordernd zur leeren Erzählbühne hinüber. "Na, was kannst Du uns erzählen?"\footnote{Sendet neue Fan-Geschichten an \href{mailto:gilda@legenden-von-andor.de}{\texttt{gilda@legenden-von-andor.de}} !}








\begin{chapterbox}
    \chapter{Was ist ein Butterbrotbär? (2023)}
    \label{Was ist ein Butterbrotbär? (2023)}
    \az{diverse Jahre}
    
    Ende 2019, als die Arbeit im Andorwiki nach einigen Jahren des Brachliegens wieder etwas anlief, schrieb ich die Sage vom Butterbrotbären\footnote{Siehe \hypref{Stürmische Rätselnacht in der Taverne (2019 bis 2020)}} in die Wiki-Blogs, in Anlehnung an Tosts Sage von Tapsi dem Waldtost\footnote{Aus dem Fan-Wiki, "Benutzer Blog:Tapsi das Waldtost: Das Waldtost": Na? Wie geht es euch so? Habt ihr es schön warm in euren Wohnungen? Dann kuschelt euch mal aufs Sofa und lasst mich euch etwas über das Waldtost erzählen. [...]}. Damals stellte ich mir unter einem Butterbrotbären noch einen großen Brummbären mit einem Hunger für Butterbrote vor. Inzwischen male ich mir meinen Forencharakter eher als tiny toastförmigen Teddy aus.

    Folglich folgt hier nun als Nachsatz eine selbstverliebte Übersicht darüber, was es, Stand Ende 2023, alles für verschiedene Butterbrotbären in den verschiedenen Andorversen gibt.
\end{chapterbox}


\az{diverse Jahre}

{\parindent0pt

Im offiziellen großen Andorversum und in der offiziellen Junior-Welt wird natürlich kein Butterbrotbär erwähnt, wenn man von der Danksagung im DeK-Begleitheft\footnote{Aus "Die ewige Kälte (2022)": Ein besonderer Dank geht an "Butterbrotbär" aus dem Andor-Forum, der für unseren geflügelten Freund die Bezeichnung Flederfuchs ersonnen hat.} absieht. Immerhin war im Kanon-nahen Abenteuer Andor 2022 ein Gast mit dem Spitznamen Butterbrotbär im Frühjahr 64 a.Z. in der Taverne anwesend, als Tenaya dort vorbeischaute.\footnote{Aus der Taverne, "Tavernen-Party 2022": [Tenaya:] Hallo, ich hoffe ich störe nicht? [Butterbrotbär:] Willkommen in unserer gemütlichen Stube! :P}\newpage


In den kanon-fernen Fan-Tavernenpartys\footnote{Aus der Taverne, "1. Tavernenstammtischparty!" und "2. Tavernenstammtischparty 2021"} war schon mehrmals ein Gast (später sogar Organisator) mit dem Spitznamen Butterbrotbär anwesend. Dieser wurde in einem Gruppenbild vom Wachsamen Waldläufer festgehalten: Ein großer brauner Teddy mit schwarzen Knopfaugen.\footnote{Aus der Taverne, "1. Tavernenstammtischparty!":\\\includegraphics[height=120px]{Das Erbe des Wunderkindes/Bilder/Tavernenstammtischparty 2020 Gruppenbild.jpg}}\bigskip


In der Welt der bärig gebutterten Geschichtchen lebt, wie in der Sage des Butterbrotbären\footnote{Siehe \hypref{Stürmische Rätselnacht in der Taverne (2019 bis 2020)}} angetönt und in \hypref{Der Feuerkrieger und die Wassermagierin (2023)} konkretisiert, der Alchemist Naraven im Südlichen Wald. Er blickt mit seinem Hadrischen Spiegel tief in der Vergangenheit und hält seine Erkenntnisse auf danwarischen Steintafeln fest. Der Butterbrotbär dieser Welt ist eines der vielen Tiere, welche Naraven zu seiner magisch versteckten Hütte führen können – ebenjener Brummbär aus der Sage vom Butterbrotbären, welcher gerne Naravens Steintafeln stibitzt und gegen Gildas Butterbrote tauscht – aber die Taverne nicht betreten darf, weil Bären bekanntlich draußen bleiben. müssen\footnote{ Aus der Taverne, "Wie vertragen Dunkle Helden Gildas Met???":\\\includegraphics[height=120px]{Das Erbe des Wunderkindes/Bilder/Bären müssen draußen bleiben.jpg}}\bigskip



In der Welt von Frodos Mini-Erweiterung lebt ebenfalls ein Butterbrotbär im Südlichen Wald, dieser ist aber kleiner und weniger animalisch als derjenige aus der Sage. Er bietet ein beidseitig gebuttertes Bärenbrot an, welches einem Helden kurzzeitig Drukils Bärenstärke verleiht.\footnote{Aus der Taverne, "Minierweiterung: Die Tavernengäste": \textit{Ein kleiner brauner Bär stolperte aus dem Südlichen Wald und bot den Helden bereitwillig ein beidseitig gebuttertes Brot an.} 1x pro Legende kann ein Held, welcher an der Taverne (Feld 72) steht, das Bärenbrot verspeisen. Bis dieser Held seinen Tag beendet, hat er 3 Stärkepunkte mehr, dafür kann er solange keine Gegenstände, Gold, Edelsteine oder Holz tragen.}\bigskip


In der Welt der kollaborativen Fan-Geschichte "Das Abenteuer der Tavernengäste", die Albus der Bewahrer anstieß, kommt ein kleiner sprechender Bär vor, welcher allerlei Andor-Lore von sich gibt, gerne Butterbrote mampft und hier die Wunder magischer Waffeln kennenlernt.\footnote{Aus der Taverne, "Story: Das Abenteuer der Tavernengäste (Story)": [Butterbrotbär:] \textit{Die Tür schwang auf und eine gedrungene Gestalt stürzte in die Taverne hinein. Kleine Äuglein blinzelten unter einer beigen Kapuze hervor. Zwei Hände - oder war es Pfoten? - schoben sich aus Ärmeln hervor und wärmten sich am Feuer. Bibbernd streichelte die Gestalt die Tavernenkatze und fragte in die Runde: "Hat jemand ein Butterbrot? Ich komme gleich um vor Hunger!"} [Steppenechse111777:] \textit{Das Gegenüber von der Steppenechse war ein sonderbares Geschöpf: Es war ein kleiner Bär, in einen Umhang gehüllt, das besonderste Merkmal des Butterbrotbären war jedoch sein knurrender Magen. Er war in ganz Andor bekannt. Einmal als die Steppenechse zum ersten Mal die Taverne von Andor besucht hatte war sie ihm schon mal begegnet. Er hatte am Stammtisch gesessen und genüsslich ein Butterbrot gegessen.} [nic:] \textit{["]Hast du ein Butterbrot?" "Nein aber ein magisches Waffeleisen." antwortete Nic. "Äh...Was für eine magische Waffe?" fragte der Butterbrotbär verwirrt. "Ein magisches Waffeleisen. Damit presst man Teig zusammen und backt man ihn und schliesslich hat man eine sogennante Waffel" Er gab dem Bären das Waffeleisen und ein bisschen Teig.}} Nach eigener Aussage ist dies dieselbe Welt und derselbe Butterbrotbär wie damals, als Albus der Bewahrer zum ersten Mal Gildas (damals Garzes) gemütliches Gasthaus betrat.\footnote{Aus der Taverne, "Ein neuer Gast in der Taverne...": [Albus der Bewahrer:] \textit{Die Tür der Taverne schwang auf. Herein trat ein junger Bewahrer. "Bei Mutter Natur, ist das kalt da draußen!", beschwerte er sich. "Hallo, mein Name ist Albus.", stellte er sich anschließend vor, "Garz, ich hätte gerne eine warme Ziegenmilch!". Als er die Taverne betrat, sah er sich um. Etwas misstrauisch musterte er den Bären, der gerade ein Butterbrot verspeiste, oder den TroII, der gerade lebhaft, aber ruhig, mit einem anderen Gast diskutierte. Albus fühlte sich sofort wie zu Hause.} [Butterbrotbär:] \textit{Der Bär blickte schmatzend von seinem Butterbrot auf, um dem Neuankömmling fröhlich zuzuwinken. "Willkommen, holder Bewahrer Albus! Was für Neuigkeiten gibt es vom Baum der Lieder? Setze dich an den Kamin und genieße die Wärme – sofern die Schlafende Katze noch genug Platz davor gelassen hat. Wenn die Ewige Kälte noch viel länger anhält, werden wir bald von unserem hauseigenen Feuerdämon abhängig sein, um die Taverne gemütlich warm zu halten."}}\bigskip


In der Welt der kollaborativen Stammtischparty-Geschichte taucht in Kapitel 2 ganz kurz ein Butterbrotbär auf.\footnote{Aus der Taverne, "Kollaborative Geschichte Stammtischparty": [Schlafende Katze:] Im Laufe des Abends wohnten der Feier unter anderem Aoto, Galaphil, ein Schallmagier namens Liteus, ein Wesen, dass Hobbit genannt wurde, ein Elf aus den entfernten Elfenwäldern und ein andorischer Krieger mit dem Namen Piepe bei. Auch mehrere andere tierische Besucher gesellten sich zur Tavernenkatze: Der legendäre Butterbrotbär mit seinem Spielzeugtroll, für den Gilda extra das Bärenverbot der Taverne lockerte, ein danwarischer Phönix, den Trieest von dort mitgebracht hatte und ein andorischer Adler, der auf einem seltsamen Konstrukt, das wohl in Tulgor hergestellt worden war und sich "Einrad" nannte, saß. Sie alle saßen am schon fast legendären Tavernenstammtisch unter dem Bild eines tapferen Schildzwergs namens Brogg und einem Zauberfenster, durch das man eine sehr weit entfernte, sagenhafte Inselkette, auf der wohl von Mutter Natur persönlich erschaffen Steinstatuen standen, sehen konnte.} Wenn ich mich nicht irre, beschreibt diese Szene des Wachsamen Waldläufers Bild der Tavernengemeinschaft, welches weiter oben bereits genannt wurde.\bigskip

Der Tolle Troll, Barde der Kreaturen, führt ein Geschichtenbuch mit sich, auf dessen Cover\footnote{Aus der Taverne, "Bärig gebutterte Geschichtchen": [Butterbrotbär:] Ha! Ich habe in den Untiefen unserer Skype-Verläufe noch ein Original des Covers des Geschichtenbuchs des Tollen Trolls aufgespürt! :P\\ \includegraphics[height=120px]{Das Erbe des Wunderkindes/Bilder/Der Tolle Troll Geschichtenbuch Cover.jpg}} eine Skizze seiner selbst mit einigen Eastereggs ist. In der rechten unteren Ecke kann man einen weiteren Butterbrotbären erkennen, diesmal als Teddy mit toastförmigem Körper. So zeichnete ich den Butterbrotbären bereits in gemeinsamen Gartic-Phone-Partien.\bigskip

Natürlich gibt es in unserer Welt einen großen Andor-Fan, welcher unter dem Nicknamen Butterbrotbär auftritt, diesen Quark hier verfasst und schon so überhebliche Titel wie "Hüter der Andorchronik" zugesprochen bekam.\footnote{Aus der Taverne, "2020 – Neue Helden – Neue Legenden": [Bewahrer Melkart:] Ich werde dem obersten Hüter der Andorchronik nie mehr widersprechen :lol:} Towa zeichnete ihn in einem unglaublich coolen Jubiläums-Kreuzworträtsel,\footnote{Aus der Taverne, "26. Februar 2022: 3-Jahres-Jubiläum des Butterbrotbären": [Towa:] Jetzt sind es schon 3 Jahre, die der Butterbrotbär und ich in der Taverne sind! :D Herzlichen Glückwunsch zum Jubiläum, lieber Butterbrotbär! :mrgreen: Da darf natürlich ein Rätsel nicht fehlen. ;) Es sind lauter Begriffe aus der Andor-Welt versteckt, die gefunden werden müssen. Alle Richtungen sind möglich. Und alle Felder, die zu keinem Wort gehören, bilden ein passendes Bild. :P\\\includegraphics[height=120px]{Das Erbe des Wunderkindes/Bilder/Towa Raetsel BBB.jpg}} und Galaphil verfasste für ihn (als \textit{beste Beglückwünschungen bezüglich 2000 beachtlicher, belesener, beflissener, bedeutsamer, besonnener, beglückender, beispielloser, beredter und buchstäblich bemerkenswerter Beiträge in beige und braun},\footnote{Aus der Taverne, "Tavernentratsch": [TroII:] Bei Berthas bauchigem Bierfass! Ich bemerke, beinahe hätte ich Butterbrotbärs beeindruckende Beitragszahl bar jeder Beachtung belassen. :oops: Beste Beglückwünschungen bezüglich 2000 beachtlicher, belesener, beflissener, bedeutsamer, besonnener, beglückender, beispielloser, beredter und buchstäblich bemerkenswerter Beiträge in beige und braun! :P} habe ich mir sagen lassen :mrgreen:) sogar schon einen Fan-Helden.\footnote{Aus der Taverne, "Tavernentratsch": [Galaphil:] Butterbrotbär, Gelehrter von Andor, Rang 42, 1 Helm, 1 großer und 3 kleine Ablagefelder. 14 Stärke, 3 Zeilen Willenspunkte von 0 bis 20. Je ein Würfel pro Zeile. Sonderfähigkeit: bei der Aktion Laufen kann der Butterbrotbär immer ein Feld weiter laufen als er Stunden ausgibt. Bei der Aktion Kämpfen bekommt er pro Willenspunktzeile 1 Würfel extra, den er für selbst oder mitkämpfende Helden verwenden können. Im geeinsamen Kampf gilt: Jede/-r HeldIn darf dabei maximal einen Würfel bekommen. Kämpft er alleine, darf er natürlich alle Würfel selbst verwenden.}\bigskip

Und zu guter Letzt gibt es einen beigen Teddy, welcher sich bereits als digitaler Butterbrotbär in einen Sherwood Forest begab.\footnote{Vom Zuckerberg, "Robin Hood geht immer!":\\\includegraphics[height=120px]{Das Erbe des Wunderkindes/Bilder/Robin Hood geht immer!.jpeg}} Diesen Teddy kenne ich seit Geburt, und er sieht aktuell aus einem Regal zu, wie ich diese Worte schreibe. :D\bigskip


Tja, und wenn ich nichts übersehen habe, war es das auch schon – bislang. Wenn ich Leanders wäre, würde ich wohl voraussagen, dass Thorald eines Tages in einem Trunkenen Traum ... aber nein, diese Fan-Legende liegt noch in weit entfernter Zukunft. ;)

}


% TODO: Erwähne den Butterbrotbär in Thoralds Trunkenen Traum \url{https://legenden-von-andor.de/forum/viewtopic.php?f=5&t=10208}

% TODO: Erwähne mivos Botenbrief \url{https://legenden-von-andor.de/forum/viewtopic.php?f=11&t=4695&start=7439}

% TODO: Erwähne den Butterbrotbär im DfL-Begleitheft





%

\part{Weitere Werke}

\fancyhead[L]{\nouppercase{\lastrightmark}}



\begin{chapterbox}
    \chapter{Weitere wilde Werke eines Butterbrotbären}
    \label{Weitere wilde Werke eines Butterbrotbären}
    Andorische Fan-Werke meiner Wenigkeit, welche im Netz zu finden sind. Für eine vollständige Liste, siehe \url{https://legenden-von-andor.de/forum/viewtopic.php?p=116297\#p116297}.
\end{chapterbox}




{\parindent0pt



\az{diverse Jahre}

\section{Viel zu viele Fan-Held:innen (ab 2019)}

\begin{center}
    Fan-Held:innen aus der Taverne
\end{center}

\bildmitts[width=0.67\textwidth]{Heldencompilation.jpg}





\newpage


\section{Würfelerwartungswertberechnungen (ab 2019)}


\begin{center}
    Berechnungen aus der Taverne

    \url{https://legenden-von-andor.de/forum/viewtopic.php?p=110899#p110899}
\end{center}

\bildmitts[width=\textwidth]{Würfelerwartungswerte Icons.jpg}

\bildmitts[width=\textwidth]{Würfelerwartungswerte Zahlen.jpg}








\newpage
\section{Der Tolle Troll (2021)}

\begin{center}
    Fan-Held aus der Taverne

    \url{https://legenden-von-andor.de/forum/viewtopic.php?f=13&t=6119}
\end{center}

\bildmitts{Der Tolle Troll (2021).jpg}



\textbf{Ideen, Texte und Gestalltung:} Boggart, Butterbrotbär, Lost in the Echo und Schlafende Katze mit allerlei Impulsen vom Troll persönlich!\bigskip

Die Tavernengemeinschaft gratuliert dem ebenso rundum einzigartigen wie ideenreichen, Geschichten erzählenden, recht außergewöhnlichen Troll ganz herzlich zu 3000 Beiträgen hier im Forum!!!

Angesichts dieses außerordentlichen Anlasses konnten wir Anhänger der ausgezeichneten andorischen Allgemeinheit nicht davon absehen, als Anerkennung aller Aktivität und Arbeit dieses Auskenners einen frischen Fan-Helden anzufertigen.

Die Idee zur Willenspunkte-Vulnerabilität dieses Wesens haben wir womöglich von jemandem entwendet ;)

Im strahlend sonnigen Sommer, so sagt man, sieht man dieses sprachbegabte, selbstsichere Subjekt auf dem schönen Feld Sechs, sowohl beim souverän sarkastischen, schadenfrohen Scherzen als auch stundenlangem, sehr seriösen Schwafeln mit skeptischen, schreckhaft schüchternen, nach sicheren, schmalen Straßen suchenden Spaziergängern, beim schlauen Sinnieren über schlagfertige, stilbewusste Strategien sowie beim sinnlichen, schläfrigen Schnarchen im schattigen Schutz sanfter Sträucher.
Zur zwielichtigen, zeitweise zappendusteren (Winter-)Zeit zieht es diesen zuversichtlichen Zweibeiner zielstrebig zügig zum ziemlich zentralen Zugang einer Zwergenhöhle zurückliegender Zeiten, seinem zeitweilig zukünftigen Zuhause Feld Zwohundertneunzehn. Dort argumentiert er allabendlich mit allerlei anwesenden Argen in ausschweifenden Alliterationen über allgemeine Ausnahmen absurden Ausmaßes.

Karrenweise kommen kreuzende Kreaturen kurzum zu diesem Kenner kreativer Kurzgeschichten.
Viele Feinde finden völlig fälschlich Freude an Forns vielversprechend frohlockendem Freund.
Und wie bei Freund Forns Fähigkeit haben fesche Päsche fiese Folgen für freche Feinde.

\textbf{Dreitausend Dank an den tollen Troll im Namen aller Tavernengäste!!!}
\textbf{Die coole Creativ-Gaming-Force gegen chaotischen Covid-Frust.}

PS: Zu den lästigen Legenden der letzten Hoffnung liegen lauwarme Ideen vor.






\newpage
\section{Die Prinzessin von Andor (2021)}


\begin{center}
    Fan-Mini-Erweiterung aus der Taverne

    \url{https://legenden-von-andor.de/forum/viewtopic.php?f=8&t=6186}
\end{center}

\bildmitts{Die Prinzessin von Andor (2021).jpg}

\textbf{Erfinder:innen:} Boggart/Kar éVarin, Butterbrotbär,
Lost in the Echo, Schlafende Katze und Troll\bigskip

\textit{Es kam daher, vor langer Zeit,} \textit{ein Gast in die Taverne, ganz gescheit.}

\textit{Unbekannt waren Weg und Ziel,} \textit{wir wussten nur sein' Namen: Galaphil.}

\textit{In der Taverne, wohl bekannt,} \textit{leiht er stets eine helfende Hand.}

\textit{Und vor bald schon zwei Jahr',} \textit{brachte er die Idee des Stammtisches dar.}

\textit{Seitdem sind noch mehr bei Gilda zu Gast} \textit{und machen da mit Freuden Rast.}

\textit{Wir hoffen, es bleibt noch lange so,} \textit{und sind darüber äußerst froh.}

\textit{3000 Beiträge konntest du nun schon schreiben,} \textit{wir sind sicher, dabei wird es nicht bleiben.}

\textit{Und laut erschallt's in Andor noch,} \textit{der Galaphil, der lebe hoch!}








\newpage
\section{Kreaturenkarten statt Kreaturenplättchen (2023)}

\begin{center}
    Fan-Spielvariante aus der Taverne

    \url{https://legenden-von-andor.de/forum/viewtopic.php?f=8&t=9668}
\end{center}

\bildmitts{Kreaturenkarten (2023).jpg}
 

Wann immer ihr ein gewöhnliches Kreaturenplättchen ausführen würdet, zieht stattdessen zwei zufällige Kreaturenkarten und legt sie zufällig nebeneinander. Die beiden schwarzen Ziffern bestimmen das Zielfeld (falls dieses nicht existiert, wechselt die 10er-Stelle zur weißen Ziffer) und das einzige vollständige Symbol zwischen den beiden Karten gibt an, was auf dort erscheint (von links nach rechts). Danach werden die zwei Kreaturenkarten zurück in den Stapel gemischt.

Falls eine Kreatur durch Kreaturenkarten erscheinen sollte, aber es keine dieser Art mehr im Vorrat hat, erscheint stattdessen eine Kreatur der nächstschwächeren noch vorhandenen Art.

Falls eine Kreatur durch Kreaturenkarten auf ein Feld gestellt werden sollte, von dem kein Pfeil wegzeigt und/oder welches ein Kreaturen-Zielfeld ist (z.B. Rietburg, Lager der Tulgori oder Lager der Trolle), wird die Position dieser Kreatur stattdessen mit einem roten (10er-Stelle) und einem Heldenwürfel (1er-Stelle) ausgewürfelt, gegebenenfalls mehrfach.

Falls eine Kreatur durch Kreaturenkarten auf ein Feld gestellt wird, auf dem Bauern liegen, und von dem ein Pfeil wegzeigt, dürft ihr die Bauern auf das nächste kreaturenfreie Feld entlang der Pfeile versetzen.









\newpage
\section{ANDOR JUNIOR Erweiterungspaket (2020)}

\begin{center}
    Fan-Spielvariante aus der Taverne

    \url{https://legenden-von-andor.de/forum/viewtopic.php?f=8&t=5567}
\end{center}

\bildmitts{ANDOR JUNIOR Erweiterungspaket (2020).jpg}

\textbf{Erfinder:innen:} Schlafende Katze, Troll und Butterbrotbär

Kürzlich ist mir ein mysteriöses Paket per Falke zugestellt worden. Daran war ein Brief befestigt, welcher sich wie folgt anhört:

\textit{Dieses Andor-Junior-Fan-Erweiterungs-Paket besteht aus 11 Bonus-Helden, 9 Bonus-Aufgaben sowie einem Bonus-Endgegner.}

\textit{Die 11 Bonus-Helden sind Junior-Varianten der 11 noch nicht als Junior-Helden erschienenen offiziellen Helden (ausser Stinner). Jeder dieser Bonus-Helden ersetzt den offiziellen Junior-Helden seiner Würfelfarbe (bei Darh einer der drei). Prinzipiell könnt ihr aber auch mehrere Helden derselben Würfelfarbe gleichzeitig spielen, solange ihr im Voraus bestimmt, welcher Held welcher Heldenfarbe entspricht, wenn das für eine Aufgabe relevant sein sollte.}

\textit{Die 9 Bonus-Aufgaben können beliebig miteinander und mit den ursprünglichen Aufgaben gemischt werden. Sie benötigen oftmals Plättchen und Figuren aus dem Grundspiel oder einer Grundspiel-Erweiterung, diese können aber leicht ersetzt werden, falls ihr sie nicht besitzt.}

\textit{Mit dem neuen Endgegner könnt ihr euch nun endlich dem Drachen Tarok im Kampf stellen und die riesige Echse furchtlos vertreiben.}

\textit{Zu guter Letzt sei noch erwähnt, dass wir aus Gründen des Helden-Balancing empfehlen, bei der originalen Junior-Magierin gewürfelte Blitze als Fackeln zählen zu lassen.}

Spannend! Nur leider sieht es so aus, als hätte ein gewisser Schmierfink die Karten bereits vor mir in die Finger gekriegt und seine Meinung mit dunkler Tinte kundgetan.


\begin{center}
    Beiträge aus der Taverne

    "1. Tavernenstammtischparty!" (2020)
\end{center}

[Schlafende Katze:] Naja, eigentlich war ich ja schon die ganze Zeit da ;) Ich darf euch heute noch was zu Andor Ju...*\textbf{FAUCH!!!}*

[Varkur himself:] Muhahaha! :twisted: Seid gegrüßt, erbärmliches Gesindel! Habt ihr gedacht, ihr wäret mich los? Von Stund an werde ich sogar eure Kinder nicht mehr in Ruhe lassen! {\footnotesize Muhahahahaa! :twisted:}

[Towa:] Oh! Mit dem habe ich nicht gerechnet. Was willst du hier?

[Varkur himself:] Ich werde das Rietland brennen lassen! {\footnotesize Muhahahahaa! :twisted:}

[Tavernengaukler:] Der schon wieder... :x :o :lol:

[Varkur himself:] Schweig, Unwürdiger! Außer du willst um Gnade betteln. {\footnotesize Muhahahahaa! :twisted:}

[Lost In The Echo:] Varkur? Was hast du uns mitzuteilen? Sprich, bevor du aus diesem Ort des Friedens verstoßen wirst!

[Varkur himself:] Egal wer von euch sich mir in den Weg stellt, ob billiger Bettvorleger, wolfsvernarrter Wilder, würdelose Wasserplanscherin oder mickriger Möchtegernmensch: Ihr werdet scheitern! Ich kann es mit allen Helden aufnehmen! \textbf{ALLEN!} {\footnotesize Muhahahahaa! :twisted:}

Ich bin Varkur! Der mächtigste Dunkle Magier aller Zeiten! {\footnotesize Muhahahahaa! :twisted:}

Euch erwarten zahllose (neun) neue Herausforderungen: Torkelnde Trolle, aggressive Arbaks, fiese Fluggors... und meine Wenigkeit! Und selbst wenn ihr all das übersteht, wird es nicht mehr reichen, irgendwelche winselnden Welpen zu ihrer Mama zurückzuschleifen! Ihr werdet es mit Tarok persönlich aufnehmen müssen! :twisted: {\footnotesize Muhahahahaa! :twisted:}

[Lost In The Echo:] Da du ja scheinbar nicht an einer konstruktiven Unterhaltung interessiert bist, möchte ich in die Gruppe fragen, wer wohl mit dem billigen Bettvorleger gemeint ist : :?

[Varkur himself:] Der provisorische Pelzkragen natürlich... {\footnotesize Muhahahahaa! :twisted:}

[Schlafende Katze:] Was zu trinken? Oder doch lieber die Tür?

[Varkur himself:] Ein Rachenputzer … mit Olive. :) {\footnotesize Muhahahahaa! :twisted:}

[Schlafende Katze:] NARAVEN!!! Hast du noch was übrig von deinem Cocktail für unseren Gast?

[Boggart:] Einmal Thoraldmilch mit Olive, bitteschön! Wohl bekomms!

[Varkur himself:] Danke sehr. :P {\footnotesize Muhahahahaa! :twisted:}

[Varkur himself:] Ich habe schon genug Zeit in dieser verranzten Taverne verschwendet! Ich gehe wann ich will, nicht weil mich jemand vertreibt, das muss ich klarstellen. Aber ich werde wiederkommen! Ihr werdet gar nicht erst gegen den Drachen kämpfen, weil ihr davor an mir vorbeimüsst. An MIR!
Bis dann… {\footnotesize Muhahahahaa! :twisted:}

Muhahahaha! :twisted: *puff*

[Schlafende Katze:] Was ich eigentlich sagen wollte, bevor Varkur mich unterbrochen hat:

Der Troll, Butterbrotbär und ich haben uns ein bisschen mit Andor Junior beschäftigt.

Es sind \textbf{ALLE} offiziellen Helden jetzt spielbar und es gibt insgesamt neun neue Aufgaben.
Zusätzlich zu denen, die Varkur schon angedeutet hat gilt es Runensteine den Gors abzuluchsen, die Tulgori durch das Rietland zu begleiten und dem Trunkenen Troll unter die Arme zu greifen. Natürlich alles Kindgerecht!

Und als wäre das nicht genug, gibt es statt der Wolfssuche einen spannenden Drachenkampf am Schluss :)




\newpage
\az{Jahr 61}
\section{Arbon auf der Flucht (2021)}

\begin{center}
    Fan-Legende aus der Taverne

    \url{https://legenden-von-andor.de/forum/viewtopic.php?f=5&t=6396}
\end{center}

\bildmitts{Arbon auf der Flucht (2021).jpg}

\textbf{Erfinder:innen:} Butterbrotbär, Schlafende Katze, Troll

\textbf{Gestaltung:} Schlafende Katze

\textbf{Inhaltsangaben:} Diese Legende spielt nach dem Storytext „Die Geschichte des drittbesten Bogenschützen“ und vor dem Storytext „Die Hüterin und die Hexe“. Nachdem Arbon von Melkart, Pago und Folla in den Schwarzen Archiven überrascht wurde, befindet er sich nun auf der Flucht vor dem jungen Fährtenleser Fenn und den wütenden Bewahrern. In dieser Legende spielt einer der Mitspieler den Helden Arbon, der sich unsichtbar über das Spielfeld bewegt, während die Jäger sich auf die Suche nach ihm begeben. Dabei müssen sie den Verräter nicht nur finden, sondern auch zum Baum der Lieder zurück schleifen. Wenn nur die zusätzlichen Aufgaben und die vielen nervigen Gors nicht wären. Doch auch für Arbon ist die Flucht nicht leicht, denn scharfe Vogelaugen und die Jäger machen ihm das Leben schwer.

\textbf{Spielerzahl:} 2-4 Spieler

\textbf{Spielmaterial:} Grundspiel + Neue Helden

Erlebt ein unvergessliches (Schlafende-)Katz-und-Maus-Spiel quer durch Andor. Schwitzt mit Arbon, verzweifelt mit den Jägern und lasst euch nicht von unnötigen Storykarten irritieren!

\begin{center}
    Beiträge aus der Taverne

    "2. Tavernenstammtischparty 2021"
\end{center}

[Hogo:] für mich ein wasser, bitte. 
danke. mein gesicht ist sehr langweilig.

[Schlafende Katze:] Hogo, ein bisschen destruktiv eingestellt? Sooo hässlich bist du auch nicht :P

[Hogo:] hässlich ist auch nicht langweilig.

[Schlafende Katze:] mit Kapuze schwer zu beurteilen :P

[Hogo:] ja.

[Butterbrotbär:] Sind alle hier? Können wir loslegen? :D

[Hogo:] alle da! jetzt tür verriegeln und niemanden mehr reinlassen.

[...]

[AB von dem Andorwiki:] Hogo, beherrschst du die Handwerkkunst?

[Hogo:] nein. ich beherrsche gar nichts. beachtet mich nicht.

[Pago:] *In die Taverne stürz* Habt ihr ihn gesehen?
So einen jungen Barbar... Bisschen kleiner wie ich. Rote Haare. Rabe? Wenn, Denn, Venn... oder irgendwie so müsste er heißen. Ich hab eine Nachricht von ihm bekommen, dass er hier irgendwo wäre

[AB von dem Andorwiki:] Fenn?

[Pago:] Ja, genau so war der Name!!! Habt ihr ihn gesehen?

[Troll:] Nö, war schon seit 2016 (?) nicht mehr hier.

[Pago:] :shock: Das kann gar nicht sein! Ich habe ihn erst vor kurzem am Baum der Lieder getroffen, als er einen Auftrag angenommen hat.

[Butterbrotbär:] Wie kommst du denn darauf, dass Fenn hier wäre? Wollte er auch das Stammtisch-Jubiläum feiern?

[Pago:] Schön für euren Stammtisch. Dabei gibt es hier viel wichtigeres!!!

[Lost In The Echo:] Kannst du uns beschreiben, was er für Kleidung trug, als du ihn das letzte Mal gesehen hast?

[Pago:] Nunja. Keine Ahnung. Braun eben. Ist auch nicht so wichtig.... Wir hatten einige Schwierigkeiten am Baum der Lieder und Fenn sollte und bei der Suche nach einem verdächtigen Subjekt helfen
Ein mieser Verräter. Ehemals ein Wächter des Schwarzen Archivs. Hat Melkart mit einem Messer bedroht und ist dann abghauen. Arbon ist der Name dieses unwürdigen Stück Drecks...
Habt ihr wenigstens Kunde von IHM? Oder versteckt ihr ihn sogar in eurer Mitte?
*Durch die Taverne geh* *Besucher muster*

[AB von dem Andorwiki:] *zu Hogo schau* :?

[Hogo:] *sich ein bisschen in den schatten duck und leise die kekse mümmel

[Pago:] *vor dem Mann mit der schwarzen Kapuze stehen bleib, der Kekse mümmelt wie ein Kaninchen* *mit Bogen Kapuze aus seinem Gesicht streich* Wen haben wir denn hier? ARBON So sieht man sich wieder!

[Hogo:] abron? welch ein unvertraut klingender name. Ich bin hogo, ein einfacher knecht, dem der segen der bildung nicht zuteil wurde. ich kann auch nicht schreiben (außer das hier) und in den wachsamen wald habe ich zu lebzeiten noch keinen fuß gesetzt.
(und wenn ich doch dieser arbon wäre, wäre ich sicher nur hier, um mitzufeiern, und bestimmt nicht um in der menge der feiernden unterzutauchen)

[Pago:] Ach komm, erzähl keinen Stuss. Den kauft dir hier eh keiner ab! *Arbon am Schlaffitchen pack und vom Stuhl zerr*

[Hogo:] *dagegenzerr!!! ha! 6! das übertriffst du nicht! 8-)

[Pago:] Mist, eine 3...

[Hogo:] *schnell einen keks in den mund steck* sag ich doch! 8-)

[Pago:] *Arbon versuch noch eines ruterzuhauen* 4! du entkommst mir nicht! *Über Stuhl im Weg hechte*

[Hogo:] *bein stell

[Towa:] Arbon, brauchst du noch ein Heilkraut? Ich habe eines dabei!

[Hogo:] *wegrenn

[Pago:] *der Länge nach hinschlag* *Blut aus dem Gesicht wisch* *Schwer schnauf*

[Hogo:] kopf hoch, pago! (der ist so leicht, den kannst sogar du heben.) in anbetracht der schieren vielzahl deiner fehlschläge sind gewiss auch welche dabei, die nicht auf deine mangelden fähigkeiten zurückgehen :P *heldenfigur vom spielplan nehm

[Pago:] *Towa das Heilkraut abnehm* Ich danke

[Hogo:] Leise Stimme aus der Ferne: tut mir leid für die ziegenmilch und die gläser! pago bezahlt sicher gerne! tschühüss!

[Pago:] Verdammt, jetzt ist diese miese kleine Ratte schon wieder entwischt. Und von Fenns Raben keine Spur, wenn man mal einen Vogel braucht. So ein Mist! Dann muss ich wohl so nach ihm suchen. Vielleicht finde ich ja noch irgendwo Folla. *Hinter Arbon her aus der Taverne renn*
*umdreh und nochmal zur Türe reinschau* Wenn ihr mir doch noch helfen wollt, ihr wisst ja, wo ihr mich findet!!! [Link zu Arbon auf der Flucht] *Tür hinter sich zuschlag*

[Hogo:] *zum Fenster hereinschau* Wenn ihr mir helfen wollt übrigens auch. 8-)


\newpage
\az{Jahr 64}
\section{Flapsige Helden ins Land der Lagerfeuer (2023)}

\begin{center}
    Fan-Spielvarianten aus der Taverne

    \url{https://legenden-von-andor.de/forum/viewtopic.php?f=8&t=8615}
\end{center}

\bildmitts{Flapsige Helden ins Land der Lagerfeuer (2023).png}

\textbf{Alle Helden ins Land der Steppe}

Fan-Spielvariante, um mehr Helden, auch aus (annähernd) allen anderen Andor-Teilen, in DeK einzusetzen (nicht zwingend für die Einführungslegende geeignet).\bigskip

\textbf{Wächter der Lagerfeuer}

Fan-Mini-Erweiterung, um Helden aus DeK auch in (annähernd) allen anderen Andor-Legenden einzusetzen, oder einfach um mehr Variabilität ins Spiel zu bringen.
Einige Brunnen/Quellen kommen aus dem Spiel, dafür werden DeK-Feuerplättchen ins Spiel gebracht. Damit die Brunnen-Helden nicht traurig werden, könnt ihr auf die Feuer-Hausregeln aus "Alle Helden ins Land der Steppe" zurückgreifen.\bigskip

\textbf{Flapsig, fledrig und fuchsig}

Fan-Mini-Erweiterung, um Flaps Flederfuchs auch in anderen Andor-Teilen einzusetzen.




\newpage
\section{Zeit entzweit (2023)}

\begin{center}
    Fan-Legende zu "Die Kreaturen des Ava (2023)", aus der Taverne

    \url{https://legenden-von-andor.de/forum/viewtopic.php?f=25&t=9655}
\end{center}

\bildmitts{Zeit entzweit (2023).jpg}

\textbf{Erfinder:} Butterbrotbär

\textbf{Inhaltsangabe:} Instabile Zeitportale verbinden zwei Zeiten: Das herbstliche Andor, dessen Bewohner noch nichts von der ewigen Kälte ahnen, und das winterliche Andor, wo einige letzte Helden die Rietburg tapfer verteidigen. Ein mysteriöser Gegner ruft verschiedene Kreaturen des Ava herbei, die die Helden beschäftigen sollen. Doch nicht alle Kreaturen des Ava müssen im Kampf besiegt werden, denn sie wurden durch magische Banne in die Kontrolle des Gegners gebracht. Und es gibt diverse Möglichkeiten, solche Bänne zu brechen...

\textbf{Spielmaterial:} Grundspiel und "Die ewige Kälte", Andor-Spielpläne aus Grundspiel und "Die ewige Kälte"

\textbf{Schwierigkeitsgrad:} mittel bis schwer

\textbf{Chronologische Einordnung:} Während der vierten Legende aus Die ewige Kälte – unter anderem. ;)

\textbf{Heldenanzahl:} 2-4 getestet (empfohlen: 4 Helden), 5/6 theoretisch möglich

\textbf{Anmerkung:} Enthält signifikante SPOILER für Die ewige Kälte. Die Karten „Vorgeschichte“ und „Epilog“ enthalten ganz leichte Spoiler für „Düstere Zeiten“, können aber problemlos übersprungen werden.






\newpage
\az{Jahr 65}
\section{Legende 5 + Befreiung der Rietburg (2020)}

\begin{center}
    Fan-Spielkombination aus der Taverne

    \url{https://legenden-von-andor.de/forum/viewtopic.php?f=8&t=5295}
\end{center}

\bildmitts{Legende 5 + Befreiung der Rietburg (2020).jpg}

\textbf{Erfinder:} Boggart/Kar éVarin, unter Mitwirkung von Butterbrotbär, Galaphil, Schlafende Katze und Troll

\textbf{Inhaltsangabe:} Seit ich wusste, um was es in "Die Befreiung der Rietburg" geht, habe ich mich gefragt, ob sich das Spiel mit Legende 5: "Der Zorn des Drachen" kombinieren lässt. Vor euch liegt meine Antwort auf diese Frage!

Ab jetzt müssen die Kreaturen in der Rietburg nicht mehr mit Bögen bekämpft werden. Stattdessen können die Helden die Burg betreten und ihnen vor Ort den Garaus machen.
Der Schwierigkeitsgrad der Legende sollte sich dadurch nicht übermäßig ändern, maximal der Glücksfaktor wird noch etwas erhöht.



\newpage
\az{Jahr 66}
\section{Die verschwundene Krone (2021)}

\begin{center}
    Fan-Legende aus der Taverne

    \url{https://legenden-von-andor.de/forum/viewtopic.php?f=5&t=6445}
\end{center}

\bildmitts{Die verschwundene Krone (2021).jpg}

\textbf{Erfinder:} Butterbrotbär

\textbf{Inhaltsangabe:} Thoralds Krönung zum König von Andor kann nicht stattfinden, weil die Rietgraskrone verschwunden ist. Währenddessen geraten einige Tulgori im Grauen Gebirge in Schwierigkeiten.

\textbf{Chronologische Einordnung:} Frühjahr 66 a.Z.

\textbf{Benötigtes Material:} Grundspiel, Die Reise in den Norden und Die letzte Hoffnung.

\textbf{Spielpläne:} Graues Gebirge, Andor und Nebelinseln

\textbf{Schwierigkeitsgrad:} Schwer

\textbf{Heldenanzahl:} 2-4 getestet, 5/6 theoretisch möglich



\begin{center}
    Beiträge aus der Taverne

    "1. Tavernenstammtischparty!" (2020)
\end{center}

\bildmitts{Die verschwundene Krone Ankündigung Bild 1.jpeg}

Ihr habt ja bestimmt auch schon die Gerüchte gehört, dass die Helden von Andor einst eine ereignisreiche Epoche erlebt hätten, in welcher manche von ihnen auf der Aldebaran ins Hadrische Meer stachen, während andere zur selben Zeit Meilen entfernt durchs Graue Gebirge stapften. Doch eine solche Legende wurde zumindest hier in der Taverne noch nicht erzählt...

So suchte ich eines gemütlichen Abends hier in ebendiesem Schankraum die Gesellschaft von Fenn dem Fährtenleser und fragte ihn, ob an jenen Gerüchten etwas Wahres dran sei.

\bildmitts{Die verschwundene Krone Ankündigung Bild 2.jpeg}

Fenn ist begnadeter Geschichtenerzähler, und so dauerte es nicht lange, bis er begeistert eine Erzählung zum Besten gab, von einer Zeit, als Prinz Thorald eigentlich zum König von Andor gekrönt hätte werden sollen, als gleich zwei unerwartete Angelegenheiten dazwischen kamen, welche die Helden in entgegengesetzte Himmelsrichtungen schickten...

Zurück in meiner gemütlichen Hütte versuchte ich bei einer dampfenden Tasse kostbaren Schokogebräus, das Gehörte, so gut ich konnte, zu Pergament zu bringen.

\bildmitts{Die verschwundene Krone Ankündigung Bild 3.jpeg}

Laut Fenns Bericht waren einige Helden vor Sidra, der Küste Andors, an Bord der Aldebaran gegangen... doch als ich dies mit einigen alten Karten Andors abglich, war vor Sidra gar kein Anlegeplatz eingezeichnet! Kurzerhand musste ich mir einen hinzudichten. Ähnlich erging es mir bei der Suche nach einem durchgehenden Weg ins Graue Gebirge. Und an manchen Stellen könnte Fenn im Erzählfieber die Wahrheit etwas ausgeschmückt haben... ein waschechter Arrog am Baum der Lieder?!

Kleine Hindernisse waren etwa meine Unerfahrenheit in der Kunst des Legendenschreibens oder auch mein Stubentiger, der in all meinen Notizen immer bloss Schlafgelegenheiten sah. :P

\bildmitts{Die verschwundene Krone Ankündigung Bild 4.jpeg}

Zum Glück konnte ich auf die umfangreiche Unterstützung anderer Andori zählen!
Nun hoffe ich, diese verworrene Geschichte bald hier vorstellen zu können – und dass sie vielleicht andere Tavernengäste dazu anregt, weitere Berichte dieser Sorte zusammenzutragen.




\newpage
\az{Jahr 71}
\section{Krankheit in Cavern (2022)}

\begin{center}
    Fan-Legende aus der Taverne

    \url{https://legenden-von-andor.de/forum/viewtopic.php?f=5&t=7093}
\end{center}

\bildmitts{Krankheit in Cavern (2022).jpg}


\textbf{Erfinder:} Butterbrotbär

\textbf{Inhaltsangabe:} Unsere Helden wollen kranke Zwerge in der Schatzkammer Caverns versammeln, auf dass diese dort geheilt werden können. Wenn ihnen nur nicht all dieses Geröll und all diese Kreaturen den Weg versperrten …

\textbf{Benötigtes Material:} Grundspiel, Alte Geister („Feld 37“-Überleger, eine zufällige Quelle, „Baum der Lieder“-Plättchen und Hallgard) und Düstere Zeiten („Feld 83“-Überleger, Mart und Kjall).

\textbf{Benötigte Vorkenntnisse:} Minen-Regeln aus Legende 4 und Gift-Regeln aus Legende 5.

\textbf{Heldenanzahl:} 2-4 getestet, 5/6 theoretisch möglich

\textbf{Chronologische Einordnung:} Ende 71 a.Z.

\textbf{Schwierigkeitsgrad:} Leicht

\textbf{Spielplan:} Cavern



\newpage
\az{Jahr 74}
\section{Der sagenumstohlene Eismet (2021)}

\begin{center}
    Fan-Abenteuer aus der Taverne

    \url{https://legenden-von-andor.de/forum/viewtopic.php?f=8&t=6658}
\end{center}

\bildmitts{Der sagenumstohlene Eismet (2021).jpg}


\textbf{Erfinder:} Butterbrotbär

\textbf{Inhaltsangabe:} Das einst vom mächtigsten Magier Orweyn über den Eingang des Eisernen Turms gelegte Siegel war schon vor langer Zeit vom jungen Magier Varkur gesprengt worden. Doch das hinderte die Zauberer des Turms nicht daran, die darin eingeschlossenen Geheimnisse wacker zu wahren. Bis jetzt.

In späteren Jahren sollten sich Chronisten darüber uneinig sein, wann genau die Tore zum Eisernen Turm geöffnet wurden. Drangen Zauberer des Feuers mithilfe der Helden von Andor in sein Inneres, um die legendären Magischen Waffen zu erringen? Brach der Trunkene Troll auf der Suche nach dem sagenumwobenen Eismet aus der Vergangenheit Hadrias in das geschichtsträchtige Gemäuer ein? Fest steht, dass der Eisernen Turm in diesen chaotischen Zeiten kurzzeitig nicht mehr gesichert wurde.

So gelang es einem diebischen und außergewöhnlich willensstarken Nerax, ein wertvolles Pergament aus dem jahrhundertelang verschlossen gebliebenen Turm zu erringen. War dies etwa das Geheimrezept für den sagenumwobenen Eismet, von dem Braumeister Borabil einst dem wackeren Andori Erloth berichtet hatte?

Earas Aufgabe: Überwinde den diebischen Nerax und nimm das gestohlene Pergament wieder an dich!


\textbf{Benötigtes Material:} Modifizierte "Adventures of Hadria"-Datei.

\textbf{Spielplan:} Feste von Yra

\textbf{Schwierigkeitsgrad:} Sehr leicht

\textbf{Heldenanzahl:} 1 (Eara)






\newpage
\az{Jahr 75}
\section{Alle Helden ins Graue Gebirge! (2019)}

\begin{center}
    Fan-Spielvariante aus der Taverne

    \url{https://legenden-von-andor.de/forum/viewtopic.php?f=8&t=4871}
\end{center}


\bildmitts{Alle Helden ins Graue Gebirge! (2019).jpg}

\textbf{Erfinder:} Butterbrotbär

\textbf{Inhaltsangabe:} Mit dieser Spielvariante sollte man alle offiziellen Helden des Grundspielplans und viele Fan-Helden jetzt auch in Andor Teil III spielen können.

Diese Karten hier können nicht und sollen auch nicht die auf einzelne Helden, deren SF und deren Charakterentwicklung abgestimmten Helden-, Erschöpfungs- und Hoffnungskarten ersetzen. Vielmehr ging es mir darum, möglichst allgemeine Helden-, Erschöpfungs- und Hoffnungskarten zu erstellen, damit man auch in der letzten Hoffnung eine möglichst offene Heldenwahl hat.

Praktisch ist, dass man die allgemeinen Karten aus dieser Spielvariante problemlos mit spezifischen Karten mixen kann: Falls jemandem beispielsweise eine gute zweite Sonderfähigkeit für Orfen einfällt, kann er diese selbstverständlich benutzen – und für alle noch fehlenden Karten (z.B. Geschenkaufgaben und Erschöpfungskarten) einfach auf Karten aus dieser Fan-Erweiterung zurückgreifen.






\newpage
\az{Jahr 76}
\section{Renn, Forn, renn! (2022)}

\begin{center}
    Fan-Legende aus der Taverne

    \url{https://legenden-von-andor.de/forum/viewtopic.php?f=5&t=7893}
\end{center}

\bildmitts{Renn, Forn, renn! (2022).jpg}

\textbf{Erfinder:} Butterbrotbär


\textbf{Inhaltsangabe:} Eine Horde gepanzerter Trolle trottet nach der Befreiung der Winterburg auf den ungeschützten Tross der Andori zu. Forn, der scheue Schattenskral, ist stark und schnell, doch auch weit weg. Wird der Halbskral noch rechtzeitig zur Rettung heranrasen können? Manch ein anderer Dunkler Held wird ihn auf seinem Weg unterstützen – oder selbst Unterstützung benötigen.

\textbf{Benötigtes Material:} Die letzte Hoffnung \& Dunkle Helden

\textbf{Chronologische Einordnung:} 76 a.Z., nach Legende 16

\textbf{Schwierigkeitsgrad:} Leicht

\textbf{Spielplan:} Graues Gebirge

\textbf{Heldenanzahl:} 1 (Forn)





}










\begin{chapterbox}
    \chapter{Weitere wunderbare Werke der Andor-Fan-Community}
    \label{Weitere wunderbare Werke der Andor-Fan-Community}
    Einige außergewöhnliche andorische Fan-Werke, welche im Netz zu finden sind. 
\end{chapterbox}

\extramarks{}{}

{\parindent0pt

\section{Die Taverne in Aufruhr (2014)}

\begin{center}
    Fan-Legende aus der Taverne

    \url{https://legenden-von-andor.de/forum/viewtopic.php?f=5&t=943}
\end{center}


\bildmitts{Gildas Bild der Woche Das brachte der Osterhase! (2015).jpg}


\textbf{Fan-Legenden-Erfinder:} Diese Legende wurde zwischen Mai und September 2014 von den Mitgliedern des offiziellen Forums “Die Taverne von Andor” entwickelt.

\textbf{Inhaltsangabe:} Vor langer Zeit hatte Erloth aus den Überresten eines zerstörten Karrens einen Marktstand errichtet, verkaufte Met und Wein, und bald war daraus eine kleine Hütte geworden - “Die Taverne zum Trunkenen Troll”. Händler, Reisende und Helden gingen ein und aus, und die Wirtin Gilda, Erloths Enkelin, war im ganzen Land bekannt und beliebt. Oft genug waren Kreaturen durch Andor gezogen, doch nie hatte eine von ihnen gewagt, die Taverne anzugreifen, und kein Mensch hatte das jemals für möglich gehalten. Bis zu jenem Morgen, als ein Bauer die Helden erreichte und rief: “Kommt schnell! Der Trunkene Troll verwüstet die Taverne!”

\textbf{Schwierigkeitsgrad:} mittel

\textbf{Spieleranzahl:} Die Legende kann mit 2-6 Helden gespielt werden.

\textbf{Info:}
Die Legende spielt auf der Vorderseite des Grundspiel-Plans. Es werden die Sternenschild-Erweiterung und die Ergänzung "Neue Helden" benötigt.









\newpage
\section{Der sagenumwobene Eismet (2016)}

\begin{center}
    Fan-Legende aus der Taverne

    \url{https://legenden-von-andor.de/forum/viewtopic.php?f=5&t=2744}
\end{center}



\bildmitts{Der sagenumwobene Eismet (2016).jpg}


\textbf{Fan-Legende:} Der sagenumwobene Eismet
\textbf{2. Fan-Legende der Tavernengemeinschaft}

\textbf{Fan-Legenden-Erfinder:} Diese Legende wurde zwischen Mai 2015 und August 2016 von den Mitgliedern des offiziellen Forums “Die Taverne von Andor” entwickelt.

\textbf{Inhaltsangabe:} Qurun war besiegt! Doch seine letzten Worte hallten den Helden in den Ohren nach. `Andor brennt und ihr seid weit weg...´ hatte er ihnen noch im Sterben höhnisch ins Gesicht geschleudert. Die Helden verloren keine Zeit und brachen sofort auf. `Wir müssen zurück nach Andor!´Doch als sie endlich den Steg erreichten, wo die Aldebaran vor Anker lag, sahen sie mit Entsetzen, dass der Mast gebrochen und das Segel zerfetzt war...

\textbf{Schwierigkeitsgrad:} variabel, von leicht bis sehr schwer

\textbf{Spieleranzahl:} Die Legende kann mit 2-6 Helden gespielt werden.

\textbf{Info:}
Die Legende spielt auf der Rückseite des Nord-Spielplans (Hadria). Es werden neben dem Grundspiel die Erweiterungen „Die Reise in den Norden“, „Der Sternenschild“ und „Neue Helden“ benötigt.
Die Legende schließt inhaltlich direkt an die Geschehnisse der Legende 10 an. Bei der Entwicklung der Legende wurde ein ganz besonderes Augenmerk auf den erzählerischen Aspekt gelegt.

\textbf{Spieldauer:} min. 2 Stunden








\newpage
\section{Das Lied des Königs – Die Neuen Helden (2019)}

\begin{center}
    Fan-Legende aus der Taverne

    \url{https://legenden-von-andor.de/forum/viewtopic.php?f=5&t=4817}
\end{center}

\bildmitts{Gildas Bild der Woche Legendäre Freuden (2019).jpg}


\textbf{Fan-Legenden-Erfinder:}
Entstanden als Gemeinschaftsprojekt in der Taverne: Projekt: Das Lied des Königs

\textbf{Inhaltsangabe:}
Derzeit erreichen uns Berichte, wie die Neuen Helden nach Andor kamen. Auch wir stießen auf solche Geschichten, doch berichten unsere Quellen von ganz anderen Geschehnissen.
Lasst Euch erzählen, was wir über Arbon, Bragor, Fenn, Kheela, Orfen und Forn zur Zeit der Anfänge der Helden von Andor hörten!

\textbf{Schwierigkeitsgrad:}
mittel bis schwer, abhängig von Helden- und Endgegnerwahl

\textbf{Spielerzahl:} 1-6 Helden

\textbf{Info:}
Die Legende spielt auf dem Andor-Spielplan. Neben dem Grundspiel werden die Erweiterungen "Der Sternenschild" und "Die BonusBox", die Feinde von Andor sowie, je nach Heldenwahl, die Merhspielererweiterungen "Neue Helden" und "Dunkle Helden" benötigt.
Die Legende spielt parallel zum Roman "Das Lied des Königs" und nimmt gelegentlich Bezug auf dessen Handlung. Die Kenntnis des Romans wird nicht vorausgesetzt, doch können wir jedem, der ihn noch nicht gelesen hat, dies nur empfehlen!

\textbf{Spieldauer:}
Abhängig von der Heldenwahl, da jeder Held seine eigenen Aufgaben mit sich bringt.





\newpage
\section{Ära der Händler (2014)}

\begin{center}
    Fan-Erweiterung aus der Taverne

    \url{https://legenden-von-andor.de/forum/viewtopic.php?f=8&t=1162}
\end{center}

\bildmitts{Ära der Händler Tafeln.jpg}

548 Beiträge, 59 Seiten, 5 Threads und etwas mehr als 2 Monate hat es gedauert bis wir euch eine (fast) fertige Version der großen Gegenstandserweiterung - \textbf{Ära der Händler} heute vorstellen können. Auf \textbf{5} neuen Händlertableaus findet ihr über \textbf{30} neue Gegenstände mit einzigartigen Funktionen, die nur darauf warten von euren Helden, im Kampf gegen das Böse, eingesetzt zu werden. Ich wünsche euch viel Spaß und Würfelglück ;)!

Ein großer Dank an \textbf{alle} Beteiligten die stets getestet, hinterfragt und die nötigen Ideen geliefert haben, ohne dies wäre diese Erweiterung nie so toll geworden!
Ganz besonders möchte ich mich bei Garz der Händler, Fussssohle, Speerkämpfer, Ent, Tost, RoninBubu, Mjölnir und Karl Schlag bedanken, die viel Zeit in die Entwicklung gesteckt haben.



\newpage
\section{Tulgorische Abenteuer (2020)}

\begin{center}
    Fan-Legenden aus der Taverne
\end{center}

\bildmitts{Tulgorische Abenteuer (2020).jpg}



\begin{center}
    "Wüstenabenteuer: Der Merasteintransport"

    \url{https://legenden-von-andor.de/forum/viewtopic.php?f=5&t=5370}
\end{center}

\textbf{Fan-Legenden-Erfinder:} Falkenstadl

\textbf{Einleitung}
Die Zeiten nach den vielen Kämpfen in Andor waren von Hoffnung erfüllt. Hoffnung auf einen Frieden, der es allen erlaubte in Sicherheit mit ihren Familien zu leben und zu arbeiten. Die Arbeit war schwer, ja – aber sie war erfüllend und ernährte die ganze Sippschaft.
Nach getaner Arbeit konnte man sich auch wieder gefahrlos in der Umgebung bewegen um die Taverne zu besuchen. Sie war nicht nur eine Zuflucht in Notzeiten, sondern auch stets der Ort zum Austausch der interessentesten Neuigkeiten. So kam es an einem Abend im „Trunkenen Troll“ zu einem Treffen unserer Helden von Andor. Man gönnte sich ein gut gefülltes Horn und sprach über allerlei Belangloses und über Heldentaten, aber auch über so manchen traurigen Verlust.




\begin{center}
    "Wüstenlegende: Verschollene Freunde"

    \url{https://legenden-von-andor.de/forum/viewtopic.php?f=5&t=5590}
\end{center}

\textbf{Fan-Legenden-Erfinder:} Phoenixpower


\textbf{Inhaltsangabe:} Bei der Überquerung des Fahlen Gebirges wurden Thorn, Kram und Marek, der Sohn Merriks von Chada und Eara getrennt. Nun versuchen Thorn und Kram gemeinsam mit ihren neuen Freunden Tugor und Tugira die beiden wiederzufinden.

\textbf{Schwierigkeitsgrad}: Variabel
Je weniger Helden mitspielen, desto leichter ist die Legende, zudem könnt ihr die von Falkenstadl beschriebene „\textit{Wünsch dir was-Methode}“ nutzen.

\textbf{Benötigtes Material:} Grundspiel, das Wüstenabenteuer von Falkenstadl, diese Legende





\begin{center}
    "Abenteuer in Tulgor: Der Spruch des Gepodon"

    \url{https://legenden-von-andor.de/forum/viewtopic.php?f=5&t=5851}
\end{center}


\textbf{Fan-Legenden-Erfinderin:} Flederfluse

\textbf{Legendenziel:} Beschützt die Erträge der tulgorischen Feldwirtschaft. Das Getreide, die Schafe, die Hühner und die Obstbäume.

Verhindert, dass die Kreaturen auf ihre Zielfelder gelangen. Sobald eine Kreatur ihr zweites Zielfeld erreicht hat, kommt sie auf eines der beiden „Goldenen Schilde“. Es dürfen nur zwei verschiedene Kreaturen auf die Schilde.
Sobald eine dritte Kreatur, oder eine zweite Kreatur der gleichen Art ihr zweites Zielfeld erreicht, ist die Legende verloren.




\newpage
\section{Die Odyssee (2021)}

\begin{center}
    Fan-Legende aus der Taverne

    \url{https://legenden-von-andor.de/forum/viewtopic.php?f=5&t=6037}
\end{center}


\bildmitts{Die Odyssee (2021).jpg}

\textbf{Fan-Legenden-Erfinder:} Vakmar \& Vakur
\textbf{Audio-Dateien: }Cassia u. Vakur

\textbf{Kurzbeschreibung:} Die Helden erleben eine Irrfahrt im Hadrischen Meer.

\textbf{Informationen:} Die Legende ist an Homers Odyssee angelehnt; ansonsten ist sie frei erfunden und spielt nicht vor, zwischen, während oder nach den offiziellen Legenden.
\textit{Hinweis:} Die Spielvariante „Grenolin“ und die Heldenfigur Stiana können in dieser Legende nicht genutzt werden.

\textbf{Erfahrung:} L10; Im Dienste des Feuers (Die Handhabung der Magischen Waffen sollte bekannt sein.)

\textbf{Material:} Die Legende spielt auf der Vorderseite der Erweiterung „Die Reise in den Norden“ (Nebelinseln).
Neben dem Grundspiel wird noch die Erweiterung „Die Reise in den Norden“ benötigt.
Auf der letzten Seite gibt es einen „Bastelbogen“. Dieser enthält Material, das ebenfalls benötigt wird; alternativ kann aber auch anderes Spielmaterial verwendet werden.
Die Seeschlange ist der sehr zu empfehlenden Fan-Legende „Das Juwel der Mächte“ entnommen.

\textbf{Schwierigkeitsgrad:} je nach Heldenwahl und -anzahl leicht bis schwer

\textbf{Heldenanzahl:} Optimal vier (ohne Stinner), mit zwei Helden ist sie etwas schwerer, mit drei Helden etwas leichter




\newpage
\section{Die Feinde von Andor (2017 bis 2018)}


\begin{center}
    Fan-Legenden aus der Taverne

    "Schattenjagd"

    \url{https://legenden-von-andor.de/forum/viewtopic.php?f=19&t=3436}
\end{center}


\bildmitts{Moais Feinde von Andor (2017 bis 2018).jpg}

\textbf{Fan-Legenden-Erfinder:} Moai

\textbf{Inhaltsangabe:}
Während eines abendlichen Tavernenbesuchs erfahren die Helden von einem alten Seefahrer von einer Legende um die Drei Schwestern, drei mächtige Hexen, die vor hunderten von Jahren in Hadria lebten und die mit ihren magischen Kräften viel Gutes für das Land und seine Bewohner taten. Mit der Zeit aber verfielen sie dem Bösen und richteten furchtbares Unheil an. Nur ein Bündnis aus Menschen und Zwergen gelang es, sie zu besiegen und in der Unterwelt einzukerkern. Nun aber kehren sie zurück, nach wie vor von einem unbändigen Hass auf alles Gute beseelt, und sie suchen nach drei geheimnisvollen Gegenständen, an die sie einst ihre Kraft gebunden hatten. Sollte es ihnen gelingen, dieser Gegenstände habhaft zu werden und sich wieder zu vereinen, droht Andor größeres Unheil, als sich irgendjemand vorstellen könnte…

\textbf{Schwierigkeitsgrad:} mittel bis schwer

\textbf{Spieleranzahl:} 2 bis 4

\textbf{Info:}
Diese Legende ist der 1.Teil einer Kampagne um die Feinde von Andor. Das Hauptaugenmerk liegt dabei auf den Drei Schwestern. Sie und weitere „Feinde“ werden in mehreren Legenden auftauchen, die wechselweise auf verschiedenen Spielplänen vom Grundspiel bis zur „Letzten Hoffnung“ spielen werden. In den Legenden wird es auch möglich sein, die „Neuen Helden“ einzusetzen, auch wenn sie bislang nicht für Teil III vorgesehen sind. Und auch der Held Kram/Bait kann im „Norden“ eingesetzt werden. (Kleine Vorabinfo: Die „Neuen Helden“ erhalten in Teil III zusätzliche Fähigkeiten)
Diese Legende benötigt Material aus dem Grundspiel und dem Sternenschild.




\begin{center}
    "Schwarze Blitze"

    \url{https://legenden-von-andor.de/forum/viewtopic.php?f=5&t=3464}
\end{center}

\textbf{Fan-Legenden-Erfinder:} Moai

\textbf{Inhaltsangabe:}
Nach einer langen und relativ ereignisarmen Seereise erreichen die Helden endlich die Insel Hadria. Doch ein plötzlich auftretendes Unwetter zwingt sie, das Schiff zu verlassen und an der Küste der Insel zu stranden. In Nordgard angekommen erfahren sie von den Zauberern des Feuers, was alles vor ihrer Ankunft in Hadria geschehen ist und was sie tun müssen, um den Norden von einer schrecklichen Gefahr zu befreien.

\textbf{Schwierigkeitsgrad:} mittel bis schwer

\textbf{Spieleranzahl:} 2-4

\textbf{Info:}
Diese Legende ist der 2.Teil der Kampagne um die „Feinde von Andor“, die in mehreren Legenden wechselweise auf verschiedenen Spielplänen vom Grundspiel bis zur „Letzten Hoffnung“ spielen wird. In den Legenden wird es auch möglich sein, die „Neuen Helden“ einzusetzen, auch wenn sie bislang nicht für Teil III vorgesehen sind. Und auch der Held Kram/Bait kann im „Norden“ eingesetzt werden. (Kleine Vorabinfo: Die „Neuen Helden“ erhalten in Teil III zusätzliche Fähigkeiten). Diese Legende spielt auf der Insel Hadria.
\textit{Benötigtes Spielmaterial:} Grundspiel, Neue Helden und Die Reise in den Norden






\begin{center}
    "Dunkle Schwingen"

    \url{https://legenden-von-andor.de/forum/viewtopic.php?f=5&t=3506}
\end{center}


\textbf{Fan-Legenden-Erfinder:} Moai

\textbf{Inhaltsangabe:}
Nach dem Sieg über die zweite der Drei Schwestern, verließen die Helden Hadria und begaben sich auf die Rückreise nach Andor. Da sie erfahren hatten, dass sich die letzte Schwester im Grauen Gebirge aufhalten sollte, wollten sie das Rietland nur schnell durchqueren, um ihr Ziel so bald wie möglich zu erreichen. Unterwegs erhielten sie Unterstützung von den Bewohnern Andors, die ihnen dabei halfen, ihre Fähigkeiten zu verbessern.
Im Grauen Gebirge angekommen mussten die Helden jedoch feststellen, dass die Bewohner der Berge einer Gefahr ausgesetzt waren, von der sie vorher noch nichts geahnt hatten …

\textbf{Schwierigkeitsgrad:} mittel bis schwer

\textbf{Spieleranzahl:} 2-4

\textbf{Info:}
Diese Legende ist der 3. Teil der Kampagne um die "Feinde von Andor" und spielt im Grauen Gebirge. In dieser Legende ist es auch möglich, die "Neuen Helden" zu verwenden, da sie Karten mit speziellen Sonderfähigkeiten erhalten können.
Mit den "Dunklen Helden" wurde die Legende noch nicht getestet.
Benötigtes Spielmaterial: Die Letzte Hoffnung.








\begin{center}
    "Totenfeuer"

    \url{https://legenden-von-andor.de/forum/viewtopic.php?f=5&t=3550}
\end{center}

\textbf{Fan-Legenden-Erfinder:} Moai

\textbf{Inhaltsangabe:}
Endlich! Nach einer langen Reise haben die Helden Krahd erreicht. Aber bereits kurz nach ihrer Ankunft müssen sie feststellen, dass die Dunkle Schwester nicht die einzige Gefahr ist, der sich im Land der Sklavenschinder stellen müssen.
Und selbst wenn sie den Kampf gegen die dritte Schwester gewinnen, wird das wirklich das Ende ihres Abenteuers sein?

\textbf{Schwierigkeitsgrad:} schwer

\textbf{Spieleranzahl:} 2-4

\textbf{Info:}
Diese Legende ist der 4.Teil der Kampagne um die „Feinde von Andor“ und spielt in Krahd. Weitere Teile sind in Vorbereitung. Spieler, welche die „Neuen Helden“ verwenden wollen, benötigen Karten aus Teil 3 (Dunkle Schwingen), auf denen Sonderfähigkeiten für „Die Letzte Hoffnung“ beschrieben wurden. Mit den Dunklen Helden ist die Legende noch nicht getestet worden.
Benötigtes Spielmaterial: Die letzte Hoffnung, Der Sternenschild und Die Reise in den Norden.












\begin{center}
    "Finstere Seelen"

    \url{https://legenden-von-andor.de/forum/viewtopic.php?f=5&t=3784}
\end{center}

\textbf{Fan-Legenden-Erfinder:} Moai

\textbf{Inhaltsangabe:}
Nach dem Sieg über die Dritte der drei Schwestern atmeten die Helden erleichtert auf. Die Bedrohung durch die Hexen war abgewendet worden und nun sehnten sich alle nach Ruhe und Frieden. Doch eine plötzlich eingetroffene Nachricht aus der Mine der Schildzwerge versprach weiteres Unheil. Neue finstere Wesen waren aufgetaucht und wieder drohte Andor der Untergang.

\textbf{Schwierigkeitsgrad:} leicht bis schwer

\textbf{Spieleranzahl:} 2-4

\textbf{Info:}
Diese Legende ist der 5. Teil der Kampagne um die "Feinde von Andor" und spielt in der Mine. In dieser Legende wird wieder die Heldentafel aus dem Grundspiel benötigt. Außerdem wird die Figur "Der Fluch" gebraucht. Diese habe ich zum Ausdrucken beigefügt. Spieler, die im Besitz von "Chada \& Thorn" sind, können natürlich auch diese "Original-Figur" benutzen. Alternativ kann auch die Figur "Dunkler Magier" verwendet werden.
\textit{Benötigtes Material:} Grundspiel, Die Reise in den Norden, Der Sternenschild und Die Letzte Hoffnung.









\begin{center}
    "Schrecken der Vergangenheit"

    \url{https://legenden-von-andor.de/forum/viewtopic.php?f=5&t=3849}
\end{center}

\textbf{Fan-Legenden-Erfinder:} Moai

\textbf{Inhaltsangabe:}
Nach Hellas Enthüllungen bezüglich des bevorstehenden Angriffs des Eisernen Königs waren die Helden zunächst sprachlos und wollten kaum glauben, dass all ihre Mühen, die darauf ausgelegt waren, dem Land endlich Frieden zu bringen, so ins Gegenteil gekehrt worden waren. Bevor sie sich aber dann bereit machten, den Kampf aufzunehmen, wollten sie von den Bewahrern zunächst jedoch mehr über den Eisernen König und seine Herrschaft erfahren.
Was sie erzählt bekamen, war eine Geschichte voller Angst und Zerstörung. Und eine Geschichte über eine Gruppe außergewöhnlicher Helden …

\textbf{Schwierigkeitsgrad:} mittel bis schwer

\textbf{Spieleranzahl:} 2-4

\textbf{Info:}
Diese Legende ist der 6. Teil der Kampagne um die Feinde von Andor und spielt im Rietland. Sie ist speziell für die Dunklen Helden geschrieben worden, da sie in einer entfernten Vergangenheit spielt. Auf der letzten Seite findet ihr die Figur Narraven, die für die Legende benötigt wird. Alternativ kann aber auch die Figur der Hexe verwendet werden.
Außerdem habe ich für diese Legende neue Ereigniskarten für die Dunklen Helden geschaffen. Diese können allerdings auch unabhängig von der Kampagne in einer anderen Legende benutzt werden.
\textit{Benötigtes Material:} Grundspiel, Der Sternenschild und Dunkle Helden.









\begin{center}
    "Die Rückkehr des Eisernen Königs"

    \url{https://legenden-von-andor.de/forum/viewtopic.php?f=5&t=4005}
\end{center}


\textbf{Fan-Legenden-Erfinder:} Moai

\textbf{Inhaltsangabe:}
Gewarnt vor der Ankunft des Eisernen Königs und dessen herannahender Flotte reisen die Helden zu den Nebelinseln, um sich gemeinsam mit den Inselbewohnern auf ihren hoffentlich letzten Kampf vorzubereiten.

\textbf{Schwierigkeitsgrad:} mittel bis schwer

\textbf{Spieleranzahl:} 2-4

\textbf{Info:}
Diese Legende ist der abschließende Teil der Kampagne um die "Feinde von Andor" und spielt auf den Nebelinseln.
Zu dieser Legende gehören insgesamt sieben Schiffsfiguren, die auf der zusätzlichen Datei zu finden sind.
\textit{Benötigtes Material:} Grundspiel, Die Reise in den Norden, Der Sternenschild und die Andor-Bonus-Box.









\newpage
\section{Andors Beginn (2022)}


\begin{center}
    Fan-Chroniken aus der Taverne

    \url{https://legenden-von-andor.de/forum/viewtopic.php?f=5&t=7645}
\end{center}

\bildmitts{Andors Beginn (2022).jpg}


\textbf{Fan-Legenden-Erfinder:} TroII

\textbf{Inhaltsangabe:} Eine kleine Schar von Unfreien flieht vor grausamen Herrschern in ein leeres Land. Auf dem Weg dorthin müssen sie das Graue Gebirge passieren und werden von einem Drachen beinahe vernichtet.

\textbf{Schwierigkeitsgrad:} mittel bis schwer

\textbf{Spieleranzahl:} 3 bis 6 (bis auf Teil 2.5, dort 1)

\textbf{Spielpläne:}
Teil 1: Krahd
Teil 2: Graues Gebirge
Teil 2.5: Graues Gebirge
Teil 3: Andor

\textbf{Benötigt Material:} Die letzte Hoffnung und Grundspiel. (Für die Teile 1 und 2.5 wird das Grundspiel nicht benötigt.)
Wer die „Düsteren Zeiten“ besitzt, kann an manchen Stellen optional z.B. die Figur Harthalt nutzen.

Teil 2.5, „Harthalts Marsch“, ist eine Solo-Chronik und für nur einen Helden vorgesehen. (Natürlich können aber auch mehrere Spieler diesen einen Helden steuern.) Sie beleuchtet den Weg von Harthalt nach dessen Trennung von Brandur. Grundsätzlich können Teil 2.5 und die Haupt-Trilogie vollkommen unabhängig voneinander gespielt werden, sie sind auch jeweils für sich verständlich.


\begin{center}
    Beiträge aus der Taverne

    "Andors Chroniken"
\end{center}

\textit{Im „Lied des Königs“ wird berichtet, wie sich eine kleine Gruppe von Helden zusammenfand, um Andor vor dem Untergang zu retten. Und (unter anderem) diese Gruppe von Helden sollte in den folgenden Jahren noch Unglaubliches vollbringen. Sie töteten den letzten Drachen Tarok, eine der Mächte des Meeres und den Dunklen Magier Varkur. Sie errangen mächtige Schilde und Magische Waffen, sie brachen Flüche und die Ketten der Krahder – ihre großen Heldentaten sind bekannt, ebenso wie die unvorstellbaren Übel, die sie zu überwinden hatten. Über 200 (Fan-)Legenden befassen sich mit jenen ereignisreichen 17 Jahren, die sie seit dem „Lied des Königs“ durchlebten.}

\bildmitts{Andors Chroniken 1.jpg}

\textit{Doch im Baum der Lieder gibt es noch andere Schriften. Lieder aus anderen Epochen, von den Helden längst vergangener Zeiten, und den Gefahren, denen sie trotzten. Berichte von Trollkriegen, von entflohenen Sklaven auf ihrem Marsch durch ein einsames Gebirge, vom Untergang des großen Seereichs Varatanien, von einem Fluss voller Blut und einem unterirdischen Krieg zwischen Zwergen und Drachen. Berichte von heroischen Siegen und fatalen Niederlagen. Berichte von Freundschaft, Liebe und Tod…}

\bildmitts{Andors Chroniken 2.jpg}

\textit{Diese Ereignisse sind in Ansätzen bekannt, und finden doch – so schien es mir – höchstens beiläufige Erwähnung. Wenngleich auch vieles in Vergessenheit geraten ist, so sollten die bekannten Fragmente (ergänzt durch Vermutungen, Hoffnungen und – sagen wir es, wie es ist – Erfindungen) doch ausreichen für einen Einblick in diese Geschehnisse. Also habe ich begonnen, ihnen den Platz neben den mehr oder weniger bekannten Taten der Helden zu verschaffen, den sie verdienen. Ich habe begonnen, die alten Schriften zu sammeln, und ich gab meiner noch längst nicht vollendeten Sammlung den Namen:} „\textit{\textbf{Andors Chroniken}}“

\bildmitts{Andors Chroniken 3.jpg}

\textit{Nach Jahren, die sie teilweise in verstaubten Schubladen verbracht haben, lasse ich sie nun endlich in die Freiheit:}
\textit{Andors Chroniken - Legenden die lange vor der Zeit der Helden spielen. Schlüpft selbst in die Rollen von Brandur, Reka, Harthalt, Orweyn, Varatan und anderen. Jetzt in dieser Taverne: Der Drei(einhalb-)teiler „Andors Beginn“, der von Brandurs Flucht handelt.}

\textit{Und dabei wird es nicht bleiben...}






\newpage
\section{Der ewige Rat (2019 bis 2021)}

\begin{center}
    Fan-Geschichte aus der Taverne

    \url{https://legenden-von-andor.de/forum/viewtopic.php?f=11&t=6885}
\end{center}

\bildmitts{Der ewige Rat.jpg}

Hallo liebe Andori,\bigskip

schon vor längerer Zeit hatte ich die Idee für die Handlung einer sehr storylastigen Legende, die nach Teil III angesiedelt war.

Die Geschichte wurde weiter verfeinert, bald hatte ich die Absicht, sie über alle drei Spielpläne gehen zu lassen. Nach einer Weile waren dann stattdessen drei Legenden über je einen Spielpan in Planung. Der Umfang der Geschichte wuchs, die Ideen sprudelten und es entstanden seitenlange Storytexte, die am Anfang und Ende jeder der drei Legenden stehen sollten. Und irgendwann wurde mir klar, dass es mir mehr und mehr um die Geschichte ging und weniger darum, eine (oder vielmehr drei) zu spielende Legende(n) zu entwickeln. :roll:

Als sich diese Erkenntnis erst mal durchgesetzt hatte, wurde umdisponiert. Wer Andor spielen möchte, braucht keine seitenlangen Texte, wer dagegen gerne eine Geschichte hört, muss sie nicht in Legendenform verpackt bekommen. Ideen wurden verworfen, Pläne fallengelassen und stattdessen nahm ich mir vor, das zu tun, was längst der eigentliche Zweck geworden war: Die Geschichte zu erzählen. :D\bigskip

Ich bin nun wirklich nicht der erste, der auf diese Idee kam. Man denke nur an Tost, die alte Bait-Geschichte von ChrisW und natürlich die wirklich brillante Story von Bird, die ich nur jedem wärmstens ans Herz legen kann und auf deren Fortsetzung ich mal wieder sehnsüchtig warte. ;)

Ich will keinesfalls irgendeine Form von Wettbewerb, sondern nur eine Ergänzung schaffen. Wer möchte, kann die Geschichte lesen, und wenn sie keiner liest, schreibe ich eben um des Schreibens willen. 8-)\bigskip

So, dann noch ein paar einleitende Worte: Ich werde mich bemühen, dem offiziellen Kanon nicht zu widersprechen. Das bezieht sich nur auf die Erweiterungen, die bis jetzt (Stand Januar 2019) erschienen sind, spätere Neuerscheinungen werden nicht mehr zwingend berücksichtigt. Dass ich gerade jetzt anfange liegt tatsächlich an der App, die ich noch abwarten wollte, um Widersprüche zu vermeiden. Die Zeit habe ich genutzt, um noch ein bisschen zu planen und schon etwas vorzuschreiben.

Falls jemandem ein Widerspruch zur offiziellen Geschichte auffällt, freue ich mich darüber, wenn er genannt wird. Wo es mehrere Möglichkeiten gibt (z.B. variable Endgegner) wähle ich u.U. aber die aus, die mir am besten passen. Und dass ich dem Kanon nicht widerspreche bedeutet natürlich noch lange nicht, dass ich nichts dazuerfinde. ;)

Die Geschichte spielt aus gegebenem Anlass (s.o.) NACH Teil III. Ich setze dennoch kein Andorwissen voraus, muss aber schon mal eine \textbf{DICKE FETTE SPOILERWARNUNG} loswerden! Manche Enthüllungen lassen sich einfach nicht vermeiden, auch wenn ich versuchen werde, mich in dieser Hinsicht auf das Notwendigste zu beschränken. :)\bigskip

Ich freue mich, wenn ich hiermit auch anderen eine Freude machen kann, ansonsten hinterlasse ich eben meinen Müll für eventuelle spätere Leser.





\newpage
\section{Das Kochbuch der Andori (2018)}

\begin{center}
    Fan-Dokument aus der Taverne

    \url{https://legenden-von-andor.de/forum/viewtopic.php?f=11&t=1628&start=110}
\end{center}

\bildmitts{Das Kochbuch der Andori (2018).jpg}

Hallo liebe Andori,

welch feine Düfte locken in die Taverne von Andor? Nun, fleißige Fans haben ein andorisches Kochbuch zusammengetragen.\bigskip

Damit zaubert ihr die die perfekte kulinarische Unterstützung für ein großes Abenteuer im Lande Andor auf den Tisch!





\newpage
\section{3D mit der Heldenschmiede (ab 2020)}

\begin{center}
    3D-Heldenfiguren aus der Taverne

    \url{https://legenden-von-andor.de/forum/viewtopic.php?f=11&t=5725}
\end{center}

\begin{figure}[ht!]
    \centering
    \includegraphics[width=0.3\linewidth]{Das Erbe des Wunderkindes/Bilder/3D mit der Heldenschmiede 1.jpg}
    \includegraphics[width=0.3\linewidth]{Das Erbe des Wunderkindes/Bilder/3D mit der Heldenschmiede 2.jpg}\\
    \includegraphics[width=0.3\linewidth]{Das Erbe des Wunderkindes/Bilder/3D mit der Heldenschmiede 3.jpg}
    \includegraphics[width=0.3\linewidth]{Das Erbe des Wunderkindes/Bilder/3D mit der Heldenschmiede 4.jpg}
\end{figure}

Hallo liebe Andori,\bigskip

immer mal wieder kam das Thema "3D-Miniaturen für Andor" auf. Solche Figuren hätten sicherlich ihren Reiz, wobei ich die schönen gemalten Charaktere eigentlich doch lieber habe zum Spielen. Nichtsdestotrotz habe ich mal die Heldenschmiede (https://www.heroforge.com/) angeworfen und ein paar Beispielcharaktere erstellt. Vieles hat natürlich nicht perfekt funktioniert. Ich hab versucht, das beste rauszuholen :D

Drucken lassen werde ich mir keine, es hat mir einfach Spaß gemacht, sie zu designen und ich dachte, ich teile die Bilder mit euch :D

Ich finde die Seite auch ganz praktisch, um für etwaige Bilder auf Fan-Legenden Bild-Modelle von verschiedenen Charakteren zu erstellen, die man dann evtl. mit einem Bildbearbeitungsprogramm noch verfeinern kann.\bigskip

LG

Giftknödel


}



\newpage

\hintergrund{Andor Fan-Karte Inkarnate v2.0 textlos A4.jpg}

\thispagestyle{empty}%keine Header oder Footer auf dieser Seite

\textbf{ }

\end{document}