% !TeX program = lualatex

\documentclass[10pt, a4paper, oneside]{book}

\usepackage[left=1.5cm,right=1.5cm, top=1cm,bottom=1cm,includeheadfoot]{geometry} %definiere Ränder und so

\usepackage[ngerman]{babel} % für deutsche Titelbezeichnungen

\usepackage[autocite=superscript,  backend=biber,sortcites=true, style=numeric,]{biblatex}
    \addbibresource{Alle Helden ueberall/quellen.bib}
    \DeclareFieldFormat{url}{\href{#1}{\nolinkurl{#1}}} % Damit nicht vor jeder cite-url "URL:" steht.

\usepackage{comment}

\usepackage{contour}
    \contourlength{1.2px}

\usepackage{csquotes} % für Anführungszeichen
    \MakeOuterQuote{"}

\usepackage{environ}
    \usepackage[most]{tcolorbox} % für farbige Boxen
    \NewEnviron{chapterbox} %box für chapter-Anfänge
    {   
        \newpage
        \begin{tcolorbox}[arc=0pt, colback=white, boxrule=0pt]
            \BODY
            \vfill
        \end{tcolorbox}
    }

\usepackage{extramarks} %für \lastrightmark und xmark's

\usepackage{fancyhdr} %für fancy Headers und Footers
    \pagestyle{fancy}
    \fancyhf{}%lösche alle Header-Footer-default-Einstellungen
    \fancyfoot[C]{\thepage} %Seitenzahl im Footer, zentriert 
    \fancyhead[L]{}
    \fancyhead[C]{\nouppercase{\lastleftmark}}
    \fancyhead[R]{}

\usepackage{fontspec}
    \newfontfamily{\andorfont}{Andor-Schriftart}[Extension=.otf]

\usepackage{eso-pic} % unter anderem für die Hintergrundbilder

\usepackage{graphicx} % für Bildereinfügen

\usepackage{hyperref}
    \hypersetup{
        colorlinks=true,
        linkcolor=blue,
        filecolor=blue,      
        urlcolor=blue,
        citecolor=blue,
        pdftitle={Alle Helden ueberall},
        pdfpagemode=FullScreen,
    }

\usepackage{multicol}

\usepackage{sectsty} %section stiling
    \partfont{\fontsize{50}{60}\andorfont}
    \chapterfont{\centering}
    \sectionfont{\centering}
    \renewcommand\thesection{} %damit sections im Inhaltsverzeichnis nicht nummeriert werden
    \renewcommand\thesubsection{} %damit subsections im Inhaltsverzeichnis nicht nummeriert werden

\usepackage[most]{tcolorbox}
    \newtcolorbox[]{bgcolor}[1][color-DLvA]{
        breakable,enhanced,opacityfill=0.9,
        arc=0pt,boxrule=0pt,frame hidden, % no frame
        colback={#1}, % set background color
        left=0.05cm,right=0.05cm,top=0.05cm,bottom=0.05cm, % set positioning of text inside box
    }
    \newtcolorbox[]{nobgcolor}[1][color-DLvA]{
        breakable,enhanced,opacityfill=0,
        arc=0pt,boxrule=0pt,frame hidden, % no frame
        colback={#1}, % set background color
        left=0.05cm,right=0.05cm,top=0.05cm,bottom=0.05cm, % set positioning of text inside box
    }

\usepackage{tikz} % unter anderem für die Hintergrundbilder

\usepackage{titlesec}
    \titlespacing*{\chapter}{0pt}{30pt}{20pt}
    \titleformat{\chapter}{\penalty-100\centering\normalfont\Large\bfseries}{}{0pt}{\huge}
    \titleformat{\section}{\penalty-100\centering\normalfont\large\bfseries}{}{0pt}{\large\vspace{-0.3cm}}
    \titleformat{\subsection}{\vspace{0.3cm}\penalty-100\normalfont\bfseries}{}{0pt}{\vspace{-0.2cm}}

\usepackage{tocloft}
    \setlength{\cftsecindent}{-7pt} % entferne Einrücken im Inhaltsverzeichnis

\usepackage{xcolor}
    \definecolor{color-DLvA}{HTML}{FF8080}
    \definecolor{color-DRidN}{HTML}{40E0D0}
    \definecolor{color-DlH}{HTML}{C0C0C0}
    \definecolor{color-DeK}{HTML}{FFFFFF}
    \definecolor{color-DfL}{HTML}{FFD040}
    \definecolor{color-StSch}{HTML}{FFB040}
    \definecolor{color-BB}{HTML}{F0D8C0}
    \definecolor{color-AG}{HTML}{FFA0E0}
    \definecolor{color-DZ}{HTML}{80E0FF}



\newcommand{\fan}[1]{\textit{#1}}

\newcommand{\hypref}[1]{%
    \hyperref[#1]{#1}%
}

\newcommand{\bildmitts}[2][height=0.5\textwidth,width=0.75\textwidth,keepaspectratio]{%
    \begin{center}
        \includegraphics[#1]{Bilder/#2}
    \end{center}
}

\newcommand{\bildlinks}[2][height=0.32\textwidth,width=0.48\textwidth,keepaspectratio]{%
    \begin{wrapfigure}{L}{0px}
        \includegraphics[#1]{Bilder/#2}
    \end{wrapfigure}
}

\newcommand{\hintergrund}[1]{%
    \AddToShipoutPictureBG{%
        \tikz[remember picture, overlay] \node[opacity=1, inner sep=0pt] at(current page.center){\includegraphics[width=\paperwidth,height=\paperheight]{Alle Helden ueberall/Hintergrund/#1}};%
    }
}

\newcommand{\titelseite}[1]{%
    \AddToShipoutPictureBG*{%
        \tikz[remember picture, overlay] \node[opacity=1, inner sep=0pt] at(current page.center){\includegraphics[width=\paperwidth,height=\paperheight]{Alle Helden ueberall/Hintergrund/#1}};%
    }
}






\setlength{\parindent}{0cm} % damit Zeilen nicht eingerückt werden.






\renewcommand{\footnoterule}{%stretche footnote-Linie über die ganze Länge
  \kern -3pt
  \hrule width \textwidth
  \kern 2pt
}

\renewcommand\thechapter{ } %damit chapters nicht mit Zahlen bezeichnet werden
\renewcommand{\chaptermark}[1]{\markboth{#1}{}} %keine Nummerierung für die chapter-header
\renewcommand{\sectionmark}[1]{\markright{#1}} %keine Nummerierung für die section-header






\title{Alle Helden überall\\Stand 2026}
\author{Fan-Spielvariante von Butterbrotbär,\\basierend auf den \enquote{Die Legenden von Andor}-Spielen von Michael Menzel,\\Christoph Kling, Dorothea Michels, Andreas Kälber und Matthias Miller\\sowie vielen fruchtvollen Diskussionen der fantastisch fantasievollen Fans\\Albus, Bennie, Bewahrer Melkart, Cardiodrone, Dagain, Fenn-Fan, Galaphil,\\ Kar éVarin, Legolas Grünblatt, namenloser Wanderer, Phoenixpower,\\ Qurunatobra213, Ragnar, Steppenechse111777, Towa,\\und insbesondere Mivo, Orpheus und TroII.}
\date{\vfill Dieses Dokument wurde erstellt mit \LaTeX{}.\\
Für den Quelltext siehe \url{https://butterbrotbaer.github.io}.\\Für die neuste Version siehe \url{https://legenden-von-andor.de/forum/viewtopic.php?f=11&t=11151}.}






\begin{document}

\hintergrund{Hintergrund.jpg}

\begin{titlepage}
    \titelseite{Helden Cover.jpg}
    \centering

    \textbf{ }

    \vspace{400pt}

    \fontsize{108}{150}
    \textcolor{white}{\andorfont \contour{black}{Alle Helden}}
    
    \fontsize{108}{150}
    \textcolor{white}{\andorfont \contour{black}{überall}}

    \fontsize{54}{75}
    \textcolor{white}{\andorfont \contour{black}{Stand 2026}}
\end{titlepage}

\thispagestyle{plain}


\maketitle

\thispagestyle{plain}


\setcounter{page}{3} %lasse bereits das Titelbild als Seite 1 zählen






\begin{multicols*}{2}\raggedcolumns



\textbf{Liebe Andori}
\bigskip

\textbf{Dieses Dokument dient zweierlei Zweck:} 

Zum einen versammelt es alle offiziellen Regeln zum Spielen der verschiedenen offiziellen Helden in den Legenden aus den verschiedenen offiziellen Andor-Produkten und Internetauftritten.

Zum anderen ergänzt es diese gesammelten Regeln um Vorschläge für Hausregeln, mit denen endlich alle offiziellen Helden in (fast) allen offiziellen Legenden gespielt werden können.

Zur Unterscheidung sind offizielle Regeln jeweils gerade geschrieben, während Fan-Hausregeln \fan{kursiv} sind.\bigskip


\textbf{Dieses Dokument ist ein Work in Progress.} Ich werde Schreibfehler und Regelfehler begangen haben, suboptimale Formulierungen gewählt und Inkonsistenzen übersehen. Bitte, werdet aktiv! Gebt Rückmeldungen, idealerweise unter \url{https://legenden-von-andor.de/forum/viewtopic.php?f=8&t=11151}, und verfasst eure eigenen Versionen dieses Dokuments, mit euren eigenen liebsten Hausregeln.
\bigskip

\textbf{Hausregeln sind in Andor stets willkommen!} Diese Goldene Regel sei hier mit Zitaten untermauert:

\bigskip

"Wenn es nach mir geht, soll jeder Andor so spielen können wie er es gerne mag. Mit 'Leichter Spielen'-PDF oder ohne. Mit einem sehr starken selbst ausgedachten 'Fremden' oder ohne. Mit Hausregeln oder ohne. Mit Arbon oder ohne." 

– Michael Menzel
\supercite{6390}

\bigskip

"Manchmal ist es nötig, Entscheidungen zu treffen, die gegen die Regeln verstoßen, denn diese sind nicht immer sinnvoll."

– König Brandur zu \hyperref[Bogenschützin]{Chada}
\supercite{LdK:Der_König_von_Andor}

\bigskip

"Ich persönlich bin [...] ein großer Freund von Hausregeln und Anpassungen und wenn das Spiel für euch so gut ist, dann ist das doch prima." 

– Michael Menzel
\supercite{32174}

\bigskip

"Am besten richtet man sich wohl nach dem, was Bennie als Michael Menzels
'Goldene Regel' bezeichnet hat. Das passt immer." 

– Ragnar
\supercite{47207}

\bigskip

"Michas Goldene Regel: Macht es so, wie es euch am meisten Spaß macht, und wenn es dazu Hausregeln braucht, ist es für euch vielleicht genau so richtig!" 

– Kar éVarin
\supercite{97401}

\bigskip

"Im Zweifelsfall gewinnt schließlich immer Michas Goldene Regel: Spielt so, wie es euch am meisten Spaß macht." 

– TroII
\supercite{118167}

\bigskip

\textbf{Und nun springt zu eurer Lieblingsheld*in und spielt sie in eurer Lieblingslegende.}\bigskip

Liebe Grüße

Butterbrotbär

\vspace*{\fill}
\columnbreak




\textit{\fan{Stinner stand stolz im Schatten von Klippenwacht, der schönen Seefeste, deren Wiederaufbau nach über einem Jahrhundert als Ruinenstadt endlich in vollem Gange war. Das Hadrische Meer wogte wohlig warm um seine Hüfte. Sein treuer Streifenmarder schmiegte sich fiepsend um seinen Hals. Seine Hand lag in der seiner Geliebten. Die Nixe, deren Oberkörper vor ihm aus dem Meer erhob, blubberte ihm Worte der Freude zu, die Stinner inzwischen genau verstehen, wenn auch nicht nachsprechen konnte. Ihr Schuppenbauch zeigte bereits eine Wölbung. Alles war perfekt. Nichts könnte ihren gemeinsamen Moment der Ruhe stören.}}

\textit{\fan{Ein Zischen ertönte hinter ihnen. Blauer Dampf wallte hervor, waberte, wogte, wirbelte und formte ein rotierendes Portal. Heraus stolperte eine grünhäutige Frau mit kahl rasiertem Kopf, ein tickendes silbriges Konstrukt aus ineinander verschlungenen Streben in ihrer linken Hand.}}

\textit{\fan{"Du?! Dich habe ich schon einmal getroffen!", rief Stinner, "Damals bliebst du mir eine Antwort schuldig, doch nun sprich: Wer bist du, die du durch die Zeit reist und die Gefüge dieser Welt zu lenken ersuchst?"}}

\textit{\fan{Die Fremde blieb Stinner erneut eine Antwort schuldig, als sie zunächst nach Atem rang und dann dramatisch sprach: "Stinner, der Seekrieger aus Werftheim? Kommt mit mir! Wir schützen gemeinsam das Land Andor und erleben fantastische Abenteuer!"}}

\textit{\fan{"Dem Königreich Andor geht es gut", blubberte Stinners Geliebte zischend. "Königin Chada macht ihre Sache besser, als es selbst mein Schatz könnte. Du bist hier nicht erwünscht." }}

\textit{\fan{"Nicht das jetzige Andor", meinte die Fremde kopfschüttelnd. "Sondern das aus der Vergangenheit. Eine Ewige Kälte lag darüber, und Stinner wird gebraucht, diese Ewige Kälte zu beenden! Dachtet Ihr etwa, Stinner hätte damals keine Rolle gespielt? Ich bin hier, um ihn dorthin zurückzubringen, damit er sein Schicksal erfüllen kann. Danach mag er wieder hierher zurückkehren und Varatans Erbe antreten."}}

\textit{\fan{Stinner trat vor das bläulich schimmernde Zeitportal. Durch die milchige Oberfläche erkannte er Personen, die sich um ein von Schneewirbeln umspieltes Lagerfeuer versammelt hatten. Ein Feuerkrieger mit einem glühenden Lavastein in seiner Rüstung zeigte sein bewegliches Flammenschwert einer blau gewandten andorischen Kriegerin auf einem weißen Pferd. Vor ihnen saßen im Schneidersitz zwei Gestalten, die Schwerter-Geröll-Pergament spielten – eine jüngere Frau in einem braunen Zauberergewand, und eine ältere, die ihre blonden Haare offen trug und sich einen dunklen Magiermantel hüllte. Beide sahen Stinners Freundin Eara zum Verwecheln ähnlich.}}

\textit{\fan{"Wer sind sie?", fragte Stinner die Fremde argwöhnisch.}}

\textit{\fan{"Das sind Helden von Andor", antwortete diese schlicht, "Aus verschiedenen Zeiten, aus fernen Ländern, manche mehrfach vertreten, manche anders, als Ihr sie kanntet ... so vieles, von dem Ihr dachtet, es sei unmöglich, ist nur einen Schritt entfernt. Kommt, sie warten nur noch auf Euch!"}}

\textit{\fan{Stinner warf einen Blick zurück zu seiner Geliebten, dann wieder zu den Helden. Er atmete tief durch. Eine Entscheidung stand an.}}



\newpage


\setcounter{tocdepth}{0}
\tableofcontents






\end{multicols*}


\chapter{Grundlagen}
\label{Grundlagen}


\begin{multicols*}{2}\raggedcolumns




\begin{nobgcolor}[color-DLvA]

\section{Allgemein}





\subsection{Heldenauswahl:}

Ihr dürft \fan{(fast) alle} Legenden mit 2 bis 6 Helden spielen. Im Spiel mit 5 oder 6 Helden gelten die \hyperref[5 oder 6 Helden]{Regeln fürs Spiel mit 5 oder 6 Helden}.

\fan{Ihr dürft mit (fast) beliebig vielen Spielern spielen. Ein Spieler darf mehrere Helden spielen. Mehrere Spieler dürfen gemeinsam einen Helden spielen. Sprechen Regeln von "Spieleranzahl" oder "pro Spieler", ist "Heldenanzahl" oder "pro Held" gemeint.}

\fan{Ihr dürft (fast) alle Legenden mit beliebigen Helden spielen. Ihr dürft Helden auf Spielplänen spielen, für die sie ursprünglich nicht gedacht waren. Ihr dürft Helden kombinieren, die denselben Rang oder dieselbe offizielle Farbe haben. Ihr dürft mehrere gleiche Helden spielen, wenn ihr deren Spielmaterial mehrfach besitzt.}



Es gibt einzelne Ausnahmen:

\bullet{} Legende 3, "Die Tage des Widerstands", darf mit der Solo-Variante auch mit 1 Held gespielt werden.

\bullet{} Die Solo-Legenden "Der Vorbote des Feuers" und "Die Jagd" können nur mit 1 Held gespielt werden, und nur mit dem \hyperref[Wolfskrieger]{Wolfskrieger}.

\bullet{} Die Mini-Legende "Die Rückkehr" kann nur mit 2 Helden gespielt werden, und nur mit einer \hyperref[Bogenschützin]{Bogenschützin} und einem \hyperref[Krieger]{Krieger}.

\bullet{} Die Bonus-Legende "Im Bann von Choranat" kann nur mit 2 bis 4 verschiedenen Helden gespielt werden, und nur mit dem \hyperref[Eis-Dämon]{Eis-Dämon}, der \hyperref[Runenmeisterin]{Runenmeisterin}, dem \hyperref[Steppennomade]{Steppennomaden} oder der \hyperref[Takuri-Hüterin]{Takuri-Hüterin}.

\bullet{} Spielt ihr Legende 15, "Der vergiftete Geist", mit 4 bis 6 Helden, kann ein Spieler nicht mehrere Helden spielen, da es versteckte Informationen gibt. \fan{Es dürfen aber immer noch mehrere Spieler gemeinsam einen Helden spielen.}



\subsection{\fan{Gleiche Ränge:}}

\fan{Spielt ihr mehrere Helden mit gleichem Rang, bestimmt für alle Helden mit selbem Rang eine Reihenfolge, welche eigentlich gleich großen Ränge als höher und welche als tiefer gelten sollen. Diese Reihenfolge gilt dann für die gesamte Legende (z.B. für die Startheldbestimmung).}


\subsection{\fan{Gleiche Farben:}}

\fan{Ihr dürft Helden Standfüße, Stärkesteine, Willenssteine und Heldenwürfel in beliebigen Farben geben, egal was ihre offiziellen Farben sind. Ein Held darf Spielmaterial in verschiedenen Farben nutzen. Verschiedene Helden dürfen Spielmaterial derselben Farbe nutzen.} 

\fan{Helden mit besonderen Würfeln (\hyperref[Eis-Dämon]{Eis-Dämon},  \hyperref[Runenmeisterin]{Runenmeisterin} und \hyperref[Steppennomade]{Steppennomade}) müssen ihre besonderen Würfel nutzen. Andere Helden können diese besonderen Würfel nicht nutzen}

\fan{Alle Zeitsteine müssen verschiedene Farben haben (außer ihr spielt mit 4 neutralen Zeitsteinen). In Legenden mit Heldenwappen müssen alle Heldenwappen unterscheidbar sein. Ihr dürft ein Spielmaterial der passenden Farbe zu einer Heldentafel legen, um euch an die Farbe des Zeitsteins oder Heldenwappens dieses Helden zu erinnern.}


\fan{Verwechslungen sind unwahrscheinlicher, wenn ihr für jeden Helden Spielmaterial in passenderen Farben wählt.}




\subsection{Geschlecht ist gleich:}

Regeln gelten für Helden aller Geschlechter, egal welche Geschlechter bei der Regelformulierung genutzt werden.
\supercite{DLvA:Losspiel-Anleitung,NH:Begleitheft,DlH:Losspiel-Anleitung,DH:Begleitheft,MH:Begleitheft,DeK:Losspiel-Anleitung,DfL:Losspiel-Anleitung}










\subsection{Allgemeine Regeln:}

Diese Regeln gelten, außer es steht explizit anders:

\bullet{} Die \hyperref[Ablagen und Leisten]{Regeln für Ablagen und Leisten}.

\bullet{} Die \hyperref[Bonusstärke]{Regeln für Bonusstärke}.

\bullet{} Die \hyperref[Helfer]{Regeln für \fan{Helfer}}.

\bullet{} Gibt es mehrere Möglichkeiten, eine Anweisung zu befolgen (z.B. wenn eine Figur entlang eines kürzesten Wegs laufen muss und es mehrere kürzeste Wege gibt), dürfen die Spieler entscheiden, welche Möglichkeit ausgeführt wird.

\bullet{} Ein Held, der seinen Tag beendet hat, kann keine freiwilligen Handlungen ausführen, bis der Sonnenaufgang abgeschlossen wurde, außer Ereigniskarten oder Sturmkarten abzuwehren.
\supercite{32754,32768}

\bullet{} Ein Held kann eine Handlung \textbf{nicht} ausführen, wenn er mehr Ressourcen \textbf{abgeben} müsste, als er hat, oder wenn er so viele Willenspunkte \textbf{abgeben} müsste, dass er auf 0 Willenspunkte fallen würde. Das gilt auch, wenn steht, dass eine Handlung so viel \textbf{kostet}.

\bullet{} Ein Held darf eine Handlung \textbf{sehr wohl} ausführen, wenn er mehr Ressourcen \textbf{verlieren} würde, als er hat, oder wenn er so viele Willenspunkte \textbf{verlieren} würde, dass er auf 0 Willenspunkte fällt.

\bullet{} \fan{Wirft ein Held im Kampf (nicht im \hyperref[Fernkämpfer]{Fernkampf}) Würfel, muss er einen beliebigen seiner geworfenen Würfel zu seinem Kampfwert addieren. Dies darf auch ein Würfel sein, der nicht den höchsten Wert hat. Mit einem Helm oder Reiterhelm darf er beliebig viele gleiche Würfelwerte addieren. Er muss aber nicht alle gleichen Würfelwerte addieren.}

\bullet{} \textit{Goldmünzen können jederzeit gegen Goldmünzen gleicher Summe aus dem Vorrat ausgetauscht werden. }

\bullet{} Spielmaterial ist wiederkehrend, aber begrenzt. Spielmaterial, das aus dem Spiel geht, geht in den Vorrat und mag später erneut ins Spiel kommen. Sollte Spielmaterial ins Spiel kommen, das es aktuell im Vorrat nicht mehr gibt, geschieht stattdessen nichts.\newline
Ausnahme: Sternchen sind unbegrenzt. Sollen mehr Sternchen ins Spiel kommen, als es im Vorrat gibt, nutzt anderes ungenutztes Material (schlimmstenfalls aus anderen Spielen) als Ersatz.\newline
Ausnahme: In "Die letzte Hoffnung" gehen Höhlenplättchen\supercite{40676} und Sternkraut\supercite{DlH:Ausrüstungstafel} nicht zurück in den Vorrat.\newline
Ausnahme: In "Das ferne Land" gehen Sporne nicht zurück in den Vorrat. Sollten mehr Spornen ins Spiel kommen, als es im Vorrat gibt, kommt stattdessen die Muttersporne ins Spiel.\supercite{DfL:Begleitheft}



\end{nobgcolor}






\begin{bgcolor}[color-DLvA]
\section{Die Legenden von Andor (2012)}





\subsection{Legende 1, Die Ankunft der Helden:}

\fan{Ihr müsst am Anfang nicht bestimmte Helden auf bestimmte Felder stellen. Stellt stattdessen alle Helden auf verschiedene der folgenden Felder: 9, 25, 30, 43, 53, 72.}



\subsection{Legende 2, Die Heilung des Königs:}

Stellt am Anfang alle Helden auf das Feld, das ihrer Rangzahl entspricht.
\supercite{DLvA2:A,NH:Begleitheft,DH:Begleitheft,MH:Begleitheft}



\subsection{Bonus-Legende "Der Kampf um Cavern":}

Es muss ein "Kram" mitspielen.
\supercite{BL_KuC:A}
\fan{Es muss jedoch nicht zwingend ein \hyperref[Zwerg]{Zwerg} mitspielen. Wählt am Anfang einen beliebigen Helden. Dieser (und nur dieser) gilt als Kram, wenn eine Legendenkarte von "Kram" spricht. Ihr dürft zur Erinnerung den Namen dieses Helden mit einem Sternchen markieren.}

\end{bgcolor}








\begin{bgcolor}[color-DlH]
\section{Die letzte Hoffnung (2016)}





\subsection{Heldenkarten:}

\fan{Wählt am Anfang für jeden Helden, ob er Heldenkarten nutzt, und falls ja, welche.}

\fan{Diese Fan-Spielvariante garantiert Funktionalität, wenn jeder Held seine offiziellen oder gar keine Heldenkarten nutzt und bei mehreren gleichen Helden höchstens einer von ihnen die offiziellen Heldenkarten nutzt.}

\fan{Auf eigene Verantwortung dürft ihr mehreren Helden dieselben Heldenkarten geben, Helden Karten anderer Helden nutzen lassen, selbst Heldenkarten erfinden oder Fan-Heldenkarten aus der Taverne wählen:}

\bullet{} \fan{"Neue Helden in Die letzte Hoffnung" (Darhen)}
\supercite{103303}

\bullet{} \fan{"Alle Helden ins Graue Gebirge" (Butterbrotbär)}
\supercite{56931}




\subsection{Zweite Sonderfähigkeit:}

Die zweite Sonderfähigkeit jedes Helden wird im Laufe von Legende 11 aktiviert und gilt in Legenden 12 bis 17 von Anfang an.


\fan{Wer keine Heldenkarten nutzt, hat:}

\bullet{} \fan{Wenn es offizielle Heldenkarten für ihn geben würde, seine offizielle zweite Sonderfähigkeit.}

\bullet{} \fan{Wenn es keine offiziellen Heldenkarten für ihn gibt, folgende zweite Sonderfähigkeit:}

\fan{"Gibst du beim Sonnenaufgang Apfelnüsse als Nahrung ab, darfst du, anstatt sie zurück auf die Ausrüstungstafel zu legen, mit einem roten (10er-Stelle) und einem Heldenwürfel (1er-Stelle) würfeln und sie auf dieses Feld legen."}
\supercite{128906}









\subsection{Legende 11, Das Graue Gebirge:}

\fan{Wer keine Heldenkarte nutzt, erhält bei Legendenkarte b2 folgende Heldenaufgabe:}

\fan{"Wirf einen Kreaturenwürfel (10er-Stelle) und einen Heldenwürfel (1er-Stelle) und markiere dieses Feld mit deinem Heldenwappen. Du musst einmal dort deinen Tag beenden. Wurde diese Aufgabe erfüllt, erhältst du 1 Stärkepunkt und deine zweite Sonderfähigkeit wird aktiviert."}




\subsection{Legende 14, Der Meister des Trolls:}

\fan{Wer keine Erschöpfungskarten nutzt, erhält die folgende, wenn der Erzähler sein Heldenwappen erreicht:} 

\fan{"Solange du deine Genesungsaufgabe nicht erfüllt hast, kannst du das Lager nicht mehr betreten. Wirf einen großen grauen Würfel und lege dein Heldenwappen auf dieses Feld des Karte 'Genesung'. Wirf neu, falls auf diesem Feld bereits ein Heldenwappen liegt. Deine Genesungsaufgabe ist kein Legendenziel."}

\fan{Ein Held, der wegen seiner Erschöpfungskarte das Lager nicht mehr betreten kann, kann auch nicht durch Mitbewegen eines \hyperref[Reiter]{Reiters} oder den Portalzauber der \hyperref[Zauberin]{Zauberin} zum Lager bewegt werden.}



\subsection{Legende 15, Der vergiftete Geist:}

\fan{Wer keine Geschenkkarte nutzt, erhält zwischen Legendenkarten a3 und a4 als Geschenkaufgabe:}

\fan{"Während der Erzähler auf k steht, lege auf Feld 219 einen Gegenstand, Nahrung, Edelstein oder Holzstamm ab, den man nicht beim Aktivieren eines Agren-Plättchens erhalten kann."}

\fan{Plättchen, die durch eine Geschenkaufgabe auf Feld 219 abgelegt werden, gehen sofort aus dem Spiel.}

\fan{Wurde der Verhexte enttarnt, kann er seine Sonderfähigkeiten nicht mehr nutzen.}




\subsection{Legende 16, Im Schatten der Winterburg:}

\fan{Wer keine Verlieskarte nutzt, muss auf "n" keinen Text vorlesen. Er darf selbst einen erfinden.}




\subsection{Legende 17, Die letzte Hoffnung:}

\fan{Wer keine Hoffnungskarte nutzt, erhält als dritte Sonderfähigkeit, wenn der Erzähler sein Heldenwappen erreicht: "Lege dein Heldenwappen auf ein beliebiges Feld. Fortan hast du, wenn du dort stehst, +5 \hyperref[Bonusstärke]{Bonusstärke}."}

\fan{Spielt ihr mit mehreren gleichen Helden, gilt eine offizielle Hoffnungskarte für den Helden, der sie nutzt, nicht für andere gleiche Helden mit unterscheidbarem Spielmaterial (z.B. bei der \hyperref[Beschwörerin]{Beschwörerin} "dein Golem-Symbol" statt "das Golem-Symbol").}

\end{bgcolor}



\columnbreak


\begin{bgcolor}[color-DZ]
\section{Düstere Zeiten (2020)}


\subsection{Legende 1, Schatten über Cavern:}

Es muss ein "Kram" mitspielen.
\supercite{DZ:Begleitheft} 
\fan{Es muss jedoch nicht zwingend ein \hyperref[Zwerg]{Zwerg} mitspielen.} 

\fan{Wählt am Anfang einen beliebigen Helden. Dieser (und nur dieser) gilt als Kram, wenn eine Legendenkarte von "Kram" spricht. Ihr dürft zur Erinnerung den Namen dieses Helden mit einem Sternchen markieren.}

\fan{Hat Kram keine Helmablage, kann niemand die Schildkrone tragen oder nutzen.}

\subsection{Spielvariante "Die Zwergentüren":}

Spielt ihr mit Kram, erhält er am Anfang die Schildkrone.
\supercite{DZ:Begleitheft}

\fan{Da in dieser Fan-Spielvariante nicht alle \hyperref[Zwerg]{Zwerge} Kram sind und Kram kein \hyperref[Zwerg]{Zwerg} sein muss, wählt stattdessen am Anfang einfach, ob ihr mit der Schildkrone spielt. Falls ja, gebt sie einem beliebigen Helden, der eine Helmablage hat. Nur dieser kann die Schildkrone tragen und nutzen. Ihr dürft zur Erinnerung die Helmablage dieses Helden mit einem Sternchen markieren.}

\end{bgcolor}


\begin{nobgcolor}
\section{Magische Helden (2021)}

\subsection{Bonus-Legende "Im Bann von Choranat":}

\fan{Spielt der Spieler, der den besten Zaubertrick vorführen kann, mehrere Helden, darf er einen beliebigen von deren Zeitsteinen auf den Hahn legen.}

\end{nobgcolor}

\begin{bgcolor}[color-DeK]
\section{Die ewige Kälte (2022)}



\subsection{Legende 1, Der Winterstein:}


\fan{Ihr müsst am Anfang nicht bestimmte Helden auf bestimmte Felder stellen. Stellt stattdessen alle Helden auf verschiedene der folgenden Felder: 9, 19, 28, 43, 48, 72.}

\end{bgcolor}



\begin{bgcolor}[color-DfL]
\section{Das ferne Land (2025)}



\subsection{Farbige Perlen:} 

\fan{Ordnet in den Legenden 3 und 4 jedem Helden am Anfang eine beliebige unterschiedliche Perlenfarbe (Blau, Grün, Braun oder Violett) zu.}

\fan{Ihr dürft ein Spielmaterial der passenden Farbe zu einer Heldentafel legen, um euch an die Perlenfarbe dieses Helden zu erinnern.}

\fan{Verwechslungen sind unwahrscheinlicher, wenn ihr für jeden Helden Spielmaterial in passenderen Farben wählt.}




\subsection{Legende 1, Der wahre Meister:}



\fan{Ihr müsst am Anfang nicht bestimmte Helden auf bestimmte Felder stellen. Stellt stattdessen alle Helden auf verschiedene der folgenden Felder: 404, 410, 411, 422, 443, 448.} Nutzt ihr die Karte "Für erfahrene Andor-Spieler", stellt stattdessen alle Helden auf Feld 401.
\supercite{DfL1:erfahren}






\end{bgcolor}









\end{multicols*}




\begin{comment}
\chapter{Neue Terminologie}



\begin{multicols*}{2}\raggedcolumns



\section{Festhaltend}

Plättchen und Figuren, auf deren Feld Helden die Aktion "Laufen" nicht wählen können:

\bullet{} In "Die letzte Hoffnung": Höhlenwichte, Skelette, der Bleiche König (in Legende 16), Wachtrolle und Krahder.

\bullet{} In "Dunkle Helden": Sporne.

\bullet{} In "Alte Geister": Maasavi.

\bullet{} In "Die ewige Kälte": Der Krahder Tuavahar.

\bullet{} In "Das ferne Land": Sporne, Wachtrolle, der Trollkönig und das Büffelmonster.





\section{Mitbewegbar}

Plättchen und Figuren, die Helden mit ihrer Heldenfigur mitbewegen dürfen:


\bullet{} In "Die Legenden von Andor": Bauern.

\bullet{} In "Der Sternenschild": Der Kartograph Merrik und die Hexe Reka (bei der Fürstenaufgabe "Fieber").

\bullet{} In "Die Reise in den Norden": Der Kartograph Merrik, der Barde Grenolin, Silberzwerge und Taren.

\bullet{} In "Die letzte Hoffnung": Bauern und der Trosswagen.

\bullet{} In "Dunkle Helden": Der Kartograph Merrik.

\bullet{} In "Die Bonus-Box": König Brandur und Melkart.

\bullet{} In "Alte Geister: Tulgori und Hallgard.

\bullet{} In "Düstere Zeiten": Der verletzte Mart und Schildzwerge (in Legende 2 "Fremde Heimat").

\bullet{} In "Die ewige Kälte": Iquar.

\bullet{} In "Das ferne Land": Das Steppenvolk und Azturia.





\section{Von Kreaturen entfernbar}

Plättchen und Figuren, die aus dem Spiel gehen (oder die die Legende ein böses Ende nehmen lassen), wenn eine Kreatur auf ihrem Feld steht:

\bullet{} In "Die Legenden von Andor": Bauern und Edelsteine.

\bullet{} In "Die letzte Hoffnung": Bauern und Edelsteine.

\bullet{} In "Die Bonus-Box": König Brandur, Pergamente (in der Legende "Der Vorbote des Feuers").

\bullet{} In "Düstere Zeiten": Der verletzte Mart.








\section{Zusatzwürfel}

Würfel, die Helden im Kampf gemeinsam mit ihren Heldenwürfeln werfen:

\bullet{} In "Die Reise in den Norden": Ein vom diebischen Streifenmarder geklauter Kreaturenwürfel.

\bullet{} In "Alte Geister": Ein von Meres geklauter roter Würfel einer Zwergenstatue.







\section{Ersatzwürfel}

Würfel, die Helden im Kampf anstelle all ihrer Heldenwürfel werfen:

\bullet{} In "Die Legenden von Andor": Der große schwarze Würfel der Runensteine.

\bullet{} In "Neue Helden": Der große weiße Würfel des Wassergeists.

\bullet{} In "Die Reise in den Norden": Der große rote Würfel des Flammenschwerts Varlion.

\bullet{} In "Die letzte Hoffnung": Der große graue Würfel der bronzenen Ereigniskarte 244 oder der Hoffnungskarte der \hyperref[Dunkle Magierin]{Dunklen Magierin}.









\section{Willenspunkte-beschränkte Felder}

Felder, die nur von Helden mit genug Willenspunkten betreten werden können:

\bullet{} In "Düstere Zeiten": Das Lager der Trolle (Feld 39).

\bullet{} In "Die ewige Kälte": Die Höhle der Verwandlung (Feld 414).



\section{Willenspunkte-beschränkte Verbindungen}

Verbindungen zwischen zwei Feldern, die nur von Helden mit genug Willenspunkten überquert werden können:

\bullet{} In "Die letzte Hoffnung": Sprungfelder.

\bullet{} In "Das ferne Land": Das Boot zur Pfahlbausiedlung und der steile Felsen.




\end{multicols*}
\end{comment}








\chapter{Archetyp: Fernkämpfer}
\label{Fernkämpfer}





\begin{multicols*}{2}\raggedcolumns


\begin{nobgcolor}
Diese Regeln gelten immer für den \hyperref[Bewahrer]{Bewahrer}, die \hyperref[Bogenschützin]{Bogenschützin}, den \hyperref[Steppennomade]{Steppennomaden} und den \hyperref[Steppenreiter]{Steppenreiter} sowie für alle Helden, während sie an Bord des Schiffs stehen.

Für den \hyperref[Tarus]{Tarus} und Helden, die den großen Gegenstand Bogen tragen, gelten diese Regeln nur, wenn sie Gegner auf angrenzenden Feldern bekämpfen, nicht, wenn sie Gegner auf demselben Feld bekämpfen.


\section{Allgemein}



\subsection{Fernkampf:}

Der Fernkämpfer muss im Kampf seine Würfel nacheinander werfen und entscheiden, wann er aufhört. Er muss mindestens einen Würfel werfen. Addiert den letzten geworfenen Würfel zu seinem Kampfwert.

Er darf Gegner auf angrenzenden Feldern bekämpfen.

Er kann im Kampf Willenspunkte verlieren, auch wenn er von einem angrenzenden Feld kämpft.

Er kann mit einem Helm \fan{oder Reiterhelm} keine gleichen Würfelwerte addieren.
\supercite{DLvA:Begleitheft}

Er muss seine Würfel auch nacheinander werfen, wenn der \hyperref[Fährtenleser]{Fährtenleser} sein Horn nutzte.
\supercite{NH:Begleitheft}



\subsection{Willenspunkte-Schranken:}

Kann eine Verbindung zwischen zwei Feldern nur von Helden mit genug Willenspunkten überquert werden (z.B. Sprungfelder in "Die letzte Hoffnung"\supercite{DlH:Begleitheft,DlH11:b} oder der steile Felsen in "Das ferne Land"), darf der Fernkämpfer Gegner über diese Verbindung bekämpfen, egal wie viele Willenspunkte er hat.




\end{nobgcolor}






\begin{bgcolor}[color-DLvA]
\section{Die Legenden von Andor (2012):}





\subsection{Brücken:}

Der Fernkämpfer darf über Brücken angrenzend kämpfen.
\supercite{38688}

Der Fernkämpfer kann nicht über eine Bücke kämpfen, die durch ein rotes X \fan{oder Geröll} blockiert ist.


\subsection{Alter Wehrturm:}

Der Fernkämpfer darf vom Turm aus auch angrenzende Gegner mit +2 \hyperref[Bonusstärke]{Bonusstärke} bekämpfen.
\supercite{DLvA3:Solo,DLvA5:Turm}




\subsection{Bonus-Legende "Die Befreiung der Mine":}

Helden, die ohne den großen Gegenstand "Bogen" fernkämpfen dürfen, dürfen Varkur im Fernkampf angreifen, auch wenn im Kampf gegen ihn keine Gegenstände genutzt werden können (d.h. wenn Varkur bei der Karte "Varkurs Versteck" in der Waffenkammer auftaucht).
\supercite{BL_BdM:Versteck}



\end{bgcolor}






\begin{bgcolor}[color-StSch]
\section{Der Sternenschild (2013)}

\subsection{Wolfskampf:}

Bewegt ein Wolf sich während eines Kampfs gegen den Fernkämpfer, darf der Fernkämpfer weiterkämpfen, sofern der Wolf auch nach seiner Bewegung bei oder angrenzend zum Fernkämpfer steht.
\supercite{StSch:Begleitheft}




\end{bgcolor}


\begin{bgcolor}[color-DRidN]
\section{Die Reise in den Norden (2014)}


\subsection{Meereskampf:}

Der Fernkämpfer darf von einem Landfeld aus eine Kreatur auf einem angrenzenden Meeresfeld bekämpfen.
\supercite{DRidN:Begleitheft}


\subsection{Bootsstrecken:}

Der Fernkämpfer kann nicht entlang Bootsstrecken fernkämpfen.
\supercite{DRidN:FAQ}


\subsection{Varatans Helm der Macht:}

Der Fernkämpfer kann mit Varatans Helm nur seine bis zu 3 zuletzt geworfenen Würfel addieren.
\supercite{DRidN:Begleitheft}

\end{bgcolor}


\begin{bgcolor}[color-DlH]
\section{Die letzte Hoffnung (2016)}


\subsection{Alte Waffen:}

Der Fernkämpfer kann mit dem Eisernen Handschuh keine Heldenwürfel addieren.
\supercite{DlH:Begleitheft}

Er kann mit dem Messer nur seinen zuletzt geworfenen Würfel neu werfen,\supercite{DlH:Begleitheft} \fan{bei dem er aufhört}.

Er darf Messer, Axt, Schwertgriff und Schwertklinge auch nutzen, wenn er angrenzende Gegner bekämpft.
\supercite{DlH:Begleitheft}


\end{bgcolor}



\begin{bgcolor}[color-AG]
\section{Alte Geister (2018)}

\subsection{Rekas Schlange Maro:}

Der Fernkämpfer kann mit Rekas Schlange nur seine beiden zuletzt geworfenen Würfel addieren.
\supercite{AG:Begleitheft}


\subsection{Legende 2, Der Hexer aus Andor:}

\fan{Ist der Endkampf gegen Siantari und auf Feld 30, muss der Fernkämpfer pro gewähltem Würfel 1 Willenspunkt abgeben, auch wenn er früher aufhörte.}


\subsection{Bonus-Legende "Die vierte Statue":}

\fan{Für Zors Drachenfeuer-Gift-Effekt gilt der höchste Würfelwert des Fernkämpfers, nicht zwingend der letzte geworfene Würfel.}


\end{bgcolor}



\columnbreak





\begin{bgcolor}[color-DZ]
\section{Düstere Zeiten (2020)}

\subsection{Zwergentüren:}

\fan{Der Fernkämpfer darf über zwei passende offene Zwergentüren angrenzend kämpfen.}


\subsection{Alter Wehrturm:}

Der Fernkämpfer darf vom Turm aus auch angrenzende Gegner mit +2 \hyperref[Bonusstärke]{Bonusstärke} bekämpfen.
\supercite{DZ:Begleitheft}


\subsection{Schwarzes Einhorn:}

Kämpft der Fernkämpfer als Einhornreiter, darf er weitere Würfel nur in umgekehrter Reihenfolge (vom zweitletzt bis zum zuerst geworfenen) addieren.
\supercite{DZ:Begleitheft}



\end{bgcolor}







\begin{bgcolor}[color-DeK]
\section{Die ewige Kälte (2022)}

\subsection{\fan{Askimar-Klinge:}}

\fan{Der Fernkämpfer kann mit einer Askimar-Klinge nur seinen zuletzt geworfenen Würfel neu werfen, bei dem er aufhört.}

\end{bgcolor}





\begin{bgcolor}[color-DfL]
\section{Das ferne Land (2025)}


\subsection{Gorlots:}

Wurde ein Gorlot nicht in der ersten Kampfrunde besiegt, wird der Kampf abgebrochen, auch wenn der Gorlot noch bei oder angrenzend zum Fernkämpfer stehen würde.
\supercite{127533}





\subsection{Geister:}

Der Fernkämpfer ist immer ein Fernkämpfer, auch wenn er bei oder angrenzend zu einem Geist steht.
\supercite{DfL:Begleitheft,DfL3:G}



\subsection{Wurfmesser:}

\fan{Der Fernkämpfer kann mit einem Wurfmesser nur seinen zuletzt geworfenen Würfel neu werfen, bei dem er aufhört.}


\end{bgcolor}


\end{multicols*}











\chapter{Archetyp: Reiter}
\label{Reiter}


\begin{multicols*}{2}\raggedcolumns


\begin{nobgcolor}

Diese Regeln gelten für den \hyperref[Steppenreiter]{Steppenreiter} und den \fan{\hyperref[Ritter]{Ritter}}.


\section{Allgemein}

\subsection{Treues Reittier:}

Läuft der Reiter (auch, wenn durch das Leuchtfeuer der \hyperref[Feuerbezwingerin]{Feuerbezwingerin}\supercite{127533}), darf er einen anderen, damit einverstandenen\supercite{48981} Helden mit sich mitbewegen. Er darf vor jedem Schritt neu entscheiden, ob er einen anderen Helden mitbewegt, und falls ja, welchen. 
Der mitbewegte Held darf seinen Tag bereits beendet haben.
\supercite{DfL:Begleitheft}

Während der Reiter einen anderen Helden mitbewegt, können er und der andere Held keine weiteren mitbewegbaren Plättchen (z.B. Bauern) oder Figuren (z.B. Merrik) mitbewegen.
\supercite{40250,41284,45835}

Der Reiter kann den \hyperref[Hautwandler]{Bären} nicht mitbewegen.
\supercite{DH:Begleitheft}

Der Reiter kann eine Figur, die kein Held ist, nur mitbewegen, wenn ein Effekt ihm dies erlaubt, nicht durch seine Sonderfähigkeit.







\subsection{Ausgelöste Effekte:}

Löst der Reiter durch das Betreten oder Stehenbleiben auf einem Feld einen Effekt aus (z.B. Nebelplättchen), während er einen Helden mitbewegt, dürft ihr \fan{direkt bevor der Effekt ausgelöst wird} entscheiden, ob der Reiter oder der mitbewegte Held ihn auslöst.
\supercite{DfL:Begleitheft}

Wird durch das Mitbewegen ein anderer Held alleine stehen gelassen, werden durch das Stehenbleiben ausgelöste Effekte (z.B. Nebelplättchen) erst ausgelöst, nachdem alle durch die vollständige Bewegung des Steppenreiters ausgelösten Effekte abgehandelt wurden.
\supercite{DlH:FAQ}
\fan{Diese Effekte werden in umgekehrter Reihenfolge ausgelöst, vom zuletzt stehen gelassenen Helden zum zuerst stehen gelassenen. Diese Effekte werden auch ausgelöst, wenn der mitbewegte Held seinen Tag bereits beendet hat.}




\subsection{Willenspunkte-Schranken:}

\fan{Kann ein Feld nur von Helden mit genug Willenspunkten betreten werden (z.B. das Lager der Trolle in "Düstere Zeiten" oder die Höhle der Verwandlung in "Die ewige Kälte"), kann der Reiter, auch wenn er selbst genug Willenspunkte hat, einen Helden mit zu wenig Willenspunkten nicht auf dieses Feld mitbewegen.}

Will der Reiter einen Helden über eine Verbindung zwischen zwei Feldern mitbewegen, die nur von Helden mit genug Willenspunkten überquert werden kann (z.B. Sprungfelder in "Die letzte Hoffnung"\supercite{DlH:Begleitheft,DlH:FAQ} oder der steile Felsen in "Das ferne Land"\supercite{DfL:Begleitheft,DfL2:Erkungsungskarte_555}), muss der Reiter selbst genug Willenspunkte haben. Es ist aber egal, ob der mitbewegte Held zu wenig Willenspunkte hat.




\subsection{Festhaltende Plättchen und Figuren:}

Der Reiter kann festgehaltene Helden (z.B. von Skeletten in "Die letzte Hoffnung" oder Spornen in "Das ferne Land") nicht mitbewegen,\supercite{49724,127533} \fan{auch nicht der \hyperref[Ritter]{\fan{Ritter}} in Legende 17, "Die letzte Hoffnung", nach seiner offiziellen Hoffnungskarte.}





\end{nobgcolor}








\begin{bgcolor}[color-DLvA]
\section{\fan{Die Legenden von Andor (2012)}}


\subsection{\fan{Geheimer See:}}

\fan{Bewegt der Reiter einen anderen Helden auf den Geheimen See, lösen der Reiter und der mitbewegte Held in beliebiger Reihenfolge je eine Ereigniskarte "Geheimer See" aus.}



\subsection{\fan{Legende 4, Das Geheimnis der Mine:}}

\fan{Bewegt der Reiter einen anderen Helden über die alte Brücke, während ein Pergament dort liegt, wird nur ein Pergament weggelegt, nicht zwei.}


\end{bgcolor}




\begin{bgcolor}[color-DRidN]
\section{\fan{Die Reise in den Norden (2014)}}


\subsection{\fan{Schiffe und Boote:}}

\fan{Der Reiter darf einen anderen Helden vom Land aufs Schiff, vom Schiff aufs Land und über Bootsstrecken mitbewegen.}

\end{bgcolor}






\begin{bgcolor}[color-DlH]
\section{Die letzte Hoffnung (2016)}





\subsection{Legende 14, Der Meister des Trolls:}

\fan{Ein Held, der wegen seiner Erschöpfungskarte das Lager nicht mehr betreten kann, kann auch nicht vom \hyperref[Reiter]{Reiter} zum Lager mitbewegt werden.}







\end{bgcolor}


\begin{bgcolor}[color-AG]
\section{\fan{Alte Geister (2018)}}


\subsection{\fan{Eisplättchen:}}

\fan{Der Reiter darf ein Feld mit einem offenen Eisplättchen betreten und einen anderen Helden dabei mitbewegen, wenn der Reiter selbst oder der andere Held die Bedingung auf dem Eisplättchen erfüllt.}


\end{bgcolor}




\columnbreak




\begin{bgcolor}[color-DZ]
\section{\fan{Düstere Zeiten (2020)}}

\subsection{\fan{Schwarzes Einhorn:}}

\fan{Der Reiter kann den Einhornreiter nicht mitbewegen.} 

\fan{Ist der Reiter der Einhornreiter, kann er keine anderen Helden mitbewegen.}



\subsection{\fan{Zwergenstiefel:}}

\fan{Der Reiter kann die Zwergenstiefel nicht nutzen, während er einen anderen Helden mitbewegt.}


\subsection{\fan{Boot:}}

\fan{Der Reiter darf einen anderen Helden auf das Boot und vom Boot mitbewegen.}



\subsection{\fan{Legende 1, Schatten über Cavern:}}

\fan{Bewegt der Reiter einen anderen Helden auf den Geheimen See oder über die Alte Brücke, werden zwei Kreaturenplättchen ausgelöst.}


\end{bgcolor}


\begin{bgcolor}[color-DeK]
\section{\fan{Die ewige Kälte (2022)}}

\subsection{\fan{Treues Reittier:}}


\fan{Der Reiter kann einen Helden nicht auf ein Seefeld mitbewegen.}

\fan{Ausnahme: Mit der Mini-Erweiterung "Die Bruderfeuer", wenn ein Feuer auf der Pfahlbausiedlung (Feld 460) brennt.} 



\subsection{\fan{"Schwerer Spielen"–Winterstein:}} 

\fan{Bewegt der Reiter den Winterstein-Träger mit, verliert nur der Winterstein-Träger die Schwerer-Spielen-Willenspunkte, nicht der Reiter.}

\end{bgcolor}



\begin{bgcolor}[color-DfL]
\section{Das ferne Land (2025)}


\subsection{Geister:}

Der Reiter kann einen Helden nicht \fan{auf ein oder von einem} Feld mit oder angrenzend zu Geistern bewegen.


\end{bgcolor}



\end{multicols*}












































\chapter{\fan{2.} Trieest (Taleena), der Feuerkrieger}
\label{Feuerkrieger}


\begin{multicols*}{2}\raggedcolumns

\begin{nobgcolor}

\section{Allgemein}


\subsection{Heldentafel:}

\fan{Der Feuerkrieger hat Rang 2.}

Seine offizielle Farbe ist Orange.

Er hat:

\bullet{} 1 Würfel bei 1 bis 6 Willenspunkten,

\bullet{} 2 Würfel bei 7 bis 13 Willenspunkten,

\bullet{} 3 Würfel bei 14 bis 20 Willenspunkten.





\subsection{Danwarischer Lavastein:}

Der Feuerkrieger darf im Kampf, direkt nach seinem Würfeln (nach einem etwaigen Drehen der \hyperref[Zauberin]{Zauberin}), einen seiner niedrigsten nicht zu seinem Kampfwert addierten Würfel doch noch zusätzlich zu seinem Kampfwert addieren. Für alle weiteren Kampfrunden dieses Kampfes hat er pro Einsatz dieser Fähigkeit in diesem Kampf je einen Heldenwürfel weniger als auf seiner Heldentafel angegeben.

\fan{Er hat auch so viele Würfel weniger, wenn der \hyperref[Fährtenleser]{Fährtenleser} sein Horn nutzte.}

\fan{Hätte der Feuerkrieger keine oder eine negative Anzahl Heldenwürfel, addiert er keinen Würfelwert zu seinem Kampfwert.}

Ein durch seinen Lavastein zu seinem Kampfwert addierten Würfel kann nicht auf andere Arten (z.B. mit einem Helm oder Eisernen Handschuh in "Die letzte Hoffnung") zu seinem Kampfwert addiert werden \fan{und kann nicht mit dem Trank der Hexe verdoppelt werden}.

Er kann diese Fähigkeit nicht im \hyperref[Fernkämpfer]{Fernkampf} nutzen.
\supercite{112103}





\subsection{\fan{Bonuswürfel:}}

\fan{Wirft der Feuerkrieger im Kampf einen Gegnerwürfel gemeinsam mit seinen Heldenwürfeln (z.B. durch den Streifenmarder in "Die Reise in den Norden" oder Meres in "Die steinernen Drei"), darf er diesen mit dem Lavastein zu seinem Kampfwert addieren (wenn er am niedrigsten ist), und hat auch dann in allen weiteren Kampfrunden einen Würfel weniger als auf seiner Heldentafel angegeben.}

\fan{Hätte der Feuerkrieger keine oder eine negative Anzahl Heldenwürfel, kann er auch keinen Ersatzwürfel (z.B. den großen weißen Würfel der \hyperref[Hüterin]{Hüterin}) statt seiner Heldenwürfel werfen. Ausnahme: Hat er einen Gegnerwürfel, darf er den Ersatzwürfel statt des Gegnerwürfels werfen.}









\subsection{Kein Krieger aus Andor:}

Spricht eine Regel oder Karte vom \hyperref[Krieger]{Krieger}, ist der Feuerkrieger nicht gemeint.

\end{nobgcolor}




\columnbreak

\begin{bgcolor}[color-AG]
\section{\fan{Alte Geister (2018)}}

\subsection{\fan{Legende 2, Der Hexer aus Andor:}}

\fan{Ist der Endkampf gegen Siantari und auf Feld 30, beeinflusst die Anzahl in früheren Kampfrunden durch den Lavastein addierter Würfel nur, wie viele Würfel der Feuerkrieger maximal wählen kann, nicht, wie viele Willenspunkte er abgeben muss.}


\end{bgcolor}





\begin{bgcolor}[color-DeK]
\section{\fan{Die ewige Kälte (2022)}}

\subsection{\fan{Große Kräuterkunde:}}

\fan{Der Feuerkrieger darf durch Blaubachbeeren + rotes Blutkraut den Würfelwert 10 erhalten, auch wenn er keine oder eine negative Anzahl Würfel hat.}


\end{bgcolor}


\begin{bgcolor}[color-DfL]
\section{\fan{Das ferne Land (2025)}}

\subsection{\fan{Geister:}}

\fan{Steht der Feuerkrieger auf oder angrenzend zu Geistern, kann er im Kampf durch seinen Lavastein keinen Würfel addieren.}

\fan{Es ist egal, ob sein Gegner bei oder angrenzend zu Geistern steht.}

\end{bgcolor}










\vfill

\end{multicols*}










\chapter{3. Jarid (Jarim), die Wassermagierin}
\label{Wassermagierin}


\begin{multicols*}{2}\raggedcolumns

\begin{nobgcolor}

\section{Allgemein}


\subsection{Heldentafel:}

Die Wassermagierin hat Rang 3.

Ihre offizielle Farbe ist Blau.

Sie hat:

\bullet{} 2 Würfel bei 1 bis 6 Willenspunkten,

\bullet{} 3 Würfel bei 7 bis 20 Willenspunkten.





\subsection{Teleport:}

Die Wassermagierin darf sich als freie Handlungsmöglichkeit von einem ungeleerten Brunnen-/\fan{Quellen}plättchen zu einem anderen ungeleerten Brunnen-/\fan{Quellen}plättchen teleportieren, indem sie beide Plättchen auf die graue Seite dreht. Sie verteilt \fan{bis zur} Summe der auf beiden Plättchen gedruckten Willenspunkte auf die Heldengruppe, egal wo die Helden stehen und ob sie ihren Tag bereits beendet haben.

Der \hyperref[Krieger]{Krieger} kann dabei keine zusätzlichen Willenspunkte durch seine Sonderfähigkeit erhalten. Der \hyperref[Tarus]{Tarus} kann dabei keine Stärkepunkte erhalten.


\fan{Der Teleport gilt als Betreten des Felds und Stehenbleiben auf dem Felds. Er löst entsprechende Effekte (z.B. Nebelplättchen oder Arbaks in "Alte Geister") aus.}


Direkt vor einem Teleport muss die Wassermagierin alle Plättchen von ihrer Heldentafel ablegen, die sie ablegen kann (z.B. Gegenstände, Gold oder Holzstämme/Eisen).

Sie kann keine mitbewegbaren Plättchen (z.B. Bauern) oder Figuren (z.B. Merrik) mitteleportieren.





\subsection{Festhaltende Plättchen und Figuren:}


\fan{Die Wassermagierin darf sich auch teleportieren, wenn sie festgehalten wird (z.B. von Skeletten in "Die letzte Hoffnung" oder Spornen in "Das ferne Land"). Festhaltende Plättchen und Figuren werden nicht mitteleportiert. Ausnahme: Tuavahr in "Die ewige Kälte".}


\end{nobgcolor}



\begin{bgcolor}[color-DRidN]
\section{Die Reise in den Norden (2014)}


\subsection{Nord-Brunnen:}

\fan{Dreht die Wassermagierin beim Teleport einen Nord-Brunnen um, auf dem ein Trinkschlauch oder Gold abgebildet ist, wird jenes auf das entsprechende Brunnenfeld gelegt.}


\end{bgcolor}






\begin{bgcolor}[color-DlH]
\section{\fan{Die letzte Hoffnung (2016)}}

\subsection{\fan{Teleport:}}

\fan{Das Umdrehen der Quellen beim Teleport gilt für Ereigniskarten nicht als Leeren der Quelle.}



\end{bgcolor}





\begin{bgcolor}[color-AG]
\section{Alte Geister (2018)}


\subsection{Legende 1, Die Spur des Drachen:}

\fan{Ist die Karte "F" (Version Brunnen) im Spiel, darf die Wassermagierin, wenn sie sich zwischen zwei Brunnen teleportiert, 12 Willenspunkte auf die Heldengruppe verteilen (statt 6). Danach gehen beide Brunnen aus dem Spiel.}



\subsection{Legende 3, Die steinernen Drei:}

Die Wassermagierin kann sich nicht von oder zur Quelle auf Feld 37 teleportieren.
\supercite{AG:Begleitheft}

\fan{Teleportiert sie sich von oder zu einem Brunnen, bei dem Fürst Hallgard steht, wird sein Stärkemarker nicht nach rechts verschoben.}



\end{bgcolor}





\begin{bgcolor}[color-DeK]
\section{\fan{Die ewige Kälte (2022)}}

\subsection{\fan{Feuerteleport:}}

\fan{Die Wassermagierin darf sich als freie Handlungsmöglichkeit von einem brennenden Feuer zu einem anderen brennenden Feuer teleportieren, indem sie beide Feuer auf die erloschene Seite dreht. Es werden keine Willenspunkte auf die Heldengruppe verteilt.} 

\fan{Sie kann sich nicht von oder zu einem Feuer auf der Heldentafel der \hyperref[Feuerwächterin]{Feuerwächterin} teleportieren.}


\fan{Der Teleport gilt als Betreten des Felds und Stehenbleiben auf dem Feld. Er löst entsprechende Effekte (z.B. Nebelplättchen) aus.}

\fan{Sie kann keine Iquar oder Figuren mitteleportieren.}

\fan{Direkt vor einem Teleport muss die Wassermagierin alle Plättchen von ihrer Heldentafel ablegen, die sie ablegen kann (z.B. Gegenstände oder Edelsteine). Auf einem Händlerfeld abgelegte Gegenstände gehen in den Besitz der Händler über und können üblicherweise erst durch Tausch wieder aufgenommen werden.}

\fan{Zielmarker darf sie aber mitteleportieren.}





\subsection{\fan{Legende 4, Die Pranken des Winters:}}

\fan{Die Wassermagierin darf sich auch teleportieren, wenn sie von Tuavahr festgehalten wird. Tuavahr wird jeweils sofort auf ihr neues Feld mitversetzt.}




\end{bgcolor}



\columnbreak


\begin{bgcolor}[color-DfL]
\section{\fan{Das ferne Land (2025)}}


\subsection{\fan{Teleport:}}

\fan{Ruinengeister bleiben beim Teleport auf der Heldentafel.}


\subsection{\fan{Geister:}}

\fan{Die Wassermagierin kann sich nicht von oder zu einem Brunnen bei oder angrenzend zu Geistern teleportieren.}

\fan{Sie darf beim Teleportieren auch Willenspunkte auf Helden verteilen, die bei oder angrenzend zu Geistern stehen.}

\end{bgcolor}





\end{multicols*}












\chapter{4. Leander (Leane), der Seher}
\label{Seher}

\begin{multicols*}{2}\raggedcolumns

\begin{nobgcolor}

\section{Allgemein}


\subsection{Heldentafel:}

Der Seher hat Rang 4.

Seine offizielle Farbe ist Dunkelblau.

Er hat höchstens 14 Willenspunkte (statt 20).

Er hat 2 kleine rechteckige Ablagefelder (statt 3).

Seine Würfelleiste hat Platz für bis zu 5 Würfel.




\subsection{Würfelleiste:}

Der Seher hat 5 Würfel, egal wie viele Willenspunkte er hat.

Ist seine Würfelleiste leer, wirft der Seher alle seine Würfel und legt:

\bullet{} bei 5 bis 9 Willenspunkten bis zu\supercite{DH:FAQ,DH:Begleitheft} 1 Würfel weg,

\bullet{} bei 10 bis 14 Willenspunkten bis zu\supercite{DH:FAQ,DH:Begleitheft} 2 Würfel weg.

Die übrigen Würfel legt er in beliebiger Reihenfolge auf seine Würfelleiste.



\subsection{Kampf:}

Statt Heldenwürfel zu werfen, legt der Seher den nächsten Würfel seiner Würfelleiste weg und wertet diesen. Der Würfel darf vor der Wertung auf die gegenüberliegende Seite gedreht werden (z.B. von der \hyperref[Zauberin]{Zauberin})\supercite{DH:FAQ} und mit dem Trank der Hexe verdoppelt werden.

Der Seher kann auch mit einem Helm nicht mehrere Würfel seiner Würfelleiste addieren.
\supercite{DH:FAQ,DH:Begleitheft}

\fan{Das Horn des \hyperref[Fährtenleser]{Fährtenlesers} hat keinen Effekt auf ihn.}

Legt der Seher in der letzten Kampfrunde seinen letzten Würfel weg, darf er mit dem Neuwerfen bis nach der Belohnung warten.
\supercite{DH:FAQ,DH:Begleitheft}






\subsection{Bonuswürfel:}

Der Seher kann keinen Gegnerwürfel gemeinsam mit seinen Heldenwürfeln werfen (z.B. durch \fan{den Streifenmarder in "Die Reise in den Norden" oder} Meres in "Die steinernen Drei"\supercite{AG:Begleitheft}). Dennoch wirft der Gegner einen Würfel weniger.

\fan{Wirft der Seher einen Ersatzwürfel (z.B. den großen weißen Würfel der \hyperref[Hüterin]{Hüterin}), legt er keinen Würfel von seiner Würfelleiste weg.}
\supercite{49630}







\subsection{Sehendes Auge:}

Zeigt der nächste Würfel seiner Würfelleiste "1", darf er die Aktion "Sehendes Auge" wählen. Dann gibt er 1 Stunde ab, legt die "1" weg und deckt ein \hyperref[Aufdecken]{\fan{vorhersehbares}} Plättchen auf dem Spielplan (auch an der Tagesleiste oder Legendenleiste) auf. Zeigt der nächste Würfel seiner Würfelleiste (\fan{nach etwaigem Neuwerfen}) jeweils erneut "1", darf er jeweils erneut für 1 Stunde die "1" weglegen und ein Plättchen aufdecken.\supercite{46520}

Aktiviert werden diese Plättchen dadurch nicht. Plättchen in Stapeln dürfen auch aufgedeckt werden, ihre Reihenfolge bleibt aber gleich.
\supercite{BL_BdM:A3_en}

Die \hyperref[Zauberin]{Zauberin} kann einen Würfel nicht umdrehen, damit der Seher ihn für das sehende Auge nutzen könnte.
\supercite{DH:FAQ} 





\subsection{Doch nicht so blind?}

\fan{Der Seher darf ein Fernrohr nutzen.}


\end{nobgcolor}



\begin{bgcolor}[color-DLvA]
\section{Die Legenden von Andor (2012)}


\subsection{Legende 3, Die Tage des Widerstands:}

\fan{Nutzt Varkur die Dunkle Magie "Die Schwächung der Helden", legt der Seher den obersten Würfel seiner Würfelleiste weg, statt einen Würfel zu werfen. Der weggelegte Würfel wird auf die gegenüberliegende Seite gedreht, wenn diese kleiner ist.}



\subsection{Ereigniskarte "Geheimer See" 9:}

\fan{Statt all seine Heldenwürfel zu werfen, legt der Seher alle Würfel seiner Würfelleiste weg und guckt, ob eine 1 oder 2 darunter war.}


\end{bgcolor}



\begin{bgcolor}[color-StSch]
\section{Der Sternenschild (2013)}

\subsection{Andorische Flöte:}

\fan{Nutzt der Seher die andorische Flöte, legt er den nächsten Würfel seiner Würfelleiste weg und wertet diesen, statt einen Heldenwürfel zu werfen.}


\end{bgcolor}






\begin{bgcolor}[color-DRidN]
\section{\fan{Die Reise in den Norden (2014)}}



\subsection{\fan{Achterballista und Varatans Helm der Macht:}}

\fan{Mit der Achterballista oder Varatans Helm darf der Seher, sofern er 2 oder mehr Würfel auf seiner Würfelleiste trägt, die 2 nächsten Würfel seiner Würfelleiste weglegen.}

\fan{Mit Varatans Helm darf er, sofern er 3 oder mehr Würfel auf seiner Würfelleiste trägt, die 3 nächsten Würfel seiner Würfelleiste weglegen.}

\fan{Die Summe aller weggelegten Würfel wird zu seinem Kampfwert addiert.}


\subsection{\fan{Legende "Die Rückkehr der Schwarzen Kogge":}}

\fan{Wirft der Unbekannte Krieger die Würfel des Sehers, wirft er alle Würfel des Sehers, die gerade nicht auf seiner Würfelleiste liegen. Würfel auf seiner Würfelleiste bleiben liegen und werden nicht gewertet.}



\end{bgcolor}



\columnbreak


\begin{bgcolor}[color-DlH]
\section{Die letzte Hoffnung (2016)}



\subsection{Zweite Sonderfähigkeit: Ereignis sehen}

Nur der Seher darf die Aktion "Ereignis sehen" wählen, und nur, wenn die oberste Ereigniskarte verdeckt ist. Für 1 Stunde legt er den obersten Würfel seiner Würfelleiste weg. Dann darf er die oberste Ereigniskarte aufdecken, aber noch nicht aktivieren.

Wird eine so aufgedeckte Ereigniskarte aktiviert, kann sie nicht mit einem Schild abgewehrt werden.
\supercite{DH11:Heldenkarte_Seher}



\subsection{Alte Waffen:}

Der Seher kann den Eisernen Handschuh nicht nutzen, um mehrere Würfel seiner Würfelleiste zu addieren.
\supercite{DH:FAQ,DH:Begleitheft}


\fan{Der Seher darf mit dem Messer den nächsten Würfel seiner Würfelleiste neu werfen. Eine so gewürfelte 1 kann aber nicht fürs Sehende Auge genutzt werden.}






\subsection{Legende 17, Die letzte Hoffnung:}

Da es keine Ereigniskarten gibt, kann der Seher seine zweite Sonderfähigkeit nicht nutzen.
\supercite{DH:Begleitheft}

\end{bgcolor}













\begin{bgcolor}[color-AG]
\section{Alte Geister (2018)}



\subsection{Legende 2, Der Hexer aus Andor:}

Mit Rekas Schlange darf der Seher, sofern er 2 oder mehr Würfel auf seiner Würfelleiste trägt, die zwei nächsten Würfel seiner Würfelleiste weglegen und ihre Summe zu seinem Kampfwert addieren.
\supercite{AG:Begleitheft}

\fan{Ist der Endkampf gegen Siantari und auf Feld 30, muss der Seher 1 Willenspunkt pro zu seinem Kampfwert addierten Würfel abgeben. Er kann nicht zusätzliche Willenspunkte abgeben, um zusätzliche Würfel von seiner Würfelleiste zu wählen. Ist seine Würfelleiste leer, muss er keine Willenspunkte abgeben, um seine Würfel wieder zu werfen.}

\subsection{Bonus-Legende "Die vierte Statue":}

\fan{Für Zors Drachenfeuer-Gift-Effekt zählt der in dieser Kampfrunde von der Würfelleiste weggelegte Würfel, keine auf der Würfelleiste liegen gebliebene Würfel.}


\end{bgcolor}



\begin{bgcolor}[color-DZ]
\section{Düstere Zeiten (2020)}


\subsection{Schwarzes Einhorn:}

Kämpft der Seher als Einhornreiter, kann er für jeden Willenspunkt nur den nächsten Würfel seiner Würfelleiste addieren und weglegen, bis er aufhören will oder seine Würfelleiste leer ist.
\supercite{DZ:Begleitheft}



\subsection{Shans Fluch:}

Muss der Seher wegen Shans Fluch einen Würfel abgeben, muss er jeweils den nächsten Würfel seiner Würfelleiste abgeben.
\supercite{DZ:Begleitheft}
\fan{Wirft der Seher bei leerer Würfelleiste neu, bleiben seine Würfel auf Schattenplättchen liegen und er kann für jeden seiner Würfel auf Schattenplättchen einen Würfel weniger weglegen.}


\end{bgcolor}




\columnbreak



\begin{bgcolor}[color-DeK]
\section{\fan{Die ewige Kälte (2022)}}




\subsection{\fan{Askimar-Klinge:}}

\fan{Der Seher darf mit einer Askimar-Klinge seinen genutzten Würfel neu werfen. Eine so gewürfelte 1 kann aber nicht fürs Sehende Auge genutzt werden.}


\subsection{\fan{Große Kräuterkunde:}}

\fan{Hat der Seher durch Blaubachbeeren + rotes Blutkraut den Würfelwert 10, legt er in dieser Kampfrunde keinen Würfel seiner Würfelleiste weg.}

\end{bgcolor}




\begin{bgcolor}[color-DfL]
\section{\fan{Das ferne Land (2025)}}

\subsection{\fan{Pilze:}}


\fan{Das sehende Auge darf einen Pilz auf dem Spielplan oder auf einer Heldentafel aufdecken, ohne ihn zu aktivieren. Ein so aufgedeckter Pilz wird wie ein verdeckter behandelt und darf von seinem Träger als freie Handlungsmöglichkeit aktiviert werden.}






\subsection{\fan{Wurfmesser:}}

\fan{Der Seher darf mit einem Wurfmesser seinen genutzten Würfel neu werfen. Eine so gewürfelte 1 kann aber nicht fürs Sehende Auge genutzt werden.}




\subsection{\fan{Geister:}}

\fan{Steht der Seher bei oder angrenzend zu Geistern, kann er sein Sehendes Auge nicht nutzen. Er darf mit seinem Sehenden Auge Plättchen aufdecken, die bei oder angrenzend zu Geistern liegen.}

\fan{Er nutzt im Kampf Würfel von seiner Würfelleiste und wirft Würfel bei leerer Würfelleiste neu, auch wenn er bei oder angrenzend zu Geistern steht.}

\end{bgcolor}



\end{multicols*}










\chapter{5. Ijsdur (Ijsdora), der Eis-Dämon}
\label{Eis-Dämon}

\begin{multicols*}{2}\raggedcolumns


\begin{nobgcolor}

\section{Allgemein}



\subsection{Aufbau:} 

Legt den Eiskristall mit der blauen Seite nach oben oberhalb des ersten Symbols des Sonnenaufgang-Felds.
\supercite{MH:Kälte,MH:Begleitheft}

Legt 5 Eisblitze bereit.





\subsection{Heldentafel:}

Der Eis-Dämon hat Rang 5.

Seine offizielle Farbe ist Weiß.

Er hat höchstens 12 Stärkepunkte (statt 14).

Er hat höchstens 13 Willenspunkte (statt 20).

Er hat 2 kleine rechteckige Ablagefelder (statt 3).

Er hat keine Helmablage (statt 1).

Er hat kein großes Ablagefeld (statt 1).

Er hat 3 besondere Ablagefelder für Eisblitze.

Er hat 1 Würfel, egal wie viele Willenspunkte er hat: Den Pyramidenwürfel. 





\subsection{Pyramidenwürfel:}

Der Pyramidenwürfel hat die Seiten 1/2/3/4. Er kann nicht auf die gegenüberliegende Seite gedreht werden (z.B. von der \hyperref[Zauberin]{Zauberin}), da er keine gegenüberliegenden Seiten hat.
\supercite{91842}
Er kann nicht genutzt werden, wenn eine Legende das Werfen eines beliebigen Heldenwürfels außerhalb eines Kampfes verlangt (z.B. für das Bestimmen einer Feldzahl).
\supercite{88552}




\subsection{Blaue Kälte:}

Liegt der Eiskristall am Anfang eines Sonnenaufgangs mit der blauen Seite nach oben, erhält der Eis-Dämon auf seine Ablage:

\bullet{} bei 7 bis 9 Willenspunkten 1 Eisblitz,

\bullet{} bei 10 bis 13 Willenspunkten 2 Eisblitze.

Dreht den Eiskristall danach auf die orange-rote Seite.
\supercite{MH:Kälte,MH:Begleitheft}


\subsection{Orange-rote Kälte:}

Liegt der Eiskristall am Anfang eines Sonnenaufgangs mit der orange-roten Seite nach oben, werden alle Eisbrücken vom Spielplan und alle Kampfblitze von Stärkeleisten entfernt. Eisblitze auf Ablagefeldern bleiben liegen.

Dreht den Eiskristall danach auf die blaue Seite.
\supercite{MH:Kälte,MH:Begleitheft}




\subsection{Eisblitze:}

Eisblitze sind keine Gegenstände. Der Eis-Dämon kann sie nicht wie Gegenstände ablegen oder aufnehmen. Würde er mehr Eisblitze erhalten, als er tragen kann, verfallen überzählige.
\supercite{MH:Begleitheft}



\columnbreak


\subsection{Kampfblitz:}


Jederzeit während seines Zugs darf der Eis-Dämon 1 Eisblitz von seiner Ablage als Kampfblitz einsetzen. Das geht nur, wenn er 7 oder weniger Stärkepunkte hat und kein Eisblitz auf seiner Stärkeleiste liegt.
\supercite{MH:Kälte}
Dann legt er den Kampfblitz rechts an seinen Stärkestein auf seine Stärkeleiste. Solange der Kampfblitz dort liegt, hat er +5 \hyperref[Bonusstärke]{Bonusstärke}.
\supercite{MH:Begleitheft}
Wir der Stärkestein des Eis-Dämons bewegt, bewegt den Kampfblitz mit.
\supercite{88778} 
\fan{Erreicht der Eis-Dämon 8 oder mehr Stärkepunkte, entfernt den Kampfblitz.}




\subsection{Eisbrücke:}

Jederzeit während seines Zugs darf der Eis-Dämon zwei verschiedene Spielplanfelder wählen und 1 oder 2 Eisblitze von seiner Ablage als Eisbrücke zwischen diesen Feldern auf den Spielplan legen.
\supercite{MH:Begleitheft}
Das geht nur, wenn beide gewählten Felder bis zu 4.7 cm pro verwendetem Eisblitz entfernt sind. 

Solange die Eisbrücke im Spiel ist, gelten die beiden Felder als angrenzend für Helden, von Helden bewegte Figuren und \hyperref[Helfer]{Helfer}. 
\supercite{88780}

Es können mehrere Eisbrücken gleichzeitig im Spiel sein.

Gegner ignorieren Eisbrücken und beachten durch sie verbundene Felder nur als angrenzend, wenn sie es wirklich sind.
\supercite{MH:Kälte}

Eisbrücken dürfen über den Fluss, Brücken, Gebirge, Felsen, Seen, Meere und Schluchten gebaut werden.
\supercite{MH:Begleitheft}




\subsection{Feld für besiegte Kreaturen:}

Eisbrücken dürfen auch aufs Feld für besiegte Kreaturen gebaut werden.\supercite{88778} \fan{Dort liegende Kreaturen können aber nicht erneut bekämpft werden. Ein dort liegender Knochen-Golem der \hyperref[Beschwörerin]{Beschwörerin} kann nicht bewegt werden.}



\subsection{Willenspunkte-Schranken:}

Jeder Held kann eine Eisbrücke nutzen, egal wie viele Willenspunkte er hat, auch wenn zwischen ihren Endfeldern eine Verbindung ist, die nur von Helden mit genug Willenspunkten überquert werden kann (z.B. Sprungfelder in "Die letzte Hoffnung"\supercite{MH11:Heldenkarte_Eis-Dämon} \fan{oder der steile Felsen in "Das ferne Land"}).




\subsection{\fan{Mehrere Eis-Dämonen:}}

\fan{Legt am Anfang 5 Eisblitze pro Eis-Dämon bereit, aber nur ein Eiskristall-Plättchen auf das Sonnenaufgang-Feld. Dieses gilt für alle Eis-Dämonen.}

\fan{Jeder Eis-Dämon kann Eisblitze von seiner Heldentafel nicht anderen Eis-Dämonen geben und einen Kampfblitz nur auf seine eigene Stärkeleiste legen.}


\end{nobgcolor}



\columnbreak


\begin{bgcolor}[color-DLvA]
\section{Die Legenden von Andor (2012)}

\subsection{Eisbrücke:}

Eisbrücken dürfen zu Feld 83 gebaut werden. 
\supercite{88778}

\fan{Eine Eisbrücke kann auch genutzt werden, wenn zwischen ihren Endfeldern eine durch ein rotes X oder Geröll blockierte Brücke liegt.}



\subsection{Ausscheiden:}

\fan{Scheidet der Eis-Dämon aus (z.B. weil er in Cavern auf 0 Willenspunkte fiel), gehen Eisblitze von seiner Heldentafel aus dem Spiel. Eisbrücken bleiben aber im Spiel, bis der Eiskristall sie entfernt.}


\subsection{Legende 5, Der Zorn des Drachens:}

\fan{Der Eis-Dämon darf über eine Eisbrücke die von Kreaturen besetzte Rietburg (Feld 0) betreten. Andere Helden können dies nicht. Wird die Eisbrücke, über die der Eis-Dämon die Rietburg betrat, entfernt, während er gemeinsam mit Kreaturen dort steht, wird seine Heldenfigur sofort zum anderen Ende der Eisbrücke versetzt.}

\fan{Bei der Drachenkampfkarte "Gegenstand in 2 Stärkepunkte tauschen" kann der Eis-Dämon keinen Eisblitz in Stärke tauschen.}





\subsection{Goldene Ereigniskarte 6:}

\fan{Der Eis-Dämon wirft einen normalen Heldenwürfel, nicht den Pyramidenwürfel.}




\subsection{Ereigniskarte "Geheimer See" 9:}

\fan{Der Eis-Dämon wirft seinen Pyramidenwürfel, keine normalen Heldenwürfel.}


\end{bgcolor}




\begin{bgcolor}[color-StSch]
\section{Der Sternenschild (2013)}

\subsection{Sternenschild:}

\fan{Der Eiskristall darf durch den Sternenschild abgedeckt werden. Dann wird kein Kälte-Effekt ausgelöst und der Eiskristall nicht auf die gegenüberliegende Seite gedreht.}



\subsection{Andorische Flöte:}

\fan{Nutzt der Eis-Dämon die andorische Flöte, wirft er den Pyramidenwürfel, keinen normalen Heldenwürfel.}



\subsection{Fackel von Cavern:}

Kreaturen können mit der Fackel von Cavern nicht entlang Eisbrücken versetzt werden (außer die Felder sind wirklich angrenzend).
\supercite{88781}


\end{bgcolor}




\columnbreak


\begin{bgcolor}[color-DRidN]
\section{Die Reise in den Norden (2014)}

\subsection{Eisbrücke:}

Eine Eisbrücke darf zwei Landfelder, zwei Meeresfelder oder ein Landfeld und ein Meeresfeld angrenzend machen.
\supercite{MH:Begleitheft}

\fan{Eine Eisbrücke kann nicht auf Feld 90 gebaut werden.}


\fan{Der \hyperref[Seekrieger]{Seekrieger} darf das Schiff mit seiner Sonderfähigkeit entlang einer Eisbrücke versetzen.}

\fan{Verbindet eine Eisbrücke zwei Meeresfelder horizontal, vertikal oder 45$^\circ$ diagonal, darf das Schiff beim Segeln entlang dieser Eisbrücke bewegt werden, als wären die beiden Felder in dieser Richtung angrenzend.}



\end{bgcolor}





\begin{bgcolor}[color-DlH]
\section{Die letzte Hoffnung (2016)}



\subsection{Zweite Sonderfähigkeit: Längere Eisbrücken}

Eisbrücken dürfen aus bis zu 3 Eisblitzen gebaut werden.
\supercite{MH11:Heldenkarte_Eis-Dämon}




\subsection{Legende 11, Das Graue Gebirge}

Spielt ihr mit der offiziellen Heldenkarte, legt am Anfang 6 Eisblitze bereit (statt 5).



\subsection{Legende 15, Der vergiftete Geist:}

\fan{Wird der Eis-Dämon enttarnt, gehen Eisblitze von seiner Heldentafel aus dem Spiel. Eisbrücken bleiben aber liegen, bis der Eiskristall sie entfernt. Er kann Eisblitze nicht in Stärke tauschen.}



\subsection{Legende 17, Die letzte Hoffnung:}

\fan{Der Eiskristall darf durch den Sternenschild abgedeckt werden. Dann wird er nicht ausgelöst und nicht auf die gegenüberliegende Seite gedreht.}

\fan{Spielt ihr mit mehreren Eis-Dämonen: Da alle Eisbrücken ununterscheidbar sind, gilt der Effekt der offiziellen Hoffnungskarte für Eisbrücken aller Eis-Dämonen, nicht nur desjenigen, der die Hoffnungskarte nutzt.}


\end{bgcolor}






\begin{bgcolor}[color-BB]
\section{Die Bonus-Box (2017/2024)}

\subsection{Spielvariante "Die Fluggors":}

In dieser Spielvariante ist Feld 83 nie zu einem anderen Feld angrenzend, auch nicht mit einer Eisbrücke.
\supercite{88779}

\end{bgcolor}


\begin{bgcolor}[color-AG]
\section{Alte Geister (2018)}


\subsection{Kraft des roten Mondes:}

\fan{Der Eiskristall kann nicht durch die Kraft des roten Mondes abgedeckt werden, auch nicht, wenn das Ereigniskarten-Symbol abgedeckt wird.}

\end{bgcolor}




\begin{bgcolor}[color-DeK]
\section{\fan{Die ewige Kälte (2022)}}

\subsection{\fan{Eisbrücke:}}

\fan{Eisbrücken dürfen zu Feld 83 gebaut werden.}

\end{bgcolor}



\columnbreak


\begin{bgcolor}[color-DfL]
\section{\fan{Das ferne Land (2025)}}

\subsection{\fan{Eisbrücke:}}

\fan{Eisbrücken dürfen auch zur Pfahlbausiedlung (Feld 453) gebaut werden. Die Grenze von Feld 453 ist zwischen der Pfahlbausiedlung und dem See, nicht am äußeren Seeufer.}

\fan{Eisbrücken dürfen zwischen zwei Feldern des Schachtelbodens gebaut werden, aber nicht zwischen dem Spielplan und dem Schachtelboden.}


\subsection{\fan{Geister:}}

\fan{Steht der Eis-Dämon bei oder angrenzend zu Geistern, erhält er bei Sonnenaufgang keine Eisblitze, kann keine Eisblitze von seiner Heldentafel einsetzen und erhält keine \hyperref[Bonusstärke]{Bonusstärke} durch Kampfblitze. Es ist egal, ob der Gegner bei oder angrenzend zu Geistern steht.}

\fan{Solange Geister bei oder angrenzend zu einem Endfeld einer Eisbrücke stehen, hat diese Eisbrücke keinen Effekt. Dennoch darf der Eis-Dämon Eisbrücken von und zu Feldern mit oder angrenzend zu Geistern bauen.}

\fan{Eisbrücken auf dem Spielplan funktionieren, auch wenn der Eis-Dämon bei oder angrenzend zu Geistern steht.}

\fan{Bei der orange-roten Seite des Eiskristalls werden Eisblitze vom Spielplan und der Stärkeleiste entfernt, auch wenn der Eis-Dämon oder Eisblitze sich bei oder angrenzend zu Geistern befinden.}


\end{bgcolor}







\end{multicols*}








\chapter{6. Orfen (Marfa), der Wolfskrieger}
\label{Wolfskrieger}

\begin{multicols*}{2}\raggedcolumns



\begin{nobgcolor}

\section{Allgemein}

\subsection{Aufbau:}

Wählt die Würfelanzahl des Wolfskrieger bei 1 bis 6 Willenspunkten.


\subsection{Heldentafel:}

Der Wolfskrieger hat Rang 6.

Seine offiziellen Farben sind Gelb, Grau und Dunkelblau.

Er hat:

\bullet{} 1 oder 2 Würfel bei 1 bis 6 Willenspunkten,

\bullet{} 2 Würfel bei 7 bis 13 Willenspunkten,

\bullet{} 3 Würfel bei 14 bis 20 Willenspunkten.






\subsection{Geländekunde:}

Im Kampf (außer gegen Trolle) hat der Wolfskrieger, wenn er weniger Stärkepunkte hat als die 10er-Stelle seiner Feldzahl, die Differenz zwischen der 10er-Stelle seiner Feldzahl und seinen Stärkepunkten als \hyperref[Bonusstärke]{Bonusstärke}.

Beispiel: Steht er mit 3 Stärkepunkten auf Feld 72, hat er +4 Bonusstärke, sodass die Summe aus Stärkepunkten und Bonusstärke 7 ist.

Das gilt auch gegen Endgegner.



\subsection{Trollhass:}

Im Kampf gegen Trolle hat der Wolfskrieger die 10er-Stelle seiner Feldzahl als \hyperref[Bonusstärke]{Bonusstärke}.

Beispiel: Steht er auf Feld 72, hat er +7 Bonusstärke.

Das gilt auch gegen Endgegner.





\subsection{Dreistellige Feldzahlen:}

Auch bei dreistelligen Feldzahlen zählt nur die 10er-Stelle. Zum Beispiel hat er auf Feld 144 einen Trollhass-Bonus von +4, nicht +14.
\supercite{10525}




\subsection{Kein Krieger des Königs mehr:}

Spricht eine Regel oder Karte vom \hyperref[Krieger]{Krieger}, ist der Wolfskrieger nicht gemeint.

\end{nobgcolor}


\begin{bgcolor}[color-StSch]
\section{Der Sternenschild (2013)}

\subsection{Geländekunde und Trollhass:}

\fan{Für seine \hyperref[Bonusstärke]{Bonusstärke} gelten auch der Belagerungsturm und die Trolle des Katapults als Trolle.}

\end{bgcolor}




\begin{bgcolor}[color-DRidN]
\section{Die Reise in den Norden (2014)}

\subsection{Geländekunde und Trollhass:}

Ist der Wolfskrieger an Bord des Schiffes, erhält er keine \hyperref[Bonusstärke]{Bonusstärke} durch eine Feldzahl.

Für seine \hyperref[Bonusstärke]{Bonusstärke} gelten Meerestrolle nicht als Trolle.
\supercite{10469}

\end{bgcolor}


\begin{bgcolor}[color-DlH]
\section{\fan{Die letzte Hoffnung (2016)}}

\subsection{\fan{Geländekunde und Trollhass:}}

\fan{Für seine \hyperref[Bonusstärke]{Bonusstärke} gelten auch der Urtroll und Wachtrolle als Trolle.}

\end{bgcolor}



\begin{bgcolor}[color-DZ]
\section{Düstere Zeiten (2020):}

\subsection{Geländekunde und Trollhass:}


\fan{Ist der Wolfskrieger an Bord des Boots, erhält er keine \hyperref[Bonusstärke]{Bonusstärke} durch eine Feldzahl.}

\fan{Für seine \hyperref[Bonusstärke]{Bonusstärke} gelten auch die drei großen Trolle als Trolle.}

\end{bgcolor}



\begin{bgcolor}[color-DeK]
\section{\fan{Die ewige Kälte (2022)}}

\subsection{\fan{Geländekunde und Trollhass:}}

\fan{Für seine \hyperref[Bonusstärke]{Bonusstärke} gelten auch Felltrolle als Trolle.}


\end{bgcolor}





\begin{bgcolor}[color-DfL]
\section{\fan{Das ferne Land (2025)}}

\subsection{\fan{Geländekunde und Trollhass:}}

\fan{Für seine \hyperref[Bonusstärke]{Bonusstärke} gelten auch Sumpftrolle und Wachtrolle als Trolle. Der Trollkönig ebenfalls, aber dessen Feld hat keine Zehnerstelle.}




\subsection{\fan{Geister:}}

\fan{Steht der Wolfskrieger auf oder angrenzend zu Geistern, erhält er keine \hyperref[Bonusstärke]{Bonusstärke} durch Geländekunde und Trollhass. Es ist egal, ob sein Gegner bei oder angrenzend zu Geistern steht.}


\end{bgcolor}




\end{multicols*}








\chapter{7. Kram (Bait), der Zwerg}
\label{Zwerg}

\begin{multicols*}{2}\raggedcolumns


\begin{nobgcolor}

\section{Allgemein}


\subsection{Aufbau:}

Wählt den Rang und die Sonderfähigkeit des Zwergs.


\subsection{Heldentafel:}

Der Zwerg hat Rang 7 oder Rang 71.

Seine offizielle Farbe ist Gelb.

Er hat:

\bullet{} 1 Würfel bei 1 bis 6 Willenspunkten,

\bullet{} 2 Würfel bei 7 bis 13 Willenspunkten,

\bullet{} 3 Würfel bei 14 bis 20 Willenspunkten.



\subsection{Sonderfähigkeit:}

\fan{Wählt eine der folgenden Sonderfähigkeiten:}

\bullet{} \textbf{Feld 71} (nur in "Die Legenden von Andor" geeignet):

Der Zwerg darf auf Feld 71 Stärkepunkte für je 1 Gold kaufen (statt 2).

\bullet{} \textbf{Nahrung} (nur in "Die letzte Hoffnung" geeignet):

Gibt der Zwerg bei Sonnenaufgang keine Nahrung ab, verliert er 4 Willenspunkte (statt 8). Gibt der Zwerg bei Sonnenaufgang Apfelnüsse oder Sternkraut als Nahrung ab (keine Mondbeeren\supercite{MH:Mondbeeren,MH:Begleitheft}), erhält er 1 Stärkepunkt oder 6 Willenspunkte. Er darf bei Sonnenaufgang immer Nahrung abgeben, auch wenn er nicht muss (z.B. weil er im Lager steht).
\supercite{DlH15:k_4}

\bullet{} \textbf{Zwergenwut} (\fan{für alle Legenden geeignet}):

Der Zwerg darf im Kampf, direkt nach seinem Würfeln, beliebig viele Willenspunkte abgeben, um so viel \hyperref[Bonusstärke]{Bonusstärke} zu erhalten. 
\supercite{GdK}

\bullet{} \textbf{Kampfbelohnung} (\fan{für alle Legenden geeignet}):

Erhält der Zwerg fürs Besiegen eines auf der Kreaturenanzeige abgebildeten Gegners eine Belohnung, darf er 1 zusätzlichen Stärkepunkt erhalten. Das gilt auch, wenn der Zwerg gemeinsam mit anderen Helden kämpfte und die Belohnung nur auf die anderen Helden verteilt wurde.
\supercite{BGG:extrastaerke-zwerg}
\fan{Das gilt nicht, wenn der Gegner gar keine Belohnung gibt (z.B. durch den \hyperref[Bewahrer]{Bewahrer}, das Gift in "Die Legenden von Andor" oder die Schattenklinge in "Das ferne Land").}


\end{nobgcolor}


\begin{bgcolor}[color-DLvA]
\section{Die Legenden von Andor (2012)}

\subsection{Bonus-Legende "Der Kampf um Cavern":}

Es muss ein "Kram" mitspielen.
\supercite{BL_KuC:A}
\fan{Es muss jedoch nicht zwingend ein Zwerg mitspielen. Wählt am Anfang einen beliebigen Helden. Dieser (und nur dieser) gilt als Kram, wenn eine Legendenkarte von "Kram" spricht. Ihr dürft zur Erinnerung den Namen dieses Helden mit einem Sternchen markieren.}


\end{bgcolor}


\begin{bgcolor}[color-DlH]
\section{Die letzte Hoffnung (2016)}


\subsection{Zweite Sonderfähigkeit: Höhlenkraft}

Steht der Zwerg auf einem Höhlenfeld, hat er +5 \hyperref[Bonusstärke]{Bonusstärke} und 3 Würfel, egal wie viele Willenspunkte er hat.
\supercite{DlH:Begleitheft,DlH11:Heldenkarte_Zwerg}

Das gilt auch, wenn der Höhlenplättchen-Platz auf dem Höhlenfeld mit einem roten X\supercite{DlH13:a} oder Sternchen\supercite{DlH15:a} abgedeckt ist.



\subsection{Legende 14, Der Meister des Trolls:}

\fan{Hat der Zwerg eine andere Sonderfähigkeit als "Nahrung", gilt bei der offiziellen Erschöpfungskarte "verlierst 8 Willenspunkte und erhältst keinen Stärkepunkt bzw. Willenspunkte" stattdessen "Solange du deine Genesungsaufgabe nicht erfüllt hast, darfst du deine erste Sonderfähigkeit nicht nutzen."}


\subsection{Mini-Erweiterung "Das Chaos":}

\fan{Hat der Zwerg die Sonderfähigkeit "Nahrung", verliert er bei Sonnenaufgang, wenn er keine Nahrung abgibt, 4 Willenspunkte (statt des Werts des großen grauen Würfels).}

\end{bgcolor}




\begin{bgcolor}[color-BB]
\section{Die Bonus-Box (2017/2024)}

\subsection{Weiße Ereigniskarte 11:}

Der Zwerg erhält das Zwergenseil auf Feld 71 umsonst, 
\supercite{BB:Ereigniskarte_11}
\fan{egal welche Sonderfähigkeit er hat}.

\end{bgcolor}


\begin{bgcolor}[color-DZ]
\section{Düstere Zeiten (2020)}

\subsection{Legende 1, Schatten über Cavern:}

Es muss ein "Kram" mitspielen.
\supercite{DZ:Begleitheft}
\fan{Es muss jedoch nicht zwingend ein \hyperref[Zwerg]{Zwerg} mitspielen.} 

\fan{Wählt am Anfang einen beliebigen Helden. Dieser (und nur dieser) gilt als Kram, wenn eine Legendenkarte von "Kram" spricht. Ihr dürft zur Erinnerung den Namen dieses Helden mit einem Sternchen markieren.}

\fan{Hat Kram keine Helmablage, kann niemand die Schildkrone tragen oder nutzen.}

\subsection{Spielvariante "Die Zwergentüren":}

Spielt ihr mit Kram, erhält er am Anfang die Schildkrone. 
\supercite{DZ:Begleitheft}
\fan{Da in dieser Fan-Spielvariante nicht alle \hyperref[Zwerg]{Zwerge} Kram sind und Kram kein \hyperref[Zwerg]{Zwerg} sein muss, wählt stattdessen am Anfang einfach, ob ihr mit der Schildkrone spielt. Falls ja, gebt sie einem beliebigen Helden, der eine Helmablage hat. Nur dieser kann die Schildkrone tragen und nutzen. Ihr dürft zur Erinnerung die Helmablage dieses Helden mit einem Sternchen markieren.}


\end{bgcolor}



\begin{bgcolor}[color-DfL]
\section{\fan{Das ferne Land (2025)}}

\subsection{\fan{Geister:}}

\fan{Steht der Zwerg auf oder angrenzend zu Geistern, kann er seine Sonderfähigkeit nicht nutzen.}

\fan{Es ist egal, ob sein Gegner bei oder angrenzend zu Geistern steht.}



\end{bgcolor}




\end{multicols*}








\chapter{9. Stinner (Stianna), der Seekrieger}
\label{Seekrieger}

\begin{multicols*}{2}\raggedcolumns


\begin{nobgcolor}

\section{Allgemein}

\subsection{\fan{Aufbau:}}

\fan{Der Seekrieger erhält den treuen Streifenmarder.}



\subsection{Heldentafel:}

Der Seekrieger hat Rang 9.

Seine offizielle Farbe ist Beige.

Er hat:

\bullet{} 2 Würfel bei 1 bis 13 Willenspunkten,

\bullet{} 3 Würfel bei 14 bis 20 Willenspunkten.




\subsection{\fan{Treuer Streifenmarder:}}

\fan{Der treue Streifenmarder ist ein kleiner rechteckiger Gegenstand, der von anderen Helden getragen werden darf. Nur der Seekrieger darf ihn 1x pro Tag einsetzen, vor einer Kampfrunde gegen einen Gegner (kein Endgegner) mit 2 oder mehr Würfeln. Dann wirft der Gegner nur in dieser Kampfrunde einen Würfel weniger und der Seekrieger darf diesen Würfel mit seinen Heldenwürfeln werfen. Klaut der treue Streifenmarder einem Gegner einen großen schwarzen oder großen grauen Würfel, kann der Seekrieger diesen nur anstatt all seiner Heldenwürfel werfen und nicht mit dem Trank der Hexe verdoppeln.}

\fan{Jeden Sonnenaufgang, wenn der Seekrieger den treuen Streifenmarder trägt, darf er einen halbvollen Trinkschlauch erhalten.}

\fan{Dies ist nur eine von vielen Möglichkeiten, Stinner außerhalb von "Die Reise in den Norden" zu spielen:}

\bullet{} \fan{"Fan-Stinner in anderen Andor-Teilen"}
\supercite{126785}








\subsection{Kein Krieger aus Andor:}

Spricht eine Regel oder Karte vom \hyperref[Krieger]{Krieger}, ist der Seekrieger nicht gemeint.




\subsection{\fan{Mehrere Seekrieger:}}

\fan{Die treuen Streifenmarder sind alle ununterscheidbar. Jeder Seekrieger kann jeden treuen Streifenmarder nutzen. Pro Kampfrunde kann insgesamt nur ein treuer Streifenmarder genutzt werden.}

\end{nobgcolor}



\begin{bgcolor}[color-DLvA]
\section{\fan{Die Legenden von Andor (2012)}}





\subsection{\fan{Ausscheiden:}}

\fan{Scheidet der Seekrieger aus (z.B. weil er in Cavern auf 0 Willenspunkte fiel), während er einen treuen Streifenmarder trägt, geht dieser aus dem Spiel.}

\subsection{\fan{Legende 5, Der Zorn des Drachens:}}

\fan{Bei der Drachenkampfkarte "Gegenstand in 2 Stärkepunkte tauschen" darf der Seekrieger den treuen Streifenmarder oder einen von ihm erhaltenen Trinkschlauch in Stärke tauschen.}

\end{bgcolor}


\begin{bgcolor}[color-DRidN]
\section{Die Reise in den Norden (2014)}

\subsection{Diebischer Streifenmarder:}

\fan{Der Seekrieger erhält keinen treuen Streifenmarder. Der diebische Streifenmarder, der als Teil der Gaben des Nordens ins Spiel kommt, hat keine besondere Verbindung zum Seekrieger und gibt keine Trinkschläuche.}



\subsection{Schiff:}

Der Seekrieger darf einmal pro Zug, vor oder nach seiner Aktion, wenn er an Bord des Schiffs ist, das Schiff auf ein angrenzendes Meeresfeld versetzen (\fan{auch über eine Eisbrücke des \hyperref[Eis-Dämon]{Eis-Dämons}}). Das kostet ihn keine Stunden.

Durch dieses Versetzen werden Effekte wie beim Stehenbleiben aktiviert (z.B. Wrackplättchen), auch wenn der Seekrieger das Schiff vor seiner Aktion versetzt und danach weiter bewegt.
\supercite{DRidN:FAQ}

\end{bgcolor}


\begin{bgcolor}[color-DlH]
\section{\fan{Die letzte Hoffnung (2016)}}


\subsection{\fan{Legende 15, Der vergiftete Geist:}}

\fan{Wird der Seekrieger enttarnt, darf er den treuen Streifenmarder und von ihm erhaltene Trinkschläuche in Stärke tauschen.}

\end{bgcolor}





\begin{bgcolor}[color-DeK]
\section{\fan{Die ewige Kälte (2022)}}

\subsection{\fan{Händler:}}

\fan{Der Seekrieger darf seinen treuen Streifenmarder und von ihm erhaltene Trinkschläuche bei Händlern tauschen.}

\end{bgcolor}



\begin{bgcolor}[color-DfL]
\section{\fan{Das ferne Land (2025)}}

\subsection{\fan{Geister:}}

\fan{Steht der Seekrieger bei oder angrenzend zu Geistern, kann er den treuen Streifenmarder nicht nutzen und erhält bei Sonnenaufgang keinen Trinkschlauch.}

\fan{Er darf Trinkschläuche nutzen, auch wenn er bei oder angrenzend zu Geistern steht.}

\fan{Für den Einsatz des Streifenmarders ist es egal, ob der Gegner bei oder angrenzend zu Geistern steht.} 


\end{bgcolor}



\end{multicols*}









\chapter{14. Thorn (Mairen), der Krieger}
\label{Krieger}


\begin{multicols*}{2}\raggedcolumns


\begin{nobgcolor}

\section{Allgemein}



\subsection{Aussprache:}


Der Name wird wie "Torn" gesprochen.
\supercite{33469}


\subsection{Heldentafel:}

Der Krieger hat Rang 14.

Seine offizielle Farbe ist Blau.

Er hat:

\bullet{} 2 Würfel bei 1 bis 6 Willenspunkten,

\bullet{} 3 Würfel bei 7 bis 13 Willenspunkten,

\bullet{} 4 Würfel bei 14 bis 20 Willenspunkten.





\subsection{Noch kein Ritter:}

Spricht eine Ereigniskarte vom Krieger, sind davon immer der Krieger und der \hyperref[Ritter]{\fan{Ritter}} betroffen.

Sprechen Regeln aus "Die letzte Hoffnung" vom Krieger, ist damit der \hyperref[Ritter]{\fan{Ritter}} gemeint, nicht der Krieger. Sprechen andere Regeln vom Krieger, ist damit der Krieger gemeint.





\subsection{Brunnen:}

Leert der Krieger einen Brunnen, erhält er 5 Willenspunkte (statt 3).


\subsection{Feuerwächterin:}

Ein brennendes Feuer der \hyperref[Feuerwächterin]{Feuerwächterin} gibt dem Krieger +2 \hyperref[Bonusstärke]{Bonusstärke} (statt +1).


\subsection{Wassermagierin:}

Der Krieger kann keine zusätzlichen Willenspunkte erhalten, wenn die \hyperref[Wassermagierin]{Wassermagierin} zwischen Brunnen/\fan{Quellen} teleportiert.

\end{nobgcolor}

\begin{bgcolor}[color-DRidN]
\section{Die Reise in den Norden (2014)}


\subsection{Nord-Brunnen:}

Leert der Krieger einen Nord-Brunnen, erhält er:
\supercite{DRidN:Begleitheft}

\bullet{} 5 Willenspunkte und 1 Gold (statt 3 und 1),

\bullet{} 5 Willenspunkte und 1 Trinkschlauch (statt 3 und 1),

\bullet{} 6 Willenspunkte (unverändert).

\end{bgcolor}


\begin{bgcolor}[color-DlH]
\section{\fan{Die letzte Hoffnung (2016)}}

\subsection{\fan{Quellen:}}

\fan{Leert der Krieger eine Quelle, erhält er 5 Willenspunkte, egal was auf dem Quellenplättchen steht.}


\subsection{\fan{Heldenkarten:}}

\fan{Der Krieger hat keine offiziellen Heldenkarten. Die offiziellen Heldenkarten des \hyperref[Ritter]{Ritters} passen nicht alle zu ihm.}



\subsection{\fan{Bronzene Ereigniskarte 239:}}

\fan{Der Krieger verliert den Wert der Quelle an Willenspunkten, nicht 5.}


\subsection{\fan{Bronzene Ereigniskarte 242:}}

\fan{Der Krieger erhält bei der 2er-Quelle 5 Willenspunkte, bei anderen Quellen den doppelten Quellenwert.}

\end{bgcolor}


\begin{bgcolor}[color-BB]
\section{Die Bonus-Box (2017/2024)}

\subsection{Legende 1, Der Angriff der Babaren:}

Ist auf einer Legendenkarte von "Krieger" die Rede, sind damit Barbaren-Krieger gemeint und nicht Thorn (Mairen).
\supercite{BB1:A}

\end{bgcolor}





\begin{bgcolor}[color-AG]
\section{Alte Geister (2018)}



\subsection{Legende 1, Die Spur des Drachen:}

Ist die Karte "F" (Version Brunnen) im Spiel, erhält der Krieger an einem Brunnen 8 Willenspunkte (statt 6).
\supercite{AG1:F_Brunnen}


\subsection{Legende 3, Die steinernen Drei:}

Leert der Krieger die Quelle auf Feld 37, erhält er nur das, was auf dem Quellenplättchen steht.
\supercite{AG:Begleitheft}

\end{bgcolor}




\begin{bgcolor}[color-DeK]
\section{Die ewige Kälte (2022)}

\subsection{Feuer:}

Ein brennendes Feuer auf dem Feld des Kriegers gibt ihm +2 \hyperref[Bonusstärke]{Bonusstärke} (statt +1).

\end{bgcolor}



\begin{bgcolor}[color-DfL]
\section{\fan{Das ferne Land (2025)}}

\subsection{\fan{Geister:}}

\fan{Steht der Krieger bei oder angrenzend zu Geistern, erhält er beim Leeren eines Brunnens nur 3 Willenspunkte und keine \hyperref[Bonusstärke]{Bonusstärke} durch ein Feuer auf der Heldentafel der \hyperref[Feuerwächterin]{Feuerwächterin}.}



\subsection{\fan{Ereigniskarte 8:}}

\fan{An diesem Tag erhält der Krieger bei Brunnen 3 Willenspunkte (statt 1).}

\end{bgcolor}




\end{multicols*}







\chapter{14. Thorn (Mairen), der \fan{Ritter}}
\label{Ritter}



\begin{multicols*}{2}\raggedcolumns




\begin{nobgcolor}

\section{Allgemein}


\subsection{Aussprache:}


Der Name wird wie "Torn" gesprochen.
\supercite{33469}



\subsection{Noch immer ein Krieger:}

Spricht eine Ereigniskarte vom Krieger, sind davon immer der \hyperref[Krieger]{Krieger} und der \fan{Ritter} betroffen.

Sprechen Regeln aus "Die letzte Hoffnung" vom Krieger, ist damit der \fan{Ritter} gemeint. Sprechen andere Regeln vom Krieger, ist der \fan{Ritter} damit nicht gemeint.




\subsection{Heldentafel:}

Der \fan{Ritter} hat Rang 14.

Seine offizielle Farbe ist Blau.

Er hat höchstens 19 Willenspunkte (statt 20).

Er hat 4 kleine rechteckige Ablagefelder (statt 3).

Er hat:

\bullet{} 2 Würfel bei 1 bis 4 Willenspunkten,

\bullet{} 3 Würfel bei 5 bis 14 Willenspunkten,

\bullet{} 4 Würfel bei 15 bis 19 Willenspunkten.






\subsection{Treues Pferd Ambra:}

Der \fan{Ritter} ist ein \hyperref[Reiter]{Reiter}.

\end{nobgcolor}



\begin{bgcolor}[color-DLvA]
\section{\fan{Die Legenden von Andor (2012)}}

\subsection{\fan{Trank der Hexe und Runensteine:}}

\fan{Der Ritter darf gleichzeitig drei Runensteine und einen Trank der Hexe tragen. Er kann den großen schwarzen Runenstein-Würfel aber nicht mit dem Trank der Hexe verdoppeln.}

\end{bgcolor}



\begin{bgcolor}[color-DlH]
\section{Die letzte Hoffnung (2016)}


\subsection{Zweite Sonderfähigkeit: Ambras Wut}

Der \fan{Ritter} hat in der ersten Kampfrunde jedes Kampfes +2 \hyperref[Bonusstärke]{Bonusstärke}.

\fan{Hat er 13 Stärkepunkte, ist die Bonusstärke nur +1.}

\fan{Hat er 14 Stärkepunkte, hat er keine Bonusstärke.}
\supercite{39829}

Ihr dürft die Summe seiner Stärkepunkte und dieser Bonusstärke durch einen weißen Holzstein markieren.\supercite{DlH:Begleitheft,DlH11:Heldenkarte_Krieger}

\end{bgcolor}


\end{multicols*}











\chapter{15. Bragor (Rhega), der Tarus}
\label{Tarus}

\begin{multicols*}{2}\raggedcolumns


\begin{nobgcolor}

\section{Allgemein}



\subsection{Heldentafel:}

Der Tarus hat Rang 15.

Seine offizielle Farbe ist Braun.

Er hat höchstens 19 Willenspunkte (statt 20).

Er hat keine Helmablage (statt 1).

Er hat kein großes Ablagefeld (statt 1).

Er hat:

\bullet{} 1 Würfel bei 1 bis 4 Willenspunkten,

\bullet{} 2 Würfel bei 5 bis 14 Willenspunkten,

\bullet{} 3 Würfel bei 15 bis 19 Willenspunkten.




\subsection{Speerkämpfer:}

Der Tarus ist ein \hyperref[Fernkämpfer]{Fernkämpfer}. Die Fernkampf-Regeln gelten aber nur, wenn er einen angrenzenden Gegner bekämpft, keinen Gegner auf demselben Feld.
\supercite{NH:Begleitheft}


\subsection{Kein Bogenschütze:}

Spricht eine Regel oder Karte von der \hyperref[Bogenschützin]{Bogenschützin}, ist der Tarus nicht gemeint.



\subsection{Wasser bringt Stärke:}

Leert der Tarus mit 15 oder mehr Willenspunkten ein Brunnen-/\fan{Quellen}plättchen, darf er 1 Stärkepunkt statt der Willenspunkte erhalten.



\subsection{Wassermagierin:}

Teleportiert die \hyperref[Wassermagierin]{Wassermagierin} zwischen Brunnen/\fan{Quellen}, kann der Tarus keinen Stärkepunkt statt Willenspunkten erhalten.


\end{nobgcolor}







\begin{bgcolor}[color-DRidN]
\section{Die Reise in den Norden (2014)}


\subsection{Nord-Brunnen:}

Wählt der Tarus beim Leeren eines Nord-Brunnens 1 Stärkepunkt statt der Willenspunkte, erhält er kein Gold und keinen Trinkschlauch.
\supercite{DRidN:Begleitheft}

\end{bgcolor}





\columnbreak



\begin{bgcolor}[color-AG]
\section{Alte Geister (2018)}


\subsection{Legende 1, Die Spur des Drachen:}

Ist die Karte "F" (Version Brunnen) im Spiel, darf der Tarus durch seine Sonderfähigkeit bei einem Brunnen 1 Stärkepunkt und 3 Willenspunkte erhalten (statt 6 Willenspunkten), nicht 2 Stärkepunkte.
\supercite{AG:Begleitheft}

\subsection{Legende 3, Die steinernen Drei:}

Leert der Tarus die Quelle auf Feld 37, kann er keinen Stärkepunkt statt Willenspunkten erhalten.
\supercite{AG:Begleitheft}

\fan{Fürst Hallgards Stärkemarker wird auch nach rechts verschoben, wenn der Tarus beim Leeres eines Brunnens bei Hallgard 1 Stärkepunkt anstelle der Willenspunkte wählt.}


\end{bgcolor}



\begin{bgcolor}[color-DeK]
\section{\fan{Die ewige Kälte (2022)}}

\subsection{\fan{Feuer:}}

\fan{Steht der Tarus bei Sonnenaufgang mit 15 oder mehr Willenspunkten bei einem brennenden Feuer, darf er 1 Stärkepunkt statt 5 Willenspunkten erhalten. }


\end{bgcolor}


\begin{bgcolor}[color-DfL]
\section{\fan{Das ferne Land (2025)}}

\subsection{\fan{Geister:}}

\fan{Steht der Tarus bei oder angrenzend zu Geistern, kann er beim Leeren eines Brunnens keinen Stärkepunkt statt Willenspunkten erhalten.}

\end{bgcolor}




\end{multicols*}











\chapter{22. Fenn (Fennah), der Fährtenleser}
\label{Fährtenleser}

\begin{multicols*}{2}\raggedcolumns


\begin{nobgcolor}

\section{Allgemein}


\subsection{Aufbau:}

Der Fährtenleser erhält die 3 Plättchen Rabe, Messer und Horn, alle aufgedeckt.



\subsection{Heldentafel:}

Der Fährtenleser hat Rang 22.

Seine offizielle Farbe ist Orange.

Er hat höchstens 10 Stärkepunkte (statt 14).

Er hat höchstens 14 Willenspunkte (statt 20).

Er hat 3 besondere Ablagefelder für Rabe, Messer und Horn.

Er hat:

\bullet{} 1 Würfel bei 1 bis 4 Willenspunkten,

\bullet{} 2 Würfel bei 5 bis 14 Willenspunkten.



\subsection{Rabe, Messer und Horn:}

Nur der Fährtenleser kann Rabe, Messer und Horn tragen. Sie gelten nicht als Gegenstände und können nicht abgelegt oder abgegeben werden.
\supercite{6952}

Jeden Sonnenaufgang werden diese Plättchen aufgedeckt.
\supercite{NH:Begleitheft}







\subsection{Der Rabe Morar von Ex-Freundin Neja:}

Trägt der Fährtenleser den Raben aufgedeckt, darf er ihn als freie Handlungsmöglichkeit zudecken. Dann wählt er ein Spielplanfeld (kein Buchstabenfeld der Legendenleiste) und deckt \fan{beliebig viele} \hyperref[Aufdecken]{\fan{von nah aufdeckbare}} Plättchen auf diesem Feld auf.
\supercite{NH:Begleitheft}
Aktiviert werden diese Plättchen dadurch nicht. Plättchen in Stapeln dürfen auch aufgedeckt werden, ihre Reihenfolge bleibt aber gleich.
\supercite{BL_BdM:A3_en}


\subsection{Das Messer von Vater Vann:}

Trägt der Fährtenleser das Messer aufgedeckt, darf er es im Kampf direkt nach seinem Würfeln zudecken. Dann wirft er einen seiner geworfenen Heldenwürfel neu.
\supercite{NH:Begleitheft}
\fan{Das geht, auch wenn der Würfel zuvor bereits durch einen anderen Effekt neu geworfen wurde.} Im \hyperref[Fernkämpfer]{Fernkampf} kann er mit dem Messer nur den zuletzt geworfenen Würfel neu werfen,\supercite{DRidN:Begleitheft} \fan{bei dem er aufhörte}.


\fan{Wirft der Fährtenleser einen Gegnerwürfel gemeinsam mit seinen Heldenwürfeln (z.B. durch den Streifenmarder in "Die Reise in den Norden" oder Meres in "Die steinernen Drei"), kann das Messer diesen nicht neu werfen.}

Wirft der Fährtenleser statt seiner Heldenwürfel einen Ersatzwürfel (z.B. den großen weißen Würfel der \hyperref[Hüterin]{Hüterin}), kann das Messer diesen nicht neu werfen.
\supercite{5460}

\columnbreak

\subsection{Das Signalhorn von Händler Nader:}

Trägt der Fährtenleser das Horn aufgedeckt, darf er es am Anfang einer Kampfrunde, in der er mitkämpft, zudecken. Dann haben alle Helden in dieser Kampfrunde ihre höchste Anzahl Heldenwürfel, egal wie viele Willenspunkte sie haben. 
\hyperref[Fernkämpfer]{Fernkämpfer} würfeln ihre Würfel weiterhin nacheinander.
\supercite{NH:Begleitheft}

\fan{Der \hyperref[Feuerkrieger]{Feuerkrieger} und \hyperref[Steppennomade]{Steppennomade} erhalten Würfel nicht zurück, die sie für ihre Sonderfähigkeit abgaben.}

\fan{Das Horn hat keinen Effekt auf die \hyperref[Runenmeisterin]{Runenmeisterin} und den \hyperref[Seher]{Seher}.}







\subsection{\fan{Mehrere Fährtenleser:}}

\fan{Ein Fährtenleser kann seinen Raben, sein Messer und sein Horn keinem anderen Fährtenleser geben. Ein Fährtenleser kann mit seinem Messer nur einen eigenen Würfel neu werfen.}



\end{nobgcolor}


\begin{bgcolor}[color-DLvA]
\section{Die Legenden von Andor (2012)}

\subsection{Ausscheiden:}

\fan{Scheidet der Fährtenleser aus (z.B. weil er in Cavern auf 0 Willenspunkte fiel), gehen Rabe, Messer und Horn aus dem Spiel.}


\subsection{Legende 3, Die Tage des Widerstands:}

\fan{Nutzt Varkur die Dunkle Magie "Die Schwächung der Helden", kann der Fährtenleser das Horn nicht im Endkampf nutzen.}


\subsection{Legende 5, Der Zorn des Drachens:}

Bei der Drachenkampfkarte "Gegenstand in 2 Stärkepunkte tauschen" kann der Fährtenleser Rabe, Messer und Horn nicht in Stärke tauschen.
\supercite{6952}



\subsection{Bonus-Legende "Die Befreiung der Mine":}

Der Fährtenleser darf Messer und Horn im Kampf gegen Varkur nutzen, auch wenn im Kampf gegen ihn keine Gegenstände genutzt werden können (d.h. wenn Varkur bei der Karte "Varkurs Versteck" in der Waffenkammer auftaucht).
\supercite{6952}


\subsection{Ereigniskarte "Geheimer See" 9:}

\fan{Der Fährtenleser kann keinen Würfel mit seinem Messer neu werfen.}





\end{bgcolor}


\begin{bgcolor}[color-StSch]
\section{Der Sternenschild (2013)}

\subsection{Andorische Flöte:}

\fan{Nutzt der Fährtenleser die andorische Flöte, kann er den Würfel nicht mit seinem Messer neu werfen.}

\end{bgcolor}



\columnbreak



\begin{bgcolor}[color-DlH]
\section{\fan{Die letzte Hoffnung (2016)}}


\subsection{\fan{Legende 15, Der vergiftete Geist:}}

\fan{Wird der Fährtenleser enttarnt, kann er Rabe, Messer und Horn nicht in Stärke tauschen.}

\end{bgcolor}



\begin{bgcolor}[color-AG]
\section{Alte Geister (2018)}

\subsection{Legende 2, Der Hexer aus Andor:}

\fan{Nutzt der Fährtenleser das Horn gegen Siantari auf Feld 30, dürfen alle Helden aus bis zu ihrer höchsten Anzahl Heldenwürfel wählen, sie müssen aber weiterhin 1 Willenspunkt pro gewähltem Würfel abgeben.}

\end{bgcolor}




\begin{bgcolor}[color-DZ]
\section{Düstere Zeiten (2020)}

\subsection{Shans Fluch:}

\fan{Würfel, die auf einem Schattenplättchen liegen, können nicht geworfen werden, auch wenn der Fährtenleser sein Horn nutzte.}

\end{bgcolor}




\begin{bgcolor}[color-DeK]
\section{\fan{Die ewige Kälte (2022)}}

\subsection{\fan{Rabe, Messer und Horn:}}

\fan{Der Steppennomade kann Rabe, Messer und Horn nicht bei Händlern tauschen.}

\end{bgcolor}


\begin{bgcolor}[color-DfL]
\section{\fan{Das ferne Land (2025)}}

\subsection{\fan{Pilzvergiftung:}}

\fan{Deckt der Rabe den "-3 Willenspunkte" Pilz auf, geht der Rabe für den Rest der Legende aus dem Spiel.}

\subsection{\fan{Geister:}}

\fan{Steht der Fährtenleser bei oder angrenzend zu Geistern, kann er Rabe, Messer und Horn nicht nutzen.}

\fan{Rabe, Messer und Horn werden bei Sonnenaufgang aufgedeckt, auch wenn der Fährtenleser steht bei oder angrenzend zu Geistern steht. Der Rabe darf Plättchen bei oder angrenzend zu Geistern aufdecken. Mit dem Horn dürfen auch Helden bei oder angrenzend zu Geistern ihre höchste Anzahl Würfel werfen, solange der Fährtenleser selbst nicht bei oder angrenzend zu Geistern steht.}

\end{bgcolor}


\end{multicols*}







\chapter{23. Drukil (Drukia), der Hautwandler}
\label{Hautwandler}


\begin{multicols*}{2}\raggedcolumns


\begin{nobgcolor}

\section{Allgemein}

\subsection{Aufbau:}

Der Hautwandler beginnt mit der Heldenfigur "Mensch". Legt die Figur "Bär" bereit.




\subsection{Heldentafel:}

Der Hautwandler hat Rang 23.

Seine offizielle Farbe ist Braun.

Er hat höchstens 6 Stärkepunkte (statt 14).

Er hat höchstens 14 Willenspunkte (statt 20).

Er hat:

\bullet{} 1 Würfel bei 1 bis 4 Willenspunkten,

\bullet{} 2 Würfel bei 5 bis 14 Willenspunkten.




\subsection{Hautwandel:}

Beim Brunnen/Quellen-Symbol des Sonnenaufgang-Felds, wenn der Hautwandler auf einem Waldfeld steht, tauscht er seine Heldenfigur: Von Mensch zu Bär bzw. von Bär zu Mensch.
\supercite{46300}

Steht er nicht auf einem Waldfeld oder wird das Brunnen/Quellen-Symbol nicht ausgeführt (z.B. wegen des Sternenschilds in "Der Sternenschild" oder der Kraft des roten Mondes in "Alte Geister"), bleibt seine Heldenfigur gleich.



\subsection{Bär:}

Als Bär hat der Hautwandler \hyperref[Bonusstärke]{Bonusstärke}:

\bullet{} +2, wenn er 2 Stärkepunkte hat,

\bullet{} +3, wenn er 1, 3 oder 5 Stärkepunkte hat,

\bullet{} +4, wenn er 4 oder 6 Stärkepunkte hat.

Die Summe seiner Stärkepunkte und dieser \hyperref[Bonusstärke]{Bonusstärke} wird oberhalb seiner Stärkeleiste angezeigt.


Der Bär kann nur Gegenstände tragen, die ihm explizit erlaubt werden (auch keine Plättchen auf der Goldablage\supercite{DH:FAQ} und keine Holzstämme\supercite{46402}\fan{/Eisen}). Was er nicht tragen kann, wird sofort auf sein Feld gelegt.





\subsection{Reiner Instinkt / Erinnerung ans Menschsein:}

Hat der Bär 1 bis 9 Willenspunkte, gilt folgendes:

Er kann nur die Aktionen "Laufen" und "Passen" wählen.
\supercite{46612,DH:Bär}
\fan{Fällt er in einer anderen Aktion auf 9 oder weniger Willenspunkte, muss er sie sofort beenden.}

\bullet{} Er kann nur Plättchen aufdecken oder aktivieren, die ihm explizit erlaubt werden. Er kann auch keine Plättchen aufdecken oder aktivieren, deren Aktivierung sonst obligatorisch wäre (z.B. Nebelplättchen). Er darf Brunnen/Quellen leeren.
\supercite{DH:Bär}

\bullet{} Er kann keine mitbewegbaren Plättchen (z.B. Bauern) oder Figuren (z.B. Merrik) mitbewegen\supercite{AG:Begleitheft,DZ:Begleitheft}.

\bullet{} Von Kreaturen entfernbare Plättchen und Figuren (z.B. Bauern) gehen aus dem Spiel, wenn der Bär auf ihrem Feld steht. Das Feld des Bären darf aber gefahrlos überquert werden.
\supercite{DH:Bär,DH:Begleitheft}
Ausnahme: Edelsteine gehen durch den Bären nicht aus dem Spiel.

\bullet{} Betritt er ein Feld mit (goldenen \fan{oder Steppenvolk-}) Schilden \fan{oder steht er dort}, ist die Legende verloren. 
\supercite{DH:Bär,DZ:Begleitheft}

\bullet{} Steht ein Held auf dem Feld des Bären, verliert der Held 1 Stärkepunkt. Trägt er einen Schild, darf er eine Seite verbrauchen, um sich davor zu schützen, \fan{auch wenn er seinen Tag bereits beendet hat}. \fan{Danach ist er sicher, bis er oder der Bär dieses Feld verlässt.}
Das Feld des Bären darf immer gefahrlos überquert werden, solange man nicht dort stehen bleibt.
\supercite{DH:Bär,DH:Begleitheft,46300} 


\subsection{Der Ruf des Waldes:}

Hat der Bär 1 bis 4 Willenspunkte, darf er zwar immer noch entscheiden, welche freien Handlungsmöglichkeiten er ausführt, aber seine Aktionen kann er nicht mehr frei wählen.  

Ist er am Zug und steht nicht auf einem Waldfeld, muss er 1 Schritt in Richtung des nächstgelegenen Waldfelds mit der kleinsten Feldzahl laufen. Dafür darf er auch Eisbrücken des \hyperref[Eis-Dämon]{Eis-Dämons} nutzen. \fan{Dafür darf er Verbindungen zwischen zwei Feldern überqueren, die nur von Helden mit genug Willenspunkten überquert werden können (z.B. Sprungfelder in "Die letzte Hoffnung" oder der steile Felsen in "Das ferne Land"), egal wie viele Willenspunkte er hat.}

Ist er am Zug ist und steht auf einem Waldfeld oder kann nicht näher an eines laufen, muss er passen. 

Würde er eine Überstunde machen, muss er stattdessen seinen Tag beenden.
\supercite{DH:Bär}

\fan{Fällt er in einer Aktion auf 4 oder weniger Willenspunkte, muss er sie sofort beenden.}










\subsection{Ambra und Mamoru:}

Der Bär kann nicht von einem \hyperref[Reiter]{Reiter} mitbewegt werden.
\supercite{DH:Begleitheft}








\subsection{\fan{Mehrere Hautwandler:}}

\fan{Ein Bär verliert keine Stärkepunkte durch einen anderen Bären auf seinem Feld. Ein Hautwandler als Mensch kann aber durchaus Stärkepunkte durch einen Bären auf seinem Feld verlieren.}



\end{nobgcolor}




\columnbreak



\begin{bgcolor}[color-DLvA]
\section{Die Legenden von Andor (2012)}





\subsection{Waldfelder:}

Als Waldfeld gelten:

\bullet{} In Andor (Spielplan-Vorderseite) die Felder 22–25, 47–60, 62 und 63.
\supercite{232}

\bullet{} In Cavern (Spielplan-Rückseite) die Felder 41–47, 49–52, 54–63, 65 und 70.
\supercite{17913}




\subsection{Bär:}

Der Bär kann Felder mit Geröllplättchen nicht betreten.

Feld 0 von Cavern (Spielplan-Rückseite) darf er gefahrlos betreten.







\end{bgcolor}




\begin{bgcolor}[color-StSch]

\section{Der Sternenschild (2013)}

\subsection{Blaue Lichter:}

\fan{Müssen die Helden eine Anzahl blaue Lichter sammeln, um die Karte "Sternenlicht" zu aktivieren, darf der Bär blaue Lichter aufnehmen und tragen.}


\end{bgcolor}



\begin{bgcolor}[color-DRidN]
\section{\fan{Die Reise in den Norden (2014)}}






\subsection{\fan{Waldfelder}:}

\fan{Als Waldfeld gelten:}

\bullet{} \fan{Auf den Nebelinseln (Spielplan-Vorderseite) die Felder 52, 55, 57, 59, 60, 76, 77 und 114,}

\bullet{} \fan{In Hadria (Spielplan-Rückseite) die Felder 147 und 152.}





\subsection{\fan{Bär:}}

\fan{Der Bär darf Bootsstrecken nutzen. Das gilt für den "Ruf des Waldes" als 1 Schritt und kostet 2 Stunden.}

\fan{Er darf das Schiff betreten und verlassen, wenn es an ein Landfeld angrenzt. Er kann nicht segeln oder Schiffsausbauten nutzen. Die Legende ist nie verloren dadurch, dass er an Bord des Schiffes ist.}

\fan{Hat er 10 bis 14 Willenspunkte, darf er vom Schiff aus kämpfen und dabei Schiffsausbauten nutzen.}






\subsection{\fan{Reiner Instinkt / Erinnerung ans Menschsein:}}

\fan{Hat der Bär 1 bis 9 Willenspunkte, gilt folgendes:}

\bullet{\fan{Steht das Schiff nicht angrenzend an ein Landfeld, während er an Bord ist, muss er passen.}}

\bullet{} \fan{Da er keine Schneeplättchen aktivieren kann, muss er nicht stehen bleiben, wenn er ein Feld mit Schneeplättchen betritt.}

\bullet{} \fan{Stehen ein Held und der Bär an Bord des Schiffs, verliert der Held 1 Stärkepunkt. Trägt er einen Schild, darf er eine Seite verbrauchen, um sich davor zu schützen, auch wenn er seinen Tag bereits beendet hat. Danach ist er sicher, bis er oder der Bär von Bord geht. Ein Held darf gefahrlos an Bord gehen und segeln, wenn er in derselben Aktion wieder von Bord geht.}






\subsection{\fan{Legende 8, Ein Sturm zieht auf:}}

\fan{Ist das Schiff nach einem Unwetterplättchen auf einem nicht an Land angrenzenden Meeresfeld und ist der Bär der einzige Held an Bord, werft sofort einen roten Würfel und versetzt das Schiff auf dieses Feld.}

\end{bgcolor}




\begin{bgcolor}[color-DlH]
\section{Die letzte Hoffnung (2016)}


\subsection{Heldentafel:}

Der Hautwandler hat ein besonderes Ablagefeld für seinen Armreif.




\subsection{Zweite Sonderfähigkeit: Armreif}

Der Hautwandler erhält den kleinen runden Gegenstand Armreif, den nur Hautwandler tragen können.

Er darf als eigene Aktion für 1 Stunde den Armreif abgeben, um seine Heldenfigur zu wechseln, von Mensch zu Bär bzw. von Bär zu Mensch.
\supercite{DH11:Heldenkarte_Hautwandler}

\fan{Spielt ihr mit mehreren Hautwandlern: Alle Armreife sind ununterscheidbar. Alle Hautwandler dürfen alle Armreife tragen, aber höchstens einen auf einmal. Ein Hautwandler darf im Laufe einer Legende mehrere Armreife nutzen.}



\subsection{Waldfelder:}

Als Waldfeld gelten:

\bullet{} Im Grauen Gebirge (Spielplan-Vorderseite) die Felder 220, 227, 235, 246 und 256,

\bullet{} In Krahd (Spielplan-Rückseite) nur Feld 340.
\supercite{DH:Begleitheft}




\subsection{Bär:}

Der Bär darf Apfelnüsse, Sternkraut, \fan{Mondbeeren} und seinen Armreif aufnehmen, tragen, nutzen, \fan{ablegen und tauschen}.
\supercite{DH:Bär}

\subsection{Reiner Instinkt / Erinnerung ans Menschsein:}

Hat der Bär 1 bis 9 Willenspunkte, gilt folgendes:

\bullet{} \fan{Höhlenwichte und Sporne halten ihn nicht fest.} Skelette, \fan{Wachtrolle und Krahder} können ihn weiterhin verfolgen und festhalten.
\supercite{46529}


\subsection{Legende 14, Der Meister des Trolls:}

\fan{Für die offizielle Erschöpfungskarte "gleich viele Stärkepunkte" gilt stattdessen "Solange du deine Genesungsaufgabe nicht erfüllt hast, hat der Bär keine \hyperref[Bonusstärke]{Bonusstärke}."}

\fan{Spielt ihr mit mehreren Hautwandlern, betreffen die offiziellen Erschöpfungskarten nur dich selbst, keine anderen Hautwandler.}


\subsection{Legende 15, Der vergiftete Geist:}

\fan{Betritt der Bär mit 1 bis 9 Willenspunkten das Lager oder steht dort, ist die Legende ausnahmsweise nicht verloren, sondern es geschieht nichts.}


\subsection{Legende 17, Die letzte Hoffnung:}

\fan{Nur Bauern mit der Bauern-Seite oben gehen aus dem Spiel, wenn der Bär mit 1 bis 9 Willenspunkten auf ihrem Feld steht. Mit der Gitterseite oben bleiben sie im Spiel.}

Nachdem der Hautwandler seine offizielle Hoffnungskarte aktivierte, kann er nicht mehr zum Menschen werden, auch nicht mit dem Armreif.
\supercite{92358}

\end{bgcolor}



\columnbreak




\begin{bgcolor}[color-AG]
\section{Alte Geister (2018)}

\subsection{Bär:}

Der Bär darf Rietgrasblüten aufnehmen, tragen, nutzen, \fan{ablegen und tauschen}. 
\supercite{AG:Begleitheft}


\subsection{Reiner Instinkt / Erinnerung ans Menschsein:}

Hat der Bär 1 bis 9 Willenspunkte, gilt folgendes:

\bullet{} Betritt er das Lager der Tulgori auf Feld 18 \fan{oder steht er dort}, ist die Legende verloren.
\supercite{AG:Begleitheft}

\bullet{} Er kann Felder mit Eisplättchen nicht betreten.
\supercite{AG:Begleitheft}


\subsection{Arbaks:}

Arbaks nehmen den Bären als Waldbewohner war. Er darf Felder mit oder angrenzend zu Arbaks betreten und passieren, ohne anhalten zu müssen, Willenspunkte zu verlieren oder aus dem Spiel zu scheiden. 

Er darf eine Rietgrasblüte nutzen, um einen Arbak auf oder angrenzend zu seinem Feld hinzulegen und 1 Stärkepunkt zu erhalten.

Verwandelt der Hautwandler sich angrenzend zu einem Arbak zum Menschen, gilt das nicht als Betreten des Feldes und er muss keine Willenspunkte abgeben.

Verwandelt der Hautwandler sich auf dem Feld eines Arbaks zum Menschen, scheidet er aus, außer er kann sofort eine Rietgrasblüte einsetzen.
\supercite{AG:Begleitheft}

\subsection{Maasavi:}

Auch der Bär wird von Maasavi festgehalten.
\supercite{AG:Begleitheft}


\subsection{Casamatuc:}

\fan{Hat der Hautwandler den Casamatuc eingesetzt, wird dieser auf seine Willensleiste gelegt und bleibt bis zum Legendenende dort liegen, egal welche Gestalt der Hautwandler hat.}


\subsection{Legende 1, Die Spur des Drachen:}

\fan{Hat der Bär 10 bis 14 Willenspunkte, darf er die Aktion "Gehöft reparieren" nutzen. Bauernplättchen auf vollständig reparierten Gehöften sind vom Bären sicher, egal wie viele Willenspunkte er hat.}

\subsection{Bonus-Legende "Die vierte Statue":}

\fan{Hat der Bär 10 bis 14 Willenspunkte, darf er mit dem Casamatuc die Tiefminen besuchen.}

\end{bgcolor}





\columnbreak

\begin{bgcolor}[color-DZ]
\section{Düstere Zeiten (2020)}



\subsection{Reiner Instinkt / Erinnerung ans Menschein:}

Hat der Bär 1 bis 9 Willenspunkte, gilt folgendes:

\bullet{} Bauern auf Harthalts Feld \fan{und Bauern mit Sternchen (noch nicht rekrutiert)} bleiben im Spiel, auch wenn der Bär auf ihrem Feld steht.
\supercite{DZ:Begleitheft}

\bullet{} \fan{Legende 1, der Schatten über Cavern, ist verloren, wenn er auf dem Feld des verletzten Mart steht. Sein Feld kann aber gefahrlos überquert werden.}



\subsection{Der Ruf des Waldes:}

Hat der Bär 1 bis 4 Willenspunkte, darf er das Lager der Trolle betreten. Stellt ihn dann auf ein freies Fass, sonst ist die Legende verloren. Steigt er auf 5 oder mehr Willenspunkte, während er auf einem Fass steht, versetzt ihn auf Feld 38. 
\supercite{DZ:Begleitheft}
\fan{Rekas Gift hat keine Wirkung auf ihn, auch wenn er auf einem Fass steht.}





\subsection{Zwergentüren:}

\fan{Der Bär darf Zwergentüren nutzen, egal wie viele Willenspunkte er hat, obwohl er sie mit 1 bis 9 Willenspunkten nicht öffnen kann.}




\subsection{Schwarzes Einhorn:}

Der Hautwandler darf das Schwarze Einhorn nur zähmen, wenn er \fan{am Ende des Sonnenaufgangs} ein Mensch ist.
\supercite{DZ:Begleitheft}
\fan{Beendet er seinen Tag als Einhornreiter auf einem Waldfeld, muss er sofort absteigen.}






\subsection{Legende 2, Fremde Heimat:}

\fan{Der Bär kann die Aktion "Bauern rekrutieren" nur nutzen und das Boot nur fahren, wenn er 10 bis 14 Willenspunkte hat.}

\fan{Der Bär darf das Boot betreten, verlassen und von einem anderen Helden im Boot gefahren werden, egal wie viele Willenspunkte er hat. Das Boot gilt für den Stärkepunkte-Verlust als eigenes Feld. Das Boot gilt nie als Waldfeld, auch wenn es auf ein Waldfeld zeigt.}




\subsection{Legende 3, Der letzte Schatten:}

Hat der Bär 1 bis 4 Willenspunkte, darf er ein Feld mit Schattenplättchen betreten. Stellt ihn dann auf das Schattenplättchen. Er kann das Feld erst wieder verlassen, wenn er 5 oder mehr Willenspunkte hat oder das Schattenplättchen entfernt wurde. Solange er dort steht, überspringen Kreaturen dieses Feld und werden auf das nächste freie Schattenfeld gestellt.
\supercite{DZ:Begleitheft}


\subsection{Bonus-Legende "Der Marsch der Trolle":}

\fan{Ein vom Bären besetztes Fass zählt nicht zum Entfernen von Geröllplättchen. Steigt er auf einem Fass auf 5 oder mehr Willenspunkte, während die Brücke blockiert ist, stellt ihn auf Feld 42.}

\fan{Der Bär wird nie auf ein Schattenplättchen gestellt.}

\fan{Würde ein Bauernplättchen auf dem Feld des Bären mit 1 bis 9 Willenspunkten ins Spiel kommen, kommt es stattdessen auf das kreaturenfreie Feld mit der nächstkleineren Feldzahl.}

\end{bgcolor}



\begin{nobgcolor}

\section{Magische Helden (2021)}

\subsection{Spielvariante "Die Mondbeeren":}

\fan{Der Bär darf Mondbeeren erhalten, aufnehmen, tragen, nutzen, ablegen und tauschen.}


\end{nobgcolor}




\begin{bgcolor}[color-DeK]
\section{\fan{Die ewige Kälte (2022)}}

\subsection{\fan{Waldfelder:}}

\fan{Als Waldfeld gelten:}

\bullet{} \fan{Im Land der Steppe (Spielplan-Vorderseite) die Felder 406, 408 und 410,}

\bullet{} \fan{In Andor (Spielplan-Rückseite) die Felder 22–25, 47–60, 62 und
63.}


\subsection{\fan{Hautwandel:}}

\fan{Beim Wärmendes-Feuer-Symbol des Sonnenaufgang-Felds, wenn der Hautwandler auf einem Waldfeld steht, tauscht er seine Heldenfigur.}


\subsection{\fan{Bär:}}

\fan{Der Bär kann ein Seefeld nie betreten, egal ob ein Eisplättchen darauf liegt.}

\fan{Ausnahme: Sollte er auf einem Seefeld oder Feld 460 stehen, darf er Seefelder auf einem kürzesten Weg zum Ufer betreten, egal ob dort Eisplättchen liegen.}

\fan{Der Bär kann Eisplättchen weder aktivieren noch aufdecken und muss auf Feldern mit Eisplättchen nicht stehen bleiben.}

\fan{Der Bär darf Steppenkräuter (Blaubachbeeren, weiße Warzwurzeln, rotes Blutkraut und schwarze Nachtstacheln) aufnehmen, tragen, für die "Kleine Kräuterkunde" (+2 Willenspunkte) nutzen, ablegen und tauschen.}

\fan{Der Bär darf Zielmarker aufdecken, aufnehmen und tragen.}

\fan{Die "Große Kräuterkunde" kann der Bär nie nutzen.}

\fan{Er kann erloschene Feuer nur entfachen, wenn er 10 bis 14 Willenspunkte hat, darf aber immer Willenspunkte von einem brennenden Feuer erhalten.}

\end{bgcolor}





\columnbreak


\begin{bgcolor}[color-DfL]
\section{\fan{Das ferne Land (2025)}}



\subsection{\fan{Waldfelder:}}

\fan{Als Waldfeld gelten:}

\bullet{} \fan{In Azturien (Spielplan-Vorderseite) die Felder 500, 501, 502, 506, 508, 513, 514, 516, 517, 519, 521, 522, 535, 539–541, 551, 553, 556, 557 sowie 559–562.}

\bullet{} \fan{Im Land der Steppe (Spielplan-Rückseite) die Felder 405, 412, 413, 415, 416, 424 und 443.}


\subsection{\fan{Bär:}}

\fan{Der Bär darf Kräuter und Pilze aufnehmen, tragen, nutzen, ablegen und tauschen.}



\subsection{\fan{Reiner Instinkt / Erinnerung ans Menschsein:}}

\fan{Hat der Bär 1 bis 9 Willenspunkte, gilt folgendes:}

\bullet{} \fan{Betritt er das Feld des Steppenvolk-Plättchens oder steht dort, ist die Legende verloren. Das Steppenvolk darf sein Feld aber gefahrlos überqueren.}

\bullet{} \fan{Sporne halten ihn nicht fest.}


\subsection{\fan{Geister:}}

\fan{Steht der Bär bei oder angrenzend zu Geistern, hat er keine \hyperref[Bonusstärke]{Bonusstärke}. Alle Regeln zum Hautwandel und den anderen Bäreneffekten gelten, auch wenn er bei oder angrenzend zu Geistern steht.}

\end{bgcolor}







\end{multicols*}








\chapter{24. Avida (Avid), die Heilerin}
\label{Heilerin}


\begin{multicols*}{2}\raggedcolumns


\begin{nobgcolor}

\section{Allgemein}




\subsection{Heldentafel:}

Die Heilerin hat Rang 24.

Ihre offizielle Farbe ist Blau.

Sie hat:

\bullet{} 1 Würfel bei 1 bis 6 Willenspunkten,

\bullet{} 2 Würfel bei 7 bis 20 Willenspunkten.






\subsection{Heilkraft:}

Hat die Heilerin 7 oder mehr Willenspunkte, darf sie in jeder Kampfrunde (nicht im \hyperref[Fernkämpfer]{Fernkampf}\supercite{DfL:Begleitheft}), direkt nach ihrem Würfeln einen ihrer nicht zu ihrem Kampfwert addierten Würfel wählen und \fan{bis zu} so viele Willenspunkte auf die kämpfenden Helden verteilen (auch auf Helden, die durch \hyperref[Fernkämpfer]{Fernkampf} mitkämpfen).

\fan{Der gewählte Würfel kann nicht auf andere Arten (z.B. mit einem Helm \fan{oder Eisernen Handschuh in "Die letzte Hoffnung"}) zu ihrem Kampfwert addiert werden und kann nicht mit dem Trank der Hexe verdoppelt werden}.

Nach der Heilerin würfelnde Helden können durch von ihr verteilte Willenspunkte mehr Würfel zum Werfen haben.
\supercite{DfL:Begleitheft}



\fan{Zur Erinnerung: Helden müssen jede Kampfrunde einen ihrer geworfenen Würfel zu ihrem Kampfwert addieren, aber sie dürfen wählen, welchen.} Es muss nicht der höchste geworfene Würfel sein.
\supercite{127533}







\subsection{Kampfkraft:}

Hat die Heilerin 14 oder mehr Willenspunkte, darf sie im Kampf (nicht im \hyperref[Fernkämpfer]{Fernkampf}\supercite{DfL:Begleitheft}) zwei gleiche Würfelwerte zu ihrem Kampfwert addieren.
\fan{Diese Fähigkeit kann nicht mit dem Trank der Hexe kombiniert werden.}



\subsection{\fan{Bonuswürfel:}}

\fan{Wirft die Heilerin im Kampf einen Gegnerwürfel gemeinsam mit ihren Heldenwürfeln (z.B. durch den Streifenmarder in "Die Reise in den Norden" oder Meres in "Die steinernen Drei"), darf sie durch ihre Kampfkraft zwei gleiche Würfelwerte addieren (auch von Gegnerwürfeln), aber nicht 3 oder mehr gleiche Würfelwerte.}



\end{nobgcolor}







\begin{bgcolor}[color-DfL]
\section{Das ferne Land (2025)}

\subsection{Geister:}

Steht die Heilerin bei oder angrenzend zu Geistern, kann sie im Kampf ihre Heilkraft und Kampfkraft nicht nutzen. \fan{Sofern sie selbst nicht bei oder angrenzend zu Geistern steht, darf sie Willenspunkte auch auf Helden verteilen, welche bei oder angrenzend zu Geistern stehen.}

\end{bgcolor}


\end{multicols*}











\chapter{24. Santa Gor, der Weihnachts-Gor}
\label{Weihnachts-Gor}

\begin{multicols*}{2}\raggedcolumns




\begin{nobgcolor}

\section{Allgemein}

\subsection{Heldentafel:}

Der Weihnachts-Gor hat Rang 24.

Seine offizielle Farbe ist Grün.

Er hat höchstens 11 Stärkepunkte (statt 14).

Er hat höchstens 11 Willenspunkte (statt 20).

Er hat 2 kleine rechteckige Ablagefelder (statt 3).

Er hat keine Helmablage (statt 1).

Er hat:

\bullet{} 1 Würfel bei 1 bis 5 Willenspunkten,

\bullet{} 2 Würfel bei 6 bis 11 Willenspunkten.



\subsection{Geschenke bescheren:}

Bleibt der Weihnachts-Gor auf einem Feld mit Gebäude stehen, darf er einen Gegenstand von der Ausrüstungstafel auf das Feld legen (keinen Trank der Hexe).

Der Weihnachts-Gor kann einen bescherten Gegenstand erst tragen, nachdem er von einem anderen Helden aufgenommen wurde.

Der Weihnachts-Gor kann jedes Feld nur einmal pro Legende beschenken. Nutzt Markierungsplättchen (z.B. Sternchen), um euch zu merken, wo der Weihnachts-Gor bereits Geschenke bescherte, und welche Geschenke bereits aufgenommen wurden.




\subsection{Kein Halbskral:}

Anders als der \hyperref[Halbskral]{Halbskral} wird der Weihnachts-Gor nicht von Kreaturen übersprungen und kann im Kampf keine gleichen Würfelwerte addieren.



\subsection{Neugier:}

\fan{Der Weihnachts-Gor darf \hyperref[Aufdecken]{von nah aufdeckbare} Plättchen auf seinem Feld aufdecken, auch wenn er sie nicht tragen kann.}



\subsection{\fan{Mehrere Weihnachts-Gors:}}

\fan{Jedes Feld kann nur einmal beschenkt werden, nicht von verschiedenen Weihnachts-Gors.}


\fan{Spielt ihr mit mehreren Weihnachts-Gors, erhält zu Beginn der Legende einer von ihnen das kleine runde Plättchen "Geschenkesack" (Sternchen). Dieses Plättchen wird auf der Goldablage getragen und darf anderen Weihnachts-Gors, die auf demselben Feld stehen, übergeben werden. Es muss immer ein Weihnachts-Gor den Geschenkesack tragen. Er kann nicht per Falke verschickt werden.}

\fan{Nur derjenige Weihnachts-Gor, der aktuell den Geschenkesack trägt, darf Geschenke bescheren. Solange er den Geschenkesack trägt, kann er keine bescherten Gegenstände aufnehmen.}

\fan{Jeder Weihnachts-Gor, der den Geschenkesack nicht trägt, hat 3 kleine rechteckige Ablagefelder (links neben dem Heldenbild). Er darf dort auch Orweyns Hammer aus "Die Reise in den Norden" oder die Spitzhacke aus "Düstere Zeiten" tragen. Er darf bescherte Gegenstände aufnehmen, auch wenn er selbst sie bescherte. Erhält er den Geschenkesack, muss er alle Gegenstände, die er nun nicht mehr tragen kann, auf seinem Feld zurücklassen.}


\end{nobgcolor}


\begin{bgcolor}[color-DLvA]
\section{Die Legenden von Andor (2012)}

\subsection{Gebäude:}

\fan{Als Feld mit Gebäude gelten:}

\bullet{} \fan{In Andor (Spielplan-Vorderseite) die Felder 0, 2, 6, 15, 17–19, 24, 28, 32, 40, 57, 64, 71, 72 und 83,}

\bullet{} \fan{In Cavern (Spielplan-Rückseite) die Felder 6, 24, 25, 27, 28, 40, 42, 45, 61, 63–65, 68, 69 und 71.}


\subsection{Legende 2, Die Heilung des Königs:}


\fan{Solange die Skral-Festung im Spiel ist, gilt ihr Feld als Feld mit Gebäude.}


\subsection{Legende 3, Die Tage des Widerstands:}

\fan{Spielt ihr diese Legende mit der Solo-Variante alleine mit dem Weihnachts-Gor, hat er 3 kleine rechteckige Ablagefelder (links neben dem Heldenbild) und darf bescherte Geschenke selbst aufnehmen.}


\end{bgcolor}



\begin{bgcolor}[color-StSch]
\section{Der Sternenschild (2013)}

\subsection{Gebäude:}

\fan{Solange der Dunkle Tempel auf Feld 61 steht, gilt dieses Feld als Feld mit Gebäude.}

\end{bgcolor}

\begin{bgcolor}[color-DRidN]
\section{Die Reise in den Norden (2014)}

\subsection{Gebäude:}

\fan{Als Feld mit Gebäude gelten:}

\bullet{} \fan{Auf den Nebelinseln (Spielplan-Vorderseite) die Felder 91, 100, 103, 107–109, 113 und 114,}

\bullet{} \fan{In Hadria (Spielplan-Rückseite) die Felder 133, 135, 141, 143, 144 und 148.}

\end{bgcolor}




\columnbreak


\begin{bgcolor}[color-DlH]
\section{\fan{Die letzte Hoffnung (2016)}}

\subsection{\fan{Nahrung:}}

\fan{Der Weihnachts-Gor darf Apfelnüsse bescheren, kann aber kein Sternkraut bescheren.}





\subsection{\fan{Gebäude:}}

\fan{Das Feld des Trosswagens gilt nicht als Feld mit Gebäude. Solange das Lager auf einem Feld steht, gilt dieses Feld als Feld mit Gebäude. Als Feld mit Gebäude gelten zudem:}

\bullet{} \fan{Im Grauen Gebirge (Spielplan-Vorderseite) die Felder 223, 228, 242, 245, 247–249, 252, 267 und 268.}

\bullet{} \fan{In Krahd (Spielplan-Rückseite) die Felder 320, 340, 350 sowie 360–370.}




\end{bgcolor}




\begin{bgcolor}[color-BB]
\section{Die Bonus-Box (2017/2024)}

\subsection{Rucksack:}

\fan{Der Weihnachts-Gor darf, wenn er den Rucksack trägt, mit der Aktion "Umpacken" kleine rechteckige Gegenstände von seinem Feld in den Rucksack legen oder vom Rucksack auf sein Feld ablegen. Er kann wie üblich Gegenstände im Rucksack nicht nutzen.}

\subsection{Weiße Ereigniskarten:}

\fan{Der Weihnachts-Gor kann keine Gegenstände von der kleinen Ausrüstungstafel der weißen Ereigniskarten bescheren.}


\end{bgcolor}


\begin{bgcolor}[color-AG]
\section{Alte Geister (2018):}

\subsection{Tulgorischer Handel:}

\fan{Der Weihnachts-Gor kann keine Gegenstände von der Karte "Tulgorischer Handel" bescheren.}


\subsection{Bonus-Legende "Die vierte Statue":}

\fan{Es gibt keine Ausrüstungstafel. Der Weihnachts-Gor beschert Gegenstände von Feld 71.}

\end{bgcolor}



\begin{bgcolor}[color-DZ]
\section{Düstere Zeiten (2020)}


\subsection{Gebäude:}

\fan{Feld 83 auf dem Spielplan-Überleger gilt als Feld mit Gebäude.}

\fan{Während der Überleger "Zerstörter Karren" auf Feld 72 liegt, gilt dieses nicht als Feld mit Gebäude.}

\end{bgcolor}





\columnbreak




\begin{bgcolor}[color-DeK]
\section{\fan{Die ewige Kälte (2022)}}

\subsection{\fan{Gebäude:}}

\fan{Als Feld mit Gebäude gelten:}

\fan{\bullet{} Im Land der Steppe (Spielplan-Vorderseite) die Felder 401, 414, 423, 425, 442, 450 und 460.}

\fan{\bullet{} In Andor (Spielplan-Rückseite) die Felder 0, 2, 6, 15, 17, 19, 24, 28, 32, 40, 57, 64, 71, 72 und 83.}

\subsection{\fan{Geschenke bescheren:}}

\fan{Bleibt der Weihnachts-Gor auf einem Feld mit einem Gebäude stehen, darf er einen Gegenstand aus dem Vorrat oder von einem Händlerfeld auf dieses Feld legen, der auf einem Händlerfeld abgebildet ist. Dieser darf wie üblich erst vom Weihnachts-Gor getragen werden, nachdem ein anderer Held ihn aufgenommen hat. Das Feld wird wie üblich als beschenkt markiert.}

\end{bgcolor}


\begin{bgcolor}[color-DfL]
\section{\fan{Das ferne Land (2025)}}

\subsection{\fan{Gebäude:}}

\fan{Als Feld mit Gebäude gelten:}

\fan{\bullet{} In Azturien (Spielplan-Vorderseite) die Felder 501, 502, 505–512, 515, 518, 521–523, 530–532, 536, 539, 542, 550, 551, 553, 563 und 599.}

\fan{\bullet{} Im Land der Steppe (Spielplan-Rückseite) die Felder 401, 403, 405, 409, 412, 413, 415, 419, 426, 437, 442.}

\fan{Im Schachtelboden gelten keine Felder als Gebäude.}




\subsection{\fan{Geister:}}

\fan{Der Weihnachts-Gor kann kein Geschenk bescheren, wenn er auf einem Feld mit oder angrenzend zu Geistern stehen bleibt. Seine Geschenke dürfen aufgenommen werden, auch wenn sie bei oder angrenzend zu Geistern liegen.}

\fan{Spielt ihr mit mehreren Weihachts-Gors, gelten die Regeln zum Geschenkesack alle auch bei oder angrenzend zu Geistern.}





\subsection{\fan{Ruinengeist:}}

\fan{Würde der Weihnachts-Gor einen Ruinengeist aktivieren, geht das aktivierte Plättchen stattdessen ohne Effekt aus dem Spiel.}

\end{bgcolor}


\end{multicols*}









\chapter{25. Chada (Pasco), die Bogenschützin}
\label{Bogenschützin}


\begin{multicols*}{2}\raggedcolumns


\begin{nobgcolor}

\section{Allgemein}

\subsection{Aussprache:}


Der Name wird wie "Kada" gesprochen.
\supercite{26518,33469}




\subsection{Aufbau:}

Wählt den Rang der Bogenschützin.



\subsection{Heldentafel:}

Die Bogenschützin hat Rang 25 oder Rang 82.

Ihre offizielle Farbe ist Grün.

Sie hat:

\bullet{} 3 Würfel bei 1 bis 6 Willenspunkten,

\bullet{} 4 Würfel bei 7 bis 13 Willenspunkten,

\bullet{} 5 Würfel bei 14 bis 20 Willenspunkten.




\subsection{Bogen:}

Die Bogenschützin ist eine \hyperref[Fernkämpfer]{Fernkämpferin}.



\end{nobgcolor}


\begin{bgcolor}[color-DLvA]
\section{Die Legenden von Andor (2012)}


\subsection{Rekas Liebling:}

Der Trank der Hexe kostet die Bogenschützin 1 Gold weniger.
\supercite{DLvA:Ausrüstungstafel,DLvA2:Hexe,DLvA:Begleitheft}

\end{bgcolor}




\begin{bgcolor}[color-DlH]
\section{Die letzte Hoffnung (2016)}

\subsection{Zweite Sonderfähigkeit: Rekas Schülerin}

Die Bogenschützin erhält den Kessel. Der Kessel ist ein kleiner rechteckiger Gegenstand, der von anderen Helden getragen werden darf. 

Nur die Bogenschützin kann, wenn sie den Kessel trägt, die Aktion "Trank brauen" wählen. Dann gibt sie 1 oder mehr Sternkraut sowie 1 Stunde ab, um pro 1 Sternkraut 1 halb vollen Trank der Hexe oder pro 2 Sternkraut 1 vollen Trank der Hexe zu erhalten (in beliebiger Kombination, sie darf also z.B. für 4 Sternkräuter 1 vollen und 2 halb volle Tränke der Hexe erhalten). 
\supercite{DlH:Begleitheft,DlH11:Heldenkarte_Bogenschütze}

Mit Mondbeeren geht das nicht.\supercite{MH:Mondbeeren,MH:Begleitheft}


\end{bgcolor}





\end{multicols*}













\chapter{26. Aćh (Aćhan), die Takuri-Hüterin}
\label{Takuri-Hüterin}

\begin{multicols*}{2}\raggedcolumns


\begin{nobgcolor}

\section{Allgemein}


\subsection{Aussprache:}

Der Name wird wie "Ack" ("Ackan") gesprochen.
\supercite{MH:Begleitheft}


\subsection{Aufbau:}

Die Takuri-Hüterin erhält die tulgorische Steinflöte.

Stellt den Takuri-Marker auf +0.
\supercite{MH:Takuri}

Stellt den Takuri zur Takuri-Hüterin.
\supercite{MH:Begleitheft}




\subsection{Heldentafel:}

Die Takuri-Hüterin hat Rang 26.

Ihre offizielle Farbe ist Orange.

Sie hat höchstens 10 Stärkepunkte (statt 14).

Sie hat höchstens 19 Willenspunkte (statt 20).

Sie hat:

\bullet{} 1 Würfel bei 1 bis 4 Willenspunkten, 

\bullet{} 2 Würfel bei 5 bis 19 Willenspunkten.

Der Takuri-Marker kann folgende Werte haben: +0/+2/+3/+5/+7.








\subsection{Treuer Takuri Turr:}

Die Takuri-Hüterin hat einen \hyperref[Helfer]{\fan{Helfer}}, die Figur "Takuri".

Steht der Takuri auf dem Feld eines kämpfenden Helden, gibt er den Wert des Takuri-Markers als \hyperref[Bonusstärke]{Bonusstärke}.
\supercite{MH:Takuri}

Er darf alle Felder betreten und Feldverbindungen nutzen, auch wenn Helden diese nicht oder nur unter gewissen Bedingungen betreten oder passieren können (z.B. Geröllfelder, blockierte Brücken, Felder mit Arbaks oder das Lager der Trolle).
\supercite{MH:Begleitheft}


\subsection{Feueriger Lebenszyklus:}

Jedes Mal, wenn der Erzähler vorgesetzt wird, bewegt sich der Takuri-Marker auf das nächste Feld. Wird der Erzähler zurückgesetzt, geschieht nichts.
\supercite{MH:Begleitheft}

Steht der Takuri-Marker auf +7, darf die Takuri-Hüterin 5 Willenspunkte abgeben, um den Takuri zu besänftigen und den Marker auf +5 zurückzusetzen.
\supercite{MH:Takuri}
Das darf sie jederzeit, insbesondere nach einem Kampf zwischen der Belohnung und der Erzählerbewegung\supercite{89455}, und egal wie weit vom Takuri sie entfernt ist.





\subsection{Tulgorische Steinflöte:}

Die tulgorische Steinflöte ist ein kleiner rechteckiger Gegenstand, der von anderen Helden getragen werden darf. 
Nur die Takuri-Hüterin darf die Steinflöte von der Vorderseite auf die Rückseite drehen, um den Takuri auf ein beliebiges Spielplanfeld zu versetzen.
Bei Sonnenaufgang wird die Steinflöte auf die Vorderseite gedreht.
Trägt die Takuri-Hüterin die Steinflöte auf ihrer Rückseite, darf sie als freie Handlungsmöglichkeit 2 Stunden abgeben, um sie wieder auf die Vorderseite zu drehen. Das darf sie mehrmals pro Tag.
\supercite{MH:Begleitheft}






\subsection{\fan{Mehrere Takuri-Hüterinnen:}}

\fan{Alle Steinflöten sind ununterscheidbar und dürfen von allen Takuri-Hüterinnen genutzt oder aufgefrischt werden. Jede Takuri-Hüterin hat aber ihren eigenen Takuri, den nur sie bewegen oder besänftigen kann. Ihr dürft den Takuri Standfüße in unterschiedlichen Farben geben.}





\subsection{Neutrale Zeitsteine:}

Im Spiel mit neutralen Zeitsteinen darf die Takuri-Hüterin vor dem Auffrischen der Flöte auswählen, welchen Zeitstein sie 2 Stunden vorsetzen will.

\end{nobgcolor}



\begin{bgcolor}[color-DLvA]
\section{Die Legenden von Andor (2012)}

\subsection{Ausscheiden:}

\fan{Scheidet die Takuri-Hüterin aus (z.B. weil sie in Cavern auf 0 Willenspunkte fiel), geht ihr Takuri aus dem Spiel.}


\subsection{Legende 5, Der Zorn des Drachens:}

\fan{Bei der Drachenkampfkarte "Gegenstand in 2 Stärkepunkte tauschen" darf die Takuri-Hüterin die tulgorische Steinflöte in Stärke tauschen.}


\end{bgcolor}






\begin{bgcolor}[color-DlH]
\section{Die letzte Hoffnung (2016)}



\subsection{Zweite Sonderfähigkeit: Takuri-Federn}

Die Takuri-Hüterin erhält die zwei Takuri-Federn, welche auf der Nahrungsablage getragen werden. Nur die Takuri-Hüterin kann Takuri-Federn tragen. Sie darf 1 Takuri-Feder abgeben, damit sie alle Überstunden an diesem Tag keine Willenspunkte kosten.
\supercite{MH11:Heldenkarte_Takuri-Hüterin}

\fan{Spielt ihr mit mehreren Takuri-Hüterinnen: Alle Takuri-Federn sind ununterscheidbar und dürfen von allen Takuri-Hüterinnen getragen, getauscht und genutzt werden. Eine Takuri-Hüterin darf beliebig oft pro Legende Takuri-Federn nutzen.}


\subsection{Legende 15, Der vergiftete Geist:}

Für das offizielle Geschenk der Takuri-Hüterin darf sie die Steinflöte auch auf die Rückseite drehen, ohne den Takuri zu versetzen.
\supercite{MH:Begleitheft}
\fan{Wird die Takuri-Hüterin enttarnt, gehen von ihr getragene Takuri-Federn und ihr Takuri aus dem Spiel, ohne zu Stärke zu werden. Sie darf die tulgorische Steinflöte in Stärke tauschen.}


\subsection{Legende 17, Die letzte Hoffnung:}

Die Spiegelscherben von der offiziellen Hoffnungskarte können Wachtrolle nicht über Sprungfelder versetzen. 
\supercite{MH:Begleitheft}

\fan{Mit den Spiegelscherben können Gegner nicht über Eisbrücken des \hyperref[Eis-Dämon]{Eis-Dämons} versetzt werden.}

\fan{Mit den Spiegelscherben dürfen auch festgehaltene Helden (z.B. durch Skelette oder Spornen) versetzt werden.}

\end{bgcolor}



\begin{bgcolor}[color-DeK]
\section{\fan{Die ewige Kälte (2022)}}

\subsection{\fan{Tulgorische Steinflöte:}}

\fan{Die Takuri-Hüterin kann die Steinflöte nur auf ihrer Vorderseite bei Händlern tauschen.}


\end{bgcolor}




\begin{bgcolor}[color-DfL]
\section{\fan{Das ferne Land (2025)}}



\subsection{\fan{Geister:}}

\fan{Steht die Takuri-Hüterin bei oder angrenzend zu Geistern, kann sie die tulgorische Steinflöte nicht nutzen. Steht der Takuri bei oder angrenzend zu Geistern, gibt er keine \hyperref[Bonusstärke]{Bonusstärke}. Für seine Bonusstärke ist es egal, ob die Takuri-Hüterin bei oder angrenzend zu Geistern steht.}

\fan{Der Takuri-Marker wird bewegt, wenn der Erzähler vorgesetzt wird, egal ob der Takuri bei oder angrenzend zu Geistern steht. Die Takuri-Hüterin darf den Takuri-Marker von +7 auf +5 zurücksetzen und die tulgorische Steinflöte auf die Vorderseite drehen, auch wenn sie oder der Takuri bei oder angrenzend zu Geistern stehen.}

\end{bgcolor}


\end{multicols*}










\chapter{29. Kheela (Kheelan), die Hüterin}
\label{Hüterin}

\begin{multicols*}{2}\raggedcolumns


\begin{nobgcolor}

\section{Allgemein}


\subsection{Aufbau:}

Stellt den Wassergeist zur Hüterin.
\supercite{NH:Begleitheft}



\subsection{Heldentafel:}

Die Hüterin hat Rang 29.

Ihre offizielle Farbe ist Weiß.

Sie hat 1 Würfel, egal wie viele Willenspunkte sie hat.




\subsection{Treuer Wassergeist Vara:}

Die Hüterin hat eine \hyperref[Helfer]{\fan{Helferin}}, die Figur "Wassergeist".






\subsection{Großer weißer Würfel:}

Der große weiße Würfel hat die Seiten 4/5/5/6/6/7.

Ein Held, der mit dem Feld des Wassergeists steht, darf im Kampf, \fan{wenn er 1 oder mehr seiner Heldenwürfel werfen würde}, stattdessen den großen weißen Würfel statt aller seiner Heldenwürfel werfen.

Im gemeinsamen Kampf darf in jeder Kampfrunde höchstens 1 Held den großen weißen Würfel nutzen.
\supercite{5300}

Der große weiße Würfel darf auf die gegenüberliegende Seite gedreht werden (z.B. von der \hyperref[Zauberin]{Zauberin}\supercite{5335}) und mit dem Trank der Hexe verdoppelt werden.
\supercite{5325}

Das Messer des \hyperref[Fährtenleser]{Fährtenlesers} kann ihn nicht neu werfen.
\supercite{5460}




\subsection{Bonuswürfel:}

\fan{Wirft ein Held im Kampf einen Gegnerwürfel gemeinsam mit seinen Heldenwürfeln (z.B. durch den Streifenmarder in "Die Reise in den Norden" oder Meres in "Die steinernen Drei") und wirft den großen weißen Würfel statt seiner Heldenwürfel, wirft er nur den großen weißen Würfel. Dennoch wirft der Gegner einen Würfel weniger.}







\subsection{\fan{Mehrere Hüterinnen:}}

\fan{Jede Hüterin hat einen eigenen Wassergeist, den nur sie bewegen kann. Ihr dürft den Wassergeistern Standfüße in unterschiedlichen Farben geben. Stehen mehrere Wassergeister auf demselben Feld, dürfen im gemeinsamen Kampf bis zu so viele Helden auf diesem Feld den großen weißen Würfel nutzen.}


\end{nobgcolor}


\begin{bgcolor}[color-DLvA]
\section{Die Legenden von Andor (2012)}

\subsection{Ausscheiden:}

Scheidet die Hüterin aus (z.B. weil sie in Cavern auf 0 Willenspunkte fiel), geht ihr Wassergeist aus dem Spiel.
\supercite{35653}

\end{bgcolor}



\begin{bgcolor}[color-StSch]
\section{Der Sternenschild (2013)}

\subsection{Andorische Flöte:}

\fan{Nutzt ein Held die andorische Flöte, kann er nicht den großen weißen Würfel werfen, auch wenn er beim Wassergeist steht.}

\end{bgcolor}



\begin{bgcolor}[color-DRidN]
\section{Die Reise in den Norden (2014)}



\subsection{Treuer Wassergeist Vara:}

Der Wassergeist startet nicht immer bei der Hüterin, sondern immer auf dem Meeresfeld, auf dem das Schiff startet. 
\supercite{DRidN:Begleitheft}



\subsection{Großer weißer Würfel:}

Steht der Wassergeist auf dem Meeresfeld des Schiffes, darf ein Held an Bord im Kampf den großen weißen Würfel nutzen.
\supercite{DRidN:Begleitheft}


\subsection{Legende 9, Die Mächte des Meeres:}

Für Kenvilars Tücke "auf die 1" wird auch der große weiße Würfel als 1 gewertet.
\supercite{15179}

\end{bgcolor}


\begin{bgcolor}[color-DlH]
\section{\fan{Die letzte Hoffnung (2016)}}

\subsection{\fan{Alte Waffen:}}

\fan{Das Messer kann den großen weißen Würfel nicht neu werfen.}



\subsection{\fan{Legende 15, Der vergiftete Geist:}}

\fan{Wird die Hüterin enttarnt, geht ihr Wassergeist aus dem Spiel.}



\end{bgcolor}






\begin{bgcolor}[color-AG]
\section{Alte Geister (2018)}



\subsection{Legende 2, Der Hexer aus Andor:}

\fan{Ist der Endkampf gegen Siantari und auf Feld 30, muss ein Held, der den großen weißen Würfel wirft, genau 1 Willenspunkt abgeben.}



\end{bgcolor}



\begin{bgcolor}[color-DZ]
\section{Düstere Zeiten (2020)}



\subsection{Shans Fluch:}

\fan{Ein Held, der wegen Shans Fluch keinen Würfel mehr hat, kann nicht den großen weißen Würfel statt seiner Heldenwürfel werfen.}



\end{bgcolor}




\columnbreak


\begin{bgcolor}[color-DeK]
\section{\fan{Die ewige Kälte (2022)}}

\subsection{\fan{Askimar-Klinge:}}

\fan{Eine Askimar-Klinge darf den großen weißen Würfel neu werfen.}

\end{bgcolor}






\begin{bgcolor}[color-DfL]
\section{\fan{Das ferne Land (2025)}}

\subsection{\fan{Wurfmesser:}}

\fan{Ein Wurfmesser darf den großen weißen Würfel neu werfen.}


\subsection{\fan{Geister:}}

\fan{Steht die Hüterin bei oder angrenzend zu Geistern, kann sie den Wassergeist nicht bewegen. Steht der Wassergeist bei oder angrenzend zu Geistern, kann niemand im Kampf den großen weißen Würfel werfen. Sofern er nicht bei oder angrenzend zu Geistern steht, darf ein Held den großen weißen Würfel werfen, auch wenn die Hüterin oder der Gegner bei oder angrenzend zu einem Geist steht.}

\end{bgcolor}








\end{multicols*}












\chapter{33. Grent (Grea), der Bergkrieger}
\label{Bergkrieger}

\begin{multicols*}{2}\raggedcolumns


\begin{nobgcolor}

\section{Allgemein}


\subsection{Heldentafel:}

Der Bergkrieger hat Rang 33.

Seine offizielle Farbe ist Violett.

Er hat 1 Würfel, egal wie viele Willenspunkte er hat.






\subsection{Würfeleffekte:}

Je nachdem, welche Zahl der Heldenwürfel des Bergkriegers im Kampf zeigt, aktiviert er zusätzliche Effekte:

\bullet{} 1: Ist der Kampfwert der Helden in dieser Kampfrunde kleiner als der Kampfwert des Gegners, darf der Bergkrieger sich entscheiden, dass er selbst\supercite{126498} dadurch keine Willenspunkte verliert.

\bullet{} 2: Der Bergkrieger darf seinen Heldenwürfel neu werfen und den neuen Effekt aktivieren. So darf er seinen Würfel auch mehrmals nacheinander neu werfen.
\supercite{DfL:Begleitheft}

\bullet{} 3: Der Bergkrieger darf diesen Würfelwert behandeln, als wäre er eine 6. \fan{Er darf diesen Wert mit dem Trank der Hexe verdoppeln.}

\bullet{} 4: Der Bergkrieger darf den Zeitstein, den er für diese Kampfrunde vorrückte, eine Stunde zurücksetzen, \fan{sogar bis auf das Sonnenaufgang-Feld}.

\fan{Diese Effekte werden aktiviert, egal ob der Würfel durch den Wurf des Bergkriegers auf dieser Zahl landete oder weil er auf die gegenüberliegende Seite gedreht (z.B. von der \hyperref[Zauberin]{Zauberin}) oder durch einen Gegenstand neu geworfen wurde.}

\fan{Effekte 1 und 3 gelten nur, wenn der Würfel des Bergkriegers am Ende eine 1 bzw. 3 zeigt, nicht, wenn diese 1 bzw. 3 später durch Effekte neu geworfen oder weggedreht wird. Die Effekte 2 und 4 wirken sofort, auch wenn diese 2 bzw. 4 später durch Effekte neu geworfen oder weggedreht wird.}





\subsection{Bonuswürfel:}


\fan{Wirft der Bergkrieger einen Gegnerwürfel gemeinsam mit ihren Heldenwürfeln (z.B. durch den Streifenmarder in "Die Reise in den Norden" oder Meres in "Die steinernen Drei"), werden durch diesen keine Effekte aktiviert. Ein Effekt seines geworfenen Heldenwürfels wird aktiviert, auch wenn er diesen gar nicht zu seinem Kampfwert addiert.}

\fan{Mit einem Helm darf er eine 3 seines Heldenwürfels zu einer 6 eines Gegnerwürfels addieren.}

\fan{Im \hyperref[Fernkämpfer]{Fernkampf} muss er zuerst seinen Heldenwürfel werfen und sich danach entschieden, ob er aufhört oder den Gegnerwürfel wirft.}

\fan{Wirft der Bergkrieger statt seiner Heldenwürfel einen Ersatzwürfel (z.B. den großen weißen Würfel der \hyperref[Hüterin]{Hüterin}), werden keine Effekte aktiviert.}



\columnbreak

\subsection{Scharfe Augen:}

Während der Bergkrieger mit 14 oder mehr Willenspunkten auf einem Feld steht (nicht im Vorbeigehen), darf er beliebig viele \hyperref[Aufdecken]{\fan{von fern aufdeckbare}} Plättchen auf angrenzenden Feldern aufdecken. Aktiviert werden diese Plättchen dadurch nicht. Plättchen in Stapeln dürfen auch aufgedeckt werden, ihre Reihenfolge bleibt aber gleich.
\supercite{BL_BdM:A3_en}


\subsection{Kein Krieger aus Andor:}

Spricht eine Regel oder Karte vom \hyperref[Krieger]{Krieger}, ist der Bergkrieger nicht gemeint.

\end{nobgcolor}






\begin{bgcolor}[color-DLvA]

\section{Die Legenden von Andor (2012)}



\subsection{Legende 3, Die Tage des Widerstands:}

\fan{Nutzt Varkur die Dunkle Magie "Die Schwächung der Helden", dreht er eine 4 des Bergkrieger nicht auf eine 3. Er dreht aber eine 5 auf eine 2 (die dann neu geworfen werden darf).}

\fan{Beim Dunklen Magier "Der verhexte Prinz" (aus "Varkurs Erwachen") wird keine Prinz-Bewegung ausgelöst, auch wenn Würfeleffekt 4 den Zeitstein auf die 3. Stunde verschiebt.}







\subsection{Ereigniskarte "Geheimer See" 9:}

\fan{Es werden keine Würfeleffekte des Bergkriegers aktiviert.}

\end{bgcolor}



\begin{bgcolor}[color-StSch]
\section{Der Sternenschild (2013)}

\subsection{Andorische Flöte:}

\fan{Nutzt der Bergkrieger die andorische Flöte, werden keine seiner Würfeleffekte aktiviert.}



\subsection{Wolfssymbol:}

Wird der Zeitstein \fan{durch Würfeleffekt 4} rückwärts auf das Wolfssymbol verschoben, werden die Wölfe nicht bewegt.
\supercite{StSch:Begleitheft}

\end{bgcolor}





\columnbreak


\begin{bgcolor}[color-DRidN]
\section{Die Reise in den Norden (2014)}

\subsection{Legende 9, Die Mächte des Meeres:}

\fan{Für Kenvilars Tücke "auf die 1" wird der Würfel des Bergkriegers direkt auf die 1 gedreht, bevor er andere Effekte aktivieren könnte. Er darf aber Würfeleffekt 1 aktivieren.}

\fan{Für Kenvilars Tücke "nur gerade Würfelwerte" darf der Bergkrieger Würfeleffekte aktivieren, egal ob die Zahl gerade ist. Er darf eine 3 als gerade 6 behandeln.}


\subsection{Legende "Die Rückkehr der Schwarzen Kogge":}

\fan{Wirft der Unbekannte Krieger den Würfel des Bergkriegers, werden keine Würfelffekte aktiviert.}

\end{bgcolor}


\begin{bgcolor}[color-DlH]
\section{\fan{Die letzte Hoffnung (2016)}}

\subsection{\fan{Bewegungsplättchen:}}

\fan{Wird der Zeitstein durch Würfeleffekt 4 rückwärts von der 4. Stunde verschoben, wird kein Bewegungsplättchen aktiviert. Es ist aber möglich, dass der Bergkrieger durch Würfeleffekt 4 mehrere Bewegungsplättchen am selben Tag aktiviert.}




\subsection{\fan{Legende 17, Die letzte Hoffnung:}}

\fan{Hat der Bergkrieger 14 oder mehr Willenspunkte, darf er auch ohne Fernrohr Ortskarten aktivieren.}


\subsection{\fan{Bronzene Ereigniskarte 226:}}

\fan{Hat der Bergkrieger 14 oder mehr Willenspunkte, erhält er auch ohne Fernrohr 2 Willenspunkte.}

\end{bgcolor}



\begin{bgcolor}[color-BB]

\section{Die Bonus-Box (2017/2024)}

\subsection{Weiße Ereigniskarte 5:}


\fan{Hat der Bergkrieger 14 oder mehr Willenspunkte, darf er auch ohne Fernrohr bis zu 2 Nebelplättchen aufdecken.}

\end{bgcolor}





\begin{bgcolor}[color-AG]
\section{Alte Geister (2018)}

\subsection{Legende 2, Der Hexer aus Andor:}

Wird der Zeitstein \fan{durch Würfeleffekt 4} rückwärts auf die Stunde der Kartographie-Marker verschoben, darf kein Kartographie-Marker entfernt werden. 
\supercite{AG:Begleitheft}
\fan{Es ist aber möglich, dass der Bergkrieger durch Würfeleffekt 4 mehrere Kartographie-Marker am selben Tag aktiviert.}

\fan{Wird der Zeitstein durch Würfeleffekt 4 auf eine Überstunde mit Gold und Kreatur verschoben, erhält er das Gold und die Kreatur wird eingewürfelt.}


\subsection{Bonus-Legende "Die vierte Statue":}

\fan{Für Zors Drachenfeuer-Gift-Effekt gilt eine vom Bergkrieger gewürfelte 3 als 3, auch wenn der Bergkrieger sie als 6 behandelt.}

\end{bgcolor}




\columnbreak




\begin{bgcolor}[color-DZ]
\section{Düstere Zeiten (2020)}

\subsection{Legende 2, Fremde Heimat:}

\fan{Wird der Zeitstein durch Würfeleffekt 4 rückwärts auf die Stunde der Geröllplättchen oder des Thogger-Plättchens verschoben, wird kein Geröll ausgelöst und Thogger nicht bewegt.}

\end{bgcolor}



\begin{bgcolor}[color-DfL]
\section{Das ferne Land (2025)}




\subsection{Geister:}

Während der Bergkrieger bei oder angrenzend zu Geistern steht, hat sein Würfel im Kampf keine Würfeleffekte und er kann seine scharfen Augen nicht nutzen.

\fan{Er darf mit seinen scharfen Augen auch Plättchen bei oder angrenzend zu Geistern aufdecken.}

\end{bgcolor}




\end{multicols*}














\chapter{34. Eara (Liphardus), die Zauberin}
\label{Zauberin}


\begin{multicols*}{2}\raggedcolumns


\begin{nobgcolor}

\section{Allgemein}



\subsection{Heldentafel:}

Die Zauberin hat Rang 34.

Ihre offizielle Farbe ist Violett.

Sie hat 1 Würfel, egal wie viele Willenspunkte sie hat.




\subsection{Würfelzauberei:}

Kämpft die Zauberin, darf sie einmal pro Kampfrunde, sofort nachdem ein Held Würfel warf, einen soeben geworfenen Würfel auf die gegenüberliegende Seite drehen. So wird z.B. aus einer "1" eine "6".

Das darf auch ein durch einen Effekt (z.B. Messer) neu geworfener Würfel sein.

Sie muss sich sofort nach dem Würfeln entscheiden und kann nicht erst abwarten, bis alle Helden gewürfelt haben.
\supercite{DLvA:Begleitheft,DlH:Begleitheft,DeK:Begleitheft}

Außerhalb des Kampfs kann sie Würfel nur drehen, wenn es ihr explizit erlaubt wird.




\subsection{Bonuswürfel und besondere Würfel:}

Die Zauberin darf auch einen Ersatzwürfel drehen (z.B. den großen weißen Würfel der \hyperref[Hüterin]{Hüterin}\supercite{5335}).
\supercite{DLvA:Begleitheft,DLvA2:Runensteine,DRidN:Begleitheft}

Sie darf auch den Runenwürfel der \hyperref[Runenmeisterin]{Runenmeisterin} drehen\supercite{88563} (aber nicht für ihre Aktion "Runen befragen"), oder den nächsten Würfel der Würfelleiste des \hyperref[Seher]{Sehers} (aber nicht für sein Sehendes Auge)\supercite{DH:FAQ}, oder einen rosa Würfel des \hyperref[Steppennomade]{Steppennomaden}. Den Pyramidenwürfel des \hyperref[Eis-Dämon]{Eis-Dämons} kann sie nicht drehen.
\supercite{91842}





\subsection{\fan{Mehrere Zauberinnen:}}

\fan{Kämpfen mehrere Zauberinnen gemeinsam, darf jede einen anderen Würfel drehen. Es dürfen auch mehrere denselben Würfel drehen (z.B. um mehrmals den Würfeleffekt 4 des \hyperref[Bergkrieger]{Bergkriegers} zu aktivieren).}

\end{nobgcolor}




\begin{bgcolor}[color-DLvA]

\section{Die Legenden von Andor (2012)}


\subsection{Legende 3, Die Tage des Widerstands:}

\fan{Nutzt Varkur die Dunkle Magie "Die Schwächung der Helden", kann die Zauberin im Endkampf keinen Würfel auf die gegenüberliegende Seite drehen.}


\subsection{Ereigniskarte "Geheimer See" 9:}

Die Zauberin darf ihren Würfel auf die gegenüberliegende Seite drehen.
\supercite{DLvA:Ereigniskarte_Geheimer_See_9}
Einen Würfel eines anderen Helden, der diese Ereigniskarte aktivierte, kann sie nicht drehen.


\end{bgcolor}



\columnbreak


\begin{bgcolor}[color-StSch]
\section{Der Sternenschild (2013)}


\subsection{Andorische Flöte:}

Nutzt die Zauberin die Flöte selbst, darf sie den geworfenen Würfel auf die gegenüberliegende Seite drehen.
\supercite{StSch:Gaben,StSch:Begleitheft}

\end{bgcolor}


\begin{bgcolor}[color-DRidN]
\section{Die Reise in den Norden (2014)}


\subsection{Legende 9, Die Mächte des Meeres:}

\fan{Für Kenvilars Tücke "auf die 1" kann die Zauberin ihren auf die 1 gedrehten Würfel nicht auf die gegenüberliegende Seite drehen.}

\fan{Für Kenvilars Tücke "nur gerade Würfelwerte" darf die Zauberin einen ungeraden Würfelwert auf die gegenüberliegende gerade Seite drehen und zählen lassen.}
 

\subsection{Legende 10, Hadria:}

Die Zauberin kann die großen schwarzen Würfel des Seeriesen\supercite{DRidN10:Turm_bewahrt} und den kleinen roten Geisterflammen-Würfel der Feuerzauberer\supercite{DRidN10:Geisterflammen} nicht drehen.

\end{bgcolor}



\columnbreak


\begin{bgcolor}[color-DlH]
\section{Die letzte Hoffnung (2016)}

\subsection{Heldentafel:}

Die Zauberin hat höchstens 10 Stärkepunkte (statt 14).

Sie hat höchstens 14 Willenspunkte (statt 20).

Sie hat 3 besondere Ablagefelder für Zauberbücher.




\subsection{Zweite Sonderfähigkeit: Zaubersprüche}

Die Zauberin erhält die drei Zauberbücher mit der geschlossenen Seite oben.

Zauberbücher gelten als Gegenstände und dürfen abgelegt und aufgenommen werden, aber nur die Zauberin kann sie tragen.

Nur sie kann die Aktion "Zauberspruch nachlesen" wählen. Dann gibt sie 1 bis 3 Stunden ab und dreht pro Stunde 1 geschlossenes Zauberbuch, das sie trägt, auf die offene Seite. 

Sie darf ein offenes Zauberbuch als freie Handlungsmöglichkeit aus dem Spiel geben, um es zu aktivieren.
\supercite{DlH:Begleitheft,DlH11:Heldenkarte_Zauberer}





\subsection{Lichtzauber (braun):} 

Versetzt eine Kreatur, einen Krahder, oder beliebig viele Skelette, die auf dem Feld der Zaubererin stehen, auf ein einzelnes angrenzendes Feld. Das geht nicht über Sprungfelder und nicht über Eisbrücken des \hyperref[Eis-Dämon]{Eis-Dämons}. Wird eine Kreatur versetzt, darf auf dem Zielfeld keine Kreatur sein.
\supercite{DlH11:Heldenkarte_Zauberer}
Endgegner oder der Urtroll\supercite{DlH:FAQ} können so nicht versetzt werden.



\subsection{Heilzauber (grün):} 

Die Zauberin verteilt \fan{bis zu} so viele Willenspunkte wie die Einerstelle ihrer Feldzahl auf die Heldengruppe, egal wo die Helden stehen und ob sie ihren Tag bereits beendet haben.

\subsection{Portalzauber (blau):} 

Die Zauberin versetzt sich und bis zu 1 anderen Helden, der auf ihrem Feld steht und seinen Tag noch nicht beendet hat, auf das Feld des Lagers oder Trosswagens. Das geht auch, wenn sie oder der andere Held festgehalten sind (z.B. von Skeletten oder Spornen).
\supercite{DlH:Begleitheft,40664}


Ein mitteleportierter Held kann keine mitbewegbaren Plättchen (z.B. Bauern) oder Figuren (z.B. Merrik) mitnehmen. Teleportiert die Zauberin keinen Helden mit, darf sie jedoch ein mitbewegbares Plättchen \fan{oder eine mitbewegbare Figur} mitteleportieren.\supercite{41284} 


\subsection{\fan{Mehrere Zauberinnen:}}
\fan{Zauberbücher verschiedener Zauberinnen sind ununterscheidbar. Zauberinnen können geschlossene und offene Zauberbücher miteinander tauschen, beliebige Zauberbücher nachlesen und beliebige Zauberbücher, die sie tragen, nutzen.}



\subsection{Legende 14, Der Meister des Trolls:}

\fan{Ein Held, der wegen seiner Erschöpfungskarte das Lager nicht mehr betreten kann, kann auch nicht durch den Portalzauber der Zauberin zum Lager mitteleportiert werden.}




\subsection{Legende 15, Der vergiftete Geist:}

Wird die Zauberin enttarnt, darf sie Zauberbücher in Stärke tauschen.
\supercite{DlH15:Verhext?}




\subsection{Legende 17, Krahd:}

\fan{Der Lichtzauber kann auch einen Wachtroll versetzen. Während der Erzähler auf "j" oder vorher steht, bewegt sich ein versetzter Wachtroll durch Bewegungsplättchen zurück zu seinem Ort.}

\fan{Bringen mehrere Zauberinnen oder \hyperref[Dunkle Magierin]{Dunkle Magierinnen} durch ihre offiziellen Heldenkarten einen Schatten ins Spiel, nutzt verschiedene Figuren. Jede Zauberin oder Dunkle Magierin kann nur ihren eigenen Schatten bewegen und nur mit ihrem eigenen Schatten die Position tauschen. Ihr dürft den Schatten unterschiedlich farbige Standfüße geben.}

\end{bgcolor}




\begin{bgcolor}[color-AG]
\section{Alte Geister (2018)}

\subsection{Meres' Hilfe:}


Die Zauberin kann den großen schwarzen Würfel nicht drehen.
\supercite{AG:Begleitheft}

\end{bgcolor}


\begin{bgcolor}[color-DZ]
\section{Düstere Zeiten (2020)}

\subsection{Shans Fluch:}

Auch wenn die Zauberin durch Shans Fluch keinen Würfel mehr hat, darf sie im gemeinsamen Kampf noch Würfel anderer Helden drehen.
\supercite{DZ:Begleitheft}

\end{bgcolor}




\begin{bgcolor}[color-DfL]
\section{\fan{Das ferne Land (2025)}}



\subsection{\fan{Geister:}}

\fan{Steht die Zauberin bei oder angrenzend zu Geistern, kann sie keine Würfel drehen. Sofern sie selbst nicht bei oder angrenzend zu Geistern steht, darf sie die Würfel von Helden drehen, die bei oder angrenzend zu Geistern stehen.}

\end{bgcolor}







\end{multicols*}


\chapter{34. Eara (Liphardus), die Dunkle Magierin}
\label{Dunkle Magierin}

\begin{multicols*}{2}\raggedcolumns


\begin{nobgcolor}

\section{Allgemein}


\subsection{Heldentafel:}

Die Dunkle Magierin hat Rang 34.

Ihre offizielle Farbe ist Violett.

Sie hat 1 Würfel, egal wie viele Willenspunkte sie hat.




\subsection{Würfelmagie:}

Kämpft die Dunkle Magierin, darf sie einmal pro Kampfrunde, sofort nachdem ein Gegner Würfel warf, einen dieser geworfenen Würfel auf die gegenüberliegende Seite drehen. So wird z.B. aus einer "6" eine "1".

Das geht auch gegen Endgegner.



\subsection{5 oder 6 Helden – Schwarzer Augenwürfel:}

Die Dunkle Magierin kann den schwarzen Augenwürfel nicht drehen.





\subsection{Keine Zauberin mehr:}

Spricht eine Regel oder Karte von der \hyperref[Zauberin]{Zauberin}, ist die Dunkle Magierin nicht gemeint.



\subsection{\fan{Mehrere Dunkle Magierinnen:}}

\fan{Kämpfen mehrere Dunkle Magierinnen gemeinsam, darf jede einen anderen Würfel des Gegners drehen.}


\end{nobgcolor}





\begin{bgcolor}[color-DLvA]

\section{Die Legenden von Andor (2012)}

\subsection{Ereigniskarte "Geheimer See" 9:}

\fan{Die Dunkle Magerin kann ihren Würfel nicht auf die gegenüberliegende Seite drehen.}


\end{bgcolor}






\begin{bgcolor}[color-StSch]
\section{Der Sternenschild (2013)}


\subsection{Andorische Flöte:}

\fan{Nutzt die Dunkle Magierin die Flöte, kann sie den geworfenen Würfel nicht auf die gegenüberliegende Seite drehen.}

\end{bgcolor}


\begin{bgcolor}[color-DRidN]
\section{Die Reise in den Norden (2014)}





\subsection{Legende 10, Hadria:}

\fan{Die Dunkle Magierin kann die großen schwarzen Würfel des Seeriesen, Torvens Würfel und den kleinen roten Geisterflammen-Würfel der Feuerzauberer nicht drehen. Sie darf aber den großen roten Feuerwürfel der Feuerzauberer drehen, bevor die dazugehörige Feuerzauber-Karte aktiviert wird.}




\subsection{Legende "Die Rückkehr der Schwarzen Kogge":}

\fan{Die Dunkle Magierin darf auch einen roten Würfel von Pera, Roa oder Thogger drehen.}

\end{bgcolor}




\begin{bgcolor}[color-DlH]
\section{Die letzte Hoffnung (2016)}

\subsection{Zweite Sonderfähigkeit: Magischer Schatten}

Stellt am Anfang die Figur "Schatten" zur Dunklen Magierin. Er ist ein \hyperref[Helfer]{\fan{Helfer}}.

Die Dunkle Magierin darf jederzeit, außer sie hat ihren Tag bereits beendet, einen großen grauen Würfel werfen. Die Heldengruppe muss insgesamt so viele Willenspunkte abgeben. Es dürfen auch Helden Willenspunkte abgeben, die ihren Tag bereits beendet haben. Dann tauscht die Dunkle Magierin die Position ihrer Heldenfigur mit der des Schattens. \fan{Das geht auch, wenn sie festgehalten wird (z.B. von Skeletten oder Spornen).}
 

Können nicht genug Willenspunkte abgeben werden, gibt stattdessen niemand Willenspunkte ab, die Position der Heldenfigur und des Schattens werden nicht getauscht und der Schatten wird bis zum nächsten Sonnenaufgang hingelegt. Solange kann er nicht bewegt werden.
\supercite{Dunkle_Eara:Regelkarte,Dunkle_Eara:Heldenkarte}

\fan{Wurde die Dunkle Magierin in Legende 15 enttarnt, muss ab dann nur sie selbst statt die Heldengruppe die Willenspunkte abgeben.}


\fan{Bringen mehrere \hyperref[Dunkle Magierin]{Dunkler Magierinnen} oder \hyperref[Zauberin]{Zauberinnen} durch ihre offiziellen Heldenkarten einen Schatten ins Spiel, nutzt verschiedene Figuren. Jede Dunkle Magierin oder Zauberin kann nur ihren eigenen Schatten bewegen und nur mit ihrem eigenen Schatten die Position tauschen. Ihr dürft den Schatten unterschiedlich farbige Standfüße geben.}




\subsection{Legende 14, Der Meister des Trolls:}

\fan{Kann die Dunkle Magierin wegen ihrer Erschöpfungskarte das Lager nicht mehr betreten, darf ihr Schatten weiterhin das Lager betreten. Sie kann sich aber nicht durch Schattentausch zum Lager versetzen.}
\supercite{40196}




\subsection{Legende 15, Der vergiftete Geist:}

\fan{Wird die Dunkle Magierin enttarnt, geht ihr Schatten aus dem Spiel.}





\end{bgcolor}





\begin{bgcolor}[color-AG]
\section{Alte Geister (2018)}

\subsection{Legende 2, Der Hexer aus Andor:}

\fan{Dreht die Dunkle Magierin im Kampf gegen eine Kreatur einen Würfel, gilt für die Bewegung der Maasavi der Würfelwert nach dem Drehen des roten Würfels.}

\fan{Dreht die Dunkle Magierin im Endkampf einen Würfel, werden die Effekte von Meres und Reka erst danach aktiviert.  Entschied die Gruppe sich gegen Meres' Hilfe, darf die Dunkle Magierin im Endkampf Meres' großen schwarzen Würfel drehen.}

\end{bgcolor}




\begin{bgcolor}[color-DfL]
\section{\fan{Das ferne Land (2025)}}

\subsection{\fan{Geister:}}

\fan{Steht die Dunkle Magierin bei oder angrenzend zu Geistern, kann sie keine Würfel drehen. Sofern sie selbst nicht bei oder angrenzend zu Geistern steht, darf sie die Würfel von Gegnern drehen, die bei oder angrenzend zu Geistern stehen.}

\end{bgcolor}


\end{multicols*}








\chapter{37. Forn (Forr), der Halbskral}
\label{Halbskral}



\begin{multicols*}{2}\raggedcolumns


\begin{nobgcolor}

\section{Allgemein}


\subsection{Heldentafel:}

Der Halbskral hat Rang 37.

Seine offizielle Farbe ist Rot.

Er hat:

\bullet{} 2 Würfel bei 1 bis 13 Willenspunkten,

\bullet{} 3 Würfel bei 14 bis 20 Willenspunkten.



\subsection{Schuppenhelm:}

Im Kampf darf der Halbskral, \fan{wenn er seine Würfel gleichzeitig wirft,} gleiche Würfelwerte addieren.

\fan{Er kann diese Fähigkeit nicht mit dem Trank der Hexe kombinieren.}




\subsection{Skralgeschwindigkeit:}

Läuft der Halbskral entlang (nicht entgegen) eines Pfeils, darf er für diesen Schritt 1 Willenspunkt statt 1 Stunde abgeben. Er darf sich bei jedem Schritt neu entscheiden.




\subsection{Kreaturen täuschen:}


Steht der Halbskral auf einem Feld, auf dem nur eine Kreatur stehen kann, wird jede Kreatur, die auf sein Feld gestellt\supercite{DH:Begleitheft} oder bewegt wird, versetzt, als würde auf diesem Feld bereits eine Kreatur stehen.

Üblicherweise wird die Kreatur also auf das angrenzende Feld entlang des Pfeils versetzt (auch wenn ein Edelstein angrenzend liegt). 
\supercite{130318}

\fan{Von Kreaturen entfernbare Plättchen und Figuren (z.B. Bauern oder Edelsteine) auf dem Feld des Halbskrals gehen aus dem Spiel, wenn eine Kreatur ihn überspringt.}

Endgegner überspringen den Halbskral nicht.\supercite{DZ:Begleitheft}

\fan{Ist eine Blockade entlang der Pfeile (z.B. eine mit rotem X abgedeckte Brücke), überspringt die Kreatur den Halbskral nicht.}



\subsection{Liegende Kreaturen:}

\fan{Wird eine liegende Kreatur (z.B. durch das Bannpulver des \hyperref[Steppennomade]{Steppennomaden}, den Dunklen Tempel in "Der Sternenschild", die bronzene Ereigniskarte 232 in "Die letzte Hoffnung" oder Meres in "Alte Geister") aufgestellt, während der Halbskral auf ihrem Feld steht, wird die Kreatur nicht versetzt.}


\end{nobgcolor}


\columnbreak


\begin{nobgcolor}
\section{Neue Helden (2014)}

\subsection{Spielvariante "Der Trunkene Troll":}

\fan{Der Trunkene Troll überspringt den Halbskral nicht.}

\end{nobgcolor}



\begin{bgcolor}[color-DRidN]
\section{Die Reise in den Norden (2014)}


\subsection{\fan{Skralgeschwindigkeit:}}

\fan{Der Halbskral kann beim Segeln nie Willenspunkte statt Stunden abgeben.}


\subsection{\fan{Kreaturen täuschen:}}

\fan{Steht der Halbskral an Bord des Schiffes, wird eine Kreatur, die auf das Meeresfeld des Schiffes gestellt oder bewegt wird, auf das angrenzende Feld entlang des Pfeils versetzt.}


\end{bgcolor}




\begin{bgcolor}[color-DlH]
\section{Die letzte Hoffnung (2016)}

\subsection{Zweite Sonderfähigkeit: Rekas Pflanzenkunde}

Der Halbskral darf gepflücktes Sternkraut (keine Mondbeeren\supercite{MH:Mondbeeren,MH:Begleitheft}), das er trägt, wieder auf die ungepflückte Seite drehen und ablegen, damit mehr Sternkraut darauf wächst.
\supercite{DH11:Heldenkarte_Halbskral}

Er darf mehrere Sternkraut-Plättchen getrennt auf demselben Feld ablegen, damit sich ihre Anzahl bei Sonnenaufgang bis zu verdoppelt.
\supercite{DH:FAQ,DH:Begleitheft}

 


\subsection{Alte Zwergenstraße:}

Läuft der Halbskral entlang der Alten Zwergenstraße (egal in welche Richtung), darf er ebenfalls 1 Willenspunkt statt 1 Stunde abgeben.



\subsection{Legende 17, Die letzte Hoffnung:}

\fan{Wachtrolle werden nicht versetzt, wenn sie auf das Feld des Halbskrals gestellt oder bewegt werden. Wachtrolle halten den Halbskral wie andere Helden fest (während der Erzähler auf "j" oder vorher steht).} 

\end{bgcolor}




\begin{bgcolor}[color-BB]
\section{Die Bonus-Box (2017/2024)}



\subsection{Fluggors:}


\fan{Fluggors überspringen den Halbskral nicht, von Fluggors transportierte Kreaturen aber schon.}


\subsection{Legende 1, Der Angriff der Barbaren:}

\fan{Barbaren überspringen den Halbskral nicht.}

\end{bgcolor}


\columnbreak


\begin{bgcolor}[color-AG]
\section{Alte Geister (2018)}

\subsection{Legende 2, Der Hexer aus Andor:}

Der Halbskral startet immer auf Feld 39. Hat ein anderer Held den höchsten Rang, startet dieser auch auf Feld 39.

Shron und seine Kreaturen gelten als Endgegner und überspringen den Halbskral nicht.
\supercite{AG:Begleitheft}

\end{bgcolor}




\begin{bgcolor}[color-DZ]
\section{Düstere Zeiten (2020)}

\subsection{Pfeilplättchen:}

Durch die Pfeilplättchen ändert sich die Richtung, in die der Halbskral durchs Abgeben von Willenspunkten laufen kann.
\supercite{DZ:Begleitheft}


\subsection{Die drei großen Trolle:}

Der Halbskral wird nicht durch die Effekte der großen Trolle beeinflusst.
\supercite{DZ:Begleitheft}

\subsection{Tiefminen-Golems:}

Tiefminen-Golems überspringen den Halbskral nicht.
\supercite{DZ:Begleitheft}

\end{bgcolor}



\begin{bgcolor}[color-DeK]
\section{\fan{Die ewige Kälte (2022)}}

\subsection{\fan{Höhle der Verwandlung:}}

\fan{Da auf der Höhle der Verwandlung (Feld 414) mehrere Kreaturen stehen dürfen, wird der Halbskral dort nicht übersprungen.} 

\end{bgcolor}




\begin{bgcolor}[color-DfL]
\section{\fan{Das ferne Land (2025)}}

\subsection{\fan{Pfad:}}

\fan{Läuft der Halbskral entlang des Pfads(egal in welche Richtung), darf er ebenfalls 1 Willenspunkt statt 1 Stunde abgeben.}


\subsection{\fan{Alwerons Erbe und seine Wachen:}}

\fan{Wachtrolle und der Trollkönig überspringen den Halbskral nicht. In Legende 6, "Die Höhle", halten Wachtrolle den Halbskral wie andere Helden fest.} 


\subsection{\fan{Geister:}}

\fan{Steht der Halbskral bei oder angrenzend zu Geistern, kann er nicht mit seinem Schuppenhelm gleiche Würfelwerte addieren.}

\fan{Betritt oder verlässt er ein Feld mit oder angrenzend zu Geistern, kann er für diesen Schritt nicht 1 Willenspunkt statt 1 Stunde abgeben.}

\fan{Kreaturen überspringen ihn, auch wenn er bei oder angrenzend zu Geistern steht.}

\end{bgcolor}


\end{multicols*}










\chapter{38. Barz (Yaza), der Steppennomade}
\label{Steppennomade}

\begin{multicols*}{2}\raggedcolumns


\begin{nobgcolor}

\section{Allgemein}




\subsection{Aufbau:}

Der Steppennomade erhält 3 Pulver und steckt 2 weitere Pulver auf Sabri.
\supercite{MH:Begleitheft}

Stellt Sabri zum Steppennomaden.
\supercite{MH:Begleitheft}



\subsection{Heldentafel:}

Der Steppennomade hat Rang 38.

Seine offizielle Farbe ist Rosa.

Er hat 3 besondere Ablagefelder für Pulver.

Er hat 4 Würfel, egal wie viele Willenspunkte er hat. 




\subsection{Rosa Würfel:}

Jeder Heldenwürfel des Steppennomaden hat die Seiten 2/3/3/4/4/5 (statt 1/2/3/4/5/6). Sie können nicht genutzt werden, wenn eine Legende das Werfen eines beliebigen Heldenwürfels außerhalb eines Kampfes verlangt (z.B. für das Bestimmen einer Feldzahl).
\supercite{88552}


\subsection{Bogen:}

Der Steppennomade ist ein \hyperref[Fernkämpfer]{Fernkämpfer}.



\subsection{Kein Bogenschütze der Bewahrer:}

Spricht eine Regel oder Karte von der \hyperref[Bogenschützin]{Bogenschützin}, ist der Steppennomade nicht gemeint.



\subsection{Treues Lasttier Sabri:}

Der Steppennomade hat eine \hyperref[Helfer]{\fan{Helferin}}, die Figur "Sabri".  Bewegt sich der Erzähler (\fan{egal ob vorwärts oder rückwärts}), läuft Sabri 1 Schritt auf dem kürzesten Weg zum Steppennomaden.
\supercite{MH:Steppennomade}
Sie ignoriert Kreaturen.
\supercite{MH:Begleitheft}
\fan{Gibt es keinen Weg zum Steppennomaden, läuft Sabri auf ein beliebiges angrenzendes Feld.}






\subsection{Magische Pulver:}

\fan{Pulver gelten nicht als Gegenstände. Der Steppennomade kann sie nicht auf einem Feld ablegen.} Steht der Steppennomade auf Sabris Feld, darf er Pulver von seiner Heldentafel mit Sabri tauschen.
\supercite{MH:Steppennomade,MH:Begleitheft}

Der Steppennomade darf als freie Handlungsmöglichkeit ein Pulver von seiner Heldentafel nutzen, indem er einen seiner Heldenwürfel bis zum Tagesende abgibt.
\supercite{MH:Begleitheft}
Er kann jedes Pulver höchstens 1x pro Tag nutzen, benutzte Pulver aber immer noch tragen oder tauschen.
\supercite{88552}
\fan{So abgegebene Würfel fehlen ihm, auch wenn der \hyperref[Fährtenleser]{Fährtenleser} sein Horn nutzt}.

Hat er alle seine Heldenwürfel abgegeben, addiert er bis zum Tagesende keinen Würfelwert zu seinem Kampfwert. 



\columnbreak



\subsection{Vorhersehungspulver (türkis):}

In einer Kampfrunde gegen einen Gegner (auch Endgegner\supercite{90008}), direkt bevor der Steppennomade Würfel werfen würde: Bestimmt bereits den gegnerischen Kampfwert (üblicherweise durch Würfeln).
\supercite{MH:Pulver}
Der Steppennomade und alle folgenden Helden entscheiden einzeln, ob sie den Kampf verlassen wollen. Der Steppennomade muss in jedem Fall die Stundenkosten für die Kampfrunde zahlen. Folgende Helden müssen dies nur, wenn sie weiterkämpfen. Eröffnete der Steppennomade den Kampf und verlassen alle Helden den Kampf, wird jeder abgebrochen. Wird der Kampf nicht abgebrochen, bestimmen alle folgenden Helden, die den Kampf nicht verlassen haben, in Zugreihenfolge ihren Kampfwert (üblicherweise durch Würfeln), und der Gesamtkampfwert der Helden wird wie üblich mit dem (bereits bestimmten) Kampfwert des Gegners verglichen.
\supercite{89624,89625,90058}




\subsection{Bannpulver (grün):}

Der Steppennomade darf eine gegnerische Figur ablegen, die auf seinem Feld steht und auf dem Sonnenaufgang-Feld abgebildet ist\supercite{90058}. Das nächste Mal, wenn dieses Sonnenaufgang-Symbol ausgeführt wird, bewegt sich die Figur nicht, sondern wird aufgestellt\supercite{MH:Pulver} (\fan{und nicht weiterversetzt, auch wenn der \hyperref[Halbskral]{Halbskral} auf ihrem Feld steht}).
\fan{Liegende Kreaturen können nicht bewegt, bekämpft oder durch andere Figuren ersetzt werden. Liegt eine Kreatur auf einem Feld, auf dem nur eine Kreatur stehen kann, wird jede andere Kreatur, die auf ihr Feld gestellt oder bewegt wird, weiterversetzt, als würde die liegende Kreatur stehen. Von Kreaturen entfernbare Plättchen oder Figuren (z.B. Bauern oder Edelsteine) bleiben im Spiel, wenn eine Kreatur auf ihrem Feld liegt, gehen jedoch aus dem Spiel, wenn eine andere Kreatur ihr Feld überspringt.}





\subsection{Meditationspulver (braun):}

Der Steppennomade darf die Position seines Zeitsteins mit dem eines anderen Helden tauschen. Dafür müssen sich beide Zeitsteine auf der Tagesleiste befinden (nicht dem Sonnenaufgang-Feld\supercite{MH:Begleitheft}) und alle betroffenen Spieler einverstanden sein.
\supercite{MH:Pulver}
Das geht nicht während einer Aktion eines Helden.
\supercite{88552}
Im Spiel mit 4 neutralen Zeitsteinen hat das Meditationspulver keinen Effekt.


\subsection{Umwandlungspulver (gelb):}

Der Steppennomade gibt einen kleinen Gegenstand aus dem Spiel, den er auf einem kleinen rechteckigen Ablagefeld oder auf seiner Helmablage trägt, \fan{egal auf welcher Seite der Gegenstand liegt}. Danach darf er einen beliebigen Gegenstand von der Ausrüstungstafel erhalten.
\supercite{MH:Pulver}
Das geht nicht während einer Aktion eines Helden.
\supercite{88552}




\subsection{Schwächungspulver (blau):}

Am Anfang der ersten Kampfrunde eines Kampfs, an dem der Steppennomade beteiligt ist, gegen eine Kreatur (kein Endgegner\supercite{89624,90008}), die 4 oder mehr Willenspunkte hat\supercite{88552,89625}: Sie verliert 3 Willenspunkte. Der Steppennomade darf dann \fan{bis zu} 3 Willenspunkte auf die kämpfenden Helden verteilen. 
\supercite{MH:Pulver}






\subsection{Experimentierpulver (rosa):}

Spielvariante, erfindet eure eigene Fähigkeit.
\supercite{MH:Steppennomade,MH:Pulver,MH:Begleitheft}




\subsection{Bonuswürfel:}

Der Steppennomade kann keinen Gegnerwürfel (z.B. durch den Streifenmarder in "Die Reise in den Norden" oder Meres in "Die steinernen Drei") und keinen Ersatzwürfel (z.B. den großen weißen Würfel der \hyperref[Hüterin]{Hüterin}) für Pulver abgeben.

Hat er all seine Heldenwürfel abgegeben, kann er bis zum Tagesende auch keinen Ersatzwürfel statt seiner Heldenwürfel werfen.
\supercite{94520} 
\fan{Ausnahme: Hat er einen Gegnerwürfel, darf er den Ersatzwürfel statt des Gegnerwürfels werfen.}







\subsection{\fan{Mehrere Steppennomaden:}}

\fan{Jeder Steppennomade hat seine eigene Sabri, die sich nur auf ihn zubewegt. Ihr dürft den Sabris Standfüße in unterschiedlichen Farben geben.}

\fan{Jeder Steppennomade wählt am Anfang, welche 3 verschiedenen Pulver er erhält und welche 2 anderen verschiedene Pulver er auf seine Sabri steckt. Pulver verschiedener Steppenomaden sind aber ununterscheidbar. Jeder Steppennomade darf jedes Pulver tragen und nutzen sowie miteinander und mit jeder Sabri tauschen.}


\end{nobgcolor}




\begin{bgcolor}[color-DLvA]
\section{Die Legenden von Andor (2012)}

\subsection{Umwandlungspulver (gelb):}

Der Steppennomade kann durch das Umwandlungspulver keinen Trank der Hexe erhalten.
\supercite{MH:Pulver}







\subsection{Ausscheiden:}

\fan{Scheidet der Steppenomade aus (z.B. weil er in Cavern auf 0 Willenspunkte fiel), gehen alle von ihm getragenen Pulver, seine Sabri und von seiner Sabri getragene Pulver aus dem Spiel.}


\subsection{Legende 3, Die Tage des Widerstands:}

\fan{Nutzt Varkur die Dunkle Magie "Die Schwächung der Helden", wirft der Steppennomade in jeder Endkampfrunde nur einen rosa Würfel und dreht ihn auf die gegenüberliegende Seite, wenn diese kleiner ist. Gab er bereits alle Würfel ab, wirft er stattdessen gar keinen Würfel.}


\fan{Bei dem Dunklen Magier "Der verhexte Prinz" (aus "Varkurs Erwachen") löst das Meditationspulver keine Prinz-Bewegung aus, auch wenn es einen Zeitstein auf der 3. Stunde austauscht.}


\subsection{Legende 5, Der Zorn des Drachens:}

\fan{Bei der Drachenkampfkarte "Gegenstand in 2 Stärkepunkte tauschen" kann der Steppennomade kein Pulver in Stärke tauschen.}


\subsection{Goldene Ereigniskarte 6:}

\fan{Der Steppennomade wirft einen normalen Heldenwürfel, nicht einen rosa Würfel.}



\subsection{Ereigniskarte "Geheimer See" 9:}

\fan{Der Steppennomade wirft 4 rosa Würfel, keine normalen Heldenwürfel, egal, wie viele Würfel er für Pulver abgab.}



\end{bgcolor}


\begin{bgcolor}[color-StSch]
\section{Der Sternenschild (2013)}

\subsection{Andorische Flöte:}

\fan{Nutzt der Steppennomade die andorische Flöte, wirft er einen seiner rosa Würfel, keinen normalen Heldenwürfel. Das geht, auch wenn er alle Würfel für Pulver abgab.}

\subsection{Wolfssymbol:}

\fan{Das Meditationspulver löst nie das Wolfssymbol aus, auch wenn es einen Zeitstein auf der Stunde des Wolfssymbols austauscht.}


\subsection{Bedrohung 3, Der Dunkle Tempel:}

\fan{Markiert durch das Bannpulver hingelegte Kreaturen (z.B. mit Sternchen).} Durch den Dunklen Tempel hingelegte Kreaturen werden beim Brunnen-Symbol aufgestellt, \fan{durch das Bannpulver hingelegte Kreaturen hingegen bei ihrem Sonnenaufgang-Bewegung-Symbol.}


\end{bgcolor}




\begin{nobgcolor}
\section{Neue Helden (2014)}

\subsection{Spielvariante "Der Trunkene Troll":}

\fan{Das Bannfeuer kan den Trunkenen Troll nicht hinlegen.}

\end{nobgcolor}




\begin{bgcolor}[color-DRidN]
\section{Die Reise in den Norden (2014)}


\subsection{Treues Lasttier Sabri:}

Startet der Steppennomade auf dem Schiff, startet Sabri auch auf dem Schiff. 
\supercite{MH:Begleitheft}


\subsection{Bannpulver (grün):}

Der Steppennomade darf eine Kreatur auf einem Meeresfeld hinlegen, wenn das Schiff auf ihrem Feld steht.
\supercite{MH:Begleitheft}




\subsection{Umwandlungspulver (gelb):}

\fan{Das Umwandlungspulver kann Magische Waffen nicht umwandeln.}



\subsection{Schwächungspulver (blau):}

Das Schwächungspulver darf vor oder nach dem Einsatz von Nixenstaub genutzt werden. Nachher geht das nur, wenn die Kreatur nach dem Nixenstaub 4 oder mehr Willenspunkte hat.
\supercite{MH:Begleitheft,88552}




\subsection{Legende 9, Die Mächte des Meeres:}

\fan{Für Kenvilars Tücke "auf die 1" werden auch rosa Würfel als 1 gewertet.}
\supercite{15179}


\subsection{Legende "Die Rückkehr der Schwarzen Kogge":}

\fan{Wirft der Unbekannte Krieger die Würfel des Steppennomaden, wirft er alle vier Würfel, egal wie viele Würfel der Steppennomade an diesem Tag für Pulver abgab.}

\end{bgcolor}



\columnbreak


\begin{bgcolor}[color-DlH]
\section{Die letzte Hoffnung (2016)}




\subsection{Zweite Sonderfähigkeit: Quellentausch}

Betritt der Steppennomade ein Feld mit einer farbigen Quelle, darf er alle farbigen Quellen auf die graue Seite drehen, mischen, zufällig verteilen und wieder auf die farbige Seite drehen. Erst danach entscheidet er, \fan{ob er stehen bleibt oder nicht und} ob er die Quelle auf seinem Feld leert oder nicht.
\supercite{MH11:Heldenkarte_Steppennomade,94521}



\subsection{Bannpulver (grün):}

Das Bannpulver darf auch ein Skelett hinlegen. Dieses gilt dann als überwunden und bleibt bis zum nächsten Sonnenaufgang liegen.
\supercite{MH:Begleitheft,88552}

Wachtrolle, Krahder oder der Urtroll können nicht hingelegt werden, da sie nicht auf dem Sonnenaufgang-Feld abgebildet sind.
\supercite{90058}

\fan{Läuft der Urtroll auf das Feld einer liegenden Kreatur, geht sie auch aus dem Spiel.}


\subsection{Meditationspulver (braun):}

\fan{Das Meditationspulver löst nie Bewegungsplättchen aus, auch wenn es einen Zeitstein auf dem Bewegungsplättchen-Stapel austauscht.}




\subsection{Umwandlungspulver (gelb):}

Der Steppennomade kann durch das Umwandlungspulver keinen Trank der Hexe\supercite{MH:Pulver} und keine Nahrung erhalten.
\supercite{88552}




\subsection{Schwächungspulver (blau):}

Das Schwächungspulver darf auch Skelette schwächen.
\supercite{MH:Begleitheft}

Krahder können nicht geschwächt werden.
\supercite{88552}

Das Schwächungspulver darf vor einem Einsatz des Feuerschilds genutzt werden, aber nicht nachher (da die Kreatur dann höchstens 3 Willenspunkte hat).
\supercite{91413}






\subsection{Legende 14, Der Meister des Trolls:}

Wird bei seiner offiziellen Erschöpfungskarte ein drittes Pulver zu Sabri gelegt, wird dieses mit Sabri mitbewegt, auch wenn sie eigentlich nur Platz für zwei Pulver hat.
\supercite{94521}


\subsection{Legende 15, Der vergiftete Geist:}

Bei seiner offiziellen Geschenkaufgabe darf der Steppennomade ausnahmsweise Pulver auf Feld 219 ablegen.

\fan{Wird der Steppennomade enttarnt, gehen alle von ihm getragenen Pulver, seine Sabri und von seiner Sabri getragene Pulver aus dem Spiel. Er kann Pulver nicht in Stärke tauschen.}


\end{bgcolor}



\begin{nobgcolor}

\section{Dunkle Helden (2017)}

\subsection{Spielvariante "Merriks Karte":}

\fan{Das Meditationspulver darf auch die Position eines farbigen Zeitsteins und Merriks neutralen Zeitsteins tauschen. Dafür müssen nur die Spieler der Steppennomaden und des Helds mit dem farbigen Zeitstein einverstanden sein.}

\end{nobgcolor}


\columnbreak




\begin{bgcolor}[color-BB]
\section{Die Bonus-Box (2017/2024)}


\subsection{Bannpulver (blau):}

Das Bannpulver kann keine Barbaren oder Fluggors hinlegen, da sie nicht auf dem Sonnenaufgang-Feld abgebildet sind.
\supercite{90058}

\fan{Eine vom Zwergenseil gefesselte Kreatur kann nicht hingelegt werden. Eine liegende Kreatur kann nicht vom Zwergenseil gefesselt werden.}


\subsection{Umwandlungspulver (gelb):}

Der Steppennomade kann durch das Umwandlungspulver keine Gegenstände von der kleinen Ausrüstungstafel der weißen Ereigniskarten erhalten.
\supercite{88659}

\end{bgcolor}



\begin{bgcolor}[color-AG]
\section{Alte Geister (2018)}




\subsection{Bannpulver (blau):}

Der Takuri-Spiegel darf auch eine liegende Kreatur versetzen. Wird sie auf ein Feld versetzt, auf dem bereits eine Kreatur steht oder liegt, wird sie auf das angrenzende Feld entlang des Pfeils versetzt. Wird sie direkt auf die Rietburg (Feld 0) versetzt, wird sie auf einen goldenen Schild gestellt. Wird sie direkt auf das Lager der Tulgori (Feld 18) versetzt, ist die Legende verloren.
\supercite{AG:Begleitheft}



\subsection{Meditationspulver (braun):}

\fan{Das Meditationspulver löst nie Kartographie-Marker aus, auch wenn es einen Zeitstein auf dem Kartographie-Marker-Stapel austauscht.}


\subsection{Umwandlungspulver (gelb):}

Der Steppennomade kann durch das Umwandlungspulver keine Gegenstände von der Karte "Tulgorischer Handel" erhalten.
\supercite{88659}



\subsection{Legende 2, Der Hexer von Andor:}

Shron ignoriert liegende Kreaturen. Das Feld einer liegenden Kreatur wird bei seiner Bewegung wie ein leeres Feld behandelt. Kreaturen, die Teil seines Heeres sind, können nicht hingelegt werden.
\supercite{AG:Begleitheft}

\fan{Ist der Endkampf gegen Siantari und auf Feld 30, muss der Steppennomade keinen Willenspunkt abgeben, um einen Würfel für ein Pulver abzugeben.}



\subsection{Bonus-Legende "Die vierte Statue":}

\fan{Es gibt keine Ausrüstungstafel. Mit dem Umwandlungspulver darf der Steppennomade einen Gegenstand von Feld 71 erhalten.}

\end{bgcolor}





\columnbreak


\begin{bgcolor}[color-DZ]
\section{Düstere Zeiten (2020)}

\subsection{Bannpulver (blau):}

\fan{Das Bannpulver darf Kreaturen hinlegen, die auf einem Bewegungsplättchen abgebildet sind, auch wenn sie nicht auf dem Sonnenaufgang-Feld abgebildet sind. Diese werden dann jeweils bei der nächsten Ausführung ihres Bewegungsplättchen wieder aufgestellt statt bewegt, egal wie lange nach dem Hinlegen dies geschieht. }

\subsection{Meditationspulver (braun):}

\fan{Das Meditationspulver löst nie Geröll oder Thogger aus, auch wenn es einen Zeitstein auf der Stunde der Geröllplättchen oder des Thogger-Plättchens austauscht.}



\subsection{Umwandlungspulver (gelb):}

\fan{Das Umwandlungspulver kann die Spitzhacke nicht umwandeln.}

\end{bgcolor}




\begin{bgcolor}[color-DeK]
\section{\fan{Die ewige Kälte (2022)}}

\subsection{\fan{Magische Pulver:}}

\fan{Der Steppennomade kann Pulver nicht bei Händlern tauschen.}

\subsection{\fan{Bannpulver (blau):}}

\fan{In der Höhle der Verwandlung (Feld 414) stehende Kreaturen können nicht mit dem Bannpulver hingelegt werden.}



\subsection{\fan{Umwandlungspulver (gelb):}}


\fan{Es gibt keine Ausrüstungstafel. Mit dem Umwandlungspulver darf der Steppennomade einen Gegenstand aus dem Vorrat oder von einem Händlerfeld erhalten, der auf einem Händlerfeld abgebildet ist.}

\fan{Wintersteine können nicht umgewandelt werden.}


\subsection{\fan{Große Kräuterkunde:}}

\fan{Der Steppennomade darf durch Blaubachbeeren + rotes Blutkraut den Würfelwert 10 erhalten, auch wenn er alle seine Heldenwürfel abgegeben hat.}

\end{bgcolor}


\columnbreak



\begin{bgcolor}[color-DfL]
\section{\fan{Das ferne Land (2025)}}


\subsection{\fan{Bannpulver (grün):}}

\fan{In Azturien (Spielplan-Vorderseite), wenn Ereigniskarten auf dem Ereigniskarten-Feld liegen, darf das Bannpulver Gors, Gorlots, Kreideskrale und Trolle hinlegen, egal welche auf der obersten Ereigniskarte abgebildet sind. Liegende Kreaturen werden bei der nächsten Ausführung ihres Symbols auf einer Ereigniskarte aufgestellt, egal wie viele Tage nach dem Hinlegen dies geschieht. Wachtrolle und Sumpftrolle können dann nicht hingelegt werden, da sie nicht auf Ereigniskarten abgebildet sind.}

\fan{In Azturien (Spielplan-Vorderseite), wenn keine Ereigniskarten auf dem Ereigniskarten-Feld liegen, darf das Bannpulver alle Kreaturen hinlegen, die auf dem Ereigniskartenfeld abgebildet sind, auch Wachtrolle und Sumpftrolle.}





\subsection{\fan{Umwandlungspulver (gelb):}}

\fan{Die Greifenkrone kann nicht umgewandelt werden.}



\subsection{\fan{Geister:}}

\fan{Steht der Steppennomade bei oder angrenzend zu Geistern, kann er keine Pulver nutzen. Er darf Pulver gegen Gegner einsetzen, die angrenzend zu Geistern stehen, solange er selbst nicht angrenzend zu Geistern steht.}

\fan{Der Steppennomade kann Pulver mit Sabri tauschen, auch wenn sie bei oder angrenzend zu Geistern stehen.}

\end{bgcolor}


\end{multicols*}












\chapter{39. Raash (Raashana), der Steppenreiter}
\label{Steppenreiter}

\begin{multicols*}{2}\raggedcolumns

\begin{nobgcolor}

\section{Allgemein}

\subsection{Heldentafel:}

Der Steppenreiter hat Rang 39.

Seine offizielle Farbe ist Braun.

Er hat höchstens 19 Willenspunkte (statt 20).

Er hat:

\bullet{} 1 Würfel bei 1 bis 4 Willenspunkten,

\bullet{} 2 Würfel bei 5 bis 9 Willenspunkten,

\bullet{} 3 Würfel bei 10 bis 14 Willenspunkten,

\bullet{} 4 Würfel bei 15 bis 19 Willenspunkten.






\subsection{Treuer Steppenbüffel Mamoru:}

Der Steppenreiter ist ein \hyperref[Reiter]{Reiter}.







\subsection{Bogen:}

Der Steppenreiter ist ein \hyperref[Fernkämpfer]{Fernkämpfer}.




\subsection{Kein Bogenschütze der Bewahrer:}

Spricht eine Regel oder Karte von der \hyperref[Bogenschützin]{Bogenschützin}, ist der Steppenreiter nicht gemeint.


\end{nobgcolor}



\end{multicols*}









\chapter{50. Arbon (Talvora), der Bewahrer}
\label{Bewahrer}


\begin{multicols*}{2}\raggedcolumns

\begin{nobgcolor}

\section{Allgemein}



\subsection{Heldentafel:}

Der Bewahrer hat Rang 50.

Seine offizielle Farbe ist Grau.

Er hat:

\bullet{} 1 Würfel bei 1 bis 6 Willenspunkten,

\bullet{} 2 Würfel bei 7 bis 13 Willenspunkten,

\bullet{} 3 Würfel bei 14 bis 20 Willenspunkten.




\subsection{Arcuballiste:}

Der Bewahrer ist ein \hyperref[Fernkämpfer]{Fernkämpfer}.


\subsection{Nur der drittbeste Bogenschütze:}

Spricht eine Regel oder Karte von der \hyperref[Bogenschützin]{Bogenschützin}, ist der Bewahrer nicht gemeint.






\subsection{Geheimwissen aus dem Schwarzen Archiv:}

Der Bewahrer darf sich vor der ersten Kampfrunde jedes Kampfes gegen einen auf der Kreaturenanzeige abgebildeten Gegner (kein Endgegner) entscheiden, den Gegner zu schwächen. 
\supercite{NH:Begleitheft}
Verschiebt den Stärkestein des Gegners auf das nächst linke Stärkefeld der Kreaturenanzeige \textit{mit einer kleineren Zahl}. Er hat neu so viele Stärkepunkte und gibt die Belohnung des Gegners, der über seinem Stärkestein abgebildet ist. Ist da kein Gegner abgebildet, gibt er die Belohnung des nächst linken auf der Kreaturenanzeige abgebildeten Gegner.
\supercite{DRidN:Begleitheft}
Sind über und links vom Stärkestein keine Gegner abgebildet, gibt er gar keine Belohnung. Wird der Gegner nicht besiegt, ist er im nächsten Kampf nicht mehr geschwächt (außer er wird erneut geschwächt).





\subsection{\fan{Mehrere Bewahrer:}}

\fan{Jeder Gegner kann in einem Kampf höchstens einmal geschwächt werden.}

\end{nobgcolor}



\begin{bgcolor}[color-DLvA]
\section{Die Legenden von Andor (2012)}

\subsection{Trolle:}

Der Bewahrer kann keine Trolle schwächen. \fan{Dies gilt nur für Legenden auf einem Grundspiel-Spielplan, nicht für Legenden auf anderen Spielplänen.}



\subsection{Mini-Erweiterung "Koram, der Gor-Häuptling":}

\fan{Der Bewahrer kann Koram und von Koram gestärkte Gors schwächen. Die beiden Stärkestein-Verschiebungen heben sich auf und der Gor hat seine normalen Werte und Belohnung.}



\subsection{Mini-Erweiterung "Die Taverne von Andor":} 

\fan{Der Bewahrer kann eine von der Taverne geschwächte Kreatur nicht weiter schwächen.}

\end{bgcolor}



\begin{bgcolor}[color-StSch]
\section{Der Sternenschild (2013)}

\subsection{Der Dieb Ken Dorr:}

\fan{Der Bewahrer kann den Dieb Ken Dorr nicht schwächen.}

\end{bgcolor}



\begin{bgcolor}[color-DlH]
\section{\fan{Die letzte Hoffnung (2016)}}

\subsection{\fan{Skelette:}}

\fan{Im Grauen Gebirge (Spielplan-Vorderseite) sinken geschwächte Skelette von 3 auf 2 Stärkepunkte und geben als Belohnung 3 Willenspunkte.}


\fan{In Krahd (Spielplan-Rückseite) sinken geschwächte Skelette von 1 Stärkepunkt auf 0 und von 3 Stärkepunkten auf 1. In beiden Fällen geben sie keine Belohnung.}

\end{bgcolor}



\begin{bgcolor}[color-BB]
\section{Die Bonus-Box (2017/2024)}

\subsection{Fluggors:}

\fan{Der Bewahrer kann keine Fluggors schwächen.}

\end{bgcolor}



\begin{bgcolor}[color-AG]
\section{Alte Geister (2018)}

\subsection{Maasavi:}

Der Bewahrer kann keine Maasavi schwächen.
\supercite{AG:Begleitheft}

\end{bgcolor}



\begin{bgcolor}[color-DZ]
\section{Düstere Zeiten (2020)}


\subsection{Riesentroll:}

\fan{Der Bewahrer darf eine durch den Riesentroll gestärkte Kreatur schwächen (außer gestärkte Trolle). Die beiden Stärkestein-Verschiebungen heben sich auf und die Kreatur gibt ihre normale Belohnung.}

\end{bgcolor}






\begin{bgcolor}[color-DfL]
\section{\fan{Das ferne Land (2025)}}


\subsection{\fan{Geister:}}

\fan{Steht der Bewahrer bei oder angrenzend zu Geistern, kann er keine Gegner schwächen. Er darf Gegner schwächen, die angrenzend zu Geistern stehen, sofern er selbst nicht angrenzend zu Geistern steht.}


\subsection{\fan{Hoffnungsstein Nr. 3:}}

\fan{Ein Gegner kann in einem Kampf höchstens einmal vom Bewahrer oder Hoffnungsstein Nr. 3 geschwächt werden, nicht von beiden.}



\subsection{\fan{Mini-Erweiterung "Der Mhourl":}}

\fan{Der Bewahrer kann den Mhourl nicht schwächen.}



\end{bgcolor}


\end{multicols*}





























\chapter{52. Iril (Ilar), die Runenmeisterin}
\label{Runenmeisterin}


\begin{multicols*}{2}\raggedcolumns


\begin{nobgcolor}

\section{Allgemein}


\subsection{Aufbau:}

Legt die Runenscheibe mit Speichenwert 1 bereit.
\supercite{MH:Begleitheft}


\subsection{Heldentafel:}

Die Runenmeisterin hat Rang 52.

Ihre offizielle Farbe ist Dunkelblau.

Sie hat höchstens 12 Stärkepunkte (statt 14).

Sie hat 2 kleine rechteckige Ablagefelder (statt 3).

Sie hat 1 Würfel, egal wie viele Willenspunkte sie hat: Den Runenwürfel. 


\subsection{Runenwürfel:}

Der Runenwürfel hat die Seiten 1/1/2/2/3/3 (statt 1/2/3/4/5/6). Er kann nicht genutzt werden, wenn eine Legende das Werfen eines beliebigen Heldenwürfels außerhalb eines Kampfes verlangt (z.B. für das Bestimmen einer Feldzahl).
\supercite{88552}



\subsection{Runenscheibe:}

Die Runenmeisterin hat eine Runenscheibe, die aus 9 Speichen besteht. Jede Speiche hat einen Speichenwert und ein Speichensymbol. Ihre Reihenfolge ist:

\bullet{} Speichenwert 1 / blauer Runenstein

\bullet{} Speichenwert 2 / Auge

\bullet{} Speichenwert 3 / 3 Willenspunkte

\bullet{} Speichenwert 4 / gelber Runenstein

\bullet{} Speichenwert 5 / Auge

\bullet{} Speichenwert 2 / Brunnen

\bullet{} Speichenwert 3 / grüner Runenstein

\bullet{} Speichenwert 4 / 3 Willenspunkte

\bullet{} Speichenwert 5 / Brunnen

Wann immer die Runenmeisterin "ihre Runenscheibe dreht", wirft sie den Runenwürfel und dreht ihre Runenscheibe um den Runenwürfel-Wert, sowie nach Wahl:

\bullet{} bei 7 bis 13 Willenspunkten bis zu\supercite{89598} 1 Speiche weiter,

\bullet{} bei 14 bis 20 Willenspunkten bis zu\supercite{89598} 2 Speichen weiter.

Die Runenscheibe kann nie rückwärts gedreht werden.
\supercite{MH:Begleitheft}




\subsection{Kampf:}

Die Runenmeisterin wertet jeweils nicht den Runenwürfel-Wert. Stattdessen dreht sie ihre Runenscheibe und wertet den erhaltenen Speichenwert.

Dieser Speichenwert wird wie ein Würfelwert behandelt und darf mit dem Trank der Hexe verdoppelt werden.
\supercite{90045}

Die \hyperref[Zauberin]{Zauberin} darf den Runenwürfel drehen\supercite{88563} ($1 \leftrightarrow 3$), aber nur, wenn er in einem gemeinsamen Kampf mit ihr geworfen wird. Die Runenscheibe kann sie nie drehen.

\fan{Das Horn des \hyperref[Fährtenleser]{Fährtenlesers} hat keinen Effekt auf die Runenmeisterin.}



\subsection{Bonuswürfel:}

Hat die Runenmeisterin im Kampf einen Gegnerwürfel (z.B. durch den Streifenmarder in "Die Reise in den Norden" oder Meres in "Die steinernen Drei"), wirft sie diesen unabhängig vom Runenwürfel und dem Runenscheiben-Drehen. Den zusätzlichen Würfelwert und den Speichenwert wertet sie wie zwei Würfelwerte (üblicherweise wird nur einer zum Kampfwert addiert, aber mit einem Helm dürfen zwei gleiche addiert werden).
\supercite{89633}

\fan{Im \hyperref[Fernkämpfer]{Fernkampf} muss sie zuerst ihre Runenscheibe drehen und sich danach entschieden, ob sie beim erhaltenen Speichenwert aufhört oder den Gegnerwürfel wirft.}


\fan{Wirft die Runenmeisterin statt ihres Heldenwürfels einen Ersatzwürfel (z.B. den großen weißen Würfel der \hyperref[Hüterin]{Hüterin}), wird die Runenscheibe nicht gedreht und kein Speichenwert gewertet, sondern der Wert des Ersatzwürfels.} 






\subsection{Runen befragen:}

Nur die Runenmeisterin kann die Aktion "Runen befragen" wählen. Dann gibt sie 1 Stunde ab, dreht danach ihre Runenscheibe und aktiviert den Effekt des erhaltenen Speichensymbols.\supercite{MH:Runen,89746} Nach jedem Effekt entscheidet sie jeweils, ob sie ihre Aktion beendet oder 1 weitere Stunde abgibt, um erneut ihre Runenscheibe zu drehen und den Effekt des erhaltenen Speichensymbol zu aktivieren. Zwischen jedem Drehen dürfen freie Handlungsmöglichkeiten ausgeführt werden (z.B. darf die Runenmeisterin einen Runenstein aufnehmen oder ein Held einen aufgefrischten Brunnen leeren).
\supercite{88563}


Die Effekte der Speichensymbole sind:\supercite{MH:Runen,MH:Begleitheft}

\bullet{} Anzahl Willenspunkte: Die Runenmeisterin erhält so viele Willenspunkte.

\bullet{} Auge: Die Runenmeisterin wählt ein Spielplanfeld und deckt \fan{beliebig viele} \hyperref[Aufdecken]{\fan{von fern aufdeckbare}}\supercite{88631} Plättchen auf diesem Feld auf. Aktiviert werden diese Plättchen dadurch nicht. Plättchen in Stapeln dürfen auch aufgedeckt werden, ihre Reihenfolge bleibt aber gleich.

\bullet{} Brunnen: Die Runenmeisterin darf 1 geleerten Brunnen auf dem Spielplan auffrischen, auch wenn ein Held auf seinem Feld steht.
\supercite{94130}

\bullet{} Runenstein: Trägt die Runenmeisterin mindestens einen Runenstein der passenden Farbe, erhält sie 1 Stärkepunkt.
\supercite{89598}







\subsection{Keine Schildzwergin:}

Spricht eine Regel oder Karte vom \hyperref[Zwerg]{Zwerg}, ist die Runenmeisterin nicht gemeint.




\columnbreak




\subsection{\fan{Zusätzliche Runensteine:}}

\fan{Ihr dürft am Anfang mancher Legenden, in denen keine Runensteine eingewürfelt würden, dennoch Runensteine einwürfeln. Mischt dann verdeckt drei zusätzliche Runensteine (blau, gelb und grün; wie üblich kleine rechteckige Gegenstände, die von anderen Helden getragen werden dürfen). Bestimmt mit je 1 Kreaturenwürfel (10er-Stelle) und Heldenwürfel (1er-Stelle) die Positionen von 2 dieser Runensteine. Nehmt den dritten Runenstein verdeckt aus dem Spiel.}
\supercite{128652}

\fan{Das gilt für:}


\bullet{} \fan{Legenden 1 bis 3 aus "Alte Geister" (aber nicht die Bonus-Legende "Die vierte Statue").}

\bullet{} \fan{Legende 2 aus "Düstere Zeiten" und die Bonus-Legende "Der Marsch der Trolle".}

\bullet{} \fan{Alle Legenden auf einem Spielplan aus "Die ewige Kälte" oder "Das ferne Land".}

\fan{Das gilt nicht für Legenden auf einem Spielplan aus "Die Reise in den Norden" oder "Die letzte Hoffnung", weil die Runenmeisterin dort eine zusätzliche Fähigkeit hat.}




\subsection{\fan{Mehrere Runenmeisterinnen:}}

\fan{Jede Runenmeisterin hat ihre eigene Runenscheibe, die nur sie drehen kann. Zusätzliche Runensteine werden nur einmal eingewürfelt.}


\end{nobgcolor}


\begin{bgcolor}[color-DLvA]
\section{Die Legenden von Andor (2012)}



\subsection{Legende 3, Die Tage des Widerstands:}

\fan{Nutzt Varkur die Dunkle Magie "Die Schwächung der Helden", muss die Runenmeisterin die Runenscheibe im Endkampf genau um den Runenwürfel-Wert drehen, keine Speiche weiter. Zudem dreht Varkur den Runenwürfel auf die gegenüberliegende Seite, wenn die Runenscheibe dadurch auf einem kleineren Speichenwert landet.}




\subsection{Goldene Ereigniskarte 6:}

\fan{Die Runenmeisterin wirft einen normalen Heldenwürfel, nicht ihren Runenwürfel.}


\subsection{Ereigniskarte "Geheimer See" 9:}

\fan{Statt Heldenwürfel zu werfen, dreht die Runenmeisterin ihre Runenscheibe und guckt, ob der erhaltene Speichenwert eine 1 oder 2 ist.}

\end{bgcolor}






\begin{bgcolor}[color-StSch]
\section{Der Sternenschild (2013)}

\subsection{Andorische Flöte:}

\fan{Nutzt die Runenmeisterin die andorische Flöte, dreht sie ihre Runenscheibe und wertet den erhaltenen Speichenwert, statt einen normalen Heldenwürfel zu werfen.}

\end{bgcolor}




\columnbreak



\begin{bgcolor}[color-DRidN]
\section{Die Reise in den Norden (2014)}

\subsection{Segeln:}

Segelt die Runenmeisterin, darf sie für 1 Stunde ihre Runenscheibe drehen und das Schiff bis\supercite{90045} zum erhaltenen Speichenwert viele Meeresfelder weit in eine beliebige Windrichtung zu segeln.
\supercite{MH:Runen,MH:Begleitheft}
Hat das Schiff den zweiten Mast, kann sie es bis zu doppelt so weit bewegen.
\supercite{90070}

Sie darf sich jede Stunde neu entscheiden, ob sie zum Segeln die aktuelle Windstärke nutzen oder ihre Runenscheibe drehen will, muss sich aber vor einem etwaigen Drehen entscheiden.
\supercite{88563}



\subsection{Legende 9, Die Mächte des Meeres:}

\fan{Für Kenvilars Tücke "auf die 1" wird nicht der Runenwürfel, sondern die Runenscheibe auf die 1 gedreht. Sie kann in dieser Kampfrunde nicht von der Runenmeisterin weitergedreht werden.}
\supercite{15179}

\fan{Für Kenvilars Tücke "nur gerade Würfelwerte" geht es nicht um den Wert des Runenwürfels, sondern der Speichenwert der Runenscheibe. Ist dieser ungerade, zählt er nicht.}



\subsection{Bonus-Legende "Die Suche nach Grenolin":}

\fan{In der ersten Kampfrunde gegen Warx darf auch der Runenwürfel auf die gegenüberliegende Seite gedreht werden.}



\subsection{Legende "Die Rückkehr der Schwarzen Kogge":}

\fan{Wirft der Unbekannte Krieger den Würfel der Runenmeisterin, wird die Runenscheibe nicht gedreht, sondern einfach der Wert des Runenwürfels gewertet.}

\end{bgcolor}



\columnbreak


\begin{bgcolor}[color-DlH]
\section{Die letzte Hoffnung (2016)}


\subsection{Zweite Sonderfähigkeit: Stärkere Runenscheibe}

In Legende 11 wird die Runenscheibe bei der Erfüllung der Heldenaufgabe auf ihre Rückseite gedreht.\supercite{MH11:Heldenkarte_Runenmeisterin} Für Legenden 12 bis 17 liegt die Runenscheibe vom Anfang an auf der Rückseite.\supercite{MH:Begleitheft} Die Speichenreihenfolge der Rückseite ist:

\bullet{} Speichenwert 1 / roter und violetter Edelstein

\bullet{} Speichenwert 2 / Auge

\bullet{} Speichenwert 3 / 2 Willenspunkte

\bullet{} Speichenwert 4 / roter und grüner Edelstein

\bullet{} Speichenwert 5 / Auge

\bullet{} Speichenwert 2 / Quelle

\bullet{} Speichenwert 3 / roter und blauer Edelstein

\bullet{} Speichenwert 4 / 5 Willenspunkte

\bullet{} Speichenwert 5 / Quelle

Für die Aktion "Runen befragen" haben die Speichensymbole leicht andere Effekte:

\bullet{} Anzahl Willenspunkte: Ein beliebiger\supercite{MH11:Heldenkarte_Runenmeisterin} Held erhält \fan{bis zu} so viele Willenspunkte, egal wo er steht und ob er seinen Tag bereits beendet hat.

\bullet{} Auge: Unverändert.

\bullet{} Quelle: Die Runenmeisterin darf 1 beliebige geleerte Quelle auf dem Spielplan auffrischen, auch wenn ein Held auf ihrem Feld steht.

\bullet{} Edelsteine: Trägt die Runenmeisterin mindestens einen Edelstein einer passenden Farbe, darf ein beliebiger\supercite{MH11:Heldenkarte_Runenmeisterin} Held 1 Stärkepunkt erhalten, egal wo er steht und ob er seinen Tag beendet hat. \fan{Jede Edelstein-Farbe kann höchstens 1x pro Tag für diesen Effekt genutzt werden. Zur Erinnerung dürft ihr Edelsteine genutzter Farben auf ihre graue Rückseite drehen.}
\supercite{98910}


\subsection{Alte Waffen:}

\fan{Das Messer kann den Runenwürfel nur im Kampf neu werfen.}


\subsection{Legende 14, Der Meister des Trolls:}

Die Genesungsaufgabe "2 Quellen leeren" ist auch erfüllt, wenn an einem Tag zweimal dieselbe Quelle geleert wird.
\supercite{94130}


\end{bgcolor}




\begin{bgcolor}[color-AG]
\section{Alte Geister (2018)}



\subsection{Legende 2, Der Hexer von Andor:}

\fan{Ist der Endkampf gegen Siantari und auf Feld 30, muss die Runenmeisterin sich nicht für eine Anzahl Würfel entschieden, sondern dafür, wie viele Speichen weit sie die Runenscheibe diese Kampfrunde maximal weiterdrehen darf. Pro mögliche Weiterdrehung muss sie 1 Willenspunkt abgeben, auch wenn sie früher aufhört.}



\subsection{Legende 3, Die steinernen Drei:}

Die Runenmeisterin kann die Quelle auf Feld 37 mit dem Brunnen-Symbol nicht auffrischen.
\supercite{AG:Begleitheft,88563}




\subsection{Bonus-Legende "Die vierte Statue":}

\fan{Für Zors Drachenfeuer-Gift-Effekt gilt der Speichenwert der Runenmeisterin als Würfelwert, nicht der Wert des Runenwürfels.}

\end{bgcolor}





\begin{bgcolor}[color-DZ]
\section{Düstere Zeiten (2020)}

\subsection{Shans Fluch:}

\fan{Der Runenwürfel ist nicht von Shans Fluch betroffen. Bei Shans Fluch mit Wert 8 wird stattdessen das Schattenplättchen auf ein freies Runenscheiben-Symbol auf der Heldentafel gelegt (7–13 oder 14–20). Solange der Willensstein der Runenmeisterin sich in einer Zeile mit Schattenplättchen befindet, muss sie ihre Runenscheibe genau um den Runenwürfel-Wert drehen, keine Speiche weiter (sowohl im Kampf als auch außerhalb).}

\end{bgcolor}

\begin{bgcolor}[color-DeK]
\section{\fan{Die ewige Kälte (2022)}}

\subsection{\fan{Runen befragen:}}


\fan{Das Brunnen-Symbol darf ein erloschenes Feuer auf dem Spielplan entfachen (ohne Willenspunktekosten).}


\fan{Ein erloschenes Feuer auf der Heldentafel der \hyperref[Feuerwächterin]{Feuerwächterin} kann so nicht entfacht werden.}



\subsection{\fan{Askimar-Klinge:}}

\fan{Eine Askimar-Klinge kann den Runenwürfel nur im Kampf neu werfen.}



\subsection{\fan{Große Kräuterkunde:}}

\fan{Hat die Runenmeisterin durch Blaubachbeeren + rotes Blutkraut den Würfelwert 10, wird die Runenscheibe nicht gedreht und kein Speichenwert gewertet. }

\end{bgcolor}






\begin{bgcolor}[color-DfL]
\section{\fan{Das ferne Land (2025)}}



\subsection{\fan{Geister:}}

\fan{Steht die Runenmeisterin bei oder angrenzend zu Geistern, kann sie die Aktion "Runen befragen" nicht nutzen.}

\fan{Sie darf Plättchen aufdecken und Brunnen auffrischen, auch wenn diese bei oder angrenzend zu Geistern liegen. Sie nutzt im Kampf ihre Runenscheibe, auch wenn sie bei oder angrenzend zu Geistern steht.}


\subsection{\fan{Wurfmesser:}}

\fan{Ein Wurfmesser kann den Runenwürfel nur im Kampf neu werfen.}

\end{bgcolor}



\end{multicols*}













\chapter{60. Tenaya (Teyan), die Feuerwächterin}
\label{Feuerwächterin}

\begin{multicols*}{2}\raggedcolumns


\begin{nobgcolor}

\section{Allgemein}


\subsection{\fan{Aufbau:}}

\fan{Die Feuerwächterin erhält 1 Hadrisches Brennholz. Legt 1 Feuerplättchen bereit.}



\subsection{Heldentafel:}

Die Feuerwächterin hat Rang 60.

Ihre offizielle Farbe ist Grün.

Ihr großes Ablagefeld hat Platz für ein Feuerplättchen.

Sie hat:

\bullet{} 1 Würfel bei 1 bis 6 Willenspunkten,

\bullet{} 2 Würfel bei 7 bis 13 Willenspunkten,

\bullet{} 3 Würfel bei 14 bis 20 Willenspunkten.




\subsection{\fan{Hadrisches Brennholz:}}

\fan{Das Hadrische Brennholz ist ein kleiner rechteckiger Gegenstand, der von anderen Helden getragen werden darf. Nur die Feuerwächterin kann es als freie Handlungsmöglichkeit nutzen, und nur, wenn ihr großes Ablagefeld frei ist. Dann muss sie das Hadrische Brennholz aus dem Spiel geben und 2 Willenspunkte abgeben und darf 1 Feuerplättchen mit der brennenden Seite oben auf ihr großes Ablagefeld legen.}

\fan{Das Hadrische Brennholz kann nicht genutzt werden, um ein Feuerplättchen auf ein Spielplanfeld zu legen.}


\subsection{Feuer:}

\fan{Bei Sonnenaufgang, beim Brunnen/Quellen-Symbol, wenn die Feuerwächterin ein brennendes Feuer trägt, erhalten alle Helden auf ihrem Feld 5 Willenspunkte. Dann wird das Feuer auf die erloschene Seite gedreht. Hat der \hyperref[Tarus]{Tarus} 15 oder mehr Willenspunkte, darf er 1 Stärkepunkt statt 5 Willenspunkten erhalten.} 

Solange die Feuerwächterin ein brennendes Feuer trägt, haben alle Helden, die auf ihrem Feld stehen, je +1 \hyperref[Bonusstärke]{Bonusstärke}. Der \hyperref[Krieger]{Krieger} hat +2 \hyperref[Bonusstärke]{Bonusstärke} (statt +1).
\supercite{DeK:Begleitheft} 

Die Feuerwächterin darf ein erloschenes Feuer auf ihrem großen Ablagefeld entfachen, indem sie 2 Willenspunkte abgibt.\supercite{DeK:Begleitheft} Andere Helden können dies nicht.\supercite{105294}

Die Feuerwächterin kann ein Feuerplättchen von ihrem großen Ablagefeld nicht ablegen oder anderen Helden geben, sie darf es aber jederzeit aus dem Spiel nehmen.
\supercite{117803}

Während die Feuerwächterin ein Feuerplättchen trägt, ist ihr großes Ablagefeld besetzt. \fan{Erhält die Feuerwächterin einen großen ovalen Gegenstand, darf sie sich entscheiden, ob sie das Feuerplättchen aus dem Spiel nimmt oder den großen Gegenstand auf ihrem Feld zurücklässt.}

\fan{Die \hyperref[Wassermagierin]{Wassermagierin} kann sich nicht von oder zu einem Feuerplättchen auf der Heldentafel der Feuerwächterin teleportieren.}



\subsection{\fan{Mehrere Feuerwächterinnen:}}

\fan{Alle Hadrischen Brennhölzer sind ununterscheidbar und dürfen von allen Feuerwächterinnen genutzt werden. Eine Feuerwächterin darf mehrmals pro Legende ein Hadrisches Brennholz nutzen.}

\fan{Eine Feuerwächterin kann ein brennendes Feuer keiner anderen Feuerwächterin geben.}

\fan{Auch wenn mehrere Feuerwächterinnen auf demselben Feld stehen und brennende Feuer tragen, erhält jeder Helden nur einmal \hyperref[Bonusstärke]{Bonusstärke} und bei Sonnenaufgang nur einmal 5 Willenspunkte.}


\end{nobgcolor}


\begin{bgcolor}[color-DLvA]
\section{\fan{Die Legenden von Andor (2012)}}

\subsection{\fan{Ausscheiden:}}

\fan{Scheidet die Feuerwächterin aus (z.B. weil sie in Cavern auf 0 Willenspunkte fiel), geht ein von ihr getragenes Feuer aus dem Spiel.}

\subsection{\fan{Legende 5, Der Zorn des Drachens:}}

\fan{Bei der Drachenkampfkarte "Gegenstand in 2 Stärkepunkte tauschen" kann die Feuerwächterin kein von ihr getragenes Feuer in Stärke tauschen.}



\subsection{\fan{Ereigniskarte "Geheimer See" 5:}}

\fan{Trägt die Feuerwächterin ein Feuerplättchen, geht dieses aus dem Spiel. Für den Rest der Legende kann sie kein Feuerplättchen mehr tragen.}

\end{bgcolor}





\begin{bgcolor}[color-DlH]
\section{\fan{Die letzte Hoffnung (2016)}}


\subsection{\fan{Legende 15, Der vergiftete Geist:}}

\fan{Wird die Feuerwächterin enttarnt, geht ein von ihr getragenes Feuer aus dem Spiel. Sie kann es nicht in Stärke tauschen.}

\end{bgcolor}





\begin{bgcolor}[color-DZ]
\section{\fan{Düstere Zeiten (2020)}}


\subsection{\fan{Eule:}}

\fan{Die Eule kann nicht an die Feuerwächterin geschickt werden, während sie ein Feuerplättchen trägt, da ihr großes Ablagefeld dann nicht frei ist.}

\end{bgcolor}



\columnbreak


\begin{bgcolor}[color-DeK]
\section{Die ewige Kälte (2022)}

\subsection{Brennholz:}

\fan{Die Feuerwächterin erhält am Anfang kein Hadrisches Brennholz.}

Ist das große Ablagefeld der Feuerwächterin frei, darf sie ein normales Brennholz und 2 Willenspunkte abgeben, um ein brennendes Feuerplättchen auf ihr großes Ablagefeld zu legen. Sie kann keine Feuerplättchen vom Spielplan aufnehmen.

Andere Helden können kein Feuerplättchen auf die Heldentafel der Feuerwächterin legen, auch nicht durch Abgabe von Brennholz und Willenspunkten.
\supercite{105294}




\subsection{Feuer:}
Bei Sonnenaufgang, beim Symbol "Wärmendes Feuer", wenn die Feuerwächterin ein brennendes Feuer trägt, erhalten alle Helden auf ihrem Feld 5 Willenspunkte.
\supercite{DeK:Begleitheft} 
\fan{Hat der \hyperref[Tarus]{Tarus} 15 oder mehr Willenspunkte, darf er 1 Stärkepunkt statt 5 Willenspunkte erhalten.}  

Beim Symbol "Die Feuer verlöschen", wenn die Feuerwächterin ein brennendes Feuer trägt, wird das Feuer auf die erloschene Seite gedreht. 

Steht die Feuerwächterin bereits auf dem Feld eines brennenden Feuers, gibt ein Feuer auf ihrer Heldentafel weder Willenspunkte noch \hyperref[Bonusstärke]{Bonusstärke}.
\supercite{101489}

\fan{Das Brunnen-Symbol der \hyperref[Runenmeisterin]{Runenmeisterin} kann ein erloschenes Feuer auf dem großen Ablagefeld der Feuerwächterin nicht entfachen.}

\fan{Die \hyperref[Wassermagierin]{Wassermagierin} kann sich nicht von oder zu einem Feuerplättchen auf der Heldentafel der Feuerwächterin teleportieren.}


\subsection{Legende 4, Die Pranken des Winters:}

Auch für ein brennendes Feuer auf der Heldentafel der Feuerwächterin dürft ihr Takotas Stärkestein um 1 nach links verschieben.
\supercite{DeK4:Takota}


\subsection{Mini-Erweiterung "Die Bruderfeuer":}


Steht die Feuerwächterin auf der Zeltstadt (Feld 401), der Pfahlbausiedlung (Feld 460) oder dem Greifenfelsen (Feld 425), während sie ein brennendes Feuer trägt, wirkt der Effekt dieses Orts, auch wenn kein brennendes Feuerplättchen auf dem Feld liegt.
\supercite{ME:Bruderfeuer}

\end{bgcolor}


\begin{bgcolor}[color-DfL]
\section{\fan{Das ferne Land (2025)}}





\subsection{\fan{Geister:}}

\fan{Steht die Feuerwächterin bei oder angrenzend zu Geistern, kann sie kein Feuerplättchen auf ihre Heldentafel legen, kann sie kein erloschenes Feuer auf ihrer Heldentafel entfachen und gibt ein Feuerplättchen auf ihrer Heldentafel weder Willenspunkte noch \hyperref[Bonusstärke]{Bonusstärke}.}

\fan{Ein Feuerplättchen auf ihrer Heldentafel wird beim Sonnenaufgang auf die erloschene Seite gedreht, auch wenn sie bei oder angrenzend zu Geistern steht.}

\end{bgcolor}







\end{multicols*}

\chapter{60. Tenaya (Teyan), die Feuerbezwingerin}
\label{Feuerbezwingerin}

\begin{multicols*}{2}\raggedcolumns


\begin{nobgcolor}

\section{Allgemein}


\subsection{Aufbau:}

Die Feuerbezwingerin erhält die 3 Feuerzauber, alle gelöscht (mit der "–2 Willenspunkte"-Seite nach oben).


\subsection{Heldentafel:}

Die Feuerbezwingerin hat Rang 60.

Ihre offizielle Farbe ist Grün.

Sie hat höchstens 10 Stärkepunkte (statt 14).

Sie hat höchstens 14 Willenspunkte (statt 20).

Sie hat 3 besondere Ablagefelder für ihre Feuerzauber.

Sie hat:

\bullet{} 1 Würfel bei 1 bis 4 Willenspunkten,

\bullet{} 2 Würfel bei 5 bis 9 Willenspunkten,

\bullet{} 3 Würfel bei 10 bis 14 Willenspunkten.





\subsection{Feuerzauber:}

Nur die Feuerbezwingerin kann die 3 Feuerzauber tragen. 

Die Feuerzauber sind keine Gegenstände und können nicht abgelegt werden.

Die Feuerbezwingerin darf 2 Willenspunkte abgeben, um einen gelöschten Feuerzauber auf die brennende Seite zu drehen und seinen Effekt zu aktivieren.

Bei Sonnenaufgang, direkt vor dem Erzähler-Symbol, darf sie alle Feuerzauber aufdecken, egal ob das Erzähler-Symbol ausgeführt wird.


\subsection{Bannfeuer (violett):}

Steht die Feuerwächterin mit brennendem Bannfeuer auf dem Feld einer Kreatur (kein Endgegner), läuft diese bei ihrem Sonnenaufgang-Feld-Symbol nicht. \fan{Durch einen Ereigniskarten-Effekt kann sie weiterhin laufen.}

Kreaturen können das Feld der Feuerwächterin weiterhin betreten und überspringen.


\subsection{Feuerstoß (rot):}

Die Feuerbezwingerin kann den Feuerstoß nur direkt vor einer Kampfrunde, in der sie mitkämpft, entfachen. Die Feuerbezwingerin hat nur in dieser Kampfrunde +2 \hyperref[Bonusstärke]{Bonusstärke}.


\subsection{Leuchtfeuer (blau):}

Ein anderer Held, dessen Zeitstein nicht auf dem Sonnenaufgang-Feld liegt und der seinen Tag noch nicht beendet hat, darf sofort {bis zu} 2 Felder weit laufen. Dies kostet ihn keine Stunden.

Effekte, die durch das Betreten eines Felds oder Stehenbleiben aktiviert werden, werden wie üblich abgehandelt. \fan{Sind für das Betreten eines Felds oder das Benutzen einer Feldverbindung eine Anzahl Willenspunkte nötig, muss der laufende Held genügend Willenspunkte haben, nicht die Feuerbezwingerin.}

\fan{Ein durch das Leuchtfeuer laufender Held darf mitbewegbare Plättchen (z.B. Bauern) und Figuren (z.B. Merrik) mitbewegen.}

Ein durch das Leuchtfeuer laufender \hyperref[Reiter]{\fan{Reiter}} darf einen anderen Helden mitbewegen.
\supercite{127533}

Ein festgehaltener Held (z.B. von \fan{Skeletten in "Die letzte Hoffnung" oder} Spornen in "Das ferne Land") kann durch das Leuchtfeuer nicht laufen.
\supercite{DfL:Begleitheft,127533} 


\subsection{\fan{Mehrere Feuerbezwingerinnen:}}


\fan{Eine Feuerbezwingerin kann ihre Feuerzauber keiner anderen Feuerbezwingerin geben.}

\fan{Eine Feuerbezwingerin darf mit ihrem Leuchtfeuer eine andere Feuerbezwingerin laufen lassen.}

\end{nobgcolor}



\begin{bgcolor}[color-DLvA]
\section{\fan{Die Legenden von Andor (2012)}}


\subsection{\fan{Ausscheiden:}}

\fan{Scheidet die Feuerbezwingerin aus (z.B. weil sie in Cavern auf 0 Willenspunkte fiel), gehen ihre Feuerzauber aus dem Spiel.}


\subsection{\fan{Legende 5, Der Zorn des Drachens:}}

\fan{Bei der Drachenkampfkarte "Gegenstand in 2 Stärkepunkte tauschen" kann die Feuerwächterin keinen Feuerzauber in Stärke tauschen.}



\subsection{\fan{Bonus-Legende "Die Eskorte des Königs":}}

\fan{Bewegt ein durch das Leuchtfeuer laufender Held König Brandur mit, muss er dafür die Willenspunkte abgeben, nicht die Feuerwächterin.}

\end{bgcolor}





\begin{nobgcolor}
\section{\fan{Neue Helden (2014)}}

\subsection{\fan{Spielvariante "Der Trunkene Troll":}}

\fan{Das Bannfeuer hindert den Trunkenen Troll nicht am Bewegen.}

\end{nobgcolor}





\begin{bgcolor}[color-DRidN]
\section{\fan{Die Reise in den Norden (2014):}}

\subsection{\fan{Leuchtfeuer (blau):}}

\fan{Durch das Leuchtfeuer darf ein Held das Schiff betreten, das Schiff verlassen oder Bootsstrecken (als 2 Felder) nutzen, aber nicht das Schiff segeln.}

\end{bgcolor}



\columnbreak


\begin{bgcolor}[color-DlH]
\section{\fan{Die letzte Hoffnung (2016)}}


\subsection{\fan{Bannfeuer (violett):}}

\fan{Steht ein Skelett, Wachtroll oder Krahder auf dem Feld der Feuerwächterin mit brennendem Feuer, läuft es nicht durch Bewegungsplättchen. Durch andere Effekte kann es weiterhin laufen. Es kann das Feld der Feuerwächterin weiterhin betreten, überqueren und überspringen.}

\fan{Der Bleiche König und der Urtroll werden vom Bannfeuer nicht festgehalten.}


\subsection{\fan{Leuchtfeuer (blau):}}


\fan{Ein durch das Leuchtfeuer laufender Held kann den Trosswagen nicht mitbewegen.}


\subsection{\fan{Legende 15, Der vergiftete Geist:}}

\fan{Wird die Feuerbezwingerin enttarnt, gehen ihre Feuerzauber aus dem Spiel. Sie kann sie nicht in Stärke tauschen.}

\end{bgcolor}







\begin{bgcolor}[color-BB]
\section{\fan{Die Bonus-Box (2017/2024)}}

\subsection{\fan{Bannfeuer (violett):}}

\fan{Das Bannfeuer hindert auch Barbaren bei Sonnenaufgang am Laufen.}

\fan{Das Bannfeuer hindert auch Fluggors am Bewegen von Kreaturen und am Davonfliegen.}






\end{bgcolor}







\begin{bgcolor}[color-DZ]
\section{\fan{Düstere Zeiten (2020)}}


\subsection{\fan{Bannfeuer (violett):}}

\fan{Würde eine Kreatur (kein Endgegner) durch ihr Nachtmodus-Bewegungsplättchen laufen, während die Feuerwächterin mit brennendem Bannfeuer auf ihrem Feld steht, läuft sie stattdessen nicht. Die Feuerwächterin darf im Nachtmodus mehrmals pro Tag mit demselben Bannfeuer Kreaturen am Laufen hindern.}

\fan{Tiefminen-Golems werden vom Bannfeuer nicht festgehalten.}


\subsection{\fan{Leuchtfeuer (blau):}}

\fan{Ein durch das Leuchtfeuer laufender Held kann nur bis zu zwei Felder weit laufen, auch wenn er die Zwergenstiefel trägt oder der Einhornreiter ist.}

\end{bgcolor}




\begin{bgcolor}[color-DeK]
\section{\fan{Die ewige Kälte (2022)}}



\subsection{\fan{Feuerzauber:}}

\fan{Die Feuerbezwingerin kann ihre Feuerzauber nicht bei Händlern tauschen.}



\subsection{\fan{"Schwerer Spielen"-Winterstein:}}

\fan{Läuft der Winterstein-Träger durch das Leuchtfeuer, verliert er die Willenspunkte, nicht die Feuerbezwingerin.}


\subsection{\fan{Höhle der Verwandlung:}}

\fan{Kreaturen auf der Höhle der Verwandlung (Feld 414) werden vom Bannfeuer nicht am Laufen gehindert.}


\end{bgcolor}



\begin{bgcolor}[color-DfL]
\section{Das ferne Land (2025)}

\subsection{Feuerzauber:}

Ein eigenes Symbol "Feuerzauber" auf dem Sonnenaufgang-Feld erinnert an das Aufdecken der Feuerzauber.

Ein durch das Leuchtfeuer laufender Held kann das Steppenvolk nicht mitbewegen.
\supercite{DfL:Begleitheft}



\subsection{Gorlots:}

\fan{Ein nicht in der ersten Kampfrunde besiegter Gorlot läuft davon, auch wenn die Feuerbezwingerin mit brennendem Bannfeuer bei ihm steht.}



\subsection{Geister:}

Steht die Feuerbezwingerin bei oder angrenzend zu Geistern, kann sie ihre Feuerzauber nicht entfachen.

\fan{Feuerzauber werden bei Sonnenaufgang aufgedeckt, auch wenn sie bei oder angrenzend zu Geistern steht.}

Kreaturen bei oder angrenzend zu Geistern werden werden vom Bannfeuer nicht am Laufen gehindert.

\fan{Der Feuerstoß wirkt auch gegen Gegner bei oder angrenzend zu Geistern, sofern die Feuerbezwingerin selbst nicht bei oder angrenzend zu Geistern steht.}

\fan{Das Leuchtfeuer kann Helden laufen lassen, egal ob diese bei oder angrenzend zu Geistern stehen oder laufen.}

\end{bgcolor}








\end{multicols*}











\chapter{60. Kirr, der Zeitzauberer}
\label{Zeitzauberer}


\begin{multicols*}{2}\raggedcolumns



\begin{nobgcolor}

\section{Allgemein}


\subsection{Heldentafel:}

Der Zeitzauberer hat Rang 60.

Seine offizielle Farbe ist Violett.

Er hat höchstens 17 Willenspunkte (statt 20).

Er hat 1 Würfel, egal wie viele Willenspunkte er hat.




\subsection{Zeitzauberei:}

Hat der Zeitzauberer 10 oder mehr Willenspunkte, erhält er pro Stunde, die sein Zeitstein (nachdem er für die Kampfrunde vorgesetzt wurde) von der 4. Stunde entfernt ist, +1 \hyperref[Bonusstärke]{Bonusstärke}. Zum Beispiel hat er +3 \hyperref[Bonusstärke]{Bonusstärke}, wenn sein Zeitstein auf der 1. oder 7. Stunde liegt.
\supercite{BL_RdSK:Kirr}


Ihr dürft zur Erinnerung die 4. Stunde markieren, zum Beispiel mit der Figur "Koraph", \fan{wenn diese nicht sonst genutzt wird.}




\subsection{Neutrale Zeitsteine:}

Im Spiel mit neutralen Zeitsteinen zählt die Endposition desjenigen Zeitsteins, den der Zeitzauberer in dieser Kampfrunde vorgesetzt hat, \fan{inklusive etwaige Bewegungen desselben Zeitsteins durch andere Helden}.




\subsection{Nicht Eara:}

Spricht eine Regel oder Karte von der \hyperref[Zauberin]{Zauberin}, ist der Zeitzauberer nicht gemeint.

\end{nobgcolor}



\begin{bgcolor}[color-DLvA]

\section{\fan{Die Legenden von Andor (2012)}}

\subsection{\fan{Ereigniskarte "Geheimer See" 9:}}

\fan{Der Zeitzauberer kann seinen Würfel nicht auf die gegenüberliegende Seite drehen.}


\end{bgcolor}





\begin{bgcolor}[color-StSch]
\section{\fan{Der Sternenschild (2013)}}


\subsection{\fan{Andorische Flöte:}}

\fan{Nutzt der Zeitzauberer die Flöte, kann er den geworfenen Würfel nicht auf die gegenüberliegende Seite drehen.}

\end{bgcolor}




\begin{nobgcolor}

\section{\fan{Dunkle Helden (2017)}}

\subsection{\fan{Spielvariante "Merriks Karte":}}

\fan{Nutzt der Zeitzauberer in einer Kampfrunde Merriks neutralen Zeitstein, zählt für die Bonusstärke dessen Position statt die des farbigen Zeitsteins.}

\end{nobgcolor}


\columnbreak


\begin{bgcolor}[color-DfL]
\section{\fan{Das ferne Land (2025)}}

\subsection{\fan{Geister:}}

\fan{Steht der Zeitzauberer bei oder angrenzend zu Geistern, erhält er keine \hyperref[Bonusstärke]{Bonusstärke} durch Zeitzauberei. Es ist egal, ob der Gegner bei oder angrenzend zu Geistern steht.}

\end{bgcolor}





\end{multicols*}







\chapter{68. Darh (Darhen), die Beschwörerin}
\label{Beschwörerin}





\begin{multicols*}{2}\raggedcolumns


\begin{nobgcolor}


\section{Allgemein}


\subsection{Aufbau:} 

Legt den Knochen-Golem und das Golem-Symbol bereit.





\subsection{Heldentafel:}

Die Beschwörerin hat Rang 68.

Ihre offizielle Farbe ist Beige.

Sie hat höchstens 15 Willenspunkte (statt 20).

Sie hat:

\bullet{} 2 Würfel bei 1 bis 7 Willenspunkten,

\bullet{} 3 Würfel bei 8 bis 15 Willenspunkten.





\subsection{Knochen-Golem:}



Die Beschwörerin hat einen \hyperref[Helfer]{\fan{Helfer}}, die Figur "Knochen-Golem".

Steht der Knochen-Golem im Kampf auf dem Feld des Gegners, gibt er +6 \hyperref[Bonusstärke]{Bonusstärke}.
\supercite{DH:Knochen-Golem}


Legt den Knochen-Golem, während er nicht im Spiel ist, auf das Feld für besiegte Kreaturen.
\supercite{DH:Knochen-Golem}
Er darf erst bewegt werden, nachdem er durch eine Beschwörung aufgestellt wurde, \fan{auch wenn das Feld für besiegte Kreaturen durch eine Eisbrücke des \hyperref[Eis-Dämon]{Eis-Dämons} angrenzend an ein anderes Feld gemacht wurde.}







\subsection{Beschwörung:}

Ist die Beschwörerin am Sieg über eine Kreatur (kein Endgegner) beteiligt, während der Knochen-Golem nicht im Spiel ist, darf sie nach einer etwaigen Belohnung, aber vor einer etwaigen Erzählerbewegung, den Knochen-Golem beschwören. Dazu wirft sie einen Kreaturenwürfel (10er-Stelle) und einen Heldenwürfel (1er-Stelle) und stellt die Kreatur und den Knochen-Golem auf das so ermittelte Feld.
\supercite{DH:Knochen-Golem}

Nur wenn die Kreatur bei der Beschwörung direkt versetzt wird, z.B. weil auf dem ermittelten Feld nur eine Kreatur stehen kann und dort bereits eine Kreatur (oder der \hyperref[Halbskral]{Halbskral}) steht, versetzt den Knochen-Golem mit ihr mit. 
\supercite{130318}
\fan{Würde die Kreatur bei der Beschwörung direkt auf ein Feld mit (goldenen \fan{oder Steppenvolk-}) Schilden versetzt, werft stattdessen die Würfel neu und ermittelt ein alternatives Feld.}

Ist der Erzähler bei der Beschwörung 4 oder mehr Felder vom Ende der Legendenleiste entfernt, legt das Golem-Symbol 4 Buchstaben über ihn an die Legendenleiste. Danach bewegt sich der Erzähler, sofern vom Sieg noch eine Erzählerbewegung aussteht. Erreicht der Erzähler das Golem-Symbol, geht der Knochen-Golem aus dem Spiel.
\supercite{DH:Knochen-Golem}

Der Knochen-Golem darf mehrmals pro Legende beschworen werden.

\fan{Soll die Kreatur für eine Aufgabe besiegt werden, gilt die Aufgabe als erfüllt, auch wenn die Beschwörerin die Kreatur danach wieder ins Spiel bringt.}









\subsection{\fan{Mehrere Beschwörerinnen:}}

\fan{Jede Beschwörerin hat ihren eigenen Knochen-Golem, den nur sie beschwören und bewegen kann. Ihr dürft den Knochen-Golems Standfüße in unterschiedlichen Farben geben. Jeder Knochen-Golem hat sein eigenes Golem-Symbol. Ihr dürft auch diese unterschiedlich markieren.}


\end{nobgcolor}


\begin{bgcolor}[color-DLvA]
\section{Die Legenden von Andor (2012)}




\subsection{Beschwörung:}

\fan{Würde die Kreatur bei der Beschwörung in Cavern (Spielplan-Rückseite) direkt auf ein Feld versetzt, durch dessen Betreten das "N"-Plättchen nach unten bewegt würde, werft stattdessen die Würfel neu und ermittelt ein alternatives Feld.}


\subsection{Ausscheiden:}

\fan{Scheidet die Beschwörerin aus (z.B. weil sie in Cavern auf 0 Willenspunkte fiel), gehen ihr Knochen-Golem und ihr Golem-Symbol aus dem Spiel.}

\end{bgcolor}


\begin{bgcolor}[color-StSch]
\section{Der Sternenschild (2013)}

\subsection{Der Dieb Ken Dorr:}

\fan{Der Knochen-Golem kann nicht nach dem Sieg über den Dieb Ken Dorr beschworen werden.}


\subsection{Bedrohung 3, Der Dunkle Tempel:}

\fan{Der Knochen-Golem kann nicht beschworen werden, während der Erzähler auf einem Buchstaben mit Feuerplättchen steht.}

\end{bgcolor}



\columnbreak




\begin{bgcolor}[color-DRidN]
\section{\fan{Die Reise in den Norden (2014)}}


\subsection{\fan{Beschwörung:}}

\fan{Wird der Knochen-Golem beschworen, markiert einen Kreaturenplatz auf Feld 90 mit einem leeren Standfuß. Sind alle Kreaturenplätze besetzt, bewegt sich der Erzähler und der Standfuß geht aus dem Spiel.}

\fan{Wird er beim Sieg über einen Gor beschworen, bestimmt die Position des Gors und des Knochen-Golems, indem ihr zwei weiße Würfel werft und:}

\bullet{} \fan{auf den Nebelinseln (Spielplan-Vorderseite) 100 addiert.}

\bullet{} \fan{in Hadria (Spielplan-Rückseite) 130 addiert.}

\fan{Wird er beim Sieg über eine Meereskreatur beschworen, werft einen großen schwarzen Würfel und stellt die Meereskreatur auf:}

\bullet{} \fan{römisch I bei 10,}

\bullet{} \fan{römisch II bei 12,}

\bullet{} \fan{römisch III bei 8,}

\bullet{} \fan{römisch IV bei 6.}

\fan{Auf den Nebelinseln (Spielplan-Vorderseite) könnt ihr dafür der gestrichelten Line folgen.}

\fan{Wurde die Meereskreatur auf ein Landfeld versetzt, stellt den Knochen-Golem zur ihr. Ansonsten stellt den Knochen-Golem auf das nächste Landfeld entlang der Pfeile.}






\subsection{\fan{Knochen-Golem:}}

\fan{Legt den Knochen-Golem, während er nicht im Spiel ist, neben (nicht auf) Feld 90.}


\end{bgcolor}





\columnbreak



\begin{bgcolor}[color-DlH]
\section{Die letzte Hoffnung (2016)}


\subsection{Zweite Sonderfähigkeit: Knochen-Käfig}

Steht der Knochen-Golem auf dem Feld einer Kreatur, läuft diese bei Sonnenaufgang oder aufgrund einer Ereigniskarte nicht.
\supercite{DH11:Heldenkarte_Beschwörer}

Kreaturen können das Feld des Knochen-Golems weiterhin betreten oder überspringen (wenn eine Kreatur dort steht).

\fan{Skelette, Krahder, Endgegner und der Urtroll werden vom Knochen-Golem nicht festgehalten.}




\subsection{Beschwörung:}

Der Knochen-Golem kann nicht nach dem Sieg über ein Skelett beschwört werden.
\supercite{DH:Begleitheft}



\subsection{Legende 15, Der vergiftete Geist:}

\fan{Spielt ihr mit mehreren Beschwörerinnen, muss für deine offizielle Geschenkaufgabe dein eigener Knochen-Golem auf Feld 219 stehen.}


\fan{Wird die Beschwörerin enttarnt, gehen ihr Knochen-Golem und ihr Golem-Symbol aus dem Spiel.}






\subsection{Legende 17, Die letzte Hoffnung:}


Der Knochen-Golem darf auch beim Sieg über einen Krahder beschworen werden. 
\supercite{DH:Begleitheft}

Wird er \fan{beim Sieg über einen Wachtroll beschworen, während der Erzähler auf "k" oder später steht, oder} beim Sieg über einen Krahder beschworen, bestimmt die Position von Knochen-Golems und Gegners mit einem Kreaturen- (10er-Stelle) und einem Heldenwürfel (1er-Stelle), indem ihr 300 addiert.

Wird der Knochen-Golem beim Sieg über einen Wachtroll beschworen, \fan{während der Erzähler auf "j" oder vorher steht}, wird ihre Position nicht gewürfelt, sondern die Beschwörerin wählt dafür ein Feld angrenzend zum Ort des Wachtrolls.  Während der Erzähler auf "j" oder vorher steht, bewegt sich ein wiederbeschworener Wachtroll durch Bewegungsplättchen zurück zu seinem Ort.
\supercite{DH:Begleitheft}


\end{bgcolor}




\begin{bgcolor}[color-BB]
\section{Die Bonus-Box (2017/2024)}


\subsection{Fluggors:}

\fan{Der Knochen-Golem darf auch beim Sieg über einen Fluggor beschworen werden.}



\subsection{Legende 1, Der Angriff der Barbaren:}

\fan{Der Knochen-Golem kann nicht beim Sieg über einen Barbaren beschworen werden. Würde der Knochen-Golem bei der Beschwörung auf das Feld eines Barbaren gestellt, werden er und der Gegner auf das nächste freie Feld entlang der Pfeile versetzt.}

\end{bgcolor}



\columnbreak


\begin{bgcolor}[color-AG]
\section{Alte Geister (2018)}


\subsection{Beschwörung:}

Der Knochen-Golem kann nicht nach dem Sieg über einen Maasavi oder eine Zwergenstatue beschworen werden.
\supercite{AG:Begleitheft}

\fan{Würde der Gegner bei der Beschwörung direkt auf das Lager der Tulgori (Feld 18) versetzt, werft stattdessen die Würfel neu und ermittelt ein alternatives Feld.}

\end{bgcolor}


\begin{bgcolor}[color-DZ]
\section{Düstere Zeiten (2020)}



\subsection{Beschwörung:}

Der Knochen-Golem kann nicht nach dem Sieg über einen Tiefminen-Golem beschworen werden.
\supercite{DZ:Begleitheft}

\fan{Würde die Kreatur bei der Beschwörung direkt auf das Lager der Trolle (Feld 39) versetzt, werft die Würfel neu und ermittelt ein alternatives Feld.}

\end{bgcolor}



\begin{bgcolor}[color-DeK]
\section{\fan{Die ewige Kälte (2022)}}

\subsection{\fan{Beschwörung:}}

\fan{Bestimmt die Position des Knochen-Golems und der Kreatur mit einem Kreaturenwürfel (10er-Stelle) und einem Heldenwürfel (1er-Stelle), indem ihr 400 addiert. Werft die Würfel neu und ermittelt ein alternatives Feld, wenn der Gegner bei der Beschwörung auf ein Feld gestellt würde, von dem kein Pfeil wegzeigt.}


\subsection{\fan{Höhle der Verwandlung:}}

\fan{Wird ein beschworener Gor/Steppenbüffel auf die Höhle der Verwandlung (Feld 414) gestellt oder versetzt, wird er hingelegt und bleibt dort liegen, bis das rosafarbene Verwandlungs-Symbol des Sonnenaufgang-Felds ausgeführt wird.}

\fan{Beschworene Wargors/Felltrolle, die die Höhle der Verwandlung betreten, werden nicht hingelegt oder verwandelt. Sie überspringen keine anderen Kreaturen auf Feld 414.}

\end{bgcolor}




\columnbreak



\begin{bgcolor}[color-DfL]
\section{\fan{Das ferne Land (2025)}}

\subsection{\fan{Geister:}}

\fan{Steht die Beschwörerin bei oder angrenzend zu Geistern, kann sie den Knochen-Golem nicht beschwören oder bewegen. Sofern sie selbst nicht bei oder angrenzend zu Geistern steht, kann sie ihn beschwören, auch wenn die Kreatur bei oder angrenzend zu Geistern stand oder das ermittelte Feld bei oder angrenzend zu Geistern ist.}

\fan{Der Knochen-Golem gibt keine \hyperref[Bonusstärke]{Bonusstärke}, wenn er bei oder angrenzend zu Geistern steht. Sofern er selbst nicht bei oder angrenzend zu Geistern steht, gibt er seine Bonusstärke, auch wenn die Beschwörerin bei oder angrenzend zu Geistern steht.}


\subsection{\fan{Legende 6, Die Höhle:}}

\fan{Bei einem Sieg in der Höhle kann der Knochen-Golem nicht beschworen werden.}


\subsection{\fan{Mini-Erweiterung "Der Mhourl":}}

\fan{Der Knochen-Golem darf auch nach dem Sieg über den Mhourl beschworen werden.}



\end{bgcolor}




\end{multicols*}














\chapter{Regeln für Ablagen und Leisten}
\label{Ablagen und Leisten}


\begin{multicols*}{2}\raggedcolumns

\section{Allgemein}


\begin{nobgcolor}

\subsection{Goldablage:}

Die Goldablage eines Helden sind die drei Kreise links neben seinem Sonderfähigkeitstext. Jeder Held darf dort beliebig viel Gold, Edelsteine, Apfelnüsse, Sternkraut, Mondbeeren, runde Kräuter, Pilze und andere auf diesen drei Kreisen getragene Plättchen tragen, egal welcher Text dort steht.








\subsection{Limitierte Ablagefelder:}

Manche Helden haben nicht so viele Ablagefelder wie andere. Manche Effekte (z.B. der \hyperref[Hautwandler]{Hautwandler} als Bär oder Ruinengeister in "Das ferne Land") können Ablagefelder blockieren.

\textit{Helden dürfen Handlungen ausführen, durch die ein Held mehr Gegenstände erhalten würde, als er tragen kann. Zum Beispiel darf ein Held beim Händler einen großen Gegenstand kaufen, auch wenn er selbst kein großes Ablagefeld hat.} Zum Beispiel darf der \hyperref[Hautwandler]{Hautwandler} auch als Bär Gold als Kampfbelohnung wählen.
\supercite{DH:FAQ}

Würde ein Held einen Gegenstand erhalten, den er nicht tragen kann, wird dieser auf sein Feld gelegt.
\supercite{DRidN:FAQ}

\fan{Ein Held darf \hyperref[Aufdecken]{von nah aufdeckbare} Plättchen auf seinem Feld aufdecken, auch wenn er sie nicht tragen kann.}

Wer kein großes Ablagefeld hat, kann keine großen Gegenstände tragen (außer welche, die auf zwei kleine rechteckige Ablagefelder gelegt werden\supercite{DRidN:Begleitheft}).


\subsection{Falke:}

Mit dem Falken können nur Gegenstände getauscht werden, die die beiden Helden aktuell tragen. Dabei kann man dem jeweils anderen Helden höchstens so viele Gegenstände schicken, wie er auch tragen kann. Während des Tauschs dürfen keine Gegenstände aufgenommen oder abgelegt werden. Gegenstände, die ein Held nicht tragen kann, können nicht an diesen Helden geschickt werden. 
\supercite{1673}







\subsection{Limitierte Stärkeleisten und Willensleisten:}

Manche Helden können nicht so viele Stärkepunkte oder Willenspunkte haben wie andere. Manche Effekte (z.B. der Feuerschild in "Die letzte Hoffnung" oder der Casamatuc in "Alte Geister") können im Laufe einer Legende beschränken.

Würde ein Held mehr Stärkepunkte oder Willenspunkte erhalten, als er haben kann, verfallen die überschüssigen Punkte. 

Manche Legenden sind ohne Spielvarianten kaum zu gewinnen, wenn ihr nur Helden mit zu kurzen Stärkeleisten spielt. Ein Beispiel ist Legende 17 mit der \hyperref[Zauberin]{Zauberin} und dem \hyperref[Eis-Dämon]{Eis-Dämon}. \fan{Nutzt in solchen Fällen Spielvarianten für zusätzliche \hyperref[Bonusstärke]{Bonusstärke}:}

\bullet{} Mini-Erweiterung "Das Licht der fünften Stunde"
\supercite{ME:LdfS}

\bullet{} \fan{Fan-Spielvarianten-Sammlung "Fan-Spielvarianten für für Siege fast erforderliche Sonderstärke"}
\supercite{129669}






\subsection{Holzstämme:}

\fan{Ein Held, der seine höchste Anzahl Stärkepunkte hat, darf einen einzelnen Holzstamm tragen, auch wenn dieser links über seine Stärkeleiste hinausragt. Er kann dann aber nicht mehrere Holzstämme gleichzeitig tragen. Das geht nicht, wenn ein Effekt (z.B. der \hyperref[Hautwandler]{Hautwandlers} als Bär) ihm das Tragen verbietet.} 




\subsection{Willenspunkte-Schranken:}

\fan{Kann ein Feld nur von Helden mit genug Willenspunkten betreten werden (z.B. das Lager der Trolle in "Düstere Zeiten" oder die Höhle der Verwandlung in "Die ewige Kälte"), darf ein Held, der die höchste Willenspunkte-Anzahl seiner Heldentafel hat, dieses Feld immer betreten, auch wenn diese Anzahl nicht genug wäre.}

\fan{Kann eine Verbindung zwischen zwei Feldern nur von Helden mit genug Willenspunkten überquert werden (z.B. Sprungfelder in "Die letzte Hoffnung" oder der steile Felsen in "Das ferne Land"), darf ein Held, der die höchste Willenspunkte-Anzahl seiner Heldentafel hat, dieses Verbindung immer überqueren, auch wenn diese Anzahl nicht genug wäre.}









\end{nobgcolor}








\begin{bgcolor}[color-DLvA]
\section{Die Legenden von Andor (2012)}




\subsection{Bonus-Legende "Die Befreiung der Mine":}

\fan{Taucht Varkur bei der Karte "Varkurs Versteck" im Geheimen See auf: Helden, die ihre höchste Anzahl Willenspunkte haben, dürfen gegen Varkur kämpfen, auch wenn sie weniger als 14 Willenspunkte haben.}




\subsection{Legende 5, Der Zorn des Drachens:}

\fan{Dürfte ein Held, der nur weniger als 14 Willenspunkte haben kann, durch "Licht und Schatten" oder eine Drachenkampf-Karte auf 14 Willenspunkte steigen, steigt er auf seine höchste Anzahl Willenspunkte.}



\subsection{Ereigniskarte "Geheimer See" 5:}

Hat der Held kein großes Ablagefeld, erhält er auch kein rotes X.





\end{bgcolor}



\columnbreak


\begin{bgcolor}[color-StSch]
\section{Der Sternenschild (2013)}

\subsection{Sternenlicht:}

Die Legendenkarte "Sternenlicht" kann nur aktiviert werden, wenn ein Held den Sternenschild trägt. Es genügt nicht, auf seinem Feld zu stehen.\supercite{6196}

Haben alle Helden kein großes Ablagefeld, kann diese Karte nicht aktiviert werden. Das Aktivieren dieser Karte ist kein Legendenziel.

\end{bgcolor}



\begin{nobgcolor}

\section{Neue Helden (2014)}

\subsection{Spielvariante "Der Bruderschild":}

Tauscht ein Held mit dem Bruderschild mehr Stärkepunkte mit einem anderen Helden, als dieser haben kann, verfallen die überschüssigen Stärkepunkte.
\supercite{NH:Begleitheft}

\end{nobgcolor}


\begin{bgcolor}[color-DRidN]
\section{Die Reise in den Norden (2014)}

\subsection{Zu viele Gegenstände auf dem Schiff:}

Würde ein Held an Bord des Schiffs mehr Gegenstände erhalten, als er tragen kann, werden überzählige Gegenstände in die Kajüte gelegt. Würden mehr Gegenstände in die Kajüte gelegt, als dort Platz haben, gehen überzählige Gegenstände aus dem Spiel.
\supercite{11056}


\subsection{Sturmtal:}

\fan{Ein Held, der nur weniger als 14 Willenspunkte haben kann, darf, wenn er auf Sturmtal (Feld 114) steht, seine Willenspunkte auf seine höchste Anzahl erhöhen.}





\subsection{Kein großes Ablagefeld:}

Orweyns Hammer wird auf zwei kleine rechteckige Ablagefelder gelegt und darf deswegen auch von manchen Helden getragen werden, die kein großes Ablagefeld haben.
\supercite{DRidN:Begleitheft}

\fan{Haben alle (!) Helden kein großes Ablagefeld, gilt folgendes (die \hyperref[Feuerwächterin]{Feuerwächterin} und der \hyperref[Hautwandler]{Hautwandler} gelten hierfür immer als ein großes Ablagefeld habend):}

\bullet{} \fan{Deckt ihr in Legende 9 "Die Mächte des Meeres" auf Feld 100 die Auftragskarte "Bogen" oder "Fass" auf, ersetzt sie mit der Auftragskarte "Eisen".}

\bullet{} \fan{Zieht ihr in Legende 10 "Hadria" die Legendenkarte Q mit der Heldenaufgabe "Fässer", zieht stattdessen eine andere Legendenkarte Q.}

\bullet{} \fan{Lest ihr in Legende 10 "Hadria" die Karte "Der Eiserne Turm wurde erobert" vor, dürft ihr direkt danach zusätzlich die Verbündetenkarte "Taren" vorlesen.}

\end{bgcolor}



\begin{bgcolor}[color-DlH]
\section{Die letzte Hoffnung (2016)}





\subsection{Legende 17, Die letzte Hoffnung:}

Tauscht ein Held mit dem Bruderschild mehr Stärkepunkte mit einem anderen Helden, als dieser haben kann, verfallen die überschüssigen Stärkepunkte.


\subsection{Bronzene Ereigniskarte 227:}

\fan{Ein Held, der nur weniger als 14 Willenspunkte haben kann, darf seine Willenspunkte auf seine höchste Anzahl erhöhen.}


\end{bgcolor}





\begin{bgcolor}[color-AG]
\section{Alte Geister (2018)}

\subsection{Eisplättchen:}

\fan{Das Eisplättchen "14 Willenspunkte besitzen" ist auch erfüllt, wenn der Held seine höchste Anzahl Willenspunkte hat, auch wenn diese höchste Anzahl kleiner als 14 ist.}

\end{bgcolor}





\begin{bgcolor}[color-DZ]
\section{Düstere Zeiten (2020)}

\subsection{Eule:}


Mit der Eule können nur Gegenstände verschickt werden, die der Sender aktuell trägt, und höchstens so viele Gegenstände, wie der Empfänger tragen kann. Gegenstände, die ein Held nicht tragen kann, können nicht an diesen Helden geschickt werden. Die Eule kann nur an einen Helden geschickt werden, der ein freies großes Ablagefeld hat.
\supercite{DZ:Begleitheft}


\subsection{Zu viele Gegenstände auf dem Boot:}

\fan{Würde ein Held auf dem Boot mehr Gegenstände erhalten, als er tragen kann, gehen überzählige Gegenstände zurück in den Vorrat.}


\subsection{Kjalls Spitzhacke:}

Die Spitzhacke wird auf zwei kleine rechteckige Ablagefelder gelegt und darf deswegen auch von manchen Helden getragen werden, die kein großes Ablagefeld haben.


\subsection{Lager der Trolle:}

\fan{Ein Held, der seine höchste Anzahl Willenspunkte hat, wird nicht vom Lager der Trolle (Feld 39) auf Feld 38 versetzt, auch wenn er weniger als 14 Willenspunkte hat.}

\end{bgcolor}



\begin{bgcolor}[color-DeK]
\section{Die ewige Kälte (2022)}


\subsection{Tafeln der drei Brüder:}

\fan{Ein Held darf die Tafeln der drei Brüder auch aktivieren, wenn sie alle auf seinem Feld liegen und das kein Händlerfeld ist. Das geht, auch wenn der Held nicht alle Tafeln gleichzeitig tragen kann.}


\subsection{Bonus-Legende "Orennah":}

\fan{Ein Held, der seine höchste Anzahl Willenspunkte hat und mit der Myrk auf Feld der Höhle der Verwandlung (Feld 414) steht, darf das Legendenziel erfüllen, auch wenn er weniger als 14 Willenspunkte hat.}

\end{bgcolor}


\begin{bgcolor}[color-DfL]
\section{Das ferne Land (2025)}



\subsection{Greifenkrone:}

\fan{Ein Held, der keine Helmablage hat, darf die Greifenkrone stattdessen auf seiner Goldablage tragen. Er kann dann aber nicht durch sie Würfel drehen.}

\end{bgcolor}



\end{multicols*}










\chapter{Regeln für Bonusstärke}
\label{Bonusstärke}



\begin{multicols*}{2}\raggedcolumns

\begin{nobgcolor}


Diese Regeln gelten für Bonusstärke durch:

\bullet{} die Kampfblitze des \hyperref[Eis-Dämon]{Eis-Dämons},

\bullet{} das Feuer der \hyperref[Feuerwächterin]{Feuerwächterin},

\bullet{} den Feuerstoß der \hyperref[Feuerbezwingerin]{Feuerbezwingerin},

\bullet{} die Hoffnungskarte des \hyperref[Halbskral]{Halbskrals},

\bullet{} der \hyperref[Hautwandler]{Hautwandler} als Bär,

\bullet{} die zweite Sonderfähigkeit des \hyperref[Ritter]{\fan{Ritters}},
 
\bullet{} die Sonderfähigkeit des \hyperref[Wolfskrieger]{Wolfskriegers},

\bullet{} die Zeitzauberei des \hyperref[Zeitzauberer]{Zeitzauberers},

\bullet{} die Zwergenwut-Sonderfähigkeit und zweite Sonderfähigkeit des \hyperref[Zwerg]{Zwergs},

\bullet{} andere Kampfstärke ändernden Effekte, die Stärkesteine nicht verschieben (z.B. durch Gegenstände, \hyperref[Helfer]{\fan{Helfer}}, das Licht der fünften Stunde, den Bannkreis von Choranat oder die Bruderfeuer).



\section{Allgemein}


\subsection{Wirkung:}

Sofern nichts anderes angegeben, wird jede Bonusstärke in jeder Kampfrunde einmal zum Kampfwert der Helden addiert, egal wie viele Helden kämpfen.
\supercite{17907}

Alle Bonusstärken werden zum Kampfwert addiert, nicht nur die höchste.

Bonusstärke verschiebt nie Stärkesteine. Sofern nichts anderes angegeben, ist Bonusstärke nicht durch die Länge der Stärkeleiste begrenzt.



\subsection{Holzstämme/Eisen:}

Für das Tragen von Holzstämmen/Eisen zählt nur die Position des Stärkesteins, keine Bonusstärke.
\supercite{DRidN:Begleitheft,DlH:Begleitheft,AG:Begleitheft,DfL:Begleitheft}


\end{nobgcolor}



\begin{bgcolor}[color-DLvA]
\section{Die Legenden von Andor (2012)}


\subsection{Legende 3, Die Tage des Widerstands:}

Die Bonusstärke durch Prinz Thorald erhöht den Kampfwert des verhexten Gors. 
\supercite{DLvA3:Der_verhete_Gor}
Andere Bonusstärke verändert den Kampfwert des verhexten Gors nicht.
\supercite{155,3450}


\subsection{Bonus-Legende "Die Befreiung der Mine":}

\fan{Taucht Varkur bei der Karte "Varkurs Versteck" in der Schatzkammer auf: Für die Bestimmung der wenigsten Stärkepunkte zählen nur die Stärkesteine der Helden, keine Bonusstärke. Bonusstärke wird weiterhin zum Kampfwert der Helden addiert.}


\subsection{Bonus-Legende "Der Kampf um Cavern":}

\fan{Pergament 14 verdoppelt den Wert des Stärkesteins, keine Bonusstärke.}

\end{bgcolor}



\begin{bgcolor}[color-StSch]
\section{Der Sternenschild (2013)}

\subsection{Wolfsstärke:}

Bonusstärke wirkt auch beim Zähmen der Wölfe. Für die Wolfsstärke (sowohl beim Zähmen als auch nach dem Zähmen) zählt nur die Position des Stärkesteins des Wolfszähmers, keine Bonusstärke.
\supercite{StSch:FAQ}

\end{bgcolor}



\begin{nobgcolor}

\section{Neue Helden (2014)}


\subsection{Spielvariante "Der Bruderschild":}

Für den Stärketausch durch den Bruderschild zählen nur die Positionen der beiden Stärkesteine, keine Bonusstärke.


\end{nobgcolor}


\begin{bgcolor}[color-DRidN]
\section{Die Reise in den Norden (2014)}

\subsection{Legende 9, Die Mächte des Meeres:}

Für Kenvilars Tücke "dieselben Stärkepunkte" zählt nur die Position des Stärkesteins, keine Bonusstärke.
\supercite{DRidN9:Kenvilars_Tücke}


\subsection{Legende 10, Hadria:}

Für die Stärke der Taren zählt nur die Position des Stärkesteins, keine Bonusstärke.
\supercite{DRidN10:Taren}

\end{bgcolor}





\begin{bgcolor}[color-DlH]
\section{Die letzte Hoffnung (2016)}


\subsection{Legende 15, Der vergiftete Geist:}

Beim Übertragen der Stärkepunkte des verhexten Helden auf Nomions Tafel zählt nur die Position des Stärkesteins, keine Bonusstärke.
\supercite{DlH15:Enttarnt}


\subsection{Legende 17, Die letzte Hoffnung:}

Für den Stärketausch durch den Bruderschild zählen nur die Positionen der beiden Stärkesteine, keine Bonusstärke.
\supercite{DH:Begleitheft}

\end{bgcolor}




\begin{bgcolor}[color-AG]
\section{Alte Geister (2018)}

\subsection{Eisplättchen:}

Für das Eisplättchen "6 Stärkepunkte besitzen" zählt nur die Position des Stärkesteins, keine Bonusstärke.
\supercite{AG:Begleitheft}

\end{bgcolor}



\end{multicols*}







\chapter{Regeln für \fan{Helfer}}
\label{Helfer}


\begin{tabular}{|l|l|l|l|}
    \hline
    & Kampfeffekt bei Held
    & Kampfeffekt bei Gegner
    & Kein Kampfeffekt
    \\
    \hline
    Helfer
    &
    \begin{minipage}[t][][t]{0.2\textwidth}

        % Eigene Bewegung, Kampfeffekt bei Held

        \bullet{} Der genesene Mart\supercite{DZ1:Am_Baum_der_Lieder}
    
        \bullet{} Radan\supercite{DlH17:Hoffnungskarte_Zwerg}

        \bullet{} Takuri Turr

        \bullet{} Wassergeist Vara
        \\
        
    \end{minipage}
    &
    \begin{minipage}[t][][t]{0.3\textwidth}

        % Eigene Bewegung, Kampfeffekt bei Gegner

        \bullet{} Bewahrer vom Baum der Lieder (auch angrenzend)\supercite{DZ:Begleitheft}

        \bullet{} Flederfuchs Flaps\supercite{DeK:Begleitheft,DfL:Begleitheft}

        \bullet{} Harthalt\supercite{DZ:Begleitheft}

        \bullet{} Hexe Reka 
        
        ("Der Hexer aus Andor")\supercite{AG2:Reka_und_die_Helden_von_Andor,AG2:Rekas_Hexenkunst}

        \bullet{} Knochen-Golem

        \bullet{} Meres\supercite{AG:Begleitheft,AG2:D,AG3:In_der_Mine}
     
        \bullet{} Prinz Thorald\supercite{DLvA:Begleitheft}

        \bullet{} Schildzwerge\supercite{DLvA:Begleitheft}

        \bullet{} Seeriese (auch angrenzend)\supercite{DRidN10:Turm_bewahrt}

        \bullet{} Tranuk\supercite{DeK:Begleitheft}/Yetohe-Krieger\supercite{DeK4:I}

        \bullet{} Trollkönig/Wachtrolle\supercite{DfL6:Der_Nachkomme_des_Oron,DfL7:I}

        \bullet{} Unbekannter Krieger\supercite{BL_RdSK:Der_Unbekannte_Krieger}

        \bullet{} Wölfe/Lonas\supercite{BL_Jagd:FAQ,BL_Rück:A}
        \\
        
    \end{minipage}
    & 
    \begin{minipage}[t][][t]{0.28\textwidth}

        % Eigene Bewegung, kein Kampfeffekt

        \bullet{} Irja, die Nixe
        
        (Die Suche nach Grenolin)\supercite{BL_SnG:Die_Nixe_Irja}

        \bullet{} Sabri

        \bullet{} Schatten\supercite{Dunkle_Eara:Regelkarte}

        \bullet{} Schwarze Ritter\supercite{DfL5:Die_Ritter_erheben_sich,DfL6:Die_Prinzen_Azturiens,DfL7:I}
        \\
        
    \end{minipage}
    \\
    \hline
    Andere
    &
    \begin{minipage}[t][][t]{0.2\textwidth}

        % Keine eigene Bewegung, Kampfeffekt bei Held

        \bullet{} Grenolin, der Barde\supercite{DRidN:Begleitheft}
    
        \bullet{} Melkart\supercite{BB4:A}

        \bullet{} Schildzwerge ("Fremde Heimat")\supercite{DZ2:Ankunft_in_der_fremden_Heimat_Berg-Pfad}

        \bullet{} Tulgori\supercite{AG:Begleitheft}
        \\
        
    \end{minipage}
    &
    \begin{minipage}[t][][t]{0.3\textwidth}

        % Keine eigene Bewegung, Kampfeffekt bei Gegner

        \bullet{} Hallgard\supercite{AG3:Hallgard}

        \bullet{} Marun, die Tapfere\supercite{DH:Marun}
    
        \bullet{} Taren (Legende 10)\supercite{DRidN10:Taren}
        \\
        
    \end{minipage}
    & 
    \begin{minipage}[t][][t]{0.3\textwidth}

        % Keine eigene Bewegung, kein Kampfeffekt

        \bullet{} Fahrende Händlerin Helea\supercite{DfL2:Erkungsungskarte_507}

        \bullet{} Garz, der Handelszwerg

        \bullet{} Hexe Reka

        \bullet{} Irja, die Nixe\supercite{ME:Irja}

        \bullet{} König Brandur\supercite{BL_EdK:A}

        \bullet{} Der verletzte Mart\supercite{DZ1:42}

        \bullet{} Merrik, der Kartograph\supercite{StSch:G_Merrik,DRidN8:O,DH:Begleitheft}

        \bullet{} Silberzwerge\supercite{DRidN8:Flut}

        \bullet{} Taren\supercite{DRidN8:Flut}

        \bullet{} Trosswagen\supercite{DlH13:e,DlH16:a}
        \\
        
    \end{minipage}
    \\
    \hline
\end{tabular}




\begin{multicols*}{2}\raggedcolumns


\begin{nobgcolor}

Diese Regeln gelten für alle \fan{Helfer}: Figuren, die Helden unterstützen und die durch Spieler unabhängig von Heldenfiguren bewegt werden können.

Manche Figuren (z.B. Wölfe, Trollkönig) unterstützen die Helden nur manchmal oder können nur manchmal unabhängig von ihnen bewegt werden. Diese Regeln gelten nur dann für sie.







\section{Allgemein}




\subsection{Bewegung:}

Diese Regeln gelten für alle Helfer, außer es steht explizit anders:

Jeder Held darf als eigene Aktion 1 oder mehr Stunden ausgeben und einen Helfer, der auf dem Spielplan steht, pro Stunde bis zu 4 Felder weit zu bewegen. In jeder Aktion kann nur ein einziger Helfer bewegt werden.

Helfer dürfen Eisbrücken des \hyperref[Eis-Dämon]{Eis-Dämons} nutzen.
\supercite{88780}

Helfer können keine Plättchen oder Karten aufdecken oder aktivieren.
\supercite{NH:Begleitheft,DlH17:Hoffnungskarte_Zwerg,DH:Begleitheft,MH:Begleitheft}

Helfer können keine Plättchen oder Figuren mit sich mitbewegen.
\supercite{NH:Begleitheft,DlH17:Hoffnungskarte_Zwerg,DH:Begleitheft,MH:Begleitheft}
Ausnahme: Sabri darf bis zu 2 Pulver tragen (bei der Schwächungskarte "nur noch 2 Pulver" in Legende 14 bis zu 3).

\subsection{Willenspunkte-Schranken:}

Kann eine Verbindung zwischen zwei Feldern nur von Helden mit genug Willenspunkten überquert werden (z.B. Sprungfelder in "Die letzte Hoffnung" oder der steile Felsen in "Das ferne Land"), darf ein Helfer diese Verbindung immer nutzen.
\supercite{DlH17:Hoffnungskarte_Zwerg,DH:Begleitheft,94521}


\subsection{Festhaltende Plättchen und Figuren:}

Helfer werden von festhaltenden Plättchen und Figuren (z.B. Skeletten in "Die letzte Hoffnung" oder Spornen in "Das ferne Land") nicht verfolgt oder festgehalten.\supercite{DlH17:Hoffnungskarte_Zwerg,AG:Begleitheft,MH:Begleitheft}



\subsection{"Alleine kämpfen" und "am Sieg beteiligt sein":}


Besagt eine Aufgabe oder Regel, dass ein Held einen Gegner alleine bekämpfen oder besiegen soll oder dass ein Held nicht gemeinsam kämpfen darf, darf er dennoch \hyperref[Bonusstärke]{Bonusstärke} und Kampf-Effekte durch Helfer erhalten.
\supercite{StSch:Varkurs_Triumph,5623,17369}

Ausnahme: Prinz Thorald darf für die Erfüllung einer Schicksalskarte nicht mitkämpfen.
\supercite{1947,17369}

Besagt eine Aufgabe oder Regel, dass ein Held am Sieg über einen Gegner beteiligt sein soll, muss der Held selbst mitkämpfen. Es genügt nicht, wenn ein von ihm gesteuerter Helfer den Kampf beeinflusst.
\supercite{28472}







\subsection{Helfer bestimmter Helden:}

\bullet{} Der Knochen-Golem kann nur von seiner \hyperref[Beschwörerin]{Beschwörerin} als eigene Aktion bewegt werden.

\bullet{} Radan kann nur von seinem \hyperref[Zwerg]{Zwerg} als eigene Aktion bewegt werden.\supercite{DlH17:Hoffnungskarte_Zwerg}

\bullet{} Sabri kann nicht von Helden bewegt werden. Bewegt der Erzähler sich (egal ob vorwärts oder rückwärts), läuft Sabri 1 Schritt auf dem kürzesten Weg zum Steppennomaden. \fan{Gibt es keinen Weg zum Steppennomaden, läuft Sabri auf ein beliebiges angrenzendes Feld.}


\bullet{} Der Schatten kann nur von seiner \hyperref[Dunkle Magierin]{Dunklen Magierin} oder \hyperref[Zauberin]{Zauberin} als eigene Aktion bewegt werden.

\bullet{} Der Takuri kann nicht als eigene Aktion bewegt werden. Nur die \hyperref[Takuri-Hüterin]{Takuri-Hüterin} darf die tulgorische Steinflöte von der Vorderseite auf die Rückseite drehen, um den Takuri auf ein beliebiges Spielplanfeld zu versetzen.

\bullet{} Der Wassergeist kann nur von seiner \hyperref[Hüterin]{Hüterin} als eigene Aktion bewegt werden.
\supercite{NH:Begleitheft}

\end{nobgcolor}



















\begin{bgcolor}[color-DLvA]
\section{Die Legenden von Andor (2012)}

\subsection{Brücken:}

Helfer können durch ein rotes X oder Geröll blockierte Brücken nicht benutzen.

Ausnahme: Der Knochen-Golem\supercite{AG:Begleitheft}, der Takuri\supercite{MH:Begleitheft} und der Wassergeist\supercite{15332,AG:Begleitheft} dürfen blockierte Brücken überqueren.



\subsection{Geröll:}

\fan{Helfer, die \hyperref[Bonusstärke]{Bonusstärke} geben, wenn sie auf dem Feld des Gegners stehen, geben diese Bonusstärke auch gegen Geröll, und das auch, wenn sie angrenzend stehen.}

\fan{Helfer können Felder mit Geröll nicht betreten. }

\fan{Ausnahme:} Der Knochen-Golem\supercite{AG:Begleitheft}, der Takuri\supercite{MH:Begleitheft,97438} und der Wassergeist\supercite{6937,AG:Begleitheft} dürfen Geröll ignorieren.



\subsection{Cavern:}

Helfer müssen auf dem Geheimen See nicht stehenbleiben\supercite{NH:Begleitheft} und sind nicht von Feuerstößen betroffen.



\subsection{Legende 4, Das Geheimnis der Mine:}

Helfer lösen keinen Alarm aus.
\supercite{NH:Begleitheft}


\subsection{Legende 5, Der Zorn des Drachens:}

\fan{Helfer können die Rietburg (Feld 0) immer betreten, auch wenn sie von Kreaturen besetzt ist.}

\fan{Ausnahme:} Wie Helden können Prinz Thorald\supercite{28830}, \fan{die Schildzwerge} und der Knochen-Golem\supercite{DH:FAQ} die Rietburg erst betreten, nachdem sie von Kreaturen befreit wurde.


\end{bgcolor}



\begin{bgcolor}[color-StSch]
\section{Der Sternenschild (2013)}

\subsection{Gezähmte Wölfe:}

Gezähmte Wölfe sind eine Ausnahme: Sie können nur vom Wolfsfreund als eigene Aktion bewegt werden und verschiedene Wölfe dürfen in verschiedenen Stunden derselben Aktion bewegt werden.
\supercite{StSch:Wolfsfreund}

\end{bgcolor}



\begin{bgcolor}[color-DRidN]
\section{Die Reise in den Norden (2014)}

\subsection{Meeresfelder:}

Helfer können Meeres- und Klippenfelder nicht betreten.

Ausnahme: Der Wassergeist sowie in Legende 10, "Hadria", der Seeriese dürfen Meeres- und Klippenfelder\supercite{10815} betreten.
\supercite{DRidN:Begleitheft,DRidN:FAQ} 
Der Seeriese kann keine Landfelder betreten.
\supercite{DRidN10:Turm_bewahrt} 

\subsection{Schiff:}

Helfer dürfen das Schiff mit einem Schritt betreten oder verlassen (von/zu einem Landfeld). 
\supercite{MH:Begleitheft}
Sie können nicht segeln.
\supercite{32649}
\fan{Jeder Helfer, der \hyperref[Bonusstärke]{Bonusstärke} gibt, wenn er auf dem Feld des Gegners steht, gibt diese Bonusstärke auch, wenn er auf dem Schiff angrenzend zum Gegner steht.}

\fan{Ausnahme:} Der Takuri darf das Schiff betreten und verlassen, auch wenn es nicht an ein Landfeld angrenzt.
\supercite{88939}

\fan{Ausnahme:} Der Wassergeist sowie in Legende 10, "Hadria", der Seeriese können das Schiff nicht betreten.
\supercite{DRidN:Begleitheft}


\subsection{Bootsstrecken:}

Helfer dürfen für je 2 Schritte Bootsstrecken benutzen.
\supercite{BL_RdSK:Der_Unbekannte_Krieger}

\fan{Ausnahme: Der Wassergeist kann keine Bootsstrecken benutzen.}

Ausnahme: Sabri darf Bootsstrecken für je 1 Schritt nutzen (statt 2).
\supercite{MH:Begleitheft}


\subsection{Schneeplättchen:}

\fan{Helfer aktivieren keine Schneeplättchen und müssen nicht stehen bleiben, wenn sie ein Feld mit Schneeplättchen betreten.}


\subsection{Legende "Die Rückkehr der Schwarzen Kogge":}

Der Unbekannte Krieger ist eine Ausnahme: Er kann pro Stunde nur 1 Schritt weit bewegt werden (statt bis zu 4) und er darf Plättchen aktivieren.
\supercite{BL_RdSK:Der_Unbekannte_Krieger}


\end{bgcolor}



\begin{bgcolor}[color-DlH]
\section{Die letzte Hoffnung (2016)}







\subsection{Legende 14, Der Meister des Trolls:}

Steht ein Helfer auf der Brücke, während diese abgerissen wird, stellt ihn auf Feld 205 oder 207.
\supercite{94128}

\end{bgcolor}



\begin{bgcolor}[color-AG]
\section{Alte Geister (2018)}

\subsection{Gesperrte Felder:}

Helfer dürfen Felder mit Arbaks und Eisplättchen betreten. 
\supercite{AG:Begleitheft,MH:Begleitheft}


\subsection{Casamatuc:}

Ein Helfer darf mit dem Casamatuc versetzt werden, wenn er mit dem Helden auf demselben Feld steht. 
\supercite{AG:Begleitheft}

\subsection{Takuri-Spiegel:}

Helfer können nicht mit dem Takuri-Spiegel versetzt werden.
\supercite{AG:Begleitheft}

\end{bgcolor}



\begin{bgcolor}[color-DZ]
\section{Düstere Zeiten (2020)}

\subsection{Boot:}

\fan{Helfer dürfen das Boot für je 1 Schritt betreten oder verlassen, aber nur ein Held kann das Boot fahren.}


\subsection{Zwergentüren:}

Helfer können Zwergentüren nicht öffnen und verhindern nicht das Zudecken von Zwergentüren auf ihrem Feld beim Sonnenaufgang.
\supercite{DZ:Begleitheft}

Helfer können für je 1 Schritt zwischen zwei offenen Zwergentüren mit passenden Symbolen laufen.

\subsection{Lager der Trolle:}


Helfer können das Lager der Trolle nicht betreten.
\supercite{DZ:Begleitheft}

Ausnahme: Der Takuri\supercite{MH:Begleitheft} und der Wassergeist\supercite{DZ:Begleitheft} dürfen das Lager der Trolle betreten.
\supercite{130389}



\subsection{Legende 2, Fremde Heimat:}

\fan{Thogger: Helfer werden versetzt, wenn ein Geröllplättchen auf ihr Feld gelegt wird, auch wenn sie Felder mit Geröllplättchen betreten dürfen.}



\subsection{Legende 3, Der letzte Schatten:}

\fan{Helfer dürfen Felder mit Schattenplättchen betreten und verlassen.}

\end{bgcolor}








\begin{bgcolor}[color-DeK]
\section{Die ewige Kälte (2022)}

\subsection{Eis und der große See Ava:}

\fan{Helfer aktivieren nie Eisplättchen und müssen auf Feldern mit Eisplättchen nicht stehen bleiben.}

Helfer dürfen Seefelder nicht betreten, egal ob dort Eisplättchen liegen.
\supercite{DeK2:A}

\fan{Ausnahme: Sollte ein Helfer auf einem Seefeld oder Feld 460 stehen, darf er Seefelder auf einem kürzesten Weg zum Ufer betreten, egal ob dort Eisplättchen liegen. }

\fan{Ausnahme: Der Knochen-Golem, der Takuri und der Wassergeist dürfen Seefelder immer betreten, egal ob dort Eisplättchen liegen.}



\subsection{Höhle der Verwandlung:}

\fan{Helfer dürfen die Höhle der Verwandlung (Feld 414) betreten.}

\end{bgcolor}




\columnbreak




\begin{bgcolor}[color-DfL]
\section{Das ferne Land (2025)}



\subsection{Großer See Ava:}

\fan{Für den Wassergeist sind alle an den See angrenzenden Felder zueinander angrenzend (auch welche im Schachtelboden).}



\subsection{Schwarze Ritter:}

\fan{Die Schwarzen Ritter als Gegner verfolgen keine Helfer.}

Die Schwarzen Ritter als Helfer sind eine Ausnahme: Sie können nicht als eigene Aktion bewegt werden, sondern je einmal pro Legende auf ein beliebiges Feld ohne Kreaturen, Oron, Steppenvolk oder Greifenfels (Feld 425) versetzt werden.
\supercite{DfL5:Die_Ritter_erheben_sich,DfL6:Die_Prinzen_Azturiens,DfL7:I}



\subsection{Schachtelboden:}

\fan{Helfer können die Pfahlbausiedlung und die geheime Trollhöhle betreten.}





\subsection{Geister:}

\fan{Helfer dürfen Felder mit oder angrenzend zu Geistern betreten und verlassen.}

\end{bgcolor}

\end{multicols*}








\chapter{Regeln fürs Spiel mit 5 oder 6 Helden}
\label{5 oder 6 Helden}


\begin{multicols*}{2}\raggedcolumns


\begin{nobgcolor}

\section{Allgemein}



\subsection{Von der Heldenanzahl abhängige Angaben:}

Es gelten die Angaben für 4 Helden, außer es steht explizit anders. 
Zum Beispiel gibt es bei 5 oder 6 Helden üblicherweise 1 goldenen Schild und jeder Endgegner hat so viele Stärkepunkte wie bei 4 Helden.
\supercite{NH:Begleitheft,DH:Begleitheft,MH:Begleitheft,BiBo:Bonus-Heft}

Effekte, die jeden Helden betreffen (z.B. "Jeder Held verliert 2 Willenspunkte"), gelten für alle 5 oder 6 Helden, nicht nur für 4.




\subsection{Spielvarianten:}

Wählt am Anfang eine (!) der Spielvarianten "Erschwerte Kämpfe"\supercite{AG:Begleitheft,DZ:Begleitheft} \fan{und eine (!) der Spielvarianten "Weniger Zeit".}



\subsection{Spielvarianten "Erschwerte Kämpfe":}

\bullet{} \fan{Addiert in jedem Kampf zum gegnerischen Kampfwert:}

\begin{tabular}{|c|c|c|}
    \hline
    & \fan{Gegner}
    & \fan{Endgegner}
    \\
    \hline
    \fan{bei 5 Helden}
    & \fan{+2}
    & \fan{+4}
    \\
    \hline
    \fan{bei 6 Helden}
    & \fan{+4}
    & \fan{+8}
    \\
    \hline
\end{tabular}


\bullet{} Kreaturenleisten (aus "Neue Helden"\supercite{NH:Begleitheft} oder "Die Reise in den Norden"\supercite{DRidN:Begleitheft}) und Schwarzer Herold (aus "Neue Helden"\supercite{NH:Begleitheft}, "Dunkle Helden"\supercite{DH:Begleitheft} oder "Big Box"\supercite{BiBo:Bonus-Heft})

\bullet{} Schwarzer Augenwürfel und Schwarzer Herold (aus "Dunkle Helden"\supercite{DH:Begleitheft} oder "Big Box"\supercite{BiBo:Bonus-Heft})

\bullet{} Bannkreis von Choranat (aus "Magische Helden")
\supercite{MH:Begleitheft}

\bullet{} \fan{Mithasis, erquickendes Kraut (Butterbrotbär\&Mivo)}
\supercite{132952}




\subsection{Spielvarianten "Weniger Zeit":}

\bullet{} \fan{Markiert bei 5 Helden die letzte, bei 6 Helden die letzte und vorletzte normale Stunde der Tagesleiste (z.B. mit Sternchen). Die markierten Stunden gelten als Überstunden, kosten also üblicherweise je 2 Willenspunkte.}

\bullet{} Neutrale Zeitsteine (aus "Neue Helden"\supercite{NH:Begleitheft}, "Dunkle Helden"\supercite{DH:Begleitheft}, "Magische Helden"\supercite{MH:Begleitheft} oder "Big Box"\supercite{BiBo:Bonus-Heft}).

\bullet{} \fan{Farbige Zeitsteine bei 5 bis 6 Helden (Mivo)}
\supercite{130491}

\bullet{} \fan{Mithasis, erquickendes Kraut (Butterbrotbär\&Mivo)}
\supercite{132952}


\end{nobgcolor}







\begin{bgcolor}[color-DLvA]
\section{Die Legenden von Andor (2012)}


\subsection{\fan{Legende 1, Die Ankunft der Helden:}}


\fan{Spielt am Anfang ganz ohne Spielvarianten für 5 oder 6 Helden. Erreicht der Erzähler "C", wählt eine (!) der Spielvarianten "Erschwerte Kämpfe" für den Rest der Legende. Es kommt keine Spielvariante "Weniger Zeit" ins Spiel.}









\subsection{Legende 3, Die Tage des Widerstands:}



Nur die beiden Helden mit dem höchsten Rang und die beiden Helden mit dem niedrigsten Rang erhalten eine Schicksalskarte.
\supercite{NH:Begleitheft,DH:Begleitheft,MH:Begleitheft,BiBo:Bonus-Heft}


Zieht ein Held Schicksal 5 "Die Bauernhochzeit", darf er ein anderes Schicksal ziehen.
\supercite{NH:Begleitheft,DH:Begleitheft,MH:Begleitheft,BiBo:Bonus-Heft}

Für Schicksal 7 oder 10 müssen die Helden nur am Sieg über diese Kreaturen beteiligt sein, statt sie alleine zu besiegen.
\supercite{NH:Begleitheft,DH:Begleitheft,MH:Begleitheft,BiBo:Bonus-Heft}

Der Schwarze Herold stärkt den verhexten Gor nicht.
\supercite{NH:Begleitheft,DH:Begleitheft,BiBo:Bonus-Heft}





\subsection{Bonus-Legende "Der Kampf um Cavern":}

Spielt ihr mit dem Schwarzen Herold, stärkt er das Geröll.
\supercite{BL_KuC:B}




\subsection{Ereigniskarte "Geheimer See" 6:}

Spielt ihr mit neutralen Zeitsteinen, werden alle neutralen Zeitsteine auf die 5. Stunde versetzt. Jeder Held, der seinen Tag bereit beendet hat, kann an diesem Tag jedoch keine Handlungen mehr vornehmen.
\supercite{NH:Begleitheft,DH:Begleitheft,MH:Begleitheft,BiBo:Bonus-Heft}

\end{bgcolor}



\begin{bgcolor}[color-StSch]
\section{Der Sternenschild (2013)}


\subsection{Erschwerte Kämpfe – Kreaturenleisten:}

Spielt ihr mit den Kreaturenleisten, stärken sie auch den Dieb Ken Dorr.
\supercite{NH:Begleitheft}

\subsection{Erschwerte Kämpfe – Der Schwarze Herold:} 

Der Schwarze Herold stärkt den Krahder Ferntahr nicht.
\supercite{NH:Begleitheft}


\subsection{Fürstenaufgabe "Zeugnisse":}

Es werden wie bei 4 Helden 4 weitere Pergamente eingewürfelt.
\supercite{NH:Begleitheft}

\end{bgcolor}





\begin{bgcolor}[color-DRidN]
\section{Die Reise in den Norden (2014)}


\subsection{Schiffsausbau:}

Schiffsausbauten kosten:
\supercite{DRidN:Begleitheft}

\bullet{} bei 5 Helden je 5 Gold,

\bullet{} bei 6 Helden je 6 Gold.



\subsection{Legende 10, Hadria:}

\fan{Spielt am Anfang ohne Spielvarianten für 5 oder 6 Helden. Erreicht der Erzähler "P", wählt eure Spielvarianten für 5 oder 6 Helden. Diese gelten dann für den Rest der Legende.}

Stellt am Anfang den Helden mit dem höchsten und den mit dem zweithöchsten Rang auf den Buchstaben "P". Erreicht der Erzähler "P", werden sie auf das Feld gestellt, auf dem dann alle anderen Helden stehen. Bis dann nehmen sie nicht am Spiel teil und können keine Startausrüstung erhalten.
\supercite{DRidN10:5-6}


Spielt ihr mit dem Schwarzen Herold, stärkt dieser die Zauberer, die ihr bekämpft, und auch Qurun.
\supercite{DRidN:Begleitheft}

\end{bgcolor}


\begin{bgcolor}[color-DlH]
\section{Die letzte Hoffnung (2016)}


\subsection{Spielvarianten:}

Wählt wie üblich eine (!) der Spielvarianten "Erschwerte Kämpfe".
Nutzt keine Spielvariante "Weniger Zeit". Wählt stattdessen nach den "a"-Karten eine (!) der Spielvarianten "Mehr Nahrung".


\subsection{Spielvarianten "Mehr Nahrung":}

\bullet{} \fan{An jedem Tag erhält derjenige Held, der seinen Tag als erstes beendet, 1x Apfelnüsse.}

\bullet{} Sporne (aus "Dunkle Helden")
\supercite{DH:Sporne,DH17:a0_5-6}

\bullet{} Mondbeeren (aus "Magische Helden")
\supercite{MH:Mondbeeren,MH:Begleitheft}

\bullet{} \fan{Mithasis, erquickendes Kraut (Butterbrotbär\&Mivo)}
\supercite{132952}




\subsection{Erschwerte Kämpfe – Schwarzer Augenwürfel:}

Nicht nur jede Kreatur, sondern auch jedes Skelett und jeder Krahder nutzt zusätzlich den schwarzen Augenwürfel.
\supercite{DH17:a0_5-6}
Dies gilt weiterhin nicht für Endgegner.






\subsection{\fan{Legende 11, Das Graue Gebirge:}}

\fan{Spielt am Anfang ohne Spielvarianten für 5 oder 6 Helden. Erreicht der Erzähler "b", wählt eure Spielvarianten für 5 oder 6 Helden. Diese gelten dann für den Rest der Legende.}


\subsection{Legende 13, Der Tross der Andori:}

Die Bewegung des Trosswagens kostet:
\supercite{DH13:e_5-6,MH:DlH_5-6}

\bullet{} bei 5 Helden 4 Stunden pro Feld,

\bullet{} bei 6 Helden 5 Stunden pro Feld.



\subsection{Legende 14, Der Meister des Trolls:}

Auf c werden Heldenwappen bereitgelegt:
\supercite{DH14:c_5-6,MH:DlH_5-6}

\bullet{} bei 5 Helden auf die Buchstaben d, f und h

\bullet{} bei 6 Helden auf die Buchstaben d, f, h und i


Spielt ihr mit dem Schwarzen Herold, stärkt er den Urtroll, aber nicht Nomion.
\supercite{DH:Begleitheft}


\subsection{Legende 15, Der vergiftete Geist:}

Es gibt wie üblich 1 goldenen Schild, obwohl es bei 4 Helden ausnahmsweise 2 goldene Schilde gäbe.
\supercite{DlH15:a4_5-6,MH:DlH_5-6}

\fan{Habt ihr nicht genug "Verhext?"-Karten für alle Helden, mischt stattdessen verdeckt so viele Bewegungsplättchen, wie Helden mitspielen, darunter das "Jede Horde 3 Felder"-Plättchen. Jeder Held erhält ein verdecktes Bewegungsplättchen, das sich nur dessen Spieler ansehen können. Redet nicht über die Rückseiten dieser Plättchen. Der Held mit dem "Jede Horde 3 Felder"-Plättchen ist der Verhexte.}

\subsection{Legende 16, Im Schatten der Winterburg:}

Die Bewegung des Trosswagens kostet:
\supercite{DH16:a2_5-6,MH:DlH_5-6}

\bullet{} bei 5 Helden 4 Stunden pro Feld,

\bullet{} bei 6 Helden 5 Stunden pro Feld.




\subsection{Legende 17, Die letzte Hoffnung:}

Es werden nur 4 Heldenwappen auf die Legendenleiste gelegt. Es kommen also nicht alle Hoffnungskarten ins Spiel.
\supercite{DH17:a0_5-6,MH:DlH_5-6}

Spielt ihr mit dem Schwarzen Herold, stärkt er Borghorn.
\supercite{DH17:a0_5-6}

\end{bgcolor}



\begin{bgcolor}[color-AG]
\section{Alte Geister (2018)}


\subsection{Legende 2, Der Hexer aus Andor:}

Maasavi haben:
\supercite{AG:Begleitheft}

\bullet{} bei 5 Helden 5 Stärkepunkte,

\bullet{} bei 6 Helden 6 Stärkepunkte.


\subsection{Legende 3, Die steinernen Drei:}

Spielt ihr mit dem Schwarzen Herold, stärkt er alle drei Zwergenstatuen.
\supercite{AG:Begleitheft}

\fan{Spielt ihr mit weniger normalen Stunden und liegt Garahs Stundenplättchen auf einer markierten neuen Überstunde, kostet diese 3 statt 2 Willenspunkte.}

\end{bgcolor}



\begin{bgcolor}[color-DZ]
\section{\fan{Düstere Zeiten (2020)}}

\subsection{Erschwerte Kämpfe – Schwarzer Herold:}

Spielt ihr mit dem Schwarzen Herold, stärkt er Thogger, Shan, Kjall und Rudnar, aber nicht den Schatten.
\supercite{DZ:Begleitheft}


\end{bgcolor}


\begin{bgcolor}[color-DeK]
\section{\fan{Die ewige Kälte (2022)}}


\subsection{\fan{Legende 1, Der Winterstein:}}


\fan{Für die dritte Aufgabe des Einführungsspiels müssen nur 4 Helden ein Schneeplättchen aktivieren, nicht alle.}

\fan{Spielt am Anfang ganz ohne Spielvarianten für 5 oder 6 Helden. Erreicht der Erzähler "D", wählt eine (!) der Spielvarianten "Erschwerte Kämpfe" für den Rest der Legende. Es kommt keine Spielvariante "Weniger Zeit" ins Spiel.}


\subsection{\fan{Weniger Zeit – Neutrale Zeitsteine:}}

\fan{Bei der "Großen Kräuterkunde", mit Blaubachbeeren + weißen Warzwurzeln, kann der Held einen beliebigen neturalen Zeistein 2 Stunden zurücksetzen.}

\end{bgcolor}




\columnbreak



\begin{bgcolor}[color-DfL]
\section{\fan{Das ferne Land (2025)}}




\subsection{\fan{Farbige Perlen:}}

\fan{Ordnet in den Legenden 3 und 4 jedem Helden am Anfang eine beliebige Perlenfarbe (Blau, Grün, Braun oder Violett) zu. Zu jeder Farbe müssen je ein oder zwei Helden gehören.}

\fan{Verfolgt der Schwarze Ritter eine Perlenfarbe, die mehreren Helden gehört, verfolgt er jeweils einen Helden dieser Farbe, der am wenigsten Felder entfernt steht.}

\fan{Ihr dürft ein Spielmaterial der Perlenfarbe neben den Namen eines Helden legen, um euch zu erinnern, welche Perlenfarbe ihm zugeordnet wurde.}



\subsection{\fan{Weniger Zeit – Neutrale Zeitsteine:}}

\fan{Wird ein Held von einem Schwarzen Ritter überwältigt, legt keinen neutralen Zeitstein, sondern ein Spielmaterial in einer passenden Farbe aus dem Vorrat auf die Legendenleiste, um zu markieren, wann die Heldenfigur wieder aufgestellt wird. Alle neutralen Zeitsteine bleiben auf der Tagesleiste und dürfen von allen nicht überwältigen Helden weiter genutzt werden.}


\subsection{\fan{Legende 1, Der wahre Meister:}}


\fan{Spielt am Anfang ganz ohne Spielvarianten für 5 oder 6 Helden. Erreicht der Erzähler "C", wählt eine (!) der Spielvarianten "Erschwerte Kämpfe" für den Rest der Legende. Es kommt keine Spielvariante "Weniger Zeit" ins Spiel.}

\end{bgcolor}





\end{multicols*}




\newpage

\chapter{Wer darf welche Plättchen aufdecken?}
\label{Aufdecken}





\begin{tabular}[t]{|c|c|c|c|}
    \hline
    \textbf{Von fern aufdeckbar} 
    & \multicolumn{3}{|c|}{\textbf{Nicht von fern aufdeckbar}}\\
    (Fernrohr, \hyperref[Bergkrieger]{Bergkrieger},
    & \multicolumn{3}{|c|}{}\\
    \hyperref[Runenmeisterin]{Runenmeisterin})
    & \multicolumn{3}{|c|}{}\\
    \hline
    \multicolumn{2}{|c|}{\textbf{Von nah aufdeckbar}} 
    & \multicolumn{2}{|c|}{\textbf{Nicht von nah aufdeckbar}}\\
    \multicolumn{2}{|c|}{(Rabe des \hyperref[Fährtenleser]{Fährtenlesers}, Flaps Flederfuchs, Kapuze)}
    & \multicolumn{2}{|c|}{}\\
    \hline
    \multicolumn{3}{|c|}{\textbf{Vorhersehbar}} 
    & \textbf{Unvorhersehbar} \\ 
    \multicolumn{3}{|c|}{(\hyperref[Seher]{Seher})}
    & \\ 
    \hline
    \begin{minipage}[t][][t]{0.22\textwidth}
    
        % von fern aufdeckbar

        \bullet{} Alte Waffen

        \bullet{} Drachenknochen
        
        \bullet{} Edelsteine

        \bullet{} \fan{Federplättchen}\supercite{DRidN:FAQ}

        \bullet{} Gaben des Nordens\\(ungenutzt)

        \bullet{} \fan{Geröllplättchen}

        \bullet{} Heilkräuter

        \bullet{} Hoffnungssteine
        
        \bullet{} Kreaturenplättchen (auf dem Spielfeld)\supercite{BL_BdM:A}

        \bullet{} Muscheln

        \bullet{} Mystikplättchen\supercite{DZ:Begleitheft}

        \bullet{} Nebelplättchen

        \bullet{} Pergamente\supercite{BL_KuC:A}

        \bullet{} Roteisensteine\supercite{DZ:Begleitheft,88631}

        \bullet{} Runensteine

        \bullet{} Schneeplättchen\supercite{DeK:Begleitheft}\\(Die ewige Kälte)

        \bullet{} Waldpilze\\(Der Sternenschild)

        \bullet{} \fan{Wrackplättchen}\supercite{DRidN:FAQ}
        
    \end{minipage}
    &
    \begin{minipage}[t][][t]{0.22\textwidth}
    
        % von nah aufdeckbar

        \bullet{} Höhlenplättchen\supercite{DlH:Begleitheft,DlH:Ausrüstungstafel}
        
        \bullet{} Schneeplättchen\supercite{DRidN:Begleitheft}\\(Die Reise in den Norden)

        \bullet{} Pilze\footnote{Deckt \fan{der Rabe des \hyperref[Fährtenleser]{Fährtenlesers} oder} Flaps Flederfuchs den "-3" Pilz auf, geht er für den Rest der Legende aus dem Spiel.\supercite{DfL:Begleitheft}}\\(Das ferne Land)
        
    \end{minipage}
    &
    \begin{minipage}[t][][t]{0.22\textwidth}
    
        % vorhersehbar

        \bullet{} \fan{Azturia}\supercite{DfL:Begleitheft}

        \bullet{} Bewegungsplättchen\footnote{Nur oberstes, nur in "Die letzte Hoffnung".\supercite{DH:Begleitheft}}

        \bullet{} \fan{Eisplättchen\supercite{DeK:Begleitheft,DeK2:Gegenstände}\\(Die ewige Kälte)}

        \bullet{} \fan{Felsen}\supercite{DfL:Begleitheft}

        \bullet{} Heldenwappen

        \bullet{} Kreaturenplättchen (an der Legendenleiste)\supercite{54168}
        
        \bullet{} \fan{Lichtplättchen}

        \bullet{} \fan{Sporne}\supercite{DH:Sporne,DfL:Begleitheft}

        \bullet{} \fan{Unwetterplättchen}

        \bullet{} \fan{Zielmarker}\supercite{DeK:Begleitheft,DeK2:E}
        
    \end{minipage}
    &
    \begin{minipage}[t][][t]{0.23\textwidth}
    
        % nicht aufdeckbar
        
        \bullet{} Bauern\footnote{Ausnahme: In der Bonus-Legende "Die Eskorte des Königs" sind Bauern von fern aufdeckbar.\supercite{BL_EdK:A}}/Iquar
        
        \bullet{} Brunnen/Quellen/Feuer

        \bullet{} \fan{Große blaue Einführungsplättchen}

        \bullet{} Eisplättchen\supercite{AG:Begleitheft}\\(Alte Geister)

        \bullet{} Kartographen\supercite{AG:Begleitheft}

        \bullet{} Mera-Steine\supercite{AG:Begleitheft}

        \bullet{} Zwergentüren\supercite{DZ:Begleitheft}
        
    \end{minipage}
    \\
    \hline
\end{tabular}
\bigskip








\textbf{Von fern aufdeckbare Plättchen:}

\bullet{} Ein Held mit Fernrohr, sowie der \hyperref[Bergkrieger]{Bergkrieger} mit 14 oder mehr Willenspunkten, dürfen, wenn sie auf einem Feld stehen, beliebig viele\supercite{DLvA:FAQ} von fern aufdeckbare Plättchen auf angrenzenden Feldern aufdecken.

\bullet{}  Das Auge-Symbol der \hyperref[Runenmeisterin]{Runenmeisterin} darf \fan{beliebig viele} von fern aufdeckbare Plättchen auf einem Feld aufdecken.
\supercite{88631}

\bigskip

\textbf{Von nah aufdeckbare Plättchen:}

\bullet{}  Der Rabe des \hyperref[Fährtenleser]{Fährtenlesers} darf \fan{beliebig viele} von nah aufdeckbare Plättchen auf einem Feld aufdecken.

\bullet{}  Flaps Flederfuchs darf \fan{beliebig viele} von nah aufdeckbare Plättchen auf seinem Feld aufdecken.
\supercite{DeK:Begleitheft,DfL:Begleitheft}

\bullet{} Ein Held darf von nah aufdeckbare Plättchen auf seinem Feld aufdecken. Manche Plättchen muss er aktivieren.

\bullet{} Ein Held mit Kapuze darf von nah aufdeckbare Plättchen auf seinem Feld ansehen, ohne sie aktivieren zu müssen.
\supercite{DlH:Ausrüstungstafel}
\bigskip

\textbf{Vorhersehbare Plättchen:}

\bullet{}  Das Sehende Auge des \hyperref[Seher]{Sehers} darf ein vorhersehbares Plättchen auf dem Spielplan (auch an der Tagesleiste oder Legendenleiste) aufdecken.

\bigskip


\textbf{Allgemein:}

\bullet{}  Plättchen werden durch diese Effekte nur aufgedeckt, nicht aktiviert.

\bullet{} Plättchen in Stapeln dürfen auch durch diese Effekte aufgedeckt werden, ihre Reihenfolge bleibt aber gleich.
\supercite{BL_BdM:A3_en}
Ausnahme: Von den Bewegungsplättchen aus "Die letzte Hoffnung" darf nur das jeweils oberste aufgedeckt werden.
\supercite{DH:Begleitheft}

\bullet{}  Karten können durch diese Effekte weder aufgedeckt noch aktiviert werden. Ausnahme: Ortskarten in Legende 17, "Die letzte Hoffnung", dürfen mit Fernrohr \fan{oder vom \hyperref[Bergkrieger]{Bergkrieger} mit 14 oder mehr Willenspunkten} aktiviert werden.

\bigskip






\newpage
\chapter{Quellenverzeichnis}
\label{Quellen}

\begin{multicols*}{2}\raggedcolumns
    \setlength\bibitemsep{1pt}
    \renewcommand*{\bibfont}{\fontsize{6}{6}\selectfont}
    \printbibliography[heading=none]
\end{multicols*}












\newpage



\hintergrund{Helden Cover.jpg}

\thispagestyle{empty}%keine Header oder Footer auf dieser Seite

\textbf{ }




\end{document}